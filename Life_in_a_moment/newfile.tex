31

\textsuperscript{\textdutch{\textbf{Life in a Moment}}}

\textdutch{Preface}

\textdutch{Vincent (Yen-Cheng Chen) had invited Acharn Sujin and her
sister Khun Suchit to Taiwan in September 2018 for a ten days sojourn.
This was the first time for Dhamma discussions with Acharn in Taiwan. As
soon as she had arrived she explained about the importance of
understanding the reality of the present moment: ``Life is only in a
moment, like now. What appears, what is experienced now? Only one dhamma
at a time''. Throughout our discussions the subject was the
understanding of this moment. }

\textdutch{Vincent had sponsored the hotel accomodation of Acharn and
her sister and he organized our whole program throughout our stay. Sarah
and Jonothan were assisting Acharn with the Dhamma discussions and
friends from Thailand and Vietnam joined this trip. Hang from Vietnam
helped me in many ways as she had done before on other trips so that I
could move about in comfort. }

\textdutch{I had come a few days earlier and stayed a little longer so
that I could have more conversations with Taiwanese friends. Vincent had
arranged for me hotel rooms with adapted bathrooms with many grips to
hold on to, even more comfortable than at home. He went out of his way
to obtain a wheel chair and a frame walker. I greatly appreciate all his
efforts to make my stay most agreeable and comfortable. The first
evening I went to his home where I played the piano together with his
wife Jane. We played a piece by Brahms for two pianos, an enjoyable
experience. Their daughter Nana and our friend Maggie I had met before
were also present with Nana's newly acquired little dog Hey hey. }

\textdutch{Every day Vincent took me out for luncheon to a different
restaurant. He arranged conversations with small groups of people in my
room or in restaurant the ``Mövenpick''. He translated our discussions
into Mandarin and helped to add more clarifications to difficult
points.}

\textdutch{He had translated my book ``The Buddha's Path'' and the
printing was just ready upon my arrival so that we could give away this
book to different peo-ple. He is now translating my ``Abhidhamma in
Daily Life''. }

\textdutch{With Acharn and the group of friends we stayed four nights in
Sun Moon Lake where we had discussions outside in a garden. During heavy
rains plastic coverings were cleverly put all around. One day we could
have luncheon outside surrounded by beautiful flowers and blooming
shrubs. After Sun Moon Lake we all stayed in Taipei. Our program was
full and very lively. We all profited from people's interest and
questions, some basic and some going into the depth of the teachings.
This is the way to learn that life is only in a moment and that there is
no one there.}

\textdutch{Just before the sessions in Taiwan, Dhamma discussions were
arranged in Kaeng Kracan, Thailand. I did not attend those, but I used
for this book some parts I heard on the audio that Sarah and Jonothan
had prepared. Moreover, I also used further discussions on our Dhamma
Study Group forum on the subjects raised in Kaeng Kracan. }

\textdutch{Chapter 1.}

\textdutch{The real cause of trouble.}

\textdutch{On arrival at the airport the young man who pushed my
wheelchair was very helpful and generous, and he did not accept a tip. I
said to him that we like pleasant objects through the senses, and that
it is natural that we are attached to family and friends. We should not
try not to have attachment, this is impossible, but we should come to
know what it is. Attachment to pleasant objects is different from
kindness and generosity, they are different qualities. The Buddha taught
about these in detail and how they arise from different conditions. When
we arrived at the garage where Vincent had parked his car he insisted
going all the way to the car, and meanwhile he listened to the
explanations of Dhamma. He did not accept Vincent's offer of further
meetings because of lack of time, but we never know whether there will
be conditions in the future for hearing the Dhamma again. }

\textdutch{The difference between ultimate reality and concepts or
conventional ideas was a subject that came up again and again. People
say that they can see a flower and that the flower is real. In reality
seeing sees only visible object and a flower is only an idea, an object
of thinking. Seeing sees visible object through the eyesense, hearing
hears sound through the earsense, smelling smells odour through the
nose, tasting tastes flavour through the tongue and when there is
touching of tactile object, hardness, softness, heat or cold, motion or
pressure are experienced through the bodysense. Throughout the
discussions it was often emphasized that there are two kinds of
realities: those that experience something, like seeing or hearing, and
those that cannot experience anything, like visible object or sound. }

\textdutch{Acharn asked time and again what dhamma is: it is reality,
that which can be directly known through one of the senses or the
mind-door. It was useful for all of us that the subject of discussions
was basic notions of the Dhamma because we are forgetful of what is
real. We are often absorbed in stories about concepts and situations
which are not real. }

\textdutch{In Sun Moon Lake I had to climb with my stick a very steep
staircase outside in order to reach the restaurant where we had
breakfast. I had my stick in one hand and Vincent supported me at the
other side. He asked me: ``What is reality now?'' It is good to be
reminded of the truth in difficult situations. I was absorbed in the
steps I had to take and in my anxiety. Aversion or dosa is real,
hardness appearing through the bodysense is real, but they appear just
for an extremely short moment. They are just passing dhammas. It happens
time and again in life that we make the circumstances and concepts we
are thinking of into something very important. }

\textdutch{During the discussions with Taiwanese friends it appeared
that one of them was the victim of gossip and that she was angry about
this. After we explained the difference between reality and concept she
understood that in reality there is no person to be blamed for one's
unhappiness. The cause of unhappiness is always within ourselves. In
reality there are no persons, only citta, cetasika and rūpa. In
understanding this we can become more patient and more tolerant. }

\textdutch{The Buddha spoke about the worldly conditions that change all
the time: gain and loss, honour and dishonour, happiness and misery,
praise and blame. We may be praised, but soon after that we may be
blamed, it all depends on conditions that are beyond control. }

\textdutch{In the ultimate sense these worldly conditions are not
situations but momentary dhammas, they arise and fall away immediately.
The Buddha spoke to people with different levels of understanding and
therefore he also spoke by way of conventional notions and situations.
We think of being blamed or praised, but in reality people are not the
cause of blame or praise but it is kamma. Akusala kamma or kusala kamma
conditions an experience of an unpleasant object or pleasant object just
for a moment, but we think for a long time about such experiences with
akusala cittas accompanied by aversion or attachment. There is no self
that can exert power over kamma, over its result, over akusala citta or
kusala citta that thinks about kamma and vipāka.}

\textdutch{Our friend who suffered from gossip said that, when looking
back, she found that she could change her attitude and become more
detached, she could let go of unhappiness. She used to think of self who
has to experience unpleasant things but now she thinks less in this way.
When we look back and consider our attitude, believing that we can let
go of unhappiness, there may be an idea of self who can change his
attitude. The idea of self comes in all the time and mostly it is
unknown. In reality there is no self, only different moments of
consciousness, cittas, accompanied by mental factors, cetasikas, that
are wholesome, kusala, unwholesome, akusala or neither wholesome nor
unwholesome. What we take for a person are only citta, cetasika and
material phenomena, rūpa. }

\textdutch{Life is only in a moment, like now. When we come to
understand that only one dhamma at a time is experienced through one of
the senses or the mind-door, we can appreciate more that Acharn said
time and again: ''There is no one there''. This is not Acharn's
teaching, but it is what the Buddha explained about the truth of anattā.
When we feel unhappy about an unpleasant event it is actually our own
aversion that has been accumulated in the citta from moment to moment,
from life to life. There is no one else to blame and we cannot say that
the situation we are in is the cause. Since each citta is succeeded by a
following one from life to life, good and bad inclinations are
accumulated. The Buddha's words saying that greed, hatred and delusion
are the cause of all trouble in the world have a deep meaning. It is the
accumulated attachment, aversion and ignorance that are the real cause
of all trouble in our own life. This is expressed in the first quoted
text of my book ``The Buddha's Path'' and we often referred to this in
our discussions. }

\textdutch{We read in the ``}\textenglish[variant=american]{Kindred
Sayings}\textdutch{''} \textdutch{(I, Chapter III, Kosala, Part 3,
}§\textenglish[variant=american]{3, The World) that King Pasenadi had a
conversation with the Buddha at S}\textdanish{ā}vatthī
\textenglish[variant=american]{about the cause of suffering. We read:}

\textsuperscript{``\ldots{}\textenglish[variant=american]{How many kinds
of things, lord, that happen in the world, make for trouble, for
suffering, for distress?}''}

\textsuperscript{``\textenglish[variant=american]{Three things, sire,
happen of that nature. What are the three? Greed, hate, and
delusion---these three make for trouble, for suffering, for
distress}\ldots{}''}

\textsuperscript{\textdutch{{By listening to the Buddha's teachings
there can be more understanding of attachment, aversion and ignorance
which are of many degrees. Whenever there is attachment we think of
ourselves. When we want to get hold of a pen or a paper, there is likely
to be attachment already, but it is mostly unknown. We are attached to
friends and family members but it is good to know that we are actually
attached to ourselves when we like the company of dear people.}}}

\textsuperscript{\textdutch{{Some degrees are very harmful such as
wanting to steal something, and some are not harmful for others. We can
learn to see the danger and disadvantage of all degrees of akusala. }}}

\textsuperscript{\textdutch{{When unwholesome qualities are strong they
can motivate uwholesome deeds, akusala kamma through body, speech and
mind. Wholesome qualities can motivate wholesome deeds, kusala kamma.
}}}

\textsuperscript{\textdutch{Kusala kamma, committed in the past,
conditions seeing, hearing and other sense-cognitions to experience
pleasant objects and akusala kamma conditions the experience of
unpleasant objects. These moments which are vipākacittas, cittas that
are result, are extremely brief and it is hard to know whether they are
results of kusala kamma, kusala vipākacittas, or of akusala kamma,
akusala vipākacittas. When we think about them they have fallen away
already. }}

\textsuperscript{\textdutch{Sense-cognitions, which are vipākacittas,
arise in a process of cittas and shortly after they have fallen away,
wholesome cittas, kusala cittas, arise which experience the object in a
wholesome way, or akusala cittas, which experience the object in an
unwholesome way. It depends on conditions what type of citta arises at a
particular moment. When we think about an object that was experienced,
this type of citta is either kusala or akusala, it cannot be otherwise.
The Buddha taught about citta, cetasika and rūpa in detail in order to
help beings to understand that whatever arises does so because of its
proper conditions and that there is no self who could control any
reality. }}

\textsuperscript{\textdutch{Someone asked what the purpose is of the
study of so many details of mental phenomena and physical phenomena.
When facing troubles in life such as the loss of a dear person, she
would realize that life is such, and she was wondering why she would
need to study the teachings. When we wonder what the purpose is of study
it is just thinking. Paññā, understanding, is needed to appreciate to
know the truth, to know what is reality. Without having heard the
teachings we would believe that a self really exists, not knowing that
we cling to what is only an idea, not a reality, and we would take what
is akusala for kusala. We would mislead ourselves all the time. }}

\textsuperscript{\textdutch{By hearing the teachings and considering
what appears at the present moment there will be a beginning of
intellectual understanding, pariyatti, and when this has become firmer,
it can condition later on direct understanding, paṭipatti. This was
emphasized many times throughout our discussions. }}

\textsuperscript{\textdutch{Seeing sees only visible object, but it
seems that we immediately see people and things. We believe that we
touch a table, but in reality only tactile object such as hardness or
temperature is experienced. We asked a friend to actually touch what she
believed was a table. When she touched the table she began to understand
that the reality of hardness appeared, not a table. She began to
understand what dhamma, reality, is. When someone begins to see what was
not understood before it is evident that understanding begins to
develop. Paññā can be accumulated little by little. Paññā is not self
but only a cetasika that can accompany kusala citta. }}

\textsuperscript{\textdutch{*****}}

\textsuperscript{}

\textdutch{Chapter 2.}

\textdutch{The sign, nimitta, of what is experienced.}

\textdutch{Seeing sees visible object and then it falls away, and very
shortly afterwards there are other moments of seeing visible object.
Realities succeed one another so rapidly that it seems that they can
stay. }\textenglish[variant=american]{It seems that visible object lasts
for a while, but in reality it arises and}\textdutch{
}\textenglish[variant=american]{falls away. Acharn Sujin used the simile
of a torch that is swung around. We}\textdutch{
}\textenglish[variant=american]{have the impression of a whole, of a
circle of light, but this is caused by}\textdutch{
}\textenglish[variant=american]{many moments of light.}\textdutch{ }

\textenglish[variant=american]{We cannot pinpoint which visible object
is experienced at}\textdutch{ }\textenglish[variant=american]{the
present moment}\textdutch{, there are many of them arising and falling
away very rapidly}. \textdutch{O}\textenglish[variant=american]{nly an
impression or mental image, nimitta, of}\textdutch{ }\textfrench{visible
object}\textdutch{ remains}\textenglish[variant=american]{. This
}\textdutch{causes}\textenglish[variant=american]{ us to think that
visible object does not}\textdutch{ }\textenglish[variant=american]{fall
away.}\textdutch{ }

\textenglish[variant=american]{We know that seeing arises at this
moment, but we cannot pinpoint the }\textdutch{
}\textitalian{citta}\textdutch{ }\textenglish[variant=american]{which
sees, it arises and falls away very rapidly and then there is
another}\textdutch{ }\textenglish[variant=american]{moment of seeing. We
only experience the impression or sign of seeing.}\textdutch{ }

\textenglish[variant=american]{The topic of nimitta or sign came up
several times during our discussions. }\textdutch{Some one asked
}\textenglish[variant=american]{what nimitta is}\textdutch{ and Acharn
answered:}

\textsuperscript{``\textenglish[variant=american]{It is there now. It is
there because of the rapid arising and}}

\textsuperscript{\textenglish[variant=american]{falling away of
realities. No matter which unit of visible object is}}

\textsuperscript{\textenglish[variant=american]{arising and falling
away, it keeps on arising and falling away so}}

\textsuperscript{\textenglish[variant=american]{that it forms up the
sign (nimitta) of it as something that can be}}

\textsuperscript{\textenglish[variant=american]{seen. Whenever there is
seeing there is the sign of the rapid arising}}

\textsuperscript{\textenglish[variant=american]{and falling away of
realities.}\textdutch{''}}

\textsuperscript{\textenglish[variant=american]{In the
}``\textenglish[variant=american]{Discourse on the Sixfold
Cleansing}''\textenglish[variant=american]{(Middle Length Sayings III,
112) the Buddha speaks about a monk who declares
}``\textenglish[variant=american]{profound
knowledge}''\textenglish[variant=american]{, who states that he has
reached the end of birth, thus, that he is an arahat. The Buddha said
that he might be questioned about his understanding so that one knows
whether he speaks the truth. In this sutta we read about all realities
appearing through the six doors which are the objects of right
understanding, no matter whether someone is a beginner on the Path or an
arahat. }}

\textsuperscript{\textenglish[variant=american]{We read about his
}``\textenglish[variant=american]{guarding of the six doors}''
\textenglish[variant=american]{through mindfulness:}}

\textsuperscript{``\textenglish[variant=american]{If I saw visible
object with the eye I was not entranced by the general appearance, I was
not entranced by the detail. If I dwelt with this organ of sight
uncontrolled, covetousness and dejection, evil unskilled states, might
flow in. So I fared along controlling it, I guarded the organ of sight,
I achieved control over it}\textdutch{\ldots{}''}}

\textsuperscript{\textdutch{The ``general appearance'' in this text
stands for
nimitta\protect\hyperlink{sdfootnote1sym}{\textsuperscript{1}}. We are
attached to the nimittas of the sense objects and to their details. }}

\textsuperscript{\textdutch{Acharn asked whether there is nimitta in a
dream. When we are dreaming it seems that we see persons and different
things, but we are not really seeing but thinking of different concepts.
The concepts of realities are experienced. Also when someone attains
jhāna the object of jhānacitta is usually a concept, such as a disc made
of earth or a coloured disc. When concentration of the level of samatha
is more developed there is no need anymore to look at the disc made of
earth, but a mental image can be experienced. }}

\textsuperscript{\textdutch{We read in the Visuddhimagga (Ch XII,8):}}

\textsuperscript{``\textenglish[variant=american]{Now, the
kasi}{{ṇ}}\textenglish[variant=american]{a preliminary work is difficult
for a beginner and only one in a hundred or a thousand can do it. The
arousing of the sign is difficult for one who has done the preliminary
work and only one in a hundred or a thousand can do it. To extend the
sign when it has arisen and to reach absorption is difficult and only
one in a hundred or a thousand can do it}\textdutch{\ldots{}''}}

\textsuperscript{\textdutch{This clearly shows us that samatha developed
to the stage of jhāna, absorption concentration, is not for everyone. }}

\textsuperscript{\textdutch{Sarah wrote:
''E}\textenglish[variant=american]{ven these
jh}{{ā}}\textenglish[variant=american]{nacittas and development of
samatha are just conditioned dhammas. No one actually
}\textdutch{`}\textenglish[variant=american]{can
do}\textdutch{'}\textenglish[variant=american]{ anything or
}\textdutch{`}\textenglish[variant=american]{has
done}\textdutch{'}\textenglish[variant=american]{ anything. It just
depends, like now, as to what dhammas there are conditions
for}\textdutch{,}\textenglish[variant=american]{ at
anytime}\textdutch{,}\textenglish[variant=american]{ and as all such
dhammas are anicc}\textdutch{ā}, dukkha and
anatt{{ā}}\textenglish[variant=american]{, they are not of any
consequence or to be attached to for an instant.}\textdutch{''}}

\textdutch{The whole day we experience concepts of people and things.
Memory, saññā, that accompanies each citta, is the condition that we
remember and think of the shape and form and details of things. We
experience nimittas of concepts. A mental image experienced of Nana's
dog Heyhey was given as an example. On account of many moments of seeing
visible object there is remembrance and thinking of the concept dog. }

\textdutch{In different contexts there are different meanings of
nimitta: it can be nimitta of each of the five khandhas, thus, nimitta
of conditioned realities, nimitta in samatha, nimitta of a concept like
in a dream or of a concept just now, in daily life. }

\textdutch{Acharn explained, in Kaeng Kracan, that what appears is only
nimitta. When there is seeing only the sign or nimitta of the rapid
arising and falling away of visible object is experienced. Visible
object cannot be a thing or a person which stays. She said: ''This is
wise consideration: that actually now saññā is remembering the nimitta
as something, all the time. That is perversity of saññā, saññā
vipallāsa.'' She explained that because saññā marks each object there
can be the remembrance of the sign that is experienced. Without saññā
there could be no thinking. Sarah said that when there is understanding
of nimitta there is no idea of things lasting long. Actually, in
understanding more about nimitta we are reminded that life is just in a
moment. }

\textdutch{In the ``Visuddhimagga'' saññā is called ``nimitta-maker'',
nimitta karaṇa. We read in the definition of saññā in the
``Visuddhimagga'' }(XIV 130)\textdutch{:}

\textsuperscript{{{"}}{{All
(sa}}{{ññ}}{{ā}}\textenglish[variant=american]{{{) has the
characteristic of recognition
(sa}}}{{ñ}}{{j}}{{ā}}\textenglish[variant=american]{{{nana); its
property is the making of representation (nimitta) that is a condition
of recognizing again, `this is the very same thing' - as carpenters and
so on do with the wood, etc.; its manifestation is the producing of
conviction by virtue of a representation (nimitta) that has been
accordingly learnt - like the blind perceiving the
elephant}}}\textdutch{{{
\protect\hyperlink{sdfootnote2sym}{\textsuperscript{2}}}}}\textenglish[variant=american]{{{.
Its basis is whatever object that has come near - like the recognition
(sa}}}{{ññ}}{{ā}}\textenglish[variant=american]{{{) `people' that arises
for young animals in respect of scarecrows.}}}\textdutch{{{''}}}}

\textsuperscript{\textdutch{{{The term ``representation'' in this text
stands for nimitta. Saññā marks each object that is experienced by citta
and cetasikas, so that it can be remembered or recognized later on.
Thus, saññā makes a sign, a nimitta, and moreover, saññā itself is
experienced by way of nimitta since it arises and falls away very
rapidly. }}}}

\textsuperscript{\textdutch{{{Sarah wrote
(}}}\textenglish[variant=american]{{{on Dhamma Study Group, the internet
forum}}}\textdutch{{{) about saññā:
``}}}\textenglish[variant=american]{{{Hence we can say we live in the
world of sa}}}{{ññ}}{{ā}}\textenglish[variant=american]{{{, or rather
the nimitta of sa}}}{{ññ}}{{ā}}\textenglish[variant=american]{{{. There
is sa}}}{{ññ}}{{ā}}{{ }}\textenglish[variant=american]{{{at each moment
but it depends on vitakka what is thought about, which memory is the
object of the dream world. It is sa}}}{{ññ}}{{ā}}{{
}}\textenglish[variant=american]{{{at moments of seeing and hearing
which also conditions kusala and akusala which follows on account of
what is seen, on account of the nimitta of visible
object.}}}\textdutch{{{''}}}}

\textdutch{Sarah also wrote about saññā: }

``\textenglish[variant=american]{Just like at night-time when we are
dreaming, so now we are usually
lo}\textdutch{st}\textenglish[variant=american]{ in dreams and ideas and
fantasies on account of what has been remembered and marked about
various sense objects experienced. Without memory of those experiences,
there would be no thinking about them, no fantasies or dreams at all.}

\textenglish[variant=american]{Therefore it is also on account of what
has been remembered and marked, i.e. the signs and details of what has
been experienced before, that kusala (wholesome) and akusala
(unwholesome) kinds of thinking arise and accumulate}\textdutch{;}
\textdutch{one may be}\textenglish[variant=american]{ attached,
hav}\textdutch{e}\textfrench{ aversion o}\textdutch{r}
\textdutch{be}\textenglish[variant=american]{ ignorant of what is
experienced, or }\textdutch{one may be}\textenglish[variant=american]{
aware and understand what appears. }

\textsuperscript{\textdutch{W}\textenglish[variant=american]{e live in
the world of sa}ññ{{ā}}\textenglish[variant=american]{. It is also why
i}\textdutch{t}\textenglish[variant=american]{ is a separate
khandha.}\textdutch{''}}

\textdutch{During our discussions in Taiwan people were wondering how
saññā can remember things. How can it remember or recognize what is gone
already? They were wondering whether it is accumulated from moment to
moment. When we speak about accumulation it is with reference to the
function of javana, the moments of kusala cittas or akusala cittas, in a
process. Sense-cognitions such as seeing arise and shortly after that
kusala cittas or akusala cittas arise. The cittas that perform the
function of javana can be kusala citta or akusala citta, and, in the
case of the arahat, kiriyacitta. Thus, only saññā arising with the
javana-citta accumulates. But since saññā accompanies every citta the
saññā of each moment conditions the succeeding saññā by way of
contiguity-condition, anantara-paccaya. By way of this condition each
citta and its accompanying cetasikas condition the succeeding citta
throughout life. Different types of conditions have to be
distinguished.}

\textdutch{Acharn said about saññā: ''Saññā marks the object and there
are uncountable saññās; they are a condition for thinking. After seeing
there is thinking. It takes a long time to understand saññā as saññā.
Just hear it again and again, no expectation to understand. As soon as
there is memory it is saññā, not `I'. When one tries very hard to know
it, it is attachment. It is there all day.''}

\textdutch{There are many kinds of saññā, such as attā-sañnnā, wrong
remembrance of self. Because of saññā we have an image of self time and
again. When paññā has been developed there can be anattā-saññā, right
remembrance as non-self. As Acharn said, we have to hear the Dhamma
again and again, without expectation of having full understanding soon.
}

\textdutch{*******}

\textdutch{Chapter 3. }

\textdutch{The precision of Abhidhamma.}

\textdutch{When we were in Sun Moon Lake I said to Sarah that I like to
hear Acharn talk again and again about seeing. She remarked that this
could be attachment. If a good Dhamma friend had not reminded me I would
not have realized this. }

\textdutch{Later on she said: ''The best thing is understanding what is
heard, little by little. When we think: 'I like to understand that', it
is not the moment of understanding''. }

\textdutch{As Acharn said, there is attachment all day and many moments
are unknown, such as liking to understand something. }

\textdutch{A question that is often asked is : ``How can I have less
anger.'' When one asks this it shows a lack of understanding what dhamma
really is: a conditioned reality. One thinks of a self who would like to
have less anger. Anger, dosa, is a cetasika that accompanies akusala
citta. It has been accumulated from life to life and, thus, it has
conditions for its arising. There are many shades and degrees of it: it
can arise when there is a slight feeling of uneasiness, or it can be
stronger in the form of anger. People are inclined to think of a
conventional idea of anger that can stay and that they would like to
master so that it goes away. They think of situations and other people
who cause them to be angry instead of understanding the reality that
arises just for a moment. It is important to see the difference between
ultimate realities and conventional ideas. Citta, cetasika and rūpa are
the ultimate realities in our life. They arise just for a moment because
of conditions and then fall away immediately. When we think of stories
about our anger, about ourselves who want to be different we are
thinking about what is not real, we are thinking of ourselves.}

\textdutch{We should remember that life is realities such as seeing,
hearing, hardness, sound, just nāmas and rūpas appearing one at a time.
Life exists just in a moment. We find situations and ideas so important
that we forget what life is at the present moment. As Azita reminded me:
}

\textsuperscript{~``\textenglish[variant=american]{Even when we hear
'there is no}\textdutch{ }\textenglish[variant=american]{one here at
all' and have heard this for a number of years,}\textdutch{
}\textenglish[variant=american]{most of the time we are obsessed with
self - my anger, my generosity. I believe there has to be reminders
again and again and again that this is the truth of life. }~}

\textsuperscript{~ \textenglish[variant=american]{Then we could ask
`what is life?' }\textdutch{R}\textenglish[variant=american]{ealities
such as seeing, hearing, tasting which arise for a very brief moment
then gone never to arise again, ever, all by conditions, no}\textdutch{
}\textenglish[variant=american]{one here to control or choose what will
be experience}\textdutch{d}\textenglish[variant=american]{ the very next
moment.}\textdutch{''}~}

\textsuperscript{\textdutch{The most valuable moments in our life are
understanding reality as it is and then we do not take dhammas which are
gone for permanent or self, we are not concerned with stories. }{{\\
}}\textdutch{Sarah spoke in Kaeng Kracan about disturbances in life:
``Every time there is disturbance it is just the citta that thinks. All
are just dhammas, not belonging to anyone, never to return, not of any
importance.'' }}

\textdutch{Acharn explained that it is a relieve to have more
understanding of anattā, to begin to see anattā. One understands that
what occurs cannot be any other way.}

\textdutch{In Sun Moon Lake we were in Acharn's room with Sarah and
Jonothan where we had an opportunity to talk informally about our life
now. In another room there was, at the same time, a meeting with
Taiwanese friends and Acharn's words spoken in English were translated
into Mandarin. }

\textdutch{Some people believe that they can decide what to think about,
but even that moment of thinking is conditioned, not a self who is
choosing an object. Sarah asked: ''Can we choose what we think now?''
There are realities such as seeing, hearing, thinking, pleasant feeling,
visible object, sound, hardness, and this is actually life from moment
to moment. Some realities can experience an object and some realities do
not experience any object. Life is just seeing, visible object, hearing,
sound or thinking. Citta, cetasika and rūpa are arising and falling
away. The table, the fruit and sweets on the table are not realities,
they are merely concepts we can think about. }

\textdutch{A young woman came to Acharn's room and asked Acharn what
love is. Acharn asked whether love is liking another person or just
liking oneself. If there would be no eyes or ears there would not be any
idea of a person. She answered that she worries about the person she
loves. Acharn answered where that person is and she answered: ''In the
mind.'' }

\textdutch{Acharn said:'' What is there in the mind? Sometimes love,
sometimes hatred. Love is real. It is not seeing. Life is feeling, is
there feeling now? What kind of feeling is there?'' }

\textdutch{Acharn often asks questions in order to help people to
understand what is real. When feeling arises and appears our life is
feeling at that moment. When we use the word love in daily language we
think of a whole situation. There is another person and ``I'', and we
think of a relationship that exists between persons. In reality there
are no persons, no relationship that exists, no situation, only citta,
cetasika and rupa that are fleeting moments. It is difficult to
understand the difference between conventional notions and ultimate
realities.}

\textdutch{Sarah asked the young woman what the difference is between
feeling and liking. When there is liking is there feeling at the same
time?}

\textdutch{Acharn said: ``When there is understanding of anything, it
has to be now, no matter what we are talking about. When we talk about
feeling, is there feeling now? One is used to having feeling but one
does not know what feeling is. It has to be known when it is there,
otherwise we are just talking.'' }

\textdutch{This is an important principle: we should verify the
realities that appear. Otherwise we just engage in theory, in the
``story'' about realities. We can so easily mislead ourselves about the
truth. }

\textsuperscript{\textdutch{The reality of feeling is quite different
from the conventional idea of feeling that we used to have before
hearing the Buddha's teaching. }\textenglish[variant=american]{The
Abhidhamma gives details about the different types of
}\textdutch{feeling}\textenglish[variant=american]{ that arise,
}\textdutch{about happy feeling, unhappy feeing, and indifferent
feeling. Moreover, there are pleasant bodily feeling and painful bodily
feeling which are conditioned by impact on the bodysense. Each citta is
accompanied by a specific type of feeling. For example, akusala citta
rooted in attachment can be accompanied by happy feeling or indifferent
feeling and kusala citta can be accompanied by happy feeling or
indifferent feeling. W}\textenglish[variant=american]{hen we feel
happy}\textdutch{ we may easily take akusala citta rooted in attachment
for kusala citta. All such details are taught in order to show that
}\textenglish[variant=american]{whatever arises is conditioned and that
we cannot be master of it. We can learn to understand whatever arises
and in this way }\textdutch{paññā}\textenglish[variant=american]{ can
develop. It is all about daily life.}\textdutch{ }}

\textsuperscript{\textdutch{If there were no citta the world could not
appear. When citta arises it experiences different objects: seeing
experiences visible object, hearing experiences sound, and the other
sense-cognitions experience their appropriate objects. When we think of
the world we think of a whole, a collection of impressions, but there
are only different objects experienced by a citta, one at a time. Each
citta is accompanied by cetasikas that experience the same object and
each have their own function while they assist citta in the experience
of the object. }}

\textsuperscript{\textdutch{At least seven cetasikas have to accompany
each citta, such as feeling, remembrance, or contact,(phassa), that
contacts the object so that citta can experience it. }}

\textsuperscript{\textdutch{Also cetanā, intention or volition,
accompanies each citta. Its function is coordinating the tasks of the
accompanying cetasikas. It can accompany kusala citta, akusala citta,
vipākacitta (}\textenglish[variant=american]{citta which is
result}\textdutch{)},\textdutch{ and kiriyacitta, citta that is neither
cause nor result. This is volition which is conascent kamma (sahajata
kamma)\protect\hyperlink{sdfootnote3sym}{\textsuperscript{3}}. Volition
accompanying kusala citta or akusala citta which is of sufficient
strength can motivate wholesome deeds or unwholesome deeds through body,
speech or mind that produce results. When it has produced result arising
later on it is asynchronous kamma, kamma working from a different time,
}n\textitalian{āṇ}akkha\textitalian{ṇ}ika\textdutch{ kamma. People
usually think of the ``story of kamma and vipāka'' instead of knowing
realities that present themselves one at a time. When something
unpleasant occurs to them they think: ''This is my kamma''. Through the
Buddha's teachings we learn to distinguish different realities, and to
know them more precisely. The subject of kamma and vipāka came up
several times during our discussions. }}

\textsuperscript{\textdutch{If life is not seen as different moments
some of which are cause and some results, and if one thinks of a self
who is performing deeds or receiving results one may have all kinds of
dilemmas about kamma and vipāka. People wonder why they have to be
punished for what was done in a former life.
``}\textenglish[variant=american]{I was not
there}\textdutch{''}\textenglish[variant=american]{, they think.
}\textdutch{We should}\textenglish[variant=american]{ not think of
}``I'' or person. \textdutch{There are
j}\textenglish[variant=american]{ust passing dhammas, some are cause,
some are effect.}\textdutch{ Or people wonder whether it makes sense to
do good deeds when there is nobody who will receive results in a future
life. This is again thinking of ``stories'' about self and ``my good
deeds'' instead of knowing the present reality as non-self. However,
when something unpleasant happens one tends to forget about the truth.
}}

\textsuperscript{\textenglish[variant=american]{My
sister}'\textenglish[variant=american]{s dog cannot stand a walker and
she never stopped barking in an aggressive way to me}\textdutch{ with my
walker}\textenglish[variant=american]{, even snapping at my trousers.
When we played the piano she was quiet for a while and then started
barking at me again and again. I felt lost and thought of
Acharn}'\textenglish[variant=american]{s words: there is no one there.
But how can that solve my problem: no dog there, no me who feels badly
about it. But I remembered the way I spoke to our friend who was the
victim of gossip: nobody else is to blame for our disturbances, the
cause is only in ourselves. But now, something unpleasant was happening
to me. I was forgetful of kamma and
vip}\textdutch{ā}\textenglish[variant=american]{ka. Hearing sound is
vip}\textdutch{ā}\textenglish[variant=american]{kacitta and thinking of
the meaning, thinking of how aggressive the dog was is akusala citta. It
is so helpful that the Buddha taught us different realities:
vip}\textdutch{ā}\textenglish[variant=american]{kacitta being entirely
different from being lost in stories, thinking with akusala cittas about
a long, long story. This happening was a good reminder to me. }}

\textsuperscript{\textdutch{Sarah wrote to me on account of my story
about the dog: ``}\textenglish[variant=american]{When there are
conditions for disturbance, for worry and fear and distress, it has to
be like that, it cannot be any other way. But it doesn't last, it's
dhamma which passes instantly. Like now, no disturbance or worry, but
still conditions to think }\textdutch{of}\textenglish[variant=american]{
the story about it. Very natural - same for}\textdutch{
}\textenglish[variant=american]{everyone, dwelling on what has gone and
forgetting about passing dhammas now.}\textdutch{'' }}

\textsuperscript{V\textdutch{incent remarked:}
\textdutch{``}\textenglish[variant=american]{We are reminded that citta
is always alone, there is no dog in the sound;
vip}\textdutch{ā}\textitalian{kacitta }\textdutch{is the
}\textenglish[variant=american]{result of kamma, not a self who
experiences. And the problem is the accumulated defilements that follow,
which are also anatt}\textdutch{ā}\textenglish[variant=american]{. What
a profound and beautiful meaning, though I usually don't remember it.
The sign of beauty or resentment
take}\textdutch{s}\textenglish[variant=american]{ over so quickly,
instead of apprehending what is really there.}\textdutch{''}~}

\textsuperscript{\textdutch{Sarah wrote:
``}\textenglish[variant=american]{That}\textdutch{'}\textenglish[variant=american]{s
the answer - only the understanding of dhammas, the understanding that
each moment of hearing, each moment of thinking is conditioned, not self
- can }\textdutch{`}solve\textdutch{'}\textenglish[variant=american]{
the problem for that moment only.}\textdutch{''}}

\textsuperscript{\textdutch{Vincent said:
``}\textenglish[variant=american]{Yes, only understanding can solve all
the problems. Understanding}\textdutch{
}\textenglish[variant=american]{works its way,
anatt}\textdutch{ā}!\textdutch{''}}

\textsuperscript{\textdutch{********}}

\textsuperscript{\textdutch{Chapter 4.}}

\textsuperscript{\textdutch{Conditions}}

\textsuperscript{``\textdutch{What is life'' was a question Acharn often
asked the listeners. We usually think of life as lasting for some time.
But in reality life exists in this moment: citta, cetasika and rūpa,
arising and falling away. When seeing arises, life is seeing, when
thinking arises, life is thinking and thinking may be kusala or akusala.
M}\textenglish[variant=american]{any different types of citta
aris}\textdutch{e}\textenglish[variant=american]{ in a day, and, mostly
}\textdutch{they are }\textenglish[variant=american]{akusala cittas.
Even when doing a good deed, kusala cittas and akusala cittas alternate
so fast. }\textdutch{We think of ourselves, we want to be a good person;
we cling to an idea of my kusala. }\textenglish[variant=american]{When
we think }\textdutch{about good deeds, the cittas that
think}\textenglish[variant=american]{ are gone already. }}

\textsuperscript{\textdutch{W}\textenglish[variant=american]{e often
think of }\textdutch{dāna, liberality, or sīla as
}\textenglish[variant=american]{a whole situation}\textdutch{, we think
of ourselves as being generous to others or as speaking kind words to
others, but it is the wholesome citta that motivates deeds and speech,
it is not a self. It depends on conditions what type of citta arises at
a particular moment. }}

\textsuperscript{\textdutch{The Buddha often spoke about seeing, visible
object, hearing or thinking. He asked about each reality whether it is
permanent or impermanent. This reminds us that only one reality appears
at a time and that it does not stay. When seeing appears, there is no
hearing or thinking at the same time. Acharn often explains about seeing
and visible object appearing at this moment so that we can begin to
understand the reality of the present moment instead of clinging to a
whole situation. And even when we live in the world of concepts it is
ony thinking, arising because of its own conditions. }}

\textsuperscript{\textdutch{In Sun Moon Lake some newcomers asked
questions. Two French women wanted to know more about the present
moment. Acharn asked them whether they saw anything right now. She asked
what seeing is and what is seen. She explained that we take what is seen
all the time as some ``thing''. She asked what can be known at the
moment of seeing. It is visible object that has impinged on the
eyesense. It is only a reality that can be seen, and it must be now.
Nobody can change it. Seeing arises just to see, it cannot be self. Each
moment has its proper conditions. She remarked:}}

\textsuperscript{``\textdutch{It is real, it appears now and it does not
belong to anyone. Nobody can make seeing arise, impossible.''}}

\textsuperscript{\textdutch{In the seventh book of the Abhidhamma, the
book of the }``Pa\textdutch{ṭṭhā}na''\textdutch{
}\textenglish[variant=american]{all possible relations between
phenomena}\textdutch{ have been
explained}\textenglish[variant=american]{. Each reality in our life can
only occur because of a concurrence of different conditions which
operate in a very intricate way. }}

\textsuperscript{\textenglish[variant=american]{These conditions are not
abstractions, they operate now, in our daily life. What we take for our
mind and our body are mere elements which arise because of their
appropriate conditions and are devoid of self. }\textdutch{In the planes
of existence where there are nāma and rūpa, kamma produces at the first
moment of life not only citta and cetasikas but also rūpa. Rūpas always
arise in groups (kalapas), and at that moment kamma produces three
groups of rūpa: one group with bodysense, one group with the heartbase
and one group with sex. The heartbase is the physical base of all cittas
apart from the sense-cognitions which have their appropriate
sense-bases, such as eyesense, as their base. It is called heartbase but
we do not need to think of ``heart'' in conventional sense. As to sex,
born as a male or female is due to kamma. }}

\textsuperscript{\textdutch{The element of heat is present in each group
of rūpa, and this element produces other groups of rūpa at the moment of
presence of the rebirth-consciousness
\protect\hyperlink{sdfootnote4sym}{\textsuperscript{4}}. }}

\textsuperscript{\textdutch{A group of rūpa consists of at least eight
rūpas arising and falling away together: the four great elements of
solidity, cohesion, temperature or heat, motion, as well as visible
object, odour, flavour and nutritive essence. Rūpas of the body can be
produced by kamma, citta, temperature or nutrition. Rūpas outside (not
of the body) are produced only by temperature. It is useful to learn
about such details, since it shows us that each rūpa in a group is
conditioned by the other rūpas in that group. For example, visible
objects are all dif-ferent because the accompanying rūpas in the same
group are of different qualities all the time. Understanding more about
conditions gradually eliminates the idea of being able to control
realities. }}

\textsuperscript{\textportuguese{Acharn}\textdutch{ explained that we
may hear many times about seeing, but that the idea of ``I see'' cannot
be eradicated soon. We were clinging to all realities during countless
lives. }\textenglish[variant=american]{The idea of self is so strong,
even if we say that there are only different realities.
}\textdutch{Acharn repeated that seeing is just a reality and that this
is the beginning of understanding it as not self. It is conditioned and
it arises and falls away. How could we control what arises and falls
away immediately. }}

\textsuperscript{\textdutch{The rebirth-consciousness is the first citta
in life produced by kamma and throughout our life kamma produces seeing,
hearing and the other sense-cognitions as well as the rūpas which are
the sense-bases. }\textenglish[variant=american]{The cittas which arise
are dependent on many different conditions. We tend to forget that
seeing is only a conditioned reality and that visible object is only a
conditioned reality, and }\textdutch{instead of developing understanding
of realities as they appear one at a time,
}\textenglish[variant=american]{we are easily carried away
by}\textdutch{ stories on account of what was seen or heard.
}\textenglish[variant=american]{Each citta experiences an object, be it
a sense object or }\textdutch{an }\textfrench{object}\textdutch{
appearing through the mind-door}\textenglish[variant=american]{, and the
object conditions citta by object-condition,
}\textdutch{ārammaṇa-}\textenglish[variant=american]{paccaya. It is
beneficial to remember that seeing, hearing and the other
sense-cognitions are
vip}\textdutch{ā}\textenglish[variant=american]{kacittas, cittas which
are results of kamma. They arise at their appropriate bases}\textdutch{
(}\textenglish[variant=american]{vatthus}\textdutch{)}\textenglish[variant=american]{,
which are also produced by kamma. }}

\textsuperscript{\textenglish[variant=american]{Hearing is conditioned
by sound which impinges on the earsense. Both sound and earsense are
r}\textdutch{ū}\textenglish[variant=american]{pas which also arise
because of their own conditions and fall away. Thus, hearing,
}\textdutch{a}\textenglish[variant=american]{ reality}\textdutch{ which
is conditioned by sound and earsense,}\textenglish[variant=american]{
cannot last either; it also has to fall away. Each conditioned reality
can exist just for an extremely short moment.
}\textdutch{T}\textenglish[variant=american]{here is no self who can
exert control over realities. When we move our hands, when we walk, when
we laugh or cry, when we are attached or worried, there }\textdutch{is
no self but only different realities arising because of conditions.}}

\textsuperscript{\textenglish[variant=american]{Cittas succeed one
another without any interval. The citta that has just fallen away
conditions the succeeding citta and this is by way of
proximity-condition}\textdutch{
(}anantara-paccaya\textdutch{)}\textenglish[variant=american]{. Seeing
arises time and again and after seeing has fallen away akusala cittas
usually arise. In each process of cittas there are, after the
sense-cognitions have fallen away, several moments of kusala cittas or
akusala cittas, }\textdutch{performing the function of javana
\protect\hyperlink{sdfootnote5sym}{\textsuperscript{5}}}\textenglish[variant=american]{
These experience the object in a wholesome way or unwholesome way. There
are usually seven javana-cittas and each preceding javana-citta
conditions the following one by way of repetition-condition
}\textdutch{(ā}\textitalian{sevana-paccaya}\textdutch{)}.}

\textsuperscript{\textenglish[variant=american]{We cling to visible
object, or we have wrong view about it, taking it for a being or a
person that really exists. Defilements arise because they have been
accumulated and they are carried on, from moment to moment, from life to
life. They are a natural decisive support-condition}\textdutch{
(}pakat\textdutch{ū}\textitalian{panissaya-paccaya}\textdutch{)}\textenglish[variant=american]{,
for akusala citta arising at this moment. }}

\textsuperscript{\textenglish[variant=american]{The
}``Pa\textdutch{ṭṭhāna}'' \textenglish[variant=american]{helps us to
understand the deep underlying motives for our behaviour and the
conditions for our defilements. It explains, for example, that kusala,
wholesomeness, can be the object of akusala citta, unwholesome citta. On
account of generosity which is wholesome, attachment, wrong view or
conceit, which are unwholesome realities, can arise. The
}``Pa\textitalian{ṭṭ}h\textdanish{ā}na''
\textenglish[variant=american]{also explains that akusala can be the
object of kusala, for example, when akusala is considered with
}\textdutch{right understanding as only a conditioned dhamma.
}\textenglish[variant=american]{This is an essential point which is
often overlooked. If one thinks that akusala cannot be object of
awareness and right understanding, the eightfold Path cannot be
developed. }\textdutch{One tries to ignore it instead of knowing it as
it really is: a conditioned dhamma. }}

\textsuperscript{\textdutch{People say that they try to have wholesome
thoughts, but when there is more understanding of conditions one will
see that it is impossible to try to direct one's thoughts in a certain
way. Thinking arises because of its proper conditions and it falls away
immediately.}}

\textsuperscript{\textdutch{Birth as a human or a deva is the result of
kusala kamma, whereas an unhappy rebirth such as in a hell plane or as
an animal is the result of akusala kamma. Usually people think of birth
as a situation, they think that a human being is born, or that a dog is
born. Actually, birth is only one moment of citta, vipākacitta produced
by kamma. People tend to be afraid of death since the rebirth following
upon it may be unhappy. If one is born as an animal one cannot develop
the understanding that eventually leads to the end of all defilements
and to the end rebirth. It is of no use to be afraid since it depends on
conditions what will be the next rebirth. This was one of the topics of
discussion in Taiwan before the sessions started with Acharn and the
whole group. The Buddha, when he, as a Bodhisatta, left his palace in
his last life, had a faithful horse Kandaka. Kandaka died and was reborn
as a deva; in that life he attained the first stage of enlightenment,
the stage of the sotāpanna. This shows that we can never know ahead of
time what will happen. }}

\textsuperscript{\textportuguese{Acharn}\textdutch{ explained that
}\textenglish[variant=american]{only one kamma produces
rebirth-consciousness, but }\textdutch{that in the rebirth-consciousness
}\textenglish[variant=american]{other kammas}\textdutch{ have been
accumulated}\textenglish[variant=american]{ which can produce result in
this life}\textdutch{
\protect\hyperlink{sdfootnote6sym}{\textsuperscript{6}}}.
\textdutch{Kamma needs the support of decisive natural dependent
condition in order to be able to produce result. This shows us how
complex the operation of kamma producing vipāka is. }}

\textsuperscript{\textenglish[variant=american]{In reality there are
just n}\textdutch{ā}\textenglish[variant=american]{mas and
r}\textdutch{ū}\textenglish[variant=american]{pas in our life that arise
because of different conditions. There is
accum}\textdutch{u}\textenglish[variant=american]{lation of kamma and of
good and bad qualities. What has been accumulated from life to life can
bring result and effect today. }}

\textsuperscript{\textenglish[variant=american]{During our discussions
it appeared that some people believe in a soul. The idea of a soul
suggests a mental reality that stays. However there are only
ever-changing citta and accompanying cetasikas that arise and fall away
all the time. There is no soul that passes away from this life and goes
to a next life.}}

\textsuperscript{\textdutch{Some people wonder why they have to receive
an unpleasant result produced by an ill deed done in a past life.
}\textenglish[variant=american]{It makes no sense to think in terms of
persons, or even of }``\textenglish[variant=american]{me" in a former
life or a future life, there are no persons. Just conditioned
realities.}\textdutch{ }\textenglish[variant=american]{There is
}\textdutch{a }\textenglish[variant=american]{connection of past,
present,}\textdutch{ and}\textenglish[variant=american]{ future. Only,
it is not }\textdutch{``us''}\textenglish[variant=american]{. Just
}\textdutch{dhammas}\textenglish[variant=american]{ rolling on, beyond
control.}}

\textsuperscript{\textdutch{The Buddha taught us to understand the
present reality so that there will be less confusion in life. He spoke
time and again about seeing. Seeing is vipākacitta, the result of kamma.
On account of what is seen attachment, aversion and ignorance are bound
to arise. We may worry about different situations and that is the
reality of aversion, dosa, different from vipāka. Thinking which arises
after the vipākacittas is most of the time akusala and it is important
to discern the difference between vipāka and thinking which follows so
closely. We worry mostly about ourselves, we cling to what is beneficial
for ourselves. }}

\textsuperscript{\textdutch{Acharn often reminds us that when we are
thinking of others, helping them in different ways there is no longer
worry about ourselves. When we see worry as only a passing dhamma we
shall attach less importance to our own benefit. }}

\textsuperscript{\textdutch{Acharn said once:
``}\textenglish[variant=american]{When one forgets about one's own
benefit, there can be kusala cittas thinking of the others, listening to
their useless or unpleasant talk. One thinks and thinks but it's only
thinking. Whatever happens does so by conditions. When one is flexible,
one helps the others more and forgets about one's own
}\textdutch{`}\textenglish[variant=american]{advantage}\textdutch{'},
\textdutch{in }\textenglish[variant=american]{developing kusala and
}\textdutch{there will be }\textenglish[variant=american]{less clinging
to oneself. }\textdutch{In t}\textenglish[variant=american]{his way one
can go anywhere, see anyone, help in any situation, listen to any
talk.}\textdutch{''}}

\textsuperscript{\textenglish[variant=american]{We can say many times
that seeing is not self, but there cannot be clear understanding of it
yet when pa}ñ\textdutch{ñā}\textenglish[variant=american]{ is at the
level of intellectual understanding, not yet direct understanding. This
is normal, and pa}ññ\textdutch{ā}\textenglish[variant=american]{ has to
begin, a little at the time. It can be understood that it is just a
reality. That is the way it develops. Acharn reminds us that there was
clinging to seeing as }``\textenglish[variant=american]{I see}''
\textenglish[variant=american]{during so many lives. She always says:
}\textdutch{``T}\textenglish[variant=american]{he study of what the
Buddha taught has to be at this
moment.}\textdutch{''}\textenglish[variant=american]{ What is seeing
now, hearing now, thinking now?}}

\textsuperscript{\textdutch{We read in ``Duality 2'' (``Kindred
Sayings'', Second Fifty, § 93)about conditions for the arising of
realities:}}

\textsuperscript{``\textdutch{Owing to a dual (thing), brethren,
consciousness comes into being. And what, brethren, is that dual owing
to which }\textenglish[variant=american]{consciousness comes into
being}\textdutch{?}}

\textsuperscript{\textdutch{Owing to the eye and objects arises
eye-consciousness. The eye is impermanent, changing, its state is
`becoming otherness'. Thus this dual, mobile and transitory,
impermanent, changing,- its state is
}`\textenglish[variant=american]{becoming otherness}'.}

\textsuperscript{\textdutch{Eye-consciousness
}\textenglish[variant=american]{is impermanent, changing, its state is
}`\textenglish[variant=american]{becoming otherness}'.\textdutch{ That
condition, that relation of the uprising of eye-consciousness,-they also
are impermanent, changing, their state is
}`\textenglish[variant=american]{becoming otherness}'.\textdutch{ This
eye-consciousness, arising as it does from an impermanent relation,- how
could it be permanent?}}

\textsuperscript{\textdutch{Now the striking together, the falling
together, the meeting together of these three things,- this, brethren,
is called `eye-contact.' Eye-Contact is impermanent, changing,- its
state is }`\textenglish[variant=american]{becoming
otherness}'.\textdutch{ That condition, that relation of the uprising of
eye-contact,- they also are impermanent\ldots{} This eye-contact,
arising as it does from an impermanent relation,- how could it be
permanent?}}

\textsuperscript{\textdutch{Contacted, brethren, one feels. Contacted,
one is aware. Contacted, one perceives. Thus these states also are
mobile and transitory, impermanent and changing. Their state is
}`\textenglish[variant=american]{becoming
otherness}'\textdutch{\ldots{}''}}

\textsuperscript{\textdutch{The same is said with regard to the other
doorways. }}

\textsuperscript{\textdutch{**********}}

\textsuperscript{\textdutch{Chapter 5}}

\textsuperscript{\textdutch{Different Aspects of Dhamma.}}

\textsuperscript{\textdutch{P}\textenglish[variant=american]{eople may
not see that Abhidhamma is about real life, that it is not theory at
all. }\textdutch{The Abhidhamma teaches about all realities of daily
life. We are inclined to take for wholesome what is unwholesome, we are
ignorant of realities. Through the Abhidhamma we come to understand what
is kusala and what is akusala and what }}

\textsuperscript{\textdutch{are the conditions for their arising.
}\textenglish[variant=american]{When people feel happy they believe that
this is good and wholesome. But through the Abhidhamma we learn that
happy feeling can be selfish or that it can be noble and wholesome. It
depends on what type of citta happy feeling accompanies. We are happy to
be in the company of dear people, parents and friends. But this may be
selfish affection, not wholesome. We just think of our own liking. On
the other hand, happy feeling can arise with wholesome consciousness.
One does not think of one's own pleasure or gain, but
}\textdutch{only}\textenglish[variant=american]{ about someone else's
welfare, wanting to help him. This is just an example showing that we
may mislead ourselves all the time, taking for unselfish love what is in
reality selfish affection.}~}

\textsuperscript{\textenglish[variant=american]{We mourn the loss of
dear people, but it is good to know what type of citta arises. We are
unhappy because we are no longer in the company of dear
}\textdutch{people}\textenglish[variant=american]{ we
}\textdutch{love}\textenglish[variant=american]{. This shows again how
much we cling to our own feeling. When we do not get what we like we
have aversion. Our whole life we search for what we like. }\textdutch{We
find o}\textenglish[variant=american]{ur liking and
disliking}\textdutch{ most important, but they are only passing
realities that arise for an extremely short moment and then fall away.
We should remember the following sutta from
}\textdutch{t}\textenglish[variant=american]{he }\textdutch{``}Sutta
Nipata\textdutch{'' (}\textitalian{Selected Texts from the Sutta Nipata,
}\textdutch{``The Arrow'', }\textenglish[variant=american]{translated by
John D. Ireland}\textdutch{. }\textenglish[variant=american]{Kandy:
Buddhist Publication Society, 1983)}\textdutch{.}{{\\
}}\textenglish[variant=american]{"Unindicated and unknown is the length
of life of those subject to death. Life}\textdutch{
}\textenglish[variant=american]{is difficult and brief and bound up with
suffering. There is no means by which}\textdutch{
}\textenglish[variant=american]{those who are born will not die. Having
reached old age, there is death. This is}\textdutch{
}\textenglish[variant=american]{the natural course for a living being.
With ripe fruits there is the constant}\textdutch{
}\textenglish[variant=american]{danger that they will fall. In the same
way, for those born and subject to}\textdutch{
}\textenglish[variant=american]{death, there is always the fear of
dying. Just as the pots made by a potter all}\textdutch{
}\textenglish[variant=american]{end by being broken, so death is (the
breaking up) of life.}{{\\
}}\textenglish[variant=american]{"The young and old, the foolish and the
wise, all are stopped short by the power}\textdutch{
}\textenglish[variant=american]{of death, all finally end in death. Of
those overcome by death and passing to}\textdutch{
}\textenglish[variant=american]{another world, a father cannot hold back
his son, nor relatives a relation. See!}\textdutch{
}\textenglish[variant=american]{While the relatives are looking on and
weeping, one by one each mortal is led}\textdutch{
}\textenglish[variant=american]{away like an ox to slaughter.}{{\\
}}\textenglish[variant=american]{In this manner the world is afflicted
by death and decay. But the wise do not}\textdutch{
}\textenglish[variant=american]{grieve, having realized the nature of
the world. You do not know the path by}\textdutch{
}\textenglish[variant=american]{which they came or departed. Not seeing
either end you lament in vain. If an}\textdutch{y
}\textenglish[variant=american]{benefit is gained by lamenting, the wise
would do it. Only a fool would harm}\textdutch{
}\textenglish[variant=american]{himself. Yet through weeping and
sorrowing the mind does not become calm, but}\textdutch{
}\textenglish[variant=american]{still more suffering is produced, the
body is harmed and one becomes lean and}\textdutch{
}\textenglish[variant=american]{pale, one merely hurts oneself. One
cannot protect a departed one (peta) by that}\textdutch{
}\textenglish[variant=american]{means. To grieve is in
vain.}\textdutch{''}}

\textsuperscript{\textdutch{As we read:
``}\textenglish[variant=american]{But the wise do not}\textdutch{
}\textenglish[variant=american]{grieve, having realized the nature of
the world.}\textdutch{'' The world is actually just dhammas that arise
and fall away. }}

\textsuperscript{\textdutch{Acharn often asked us during the
discussions: ``What is life?'' Life only lasts one moment, a moment of
citta that experiences an object.}}

\textsuperscript{\textdutch{We read in the ``Visuddhimagga'', `` The
Path of Purification'' (}VIII, 39\textdutch{): }}

\textsuperscript{\textenglish[variant=american]{"In the absolute sense,
beings have only a very short moment to live, life lasting as long as a
single moment of consciousness lasts. Just as a cart-wheel, whether
rolling or whether at a standstill, at all times only rests on a single
point of its periphery, even so the life of a living being lasts only
for the duration of a single moment of consciousness. As soon as that
moment ceases, the being also ceases. For it is said: 'The being of the
past moment of consciousness has lived, but does not live now, nor will
it live in future. The being of the future moment has not yet lived, nor
does it live now, but it will live in the future. The being of the
present moment has not lived, it does live just now, but it will not
live in the future.}\textdutch{' "}}

\textsuperscript{\textdutch{Before hearing the teachings we believed
that we knew a great deal about life, but after learning what the Buddha
taught we come to realize that we have ignorance about the most common
realities of life. We think that there is seeing all the time when our
eyes are open, the world seems to be bright. The Buddha taught about
cittas that experience different object at different moments, one at a
time. When there is seeing, there is no hearing or thinking at the same
time. When visible object is experienced by seeing the world is bright,
but when hearing arises sound is experienced and the world is dark. At
all other moments apart from seeing the world is dark, but we mistakenly
think that brightness lasts. Brightness seems to be all around us all
the time. Cittas arise and fall away extremely rapidly and that is why
it seems that seeing can stay on. }}

\textsuperscript{\textdutch{In order to help people to see the truth the
Buddha taught many aspects of realities. He taught realities under the
aspect of elements, of khandhas, of āyātanas (sensefields), of the
Dependent Origination (Paṭicca Samuppāda), and in many other ways. }}

\textsuperscript{\textdutch{The Dependent Origination deals with the
conditionality of all mental phenomena, nāma, and physical phenomena,
rūpa, of life. Ignorance is the first link of the Dependent Origination.
Because of ignorance we have to be reborn again and again. It conditions
kamma that produces vipāka and because of vipāka defilements arise on
account of the objects experienced through the senses and the mind-door.
Defilements condition kamma that produces vipāka again in the form of
rebirth and sense-cognitions during life. In that way the cycle of birth
and death continues. }}

\textsuperscript{\textdutch{The Dependent Origination is not theory, it
pertains to our life now. During our discussions in Taiwan Acharn asked
us a number of questions in order to remind us of reality now. She asked
us about avijjā, ignorance: }}

\textsuperscript{''\textdutch{Is avijjā a reality? Does it belong to
you? How can there be conditions for understanding
}avijj\textdanish{ā}\textdutch{ right now? What is it that
}avijj\textdanish{ā}\textdutch{ does not understand? It does not know
seeing that arises and falls away. It does not really know what seeing
is. Understanding can know what seeing is. Without eyes can there be
seeing? Usually we take it for something or for `I'. There is the idea
of `I see'. Now we talk about the Dependent Origination but we do no
mention this term. Is the visible object that is seen also Dependent
Origination? The Dependent Origination is all about now.''}}

\textsuperscript{\textdutch{The purpose of teaching the Dependent
Origination is to help beings understand that there is no self who
travels from a past life to this life and again to future lives; no self
who is ignorant, who feels, who clings.}}

\textsuperscript{\textdutch{Another subject discussed during our
sessions was the ``āyatanas''. T}\textenglish[variant=american]{he
P}\textdanish{ā}\textenglish[variant=american]{li term
}``āyatana''\textdutch{ is sometimes translated as
}``\textdutch{s}\textenglish[variant=american]{phere of
sense}''\textdutch{, or ``sense-field''.}\textenglish[variant=american]{
We read in the }``\textenglish[variant=american]{Book of Analysis}''
(Vibha\textitalian{ṅ}\textenglish[variant=american]{ga), the second book
of the Abhidhamma
\protect\hyperlink{sdfootnote7sym}{\textsuperscript{7}}}\textitalian{,
in }``\textenglish[variant=american]{Analysis of the
Bases}''\textenglish[variant=american]{, about the twelve
}\textdanish{ā}\textenglish[variant=american]{yatanas, here translated
as }``\textenglish[variant=american]{bases}''. \textdutch{There are
inner }\textdanish{ā}yatanas\textdutch{ and outward
}\textdanish{ā}yatanas\textdutch{. The inner
}\textdanish{ā}yatanas\textdutch{ are
}\textenglish[variant=american]{the eye, the ear, the nose, the tongue,
the body, }\textdutch{and }\textenglish[variant=american]{the
mind}\textdutch{, which includes all cittas. The outer
}\textdanish{ā}yatanas\textdutch{ are:
}\textenglish[variant=american]{visible object, sound, odour, flavour,
tangible object and }\textdutch{dhamm}\textdanish{ā}yatana.
\textdutch{Dhamm}\textdanish{ā}yatana \textdutch{includes: cetasikas,
subtle rūpas and nibbāna. Ā}yatana\textdutch{ deals with the association
of inner and outer }\textdanish{ā}yatanas\textdutch{. Thus, āyatana
always refers to the meeting, the association of several realities. When
seeing arises there is the association of eyesense, visible object,
seeing and its accompanying cetasikas. These are only āyatanas at the
moment they arise and associate. T}\textenglish[variant=american]{he
classification by way of }\textdanish{ā}yatanas \textdutch{shows us that
seeing and all the other realities are associating because of
conditions. Nobody can make them arise, they are not self.
}\textenglish[variant=american]{Here we see again that the Abhidhamma
points to the goal, the development of right understanding.}}

\textsuperscript{\textdutch{During our discussions it was emphasized
time and again that there are two kind of realities: those that
experience an object (nāma) and those that do not experience anything
(rūpa). We can understand this intellectually, but when paññā is more
developed the difference between nāma and rūpa can be directly known.
}\textenglish[variant=american]{Because of our accumulated ignorance we
confuse visible object and seeing, sound and hearing, we cannot clearly
distinguish their different characteristics when they appear. So long as
we do not distinguish n}\textdanish{ā}\textenglish[variant=american]{ma
and r}ū\textenglish[variant=american]{pa from each other we will not be
able to realize their arising and falling away. Then we will continue to
cling to the idea of beings or things which last. }}

\textsuperscript{\textdutch{The first stage of insight is knowing the
difference between the characteristic of nāma and of rūpa. We discussed
this during our sessions. We may be inclined to see this stage in an
abstract way but it pertains to any nāma or rūpa appearing now. As
Acharn had explained before: ``}\textenglish[variant=american]{Actually,
we don't even have to use the words
n}{{ā}}\textenglish[variant=american]{ma and
r}{{ū}}\textenglish[variant=american]{pa. When there is the
unders}\textdutch{t}\textenglish[variant=american]{anding of all kinds
of realities, those which experience an object and those which don't,
the understanding of n}{{ā}}\textenglish[variant=american]{ma and
r}{{ū}}\textenglish[variant=american]{pa is there already.}}

\textsuperscript{\textdutch{T}\textenglish[variant=american]{hey are
different but }\textdutch{there is }\textenglish[variant=american]{no
need to name them. They are different already by their nature. }}

\textsuperscript{\textenglish[variant=american]{That which sees is the
characteristic of n}\textdutch{ā}\textenglish[variant=american]{ma, now.
In that way one can become closer to understanding that it is
n}\textdutch{ā}\textenglish[variant=american]{ma, instead of knowing
beforehand, this is n}\textdutch{ā}\textenglish[variant=american]{ma,
this is r}\textdutch{ū}\textitalian{pa.}\textdutch{''}}

\textsuperscript{\textdutch{Sarah explained:
``}\textenglish[variant=american]{We may be thinking this is
n}\textdutch{ā}\textenglish[variant=american]{ma, that is
r}\textdutch{ū}\textenglish[variant=american]{pa, but the reality right
now has its own characteristic as
n}\textdutch{ā}\textenglish[variant=american]{ma or as
r}\textdutch{ū}\textitalian{pa.
}\textdutch{N}{{ā}}\textenglish[variant=american]{ma and
r}{{ū}}\textenglish[variant=american]{pa are just the realities now.
Seeing now, hearing now, lobha now are
n}{{ā}}\textenglish[variant=american]{mas, so the understanding of
n}{{ā}}\textenglish[variant=american]{ma is just the understanding of
any of these dhammas}\textdutch{
}\textenglish[variant=american]{appearing now. It's the same for
r}{{ū}}\textenglish[variant=american]{pas. The understanding of
r}{{ū}}\textenglish[variant=american]{pa is just the understanding of
visible object, sound, hardness or any other
r}{{ū}}\textenglish[variant=american]{pa appearing now.}\textdutch{
}\textenglish[variant=american]{If there is no development of
understanding and awareness of these realities in daily life now, there
can never be vipassan}{{ā}}
ñ{{ā}}\textdutch{ṇ}\textenglish[variant=american]{a, the direct and
clear understanding of dhammas.}\textdutch{''}}

\textsuperscript{\textenglish[variant=american]{When there can be
awareness of one reality at a time appearing through one doorway we will
begin to understand the present moment. }\textdutch{We can begin to
understand what the meaning is of experiencing an object, of nāma. We
learnt that kusala citta is different from akusala citta but when paññā
is of the level of intellectual understanding we do not clearly know the
truth of these realities. }}

\textsuperscript{\textdutch{Life is just citta, cetasika and rūpa,
arising and falling away. When we are walking, moving around, or
talking, there are citta, cetasika and rūpa, arising because of their
proper conditions. }}

\textsuperscript{\textdutch{Citta conditions different groups of rūpa.
As we have seen, each group of rūpa contains at least eight rūpas, and
citta can condition such groups. Moreover, citta conditions together
with the group of these eight rūpas bodily intimation,} k{{ā}}ya
viññ\textitalian{atti}\textdutch{, and it also conditions
}\textenglish[variant=american]{together with the group of these eight
r}ū\textfrench{pas}\textdutch{ and sound the rūpa which is speech
intimation, vacī viññatti. When we convey a meaning by gestures or
speech it is conditioned by citta. This reminds us that it is not a self
who does so. In the groups with bodily intimation and speech intimation
there can in addition be three vikāra
rūpas\protect\hyperlink{sdfootnote8sym}{\textsuperscript{8}} of
lightness (}(lahut\textdanish{ā})\textdutch{,
plasticity(}(mudut\textdanish{ā}\textdutch{)}\textdutch{ and
wieldiness(}(kammaññat\textdanish{ā})\textdutch{. These three rūpas
ensure the suppleness of the other rūpas in these groups.
}\textenglish[variant=american]{For the movement of the body and the
performance of its functions, these three qualities are essential.}}

\textsuperscript{\textdutch{Sarah explained that when we walk along
there may not be the intention to convey a meaning, but when there is a
deliberate movement to show a purpose such as in physiotherapy there is
bodily intimation. She gave an example of bodily intimation while typing
to convey a meaning or typing for one's own }\textdutch{use. She gave an
example of speech intimation as ``t}\textenglish[variant=american]{he
`deliberate' making of sound with movement, `conveying of a meaning'
such as when speaking, singing, chanting, making special noises whether
to others or oneself.}\textdutch{'' }}

\textsuperscript{\textdutch{Sixteen types of citta do not condition rūpa
and these are: the five pairs of sense-cognitions (five being kusala
vipāka and five being akusala vipāka), the rebirth-consciousness, the
dying-consciousness of the arahat and the four arūpa-jhāna vipākacittas.
}}

\textsuperscript{\textdutch{Someone remarked that Acharn in her early
days would often speak about awareness of the level of satipaṭṭhāna
whereas today she explains time and again about the different levels of
paññā which are pariyatti (intellectual understanding), paṭipatti
(direct understanding) and paṭivedha (direct realization by the stages
of insight and in enlightenment). Paṭipatti is paññā of the level of
direct understanding, but pariyatti, intellectual understanding of what
appears now has to be developed on and on until it is very firm so that
it can condition direct understanding. The texts of the Tipiṭaka are all
about seeing now, hearing now, but this is difficult for all of us. If I
had not listened to Acharn and discussed realities with Dhamma friends I
would be forgetful all the time. }}

\textsuperscript{\textdutch{Not only now but also a long time ago Acharn
would explain about the different levels of paññā even though she did
not always use the terms }pariyatti,
pa\textitalian{ṭ}\textenglish[variant=american]{ipatti and
pa}\textitalian{ṭ}\textenglish[variant=american]{ivedha}\textdutch{. The
Buddha explained these levels of paññā in different ways. After his
enlightenment he explained that there are three rounds of understanding
the noble truths: }\textitalian{sacca
}ña{{̄}}n{{̣}}\textenglish[variant=american]{a, the firm understanding of
what has to be known and what the Path is; kicca
}ña{{̄}}n{{̣}}\textenglish[variant=american]{a, understanding of the task,
that is, satipat}{{̣}}t{{̣}}ha{{̄}}na; kata
ña{{̄}}n{{̣}}\textenglish[variant=american]{a, understanding of what has
been realized, the realization of the truth}\textdutch{
\protect\hyperlink{sdfootnote9sym}{\textsuperscript{9}}}.}

\textsuperscript{\textdutch{When intellectual understanding has
be}\textenglish[variant=american]{come firm and more accomplished it
}\textitalian{is sacca }ña{{̄}}n{{̣}}\textenglish[variant=american]{a.
Then one does not move away from the dhamma appearing right now and turn
to other practices in order to understand the truth. Sacca
}ña{{̄}}n{{̣}}\textenglish[variant=american]{a realizes that every dhamma
that arises is conditioned. }}

\textsuperscript{\textdutch{T}\textenglish[variant=american]{he
Buddha}\textdutch{, by teaching the three rounds of understanding,
reminded us that there is a very gradual development of paññā. It takes
a long time for paññā to become direct understanding of realities.
}\textenglish[variant=american]{We need patience to gradually develop
understanding of the dhamma appearing now.}}

\textsuperscript{\textdutch{Someone asked: ``What is vipassanā?'' It is
the development of understanding of realities appearing now. People
would ask what Acharn's method is. There is no rule to be followed, no
method. Pariyatti always pertains to this moment. If we think of a
special method we are clinging to an idea of self. We never know the
next moment, it may be fear or dislike. Life is just the experience of
this moment. This moment falls away immediately never to return. She
emphasized that throughout life there are just different realities, some
realities that can experience an object and some that do not experience
anything.}}

\textsuperscript{\textdutch{During the discussions in Kaeng Kracan that
took place after the sessions in Taiwan Acharn asked: ``what is
dhamma?''. She said: ``The best thing is just understanding the word
dhamma. It is reality and includes anything which is real. We do not
know reality at all until we hear the word of the Buddha.'' She
explained that dhamma means ``Not I, not anyone, nobody.'' }}

\textsuperscript{\textdutch{Visible object appears but it does not
appear yet as a dhamma, we take it for something or someone. As soon as
seeing arises we believe that we see people and things. }}

\textsuperscript{\textdutch{Tadao asked what sampajañña is. We read
about sati sampajañña, sati and paññā. Sampajañña is not inellectual
understanding. Pariyatti is intellectual understanding of the present
moment and when it is very firm it conditions paṭipatti, direct
understanding. This is the beginning of sampajañña. Paññā knows that
there is no self at all and that is sampajañña.}}

\textsuperscript{\textdutch{Acharn explained that realities ``appear
well'' \protect\hyperlink{sdfootnote10sym}{\textsuperscript{10}} to
paññā that has been developed to direct understanding through
satipaṭṭhāna. At this moment visible object does not appear well, we
have doubts about what it is. Without sati it does not appear well. We
can see the difference between appearing well and not appearing well.
Without paññā it cannot appear well. }}

\textsuperscript{\textdutch{Paññā does not appear yet because it is so
very slight. But when there is sati it appears well. A reality is
experienced well by sati. Without that moment it is only intellectual
understanding. }}

\textsuperscript{\textdutch{Several times Acharn asked us why we study
the teachings. People have different aims. Some people want to become a
better person with less akusala. Or they wish to have more calm and
happiness in their lives. Then they cling to an idea of self with less
akusala and more happiness. }}

\textsuperscript{\textdutch{Through the Buddha's teachings we learn what
life really is. It is different from what we always thought before
hearing his teachings. We used to think that all events in life last for
a while, being sometimes pleasant, sometimes unpleasant. It takes a long
time to develop understanding of the truth. We are usually quite
absorbed in what is not reality and we tend to think of our experiences
with attachment or aversion, we are inclined to take them for self. We
want happiness to last. Through the Buddha's teachings we learn that
life is only one moment of experiencing an object through one doorway at
a time. Each moment passes away completely. Whatever arises does so
because of conditions and it cannot be controlled. }}

\textsuperscript{\textdutch{When right understanding of all that is real
is being developed there will be less ignorance of what life really is.
This does not mean that we do not cling to the idea of self. Right
understanding that is still of the level of intellectual understanding
does not eradicate the wrong view of self. But when it is firmly
established it can lead to direct understanding of the truth that life
is only in one moment and then gone forever. }}

\textsuperscript{\textdutch{*****}}

\hypertarget{sdfootnote1}{}
\textsuperscript{\protect\hyperlink{sdfootnote1anc}{1} \textdutch{The
Pali text has nimitta.}}

\hypertarget{sdfootnote2}{}
\textsuperscript{\protect\hyperlink{sdfootnote2anc}{2} \textdutch{A
blind person who touches its tusk thinks that this is the elephant, and
the same for someone who touches its tail, leg or other parts. }}

\hypertarget{sdfootnote3}{}
\textsuperscript{\protect\hyperlink{sdfootnote3anc}{3} \textdutch{We
usually think of kamma as an unwholesome or wholesome deed, but
actually, it is cetanā, volition. }}

\hypertarget{sdfootnote4}{}
\textsuperscript{\protect\hyperlink{sdfootnote4anc}{4} \textdutch{Three
``submoments'' of citta can be discerned: the moment of its arising, of
its presence and of its ceasing. }}

\hypertarget{sdfootnote5}{}
\textsuperscript{\protect\hyperlink{sdfootnote5anc}{5}
\textdutch{Javana is sometimes translated as impulsion. It can mean:
going. In the case of the arahat they are neither kusala nor akusala,
they are kiriyacittas, inoperative cittas. }}

\hypertarget{sdfootnote6}{}
\textsuperscript{\protect\hyperlink{sdfootnote6anc}{6}
\textdutch{āyūhana is a term denoting the accumulated kamma. }}

\hypertarget{sdfootnote7}{}
\textsuperscript{\protect\hyperlink{sdfootnote7anc}{7}
\textenglish[variant=american]{This book can be read together with its
commentary, the
}``\textenglish[variant=american]{Sammohavinodan}ī''\textenglish[variant=american]{,
attributed to Buddhaghosa and translated as
}``\textenglish[variant=american]{The Dispeller of
Delusion}''\textenglish[variant=american]{, in two volumes. The
commentary is most helpful for the understanding of the Abhidhamma, that
is, the understanding of one}'\textenglish[variant=american]{s own life.
Buddhaghosa illustrates the meaning of the realities taught in the
}``\textenglish[variant=american]{Book of Analysis}''
\textenglish[variant=american]{in a lively way with examples from daily
life. }}

\hypertarget{sdfootnote8}{}
\textsuperscript{\protect\hyperlink{sdfootnote8anc}{8}\textdutch{
Vikāra means changeability. }\textenglish[variant=american]{These three
r}ū\textenglish[variant=american]{pas are
r}ū\textenglish[variant=american]{pas without a distinct nature,
}\textdutch{they are }\textgerman{asabh}\textdanish{ā}va
rū\textenglish[variant=american]{pas; they are qualities of
r}ūpa\textdutch{ that can arise whenever there is bodily movement. }}

\hypertarget{sdfootnote9}{}
\textsuperscript{\protect\hyperlink{sdfootnote9anc}{9}\textdutch{
Kindred Sayings, V, Book XII, Ch II, 1: ``The setting rolling of the
wheel of Dhamma''.}}

\hypertarget{sdfootnote10}{}
\textsuperscript{\protect\hyperlink{sdfootnote10anc}{10} \textdutch{In
Pali: suvidita, well known. }}
