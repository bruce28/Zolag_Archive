% This file was converted to LaTeX by Writer2LaTeX ver. 0.4
% see http://www.hj-gym.dk/~hj/writer2latex for more info
\documentclass[12pt,twoside]{article}
\usepackage[ascii]{inputenc}
\usepackage[T1]{fontenc}
\usepackage[english]{babel}
\usepackage{amsmath,amssymb,amsfonts,textcomp}
\usepackage{color}
\usepackage{calc}
\usepackage{hyperref}
\hypersetup{colorlinks=true, linkcolor=blue, filecolor=blue, pagecolor=blue, urlcolor=blue}
% Pages styles (master pages)
\makeatletter
\newcommand\ps@Standard{%
\renewcommand\@oddhead{}%
\renewcommand\@evenhead{}%
\renewcommand\@oddfoot{}%
\renewcommand\@evenfoot{}%
\setlength\paperwidth{20.999cm}\setlength\paperheight{29.699cm}\setlength\voffset{-1in}\setlength\hoffset{-1in}\setlength\topmargin{2cm}\setlength\headheight{12pt}\setlength\headsep{0cm}\setlength\footskip{12pt+0cm}\setlength\textheight{29.699cm-2cm-2cm-0cm-12pt-0cm-12pt}\setlength\oddsidemargin{2cm}\setlength\textwidth{20.999cm-2cm-2cm}
\renewcommand\thepage{\arabic{page}}
\setlength{\skip\footins}{0.101cm}\renewcommand\footnoterule{\vspace*{-0.018cm}\noindent\textcolor{black}{\rule{0.25\columnwidth}{0.018cm}}\vspace*{0.101cm}}
}
\makeatother
\pagestyle{Standard}
\begin{document}
Introduction


\bigskip

By Nina van Gorkom


\bigskip

Ms. Sujin{\textquotesingle}s approach to the development of loving
kindness, mett{\aa} is very direct. She aims at the practical
application of the Buddha{\textquotesingle}s teachings in daily life.
The Buddha teaches the way to eradicate what is unwholesome and to
develop what is wholesome. Mett{\aa} is an essential way of
wholesomeness. However, it is difficult to develop it because we
usually think of ourselves. The development of mett{\aa} is most
beneficial both to ourselves and others: it will lead to less
selfishness and it is of vital importance to harmony and peace in
society. 

This book is a compilation of a series of lectures held in the
Bovoranives Temple in Bangkok. During her lectures questions were
brought up and therefore part of this book contains questions and
answers. Khun Sujin
({\textasciigrave}{\textasciigrave}Khun{\textquotesingle}{\textquotesingle}
is the Thai equivalent of Ms. and of Mr.) explains that in order to
develop mett{\aa} we have to know exactly what it is and when it
arises. We are likely to take selfish affection for mett{\aa} and then
mett{\aa} cannot be developed. Khun Sujin quotes from the Buddhist
scriptures, the
{\textasciigrave}{\textasciigrave}Tipi\`iaka{\textquotesingle}{\textquotesingle},
and from commentaries in order to illustrate the practice of mett{\aa}
as taught by the Buddha. She always stresses that the practice should
conform to the teachings of the Buddha as they have come to us at
present. Formerly people were used to accepting what their teachers
said without going themselves to the source of the teachings, the
Tipi\`iaka. Khun Sujin has always greatly encouraged people to read the
scriptures themselves, to consider them and to prove the truth to
themselves through the practical application of the Dhamma, the
Buddha{\textquotesingle}s teachings, in daily life. When she started
her lectures more than twenty five years ago, there were hardly any
Thai translations of the commentaries to the Tipi\`iaka. Each time she
needed the commentary to the suttas she was going to quote during her
next lecture, she asked one of the monks for a translation from
P{\aa}li into Thai of the corresponding parts of the commentaries. It
was also due to her encouragement that the interest in the teachings
and commentaries of both monks and lay people grew and more
commentaries were translated into Thai. At present the Tipi\`iaka has
been reprinted in Thailand in such a way that each sutta or each
section is immediately followed by the corresponding commentary which
gives the explanation of that text. Khun Sujin helps people not only to
investigate the sources of the teachings but also to have right
understanding of the application of the teachings, the practice in
daily life. Her lectures can be heard on the radio, morning and
evening. The radio stations which send out her lectures are in
different parts of Thailand and their number increases. Her lectures
can also be heard in neighbouring countries, such as Laos, Malaysia and
Cambodia. With my deepest appreciation of Khun Sujin{\textquotesingle}s
guidance and support and with great pleasure I offer the translation of
this book on Mett{\aa} to the English speaking readers. I made a free
translation adapted to {\textasciigrave}{\textasciigrave}Western
people{\textquotesingle}{\textquotesingle} with some changes, additions
and footnotes. The Thais are familiar with P{\aa}li terms and their
meanings, but these are difficult to understand for those who begin to
study the Buddhist teachings. In order to help the reader to understand
this book I will now explain a few notions in this book which are
essential for understanding mett{\aa} and for its application in daily
life. 

Mett{\aa} cannot be developed if people do not know their own
{\textasciigrave}{\textasciigrave}mental
states{\textquotesingle}{\textquotesingle}, in P{\aa}li: cittas. What
we take for {\textasciigrave}{\textasciigrave}my
mind{\textquotesingle}{\textquotesingle} are actually many different
moments of consciousness, cittas, which change all the time. There is
only one citta at a time which arises and then falls away immediately,
to be succeeded by the next citta. Our life is an unbroken series of
cittas arising in succession. Each citta experiences an object. Seeing
is a citta which experiences colour through the eye{}-sense. Hearing is
a citta which experiences sound through the ear{}-sense. Cittas
experience objects through the six doorways of eyes, ears, nose,
tongue, body{}-sense and mind. 

Cittas are variegated: some cittas are wholesome, kusala cittas, some
are unwholesome, akusala cittas, some are neither kusala nor akusala.
When there is mett{\aa} with the citta, the citta is kusala, but when
there is selfish affection or anger with the citta, the citta is
akusala. There is one citta at a time, but each citta is accompanied by
several mental factors, cetasikas, which each perform their own
function while they accompany the citta. Some cetasikas, such as
feeling and remembrance, accompany each citta, while other types of
cetasikas can accompany only akusala cittas or only kusala cittas.
Akusala cittas are accompanied by unwholesome mental factors, such as
attachment, lobha, or aversion, dosa, and kusala cittas are accompanied
by {\textasciigrave}{\textasciigrave}beautiful mental
factors{\textquotesingle}{\textquotesingle} such as generosity or
mett{\aa}. 

Cittas and cetasikas are realities which arise because of their
appropriate conditions. For example, wholesome qualities and
defilements which arose in the past can condition the arising of such
qualities at present. Cittas arise and then fall away, but since each
citta is succeeded by the following one, wholesome qualities and
defilements can be accumulated from moment to moment, and thus there
are conditions for their arising at the present time. 

Seeing, hearing, smelling, tasting and the experience of bodily
impressions are cittas which are neither kusala nor akusala, they are
cittas which are results of kamma, vip{\aa}kacittas. Unwholesome deeds
and wholesome deeds done in the past can bring about pleasant and
unpleasant results at present. Kamma is actually intention or volition.
The unwholesome or wholesome volition which motivates a deed is
accumulated from moment to moment and thus it can produce result later
on. Kamma produces result in the form of rebirth{}-consciousness or, in
the course of life, in the form of seeing, hearing and the other
sense{}-impressions. Seeing, hearing and the other sense impressions
experience pleasant and unpleasant objects, depending on the kamma
which produces these vip{\aa}kacittas. When a pleasant object is seen,
attachment is likely to arise after the seeing, and when an unpleasant
object is seen, aversion is likely to arise after the seeing. The sense
impressions are followed by akusala cittas more often than by kusala
cittas. There is no self who can direct or control the cittas which
arise, they arise because of their own conditions, they are non{}-self,
anatt{\aa}. Right understanding of the different cittas which arise is
the factor which can condition the development of more wholesome
qualities. 

The Buddha taught that what we take for
{\textasciigrave}{\textasciigrave}our
mind{\textquotesingle}{\textquotesingle} and
{\textasciigrave}{\textasciigrave}our
body{\textquotesingle}{\textquotesingle} are ever{}-changing phenomena
which arise and then fall away immediately, they are impermanent and
anatt{\aa}. Citta and cetasika, consciousness and mental factors, are
mental phenomena, in P{\aa}li: n{\aa}ma. Physical phenomena are called
in P{\aa}li: r\'upa. N{\aa}ma and r\'upa are ultimate realities, or
absolute realities. N{\aa}mas such as seeing, mett{\aa} or anger, and
r\'upas such as colour, sound or hardness, are ultimate realities. They
each have their own characteristic which can be directly experienced
when it appears. They are real for everybody. Their names can be
changed, but their characteristics cannot be changed. There is ultimate
truth and there is conventional truth. Without the study of the
Buddha{\textquotesingle}s teachings one knows only conventional truth:
the world of person, being, self, trees and cars. These are concepts we
can think of, but they are not ultimate realities which can be directly
experienced. Conventional truth is not denied in Buddhism, but the
difference between ultimate truth and conventional truth is pointed out
so that they can be distinguished from each other. Even when we have
understood that what we take for a person consists of n{\aa}ma and
r\'upa which arise and fall away, we can still think of persons. We can
think of them in the unwholesome way: with selfish affection or with
anger, or in the wholesome way: with mett{\aa} or with compassion. 

Buddhism teaches different ways of wholesomeness: d{\aa}na or
generosity, s\'ila or good moral conduct and bh{\aa}van{\aa} or mental
development, which includes the development of tranquillity, samatha,
and the development of insight, vipassan{\aa}. There are different
types of kusala cittas, sometimes they are accompanied by right
understanding, pa\~n\~n{\aa}, and sometimes they are not accompanied by
right understanding. D{\aa}na and s\'ila can be performed also without
right understanding, but for mental development pa\~n\~n{\aa} is
necessary. There are different levels of pa\~n\~n{\aa}. When kusala
citta with generosity arises it can be accompanied by pa\~n\~n{\aa}
which knows that generosity is kusala and which knows that a wholesome
action will produce a pleasant result. That is one level of
pa\~n\~n{\aa}. There is pa\~n\~n{\aa} at the level of intellectual
understanding of the Buddhist teachings, understanding that the
n{\aa}mas and r\'upas which arise because of conditions are impermanent
and anatt{\aa}. Pa\~n\~n{\aa} in samatha, tranquil meditation, is of a
different level again. It is not merely theoretical understanding, it
realizes precisely when the citta is kusala and when it is akusala; it
sees the disadvantage of akusala and the benefit of kusala.
Pa\~n\~n{\aa} in samatha knows the right conditions to develop calm by
means of a meditation subject. Calm accompanies kusala citta and when
its characteristic is known through direct experience it can be
developed. It is developed by concentration on one of the meditation
subjects, but if one just tries to concentrate on one object without
right understanding, calm cannot be developed. There are particular
meditation subjects of samatha, as explained in the Path of Purity,
(Visuddhimagga), which is an Encyclopedia on Buddhism, written by the
commentator Buddhaghosa. Mett{\aa} is among the meditation subjects of
samatha. When calm has been developed there can be the attainment of
jh{\aa}na, absorption. At the moment of jh{\aa}na there are no sense
impressions and one is free from defilements which are bound up with
them. The jh{\aa}nacitta is of a higher plane of consciousness.
However, after the jh{\aa}nacittas have fallen away defilements can
arise again. Through samatha defilements are temporarily subdued but
they are not eradicated. It is extremely difficult to attain jh{\aa}na
and only very few people are able to do it. Those who can attain
jh{\aa}na with mett{\aa} as meditation subject, can with a mind full of
mett{\aa} pervade the whole world and all beings. This is the
{\textasciigrave}{\textasciigrave}extension of mett{\aa} to all
beings{\textquotesingle}{\textquotesingle}, as referred to in this
book. As Khun Sujin explains, also those who do not intend to develop a
high degree of calm can and should develop mett{\aa} in daily life. If
they know precisely when the citta is kusala and when akusala and if
they know the characteristic of mett{\aa}, they can develop it, and at
such moments there is calm with the citta. 

The development of insight or vipassan{\aa} is different from the
development of samatha. Method and aim of these two ways of mental
development are different. The development of insight is the
development of right understanding of ultimate realities, of n{\aa}ma
and r\'upa, in order to eradicate the wrong view of self. Through
insight all defilements and latent tendencies of defilements can
eventually be completely eradicated. In the development of insight one
does not try to concentrate on one object, but through mindfulness or
awareness, sati, right understanding of any reality which appears
through one of the six doors is developed. Right understanding of
realities can be developed at any place and any time, in daily life;
one does not have to go to a quiet place. Sati is a wholesome cetasika
which is non{}-forgetful, aware of the n{\aa}ma or r\'upa which appears
at the present moment. At the very moment of sati direct understanding
of the reality which appears can be developed, so that realities can
eventually be seen as anatt{\aa}. Satipa\`i\`ih{\aa}na, the application
of mindfulness, is another term for the development of right
understanding of n{\aa}ma and r\'upa. In the beginning sati and
pa\~n\~n{\aa} are weak, but each moment they arise they develop, even
when this is not noticeable. They can develop from moment to moment,
from life to life. 

The reader will come across the term sati{}-sampaja\~n\~na, which stands
for sati and pa\~n\~n{\aa}. Sampaja\~n\~n{\aa} is another word for
pa\~n\~n{\aa}. Sati and pa\~n\~n{\aa} are different cetasikas which
each perform their own function, but both of them are needed in order
to develop understanding of the reality appearing at the present
moment. If there is only awareness of realities without any
understanding, the goal, seeing realities as they are, cannot be
reached. Sati{}-sampaja\~n\~na is anatt{\aa}, it cannot be induced.
There can only be sati{}-sampaja\~n\~na when there are the right
conditions. These conditions are: listening to the Dhamma as it is
explained by someone with right understanding, and careful
consideration of the Dhamma. First there has to be theoretical
understanding of n{\aa}ma and r\'upa. One has to know that n{\aa}ma is
the reality which experiences something, and that r\'upa is the reality
which does not know anything. One has to understand that seeing is
n{\aa}ma which experiences what appears through the eyes, visible
object, and that visible object is r\'upa. It is necessary to
understand that seeing is different from thinking of people which can
arise after seeing, to understand that different n{\aa}mas experience
objects through six doors. We may believe that we can touch our body,
but in reality it consists of different elements which appear one at a
time. Through touch hardness or heat can be experienced, not a body.
The body is a concept which is made up by thinking, not an ultimate
reality. When one has understood what ultimate realities are, different
from conventional truth, there can be conditions for the arising of
sati, of mindfulness. Sati is directly aware and attentive to the
n{\aa}ma or r\'upa which presents itself right now. We experience time
and again r\'upas through the body{}-sense, such as hardness or
softness, but there is forgetfulness and ignorance of these realities.
We usually pay attention to a thing or the body which is hard or soft,
to the concept of a
{\textasciigrave}{\textasciigrave}whole{\textquotesingle}{\textquotesingle}.
When sati arises it can be aware of a r\'upa such as hardness or
softness, or of a n{\aa}ma which experiences these r\'upas, and at that
moment these realities can be investigated by pa\~n\~n{\aa}. They can
be known as r\'upa or n{\aa}ma, which arise because of their own
conditions, and which are anatt{\aa}. Pa\~n\~n{\aa} is developed in
different stages of insight. First the difference between n{\aa}ma and
r\'upa has to be realized, otherwise there cannot be the direct
understanding of the arising and falling away of n{\aa}ma and r\'upa,
which is a higher stage of insight. It can be understood in theory that
n{\aa}ma is different from r\'upa, but when they actually present
themselves the difference between their characteristics is not directly
known. We tend to confuse realities such as hearing and sound or seeing
and visible object. Sati can only be aware of one reality at a time and
only if there is awareness over and over again pa\~n\~n{\aa} can
develop so that n{\aa}ma and r\'upa can be seen as they are. 

When pa\~n\~n{\aa} has been fully developed there can be the attainment
of enlightenment: the experience of nibb{\aa}na, the unconditioned
reality. Citta, cetasika and r\'upa are ultimate realities which arise
because of conditions and then fall away. Nibb{\aa}na is not r\'upa, it
is not a place where one can go to; it is n{\aa}ma. Nibb{\aa}na is the
ultimate reality which is an unconditioned n{\aa}ma, it does not arise
and fall away. Nibb{\aa}na is called the end of suffering, dukkha, the
end of the unsatisfactoriness inherent in all conditioned realities
which arise and fall away. There are four stages of enlightenment and
at each stage defilements are subsequently eradicated, until they are
all eradicated at the final stage, the stage of the arahat, the
perfected one. 

The objects of satipa\`i\`ih{\aa}na, of the development of right
understanding, are ultimate realities, n{\aa}ma and r\'upa. Mett{\aa}
is directed towards beings, it has beings or persons as object. Beings
are not ultimate realities, they are conventional truth. However, it is
most beneficial to develop both satipa\`i\`ih{\aa}na and mett{\aa}, as
Khun Sujin explains. When there is awareness of n{\aa}ma and r\'upa
there can still be thinking of beings, thinking is a type of n{\aa}ma
which arises because of conditions. We usually think of people with
akusala cittas, cittas with attachment or aversion. Instead of thinking
with akusala cittas we can learn to think with mett{\aa}{}-citta.
Mett{\aa}{}-citta is a type of n{\aa}ma and if there can be awareness
of it we will see it as a conditioned reality, non{}-self. If we do not
cling to a concept of {\textasciigrave}{\textasciigrave}my
mett{\aa}{\textquotesingle}{\textquotesingle}, mett{\aa} will be purer.
Khun Sujin emphasizes that the development of satipa\`i\`ih{\aa}na
conditions the arising of mett{\aa} more often. When
satipa\`i\`ih{\aa}na is developed defilements such as conceit, avarice
and jealousy, which are impediments to mett{\aa}, will eventually be
eradicated. The understanding that both we ourselves and other people
are only citta, cetasika and r\'upa, will condition more mett{\aa}. If
we understand that our akusala cittas arise because of conditions, we
will also understand that the akusala cittas of someone else are
conditioned. We will be less inclined to judge others and we will have
more understanding of their problems. We can learn to become, as Khun
Sujin says, {\textasciigrave}{\textasciigrave}an understanding
person{\textquotesingle}{\textquotesingle}, someone who sympathizes and
helps others. If they do not respond to our kindness we can still treat
them as friends. True friendship does not depend on the attitude of
someone else, it arises with the mett{\aa}{}-citta. 

Mett{\aa} is one of the
{\textasciigrave}{\textasciigrave}perfections{\textquotesingle}{\textquotesingle},
excellent qualities the Buddha developed during countless previous
lives when he was a Bodhisatta, a being destined for Buddhahood. People
who have confidence in the Buddha{\textquotesingle}s teachings and
develop satipa\`i\`ih{\aa}na can develop the perfections together with
mindfulness of n{\aa}ma and r\'upa. The perfections, and thus also
mett{\aa}, are necessary conditions for the attainment of
enlightenment. The aim of the development of the perfections is the
elimination of defilements. 

Khun Sujin helps people to know their own citta, to know when it is
kusala citta and when akusala citta. When they have right understanding
of their cittas, they will not delude themselves and take for mett{\aa}
what is akusala. Khun Sujin{\textquotesingle}s explanation on mett{\aa}
is essential for the understanding of what mett{\aa} is, and of the way
how it can be developed. Her explanations are very convincing and
direct and can be of great assistance to practise mett{\aa} in daily
life. The many texts she quotes from the Buddhist scriptures can be a
reminder and encouragement to practise mett{\aa} in daily life. 

The quotations in this book are taken from the Tipi\`iaka and from some
of the commentaries, including the Atthas{\aa}lin\'i (Expositor), the
commentary to the first book of the Abhidhamma (Dhammasanga\`ui), and
the Path of Purification (Visuddhimagga). The English translations of
these texts are available at the P{\aa}li Text Society, 73 Lime Walk,
Headington, Oxford OX3 7AD, England. 

I want to retain the P{\aa}li terms in this book , because it is useful
to learn some of them. The English equivalents are often unsatisfactory
since these stem from Western philosophy and therefore give an
association of meaning different from the meaning they have in the
Buddhist teachings. 

I wish to acknowledge my appreciation to the
{\textasciigrave}{\textasciigrave}Dhamma Study and Propagation
Foundation{\textquotesingle}{\textquotesingle} and to the publisher
Alan Weller who made the printing of the translation of this book
possible. 

\clearpage
Chapter 1


\bigskip


\bigskip

Conditions and impediments


\bigskip


\bigskip

Mett{\aa}, loving kindness, can be cultivated when we know its
characteristic. When there is true mett{\aa} other people are
considered as friends: there is a feeling of closeness and sympathy, we
have tender care for them and we want to do everything for their
benefit and happiness. At such moments the citta is gentle, there is no
conceit, m{\aa}na, which is the condition for asserting oneself, for
showing one{\textquotesingle}s own importance and for disparaging
others. 

If there is the earnest wish to develop mett{\aa}, we want to eliminate
akusala dhammas, also those which we usually do not notice. We do not
realize the extent of our conceit, jealousy, stinginess, aversion and
other defilements. When we develop mett{\aa} we will begin to notice
many kinds of defilements, and as mett{\aa} is accumulated more there
will be less opportunity for the arising of unwholesomeness. 

Conceit is a defilement which is an impediment to mett{\aa}. When there
is mett{\aa} we think of the well{}-being of someone else, whereas when
there is conceit we find ourselves important. If we wish to eliminate
conceit and to develop mett{\aa} we must know the characteristic of
conceit. We read in the Atthas{\aa}lin\'i (Expositor, Book II, Part II,
Chapter 2, 372) about conceit:


\bigskip

{\textasciigrave}{\textasciigrave}Conceit{\textquotesingle}{\textquotesingle},
{\textasciigrave}{\textasciigrave}overweening{\textquotesingle}{\textquotesingle}
and
{\textasciigrave}{\textasciigrave}conceitedness{\textquotesingle}{\textquotesingle}
signify mode and state.
{\textasciigrave}{\textasciigrave}Loftiness{\textquotesingle}{\textquotesingle}
is in the sense of rising upwards or of springing over others.
{\textasciigrave}{\textasciigrave}Haughtiness{\textquotesingle}{\textquotesingle},
i.e. in whom conceit arises, him it lifts up, keeps upraised.
{\textasciigrave}{\textasciigrave}Flaunting a
flag{\textquotesingle}{\textquotesingle} is in the sense of swelling
above others.
{\textasciigrave}{\textasciigrave}Assumption{\textquotesingle}{\textquotesingle}
means uplifting; conceit favours the mind all round. Of many flags the
flag which rises above others is called a banner. So conceit arising
repeatedly in the sense of excelling with reference to subsequent
conceits is like a banner. That mind which desires the banner is said
to be desirous of the banner (i.e., self advertisement). Such a state
is desire for self{}-advertisement. And that is of the citta, not of a
real self; hence {\textasciigrave}{\textasciigrave}desire of the citta
for self{}-advertisement{\textquotesingle}{\textquotesingle}. Indeed,
the citta associated with conceit wants a banner, and its state is
reckoned as banner{}-conceit. 


\bigskip

When we learn about the characteristic of conceit we can see the
difference between the moment of akusala citta and of mett{\aa}.
Akusala citta does not have the characteristic of gentleness and
tenderness, at such a moment there is no feeling of closeness and
friendship for others. If we want to develop mett{\aa} there must be
{\textasciigrave}{\textasciigrave}sati{}-sampaja\~n\~na{\textquotesingle}{\textquotesingle},
mindfulness and understanding, in order to know when there is kusala
citta and when there is akusala citta. At the moment of conceit there
cannot be mett{\aa}. 

Jealousy is another defilement which is an impediment to mett{\aa}. When
we are jealous of someone we certainly do not treat him as a friend. If
we really want to develop mett{\aa} in our daily life, we should be
aware of its characteristic of sympathy and tenderness and we should
realize that mett{\aa} cannot go together with jealousy. 

The Atthas{\aa}lin\'i (Book II, Part II, Chapter 2, 373) states about
envy:


\bigskip

In the exposition of envy, {\textasciigrave}{\textasciigrave}envy at the
gains, honour, reverence, affection, salutation, worship accruing to
others{\textquotesingle}{\textquotesingle} is that envy which has the
characteristic of not enduring, or of grumbling at the prosperity of
others, saying concerning others{\textquotesingle} gains, etc. ,
{\textasciigrave}{\textasciigrave}What is the use to these people of
all this?{\textquotesingle}{\textquotesingle}


\bigskip

The person who has attained the first stage of enlightenment, the
sot{\aa}panna, has completely eradicated jealousy because he sees the
characteristics of realities as they are: mental phenomena (n{\aa}ma
dhammas) and physical phenomena (r\'upa dhammas), arising because of
their appropriate conditions. He realizes that there
isn{\textquotesingle}t anybody who can create gains for himself, or who
can cause others to honour him, to greet him or to pay respect to him.
In fact, obtaining gains and receiving honour and respect from others
depends on conditions. Therefore, there should not be jealousy. When
there is jealousy there is no mett{\aa}. All dhammas, realities, are
anatt{\aa} (non{}-self), kusala dhammas as well as akusala dhammas;
they arise because of their appropriate conditions. So long as one is
not yet an
{\textasciigrave}{\textasciigrave}ariyan{\textquotesingle}{\textquotesingle},
a person who has attained enlightenment, there are conditions for
jealousy. One is not only jealous of those who are not
one{\textquotesingle}s relatives or friends but even of those who are
near and dear to oneself. 

Stinginess is another defilement which is an impediment to mett{\aa}.
The Atthas{\aa}lin\'i (in the same section, 373) states that there is
stinginess as to five things:


\bigskip

dwelling (the place where one stays)

family (for a monk this can be the family of servitors to a monastery or
relatives)

gain (for a monk: the acquirement of the four requisites)

beauty and praise (one does not want others to be praised because of
beauty or merits)

dhamma (one does not want to share knowledge of dhamma)


\bigskip

We read further on (375, 376):


\bigskip

{\textasciigrave}{\textasciigrave}Stinginess{\textquotesingle}{\textquotesingle}
is the expression of meanness.
{\textasciigrave}{\textasciigrave}Avariciousness{\textquotesingle}{\textquotesingle}
is the act or mode of being mean. The citta which is mean is the state
of one endowed with stinginess. {\textasciigrave}{\textasciigrave}Let
it be for me only and not for
another!{\textquotesingle}{\textquotesingle}{}-{}-thus wishing not to
diffuse all one{\textquotesingle}s own acquisitions... The state of
such a person is
{\textasciigrave}{\textasciigrave}avarice{\textquotesingle}{\textquotesingle},
a synonym for soft meanness. An ignoble person is churlish. His state
is
{\textasciigrave}{\textasciigrave}ignobleness{\textquotesingle}{\textquotesingle},
a name for hard stinginess. Verily, a person endowed with it hinders
another from giving to others. And this also has been said (Kindred
Sayings, I, 120):

Malicious, miserly, ignoble, wrong...

Such men hinder the feeding of the poor...


\bigskip

A
{\textasciigrave}{\textasciigrave}niggardly{\textquotesingle}{\textquotesingle}
person seeing mendicants causes his mind to shrink as by sourness. His
state is
{\textasciigrave}{\textasciigrave}niggardliness{\textquotesingle}{\textquotesingle}.
Another
way:{}-{}-{\textasciigrave}{\textasciigrave}niggardliness{\textquotesingle}{\textquotesingle}
is a
{\textasciigrave}{\textasciigrave}spoon{}-feeding{\textquotesingle}{\textquotesingle}.
For when the pot is full to the brim, one takes food from it by a spoon
with the edge bent on all sides; it is not possible to get a spoonful;
so is the citta of a mean person bent in. When it is bent in, the body
also is bent in, recedes, is not diffused{}-{}-thus stinginess is said
to be niggardliness. 

{\textasciigrave}{\textasciigrave}Lack of generosity of
citta{\textquotesingle}{\textquotesingle} is the state of a mind which
is shut and gripped, so that it is not stretched out in the mode of
making gifts, etc. in doing service to others. But because the mean
person wishes not to give to others what belongs to himself, and only
wishes to receive what belongs to others, therefore this meanness
should be understood to have the characteristic of hiding or seizing
one{\textquotesingle}s own property, occurring thus:
{\textasciigrave}{\textasciigrave}May it be for me and not for
another!{\textquotesingle}{\textquotesingle}


\bigskip

The commentator investigates here the citta of the ordinary person who
has not yet eradicated avarice. Only the ariyan has eradicated avarice
completely. When aversion, conceit, jealousy or stinginess arise there
is no mett{\aa} with the citta. If we want to develop mett{\aa} we
should acquire a refined knowledge of our different cittas. The
characteristics of the cittas which think of particular persons should
be investigated. Mett{\aa} should not be restricted to a particular
group of people. We should continue to develop mett{\aa} evermore.
There can never be enough mett{\aa}. 

The Buddha showed in many different suttas the benefit of the
development of mett{\aa}. We read in the Kindred Sayings (I,
Sag{\aa}th{\aa} vagga, Chapter X, The Yakkhas, {\S}4, Ma\`uibhadda):


\bigskip

The Exalted One was once staying among the Magadhese, at the
Ma\`uim{\aa}la temple, in the haunt of the yakkha Ma\`uibhadda. Then
that yakkha drew near to the Exalted One, and before him uttered the
verse: 

To one of mind alert luck ever comes;

He prospers with increasing happiness

For him tomorrow is a better day. 

And wholly from all hate is he released. 


\bigskip

The Buddha said:


\bigskip

...For him whose mind ever by night and day

In harmlessness, in kindness takes delight, 

Bearing his share in love for all that lives, 

In him no hate is found toward anyone. 


\bigskip

Thus we see the great benefit of the development of mett{\aa}. 

Mett{\aa} can be developed as a subject of tranquil meditation, samatha.
If there is right understanding of the development of calm with this
subject, a high degree of calm, even absorption, jh{\aa}na, can be
attained. The cittas which attain absorption, jh{\aa}nacittas, are of a
higher plane of citta. At the moments of jh{\aa}nacitta there are no
sense impressions and one is temporarily free from defilements.
However, after the jh{\aa}nacittas have fallen away, defilements arise
again. The development of tranquillity with mett{\aa} as meditation
subject will not lead to the eradication of anger, dosa. Only the
development of satipa\`i\`ih{\aa}na, right understanding of realities,
leads to the eradication of defilements. Defilements are eradicated
subsequently at four stages of enlightenment. Only at the fourth stage,
the stage of the arahat, all defilements are eradicated. At the third
stage, the stage of the
{\textasciigrave}{\textasciigrave}non{}-returner{\textquotesingle}{\textquotesingle},
an{\aa}g{\aa}m\'i, anger or aversion is eradicated. The
an{\aa}g{\aa}m\'i has no more anger and is full of mett{\aa}. 

The development of right understanding of realities,
satipa\`i\`ih{\aa}na, can be the condition for more mett{\aa}.
Pa\~n\~n{\aa}, right understanding, knows that what one takes for
beings, people or self are only mental phenomena, n{\aa}ma dhammas, and
physical phenomena, r\'upa dhammas. We use conventional terms and names
for the different beings and things which appear, but in reality there
are only n{\aa}mas and r\'upas which arise because of conditions and
then fall away. Each citta which falls away is succeeded by the next
one, and also r\'upas which fall away are replaced so long as there are
conditions for them to be produced. 

Someone said that while he is not engaged in any activity he finds that
he is distracted, that he has akusala cittas. He wishes, in order to
have kusala cittas, to recite stanzas about mett{\aa} for a long time.
If one develops satipa\`i\`ih{\aa}na however, one should remember that
even feeling distracted or dull can be object of awareness. In such
circumstances sati can be aware immediately of the characteristic which
appears and then there are kusala cittas. It is not easy to know the
characteristic of the reality which appears; pa\~n\~n{\aa} should
really be developed so that there can be precise knowledge of the
different characteristics of n{\aa}ma and r\'upa. There must be
awareness of the characteristic of the reality which experiences,
n{\aa}ma dhamma, and of the characteristic of the reality which does
not know anything, r\'upa dhamma. The difference between the
characteristics of n{\aa}ma and r\'upa should be clearly distinguished.
When there is awareness of the realities which appear one at a time
through the doorways of the senses and the mind, through the six doors,
their characteristics must be carefully considered and investigated. In
that way n{\aa}ma and r\'upa can be understood as they are: as
non{}-self. 

The person who believes that he should just recite texts about mett{\aa}
may not be sure whether there are at such moments kusala cittas or
akusala cittas. He may not know that awareness, sati, is necessary for
the development of pa\~n\~n{\aa}, understanding, which clearly knows
the reality appearing at the present moment. Perhaps he may not even
know to which purpose he recites texts. If we really want to cultivate
mett{\aa} we should see the disadvantage of all kinds of akusala, such
as aversion, conceit, jealousy and stinginess. 

For the development of mett{\aa} it is necessary to have a refined,
detailed knowledge of one{\textquotesingle}s different cittas. They
must be known as they really are. Kusala citta and akusala citta have
different characteristics. Even if there is kusala of a slight degree,
that moment is completely different from the moments of attachment. If
sati and pa\~n\~n{\aa} do not arise one cannot know when there is lobha
and when there is mett{\aa}. If one does not know their different
characteristics one may unknowingly develop akusala instead of
mett{\aa} since one takes for kusala what is in fact akusala. Therefore
a precise knowledge of the different characteristics of lobha and
mett{\aa} is necessary. The Atthas{\aa}lin\'i (Book II, Part II,
Chapter 2) explains about the many aspects of lobha mentioned in the
Dhammasanga\`ui. We read about
{\textasciigrave}{\textasciigrave}delight{\textquotesingle}{\textquotesingle}:


\bigskip

{\textasciigrave}{\textasciigrave}Delight{\textquotesingle}{\textquotesingle}
refers to this that by greed beings in any existence feel delight, or
greed itself is delighting in. In
{\textasciigrave}{\textasciigrave}passionate
delight{\textquotesingle}{\textquotesingle} we get the first term
combined with delight. Craving once arisen to an object is
{\textasciigrave}{\textasciigrave}delight{\textquotesingle}{\textquotesingle};
arisen repeatedly, it is {\textasciigrave}{\textasciigrave}passionate
delight{\textquotesingle}{\textquotesingle}...


\bigskip

This is daily life which should really be investigated. When mett{\aa}
does not arise citta is infatuated by objects, it delights in objects
all the time. If there is no awareness we do not know when there is
lobha. The clinging to the different objects which are experienced will
condition our behaviour, our actions through body and speech, and then
we can find out that there is no mett{\aa}. When we have learnt through
our own experience the characteristic of lobha and of mett{\aa} when
they arise, we can compare them and clearly know their difference. 

We should not only try to develop mett{\aa} when anger arises, but also
when there is attachment. We should consider with what kind of citta we
think of our friends, our circle of relatives, those who are near and
dear to us. We should find out whether there are at such moments cittas
with mett{\aa} or cittas with lobha, and we should learn by our own
experience the difference between these moments. If we earnestly wish
to develop mett{\aa} we should not waste any opportunity to learn about
the characteristics of our different cittas so that there are
conditions for the development of mett{\aa}. It is useless to think
that we should develop mett{\aa} only when we become angry. 

I will now go into some questions with regard to the development of
mett{\aa}. 

Question: the characteristic of lobha is love and attachment. If one
says that attachment to relatives and friends is lobha and that it is
therefore wrong to be attached to them I think that this does not agree
with our ordinary, daily life in the world. 

Khun Sujin: If one wants to develop mett{\aa} there must be a precise
knowledge of one{\textquotesingle}s different cittas. If people only
recite texts about mett{\aa} it is not sufficient; the characteristic
of mett{\aa} should be known precisely. When there is mett{\aa} there
is no anger. However, when we love someone and we are attached to that
person there is lobha, not mett{\aa}, and lobha can condition anger. We
should consider which reality is better, mett{\aa} or selfish love,
which is actually lobha. When we are in the company of family or
friends, there can be mett{\aa} and then we can come to know its
characteristic. When there is mett{\aa} we wish other
people{\textquotesingle}s benefit, there is no clinging, no selfish
love. True mett{\aa} towards someone else cannot condition dislike of
that person. Thus, when we have mett{\aa} instead of lobha others will
benefit from this too. Both the person who has mett{\aa} and the person
who is the object of mett{\aa} will benefit. If there is only lobha in
our daily life there are many conditions for dislike and unpleasant
feeling. However, to the extent mett{\aa} develops there will be less
opportunities for the arising of dosa. We will become more considerate
and think more often of the benefit of others. 

Question: You said that sati and pa\~n\~n{\aa} (sati{}-sampaja\~n\~na)
are necessary for the development of mett{\aa} and that one therefore
should know the characteristics of sati and pa\~n\~n{\aa}. If one does
not know them mett{\aa} cannot be developed, is that right?

Khun Sujin: There are two kinds of mental development, samatha, tranquil
meditation, and vipassan{\aa}, the development of insight or right
understanding of realities. For both kinds of mental development
sati{}-sampaja\~n\~na is necessary. However, pa\~n\~n{\aa} in samatha
is different from pa\~n\~n{\aa} in vipassan{\aa}. Pa\~n\~n{\aa} in
samatha knows the way to develop tranquillity, the temporary freedom
from defilements. Pa\~n\~n{\aa} in the development of vipassan{\aa}
knows the characteristics of mental phenomena and physical phenomena,
of the realities which appear one at a time through the six doors. 

Question: Sometimes mett{\aa} can arise when one is concerned about
other people who are in trouble. At such a moment there is sati but
there may not be pa\~n\~n{\aa} which knows the characteristic of sati.
Is there true mett{\aa} at such a moment? 

Khun Sujin: When mett{\aa} arises the citta is kusala and it is
accompanied by sati which is a wholesome reality (sobhana dhamma). One
may not have sati{}-sampaja\~n\~na so that a higher degree of calm can
be developed, but when there is mett{\aa} it has to be accompanied by
sati, because of conditions. Sati which is non{}-forgetful of kusala
accompanies each kusala citta. Because of accumulations of kusala there
can be conditions for different kinds of kusala, for d{\aa}na,
generosity, for s\'ila, abstention from unwholesome deeds or for
mett{\aa}. Those types of kusala are accompanied by sati but not
necessarily by pa\~n\~n{\aa}. However, if one wants to develop
mett{\aa} as subject of calm and attain to higher degrees of calm,
sati{}-sampaja\~n\~na is necessary. Through sati{}-sampaja\~n\~na the
difference between the characteristics of mett{\aa} and lobha can be
known precisely. 

Question: I will speak about events in my daily life. Sometimes when I
drive the car I recite: {\textasciigrave}{\textasciigrave}May all
beings be happy, may they not suffer any harm or
misfortune.{\textquotesingle}{\textquotesingle} When I happened to be
in a complicated traffic situation, however, I could at first not be
considerate to others. Later on I realized that I did not behave in
accordance with the texts about mett{\aa} I had recited. I started to
consider more those texts and I learnt to apply mett{\aa} in the
traffic situation. Thus this is the effect of thinking and considering.


Khun Sujin: When you are in a complicated traffic situation do you think
of the words, {\textasciigrave}{\textasciigrave}May all beings be
happy{\textquotesingle}{\textquotesingle}?

Question: No, I do not think of these words at such moments. 

Khun Sujin: The development of mett{\aa} is not a matter of thinking of
words, but one should know the reality of mett{\aa}{}-citta. Such a
moment is different from the moments of annoyance, anger or vengeance. 

Question: If I had not recited texts about mett{\aa} I would not be
considerate in the traffic situation, I would only think of myself. 

Khun Sujin: You should have a detailed knowledge of realities, you
should find out whether there is at the moment you recite true
mett{\aa} or just thinking of words. There is true mett{\aa} at the
moment you are considerate towards others, not when you just recite
words. 

Question: The reciting does have an effect. If I had not recited I would
not have asked myself whether I really wanted other beings to be happy.
The fact that I asked myself this was the effect of my recitation. 

Khun Sujin: When you asked yourself this you realized already that
mett{\aa} is not just reciting words but that it should be practised. 

Question: Yes, that is true. When I practised mett{\aa} in the situation
I did not recite. 

Khun Sujin: Some people only think of reciting texts about mett{\aa},
but after they have finished reciting they become angry when something
unpleasant occurs. One may recite words about mett{\aa}, but mett{\aa}
may not arise when there are beings or people present. One may recite
for a long time, but when something unpleasant happens, where is
mett{\aa}? How much longer should one then recite so that mett{\aa} can
arise?

Another questioner: If one thinks that one must recite in order to
develop mett{\aa} there will not be any result, because one has wrong
understanding about the development of mett{\aa}. Its development will
only be successful if one practises mett{\aa} in the situation of
one{\textquotesingle}s daily life. Since a year or two I have the
feeling that I have more mett{\aa} than before, and that is only due to
Khun Sujin{\textquotesingle}s lectures about Dhamma I listened to. I
always think now of doing things for the benefit and happiness of
others, no matter whether it is a small matter or something more
important. I feel that when sati arises the citta is gentle. When we
abstain from killing mosquitoes or help other beings who are in trouble
there is mett{\aa}. It happened that at first I did not want to make an
effort to help other beings, but later on I could do it, because I
considered their benefit and happiness. Sometimes people sell things I
do not want to buy, but I still buy them because mett{\aa} arises. I do
not buy them because I wish to have them or I need them. I think of
Khun Sujin{\textquotesingle}s words,
{\textasciigrave}{\textasciigrave}It does not matter whether we do a
lot or just a little for someone else, but we can consider his benefit
and happiness.{\textquotesingle}{\textquotesingle} Whenever I think of
these words kusala citta with mett{\aa} can arise. 

Khun Sujin: Anumodhan{\aa}. This is the practice of the Dhamma, it
really is the development of mett{\aa}. The P{\aa}li term for
development is
{\textasciigrave}{\textasciigrave}bh{\aa}van{\aa}{\textquotesingle}{\textquotesingle}
and this literally means: to make become more, to cause to arise often,
time and again. Development is not reciting texts with the expectation
that as a result a high degree of calm, even absorption,
jh{\aa}nacitta, will arise. There should be mett{\aa} in our daily
life. We may, when we are alone, recite texts about mett{\aa} many
times, but when we are in the situation of our daily life mett{\aa} may
not arise. The real development of mett{\aa} is done through the
practice, through our behaviour in the different circumstances of daily
life, when we are in the company of other people. 

Question: I still think that the reciting of texts on mett{\aa} may be
beneficial. Reciting is not easy. I may think of people I do not like,
such as Mr. X. who had done me wrong in the past, but now, while I
develop mett{\aa}, I think, {\textasciigrave}{\textasciigrave}May Mr.
X. be happy, may he not suffer any
misfortune{\textquotesingle}{\textquotesingle}. When I recite texts, I
do not have to spend any money or make an effort to help someone. I am
not ready yet to do these things. 

Khun Sujin: The reason is that you did not develop mett{\aa} gradually,
in daily life. Today you do not see Mr. X., but you see other people.
Can you find out whether there is mett{\aa} now, while you see other
people? When one really develops mett{\aa} one must know that when
there is mett{\aa} the citta is free from all that is unwholesome. At
such a moment there is no conceit, no idea of making oneself important.
Even when we look at other people or think of them, we can do so
without looking down on them, without conceit. Mett{\aa} can be
expressed through the body, even in our gestures, and in our way of
speech. No matter with whom we are, sati{}-sampaja\~n\~na can arise and
we can find out whether the citta at a particular moment is accompanied
by mett{\aa} or not. We can develop mett{\aa} all the time and we
should not select the persons towards whom we will have mett{\aa}, such
as Mr. X. 

Question: I will start to develop mett{\aa} all the time. When I see
other people I will think, {\textasciigrave}{\textasciigrave}May all
people be happy, may they not suffer
misfortune{\textquotesingle}{\textquotesingle}. 

Khun Sujin: Why do you think of all people?

Question: When I look at people I see them as a group. 

Khun Sujin: At this moment you know in theory that there are only
n{\aa}ma and r\'upa, no beings, people or self. However, you do not
know the characteristics of n{\aa}ma and r\'upa. There is no
sati{}-sampaja\~n\~na which considers each kind of reality which
appears. When the characteristics of n{\aa}mas and r\'upas are clearly
known, as they appear one at a time, mett{\aa} can be developed more.
Thus, there must be sati{}-sampaja\~n\~na which knows the
characteristic of the citta when there is mett{\aa} for such or such
person. Otherwise we could not know whether there is only reciting and
thinking of texts about mett{\aa}, or sincere mett{\aa} for each person
we meet. 

Question: When I recite texts on mett{\aa} there is sometimes no
pa\~n\~n{\aa}, but there is sati. I wish to extend mett{\aa} to all
beings. 

Khun Sujin: We should know the meaning of
{\textasciigrave}{\textasciigrave}developing
mett{\aa}{\textquotesingle}{\textquotesingle} and of
{\textasciigrave}{\textasciigrave}extending mett{\aa} to all
beings.{\textquotesingle}{\textquotesingle} If one has not really
developed mett{\aa} the citta does not wish happiness for anybody one
meets. One does not yet have a feeling of friendship for all people,
and thus one is not able to extend mett{\aa} to all beings. One can
begin to develop mett{\aa} for other people through body, speech and
thoughts, and thus it can gradually increase. When we think of someone
else, whoever he may be, or whenever we meet someone else, there can be
sincere mett{\aa} through body, speech and mind. By the recitation of
texts on mett{\aa} there will not be any change in the expression of
our face or in our speech; mett{\aa} will not develop through the
recitation of texts. When we meet someone we can consider the citta at
that moment, we should know whether we look down on him, even though we
do not show this outwardly, but it is just in our mind. Does it happen
that we dislike someone{\textquotesingle}s appearance, behaviour or
speech? Do we really consider that person as a friend while we speak to
him, do we sincerely seek what is beneficial for him and do we want to
help him? There is no rule that one should recite particular texts
about mett{\aa}. If we want to develop mett{\aa} we do not have to
follow any rule about recitation of texts. We can think of others with
kusala citta which is accompanied by mett{\aa}: we can think of doing
things for his wellbeing and happiness, of protecting him from
misfortune and trouble. When one recites one has to think of words, one
has to think whether one should say first
{\textasciigrave}{\textasciigrave}may all beings be
happy{\textquotesingle}{\textquotesingle}, or whether one should say
first {\textasciigrave}{\textasciigrave}may all beings be free from
suffering{\textquotesingle}{\textquotesingle}. The reality of mett{\aa}
is not the recitation of texts. Mett{\aa} arises when we give help to
someone else through actions or through speech, depending on the
situation at that moment. 


\bigskip

\clearpage
Chapter 2


\bigskip


\bigskip

Overcoming anger


\bigskip


\bigskip

If we truly know the characteristic of mett{\aa} we can develop it.
However, we should not think that we can already extend mett{\aa} to
all beings so that it is boundless. In fact, only people who have
developed samatha with mett{\aa} as meditation subject and have
attained the first stage of jh{\aa}na, are able to extend mett{\aa} to
all beings. 

Question: The commentator states that one should recite particular texts
about mett{\aa}. 

Khun Sujin: Does mett{\aa}{}-citta arise according to a particular rule?

Question: No, that is not so. 

Khun Sujin: One should know the characteristic of mett{\aa} as it is and
then one can develop it more and more. However, as I explained, one
should not try to extend mett{\aa} to all beings straightaway in order
to develop it more. 

Question: There are forty meditation subjects of samatha and it depends
on one{\textquotesingle}s inclination which subject one will develop.
Generally one has to recite texts in order to develop meditation
subjects, such as the {\textasciigrave}{\textasciigrave}earth
kasina{\textquotesingle}{\textquotesingle}. 

Khun Sujin: We should investigate the Tipi\`iaka in order to find out
whether it is said that we should recite texts. We read in the Kindred
Sayings (I, Sag{\aa}th{\aa}{}-vagga, Chapter VII, The Brahmin Suttas,
1, Arahats, {\S}1, The Dhana\~nj{\aa}ni brahminee) :


\bigskip

Thus have I heard:{}-{}-The Exalted One was once staying near
R{\aa}jagaha, in the Bamboo Grove, at the Squirrels{\textquotesingle}
Feeding ground. 

Now at that time a Dhana\~nj{\aa}ni brahminee, the wife of a certain
brahmin of the Bh{\aa}radv{\aa}ja family, was a fervent believer in the
Buddha, the Dhamma and the Sangha. And she, while serving the
Bh{\aa}radv{\aa}ja with his dinner, came before him and uttered three
times the following praise:

{\textasciigrave}{\textasciigrave}Glory to that Exalted One, Arahat,
Buddha supreme!

Glory to the Dhamma!

Glory to the Sangha!{\textquotesingle}{\textquotesingle}

And when she had said so the Bh{\aa}radv{\aa}ja brahmin exclaimed:
{\textasciigrave}{\textasciigrave}There now! At any and every
opportunity must the wretch be speaking the praises of that shaveling
friar! Now, wretch, will I give that teacher of yours a piece of my
mind!{\textquotesingle}{\textquotesingle}

{\textasciigrave}{\textasciigrave}O brahmin, I know of no one throughout
the world of gods, M{\aa}ras or Brahm{\aa}s, recluses or brahmins, no
one human or divine, who could admonish that Exalted One, Arahat,
Buddha Supreme. Nevertheless, go, brahmin, and then you will know.
{\textquotesingle}{\textquotesingle}

Then the Bh{\aa}radv{\aa}ja, vexed and displeased, went to find the
Exalted One; and coming into his presence, exchanged with him greetings
and compliments, friendly and courteous, and sat down at one side. So
seated, he addressed the Exalted One in a verse:{}-{}-


\bigskip

What must we slay if we would live happily?

What must we slay if we would weep no more?

What is it above all other things of which

The slaying you would approve, Gotama?


\bigskip

The Buddha said:


\bigskip

Wrath must you slay, if you would live happily, 

Wrath must you slay, if you would weep no more. 

Of anger, brahmin, with its poisoned root

And fevered tip, murderously sweet, 

That is the slaying by the ariyans praised;

That must you slay in truth, to weep no more. 


\bigskip

When the Exalted One had thus spoken, the Bh{\aa}radv{\aa}ja brahmin
said to him: {\textasciigrave}{\textasciigrave}Most excellent, lord,
most excellent{\textquotesingle}{\textquotesingle}...

We then read that Bh{\aa}radv{\aa}ja brahmin left the world under the
Exalted One and was ordained. Not long after his ordination he attained
arahatship. 

Question: The Buddha spoke more in general about slaying anger, but he
did not explain the way how to slay anger. 

Khun Sujin: The Buddha taught the Dhamma in many different ways and in
all details so that people could see the disadvantage of akusala and
the benefit of kusala. He taught the development of pa\~n\~n{\aa} which
can slay anger completely. 

Question: Anger can be slain. Through the development of vipassan{\aa}
anger can be slain and through the development of samatha it can be
suppressed. The development of samatha and the development of
vipassan{\aa} are different, they have different aims. I have read in
the {\textasciigrave}{\textasciigrave}Book of
Analysis{\textquotesingle}{\textquotesingle}, in the chapter on
Jh{\aa}na (Chapter XII), that if someone wants to purify the mind of
the hindrances he must sit and he must walk up and down. He must do
this in order to have right effort which is necessary for the
suppressing of the hindrances. Someone who develops vipassan{\aa},
however, does not have to sit or walk up and down in order to have
right effort. Whenever an object appears right understanding can know
its characteristic, and then there is already right effort, which is
energy for the development of understanding. Thus the development of
samatha and the development of vipassan{\aa} are different. The person
who develops samatha has to follow particular rules. 

Khun Sujin: Where does he begin and how does he develop it?

Question: He starts with reciting words. 

Khun Sujin: He should start with right understanding of the
characteristic of the meditation subject of samatha. This subject must
condition the citta to be calm, to be free from akusala. Sati
sampaja\~n\~na is needed to develop calm in the right way with the
meditation subject. 

Question: The person who develops samatha in order to attain jh{\aa}na
must concentrate on the meditation subject so that calm and
concentration can increase. 

Khun Sujin: That is too far{}-fetched, it is not related to the reality
which can be experienced now, by the person who is only a beginner. Can
you notice the characteristic of aversion in your daily life? The
brahmin Bh{\aa}radv{\aa}ja asked the Buddha,
{\textquotesingle}{\textquotesingle}What must we slay if we would live
happily?{\textquotesingle}{\textquotesingle} The Buddha answered,
{\textasciigrave}{\textasciigrave}Wrath must you slay if you would live
happily, wrath must you slay if you would weep no
more{\textquotesingle}{\textquotesingle}. When people are in daily life
busy with their work, are there no problems and unpleasant experiences
in connection with their work, with the people they meet in their work
or with their colleagues? During our work we are together with other
people and then there can be the arising of like and dislike, we may be
distressed, annoyed, displeased or sad. Whenever you feel displeasure
there is dosa, and this has many shades and degrees. We must slay dosa
when it arises in the situation of our daily life, not at some other
time. When we can subdue dosa in daily life there is a degree of calm
or samatha. When we see the disadvantages of dosa we know that there
should be mett{\aa} instead of akusala. Mett{\aa} can arise at that
moment if we develop it right away and do not delay its development
until later on. Thus, when there are difficult situations or when
problems arise in our work, contrary to our expectations, when there
are events which cause discomfort or even distress, and we can then
slay dosa, there will be happiness instead of sorrow. 

Question: Nobody likes aversion. 

Khun Sujin: It is in daily life that dosa should be overcome. It can be
subdued by developing mett{\aa} as a meditation subject of calm, or by
the development of satipa\`i\`ih{\aa}na. Sati of satipa\`i\`ih{\aa}na
is mindful of the characteristics of realities which are appearing and
thus pa\~n\~n{\aa} can be developed stage by stage, until it is so keen
that the third stage of enlightenment, the stage of the
an{\aa}g{\aa}m\'i (non{}-returner) can be reached and then dosa is
really eradicated. 

When Bh{\aa}radv{\aa}ja had become a monk under the Buddha, his younger
brothers heard that he had gained confidence in the Buddha and had
become a monk. They became angry because of this and they gave
expression to their anger in their behaviour and speech. We read in the
following sutta in the Kindred Sayings (I, Chapter VII, the Brahmins,
1, Arahats, {\S}2, Reviling):


\bigskip

The Exalted One was once staying near R{\aa}jagaha, in the Bamboo Grove,
near the Squirrels{\textquotesingle} Feeding{}-ground. 

Now
{\textasciigrave}{\textasciigrave}Reviler{\textquotesingle}{\textquotesingle}
of the Bh{\aa}radv{\aa}ja brahmins heard that the Bh{\aa}radv{\aa}ja
had left the world to enter the Sangha of Gotama the Recluse. Vexed and
displeased, he sought the presence of the Exalted One, and there
reviled and abused the Exalted One in rude and harsh speeches. 

When he had thus spoken, the Exalted One said:
{\textasciigrave}{\textasciigrave}As to this, what do you think,
brahmin? Do you receive visits from friends and colleagues, from
relatives, by blood or marriage, from other
guests?{\textquotesingle}{\textquotesingle}

{\textasciigrave}{\textasciigrave}Yes, Master Gotama, sometimes I
do.{\textquotesingle}{\textquotesingle}

{\textasciigrave}{\textasciigrave}As to that, what do you think,
brahmin? Do you prepare for them food both dry and juicy, and an
opportunity for rest?{\textquotesingle}{\textquotesingle}

{\textasciigrave}{\textasciigrave}Yes, Master Gotama, sometimes I do.
{\textquotesingle}{\textquotesingle}

{\textasciigrave}{\textasciigrave}But if they do not accept your
hospitality, brahmin, whose do those things
become?{\textquotesingle}{\textquotesingle}

{\textasciigrave}{\textasciigrave}If they do not accept those things,
Master Gotama, they are for us.{\textquotesingle}{\textquotesingle} 

Even so here, brahmin. That wherewith you revile us who do not revile,
wherewith you scold us who do not scold, wherewith you abuse us who do
not abuse, but that we do not accept from you. It is only for you,
brahmin, it is only for you! He, brahmin, who reviles again at his
reviler, who scolds back, who abuses in return him who has abused,
this, brahmin, is as if you and your visitors dined together and made
good. We neither dine together with you nor make good. It is for you
only, brahmin, it is only for you!{\textquotesingle}{\textquotesingle}

{\textasciigrave}{\textasciigrave}The king and his court believe that
Gotama the recluse is an arahat. And yet Master Gotama can indulge in
wrath!{\textquotesingle}{\textquotesingle}


\bigskip

The Exalted One said:


\bigskip

From where should wrath arise for him who, void of wrath, 

Holds on the even tenor of his way,

Self{}-tamed, serene, by highest insight free?

Worse of the two is he who, when reviled, 

Reviles again. Who does not, when reviled, 

Revile again, a two{}-fold victory wins. 

Both of the other and himself he seeks

The good; for he the other{\textquotesingle}s angry mood

Understands and has sati and calm. 

He who of both is a physician, since

Himself he heals and the other too, 

Those who do not know Dhamma think him a fool


\bigskip

When he had so said, Reviler of the Bh{\aa}radv{\aa}jas spoke thus:
{\textasciigrave}{\textasciigrave}Most excellent, Master
Gotama...{\textquotesingle}{\textquotesingle}


\bigskip

We then read that he was ordained and not long after this became an
arahat. 

If we are in a similar situation, thus, when we are reviled, can
mett{\aa} arise? Or must we, when someone else is angry, treat him
likewise? Can we change our mood and forgive him instead of being angry
in return? When there is anger, no matter whose anger it is, there is
no calm, there is the wish to cause injury, to do harm. When we see the
anger of someone else, his mood of wanting to do harm, and we
understand the disadvantage of it, do we want to treat him likewise?
When we see the disadvantage of dosa, there are conditions for the
arising of mett{\aa}. We should develop mett{\aa} so that we are able
to forgive someone else, even if he does wrong to us through body or
speech. 

We read in the following sutta in the Kindred Sayings (I, Chapter VII,
Brahmin Suttas, 1, Arahats, {\S}3, Asurinda):


\bigskip

Again, while the Exalted One was at Bamboo Grove, an Asurinda
Bh{\aa}radv{\aa}ja brahmin also heard that the Bh{\aa}radv{\aa}ja had
entered the Sangha, and he, vexed and displeased, also went and reviled
and abused the Exalted One with rude and harsh words. 

When the he had thus spoken, the Exalted One remained silent. 

Then said the Asurinda: {\textasciigrave}{\textasciigrave}You are
conquered, recluse, you are
conquered!{\textquotesingle}{\textquotesingle}

The Buddha said:


\bigskip

The fool does deem the victory his

In that he plays the bully with rude speech.

To him who knows what forbearance is,

This in itself makes him conqueror

Worse of the two is he who when reviled

Reviles again, repays in kind. 


\bigskip

We then read that also Asurinda became a monk and attained arahatship. 

The Buddha did not in any way retort angry words. We who still have
defilements may also keep silent when we are reviled, but with what
kind of cittas do we keep silent? We should consider our cittas at such
moments. There are different types of cittas for the Buddha when he
keeps silent, and also for the arahat, the perfected one who has
attained the fourth stage of enlightenment), for the an{\aa}g{\aa}m\'i
(who has attained the third stage), for the sakad{\aa}g{\aa}m\'i (who
has attained the second stage), for the sot{\aa}panna (who has attained
the first stage) and for the ordinary person; in each case there are
different types of cittas at such moments. It all depends on the degree
of wisdom. When someone has not yet eradicated dosa, he may keep silent
and not show anger outwardly, through gestures or speech, but can we
know what types of cittas he has? When satipa\`i\`ih{\aa}na does not
arise we do not know whether we have at a particular moment kusala
citta or akusala citta, we do not know whether we have true mett{\aa}.
When a person who still has defilements notices that someone else keeps
silent, he interprets this in accordance with his own accumulations.
However, the reason of someone else{\textquotesingle}s silence may be
different from what he thinks. 

When we carefully consider the meaning of the sutta which was just
quoted, we will see its benefit. But this also depends on the extent we
practise in accordance with the Dhamma. When we speak coarse words, are
we the winner or the loser? Perhaps we think that we are the winner
when we can speak such words to the other person, but in fact, we are
the loser. If we really want to be the winner we should conquer our
defilements. The person who is not angry and does not retaliate upon an
angry person has won a victory which is hard to win. 

When someone else is angry, we should not join him in his anger, we
should not be angry with him and speak harshly to him. If we repay him
in kind, we join him in his anger, we keep company with him, we keep
company with akusala dhamma. Mental development is difficult, it is
conditioned by listening to the teachings which explain the benefit of
kusala dhammas. There must also be energy and courage in order to
develop kusala dhammas. The development of all kinds of kusala is above
all conditioned by satipa\`i\`ih{\aa}na, the development of right
understanding of realities. Satipa\`i\`ih{\aa}na conditions the arising
of sati, mindfulness, which is non{}-forgetful of kusala. There are
different levels of sati: there is sati with generosity, with s\'ila
(morality, the abstaining from ill deeds), with the development of calm
and with the development of right understanding of realities. The
development of satipa\`i\`ih{\aa}na can be the condition that the
different levels of sati arise more often. It conditions sati to
consider the disadvantage of akusala which appears, and to what extent
its disadvantage is realized depends on the stage of the development of
pa\~n\~n{\aa}. When there is sati it is pa\~n\~n{\aa} which can see
akusala dhamma as it is. When pa\~n\~n{\aa} sees akusala as akusala
there are conditions for the arising of kusala instead of akusala. 

Another brother of Bh{\aa}radv{\aa}ja expressed his anger in a way
different from his brothers after he heard that the Bh{\aa}radv{\aa}ja
had entered the Sangha. We read in the following sutta,
{\textasciigrave}{\textasciigrave}The
Congey{}-man{\textquotesingle}{\textquotesingle} (Kindred Sayings I,
Chapter VII, 1, {\S}4):


\bigskip

Again, while the Exalted One was at the Bamboo Grove, the
Bh{\aa}radv{\aa}ja brahmin, known as the Congey{}-man, also heard that
the Bh{\aa}radv{\aa}ja had entered the Sangha. And he, vexed and
displeased, sought the Exalted One{\textquotesingle}s presence, and
when there sat at one side in silence. 

Then the Exalted One, discerning by his mind the thoughts of that
man{\textquotesingle}s mind, addressed him in verse: 


\bigskip

Whoso does wrong to the man that{\textquotesingle}s innocent, 

Him that is pure and from all errors free, 

His wicked act returns upon that fool 

Like fine dust that is thrown against the wind.


\bigskip

Listening to the Dhamma, even for a short time, is very beneficial. When
the Congey{}-man came to see the Buddha he was angry, although he did
not scold him or blame him. However, when he considered with respect
the Dhamma he heard, that is, when he considered cause and effect of
realities, he gained confidence in the Dhamma. He asked to be ordained
under the Buddha. Not long after that he attained the supreme goal of
the higher life, he became one of the arahats. 


\bigskip


\bigskip

\clearpage
Chapter 3


\bigskip


\bigskip

Practice in daily life


\bigskip


\bigskip

Mett{\aa}{}-citta can arise without reciting texts about mett{\aa}. We
find an example of this fact in the
{\textasciigrave}{\textasciigrave}Tu\`u\`eila
J{\aa}taka{\textquotesingle}{\textquotesingle} (III, no. 388). We read
in the Commentary to this J{\aa}taka that the Buddha told this story
while he was at Jetavana. There was a bhikkhu who had great fear of
death. He was frightened when he heard even a branch move, a stick
falling or the call of a bird or another animal. The monks assembled in
the Hall of Truth and spoke about that monk who was so frightened of
death. They said, {\textasciigrave}{\textasciigrave}now to beings in
this world death is certain, life uncertain, and should this not be
wisely born in mind?{\textquotesingle}{\textquotesingle} The Buddha
asked them what the subject of their conversation was and then said
that bhikkhu was afraid of death not only in this life, but also in a
former life. We then read in the Tu\`u\`eila J{\aa}taka that a long
time ago in Var{\aa}nas\'i the Bodhisatta was conceived by a wild sow.
In due time the sow gave birth to two male young. One day she took them
to a pit where they lay down. An old woman came home from the cotton
field with a basket of cotton, and was tapping the ground with her
stick. The sow heard the sound and in fear of death left her young and
ran away. The old woman took the two young pigs home in order to look
after them and she called the bigger one, who was the Bodhisatta,
Mah{\aa}tu\`u\`eila (big{}-snout) and the smaller one Cullatu\`u\`eila
(little{}-snout). She brought them up and treated them as her own
children, but she loved Mah{\aa}tu\`u\`eila more than Cullatu\`u\`eila.
They grew up and became fat. One day there were some young men who
liked to eat pork meat, but they did not know where to get it. They
wanted to buy the pigs from the old woman, but she said that she could
not sell them since she loved them and considered them as her children.
The young men did not give up and offered more money, but she did not
want to sell them. Then they made her drink liquor and when she was
drunk they persuaded her again to sell her pigs. She then agreed to
sell only the small pig, not the big one. She took food and called
Cullatu\`u\`eila, the smaller pig. She had always called
Mah{\aa}tu\`u\`eila first, and thus Mah{\aa}tu\`u\`eila suspected that
there was danger. Cullatu\`u\`eila saw that the trough was full of food
and he noticed that his mistress was standing nearby and that there
were also many men, with nooses in their hands. He became very
frightened and did not want to eat. He ran away to his brother, shaking
with fear. Mah{\aa}tu\`u\`eila comforted him and said that he should
eat and that he should not be sad. He explained that they were fattened
for their flesh{\textquotesingle}s sake. He said that all beings who
are born in this world must die, that nobody could escape death. Every
being, no matter whether his flesh is eatable or not eatable, must die.
He said that their mother was their refuge before, but that they now
had no refuge anymore. They should not have any fear and plunge in the
crystal pool, to wash the stains of sweat away, they would find new
ointment whose fragrance never can decay. We read that
Mah{\aa}tu\`u\`eila considered the ten perfections and set the
perfection of mett{\aa} before him as his guide. The people who heard
him preach were impressed that Mah{\aa}tu\`u\`eila comforted his
brother and then mett{\aa} and compassion arose within them. The
drunkenness left the old woman and the young men and they threw away
their nooses as they stood listening to the Dhamma. 

These men did not have to recite first so that mett{\aa} could arise,
mett{\aa} arose because of its own conditions. 

We read that Cullatu\`u\`eila asked his brother,


\bigskip

But what is that fair crystal pool,

And what the stains of sweat, I pray?

And what the ointment wonderful,

Whose fragrance never can decay?


\bigskip

Mah{\aa}tu\`u\`eila answered,


\bigskip

Dhamma is the fair crystal pool,

Akusala is the stain of sweat, they say:

Virtue{\textquotesingle}s the ointment wonderful,

Whose fragrance never will decay.


\bigskip

Kusala dhammas are like the fair crystal pool because they can purify
one from akusala which is like the stain of sweat. S\'ila is like the
ointment whose fragrance never can decay because when there is s\'ila
one does not harm anybody or do anything which is disagreeable to
others. 

We read that Mah{\aa}tu\`u\`eila said that those who are fools delight
in akusala, whereas those who are heedful do not take to what is
unwholesome. He exhorted beings not to be sad when they had to die. 

When the Buddha had told the story of the former life of that bhikkhu,
he said that Mah{\aa}tu\`u\`eila was he himself in one of his former
lives as Bodhisatta and that Cullatu\`u\`eila was the bhikkhu who was
afraid of death. 

Thus we see that mett{\aa} can arise without reciting texts. Thinking of
the words which are recited arises because of conditions. People
believe that they should recite because they are used to reciting all
the time. When they have such an idea it is a condition to think of the
words they often recite. However, if someone develops mett{\aa} there
is sati{}-sampaja\~n\~na which considers the characteristic of
mett{\aa}, and this is the opposite of akusala dhamma. When we develop
mett{\aa}{}-citta time and again there can gradually be more mett{\aa}.


Thus, we should consider and study with awareness the characteristic of
mett{\aa} as it is explained by the Buddha in many different ways. We
should remember that mett{\aa} will be more powerful if it is truly
developed whenever there is an opportunity for its application.
Mett{\aa} can become stronger and it can arise more often if we
understand the benefit of mett{\aa}. Its arising is not conditioned by
the reciting of texts for a long time. 

Questioner: I remember that when I was a child my father made me recite
texts. I could recite many texts but I did not understand their
meaning. It is the same in the case of reciting texts about mett{\aa}.
The monks are chanting texts each day, in the morning and in the
afternoon, and now I wonder what the use is of reciting. 

Khun Sujin: They may recite that the five khandhas are impermanent or
anatt{\aa}. However, the aim of reciting these words is to be reminded
to consider the characteristics of the five khandhas which are
appearing now and to know them as impermanent. 

Question: That is true, reciting can be a reminder. When I recite,
{\textasciigrave}{\textasciigrave}May Mr. X. be
happy{\textquotesingle}{\textquotesingle}, I do that in order that
there can be mett{\aa} for Mr. X. later on. 

Khun Sujin: We read in the Visuddhimagga (Chapter IX, 1, 2) about the
development of mett{\aa} from the beginning:


\bigskip

To start with, he should review and try to understand the danger in hate
and the advantage in patience. Why? Because hate has to be abandoned
and patience attained in the development of this meditation subject and
because he cannot abandon unseen dangers and attain unknown advantages.



\bigskip

It is not said that people should recite texts but they should know the
right cause which brings the appropriate result. The
{\textasciigrave}{\textasciigrave}Path of
Purification{\textquotesingle}{\textquotesingle} (Visuddhimagga IX, 4)
shows the danger of hate and the benefit of patience. If one really
understands this there are, when dosa has arisen, conditions for
sati{}-sampaja\~n\~na to be aware of it immediately and to see the
danger of dosa at that moment. The Visuddhimagga explains that people
who begin with the development of mett{\aa} as a meditation subject of
calm are advised not to develop it towards four kinds of people: a
person they dislike, a dearly loved friend, a neutral person and a
hostile person. Moreover, mett{\aa} should not be developed towards the
opposite sex and it cannot be developed towards a dead person. It is
difficult to develop mett{\aa} to the kinds of people who were just
mentioned. In the beginning one is not yet ready to do that;
defilements such as anger or attachment are likely to arise on account
of those kinds of people. Mett{\aa} cannot be developed towards a dead
person, because he is no longer the person he was before. The
dying{}-consciousness of this life is succeeded immediately by the
rebirth{}-consciousness of the next life and then there is a different
being. 

The Visuddhimagga explains that in order to make mett{\aa} grow it
should first be developed towards a person one respects, someone who
observes s\'ila, who has wisdom and other good qualities, such as
one{\textquotesingle}s teacher one loves and respects. The reason for
this is that in the beginning mett{\aa} is not yet developed to such
degree that it could be extended towards whomever one meets. In order
to be able to do this it must be developed time and again, evermore. 

Question: What is the proximate cause for the arising of mett{\aa}?

Khun Sujin: Seeing the danger of dosa, aversion or hate. 

Question: Can there be mett{\aa} for what is not alive?

Khun Sujin: That is impossible. Mett{\aa}, karu\`u{\aa} (compassion),
mudit{\aa} (sympathetic joy) and upekkh{\aa} (equanimity), which are
the {\textasciigrave}{\textasciigrave}four divine
abidings{\textquotesingle}{\textquotesingle} (brahma{}-vih{\aa}ras),
must have as object beings or people. Thus there cannot be mett{\aa}
for what is not a living being. However, as regards dosa, aversion,
there can be aversion not only towards beings, but also towards things
or circumstances. 

Question: Can mett{\aa} arise just after seeing visible object? 

Khun Sujin: Mett{\aa} has beings as object. When you see a small child
can there not be mett{\aa}? How will you act when there is mett{\aa}?
You may speak in a kind way, you may help the child to cross the street
or you may give it a sweet. This is the way to develop mett{\aa}. We
can realize ourselves to what extent mett{\aa} is already developed. We
cannot expect mett{\aa} to arise if we do not know its characteristic. 

Question: What is the difference between mett{\aa} which arises just
after seeing and mett{\aa} which arises while we are thinking? 

Khun Sujin: When you see beings and people and you are annoyed you can
be aware of this. After seeing there may be akusala or there may be
mett{\aa}. When there is mett{\aa} you consider the other person as a
friend, you wish for his happiness and want to do everything which is
beneficial for him. You feel happy and cheerful while you think of his
well{}-being, you may smile and you will not behave in any way which
will make him unhappy. Also when you give him something you can do that
in such a way that it truly makes him happy. There are many ways of
giving things to others. Some people give in such a way that the other
person feels no joy when he receives something. When mett{\aa} has
already become more developed, when it has become stronger, it
conditions our actions and speech and also our way of thinking about
other people. Even when we do not see other people we can think of them
with kindness. We can think of promoting their well{}-being and
happiness, we can consider ways to help particular persons, to support
them in different ways. Then there is mett{\aa} without the need to
recite texts. 

Reciting texts on mett{\aa} is actually not so difficult, but truly
developing mett{\aa} is difficult. This cannot be accomplished by
reciting texts. As I said before, there must be sati{}-sampaja\~n\~na
in daily life which knows precisely the characteristic of mett{\aa}. It
must know precisely when there is kusala citta and when akusala citta. 

Question: I think that one should recite in the beginning. 

Khun Sujin: You said the same about the development of
satipa\`i\`ih{\aa}na, you said that one should think before there can
be awareness. You know that seeing is the reality which experiences,
the element which experiences, and that the object which appears,
visible object, is only a physical reality which can be experienced
through the eyes. Sometimes you believe that you need to repeat to
yourself that seeing is the reality, the element, which experiences
through the eyes, and that the object which appears is a physical
reality, r\'upa{}-dhamma, which is experienced through the eyes. When
there is hearing of a sound you believe that you should repeat this to
yourself first, because you think that the reciting of words is very
useful. What you should understand correctly first of all is that each
reality arises because of its appropriate conditions. There are also
conditions for thinking to arise more often than samm{\aa}{}-sati,
right awareness. When samm{\aa}{}-sati arises it can be directly aware
of the realities which appear; it can consider them in the right way,
so that they can be understood as non{}-self. 

As to the term samm{\aa}{}-sati, samm{\aa} can be translated as right,
and sati is awareness or mindfulness. Samm{\aa}{}-sati is aware in the
right way, considers realities in the right way. How is it aware in the
right way? When there is seeing there can be right mindfulness of the
characteristic of the r\'upa{}-dhamma which appears through the eyes,
which is different from n{\aa}ma{}-dhamma. There can be right
mindfulness of n{\aa}ma{}-dhamma, the reality which experiences, the
element which experiences, which is seeing. Then there is
samm{\aa}{}-sati which is mindful in the right way, which is directly
aware of the characteristic of the reality which appears. If people
believe that they should recite, they will continue to do that, instead
of being directly aware of the characteristic of n{\aa}ma or of the
characteristic of r\'upa. If one sees realities as they are, as
non{}-self, anatt{\aa}, one will know that thinking about reciting,
thinking about words one repeats to oneself, is only a reality which
arises because of its own conditions. At such a moment samm{\aa}{}-sati
cannot yet be directly aware of the characteristics of
n{\aa}ma{}-dhammas and r\'upa{}-dhammas, and thus there is not yet
precise knowledge of them. There is only thinking about the
characteristics of realities which appear, thus, there is only
theoretical understanding of them. If pa\~n\~n{\aa} realizes this there
can be the development of samm{\aa}{}-sati, instead of thinking of
reciting or naming realities
{\textasciigrave}{\textasciigrave}n{\aa}ma{\textquotesingle}{\textquotesingle}
and
{\textasciigrave}{\textasciigrave}r\'upa{\textquotesingle}{\textquotesingle},
of repeating this to oneself. Samm{\aa}{}-sati is directly aware of
realities and considers in the right way the characteristics of
n{\aa}ma and r\'upa. 

There is no rule at all in the development of satipa\`i\`ih{\aa}na. When
someone thinks about the names of realities or thinks that he should
recite words in order to remind himself, he should remember that
thinking in this way arises because of conditions. He cannot force
himself not to think in this way. However, thinking is not
samm{\aa}{}-sati of the eightfold Path. When samm{\aa}{}-sati of the
eightfold Path arises, it is aware in the right way of the
characteristics of n{\aa}ma and r\'upa and at that moment
samm{\aa}{}-di\`i\`ihi, right understanding, can investigate the true
nature of n{\aa}ma and r\'upa. Thus pa\~n\~n{\aa} can grow and it can
realize n{\aa}ma and r\'upa as they are: not a being, not a person, not
self. 

In daily life we see beings, we see people who belong to different
families, who are different as to the colour of their skin, who have a
different rank or position in society, who speak different languages,
and who behave in different ways. When we think of people who are so
different in many ways and we have right understanding of the
characteristic of mett{\aa}, sati{}-sampaja\~n\~na can arise and be
aware of the characteristic of the citta which thinks. Then it can be
known what type of citta is thinking, mett{\aa}{}-citta or akusala
citta. When akusala citta thinks of people, it can be realized as such.
For instance the akusala citta which is rooted in attachment,
lobha{}-m\'ula{}-citta, may be accompanied by conceit or it may be
without conceit. Sometimes we think of others with conceit, and
sometimes we think only with attachment but without conceit. Or there
may be akusala citta which is rooted in aversion,
dosa{}-m\'ula{}-citta. We may think of others with aversion or even
anger. Dosa{}-m\'ula{}-citta may at times be accompanied by avarice or
by jealousy. Thus we see that there are different types of akusala
cittas which may think of other people. The understanding of our
different cittas can arise again and again so that it grows and this is
a condition for the arising of mett{\aa} when we see people or when we
think of people in daily life. In that way mett{\aa} can develop more
and more, and there can eventually be mett{\aa} for all beings. If
people want to develop mett{\aa} as a meditation subject which can
condition calm, they cannot do this without precise knowledge of the
different cittas which arise. There must be right understanding which
knows exactly the characteristic of calm which accompanies mett{\aa}
when it appears at a particular moment in daily life. When there is
calm there are at such moments no defilements. 

Question: Reciting, repeating words aloud is useful. When there is
seeing, I say to myself that this is colour, the reality which appears
through eyes, or that is the n{\aa}ma which sees. 

Khun Sujin: I do not say that it is not useful, but it is not
samm{\aa}{}-sati of the eightfold Path, and moreover, there is no rule
that one should recite words. Some people believe that there is a rule
that they should recite words and they cling to this idea. There is
attachment instead of samm{\aa}{}-sati which considers in the right way
the characteristics of n{\aa}ma and r\'upa. When people think that the
reciting of words is useful, they continue to do this again and again.
They should not forget, however, that reciting, the repeating of words,
arises because of its appropriate conditions and that it is not yet
samm{\aa}{}-sati. People should find out for themselves what is more
useful, reciting or samm{\aa}{}-sati of the eightfold Path which
considers in the right way the characteristics of n{\aa}ma and r\'upa
at the moments one does not recite. 

Question: Samm{\aa}{}-sati is certainly better, but my pa\~n\~n{\aa} is
not yet developed to that degree. 

Khun Sujin: This shows that there are conditions for thinking about
realities. However, at such a moment pa\~n\~n{\aa} should also know
that there is not yet samm{\aa}{}-sati of the eightfold Path. When
there are at a particular moment conditions for samm{\aa}{}-sati which
is directly aware of n{\aa}ma and r\'upa, you can find out that the
right understanding which can develop at that very moment is not the
same as the reciting of words. 

Question: If I recite words over and over, for a long time, sati can
arise often and then I can investigate realities with understanding. 

Khun Sujin: This is understanding of the level of thinking, it is
intellectual understanding. There is not yet direct awareness of the
characteristics of n{\aa}ma and r\'upa. You spend a lot of time
reciting, repeating words, but it would be better if there could be
samm{\aa}{}-sati which begins to be aware in the right way of the
characteristics of some n{\aa}mas and r\'upas, little by little. Even
though there is not yet precise knowledge of the characteristics of
n{\aa}ma and r\'upa and there is not yet clear understanding of their
true nature, you can begin to be mindful of their characteristics. Thus
it can gradually become one{\textquotesingle}s inclination to be
mindful of the realities which appear. The arising of samm{\aa}{}-sati
depends on conditions, but when it arises there is direct awareness of
n{\aa}ma and r\'upa and this is more useful than the reciting of words.


Question: That is right. If there can be awareness and direct
understanding of the reality which appears as r\'upa or as n{\aa}ma,
pa\~n\~n{\aa} has developed already to a certain level. However, when
someone is a beginner in the practice, pa\~n\~n{\aa} has not reached
that level yet. 

Khun Sujin: Those who are beginners have different accumulations. If
people have right understanding of the characteristic of
samm{\aa}{}-sati, it can arise. One may not yet be accomplished in the
development of pa\~n\~n{\aa}, but one knows the characteristic of
samm{\aa}{}-sati, the reality which is mindful and directly aware of
the n{\aa}ma which sees or hears or the r\'upa which appears through
one of the senses or the mind. When there is right awareness the
characteristic of the reality which appears can be studied and
investigated. It is true that we cannot prevent thinking from arising,
but we should not cling to it and believe that it is a rule that we
should think of words for a long time and repeat them to ourselves in
order that samm{\aa}{}-sati can arise afterwards. 


\bigskip


\bigskip

\clearpage
Chapter 4


\bigskip


\bigskip

Characteristics of mett{\aa}


\bigskip


\bigskip

The development of satipa\`i\`ih{\aa}na is not repeating words to
oneself, or naming realities
{\textasciigrave}{\textasciigrave}n{\aa}ma{\textquotesingle}{\textquotesingle}
and r\'upa{\textquotesingle}{\textquotesingle}, without investigating
the characteristics of the realities which appear. This becomes clearer
when we read the {\textasciigrave}{\textasciigrave}Vel{\aa}ma
sutta{\textquotesingle}{\textquotesingle} (Gradual Sayings, Book of the
Nines, Chapter II, {\S}10). We read that the Buddha, while he was near
S{\aa}vatth\'i, at the Jeta Grove, spoke to An{\aa}thapi\`u\`eika about
the gifts given by him in a former life, when he was the brahmin
Vel{\aa}ma. He compared the value of different good deeds:


\bigskip

...though with a heart full of confidence he took refuge in the Buddha,
the Dhamma and the Sangha, greater would have been the fruit thereof,
had he with confidence undertaken to keep the precepts: abstention from
taking life, from taking what is not given, from carnal lusts, from
lying and from intoxicating liquor, the cause of sloth. 

...though with confidence he undertook to keep these precepts, greater
would have been the fruit thereof, had he developed a mere passing
fragrance of mett{\aa}. 

...though he developed just the fragrance of mett{\aa}, greater would
have been the fruit thereof, had he developed, just for a
finger{}-snap, anicc{\aa}{}-sa\~n\~n{\aa}, the perception of
impermanence. 


\bigskip

Thus we see that the development of satipa\`i\`ih{\aa}na is of the
greatest value, since through satipa\`i\`ih{\aa}na the characteristics
of realities are seen as they are. 

Mett{\aa} is one of the four brahma{}-vih{\aa}ras, divine abidings. The
development of mett{\aa} is intricate and one should learn about it in
detail. The Buddha explained that mett{\aa} should be developed time
and again so that it can grow. When mett{\aa} has been developed, it
can also support the development of the other brahma{}-vih{\aa}ras of
compassion, sympathetic joy and equanimity. When someone has developed
mett{\aa}, he can have compassion: he will not hurt other beings. He
can have sympathetic joy: he will rejoice in other
people{\textquotesingle}s happiness. Whereas if one does not develop
mett{\aa} one is likely to hurt other beings and one will not rejoice
in their happiness. The Buddha stressed that the development of
mett{\aa} is very beneficial, since mett{\aa} conditions the arising of
other kusala dhammas. Therefore it is important to consider the
development of mett{\aa} more in detail. 

If someone thinks that he can develop mett{\aa} by the recitation of
texts about mett{\aa}, he should try to find out whether this is the
right approach. 

Question: It is written that one should recite:
{\textasciigrave}{\textasciigrave}May all beings be free from
misfortune, may they be free from sorrow and unhappiness, may they live
in happiness. {\textquotesingle}{\textquotesingle}

Khun Sujin: You wish this for all beings, don{\textquotesingle}t you?

Question: That is right. This is actually the extension of mett{\aa}. I
have learnt the P{\aa}li text, but since I do not know the meaning I
use the Thai translation for my recitation. In this way I can
understand the words I recite. 

I think that while I am reciting there is sati. Sometimes it happens
that I am reciting and then, without realizing it, I do not go on with
the reciting. I am at times distracted and I think of other things. But
at other moments I realize that I am reciting and that I should not
think of other things. When I notice that I stop reciting is there then
sati? When there is sati I can start again from the beginning with the
recitation of the text. 

Khun Sujin: You extend mett{\aa} to all beings, but have you attained
jh{\aa}na already? If that is not so how can you extend mett{\aa} to
all beings? When there is mett{\aa} the citta is calm. When you think
of a person you dislike, a person you love or a neutral person and
there is no calm at such moments, how can you extend mett{\aa} to all
beings? As the Visuddhimagga explains, in the beginning it is difficult
to have mett{\aa} for a person one dislikes, a person one loves or a
neutral person. When you recite that you wish happiness for all beings
can you truly extend mett{\aa} to all beings? You can only have
boundless mett{\aa}, including all beings, no matter where they are, if
you have attained jh{\aa}na. 

People should not believe that they, when they begin to develop
mett{\aa}, can truly, wholeheartedly, wish happiness to all beings.
When they really know themselves, they can find out that they do not
mean this. When they think of someone they dislike mett{\aa} does not
arise. Are they then sincere when they recite that they wish happiness
for all beings? As we have seen, the attainment of jh{\aa}na is
necessary in order to be able to extend mett{\aa} to all beings. 

When we think of a person we like, attachment is likely to arise and
this is not mett{\aa}. When we think of someone we hate or of someone
who is a hostile person there is no calm and we are simply not sincere
when we recite for ourselves the text of the mett{\aa} sutta:
{\textasciigrave}{\textasciigrave}May all beings be
happy{\textquotesingle}{\textquotesingle}. If someone wants to develop
calm, he should remember that calm is a wholesome quality arising with
kusala citta. When kusala citta arises there are no defilements and
then there is calm. If the characteristic of calm is known, it can
grow, stage by stage. Mett{\aa} is a meditation subject of samatha
which can condition the growth of calm, and it can also condition
moments of calm in daily life. However, in order to develop mett{\aa}
in the right way, it is not sufficient to think of mett{\aa}, but we
should know first of all the characteristic of mett{\aa}. It is
actually the same as in the case of the development of
satipa\`i\`ih{\aa}na. We cannot develop it if we do not know the
characteristic of sati, mindfulness. We may take thinking for
mindfulness but thinking is different from mindfulness. Sati of
satipa\`i\`ih{\aa}na is not forgetful, it is directly aware of the
reality which appears at the present moment and it considers the
characteristic of that reality. For the development of mett{\aa}
mindfulness is necessary. If there is mindfulness of mett{\aa} when it
appears, its characteristic can be known through direct experience. 

We read in the Atthas{\aa}lin\'i (II, Book II, Part II, The Summary, II,
362) about adosa, non{}-aversion. The Atthas{\aa}lin\'i which is a
commentary to the Dhammasanga\`ui, the first book of the Abhidhamma,
explains in this context the terms used in the Dhammasanga\`ui to
define the reality of adosa:


\bigskip

{\textasciigrave}{\textasciigrave}...having
love{\textquotesingle}{\textquotesingle} is exercising love,
{\textasciigrave}{\textasciigrave}loving{\textquotesingle}{\textquotesingle}
is the method of exercising love; lovingness is the nature of citta
which is endowed with love, is productive of love. Tender care is
watchfulness, the meaning is that one protects. Tenderly caring is the
method of such care. Tender carefulness is the state of tenderly
caring. Beneficence is seeking to do good.
{\textasciigrave}{\textasciigrave}Compassion{\textquotesingle}{\textquotesingle}
is the exercising of compassion...


\bigskip

Before mett{\aa} can be developed we should first of all become familiar
with the characteristic of mett{\aa}. We should carefully consider the
nature of our citta at this moment: is it really accompanied by
mett{\aa} or not? In this way we can begin to develop mett{\aa} very
gradually, by showing kindness to someone else, and then mett{\aa} can
increase. 

We should consider the words of the
{\textasciigrave}{\textasciigrave}Atthas{\aa}lin\'i{\textquotesingle}{\textquotesingle}
about friendship and the attitude of intimacy, of closeness. When we
are sitting together with others, do we have a kind disposition towards
them, do we have sincere friendship? If that is the case, we can learn
what the characteristic of mett{\aa} is. 

No matter whether we meet people in a room, or outside, on the street or
in the bus, do we consider everybody we meet as a friend? If that is
not so we should not recite the words about extending mett{\aa} to all
beings, that will not be of any use. If we see someone now, at this
moment, and we feel misgivings about him, we should not try to extend
mett{\aa} to all beings. Only those who have attained jh{\aa}na are
able to do this. When the meditation subject of mett{\aa}
brahma{}-vih{\aa}ra has been developed mett{\aa} can become boundless.
However, we should begin with simply applying sincere mett{\aa} in
daily life. 

Question: My aim is not jh{\aa}na{}-citta, I do not expect to attain
jh{\aa}na. 

Khun Sujin: Therefore mett{\aa} cannot yet be extended to all beings. 

Question: I recite the words about extending mett{\aa} to all beings
with the aim to have kusala citta. 

Khun Sujin: But when you see a hostile person or when you think of him
annoyance is likely to arise. 

Question: Yes, that is possible. 

Khun Sujin: Therefore you should not try to extend mett{\aa} to all
beings, because you don{\textquotesingle}t mean it. 

Question: I think that it is useful because while I am reciting the
citta is kusala. 

Khun Sujin: This is not possible if you do not start in the right way,
that is, knowing the true characteristic of mett{\aa}. 

Question: It is stated in the Visuddhimagga that one should begin with
extending mett{\aa} towards oneself. 

Khun Sujin: In the beginning people are not yet ready to extend
mett{\aa} to others and therefore they can take themselves as an
example. They can remind themselves that they should treat others in
the same way as they would like to be treated themselves. That is the
meaning of extending mett{\aa} towards oneself. 

Question: Thus the aim is to sympathize?

Khun Sujin: To sympathize with other people. 

Question: Thus we have to extend mett{\aa} towards ourselves, towards a
disagreeable person, towards a loved person and towards a neutral
person. 

Khun Sujin: If you cannot yet have mett{\aa} for a disagreeable person,
you cannot extend mett{\aa} at all. If you try to extend mett{\aa}
towards a dearly loved person, attachment is likely to arise and
attachment has a characteristic which is different from the
characteristic of mett{\aa}. Thus in that case you are not successful
either. Towards whom should we first extend mett{\aa}?

Question: I think towards oneself. 

Khun Sujin: This is said only by way of reminder as we have seen. Those
who are beginners and not yet accomplished should think of someone else
who excels in s\'ila, who has many good qualities which inspire love
and respect. It can be one{\textquotesingle}s teacher or someone who is
the equivalent of one{\textquotesingle}s teacher, someone who is full
of mett{\aa} and other kusala dhammas. When we think of such a person
our citta becomes soft and malleable and we can then be intent on ways
to have kusala citta. We will do everything we can for the benefit and
wellbeing of that person. That is how we can begin with the development
of mett{\aa}. 

The Visuddhimagga (IX, 93) states about the characteristic, function,
manifestation and proximate cause of mett{\aa}:


\bigskip

Mett{\aa} has the characteristic of promoting the aspect of welfare. Its
function is to prefer welfare. It is manifested as the removal of
annoyance. Its proximate cause is seeing lovableness in beings. It
succeeds when it makes ill{}-will subside, and it fails when it
produces selfish affection. 


\bigskip

It is difficult to be watchful as to our cittas, because we are so used
to having akusala. Attachment, aversion and ignorance arise time and
again. In order to develop kusala, pa\~n\~n{\aa}, right understanding
of realities, is necessary. There must be sati{}-sampaja\~n\~na which
knows the characteristic of the citta at a particular moment, which
knows whether there is kusala citta or akusala citta. When we sincerely
wish to do something for another person, not because of attachment, not
because he belongs to our circle of friends or relatives, not because
we expect affection in return, there is the characteristic of
mett{\aa}. 

In order to develop mett{\aa} we should have a detailed knowledge of our
cittas, we should carefully consider the different cittas which arise.
It is in daily life that we can truly develop mett{\aa}, when there is
sati{}-sampaja\~n\~na which knows the characteristic of mett{\aa} which
appears. We may happen to see someone who has a peculiar appearance, or
someone who is a foreigner, someone who speaks a different language.
How do we feel at such a moment? Do we have the same feeling as if we
see a friend or do we have a feeling of antipathy? If we consider that
person, whomever he may be, as a true friend, there is the
manifestation of mett{\aa}. As we have seen in the definition of
mett{\aa} in the Visuddhimagga, the manifestation of mett{\aa} is the
removal of annoyance, of displeasure. 

When we see two people who are angry with each other or who quarrel and
we are partial to one of them there is no mett{\aa} but lobha. As we
have seen in the definition, when there is selfish affection the
development of mett{\aa} fails. We can consider the two people who are
angry with one another as friends, it does not matter who of the two
acted in the proper way and who in the wrong way. When we see someone
who treated us badly, we can still have mett{\aa} towards him, we can
try to help him and we can think of his wellbeing. Then there is true
mett{\aa} which arises at such a moment. There is no mett{\aa} if we
are annoyed with the person who treated us badly, if we blame him and
cause him to be even more upset. 

If someone has mett{\aa} he considers everybody as his friend. If there
is a sincere feeling of friendship for others there can also be
compassion, karu\`u{\aa}, when someone else has to experience sorrow
and misfortune. If someone else experiences happiness, if he has
prosperity and success, there can be sympathetic joy, mudit{\aa}. If we
try to help someone but that person cannot be relieved from distress,
we can develop the brahma{}-vih{\aa}ra of equanimity, upekkh{\aa}, and
then we will not have aversion about the suffering of that person. We
can understand that all dhammas are dependant on their appropriate
conditions. The person who has to suffer receives the result of the
kamma he performed. 

The four brahma{}-vih{\aa}ras are excellent qualities which support all
other kinds of wholesome deeds so that these can develop and reach
perfection. The brahma{}-vih{\aa}ras can support, for example,
generosity. When an opportunity for giving presents itself, we can give
without partiality, whereas when we do not develop the
brahma{}-vih{\aa}ras we may be inclined to give only to a particular
group of people. The brahma{}-vih{\aa}ras are a condition for the
perfecting of s\'ila, good moral conduct through action and speech. We
can perform kusala without expecting favours in return. We can forgive
other people, whatever harm they did to us. Mett{\aa} can indeed
support the other brahma{}-vih{\aa}ras of compassion, sympathetic joy
and equanimity, if the right conditions and the proximate causes for
the other brahma{}-vih{\aa}ras are present. 

We read in the Gradual Sayings (Book of the Fives, Chapter XVII, {\S}1,
The putting away of Malice) that the Buddha teaches that we should
develop all four brahma{}-vih{\aa}ras. We should not believe that
mett{\aa} should first be developed to a high degree and that after
that the other three brahma{}-vih{\aa}ras can be developed. The text
states:


\bigskip

Monks, there are five ways of putting away malice whereby all malice
arisen in a monk ought to be put away. What five? 

Monks, in whatsoever person malice is engendered, in him loving kindness
ought to be made to become more. In this way malice in him ought to be
put away. 

Monks, in whomsoever malice is engendered, in him compassion...
equanimity ought to be made to become more. In this way malice in him
ought to be put away. 

Monks, in whomsoever malice is engendered, in that man unmindfulness,
inattention to it, ought to be brought about. In this way malice in him
ought to be put away. 

Monks, in whomsoever malice is engendered he should remember that people
are owners of their deeds. This should be firmly established in his
mind. He should think: This, reverend sir, is of one{\textquotesingle}s
own making, he is the heir of his deeds, deeds are the matrix, deeds
are the kin, deeds are the foundation; whatever one does, good or bad,
one will become heir to that. In this way malice in him ought to be put
away. 

Verily, monks, these are the five ways of putting away malice. 


\bigskip

It is natural that we are annoyed or irritated about certain people,
that we find them disagreeable. Dosa may be strong and it may last for
a long time, or it may be less intense and disappear soon. We should
remember that even when coarse dosa, such as malice or ill{}-will
arises, it can be subdued by the development of the four
brahma{}-vih{\aa}ras. 

We read in the following sutta (Gradual Sayings, Book of the Fives,
Chapter XVII, {\S}2) that the venerable S{\aa}riputta said to the monks
that, when anger arises, one should have wise consideration of the
different people one is angry with. People are different as to their
conduct through body, speech and mind. Some people may perform good
deeds through the body, but their speech and thoughts are akusala. Some
people perform akusala kamma (bad deeds) through body and mind but
their speech is wholesome. Some people are impure as to their actions
through body and speech but they can have mental calm, they listen to
the Dhamma and they are interested in it. Although they develop calm
their impurity as to body or speech appears from time to time. We can
think of these people without anger, annoyance can be subdued by the
development of mett{\aa}. There can be mett{\aa} when we think only of
someone{\textquotesingle}s good qualities which appear, we should not
pay attention to what he does wrong because then we will have aversion.
It can happen that someone is gentle in his behaviour and that he has
agreeable speech but that his way of thinking is not in accordance with
his conduct through body and speech. When we know this we should pay
attention only to his good qualities, his wholesome conduct through
body and speech, and then mett{\aa} can arise. Some people may have
compassion when they think of someone else, they think of his good
qualities, for example, his wholesome conduct through body and speech,
or, if he has bad conduct through body and speech but he has mental
calm, they think of that quality. They may have compassion and may wish
to help the other person. This shows that they have made progress with
the development of the brahma{}-vih{\aa}ras. We may not be angry with
someone else, but can there be compassion, do we really wish to help
him if he is in trouble? Can we have sympathetic joy when someone with
whom we were annoyed has prosperity, honour, praise and happiness? If
people can rejoice at such an occasion it shows that they have made
progress with the development of the brahma{}-vih{\aa}ras. 


\bigskip

In the
{\textasciigrave}{\textasciigrave}Mett{\aa}{}-sutta{\textquotesingle}{\textquotesingle}
(Gradual Sayings, Book of the Fours, Chapter XIII, {\S}5) we read about
the results of the development of the four brahma{}-vih{\aa}ras. When
someone develops calm and attains jh{\aa}na with mett{\aa} as
meditation subject and the jh{\aa}na does not decline, he is reborn in
the plane of the {\textasciigrave}{\textasciigrave}Devas of the
Brahma{}-group{\textquotesingle}{\textquotesingle} and there the
life{}-span is about one kappa. When someone develops jh{\aa}na with
compassion as subject and the jh{\aa}na does not decline he is reborn
in the plane of the {\textasciigrave}{\textasciigrave}Radiant
Devas{\textquotesingle}{\textquotesingle} and there the life{}-span is
about two kappas. When someone develops jh{\aa}na with sympathetic joy
as subject and the jh{\aa}na does not decline, he is reborn in the
plane of the {\textasciigrave}{\textasciigrave}Ever{}-radiant
Devas{\textquotesingle}{\textquotesingle} and there the life{}-span is
about four kappas. When someone develops jh{\aa}na with equanimity as
subject and the jh{\aa}na does not decline he is reborn in the plane of
the {\textasciigrave}{\textasciigrave}Vehapphala
Devas{\textquotesingle}{\textquotesingle} and there the life{}-span is
about five hundred kappas. 

The development of mett{\aa} has many benefits, it supports other ways
of kusala, such as the {\textasciigrave}{\textasciigrave}ways of
showing sympathy{\textquotesingle}{\textquotesingle}, which are:
liberality, kindly speech, beneficial actions and impartiality, as
explained in the teachings. Mett{\aa} conditions generosity in giving
and it conditions kind, agreeable speech. It makes one abstain from
rude, disgracious conduct, from doing wrong to others. We can help
people with kindness and we can consider them as fellow{}-beings who
are friends. We can learn not to think of them with conceit, as
strangers who are different. We will learn not to think of them in
terms of
{\textasciigrave}{\textasciigrave}he{\textquotesingle}{\textquotesingle}
and
{\textasciigrave}{\textasciigrave}me{\textquotesingle}{\textquotesingle},
or to consider them as superior or as inferior in comparison with
ourselves, because that is conceit. When we investigate the
characteristic of our citta we will know from our own experience that
kusala citta is completely different from akusala citta. 

The Dhammasanga\`ui (Buddhist Psychological Ethics, the first book of
the Abhidhamma, {\S}l340) refers to wholesome qualities such as
plasticity, gentleness, smoothness, pliancy, and humbleness of heart.
The commentary to this passage (Atthas{\aa}lin\'i II, Book III, 395)
describes humbleness of heart as follows:


\bigskip

{\textasciigrave}{\textasciigrave}by the absence of conceit this
person{\textquotesingle}s heart is humble; the state of such a person
is humbleness of heart.{\textquotesingle}{\textquotesingle}


\bigskip

Softness, gentleness, pliancy and humbleness of heart, these qualities
are characteristics of mett{\aa}. S{\aa}riputta was an example of
humility. He compared himself with a dust rag, an old rag without any
value. He had no arrogance, he was not conceited about it that he was
one of the foremost disciples. Even when others behaved badly towards
him through body or speech he was unaffected by it since he was an
arahat. He had eradicated conceit and all the other defilements and
thus he was of perfect gentleness and humility. 

Can we have true humility? When there is unwholesomeness in our actions
and speech we should be mindful of the characteristic of citta at such
moments. We can find out that we are full of defilements and that these
condition our behaviour and speech. When there is sincere humility
there cannot be unwholesome speech. Our behaviour and our speech
reflect our citta: kusala citta or akusala citta. Is there mett{\aa} or
is there conceit? If we want to strive earnestly for the eradication of
defilements we should be mindful of the different cittas. Then we will
notice what our normal behaviour and speech is in our daily life. We
will know when they are motivated by akusala citta and when by kusala
citta. 


\bigskip

\clearpage
Chapter 5


\bigskip


\bigskip

Mett{\aa} in action and speech


\bigskip


\bigskip

Mett{\aa} supports other kusala dhammas, it is also a condition for
patience. We read in the Dhammasanga\`ui:


\bigskip

{\S}1341: What is patience (khanti)?

That patience which is long{}-suffering, compliance, absence of rudeness
and abruptness, complacency of citta. 

{\S}1342: What is temperance (soracca\"y)?

That which is the absence of excess in deed, in word, and in deed and
word together. Besides, all moral self{}-restraint (sa\'ovara s\'ila)
is temperance. 

{\S}1343: What is amity (s{\aa}khalya\"y)?

When all such speech as is insolent, disagreeable, scabrous, harsh to
others, vituperative to others, bordering upon anger, not conducive to
concentration, is put away, and when all such speech as is innocuous,
pleasant to the ear, affectionate, such as goes to the heart, is
urbane, sweet and acceptable to people generally; when speech of this
sort is spoken, polished, friendly and gentle speech, this is what is
called amity. 


\bigskip

We read in the Atthas{\aa}lin\'i (Book II, Part II, Chapter II, 396) the
following explanation of the passage on amity in the Dhammasanga\`ui:


\bigskip

In the exposition of amity,
{\textasciigrave}{\textasciigrave}insolent{\textquotesingle}{\textquotesingle}
means, as knobs protrude in a decaying or unhealthy tree, so, owing to
faultiness, knobs are produced from words of abusing and slandering,
etc.
{\textasciigrave}{\textasciigrave}Scabrous{\textquotesingle}{\textquotesingle}
means putrid, like a putrid tree. As a putrid tree is scabrous and has
trickling, powdery tissue, so such speech is scabrous and enters as
though piercing the ear. {\textasciigrave}{\textasciigrave}Harsh to
others{\textquotesingle}{\textquotesingle} means bitter to the ears of
others, not pleasant to their hearts and productive of dosa.
{\textasciigrave}{\textasciigrave}Vituperative to
others{\textquotesingle}{\textquotesingle} means, as a branch with
barbed thorns sticks by penetrating into leather, so it sticks to
others and clings on, hindering those who want to go.
{\textasciigrave}{\textasciigrave}Bordering on
anger{\textquotesingle}{\textquotesingle} means near to anger.
{\textasciigrave}{\textasciigrave}Not conducive to
concentration{\textquotesingle}{\textquotesingle} means not conducive
to attainment concentration (appan{\aa}{}-sam{\aa}dhi) and
access{}-concentration. All these terms are synonyms of the words
{\textasciigrave}{\textasciigrave}with
hate{\textquotesingle}{\textquotesingle}... 

{\textasciigrave}{\textasciigrave}Pleasant to the
ear{\textquotesingle}{\textquotesingle}, that is, from sweetness of
diction it is pleasant to the ear; it does not produce pain to the ear,
like the piercing of a needle. And from the sweetness of sense and
meaning not producing ill{}-temper in the body, it produces affection,
and so is called
{\textasciigrave}{\textasciigrave}affectionate{\textquotesingle}{\textquotesingle}.
That speech which appeals to the heart, which enters the mind easily
without striking, we say {\textasciigrave}{\textasciigrave}goes to the
heart{\textquotesingle}{\textquotesingle}.
{\textasciigrave}{\textasciigrave}Urbane
speech{\textquotesingle}{\textquotesingle} is so called because it is
full of good qualities, and because it is refined like well{}-bred
persons, and because it is of the town (urban). It means talk of
citizens. For these use appropriate speech and address fatherly men as
fathers, and brotherly men as brothers.
{\textasciigrave}{\textasciigrave}Of{}-much{}-folk{}-sweetness{\textquotesingle}{\textquotesingle}
means sweet to many people.
{\textasciigrave}{\textasciigrave}Of{}-much{}-folk{}-pleasantness{\textquotesingle}{\textquotesingle}
means pleasant to many people and making for the growth of mind.
{\textasciigrave}{\textasciigrave}The speech which
there{\textquotesingle}{\textquotesingle}, that is, in that person,
{\textasciigrave}{\textasciigrave}is
gentle{\textquotesingle}{\textquotesingle}, i.e. polished,
{\textasciigrave}{\textasciigrave}friendly{\textquotesingle}{\textquotesingle},
that is soft,
{\textasciigrave}{\textasciigrave}smooth{\textquotesingle}{\textquotesingle},
that is, not harsh. 


\bigskip

In connection with amity there is another term, namely
{\textasciigrave}{\textasciigrave}courtesy{\textquotesingle}{\textquotesingle}
(patisanth{\aa}ro). One should not merely have speech which is
blameless, pleasant to the ears, affectionate, which goes to the heart
and which is urbane. It is important to have also courtesy through
loving kindness. When one really develops mett{\aa} one is not without
courtesy. We read in the Dhammasanga\`ui:


\bigskip

{\S}1344: What is courtesy (patisanth{\aa}ro)?

The two forms of courtesy: hospitality towards bodily needs and
considerateness in matters of the Dhamma. When anyone shows courtesy it
is in one or other of these two forms. 


\bigskip

There cannot be real courtesy if there is no mett{\aa}. When there is
sincere courtesy in daily life it is evident that there is
mett{\aa}{}-citta. If we do not have courtesy in our daily life we
should develop mett{\aa} so that we can help other people with courtesy
in our deeds and speech. 

The Atthas{\aa}lin\'i (397) explains the term courtesy:


\bigskip

In the exposition of courtesy, {\textasciigrave}{\textasciigrave}carnal
courtesy{\textquotesingle}{\textquotesingle} ({\aa}misa
patisanth{\aa}ro) is the closing, covering up, by means of bodily
needs, the gap which might exist between oneself and others owing to
those needs not getting satisfied. 


\bigskip

Thus, this refers to helping others by giving them things they need, by
looking after them. There is a gap or separation between people all the
time, between those who posses things and those who are needy. However,
there is a means to close such a gap and that is by material courtesy,
by giving assistance with material things, helping those in need. Then
there is no longer a separation or distance between people. 

As to {\textasciigrave}{\textasciigrave}Dhamma
courtesy{\textquotesingle}{\textquotesingle} (dhamma patisanth{\aa}ro),
this is the closing of the gap which might exist between oneself and
others who did not learn the Dhamma. When we see the benefit of the
Dhamma and we think it appropriate to help others by explaining the
Dhamma there is courtesy of Dhamma. Then the Dhamma covers completely
the gap or separation between people. 

We read further on in the Atthas{\aa}lin\'i (398) about material
courtesy of the monk:


\bigskip

A courteous bhikkhu, on seeing a guest arrive, should meet him and take
his bowl and robe, offer him a seat, fan him with Palmyra leaf, wash
his feet, rub him with oil; if there be butter and syrup he should give
him medicine, offer him water, scour up the monastery{}-{}-thus in one
part is material courtesy shown. 

Lay{}-followers should consider by which means they can in their own
situation show material courtesy. As to Dhamma courtesy by which people
can help one another, we read in the Atthas{\aa}lin\'i: 


\bigskip

Moreover, at eventide, if there be no junior who comes to pay his
respects, the bhikkhu should go to the presence of his guest, sit there
and, without asking him irrelevant things, question him on relevant
things. He should not ask {\textasciigrave}{\textasciigrave}What texts
do you recite? but should ask {\textasciigrave}{\textasciigrave}What
scriptural text does your teacher and spiritual adviser
use?{\textquotesingle}{\textquotesingle} and should question him on
points within his capacities. Should the guest be able to answer, that
is good; if not, he himself should give the reply. Thus in one part is
courtesy of Dhamma shown. 


\bigskip

This shows that there is thoughtfulness when we speak with mett{\aa}.
When we want to help others with Dhamma we should not explain what is
beyond the listeners capacity to understand or to receive. We should
take into consideration the accumulations and the disposition of the
listener and speak about the Dhamma in such a way that he can
understand it. 

Mett{\aa} supports other kusala dhammas and it has many benefits. If we
know about these benefits we can verify for ourselves whether mett{\aa}
is already of such degree that we can have them. Thus, reading about
them can remind us to develop mett{\aa} to that degree. 

We read in the
{\textasciigrave}{\textasciigrave}Mett{\aa}{}-sutta{\textquotesingle}{\textquotesingle}
(Gradual Sayings, Book of the Eights, Chapter I, {\S}1):


\bigskip

Thus have I heard:

Once the Exalted One was dwelling near S{\aa}vatth\'i, at Jeta Grove, in
An{\aa}thapi\`u\`eika{\textquotesingle}s Park. There the Exalted One
addressed the monks, saying:
{\textasciigrave}{\textasciigrave}Monks{\textquotesingle}{\textquotesingle}.


{\textasciigrave}{\textasciigrave}Yes, lord,
{\textquotesingle}{\textquotesingle} they replied, and the Exalted One
said:

Monks, by the release of the heart through mett{\aa} (mett{\aa}
cetovimutti), practised, made become, made much of, made a vehicle and
a basis, exercised, augmented and set going, eight advantages are to be
expected. What eight?

Happy one sleeps; happy one awakes; one sees no bad dreams; one is dear
to humans; one is dear to non{}-humans; devas guard one; neither fire,
nor poison, nor sword affects one; and though one penetrate not the
beyond, one reaches the Brahm{\aa} world. 

Monks, by the release of the heart through amity, practised, made
become, made much of, made a vehicle and a basis, exercised, augmented
and set going, these eight advantages are to be expected. 


\bigskip

Who does make mett{\aa} to grow

Boundless and thereto sets his mind,

Seeing the end of birth{\textquotesingle}s substrate

In him the fetters are worn away.

If with a heart unsoiled one feels

Mett{\aa} towards a single being, 

He is a good man (just) by that. 

Compassionate of heart to all

The ariyan boundless merit makes. 

Those royal sages who, conquering

The creature teeming earth, have ranged

Round and about with sacrifice...

Such do not share a sixteenth part

The worth of mett{\aa}{}-citta made to grow, 

Just as the radiance of the moon

Outshines all the starry host.

Who kills not nor makes others kill, 

Robs not nor makes others rob, 

Sharing goodwill with all that lives,

He has no hate for anyone. 


\bigskip

One of the benefits of the development of mett{\aa} is that one sleeps
happily. If we are angry with someone can we then sleep happily? If we
are not angry with anyone, if we have no hate and we can forgive
anybody whatever wrong he may have done, we can really sleep happily.
If sati{}-sampaja\~n\~na arises when it is time to go to sleep, we can
find out what type of citta arises before falling asleep. We can find
out whether there is at such a moment lobha, dosa, satipa\`i\`ih{\aa}na
or mett{\aa}. If we develop satipa\`i\`ih{\aa}na there can be
pa\~n\~n{\aa} which knows the characteristics of realities as they are.
When the reality which appears at a particular moment is akusala,
sati{}-sampaja\~n\~na (pa\~n\~n{\aa} arising with sati) can realize
akusala as akusala. Pa\~n\~n{\aa} can distinguish the difference
between kusala dhamma and akusala dhamma and thus it is able to
eliminate akusala more and more. The development of kusala is the only
way to have the benefit of sleeping happily. 

Waking up happily is another benefit. When it is time to get up in the
morning we can find out whether mett{\aa} has been sufficiently
developed so that we can have this benefit. If there is anger remaining
in our heart, the citta will be disturbed when we wake up; we are
preoccupied with events we can{\textquotesingle}t forget. In reality
there is no self, being or person, but there are conditions for citta
to be disturbed. As soon as we wake up sa\~n\~n{\aa} (remembrance)
remembers the event which causes us to be annoyed. Or when we have done
something wrong and we worry because of this, we cannot help thinking
of this as soon as we wake up. When we have done something wrong we are
likely to worry about it and to feel unhappy when we go to sleep, and
then we are also unhappy when we wake up. When there is akusala citta
before going to sleep there will also be akusala citta as soon as we
wake up. When there is akusala citta rooted in lobha, and there is no
mindfulness of it, we will not realize it that there is clinging as
soon as we wake up. There is clinging to the objects which appear
through eyes, ears, nose, tongue, body{}-sense and mind{}-door. We
usually do not notice attachment to the sense objects when it is of a
slight degree and we do not see its disadvantage and danger. 

Dosa is a reality which is more coarse and thus it is less difficult to
realize it as akusala than in the case of lobha. When there is dosa the
citta is disturbed and unhappy. Lobha is not coarse and fierce like
dosa, it is difficult to realize it as akusala. If we develop
satipa\`i\`ih{\aa}na naturally, in daily life, we will know the
characteristics of realities just as they are, we will know when there
is lobha and when there is dosa. 

One of the benefits of the development of mett{\aa} is not having bad
dreams. Unwholesome, impure thoughts can arise even in dreams, they
cannot be prevented. Our accumulated inclinations condition the arising
of cittas in mind{}-door processes which think about the objects which
were formerly experienced through the six doors. We remember all these
objects and dwell on them with our thoughts. People{\textquotesingle}s
accumulated defilements condition different dreams. We can sometimes
know whether there were kusala cittas or akusala cittas while we were
dreaming. Then we can scrutinize ourselves as to our accumulations, we
can see whether kusala or akusala has been accumulated. If one has
accumulated a great deal of mett{\aa} one will not have bad dreams,
thus, there will not be akusala citta which dreams. 

{\textasciigrave}{\textasciigrave}One is dear to
humans{\textquotesingle}{\textquotesingle} is another benefit of the
development of mett{\aa}. Do we know of ourselves whether we are
usually liked by others? When we investigate the characteristics of our
cittas we can know why we are liked or disliked by others. Some people
blame kamma of the past for the fact that, although they do all kinds
of good deeds they are still not liked by other people. Therefore they
feel slighted and disappointed. Other people can hurt or harm us only
through their actions and speech. When they speak in a disagreeable
way, the r\'upa which is ear{}-sense is a condition to hear different
sounds which can disturb us. However, in reality our citta cannot be
harmed by someone else at all, it can only be harmed by ourselves.
Other people can only cause us to have bodily suffering; it is our own
akusala citta which is the cause of mental suffering. Thus, instead of
thinking of all the different things which cause us to be distressed we
should cultivate mett{\aa} and we should forgive other people. Then the
citta is not disturbed and it is evident that nobody can do harm to our
citta. 

We want to be dear to others but we may forget that we ourselves should
also show affection to other people. We should not expect that other
people will first show kindness and affection; there should be no delay
in being kind and considerate to others. At such moments we have no
sadness or worry. The citta with mett{\aa} is kusala, at that moment
there is no lobha, no wish to have affection from someone else in
return. 

If one knows the characteristic of kusala citta and discerns the
difference between kusala citta and akusala citta there are conditions
to develop a great deal of kusala without being concerned about it
whether one is liked by other people or not. When there is mett{\aa}
and generosity, when one helps other people, there is the cetasika
chanda,
{\textasciigrave}{\textasciigrave}wish{}-to{}-do{\textquotesingle}{\textquotesingle},
which conditions the arising of kusala citta. The desire for kusala is
different from lobha. When lobha arises we desire to be liked by
others. Whereas when kusala chanda arises, we desire to develop loving
kindness towards others, even when we do not receive any kindness from
them. 

If satipa\`i\`ih{\aa}na is not developed, we cannot clearly distinguish
between the different characteristics of lobha and of kusala chanda
which desires the development of kusala. There may be attachment to the
development of kusala or to the benefits of kusala because clinging
cannot yet be eliminated. We know that good deeds bring their
appropriate results but when we have expectations, when we hope that
our good deeds will bring pleasant results, there is lobha. When there
is kusala chanda, desire for the development of kusala, there is no
attachment, there are no expectations with regard to the result of
kusala. Then we can develop kusala with a sincere inclination, we can
develop it naturally and spontaneously. 

{\textasciigrave}{\textasciigrave}One is dear to
non{}-humans{\textquotesingle}{\textquotesingle}, this is another
benefit of the development of mett{\aa}. When there is chanda, desire
for the development of kusala, we do not expect to be liked by human
beings nor by non{}-humans, because we do not hope for the result of
kusala, we do not hope for any benefit. When there is pure kusala one
is dear to non{}-humans. 

{\textasciigrave}{\textasciigrave}Devas guard
one{\textquotesingle}{\textquotesingle}, this is another benefit. When
we develop mett{\aa}, kusala citta has as effect that we are dear to
humans and non{}-humans and that devas guard us with mett{\aa}. The
right cause brings its appropriate effect, and there is no need to wish
for such result. 

{\textasciigrave}{\textasciigrave}Neither fire, nor poison nor sword
affects one{\textquotesingle}{\textquotesingle}, this is another
benefit. When there is pure kusala citta with mett{\aa}, it can protect
us from dangers, even if we have not attained
{\textasciigrave}{\textasciigrave}access
concentration{\textquotesingle}{\textquotesingle} or jh{\aa}na. When
someone develops calm with mett{\aa} as meditation subject and his
kusala citta is of such degree of steadfastness that jh{\aa}na can be
attained, he will not be affected by fire, poison or sword. 

{\textasciigrave}{\textasciigrave}Even when one does not reach the
highest, one will be reborn in the Brahm{\aa}
world{\textquotesingle}{\textquotesingle}, this is another benefit,
which, as I shall explain, shows clearly that satipa\`i\`ih{\aa}na
should be developed together with all the other kinds of kusala. When
someone develops samatha with mett{\aa} as subject, and he can attain
calm which is steadfast, and which is of the degree that the first
jh{\aa}na can be reached, the result can be rebirth in the
brahma{}-plane of the first jh{\aa}na. When higher stages of jh{\aa}na
are attained, the result is rebirth in higher brahma{}-planes in
accordance with the stage of jh{\aa}na which produces rebirth. However,
the highest benefit which can be reached is, after the realisation of
the four noble Truths at enlightenment, to attain the state of the
arahat, the perfected one. Then there will be the end of rebirth. The
text states that when one does not penetrate to the highest dhamma,
that is, the state of the arahat, one will be reborn in the
brahma{}-world. What is most important is the realisation of the noble
Truths. This should be one{\textquotesingle}s goal. Therefore mett{\aa}
should be developed together with satipa\`i\`ih{\aa}na and not merely
for the sake of attaining calm to the degree of access concentration or
jh{\aa}na. We should develop satipa\`i\`ih{\aa}na time and again in our
daily life, and then the other kinds of kusala will also grow. 

As we read in the sutta, the Buddha also said that the person who, with
mindfulness established, develops boundless mett{\aa} will realize the
elimination of attachment and all other
{\textasciigrave}{\textasciigrave}fetters{\textquotesingle}{\textquotesingle}.
He will not harm any being while he develops mett{\aa}{}-citta, he will
only be intent on what is wholesome. He has compassion for all beings,
he is an excellent person with abundant merit. 


\bigskip

\clearpage
Chapter 6


\bigskip


\bigskip

Benefits of mett{\aa}


\bigskip


\bigskip

We read about eleven benefits of mett{\aa} in the Gradual Sayings (Book
of the Elevens, Chapter II, {\S}5, Advantages):


\bigskip

Monks, eleven advantages are to be looked for from the release of heart
(cetovimutti) by the practice of mett{\aa}, by making mett{\aa} to
grow, by making much of it, by making mett{\aa} a vehicle and a basis,
by persisting in it, by becoming familiar with it, by well establishing
it. What are the eleven?

One sleeps happy and wakes happy; he sees no evil dream; he is dear to
human beings and non{}-human beings alike; the devas guard him; fire,
poison or sword afflict him not; quickly he concentrates his mind; his
complexion is serene; he makes an end without bewilderment; and if he
has penetrated no further (to arahatship) he reaches (at death) the
Brahma{}-world. 

These eleven advantages are to be looked for from the release of heart
by the practice of mett{\aa}...by well establishing mett{\aa}. 


\bigskip

The same eleven benefits of the development of mett{\aa} are mentioned
in the Path of Discrimination (Treatise XVI, loving kindness). The Path
of Discrimination deals with the development of mett{\aa} which is
fortified by the five {\textasciigrave}{\textasciigrave}spiritual
faculties{\textquotesingle}{\textquotesingle} or indriyas (confidence,
energy, sati, concentration and understanding), and the five powers,
balas. The indriyas develop in satipa\`i\`ih{\aa}na, and they can
become firm and unshakable, they can become
{\textasciigrave}{\textasciigrave}powers{\textquotesingle}{\textquotesingle}.
If one does not develop satipa\`i\`ih{\aa}na in one{\textquotesingle}s
daily life it is difficult to have true loving kindness, because
mett{\aa} needs the support of the indriyas and powers which develop in
satipa\`i\`ih{\aa}na. To the degree that mett{\aa} is supported by
these cetasikas, it becomes more established; there will be less
disturbance by defilements and this means more calm. When mett{\aa} is
well established it is unshakable, it does not waver because of
defilements. Thus, for the development of mett{\aa} there must be a
detailed knowledge of one{\textquotesingle}s different cittas, there
must be sati sampaja\~n\~na which knows when there is wavering and when
mett{\aa} is firm and unshakable. In order to know this, right
understanding of one{\textquotesingle}s cittas is indispensable.
Defilements can only be eradicated by pa\~n\~n{\aa} which knows the
characteristic of the reality appearing right now. Right understanding
of this very moment should be developed, because what is past has gone
already and the future has not come yet. Pa\~n\~n{\aa} which arises
falls away again but because each citta which falls away is succeeded
by the next one, pa\~n\~n{\aa} can be accumulated from moment to
moment, and in this way there are conditions for pa\~n\~n{\aa} to
become more established. 

When we read about the benefits of mett{\aa} we can, instead of wishing
for these benefits, check to what extent we have developed mett{\aa}
already. If we do not have these eleven benefits it is evident that we
have not sufficiently developed mett{\aa}. 

Question: The arahat is habitually inclined to mett{\aa}. Why did
Mah{\aa} Moggall{\aa}na have to be killed through the sword?

Khun Sujin: That was the result of past kamma. Of course, since the time
he had become an arahat, he did not commit any more kamma. 

Question:I would think that since he was an arahat he could not receive
such a result of kamma. Past akusala kamma would be in this case
{\textasciigrave}{\textasciigrave}ahosi
kamma{\textquotesingle}{\textquotesingle}, kamma which is ineffectual. 

Khun Sujin: So long as the arahat has not passed away there are still
conditions for past kamma to produce result. When the arahat has
finally passed away there is no more rebirth, no more arising of citta,
cetasika and r\'upa, and then there cannot be anymore receiving of the
result of kamma. 

When sati arises we can find out whether there is mett{\aa}, we can know
whether it is strong or weak. Sati can be aware of the characteristic
of mett{\aa}, it can find out whether there is true mett{\aa} or not.
The characteristic of mett{\aa} may be confused with the characteristic
of lobha. If there is no sati sampaja\~n\~na it cannot be known whether
there is mett{\aa} or lobha. We usually want other people to be happy,
but do we want this because we love them with attachment or because we
have true loving kindness for them without any selfishness? When there
is sati sampaja\~n\~na we will know whether there is at such a moment
lobha or mett{\aa}. When we really understand the difference mett{\aa}
can develop and lobha can decrease. 

People may doubt whether there is lobha or mett{\aa} when they want
their parents to be happy, because lobha and mett{\aa} seem to be
similar. When we think of the good qualities of our parents and we
desire their welfare there is kusala citta with mett{\aa}. When we love
our parents and we are attached to them there is lobha. It is the same
with the relationship of parents towards children, when they have
selfish affection or possessive love for their children; there is
lobha. However, if they have listened to the Dhamma and developed
satipa\`i\`ih{\aa}na and if they can distinguish the difference between
the characteristics of mett{\aa} and of lobha, they will have more
mett{\aa} towards their children and less attachment. If they do not
develop mett{\aa} there will be selfishness, they consider their child
as {\textasciigrave}{\textasciigrave}our
child{\textquotesingle}{\textquotesingle}. Attachment to
one{\textquotesingle}s child can even lead to harming someone
else{\textquotesingle}s child. In that case there is no mett{\aa}
towards one{\textquotesingle}s child but selfish affection. 

We read in the Visuddhimagga, in the section on the Divine Abiding of
Mett{\aa} (IX, 11), that if a person wants to develop mett{\aa} he
should extend it first towards someone who has moral excellence and
other good qualities, someone he esteems and respects, such as his
teacher. When we think of the qualities of such a person our mind
becomes gentle, we have no thoughts of malevolence. We wish to help our
teacher, to do everything for his benefit and happiness. Thus, the
citta which thinks of the good qualities of one{\textquotesingle}s
teacher is gentle and mellow, it is citta with mett{\aa}. When we are
happy to give assistance to someone we meet in daily life, in the same
way as we would give assistance to our teacher, it is evident that we
have mett{\aa} towards that person. 

The Buddha praised the development of mett{\aa}, even if it is just for
a short moment. We should not think that there is any kind of kusala
which is unimportant, we should remember that even a short moment of
kusala is beneficial. We read in the Kindred Sayings (II, Nid{\aa}na
vagga, Chapter XX, Kindred Sayings on Parables, {\S}4, The rich gift)
that the Buddha, while he was staying at S{\aa}vatth\'i, at the Jeta
Grove, said to the monks:


\bigskip

If anyone, monks, were to give a morning gift of a hundred
{\textasciigrave}{\textasciigrave}ukkas{\textquotesingle}{\textquotesingle},
and the same at noon and the same at eventide, or if anyone would
develop mett{\aa} in the morning, at noon or at eventide, even if it
were as slight as one pull at a cow{\textquotesingle}s udder, this
practice would be by far the more fruitful of the two. 

Wherefore, monks, thus should you train yourselves: liberation of heart
by mett{\aa} (mett{\aa} cetovimutti) we will develop, we will often
practise it, we will make it a vehicle and a base, take our stand upon
it, store it up, thoroughly set it going. 


\bigskip

The Buddha taught that all kusala dhammas can be gradually developed.
Even if one finds it difficult to develop kusala, it can be accumulated
so that it can arise more often and become more powerful. We should not
think that we can have a great deal of mett{\aa} immediately, but each
short moment of mett{\aa} is a condition that mett{\aa} develops.
Otherwise the Buddha would not have taught that mett{\aa} even for the
duration of one pull of a cow{\textquotesingle}s udder is beneficial. 

When we develop mett{\aa} we should know for what purpose we develop it.
Do we develop it in order to attain calm to the degree of access
concentration or attainment concentration? Or do we want to develop it
in our daily life? Mett{\aa} and the other
{\textasciigrave}{\textasciigrave}perfections{\textquotesingle}{\textquotesingle}
are necessary conditions for the realisation of the four Noble Truths
at enlightenment. We are bound to be for an endlessly long time in the
cycle of birth and death, and we do not know when the perfections will
have developed to the degree that enlightenment can be attained.
Therefore, we should develop all kinds of kusala in order that
eventually defilements can be completely eradicated and the state of
the arahat can be attained. Only then will there be the end of the
cycle of birth and death. Some people believe that defilements can be
eradicated without the development of mett{\aa}. Or they believe that
mett{\aa} is too difficult and therefore they do not develop it. They
do not understand that mett{\aa} should be developed in order that it
can arise again and again. Only if it arises time and again it can
gradually be accumulated. We may believe that mett{\aa} is too
difficult but we should remember that the arising of pa\~n\~n{\aa}
which realizes the noble Truths is even more difficult. We should not
be discouraged, we should not give akusala the opportunity to gain in
strength by wrongly believing that mett{\aa} is too difficult, that it
cannot arise and that it therefore should not be developed. When sati
arises we can have right understanding of the development of mett{\aa}:
we can see that it can arise, that it can be developed little by
little. In this way mett{\aa} will become more powerful, it will become
steadfast. There can be mett{\aa} with our actions, our speech and our
thoughts. 

When we begin to develop mett{\aa} it is necessary to first see the
disadvantage of dosa, aversion or anger. Dosa is the dhamma which is
opposed to mett{\aa}. Whenever dosa arises it is evident that mett{\aa}
is lacking. Dosa is the dhamma (reality) which is harsh, it causes harm
to ourselves and to others. When dosa arises it overwhelms the citta,
it inflames citta like a fire. The destructive power of dosa causes
people to harm others through body and speech, in various degrees in
accordance with its strength. We read in the Kindred Sayings (I,
Sag{\aa}th{\aa}{}-vagga, I, The Devas, 8, Slaughter suttas, {\S}1) that
a deva asked the Buddha:


\bigskip

What must we slay if we would live happily?

What must we slay if we would weep no more?

What is it above all other things, whereof

The slaughter you approve, Gotama?


\bigskip

The Buddha answered:


\bigskip

Wrath must you slay if you would live happily, 

Wrath must you slay if you would weep no more. 

Of anger, deva, with its poisoned root

And fevered climax which is sweet, 

That is the slaughter by the ariyans praised;

That must you slay to weep no more. 


\bigskip

This shows that when anger arises there is disturbance of mind, we are
unhappy. We have unkind thoughts or even malevolence, we may harm the
person we are angry with through body or speech so that he will suffer.
We can harm him in different ways, for example by violence, by hitting
him and causing him to suffer bodily injuries. Or we may utter harsh,
fierce words. When we have injured someone else through body and speech
we may be satisfied with what we have done. The Buddha said that wrath
has a poisonous root and a sweet tip. The feeling of satisfaction we
have when we have done harm to someone else is compared to the sweet
tip of anger, but its root is poisoned. Each person will receive the
result of his action. When dosa conditions someone to do harm to
another person there is akusala kamma which has a poisonous root:
akusala kamma produces an unpleasant result for the person who performs
it in the form of loss and other unpleasant experiences. It can cause
rebirth in unhappy planes such as a hell plane, the plane of ghosts
(petas) or demons (asuras), or rebirth as an animal, depending on the
degree of that akusala kamma. 

If we see the disadvantage of akusala citta and akusala kamma we will
develop mett{\aa} in order to diminish the accumulation of the
different akusala dhammas. We should consider the benefit of patience,
patience for the development of kusala and perseverance with it, so
that akusala can be eliminated. We read in the Middle Length Sayings
(I, no. 21, Discourse on the Parable of the Saw) that the Buddha, while
staying near S{\aa}vatth\'i, at the Jeta Grove, said to the monks:


\bigskip

There are, monks, these five ways of speaking in which others when
speaking to you might speak: at a right time or at a wrong time;
according to fact or not according to fact; gently or harshly; on what
is connected with the goal or on what is not connected with the goal;
with a mind of friendliness or full of hatred. Monks, when speaking to
others you might speak at a right time or at a wrong time; monks, when
speaking to others you might speak according to fact or not according
to fact; monks, when speaking to others you might speak about what is
connected with the goal or about what is not connected with the goal;
monks, when speaking to others you might speak with a mind of
friendliness or full of hatred. Herein, monks, you should train
yourselves thus: {\textasciigrave}{\textasciigrave}Neither will our
minds become perverted nor will we utter an evil speech, but kindly and
compassionate will we dwell, with a mind of friendliness, void of
hatred; and we will dwell having suffused that person with a mind of
friendliness; and, beginning with him, we will dwell having suffused
the whole world with a mind of friendliness that is far{}-reaching,
widespread, immeasurable, without enmity, without
malevolence.{\textquotesingle}{\textquotesingle} This is how you must
train yourselves, monks. 


\bigskip


\bigskip

In this sutta several similes are used to show that when there is
mett{\aa} there cannot be any anguish. Mett{\aa}{}-citta is for example
compared to a cat{}-skin bag which is supple and well cured. Even when
someone hits it with a piece of wood no noise at all can be heard. In
the same way, when there is mett{\aa}{}-citta, there cannot be anything
which could cause the arising of dosa. We read that the Buddha said:


\bigskip

Monks, as low{}-down thieves might carve one limb from limb with a
double{}-handled saw, yet even then whoever sets his mind at enmity,
he, for this reason, is not a doer of my teaching...


\bigskip


\bigskip

\clearpage
Chapter 7


\bigskip


\bigskip

The blessings of mett{\aa}


\bigskip


\bigskip

The Buddha taught Dhamma to his followers out of compassion, he taught
them Dhamma for their benefit and happiness. When they had listened to
the Dhamma they could ponder over it and put it into practice. The
Buddha taught about the ill effects of anger. Anger leads to different
kinds of suffering for the person who is angry, but the person to whom
anger is directed does not have to suffer from it if he does not have
anger himself. We read in the Gradual Sayings (Book of Sevens, Chapter
VI, {\S}10. ) about the effects of anger. The person who is angry looks
ugly. Even though he bathes himself, anoints himself, trims his hair
and beard and dresses himself in clean, white linen, for all that he is
ugly, since he is overwhelmed by anger. When someone is angry his face
is tense, and sometimes his mouth may be distorted and his speech
blurred. He may lie on a couch spread with a fleecy cover, a white
blanket, a woollen coverlet, flower{}-embroidered, with crimson
cushions, but for all that, if he is overwhelmed by anger, he lies in
discomfort. He may know what is good and what is evil, but when he is
overwhelmed by anger he does what is harmful, not what is beneficial.
When one performs unwholesome deeds through body, speech and mind, one
will have as result an unhappy rebirth in lower planes, such as a hell
plane or the animal world, depending on the kamma which produces
rebirth. 

We read in the Gradual Sayings (Book of the Sevens, Chapter VI, {\S}10.
Anger, that the Buddha said:


\bigskip

How ugly is an angry man! His sleep

Is comfortless; with fortune in his hands 

He suffers loss; and being full of wrath

He wounds by act and bitter word. Overwhelmed

By rage, his wealth he wastes away. Made mad

And crazy by his bile, his name{\textquotesingle}s bemired. 

With odium, shunned and forsaken is

An angry man by friend and relative. 

By wrath is loss incurred; by wrath, the mind

Irate, he knows not that within

Fear is engendered, nor knows the goal. 

When anger{}-bound, man Dhamma cannot see;

When anger conquers man, blind darkness reigns. 

A man in wrath finds pleasure in bad deeds, 

As in good deeds; yet later when his wrath

Is spent, he suffers like one scorched by fire:

As flame atop of smoke, he staggers on, 

When anger spreads, when youth becomes incensed.

No shame, no fear of blame, no reverence

In speech has he whose mind is anger rent;

No island of support he ever finds.

The deeds which bring remorse, far from right states,

These I{\textquotesingle}ll proclaim. Listen how they come about.

A man in anger will his father kill,

In wrath his very mother will he slay,

Arahats and ordinary men alike he will kill. 

By his mother{\textquotesingle}s care man sees the light

Of day, yet common average folk, in wrath, 

Will still destroy that fount of life. 

Self{}-mirrored all these beings are; each one

Loves self most. In wrath the ordinary men 

Kill self, by divers forms distraught: by sword 

Men kill themselves; in madness poison take;

And in some hollow of a mountain glen

They hide, and bind themselves with ropes and die. 

Thus ruin runs in wake of wrath, and they

Who act in wrath, perceive not that their deeds, 

Destroying life, bring death for themselves. 

Thus lurking in the heart is M{\aa}ra{\textquotesingle}s snare

In anger{\textquotesingle}s loathsome form. But root it out 

By insight, zeal, right view, restraint; the wise

Would one by one eradicate each akusala, 

And thus in Dhamma would he train himself:

Be not our minds obscured, but anger freed 

And freed from trouble, greed and envy. 

The well trained, the canker{}-freed. Become,

When anger is stilled, wholly, completely cool. 


\bigskip

Question: I find what I heard about mett{\aa} very beneficial. However,
mett{\aa} does not arise whenever I wish in the situations of daily
life. What should I do in order that mett{\aa} can arise? 

Khun Sujin: When someone takes realities for self he is inclined to
believe that there is a self who can, by following a particular method,
suppress dosa and develop sati and mett{\aa}. However, in reality there
isn{\textquotesingle}t anybody who can have sati and mett{\aa} if there
are no conditions for their arising. Listening to the Dhamma, wisely
considering what one heard, intellectual understanding of the Dhamma
are different moments of kusala. They are accumulated from moment to
moment, and together they make up conditions for the arising of sati
later on which is mindful of one{\textquotesingle}s different cittas.
In this way the disadvantage of dosa and the benefit of mett{\aa} can
be seen. However, if sati does not arise and there are conditions for
dosa, dosa will arise. There is nobody who can have sati and kindness
at will. If sati arises and it can, time and again, be mindful of the
Dhamma which the Buddha explained, there are conditions for the
elimination of anger. If one does not often listen to the Dhamma there
are not many conditions for wise consideration of it and then it is
difficult to subdue dosa. Whereas if one listens a great deal there are
conditions for remembrance and wise consideration of the Dhamma. One
may for example reflect on kamma and its result. People are the owners
of their deeds. There can be wise consideration of akusala kamma which
is motivated by anger, it can be remembered that anger is not helpful
for the attainment of enlightenment. People can reflect on the
development of patience by the Buddha during his lives as a Bodhisatta,
as it is described in the
{\textasciigrave}{\textasciigrave}S\'ilavan{\aa}ga
J{\aa}taka{\textquotesingle}{\textquotesingle} (I, 72), the
{\textasciigrave}{\textasciigrave}Khantiv{\aa}di
J{\aa}taka{\textquotesingle}{\textquotesingle} (III, 313), the
{\textasciigrave}{\textasciigrave}Culladhammap{\aa}la
J{\aa}taka{\textquotesingle}{\textquotesingle} (III, 358), or the
{\textasciigrave}{\textasciigrave}Chaddanta
J{\aa}taka{\textquotesingle}{\textquotesingle} (V, 514). They can apply
what they read in the Buddha{\textquotesingle}s teachings. The Buddha
taught the Dhamma out of compassion to his followers so that they would
carefully consider it and put it into practice. 

I will quote from the
{\textasciigrave}{\textasciigrave}Mah{\aa}{}-mangala
J{\aa}taka{\textquotesingle}{\textquotesingle} (IV, 453) in order that
the meaning of
{\textasciigrave}{\textasciigrave}mangala{\textquotesingle}{\textquotesingle},
auspicious sign or blessing, will be clearer. Everybody desires
blessings, things which are auspicious. Sometimes people search for it,
they believe that they have a mangala if they possess a particular
thing or if they recite particular texts. We should know what a real
mangala is. We read in the
{\textasciigrave}{\textasciigrave}Mah{\aa}{}-mangala
J{\aa}taka{\textquotesingle}{\textquotesingle} that mett{\aa} is a
mangala. When we know that, we will not search for something else. A
true mangala is the citta with mett{\aa}, mett{\aa} through body,
speech and mind. When the citta is kusala, the citta is beautiful, it
is
{\textasciigrave}{\textasciigrave}auspicious{\textquotesingle}{\textquotesingle}.


We read in the {\textasciigrave}{\textasciigrave}Mah{\aa}{}-mangala
J{\aa}taka{\textquotesingle}{\textquotesingle} that people asked the
Bodhisatta, when he was a hermit, what a mangala is which gives
blessings in this world and the next. We read that the Bodhisatta
explained:


\bigskip

Whoso the devas, and all the brahmas,

And reptiles, and all beings, which we see,

Honours forever with a kindly heart, 

The wise call this a mangala. 


\bigskip


\bigskip

Who is humble towards all beings

To men, women and children alike, 

Who to reviling does not answer back, 

His patience the wise call a mangala. 


\bigskip

Who is of clear understanding, in crisis wise,

Nor playmates nor companions does despise,

Nor boasts of birth, wisdom, caste or wealth, 

The wise call this a mangala for his friends. 


\bigskip

Who takes good men and true his friends to be, 

Who trust him, for his tongue from venom free, 

Who never harms a friend, who shares his wealth, 

The wise call this a mangala for his friends. 


\bigskip

Whose wife is friendly and of equal years, 

Devoted, good, and many children bears, 

Faithful, virtuous, and of gentle birth. 

The wise call that a mangala in wives. 


\bigskip

Whose King the mighty Lord of beings is, 

Who has purity of s\'ila, is diligent, 

And says, {\textasciigrave}{\textasciigrave}He is my
friend{\textquotesingle}{\textquotesingle}, and means no guile, 

That the wise call a mangala in Kings. 


\bigskip

Who has confidence, gives food and drink, 

Flowers, garlands and perfumes, 

With heart at peace, and spreading joy around, 

This the wise call a mangala in heavenly planes. 


\bigskip

Whom by good living virtuous sages try

With effort strenuous to purify,

Good men and wise, by tranquil life built up,

The wise call this a mangala among the company of arahats. 


\bigskip

These blessings then, that in the world befall,

Esteemed by all the wise,

Which man is prudent let him follow these,

The omens which are seen, heard or touched are not real. 


\bigskip

Some people believe that when they see something special such as a red
cow there is a mangala, that it brings them luck. Others believe that
when they hear a special sound or words by which good wishes are
conveyed to them, there is a mangala which is heard. Others again
believe that when they touch particular things, such as a white dress
or a white headgear, or when they apply white powder, there is a
mangala by touch. Or when they smell a particular flower, or taste a
special flavour they believe that there is a mangala through the senses
of smell or taste. As we have read in the J{\aa}taka, there is no truth
in such omens experienced through the senses, they are based on
superstition. Mett{\aa} is a real mangala. 

Question: Can one extend mett{\aa} to devas (heavenly beings)?

Khun Sujin: In respect to this, people should carefully consider which
cause brings which effect. In which way do we extend mett{\aa} to
devas? In the human plane mett{\aa} can be developed by d{\aa}na, by
giving other people useful things, or by s\'ila, by abstaining from
harming others, by abstaining from anger and malevolence. As regards
developing mett{\aa} towards devas, the situation is different. Birth
as a deva is produced by kusala kamma and the lifespan of devas is
extremely long. Its length depends on the degree of kusala kamma which
produced birth in that plane. Therefore we cannot extend mett{\aa} to
devas by abstaining from killing them or by abstaining from other kinds
of akusala kamma which could harm them. We can think with appreciation
of their good deeds which conditioned birth as a deva, thus, there can
be {\textasciigrave}{\textasciigrave}anumodhana
d{\aa}na{\textquotesingle}{\textquotesingle}. Or when we do good deeds
we can extend merit to the devas so that they can have anumodhana
d{\aa}na, kusala cittas with appreciation. These are ways of extending
mett{\aa} to devas. 

Question: I do not understand yet how we can extend merit to devas when
we perform d{\aa}na or other kinds of kusala. 

Khun Sujin: When we perform a good deed devas can appreciate such a
deed. However, one should not hope for their protection just by
reciting texts. When we have expectations there is lobha and that is
different from performing kusala and extending merit so that devas can
appreciate one{\textquotesingle}s kusala and also have kusala cittas. 

Question: Thus, we can extend merit to devas?

Khun Sujin: Yes, when we perform kusala we can extend merit to devas.
However, human beings cannot give things such as food to devas, because
devas take a different kind of food, more refined than our food. Devas
have great wealth, they have precious stones such as diamonds and
sapphires, they have valuable jewellery, they have more riches than any
king in the world. This is due to their great merit which caused them
to be born as devas. As a human being one cannot offer them anything,
one can only extend merit to them when one does good deeds. 

Someone may wish to extend mett{\aa} to devas by reciting texts on
mett{\aa}, and he may expect that they will protect him. However, when
he, inspite of this, meets misfortune and trouble, and thus his
expectations about being protected by the devas do not come true, he
will be disappointed and he may blame the devas. Whereas when the citta
has true calm and it is only intent on kusala, there is no expectation
of any result, and thus people will not blame anyone, there will be no
disappointment or unhappiness. 

Question: In Thailand there is the belief that one should pay respect to
guardian spirits and brahmas. Do they really exist and can they assist
us?

Khun Sujin: First of all we should consider whether there is birth in
other planes of existence, such as the deva planes, and whether there
are beings in other planes such as guardian spirits and brahmas. There
is birth in planes other than the human plane, depending on the
appropriate conditions. Birth as a deva is the result of kusala kamma
and this kind of birth is higher than birth as a human being. Birth in
a brahma plane is the result of jh{\aa}na. If samatha has been
developed to the degree of jh{\aa}na and the jh{\aa}nacitta does not
decline but arises shortly before the dying{}-consciousness, it
produces rebirth{}-consciousness in a brahma plane. Thus beings who are
brahmas really exist. 

Some people believe that there are sacred shrines or other objects they
should venerate, but why do they attach importance to such things? We
should remember that everybody is the owner of the deeds he has
performed himself. Kamma conditions people to have different pleasant
or unpleasant experiences in life. We see, hear, smell, taste and
experience through body{}-sense different objects, some pleasant, some
unpleasant. Seeing, hearing, smelling, tasting and the experience of
tangible object are cittas which are results of kamma,
vip{\aa}kacittas. If there were no kammas which have been performed and
which are capable of producing vip{\aa}ka, all those different
experiences could not arise. 

As to the question about the assistance to people given by guardian
spirits and brahmas, each person is
{\textasciigrave}{\textasciigrave}heir{\textquotesingle}{\textquotesingle}
to his own deeds; that means: pleasant and unpleasant experiences
through the senses are produced accordingly by the kamma he performed.
Someone told me about an event which happened. When he was driving his
car with a little boy sitting beside him, his car slipped off the road.
However, the driver of a jeep who was immediately behind him stopped
and could help him to get the car back on the road again, because he
had the right equipment with him. The driver of the car who had this
experience understood that if there had been conditions for akusala
kamma to produce akusala vip{\aa}ka (unpleasant result), he would not
have received help so soon and in that case he would have had to wait
much longer to get his car back on the road. We may receive help from
another person, be he human or non{}-human, but this also depends on
kamma. If there are conditions for akusala kamma to produce result,
neither human being nor non{}-human being can help us. From the example
given above we see that accumulated kusala kamma is like a close friend
who is near and who can give protection and assistance, who can solve
problems in different situations. 


\bigskip


\bigskip

\clearpage
Chapter 8


\bigskip


\bigskip

Cause and result in life


\bigskip


\bigskip

Some people may worship brahmas but they do not know where they are, how
one can be born as a brahma and what life as a brahma is like. We read
in the Kindred Sayings (I, Sag{\aa}th{\aa}{}-vagga, Chapter VI, The
Brahm{\aa} Suttas, {\S}3, Brahmadeva) that people worshipped Brahm{\aa}
already during the Buddha{\textquotesingle}s time. The text states:


\bigskip

Thus have I heard: the Exalted One was once staying at S{\aa}vatth\'i,
in the Jeta Grove, in An{\aa}thapi\`u\`eika{\textquotesingle}s Park. 

Now on that occasion Brahmadeva, son of a certain brahminee, left the
world, going from home into the homeless in the Order of the Exalted
One. And the venerable Brahmadeva, remaining alone and separate,
earnest, ardent, and strenuous, attained ere long to that supreme goal
of the higher life, for the sake of which the clansmen rightly go forth
from home into the homeless; that supreme goal did he by himself, even
in this present life, come to understand and realize. He came to
understand that birth was destroyed, that the holy life was being
lived, that his task was done, that for life as we conceive it, there
was no hereafter. And the venerable Brahmadeva thus became one of the
arahats. 

Now the venerable Brahmadeva rose early one morning, and dressing
himself, took robe and bowl and entered S{\aa}vatth\'i for alms. And
going about S{\aa}vatth\'i, house by house, he came to his
mother{\textquotesingle}s dwelling. 

At that time his mother, the brahminee, was habitually making an
oblation to Brahm{\aa}. Then it occurred to Brahm{\aa} Sahampati:
{\textasciigrave}{\textasciigrave}This mother of the venerable
Brahmadeva, the brahminee, makes her perpetual oblation to Brahm{\aa}.
What if I were now to approach and agitate
her?{\textquotesingle}{\textquotesingle} So as a strong man might
stretch forth his bent arm, or bend his arm stretched forth, Brahm{\aa}
Sahampati vanished from the Brahm{\aa} world and appeared at the
dwelling of the mother of the venerable Brahmadeva. And standing in the
air he addressed her in verses:


\bigskip

Far from here, O brahminee, is Brahma{\textquotesingle}s world, 

To whom you always serves offerings. 

And Brahm{\aa} does not take food like that. 

What babble you unwitting of the way,

O brahminee, unto the Brahma world.

Look at Brahmadeva, your son, 

A man who will never see another world, 

A man who past the gods has won his way, 

An almsman who does nothing call his own, 

Who only maintains himself, 

This man has come into your house for alms, 

Worthy of offerings, versed in the Vedas, 

With faculties developed and controlled. 

It is suitable for devas and men to offer to him

True brahmin, barring all things evil out, 

By evil undefiled, grown calm and cool, 

He moves on his alms round. 

There is no after, no before for him, 

He is at peace, no fume of vice is his;

He is untroubled, rid of hankering;

All force renouncing toward both weak and strong. 

Let him enjoy the choice food you have served. 

By all the hosts of evil unassailed, 

His heart at utter peace, he goes about

Like tamed elephant, with vices purged. 

Almsman most virtuous, and with heart well freed:

Let him enjoy the choice food you have served. 

To him so worthy of the gift do you, 

In confidence unwavering, offer your gift. 

Work merit and your future happiness, 

Now that you see here, O brahminee, 

A sage by whom the flood is overpassed. 

To him so worthy of the gift did she, 

In confidence unwavering offer her gift. 

Merit she wrought, her future happiness, 

When (at her door) the brahminee saw

A sage by whom the flood was overpassed. 


\bigskip

Should one make an offering to an arahat or to Brahm{\aa}? When one has
right understanding one will know that it is better to make an offering
to an arahat who has eradicated all defilements. He has accomplished
the task which has to be done and there is nothing more to be done by
him since all defilements have been completely eradicated. Although the
son of the brahminee had attained arahatship, the brahminee still paid
respect to Brahm{\aa} and she made continuously offerings of food to
him. The brahma planes are far away from the human plane, the distance
to those planes is immeasurable. Brahmas cannot eat food offered by
humans. The brahminee did not know how the brahma world could be
reached, but she offered food to Brahm{\aa} and in her ignorance she
mumbled words to him over and over again. 

The S{\aa}ratthappak{\aa}sin\'i, the Commentary to the Kindred Sayings,
gives an elaboration of the story about the Brahminee. 


\bigskip

When the mother of Brahmadeva had seen her son approaching her house,
she went outside to welcome him. She invited him to come inside and to
sit on a seat she had prepared. It was her custom to offer rice cakes
to Brahm{\aa} and also on that day she performed sacrificial worship.
Her whole house was decorated with fresh green leaves and puffed rice,
with precious stones and flowers. She had put up different kinds of
flags and banners and she had laid out water vessels. She had lighted
candles contained in candle holders which were decorated with garlands
and many fragrant things. People went around in procession. The
brahminee herself had got up very early in the morning. After she had
bathed herself with fragrant water taken from sixteen pots, she put on
beautiful cloths and precious jewellery. She invited her son, the
arahat, to come inside, but she had no intention to offer him even a
ladle of rice. She only wanted to attend to Mah{\aa} Brahm{\aa}, to
make sacrificial worship to him. She filled a golden tray with rice,
prepared with ghee, honey and sugar. She carried the tray to the
backyard which she had decorated with fresh green leaves. She had put a
lump of rice on each of the four corners of the tray and took one lump
at a time in her hand while the ghee was dripping on her arms. She
knelt down on the ground and recited an invitation to Mah{\aa}
Brahm{\aa} to partake of the food. 

In the meantime Brahm{\aa} Sahampati inhaled the fragrance of the s\'ila
of the arahat which rose to all deva planes and was diffused even as
far as the brahma planes. The odours of the human world do not reach
the brahma planes, it is only the fragrance of the excellent qualities
of arahats which can be diffused as far as that. It occurred to
Sahampati that he should admonish the brahminee and explain to her what
would be the right thing for her to do. He said to her:
{\textasciigrave}{\textasciigrave}You have not even given a ladle of
rice to your son after he sat down, although he is most worthy of
offerings. Instead you have only thought of offering food to Mah{\aa}
Brahm{\aa}. The situation is the same as when someone who has scales
for weighing discards them and just uses his hands, or someone who has
a drum does not make use of it but beats on his stomach instead, or
someone who has a fire does not make use of it but uses a firefly
instead. {\textquotesingle}{\textquotesingle}

Sahampati wanted to induce her to change her mind, to offer food to her
son the arahat instead of offering it to Mah{\aa} Brahm{\aa}. 

He said to himself: {\textasciigrave}{\textasciigrave}I will cause her
wrong view to disappear and save her from an unhappy plane. I will
convert her to the Buddha{\textquotesingle}s teachings so that she will
accumulate an immeasurable treasure, namely kusala kamma which will
produce as result rebirth in a heavenly plane.
{\textquotesingle}{\textquotesingle}

The distance from here to the brahma planes is difficult to fathom. If a
stone which has the size of a tall building would travel from the
lowest brahma plane as fast as 48.000 yoyanas (one yoyana being 16 km.)
a day, it would take four months before it would reach the earth. The
lowest brahma plane is as far as that, and the higher planes are still
further away. 

Sahampati said: {\textasciigrave}{\textasciigrave}Very far indeed is the
world of Brahm{\aa}, to whom you, Brahminee, are making the offering of
food. The real way to attain to the world of Brahm{\aa} are the kusala
jh{\aa}nacittas of the four stages of jh{\aa}na. These give as results
the four types of vip{\aa}ka jh{\aa}nacittas which arise in the brahma
planes. You do not know the way to attain the world of Brahm{\aa}, you
are only mumbling some prayers. Those who are in brahma planes keep
alive by jh{\aa}na rapture and not by taking rice or drinking boiled
milk. You should not trouble yourself with things which are not the
real condition for the attainment to the world of
Brahm{\aa}.{\textquotesingle}{\textquotesingle}

When Sahampati had spoken thus and respectfully took leave of the
Brahminee, he pointed to the arahat and spoke again:

{\textasciigrave}{\textasciigrave}Brahminee, your son Brahmadeva has
eradicated all defilements, he is the highest among devas, the highest
among brahmas. He is no more disturbed by defilements. He is an almsman
who has the habit of asking, who does not provide a livelihood for
someone else. The great Brahmadeva who entered your house for alms is
the person who is most worthy to receive an offering of food.
{\textquotesingle}{\textquotesingle}


\bigskip

This is the story related by the Commentary. In order to be reborn as a
brahma in a brahma plane there must be the right condition. One should
develop samatha to the stage of jh{\aa}na, which can be
r\'upa{}-jh{\aa}na or ar\'upa{}-jh{\aa}na. This is the way to
r\'upa{}-brahma planes and ar\'upa{}-brahma planes. The result of
kusala jh{\aa}nacitta is rebirth in a brahma plane where one will live
until the jh{\aa}na kusala kamma has been exhausted and one will pass
away from that plane. Beings in brahma planes do not need to eat and
they do not need to breathe in order to stay alive. Beings in the
r\'upa{}-brahma planes have very subtle r\'upas and these do not have
to be sustained by morsels of food such as is taken by human beings,
and they do not have to experience suffering due to breathing. As
regards beings in the ar\'upa{}-brahma planes, they do not have any
r\'upa. 

As the Commentary states, Brahmadeva had eradicated all defilements, he
was the highest among devas. The arahat who is not disturbed by
defilements does not have to provide a livelihood for someone else.
When one hears this one thinks of the bhikkhus who do not have a family
and who do not have a profession by which they have to provide a
livelihood for others. However, there is a deeper meaning to these
words. The meanings is that for the arahat there are no more conditions
for rebirth, for the arising of khandhas in a future life. So long as
one still has defilements, there will be a new life after this one,
conditioned by these defilements. When at the end of life the
dying{}-consciousness has fallen away there will be rebirth, there will
be n{\aa}ma khandhas and r\'upa khandhas succeeding the khandhas which
are arising and falling away in this life, which we take for
{\textasciigrave}{\textasciigrave}I{\textquotesingle}{\textquotesingle}
or
{\textasciigrave}{\textasciigrave}mine{\textquotesingle}{\textquotesingle},
for {\textasciigrave}{\textasciigrave}my
personality{\textquotesingle}{\textquotesingle}. Our present life
conditions the life of a future being, of someone else, namely the
khandhas arising in the future which are conditioned by the khandhas in
this life. In this sense it is said that we maintain or sustain the
life of someone else. 

The Brahmadeva Sutta can answer the questions about spirits and brahmas
who are venerated in Thailand, questions about whether they exist and
whether they can help us. 

Question: I believe that there is someone who is an avenger, who can
cause us to suffer misfortune. When we do good deeds and then transfer
the merit to this person can that be to our benefit? When one develops
sam{\aa}dhi can one then see such a person?

Khun Sujin: The Buddha taught about cause and effect and we should
carefully consider this. Is it true that there is someone who could
inflict retribution on us and thus control our fate? We read in the
Gradual Sayings (Book of the Tens, Chapter XXI, {\S}6) that the Buddha
taught to the monks about kamma and its result:


\bigskip

Monks, beings are owners of their deeds, heirs to their deeds, they are
the womb of their deeds, their deeds are their relatives, to them their
deeds come home again. Whatsoever deeds they do, be they good or evil,
of these deeds they are the heirs. 


\bigskip

When someone is born as this person into this world, what is the cause?
Is this caused by the kamma he performed himself or by someone else who
controls his fate?

When a person has gain, honour, praise, happiness, or when he has loss,
dishonour, blame and misery, by what are these caused? Are they caused
by someone else who controls his fate or are they results of deeds he
has performed himself?

People believe that someone to whom they in former lives caused
suffering can have power over their fate, that he follows them in this
life and causes them to be ill or to suffer different kinds of
misfortunes. Or if such a person has not caused their misfortune yet,
they believe that they should extend merit to him so that he will not
cause them to suffer. 

Who can remember his former lives and the deeds he performed during
those lives? Who can remember to which being he caused trouble and
suffering in past lives? If someone, for example, has killed another
person and then extends merit to him how could this prevent the killing
which is akusala kamma from producing result? One should know who the
owner is of the kammas which have been performed during the cycle of
birth and death. Akusala kamma such as killing can cause rebirth in a
hell plane. Or if one is born in the human plane akusala kamma can
cause one to be sick or to suffer misfortune. Kusala kamma can cause
rebirth in a happy plane, such as rebirth as a human being, or as a
deva in one of the heavenly planes. Rebirth is in accordance with the
kamma one performed oneself. If there is an unhappy rebirth it is not
due to any revenge of another being. There is no one who could rule
over someone{\textquotesingle}s destiny. 

If people believe in a person who could retaliate, how is their
relationship to such a person? If they think that there is a person who
could take revenge then they themselves could also be someone who takes
revenge on another person. However, when someone has no ill feeling
towards others could he take revenge and cause someone
else{\textquotesingle}s misfortune? We may remember ill deeds in this
life which we have committed to someone else and ill deeds which others
have committed to us, but we do not remember the deeds which were
committed in past lives. We would not be able to remember to whom we
did wrong ourselves, nor would others be able to remember such things.
Thus the belief in someone who could take revenge for the wrongs a
person formerly did to him and who could cause his misfortune in this
life is without foundation. The transfer of merit to such a person is
also useless, it does not have any effect. 

Every being has performed many kammas during countless aeons in the
past. People are born and they must die, they are born again and must
die again, and thus they are now no longer the same person they were in
the past. We should not think of a person in the past who could take
revenge, but instead we should remember that in this life one should
have no anger, no revengeful feeling, no wish to harm or hurt anyone.
People may have aversion or anger or they may even want to hurt someone
else when they think that he in this life or in a former life caused
them misfortune or suffering. However, one should subdue
one{\textquotesingle}s anger and feelings of revenge and not commit any
deed motivated by anger. Instead, one should develop mett{\aa} and make
it increase. 

\clearpage
Chapter 9


\bigskip


\bigskip

Mett{\aa}: the foundation of the world


\bigskip


\bigskip

The Buddha said that beings are owners of their deeds, heirs to their
deeds, that kamma is the womb from which they are born, that their
deeds are their relatives. To them their deeds come home again and
whatsoever deeds they do, be they good or evil, of those deeds they
receive the results. 

Everybody is the owner of his deeds, he possesses the kamma he has
performed. People cannot exchange their kammas. Other kinds of
possessions do not really belong to us, they can be destroyed or
stolen. The kamma we have performed ourselves, be it kusala kamma or
akusala kamma, cannot be stolen or damaged by fire, wind or sun. There
is no possession which can be kept as safely as kamma, because kamma is
accumulated from moment to moment, since cittas arise and fall away in
succession. 

When kamma has been performed it can cause the arising of vip{\aa}ka
(result) for the person who committed it, if there are the right
conditions for kamma to produce result. The person who has performed
kamma will receive its result accordingly, since kamma is the
{\textasciigrave}{\textasciigrave}womb{\textquotesingle}{\textquotesingle},
it can condition rebirth in a happy plane or in an unhappy plane. When
we are born, kamma is a
{\textasciigrave}{\textasciigrave}relative{\textquotesingle}{\textquotesingle}
(kinsman) to us, we are dependant on our kamma. When there are
conditions for akusala kamma to produce its result, then akusala kamma
is our
{\textasciigrave}{\textasciigrave}relative{\textquotesingle}{\textquotesingle}:
there is the arising of unpleasant experiences and misfortunes, of
which the immediate occasion can even be our circle of relatives and
friends, or other people we are acquainted with. When kusala kamma has
the opportunity to produce its result, the opposite happens, and thus
we can say that each person has kamma as his relative, that he is
dependant on his kamma. 

When we experience happiness or misery on account of visible object,
sound, odour, flavour and tangible object, it seems that other people
are the cause of such experiences. When we for example have been hurt
or harmed by others, it seems that other people are the cause of this.
However, could this really happen if there were no akusala kamma we
performed ourselves which produces such result? When akusala kamma has
the opportunity to produce result we will receive its result, even if
there are no people around who could hurt us. We may, for example,
wound ourselves with a knife, we may fall down, we may become ill, we
may suffer from an inundation or a fire. Some people may believe that
there is another person who could avenge himself and cause them to
suffer from sickness and other misfortunes. They extend merit to that
person out of fear of his retaliation. However, all this is a
superstition. 

When we have performed kusala kamma we can extend merit to others who
are able to appreciate our good deed, and this is a form of d{\aa}na,
of generosity. It is beneficial to do this, because at such a moment
the citta is accompanied by mett{\aa}. We think of the wellbeing of
someone else, we give him the opportunity to have kusala citta with
appreciation of our kusala. When somebody has
{\textasciigrave}{\textasciigrave}anumodhana
d{\aa}na{\textquotesingle}{\textquotesingle}, appreciation of another
person{\textquotesingle}s kusala, it is his kusala kamma. We all can
rejoice in each others kusala, by anumodhana d{\aa}na, and in this way
benefit from the good deeds performed by someone else. However, we
should not extend merit out of fear that there is someone who could
avenge himself and cause misfortune. The development of mett{\aa}
towards those we meet in this life is more beneficial than the
extension of merit to an avenger we have never seen and whom we do not
know. 

We read in the Commentary to the Dhammapada (vs. 136) that the Buddha
told the bhikkhus a story of the past, which happened at the time of
Buddha Kassapa. The treasurer Sumangala had a Vih{\aa}ra built for the
Buddha Kassapa. One day when Sumangala was on his way to the Teacher,
he saw a robber, hidden in a rest house at the gate of the city, his
feet spattered with mud, a robe drawn over his head. Sumangala said to
himself: {\textasciigrave}{\textasciigrave}This man must be a
night{}-prowler in hiding.{\textquotesingle}{\textquotesingle} Then
that robber conceived a grudge against Sumangala. He burned his field
seven times, cut off the feet of his cattle seven times and burned his
house seven times. However, he had not satisfied his grudge yet against
the treasurer. When he found out that Sumangala rejoiced most of all in
the Buddha{\textquotesingle}s Perfumed Chamber, he destroyed that by
fire. When Sumangala saw the Perfumed Chamber destroyed by fire he did
not have the slightest grief but he clapped his hands with joy since he
would be once more permitted to built a Perfumed Chamber for the
Buddha. He rebuilt the Perfumed Chamber and presented it to the Buddha
and his retinue of twenty thousand monks. When the robber saw that, he
decided to kill Sumangala. He took a knife and went around the
monastery for seven days. During these days Sumangala made gifts to the
Sangha presided over by the Buddha. He told the Buddha what had
happened and said that he would transfer to that man the first fruits
of the merit of his offering. When the robber heard this he realized
that he had committed a grievous sin towards the treasurer who had no
ill{}-will and even extended merit to him. He asked the treasurer
forgiveness. When the treasurer asked the robber about each particular
deed whether he had committed it, the robber answered him that he had
committed all of them, and he explained the reason. He said that he had
conceived a grudge against the treasurer when he had heard his words
while he was lying down splattered with mud near the city gate.
Sumangala asked him forgiveness for the words he had spoken then. The
robber wanted to become the treasurer{\textquotesingle}s slave and live
in his house, but Sumangala declined that, since he could not be sure
whether the robber would continue to have a grudge against him.
Although Sumangala had forgiven the robber, the akusala kamma the
robber had committed caused him to be reborn in the Av\'ici Hell. After
he had suffered there for a long time he was reborn as a peta (ghost)
on Vultures Peak in the era of this Buddha. 

The treasurer had no feelings of revenge against the robber who had a
grudge against him, but he had mett{\aa} towards him. He extended merit
to the robber who had committed very heavy akusala kamma so that he
would have kusala citta while rejoicing in Sumangala{\textquotesingle}s
good deeds. If Sumangala had been angry with the robber and had
feelings of revenge, he himself could have received the result of his
anger and of the deeds motivated by revenge. 

If one is afraid of revenge one should abstain from the five kinds of
akusala kamma which cause a fivefold guilty dread, namely: killing,
stealing, sexual misconduct, lying and the taking of intoxicants. One
should abstain from these akusala kammas. The Buddha said that people
who only fear those things which should be feared are no fools, whereas
people who only fear what should not be feared are fools. Those who put
the Dhamma into practice should fear the committing of akusala kamma,
they should not be afraid of a person who could take revenge and
control their destiny. 

Someone asked whether one, if one develops sam{\aa}dhi (concentration),
could see an image of the person who wants to take revenge. There are
misunderstandings about the development of sam{\aa}dhi, and therefore I
will explain what it is. There are two kinds of sam{\aa}dhi, namely
samm{\aa}{}-sam{\aa}dhi, right concentration, and
micch{\aa}{}-sam{\aa}dhi, wrong concentration. There is
samm{\aa}{}-sam{\aa}dhi with the development of samatha, tranquil
meditation. This is the development of kusala citta which is
established in wholesome calm so that there is more and more freedom
from lobha, attachment, dosa, aversion, and moha, ignorance. Thus in
samatha there must be kusala citta with sati sampaja\~n\~na,
pa\~n\~n{\aa} arising with sati, which is mindful time and again of the
dhammas (realities) which condition the citta to be free from akusala
and to attain true calm. Calm can be developed with meditation subjects
such as the excellent qualities of the Buddha, the Dhamma and the
Sangha. Or one can think of other people with mett{\aa}, karu\`u{\aa}
(compassion), mudit{\aa} (sympathetic joy) and upekkh{\aa}
(equanimity). One can recollect d{\aa}na (generosity) and s\'ila (good
moral conduct) one has performed, or one can recollect death. When the
citta has advanced in kusala it becomes more established in calm, in
freedom from akusala. Then the characteristic of calm which goes
together with concentration, sam{\aa}dhi, appears more clearly. Calm
can become firmer when sati sampaja\~n\~na performs its function, and
this has nothing to do with the seeing of extraordinary things or
strange experiences. 

When calm has been developed with a meditation subject and calm has
become more established, one can experience an image, nimitta, but this
is not the case with all meditation subjects. The development of the
following meditation subjects is dependant on the experience of a
nimitta: the kasinas (disks), the meditations on corpses, mindfulness
of breath ({\aa}n{\aa}p{\aa}na sati) and mindfulness with regard to the
body (k{\aa}yagat{\aa} sati). 

When someone develops the earth kasina he contemplates earth in order
that the citta becomes calm, free of akusala; he is dependant on an
image in the form of a circle, which can help him to subdue akusala
citta. When someone develops the other kasinas, namely the kasinas of
fire and wind, of the colours of blue, yellow, red and white, of light
and air, the same procedure is followed. 

The meditations on foulness (asubha) are meditations on corpses in
different stages of decay. 

As regards mindfulness of breath ({\aa}n{\aa}p{\aa}na sati), this is
mindfulness of breath where it appears on the tip of the nose or upper
lip. 

K{\aa}yagat{\aa} sati is contemplation of the foulness of the body in
each part, such as hair of the head, hair of the body, nails, teeth,
skin. 

One is dependant on a
{\textasciigrave}{\textasciigrave}nimitta{\textquotesingle}{\textquotesingle},
a mental image, only when one develops calm with the above mentioned
meditation subjects. Of each of these subjects a mental image can
appear. Citta contemplates this image in order to attain a higher
degree of calm. Citta contemplates a specific nimitta in the case of
each of these subjects and it does not
{\textasciigrave}{\textasciigrave}see{\textquotesingle}{\textquotesingle}
other nimittas such as hells, heavens, devas, ghosts or the person one
calls the avenger or controller of one{\textquotesingle}s fate. 

The development of samatha and the development of vipassan{\aa} are
intricate and difficult. For both ways of development pa\~n\~n{\aa} is
needed but pa\~n\~n{\aa} in samatha and pa\~n\~n{\aa} in vipassan{\aa}
are of different levels. Pa\~n\~n{\aa} in samatha can temporarily
subdue defilements but it cannot eradicate them. In vipassan{\aa}
pa\~n\~n{\aa} is developed which knows the reality which appears as it
is, as anatt{\aa}, non{}-self, and this kind of pa\~n\~n{\aa} can
eradicate defilements completely. People should not mistakenly think
that they develop samatha or vipassan{\aa} by way of concentration, by
trying to focus for a long time on one object. 

If someone tries to concentrate with the expectation to see special
things he concentrates with lobha. This is not the development of true
calm which is freedom from lobha, dosa and moha. He does not develop
calm, because there is lobha, not pa\~n\~n{\aa}. There is no sati
sampaja\~n\~na which knows how citta can become calm, free from
defilements. Sati sampaja\~n\~na knows correctly how citta should
contemplate a particular meditation subject in order to attain true
calm. When there is no right understanding of the development of calm
and one concentrates in order to have special experiences or to see
extraordinary things, there is no samm{\aa}{}-sam{\aa}dhi, right
concentration, but micch{\aa}{}-sam{\aa}dhi, wrong concentration. When
there is micch{\aa}{}-sam{\aa}dhi, the citta is akusala, there is citta
with attachment. There is clinging, one wants to concentrate, to focus
for a long time on one object. When there is micch{\aa}{}-sam{\aa}dhi
different mental images may appear because citta thinks of them without
realizing that there is only thinking. It is the same situation as when
people are dreaming, and they do not realize that the images in their
dream appear only because they are thinking of them. When there is
micch{\aa}{}-sam{\aa}dhi and someone sees an image he takes for the
controller of his fate or an avenger, it is only a thought, an
imagination, it is not right understanding which clearly realizes what
is true. The Buddha said that in the cycle of birth and death which is
endlessly long we all were related to each other as family members,
friends, husband and wife, parents and children, or as enemies. Even
Devadatta who tried to kill the Buddha was in a former life his father.
People should not extend merit to a person who could revenge himself
because of a bad deed they did towards him, to a person they do not
even know, since people cannot remember which bad deeds they committed
to one another. Instead, we should from now on develop mett{\aa}
towards each being, each person we meet in this life, in order to
subdue the inclination to commit evil deeds. When people lack mett{\aa}
there will be suffering. The Buddha said that mett{\aa} is the dhamma
which is the foundation of the world, it is kusala dhamma which
supports beings in the world so that they can live free from danger,
free from the sorrow resulting from akusala citta which is without
mett{\aa}.


\bigskip


\bigskip

\clearpage
Selected Texts


\bigskip


\bigskip

Mett{\aa} Sutta: Sutta{}-Nipp{\aa}ta (143{}-152)


\bigskip

This is what is to be done by one who is skilful in respect of the good,
having attained the peaceful state. He should be capable, straight, and
very upright, easy to speak to gentle and not proud, contented and easy
to support, having few duties and of a frugal way of life, with his
sense{}-faculties calmed, zealous, not impudent, (and) not greedy (when
begging) among families.

And he should not do any mean thing, on account of which other wise men
would criticize him. Let all creatures indeed be happy (and) secure;
let them be happy minded.

Whatever living creatures there are, moving or still without exception,
whichever are long or large, or middle{}-sized or short, small or
great, whichever are seen or unseen, whichever live far or near,
whether they already exist or are going to be, let all creatures be
happy minded.

One man should not humiliate another; one should not despise anyone
anywhere. One should not wish another misery because of anger or from
the notion of repugnance.

Just as a mother would protect with her life her own son, her only son,
so one should cultivate an unbounded mind towards all beings, and
loving kindness towards all the world. One should cultivate an
unbounded mind, above and below and across, without obstruction,
without enmity, without rivalry.

Standing, or going, or seated, or lying down, as long as one is free
from drowsiness, one should practise this mindfulness. This, they say,
is the holy state here.

Not subscribing to wrong views, virtuous, endowed with insight, having
overcome greed for sensual pleasures, a creature assuredly does not
come to lie again in a womb.


\bigskip

\clearpage
Sig{\aa}lov{\aa}da Suttanta: D\'igha Nik{\aa}ya, III, 180


\bigskip

Thus have I heard{}-{}-The Exalted One was once staying near
R{\aa}jagaha in the Bamboo Wood at the Squirrels{\textquotesingle}
Feeding ground.

Now at this time young Sig{\aa}la, a householder{\textquotesingle}s son,
rising betimes, went forth from R{\aa}jagaha, and with wet hair and wet
garments and clasped hands uplifted, paid worship to the several
quarters of the earth and sky{}-{}-to the east, south, west, and north,
to the nadir and the zenith.

And the Exalted One, early that morning dressed himself, took bowl and
robe and entered R{\aa}jagaha seeking alms. Now he saw young Sig{\aa}la
worshipping and spoke to him thus{}-{}-

Why, young householder, do you, rising betimes and leaving R{\aa}jagaha,
with wet hair and raiment, worship the several quarters of earth and
sky?

Sir, my father, when he was a{}-dying, said to me: Dear son, you should
worship the quarters of the earth and sky. So I, sir, honouring my
father{\textquotesingle}s word, reverencing, revering, holding it
sacred, rise betimes and, leaving R{\aa}jagaha, worship on this wise.

But in the religion of an Ariyan, young householder, the six quarters
should not be worshipped thus.

How then, sir, in the religion of an Ariyan, should the six quarters be
worshipped?

It would be an excellent thing, sir, if the Exalted One would so teach
me the doctrine according to which, in the religion of an Ariyan, the
six quarters should be worshipped.

Hear then, young householder, give ear to my words and I will speak.

So be it, sir, responded young Sig{\aa}la. And the Exalted One
said{}-{}-

Inasmuch, young householder, as the Ariyan disciple has put away the
four vices in conduct, inasmuch as he does no evil actions from the
four motives, inasmuch as he does not pursue the six channels for
dissipating wealth, he thus, avoiding these fourteen evil things, is a
coverer of the six quarters; he has practised so as to conquer both
worlds; he tastes success both in this world and the next. At the
dissolution of the body, after death he is reborn to a happy destiny
heaven. What are the four vices of conduct that he has put away? The
destruction of life, the taking what is not given, licentiousness, and
lying speech. These are the four vices of conduct that he has put away.

Thus spoke the Exalted One. And when the blessed One had thus spoken,
the master sake again{}-{}-


\bigskip

Slaughter of life, theft, lying, adultery{}-{}-

To these no word of praise the wise award.


\bigskip

By which four motives does he do no evil deed? Evil deeds are done from
motives of partiality, enmity, stupidity and fear. But inasmuch as the
Ariyan disciple is not led away by these motives, he through them does
no evil deed.

Thus sake Exalted One. And when the Blessed One had thus spoken, the
Master sake yet again{}-{}-


\bigskip

Whoso from partiality or hate

Or fear or dullness doth transgress the Norm,

All minished good name and fame become

As in the abbing month the waning moon.

Who ne{\textquotesingle}er from partiality or hate

Or fear or dullness doth transgress the Norm,

Perfect and full good name and fame become.

As in the brighter half the waxing moon.


\bigskip

And which are the six channels for dissipating wealth? The being
addicted to intoxicating liquors, frequenting the streets at unseemly
hours, haunting fairs, the being infatuated by gambling, associating
with evil companions, the habit of idleness.

There are, young householder, these six dangers through being addicted
to intoxicating liquors{}-{}-actual loss of wealth, increase of
quarrels, susceptibility to disease, loss of good character, indecent
exposure, impaired intelligence.

Six, young householder, are the perils from frequenting the streets at
unseemly hours{}-{}-he himself is without guard or protection and so
also are wife and children; so also is his property; he moreover
becomes suspected [as the doer] of [undiscovered] crimes, and false
rumours fix on him, and many are the troubles he goes out to meet.

Six, young householder, are the perils from haunting fairs{}-{}-[he is
ever thinking] where is there dancing? where is there singing? where is
there music? where is recitation? where are the cymbals? where the
tam{}-tams?

Six young householder, are the perils for him who is infatuated with
gambling: as winner he begets hatred; when beaten he mourns his lost
wealth; his actual substance is wasted; his word has no weight in a
court of law; he is despised by friends and officials; he is not sought
after by those who would give or take in marriage, for they would say
that a man who is a gambler cannot afford to keep a wife.

Six, young householder, are the perils of the habit of idleness{}-{}-he
says, it is too cold, and does no work. He says, it is too hot and does
no work; he says, it is too early...too late, and does no work. He
says, I am too hungry and does no work...too full, and does no work.
And while all that he should do remains undone, new wealth he does not
get, and such wealth as he has dwindles away.

Thus sake the Exalted One. And when the Blessed One had thus spoken, the
Master sake again{}-{}-


\bigskip

Some friends are bottle{}-comrades; some are they

Who [to your face] dear friend! dear friend! will say.

Who proves a comrade in your hour of need,

Him may ye rightly call a friend indeed.


\bigskip

Sleeping when sun has risen, adultery,

Entanglement in strife, and doing harm,

Friendship with wicked men, hardness of heart

These causes six to ruin bring a man.

Is he of evil men comrade and friend,

Doth he in evil ways order his life,

Both from this world and from the world to come

To woeful ruin such a man doth fall.


\bigskip

Dicing and women, drink, the dance and song,

Sleeping by day, prowling around at night,

Friendship with wicked men, hardness of heart{}-{}-

These causes six to ruin bring a man.


\bigskip

Playing with dice, drinking strong drink, he goes

To women dear as life to other men,

Following the baser, not th{\textquotesingle}enlightened minds,

He wanes as in the darker half the moon.


\bigskip

The tippler of strong drink, poor destitute,

Athirst while drinking, haunter of the bar,

As stone in water so he sinks in debt;

Swift will he make his folk without a name.


\bigskip

One who by habit in the day doth sleep,

Who looks upon the night as time to arise,

One who is ever wanton, filled with wine,

He is not fit to lead a household life.


\bigskip

Too cold! too hot! too late! such is the cry.

And so past men who shake off work that waits

The opportunities for good pass by..

But he who reckons cold and heat as less

Than straws, doing his duties as a man,

He no wise falls away from happiness.


\bigskip

Four, O young householder, are they who should be reckoned as foes in
the likeness of friends; to wit, a rapacious, person the man of words
not deeds, the flatterer, the fellow{}-waster.

Of these the first is on four grounds to be reckoned as a foe in the
likeness of a friend{}-{}-he is rapacious; he gives little and asks
much; he does his duty out of fear; he pursues his own interests.

On four grounds the man of words, not deeds, is to be reckoned as a foe
in the likeness of a friend{}-{}-he makes friendly profession in the
likeness of a friend{}-{}-he makes friendly profession as regards the
past; he makes friendly profession as regards the future; he tries to
gain your favour by empty sayings; when the opportunity for service has
arisen he avows his disability.

On four grounds the flatterer is to be reckoned as a foe in the likeness
of a friend{}-{}-he both consents to do wrong, and dissents from doing
right; he praises you to your face; he speaks ill of you to others.

On four grounds the fellow{}-waster companion is to be reckoned as a foe
in the likeness of a friend{}-{}-he is your companion when you indulge
in strong drink; he is your companion when you frequent the streets at
untimely hours; he is your companion when you haunt shows and fairs; he
is your companion when you are infatuated with gambling.

Thus sake the Exalted One. And when the Blessed One had thus spoken, the
Master sake yet again{}-{}-


\bigskip

The friend who{\textquotesingle}s ever seeking what to take,

The friend whose words are other than his deeds,

The friend who flatters, pleasing you withal.

The boon companion down the current ways{}-{}-

These four are foes. Thus having recognized,

Let the wise man avoid them from afar

As they were path of peril and of dread.


\bigskip

Four, O young householder, are the friends who should be reckoned as
sound at heart{}-{}-the helper; the friend who is the same in happiness
and adversity; the friend of good council; the friend who sympathizes.

On four grounds the friend who is a helper is to be reckoned as sound at
heart{}-{}-he guards you when you are off your guard, he guards your
property when you are off your guard; he is a refuge to you when you
are afraid; when you have tasks to perform he provides a double supply
[of what you may need].

On four grounds the friend who is the same in happiness and adversity is
to be reckoned as sound of heart{}-{}-he tells you his secrets; he
keeps secret your secrets; in your troubles he does not forsake you; he
lays down even his life for your sake.

On four grounds the friend who declares what you need to do is...sound
at heart{}-{}-he restrains you from doing wrong; he enjoins you to [do
what is] right; he informs you of what you had not heard before; he
reveals to you the way to heaven.

On four grounds the friend who sympathizes is to be reckoned as sound at
heart{}-{}-he does not rejoice over your misfortunes; he rejoices over
your prosperity; he restrains anyone who is speaking ill of you; he
commends anyone who is praising you.

Thus sake the Exalted One. And when the Blessed One had thus spoken, the
Master sake yet again{}-{}-


\bigskip

The friend who is a helpmate, and the friend

Of bright days and of dark, and he who shows

What{\textquotesingle}t is you need, and he who throbs for you

With sympathy{}-{}-these four the wise should know

As friends, and should devote himself to them

As mother to her own, her bosom{\textquotesingle}s child.

Whoso is virtuous and intelligent,

Shines like a fire that blazes [on the hill].

To him amassing wealth, like roving bee

Its honey gathering [and hurting naught],

Riches mount up as ant{}-heap growing high.

When the good layman wealth has so amassed

Able is he to benefit his clan.

In portions four let him divide that wealth.

So binds he to himself life{\textquotesingle}s friendly things.

One portion let him spend and taste the fruit.

His business to conduct let him take two.

And portion four let him reserve and hoard;

So there{\textquotesingle}ll be wherewithal in times of need.


\bigskip

And how, O young householder, does the Ariyan disciple protect the six
quarters? The following should be looked upon as the six
quarters{}-{}-parents as the east, teachers as the south, wife and
children as the west, friends and companions as the north, servants and
work people as the nadir, religious teachers and brahmins as the
zenith.

In five ways a child should minister to his parents as the eastern
quarter{}-{}-Once supported by them I will now be their support; I will
perform duties incumbent on them; I will keep up the lineage and
tradition of my family; I will make myself worthy of my heritage. 

In five ways parents thus ministered to, as the eastern quarter, by
their child, show their love for him{}-{}-they restrain him from vice,
they exhort him to virtue, they train him to a profession, they
contract a suitable marriage for him, and in due time they hand over
his inheritance.

Thus is this eastern quarter protected by him and made safe and secure.

In five ways should pupils minister to their teachers as the southern
quarter: by rising (from their seat, in salutation) by waiting upon
them, by eagerness to learn, by personal service, and by attention when
receiving their teaching.

And in five ways do teachers, thus ministered to as the southern quarter
by their pupils, love their pupil{}-{}-they train him in that wherein
he has been well trained; they make him hold fast that which is well
held; they thoroughly instruct him in the lore of every art; they speak
well of him among his friends and companions. They provide for his
safety in every quarter.

Thus is this quarter protected by him and made safe and secure.

In five ways should a wife as western quarter be ministered to by her
husband{}-{}-by respect, by courtesy, by faithfulness, by handing over
authority to her, by providing her with adornment.

In these five ways does the wife, ministered to by her husband as the
western quarter, love him{}-{}-her duties are well performed, by
hospitality to the kin of both, by faithfulness, by watching over the
goods he brings, and by skill and industry in discharging all her
business.

Thus is this western quarter protected by him and made safe and secure.


\bigskip

In five ways should a clansman minister to his friends and familiars as
the northern quarter{}-{}-by generosity, courtesy and benevolence, by
treating them as he treats himself, and by being as good as his word.

In these five ways thus ministered to as the northern quarter, his
friends and familiars love him{}-{}-they protect him when he is off his
guard, and on such occasions guard his property; they become a refuge
in danger, they do not forsake him in his troubles, and they show
consideration for his family.

Thus is the northern quarter by him protected and made safe and secure.

In five ways does an Ariyan master minister to his servants and
employees as the nadir{}-{}-by assigning them work according to their
strength; by supplying them with food and wages; by tending them in
sickness; by sharing with them unusual delicacies; by granting leave at
times.

In these ways ministered to by their master, servants and employees love
their master in five ways{}-{}-they rise before him, they lie down to
rest after him; they are content with what is given to them; they do
their work well; and they carry about his praise and good fame.

Thus is the nadir by him protected and made safe and secure.

In five ways should the clansman minister to recluses and brahmins as
the zenith{}-{}-by affection in act speech and mind; by keeping open
house to them, by supplying their temporal needs.

Thus ministered to as the zenith, recluses and brahmins show their love
for the clansman in six ways{}-{}-they restrain him from evil, they
exhort him to good, they love him with kindly thoughts; they teach him
what he had not heard, they correct and purify what he has heard, they
reveal to him the way to heaven.

Thus by him is the zenith protected and made safe and secure.

Thus sake the Exalted One. And when the Blessed One had so spoken, the
Master said yet further{}-{}-


\bigskip

Mother and father are the Eastern view,

And teachers are the quarters of the South.

And wife and children are the Western view,

And friends and kin the quarter to the North;

Servants and working folk the nadir are,

And overhead the brahmin and recluse.

These quarters should be worshipped by the man

Who fitly ranks as houseman in his clan


\bigskip

He that is wise, expert in virtue{\textquotesingle}s ways,

Gentle and in this worship eloquent,

Humble and docile, he may honour win.

Active in rising, foe to laziness,

Unshaken in adversities, his life

Flawless, sagacious, he may honour win.

If he have winning ways, and maketh friends,

Makes welcome with kind words and generous heart,

And can give sage councils and advice,

And guide his fellows, he may honour win.


\bigskip

The giving hand, the kindly speech, the life

Of service, impartiality to one

As to another, as the case demands{}-{}-

These be the things that make the world go round

As linchpin serves the rolling of the car.

And if these things be not, no mother reaps

The honour and respect her child should pay,

Nor doth the father win them through the child.

And since the wise rightly appraise these things,

They win to eminence and earn men{\textquotesingle}s praise.


\bigskip

When the Exalted One had thus spoken, Sig{\aa}la the young householder
said this{}-{}-Beautiful, lord, beautiful! As if one should set up
again that which had been overthrown, or reveal that which had been
hidden, or should disclose the road to one that was astray, or should
carry a lamp into darkness, saying They that have eyes will see! Even
so hath the Truth been manifested by the Exalted One in many ways. And
I, even I, do go to him as my refuge, and to the Truth and to the
Order. May the Exalted One receive me as his lay{}-disciple, as one has
taken his refuge in him from this day forth as long as life endures.

\clearpage
Ud{\aa}na: (VIII, viii)


\bigskip

Thus have I heard: On a certain occasion the Exalted One was staying
near S{\aa}vatth\'i in East Park, at the storeyed house of
Mig{\aa}ra{\textquotesingle}s mother.

Now at that time the dear and lovely grand{}-daughter of
Vis{\aa}kh{\aa}, Mig{\aa}ra{\textquotesingle}s mother, had died. So
Vis{\aa}kh{\aa}, Mig{\aa}ra{\textquotesingle}s mother, with clothes and
hair still wet (from washing) came at an unseasonable hour to see the
Exalted One, and on coming to him, saluted him and sat down at one
side. As she sat thus the Exalted One said this to Vis{\aa}kh{\aa},
Mig{\aa}ra{\textquotesingle}s mother:

{\textasciigrave}Why, Vis{\aa}kh{\aa}! How is it that you come here with
clothes and hair still wet at an unseasonable hour? 

{\textasciigrave}O, sir, my dear and lovely grand{}-daughter is dead!
That is why I come here with hair and clothes still wet at an
unseasonable hour.{\textquotesingle}

Vis{\aa}kh{\aa}, would you like to have as many sons and grandsons as
there are men in S{\aa}vatth\'i?

{\textasciigrave}Yes, sir, I would indeed!{\textquotesingle}

{\textasciigrave}But how many men do you suppose die daily in
S{\aa}vatth\'i?{\textquotesingle}

{\textasciigrave}Ten, sir, or maybe nine, or eight. Maybe seven, six,
five or four, three, two; maybe one a day dies in S{\aa}vatth\'i, sir.
S{\aa}vatth\'i is never free from men dying, sir.{\textquotesingle}

{\textasciigrave}What think you, Vis{\aa}kh{\aa}? In such case would you
ever be without wet hair and clothes?

{\textasciigrave}Surely not, sir! Enough for me, sir, of so many sons
and grandsons!

{\textasciigrave}Vis{\aa}kh{\aa}, whoso have a hundred things beloved,
they have a hundred sorrows. Whoso have ninety, eighty ... thirty,
twenty things beloved ... whoso have ten ... whoso have but one thing
beloved, have but one sorrow. Sorrowless are they and passionless.
Serene are they, I declare.{\textquotesingle}


\bigskip


\bigskip


\bigskip


\bigskip


\bigskip


\bigskip


\bigskip


\bigskip


\bigskip

All griefs or lamentations whatso{\textquotesingle}er

And divers forms of sorrow in the world, 

Because of what is dear do these become.

Thing dear not being, these do not become.

Happy are they therefore and free from grief

To whom is naught at all dear in the world.

Wherefore aspiring for the griefless, sorrowless,

Make thou in all the world naught dear to thee.

\clearpage
Itivuttaka: (III, IV,vii)


\bigskip

{\textasciigrave}Monks, these three unprofitable ways of thinking cause
blindness, loss of sight, ignorance, put an end to insight, are
associated with trouble and conduce not to nibb{\aa}na. What three ways
of thinking?

Thinking about lust ... about ill{}-will ... about harming ... causes
blindness, loss of sight ... conduces not to nibb{\aa}na. These are the
three.

{\textasciigrave}Monks, these three profitable ways of thinking cause
not blindness, but cause sight, knowledge, increase insight, are on the
side of freedom from trouble and conduce to nibb{\aa}na. What three?

Thinking about renunciation ... goodwill ... harmlessness conduce to
nibb{\aa}na. These three profitable ways of thinking ... conduce to
nibb{\aa}na.{\textquotesingle}


\bigskip


\bigskip


\bigskip


\bigskip

Three profitable ways of thought should one pursue,

And three unprofitable ways should put away,

He surely doth control a train of thought sustained, 

As a rain{}-shower lays accumulated dust,

He surely with a mind that lays its thought to rest,

In this same life (on earth) hath reached 

 the place of peace.


\bigskip

\clearpage
Glossary


\bigskip


\bigskip


\bigskip

adosa, non aversion

ahosi kamma, kamma which is ineffectual

akusala, unwholesome, unskilful

akusala kamma, a bad deed

akusala citta, unwholesome consciousness

an{\aa}g{\aa}m\'i, person who has reached the third stage of
enlightenment, he has no aversion (dosa)

{\aa}n{\aa}p{\aa}na sati, mindfulness on breath

anatt{\aa}, not self

anicca sa\~n\~n{\aa}, perception of impermanence

anumodhan{\aa}, thanksgiving, appreciation of someone
else{\textquotesingle}s kusala

appan{\aa}{}-sam{\aa}dhi, attainment{}-concentration

arahat, noble person who has attained the fourth and last stage of
enlightenment

ariyan, noble person who has attained enlightenment

ar\'upa{}-brahma plane, plane of existence attained as a result of
ar\'upa{}-jh{\aa}na. There are no sense impressions, no r\'upa
experienced in this realm.

asubha, meditations on foulness

asura, demon

Atthas{\aa}lin\'i, The Expositor, a commentary to the first book of the
Abhidhamma Pi\`iaka

balas, powers, strengths

bh{\aa}van{\aa}, mental development, comprising the development of calm
and the development of insight

bhikkhu, monk

bhikkun\'i, nun

bodhisatta, a being destined to become a Buddha

brahma, heavenly being born in the Brahm{\aa} world, as a result of the
attainment of jh{\aa}na

brahm{\aa}{}-vih{\aa}ras, the four divine abidings: loving kindness,
compassion, sympathetic joy, equanimity

Buddha, a person who becomes fully enlightened without the aid of a
teacher

Buddhaghosa, the greatest of Commentators on the Tipi\`iaka, author of
the Visuddhimagga in 5 A.D

cetasika, mental factor arising with consciousness

cetovimutti, {\textasciigrave}{\textasciigrave}deliverance of
heart{\textquotesingle}{\textquotesingle}

chanda, {\textasciigrave}{\textasciigrave}wish to
do{\textquotesingle}{\textquotesingle}

citta, consciousness, the reality which knows or cognizes an object

d{\aa}na, generosity, giving

deva, heavenly being

dhamma, the teachings, the law, reality, truth 

Dhammasanga\`ui, the first book of the Abhidhamma Pi\`iaka

di\`i\`ihi, wrong view, distorted view of realities

dosa, aversion 

dukkha, suffering, unsatisfactoriness of conditioned realities

indriya, controlling faculty, the five {\textasciigrave}spiritual
faculties{\textquotesingle} which should be cultivated, namely:
confidence, energy, awareness, concentration and wisdom.

J{\aa}takas, birth stories about the Buddha{\textquotesingle}s former
lives

jh{\aa}na, absorption which can be attained through the development of
calm

k{\aa}yagat{\aa} sati, mindfulness of the body

kamma, intention or volition; deed motivated by volition

kamma patha, course of action performed through body, speech or mind

kappa, an endlessly long period of time

karu\`u{\aa}, compassion

kasina, disk, used as an object for the development of calm

k{\aa}ya, body. It can also stand for the {\textasciigrave}mental
body{\textquotesingle}, the cetasikas

khandhas, physical and mental phenomena of life, classified as five
groups 

khanti, patience

kusala, wholesome, skilful

kusala kamma, a good deed

kusala citta, wholesome consciousness

lobha, attachment, greed

mangala, auspicious sign or blessing

m{\aa}ra, the evil one

mett{\aa}, loving kindness

mett{\aa} citta, consciousness accompanied by loving kindness

micch{\aa}{}-sam{\aa}dhi, wrong concentration 

moha, ignorance

mudit{\aa}, sympathetic joy

n{\aa}ma, mental phenomena

nibb{\aa}na, unconditioned reality, the reality which does not arise and
fall away. The destruction of lust, hatred and delusion. The deathless.
The end of suffering.

nimitta, counter{}-image in tranquil meditation

P{\aa}li, the language of the Buddhist teachings

pa\~n\~n{\aa}, wisdom

patisanth{\aa}ro, courtesy

peta, ghost

r\'upa, physical phenomena, realities which do not experience anything

r\'upa{}-brahma plane, heavenly realm of existence attained as a result
of r\'upa{}-jh{\aa}na

sakad{\aa}g{\aa}m\'i, a person who has attained the second stage of
enlightenment and will not be reborn more than once

s{\aa}khalya\"y, amity

samatha, the development of calm

samm{\aa}, right

samm{\aa}{}-di\`i\`ihi, right understanding

samm{\aa}{}-sam{\aa}dhi, right concentration

samm{\aa}{}-sati, right mindfulness

sampaja\~n\~na, discrimination, comprehension

sa\'ovara s\'ila, moral restraint

sangha, community of monks and nuns. As one of the Triple Gems it means
the community of those people who have attained enlightenment.

sa\~n\~n{\aa}, memory, remembrance

sa\`okh{\aa}ra{}-kkhandha, all cetasikas (mental factors) except feeling
and memory

S{\aa}riputta, chief disciple of Buddha

sati, awareness, non{}-forgetfulness, awareness of reality by direct
experience

sati{}-sampaja\~n\~na, clear comprehension

satipa\`i\`ih{\aa}na,development of direct understanding of realities,
or, the applications of mindfulness: body, feeling etc.{}-which are the
objects of right understanding

s\'ila, morality, virtue

sobhana, beautiful

sot{\aa}panna, person who has attained the first stage of enlightenment,
and who has eradicated wrong view of realities

sutta, part of the scriptures containing dialogues at different places
on different occasions

Therav{\aa}da Buddhism, {\textasciigrave}Doctrine of the
Elders{\textquotesingle}, the oldest tradition of Buddhism

Tipi\`iaka, the teachings of the Buddha, consisting of three parts:
Vinaya, Suttanta, Abhidhamma

upekkh{\aa}, indifferent feeling

vi\~n\~n{\aa}\`ua{}-kkhandha, all cittas (consciousness)

vip{\aa}ka, result (of kamma) e.g. rebirth and during life, the
experience of pleasant and unpleasant objects through the senses, such
as seeing, hearing, etc.

vipassan{\aa}, the development of insight

Visuddhimagga, an encyclopaedia of the Buddha{\textquotesingle}s
teaching, written by Buddhaghosa in the fifth century A.D

yakkha, non{}-human being

Other Publications


\bigskip

The Buddha{\textquotesingle}s Path By Nina van Gorkom

Explains the basic principles of Buddhism to those who have no previous
experience and knowledge of this way of life. The four noble Truths {}-
suffering {}- the origin of suffering {}- the cessation of suffering
{}- and the way leading to the end of suffering {}- are explained as a
philosophy and a practical guidance which can be followed in
today{\textquotesingle}s world. The contents include: the
Buddha{\textquotesingle}s life, the truth of suffering, the truth of
non{}-self, the mind, deeds and their results, good deeds and a
wholesome life, meditation and the Eightfold Path. 1994, paperback, 150
pages, 140mm x 210mm, ISBN 1 897633 12 2, price {\pounds}7.95. 


\bigskip

Abhidhamma in Daily Life By Nina van Gorkom

This unique book takes the reader straight to the higher doctrine of
Therav{\aa}da Buddhism {}- the Abhidhamma. It cuts through the
complexities of the original texts enabling the reader to obtain a
clearer grasp of the theory and practice of the teachings. Many
P{\aa}li terms are used in order to bring about a precise understanding
of the different realities of our daily life. Suitable for the serious
beginner to Buddhism or for advanced students. 

1992, 284pp, paperback, 130mm x 190mm, ISBN 1 897633 01 7, price
{\pounds}6.95.


\bigskip

Buddhism in Daily Life By Nina van Gorkom

A general introduction to the main ideas of Therav{\aa}da Buddhism. With
its many quotes from the P{\aa}li texts, it shows the practical
application of the teachings to daily life. The Eightfold Path, the
Four Noble Truths, the development of calm and the development of
insight are all discussed. Suitable for beginners and experts alike.
1992, 175pp, paperback, 130mm x 190mm, ISBN 1 897633 02 5, price
{\pounds}5.95


\bigskip

The World in the Buddhist Sense By Nina van Gorkom

Explains the realities in and around ourselves. Analyses the difference
between the development of calm and the development of insight.
Discusses the meditation practice of
{\textasciigrave}{\textasciigrave}mindfulness of
breathing{\textquotesingle}{\textquotesingle}. Illustrates with many
quotes from the P{\aa}li Tipi\`iaka. Suitable for those who have a
background of Buddhism but who seek a deeper understanding. October
1993, 123 pp, paperback, 210mm x 140mm. ISBN 1 897633 11 4, price
{\pounds}7.95.


\bigskip

Cetasikas{}-{}-under preparation, to be published January 1996.
\end{document}
