
\documentclass{book} 

\usepackage[T1]{fontenc}

\usepackage[sc]{mathpazo}
\linespread{1.05}
\usepackage[centering,paperwidth=6in,paperheight=9in]{geometry}
\raggedbottom
\widowpenalty=1000
\clubpenalty=1000

\title{The Buddhist Teaching\\on\\Physical Phenomena}
\author{Nina van Gorkom}
\date{2008}
\begin{document}
\maketitle
\vspace*{10pt}
\noindent First edition published in 2008 by\\
Zolag\\
32 Woodnook Road\\
Streatham\\
London\\
SW16 6TZ\\
www.zolag.co.uk\\
\vspace{10pt}

\noindent ISBN 978-1-897633-25-0\\ 
Copyright Nina van Gorkom\\
All rights reserved\\
British Library Cataloguing in Publication Data\\
A CIP record for this book is available from the British Library\\
Printed in the UK and USA by
Lightningsource.
\frontmatter
\tableofcontents
\chapter{Preface}



\begin{quote}\begin{raggedright}
That which is made of iron, wood or hemp is not a strong bond, say the
wise; (but) that longing for jewels, ornaments, children and wives is
far greater an attachment.

Dhammapada
(vs. 345).
\end{raggedright}\end{quote}




Attachment to people and possessions is strong, almost irresist\-ible. We
are infatuated by what we see, hear, smell, taste, experience through
the bodysense and through the mind. However, all the different things
we experience do not last. We lose people who are dear to us and we
lose our possessions. We can find out that attachment leads to sorrow,
but at the moments of attachment we do not want to accept the truth of
the impermanence of all things. We want pleasant objects for ourselves,
and we consider the ``self'' the most important thing in the world. 

Through the Buddhist teachings we learn that what we take for ``self'',
for ``our mind'' and for ``our body'', consists of changing phenomena.
That part of the Buddhist teachings which is the ``Abhi\-dhamma''
enumerates and classifies all phenomena of our life: mental phenomena
or n{\=a}ma and physical phenomena or r\=upa. Seeing is n{\=a}ma, it
experiences visible object through the eye-door. Visible object or
colour is r\=upa, it does not experience anything. The eyesense, that
functions as the eye-door through which visible object is
experienced, is also r\=upa. The r\=upas that are sense objects,
namely, visible object, sound, smell, flavour and tangible object, and
also the r\=upas that are the sense organs of eyes, ears, nose, tongue
and bodysense, are conditions for the n{\=a}mas to experience objects.
N{\=a}ma and r\=upa are interrelated. 

N{\=a}ma and r\=upa are ultimate realities. We should know the
difference between ultimate truth and conventional truth. Conventional
truth is the world of concepts such as person, tree or animal. Before
we learnt about Buddhism, conventional truth, the world of concepts,
was the only truth we knew. It is useful to examine the meaning of
concept, in P{\=a}li: pa\~n\~natti. The word concept can stand for the
name or term that conveys an idea and it can also stand for the idea
itself conveyed by a term. Thus, the name ``tree'' is a concept, and
also the idea we form up of ``tree'' is a concept. When we touch what
we call a tree, hardness, which is a kind of r\=upa can be experienced.
Through the eyes only the r\=upa that is visible object or colour can
be experienced. Visible object and hardness are ultimate realities,
paramattha dhammas, each with their own characteristic. These
characteristics do not change, they can be experienced without having
to name them. Colour is always colour, hardness is always hardness,
even when we give them another name. 

The whole day we touch things such as a fork, a plate or a chair. We
believe that we know instantaneously what different things are, but
after the sense-impressions such as seeing or the experiencing of
tactile object through the bodysense, there are complicated processes
of remembrance of former experiences and of classification, and these
moments succeed one another very rapidly. Concepts are conceived
through remembrance. We remember the form and shape of things, we know
what different things are and what they are used for. We could not lead
our daily life without conventional truth; we do not have to avoid
the world of conventional truth. However, in between the moments of
thinking of concepts, understanding of ultimate realities, of n{\=a}ma
and r\=upa, can be developed. The development of understanding does not
prevent us from doing all the chores of daily life, from talking to
other people, from helping them or from being generous to them. We
could not perform deeds of generosity if we would not think of
conventional truth, such as the things we are giving or the person
to whom we give. But through the development of understanding we shall
learn to distinguish between absolute truth and conventional truth. 

The ``Abhidhammattha Sangaha'', a compendium of the Abhidhamma composed
in India at a later time\footnote{This work has
been ascribed to Anuruddha. It has been translated and published into English by the
P.T.S. under the title of ``Compendium of Philosophy'', and by Ven.
N{\=a}rada, Colombo, under the title of ``A Manual of the Abhidhamma''. It
has also been translated by the Venerable Bikkhu Bodhi as ``A
Comprehensive Manual of Abhidhamma''. Moreover, it has been translated
together with its commentary as ``Summary of the Topics of Abhidhamma''
and ``Exposition of the Topics of Abhidhamma'', by R.P. Wijeratne and
Rupert Gethin.}, states that concepts are only shadows of realities.
When we watch T.V., we see projected images of people and we know that
through the eyesense only visible object is seen, no people. Also when
we look at the persons we meet, only colour is experienced through the
eyesense. In the ultimate sense there are no people. Although they seem
very real, they are only shadows of what is really there. The truth is
different from what we always assumed. What we take for a person are
only n{\=a}mas and r\=upas that arise and fall away. So long as we
have not realized the arising and falling away of n{\=a}ma and
r\=upa we continue to believe in a lasting self. 

Ultimate realities are impermanent, they arise and fall away. Concepts
are objects of thinking, they are not real in the ultimate sense.
N{\=a}ma and r\=upa, not concepts, are the objects of understanding.
The purpose of the development of the eightfold Path is seeing ultimate realities as impermanent, suffering and non-self. If the difference between concepts and ultimate realities
is not known, the eightfold Path cannot be developed. Right
understanding, the leading factor of the eightfold Path, is developed
through direct awareness of n{\=a}ma and r\=upa. However, this is
difficult and it can only be learnt very gradually. When direct
awareness arises of one object at a time as it appears through one of
the senses or through the mind-door, we do not think of a concept of
a ``whole'', of a person or thing, at that moment. The study of r\=upas
can help us to have more understanding of the sense objects and of the
doorways of the senses through which these objects are experienced. If
we do not have a foundation knowledge of objects and doorways we cannot
know how to be aware of one reality at a time as it appears at the
present moment. The study of n{\=a}ma and r\=upa can be a condition for
the arising of direct awareness later on. 

The study of r\=upas is not the study of physics or medical science. The
aim of the understanding of n{\=a}ma and r\=upa is the eradication of
the wrong view of self and freedom from enslavement to defilements. So
long as one clings to an idea of self who owns things, it can give rise
to avarice and jealousy which may even motivate bad deeds such as
stealing or killing. Defilements cannot be eradicated immediately, but
when we begin to understand that our life is only one moment of
experiencing an object through one of the six doorways, the clinging to
the idea of an abiding ego, of a person or self will decrease. 

All three parts of the Buddha's teachings, namely the Vinaya (Book of
Discipline for the monks), the Suttanta (Discourses) and the Abhidhamma
point to the same goal: the eradication of defilements. From my
quotations of sutta texts the reader can see that there is also
Abhidhamma in the suttas, thus, that the teachings are one, the
teaching of the Buddha. I have added questions at the end of each
chapter in order to encourage the reader to check his understanding. I
have used P{\=a}li terms next to the English equivalents in order to
help the reader to know the precise meaning of the realities explained
in the Abhidhamma. The English terms have a specific meaning in the
context of conventional use and they do not render the precise meaning
of the reality represented by the P{\=a}li term. The texts from which I
have quoted, including the scriptures and the commentaries, have been
translated into English by the P{\=a}li Text Society.


The first of the seven books of the Abhi\-dhamma, the ``Dham\-ma\-sanga\d ni'',
translated as ``Buddhist Psychological Ethics''\footnote{Pali Text
Society, reprint 1993.}, is a compilation of all n{\=a}ma and r\=upa,
of all that is real. The source for my book on physical phenomena is
that part of the ``Dhamma\-sanga\d ni'' which deals with this subject, as
well as the commentary to this book, the ``Atthas{\=a}lin\=\i'',
translated as ``Expositor''\footnote{Pali Text Society, 1976.},
written by the venerable Buddhaghosa. I also used the
``Visuddhimagga'', translated as ``The Path of Purification'', an
encyclopedia by the venerable Buddhaghosa\footnote{I used the
translation of Ven. Ny{\=a}\d namoli , 1964, Colombo, Sri Lanka. There
is another translation by Pe Maung Tin under the title of ``The Path of
Purity'', P.T.S.}. 

I wish to express my deepest gratefulness to Ms Sujin Boriharnwanaket who always inspired me to verify the teachings in daily life. She reminded me never to forget the goal of the teachings: the development of understanding of all that appears through the sense-doors and the mind-door so that clinging to an idea of self and all defilements can be eradicated.

May this book on r\=upas help the reader to develop right understanding
of n{\=a}ma and r\=upa! 

\chapter{Introduction}  






The Abhidhamma teaches us that in the ultimate sense our life is
n{\=a}ma and r\=upa that arise because of their appropriate conditions
and then fall away. What we take for person or self is citta\footnote{Pronounced as chitta.} or consciousness, cetasika\footnote{Pronounced as chetasika.} or mental factors arising with the citta, and r\=upa or physical
phenomena. Citta and cetasika are n{\=a}ma, they experience objects,
whereas r\=upa does not know anything. Citta experiences sense objects
through the five senses. The sense objects as well as the sense organs
are r\=upas. The five senses by means of which cittas experience an
object are called doors. When we think of something we saw or heard,
citta does not experience an object through a sense-door but through
another door: the mind-door. Thus there are six doorways. Through the
mind-door citta can experience ultimate realities, n{\=a}ma and
r\=upa, as well as concepts.

Citta experiences only one object and then it falls away to be succeeded
by the next citta. We may have thought that there is one consciousness
that lasts, that can see, hear and think, but this is not so. Only one
citta arises at a time: at one moment a citta that sees arises, at
another moment a citta that hears and at another moment again a citta
that thinks. In our life an unbroken series of cittas arise in
succession.

Cittas can be good or wholesome, kusala cittas, they can be unwholesome,
akusala cittas, or they can be neither kusala nor akusala. Seeing, for
example, is neither kusala nor akusala, it only experiences visible
object through the eye-door. After seeing has fallen away, visible
object is experienced by kusala cittas or by akusala cittas. Thus, when
an object impinges on one of the six doors different types of cittas
arise in a series or process and all of them experience that object.
They arise in a specific order within the process and there is no self
who can prevent their arising. The cittas that arise in a process
experience an object through one of the five sense-doors and through
the mind-door. 

Only one citta arises at a time, but each citta is accompanied by
several cetasikas or mental factors that share the same object with the
citta but perform each their own function. Some cetasikas such as
feeling and remembrance or ``perception'' (sa\~n\~n{\=a}) accompany
each citta, others do not. Unwholesome mental factors, akusala
cetasikas, only accompany akusala cittas, whereas ``beautiful'' mental
factors, sobhana cetasikas, accompany kusala cittas. 




As regards physical phenomena or r\=upa, there are twentyeight kinds of
r\=upa in all. R\=upas are not merely textbook terms, they are
realities that can be directly experienced. R\=upas do not know or
experience anything; they can be known by n{\=a}ma. R\=upa arises and
falls away, but it does not fall away as quickly as n{\=a}ma. When a
characteristic of r\=upa such as hardness impinges on the bodysense it
can be experienced through the bodysense by several cittas arising in
succession within a process. But even though r\=upa lasts longer than
citta, it falls away again, it is impermanent. 

R\=upas do not arise singly, they arise in units or groups. What we take
for our body is composed of many groups or units, consisting each of
different kinds of r\=upa, and the r\=upas in such a group arise
together and fall away together. The reader will come across four
conditioning factors that produce r\=upas of the body: kamma, citta,
temperature and food. The last three factors are easier to understand,
but the first factor, kamma, is harder to understand since kamma is a
factor of the past. We can perform good and bad deeds through body,
speech and mind and these can produce their appropriate results later
on. Such deeds are called kamma, but when we are more precise kamma is
actually the cetasika volition or intention (cetan{\=a}) that motivates
the deed. Kamma is a mental activity which can be accumulated. Since
cittas that arise and fall away succeed one another in an unbroken
series, the force of kamma is carried on from one moment of citta to
the next moment of citta, from one life to the next life. In this way
kamma is capable to produce its result later on. A good deed, kusala
kamma, can produce a pleasant result, and an evil deed can produce an
unpleasant result. Kamma produces result at the first moment of life:
it produces rebirth-consciousness in a happy plane of existence such
as the human plane or a heavenly plane, or in an unhappy plane of
existence such as a hell plane or the animal world. Throughout our life
kamma produces seeing, hearing and the other sense-impressions that
are vip{\=a}kacittas, cittas that are results. Vip{\=a}kacittas are
neither kusala cittas nor akusala cittas. Seeing a pleasant object is
the result of kusala kamma and seeing an unpleasant object is the
result of akusala kamma. Due to kamma gain and loss, praise and blame
alternate in our life. 

Rebirth-consciousness is the mental result of kamma, vip\-\=akacitta,
but at that moment kamma also produces r\=upas and kamma keeps on
producing r\=upas throughout life; when it stops producing r\=upas our
life-span has to end. Kamma produces particular kinds of r\=upas such
as the senses, as we shall see. Citta also produces r\=upas. Our
different moods become evident by our facial expressions and then it is
clear that citta produces r\=upas. Temperature, which is actually the
element of heat, also produces r\=upas. Throughout life the element
of heat produces r\=upas. Nutrition is another factor that produces
r\=upas. When food has been taken by a living being it is assimilated
into the body and then nutrition can produce r\=upas. Some of the
groups of r\=upas of our body are produced by kamma, some by citta,
some by temperature and some by nutrition. The four factors which
produce the r\=upas of our body support and consolidate each other and
keep this shortlived body going. If we see the intricate way in which
different factors condition the r\=upas of our body we shall be less
inclined to think that the body belongs to a self.

There are not only r\=upas of the body, there are also r\=upas which are
the material phenomena outside the body. What we take for rocks, plants
or houses are r\=upas and these originate from temperature. We may
wonder whether there are no other factors apart from the element of
heat that contribute to the growth of plants, such as soil, light and
moisture. It is true that these factors are the right conditions that
have to be present so that a plant can grow. But what we call soil,
light and moisture are, when we are more precise, different
combinations of r\=upas, none of which can arise without the element of
heat or temperature that produces them. R\=upas outside the body are
only produced by temperature, not by kamma, citta or nutrition.

R\=upas perform their functions, no matter one dresses oneself, eats,
digests one's food, moves about, gesiticulates, talks to others, in
short, during all one's activities. If we do not study r\=upas we may
not notice their characteristics that appear all the time in daily
life. We shall continue to be deluded by the outward appearance of
things instead of knowing realities as they are. We should remember
that the r\=upa which is the ``earth- element'' or solidity can
appear as hardness or softness. Hardness impinges time and again on the
bodysense, no matter what we are doing. When hardness appears it can be
known as only a kind of r\=upa, be it hardness of the body or hardness
of an external object. In the ultimate sense it is only a kind of
r\=upa. The detailed study of n{\=a}ma and r\=upa will help us to see
that there isn't anything that is ``mine'' or self. The goal of the
study of the Abhidhamma is the development of wisdom leading to the
eradication of all defilements.






\mainmatter 
\chapter[The Four Great Elements]{}  
\section*{The Four Great Elements}

R\=upas do not arise singly, they arise in units or groups. Each of
these groups is composed of different kinds of r\=upa. There are four
kinds of r\=upa, the four ``Great Elements'' (Mah{\=a}-bh\=uta
r\=upas), which have to arise together with each and every group of
r\=upas, no matter whether these are r\=upas of the body or r\=upas
outside the body. The types of r\=upa other than the four Great
Elements depend on these four r\=upas and cannot arise without them.
They are the following r\=upas:




\begin{description}
\item the Element of Earth or solidity

\item the Element of Water or cohesion

\item the Element of Fire or heat

\item the Element of Wind (air) or motion
\end{description}




Earth, Water, Fire and Wind do not in this context have the same meaning
as in conventional language, neither do they represent conceptual ideas
as we find them in different philosophical systems. In the Abhidhamma
they represent ultimate realities, specific r\=upas each with their own
characteristic. The element of earth, in P{\=a}li: pa\d thav\=\i{} 
dh{\=a}tu, translated into English as ``solidity'' or ``extension'',
has the characteristic of hardness or softness. It can be directly
experienced when we touch something hard or soft. We do not have to
name the r\=upa denoted by ``element of earth'' in order to
experience it. It is an element that arises and falls away; it has no
abiding substance, it is devoid of a ``self''. It may seem that
hardness can last for some time, but in reality it falls away
immediately. R\=upas are replaced so long as there are conditions for
them to be produced by one of the four factors of kamma, citta,
temperature or nutrition\footnote{See Introduction. This will be
explained further on.}. The hardness that is experienced now is already
different from the hardness that arose a moment ago.

We used to think that a cushion or a chair could be experienced through
touch. When we are more precise, it is hardness or softness that can be
experienced through touch. Because of remembrance of former experiences
we can think of a cushion or chair and we know that they are named
``cushion'' or ``chair''. This example can remind us that there is a
difference between ultimate realities and concepts we can think of but
which are not real in the ultimate sense. 

Viewing the body and the things around us as different combinations of
r\=upas may be new to us. Gradually we shall realize that r\=upas are
not abstract categories, but that they are realities appearing in daily
life. I shall quote the definitions of the different r\=upas given by
the ``Visuddhimagga'' and the ``Atthas{\=a}lin\=\i''. These definitions
mention the characteristic, function, manifestation and proximate cause
or immediate occasion\footnote{The Atthas{\=a}lin\=\i{} explains these
terms in Book I, Part II, Analysis of Terms, 63. } of the r\=upas
that are explained. The ``Visuddhimagga'' (XI,93)\footnote{See also
Dhamma\-sanga\d ni {\S}648, and Atthas{\=a}lin\=\i{} II, Book II, Part I, Ch
III, 332.} gives, for example, the following definition of the r\=upa
that is the earth element or solidity:




\begin{quote}\begin{flushleft}
``\ldots The earth element has the characteristic of hardness. Its function
is to act as a foundation. It is manifested as receiving.\ldots''
\end{flushleft}\end{quote}




Each reality has its own individual characteristic by which it can be
distinguished from other realities. The earth element or solidity has
hardness (or softness) as characteristic, the fire element has heat as
characteristic. Such characteristics can be experienced when they
appear. As to function, r\=upas have functions in relation to other
r\=upas or in relation to n{\=a}ma. Solidity acts as a foundation,
namely for the other r\=upas it arises together with in a group; that
is its function. Smell, for example, could not arise alone, it needs
solidity as foundation. It is the same with visible object or colour
that can be experienced through the eyesense. Visible object or colour
needs solidity as foundation or support, it could not arise alone.
Solidity that arises together with visible object cannot be seen, only
visible object can be seen. As regards manifestation, this is the way a
reality habitually appears. Solidity is manifested as receiving, it
receives the other r\=upas it arises together with since it acts as
their foundation. With regard to the proximate cause, according to the
``Visuddhimagga'' (XIV,35) each of the four Great Elements has the
other three as its proximate cause. The four Great Elements arise
together and condition one another. 

At first the definitions of realities may seem complicated but when we
have studied them, we shall see that they are helpful for the
understanding of the different realities, and this includes
understanding of the way they act on other realities and the way they
manifest themselves. The study of realities is a foundation for the
development of direct understanding, of seeing things as they really
are.

In the ``Greater Discourse on the Simile of the Elephant's Footprint''
(Middle Length Sayings I, no. 28) we read that S{\=a}riputta taught the
monks about the four Great Elements. We read about the element of earth
or solidity, which is translated here as ``extension'': 




\begin{quote}\begin{flushleft}``
\dots And what, your reverences, is the element of extension? The
element of extension may be internal, it may be external. And what,
your reverences, is the internal element of extension? Whatever is
hard, solid, is internal, referable to an individual and derived
therefrom, that is to say: the hair of the head, the hair of the body,
nails, teeth, skin, flesh, sinews, bones, marrow of the bones, kidney,
heart, liver, pleura, spleen, lungs, intestines, mesentary, stomach,
excrement, or whatever other thing is hard, solid, is internal\ldots''
\end{flushleft}\end{quote}




If the body can be seen as only elements, the wrong view of self can be
eradicated. Solidity can be internal or external, outside the body.
Solidity is also present in what we call a mountain or a rock, in all
material phenomena. S{\=a}riputta reminded the monks of the
impermanence of the element of extension:




\begin{quote}\begin{flushleft}
``There comes a time, your reverences, when the element of extension
that is external is agitated; at that time the external element of
extension disappears. The impermanence of this ancient external element
of extension can be shown, your reverences, its liability to
destruction can be shown, its liability to decay can be shown, its
liability to change can be shown. So what of this shortlived body
derived from craving? There is not anything here for saying, `I', or
`mine' or `I am'\ldots''
\end{flushleft}\end{quote}




The impermanence of the element of solidity may manifest itself in such
calamities of nature as an earthquake, but actually at each and every
moment r\=upas arise and then fall away, they do not last.

As regards the element of water (in P{\=a}li: {\=a}po dh{\=a}tu) or
cohesion, the ``Visuddhimagga'' (XI, 93) defines it as follows
\footnote{See also Dhamma\-sanga\d ni {\S}652 and Atthas{\=a}lin\=\i{}  II,
Book II, Part I, Ch III, 332.} : 




\begin{quote}\begin{flushleft} 
``\ldots The water element has the characteristic of trickling. Its function is to intensify. It is manifested as holding together.''
\end{flushleft}\end{quote}




The element of water or cohesion cannot be experienced through the
bodysense, only through the mind-door. When we touch what we call
water, it is only solidity, temperature or motion which can be
experienced through the bodysense, not cohesion. Cohesion has to arise
together with whatever kind of materiality arises. It makes the other
r\=upas it accompanies cohere so that they do not become scattered. The
``Atthas{\=a}lin\=\i`` (II, Book II, Ch III, 335) explains:




\begin{quote}\begin{flushleft}
``\dots For the element of cohesion binds together iron, etc., in masses,
makes them rigid. Because they are so bound, they are called rigid.
Similarly in the case of stones, mountains, palm-seeds,
elephant-tusks, ox-horns, etc. All such things the element of
cohesion binds, and makes rigid; they are rigid because of its
binding.''
\end{flushleft}\end{quote}




We read in the above quoted sutta that S{\=a}riputta explained to the
monks about the internal liquid element (element of water):




\begin{quote}\begin{flushleft}
``\ldots Whatever is liquid, fluid, is internal, referable to an
individual or derived therefrom, that is to say: bile, phlegm, pus,
blood, sweat, fat, tears, serum, saliva, mucus, synovial fluid, urine
or whatever other thing is liquid, fluid, is internal \ldots''
\end{flushleft}\end{quote}




When we shed tears or swallow saliva we can be reminded that what we
take for the fluid of ``my body'' are only elements devoid of self.
S{\=a}riputta reminded the monks that the external liquid element can
become agitated and can bring destruction to villages, towns, districts
and regions, or that the water of the oceans may go down and disappear.
It is liable to change and it is impermanent. Both the internal and the
external liquid element are impermanent and not self.

As to the element of fire, heat or temperature (in P{\=a}li: tejo
dh{\=a}tu), the ``Visuddhimagga'' (XI, 93) gives the following
definition of it\footnote{See also Dhamma\-sanga\d ni {\S}648 and
Atthas{\=a}lin\=\i{}  II, Book II, Part I, Ch III, 332.} :




\begin{quote}\begin{flushleft}
``\ldots The fire element has the characteristic of heat. Its function is to mature (maintain). It is manifested as a continued supply of softness\footnote{The Atthas{\=a}lin\=\i{}  (II, Book II, Part I, Ch III, 332)
states that it has ``the gift of softening (co-existent realities) as
manifestation''. It states: ``When this body is accompanied by the
life-controlling faculty, by the element of heat, by consciousness,
then it becomes lighter, softer, more wieldy.'' In a corpse there is
no body heat, it is stiff and not wieldy.}.''

\end{flushleft}\end{quote}



The element of heat or temperature can be experienced through the
bodysense and it appears as heat or cold. Cold is a lesser degree of
heat. The element of heat accompanies all kinds of materiality that
arises, r\=upas of the body and materiality outside. It maintains or
matures them. The element of heat is one of the four factors that
produce r\=upas of the body. At the first moment of life, kamma, a deed
committed in the past, produces the rebirth-consciousness and also
r\=upa. After the rebirth-consciousness has arisen temperature also
starts to produce r\=upas of the body\footnote{This will be
explained later on.}. R\=upas which are materiality outside such as
those of a plant or a rock are produced solely by temperature. 

We read in the above quoted sutta that S{\=a}riputta explained to the
monks about the internal element of heat:




\begin{quote}\begin{flushleft}
``\ldots Whatever is heat, warmth, is internal, referable to an individual
and derived therefrom, such as by whatever one is vitalized, by
whatever one is consumed, by whatever one is burnt up, and by whatever
one has munched, drunk, eaten and tasted that is properly transmuted
(in digestion), or whatever other thing is heat, warmth, is
internal\ldots''
\end{flushleft}\end{quote}




The ``Visuddhimagga'' (XI, 36) which gives an explanation of the words
of this sutta states that the element of heat plays its part in the
process of ageing: 
\begin{quotation}\begin{flushleft}
`` \ldots whereby this body grows old, reaches the
decline of the faculties, loss of strength, wrinkles, greyness, and so
on.''
\end{flushleft}\end{quotation}
As to the expression ``burnt up'', it explains that when one is
excited the internal element of heat causes the body to burn. The
element of heat also has a function in the digestion of food, it
``cooks'' what is eaten and drunk.

We may notice changes in body-temperature because of different
conditions, for instance through the digestion of our food, or when we
are excited, angry or afraid. So long as we are still alive the
internal element of heat arises and falls away all the time. When heat
presents itself and when there is awareness of it it can be known as
only a r\=upa element, not ``my body-heat''. When we are absorbed in
excitement, anger or fear we forget that there are in reality only
different kinds of n{\=a}ma and r\=upa that arise and fall away. 

The element of heat can be internal or external. S{\=a}riputta explained
that the liability to change of the external heat element and its
impermanence can be seen when it becomes agitated and burns up
villages, towns, districts and regions, and is then extinguished
through lack of fuel. Both the internal and the external element of
heat are impermanent and not self.

As to the element of wind (in P{\=a}li: v{\=a}yo dh{\=a}tu) or motion,
the ``Visuddhimagga'' (XI, 93) defines it as follows\footnote{See
also Dhamma\-sanga\d ni {\S}648 and Atthas{\=a}lin\=\i{}  II, Book II, Part
I, Ch III, 332.} : 




\begin{quote}\begin{flushleft}
``\ldots The air element (wind) has the characteristic of distending. Its
function is to cause motion. It is manifested as
conveying\footnote{Taking from one point to another, Visuddhimagga
XI, 93. The commentary explains: ``Conveying is acting as cause for the
successive arising at adjacent locations of the conglomeration of
elements.''}.''
\end{flushleft}\end{quote}




We may believe that we can see motion of objects but the r\=upa which is
motion cannot be seen. What we mean by motion as we express it in
conventional language is not the same as the element of wind or motion.
We notice that something has moved because of remembrance of different
moments of seeing and thinking of what was perceived, but that is not
the experience of the r\=upa which is motion. This r\=upa can be
directly experienced through the bodysense. When we touch a body or an
object with a certain resilience, the characteristic of motion or
pressure may present itself. These are characteristics of the element
of wind. It can also be described as vibration or oscillation. As we
read in the definition, the function of the element of wind is to cause
motion. It is, for example, a condition for the movement of the limbs
of the body. However, we should not confuse pictorial ideas with the
direct experience of this r\=upa through the bodysense. 

The element of wind or motion arises with all kinds of materiality, both
of the body and outside the body. There is also motion with dead
matter, such as a pot. It performs its function so that the pot holds
its shape and does not collapse.

S{\=a}riputta explained about the internal element of motion:




\begin{quote}\begin{flushleft}
``\ldots And what, your reverences, is the internal element of motion?
Whatever is motion, wind, is internal, referable to an individual and
derived therefrom, such as winds going upwards, winds going downwards,
winds in the abdomen, winds in the belly, winds that shoot across the
several limbs, in-breathing, out-breathing, or whatever other thing
is motion, wind, is internal\ldots''
\end{flushleft}\end{quote}







We may notice pressure inside the body. When its characteristic appears
it can be known as only a r\=upa that is conditioned. As to the words
of the sutta, ``winds that shoot across the several limbs'', the
``Visuddhimagga'' (XI, 37) explains that these are: ``winds (forces)
that produce flexing, extending, etc., and are distributed over the
limbs and the whole body by means of the network of veins (nerves)''. 

The element of wind plays its specific role in the strengthening of the
body so that it does not collapse, and assumes different postures; it
is a condition for the stretching and bending of the limbs. While we
are bending or stretching our arms and legs the element of wind may
appear as motion or pressure. We read in the ``Visuddhimagga'' (XI,
92):




\begin{quote}\begin{flushleft}
``The air element that courses through all the limbs and has the
characteristic of moving and distending, being founded upon earth, held
together by water, and maintained by fire, distends this body. And this
body, being distended by the latter kind of air, does not collapse, but stands
erect, and being propelled by the other (moving) air, it shows
intimation, and it flexes and extends and it wriggles the hands and
feet, doing so in the postures comprising walking, standing, sitting
and lying down. So this mechanism of elements carries on like a magic
trick, deceiving foolish people with the male and female sex and so
on.''
\end{flushleft}\end{quote}




We are deceived and infatuated by the outward appearance of a man or a
woman and we forget that this body is a ``mechanism of elements'' and
that it flexes and wriggles hands and feet, showing intimation by
gestures or speech, because of conditions.

The above quoted sutta mentions, in connection with the element of wind,
in-breathing and out-breathing. The ``Visuddhimagga'' (XI, 37)
explains: ``In-breath: wind in the nostrils entering in.
Out-breath: wind in the nostrils issuing out.'' We are breathing
throughout life, but most of the time we are forgetful of realities, we
cling to an idea of ``my breath''. Breath is r\=upa conditioned by
citta and it presents itself where it touches the nosetip or upperlip.
If there can be awareness of it, the characteristics of hardness,
softness, heat or motion can be experienced one at a time. However,
breath is very subtle and it is most difficult to be aware of its
characteristic. 

We read in the above quoted sutta that S{\=a}riputta explained that the
external element of motion can become agitated and carry away villages.
Its liability to change and its impermanence can be seen. Both the
external and the internal element of motion are impermanent. 

As we have seen, the four great Elements always arise together, and each
of them has the other three as its proximate cause. The
``Visuddhimagga'' (XI, 109) states that the four great Elements
condition one another: the earth element acts as the foundation of the
elements of water, fire and wind; the water element acts as cohesion
for the other three Great Elements; the fire element maintains the
other three Great Elements; the wind element acts as distension of the
other three Great Elements.

We should remember that the element of water or cohesion cannot be
experienced through the bodysense, only through the mind-door, and
that the elements of earth, fire and wind can be directly experienced
through the bodysense. The element of earth appears as hardness or 
softness, the element of fire as heat or cold and the element of wind as
motion or pressure. Time and again r\=upas such as hardness or heat
impinge on the bodysense but we are forgetful of what things really
are. We let ourselves be deceived by the outer appearance of things.
The ``Visuddhimagga'' (XI, 100) states that the four Great Elements are
``deceivers'':




\begin{quote}\begin{flushleft}
``And just as the great creatures known as female spirits (yakkhin\=\i)
conceal their own fearfulness with a pleasing colour, shape and
gesture to deceive beings, so too, these elements conceal each their
own characteristics and function classed as hardness, etc., by means of
a pleasing skin colour of women's and men's bodies, etc., and pleasing
shapes of limbs and pleasing gestures of fingers, toes and eyebrows,
and they deceive simple people by concealing their own
functions and characteristics beginning with hardness and do not allow
their individual essences to be seen. Thus they are great primaries
(mah{\=a}-bh\=uta) in being equal to the great creatures
(mah{\=a}-bh\=uta), the female spirits, since they are deceivers.''

\end{flushleft}\end{quote}



Realities are not what they appear to be. One may be infatuated by the
beauty of men and women, but what one takes for a beautiful body are
mere r\=upa-elements.

The ``Visuddhimagga'' (XI, 98) states that the four Great Elements are
like the great creatures of a magician who ``turns water that is not
crystal into crystal, and turns a clod that is not gold into gold\ldots''
We are attached to crystal and gold, we are deceived by the outward
appearance of things. There is no crystal or gold in the ultimate
sense, only r\=upas which arise and then fall away.

We may be able to know the difference between the moments that we are
absorbed in concepts and ideas and mindfulness of realities such as
hardness or heat which appear one at a time. Mindfulness (sati) arises
with kusala citta and it is mindful of one n{\=a}ma or r\=upa at a
time. When we are, for example, stung by a mosquito, we may have
aversion towards the pain and we may be forgetful of realities such as
heat experienced at that moment through the bodysense. When there are
conditions for kusala citta with mindfulness, whatever reality appears
can be object of mindfulness. This is the way gradually to develop the
understanding which knows n{\=a}ma and r\=upa as they are: only
elements that are impermanent and devoid of self.

As we read in the ``Greater Discourse of the Simile of the Elephant's
Footprint'', different ``parts of the body'', such as  the hair of the head, the hair of 
the body, nails, teeth, skin, are mentioned where the
characteristics of the four Great Elements are apparent. The aim is to
see the body as it really is. When S{\=a}riputta explained about the
four Great Elements he repeated after each section:




\begin{quote}\begin{flushleft}
``\ldots By means of perfect intuitive wisdom it should be seen of this as
it really is, thus: This is not mine, this am I not, this is not
myself\ldots''
\end{flushleft}\end{quote}








\subsection*{Questions}




\begin{enumerate}
\item Can the element of water be experienced through touch?

\item Can the characteristic of motion be experienced through \\eyesense?

\item What is the proximate cause of each of the four Great \\Elements?
\end{enumerate}



























































































\chapter[The Eight Inseparables]{}  
\section*{The Eight Inseparable R\=upas}


The four Great Elements of solidity, cohesion, temperature and motion
are always present wherever there is materiality. Apart from these four
elements there are other r\=upas, namely twentyfour ``derived r\=upas''
(in P{\=a}li: up{\=a}d{\=a} r\=upas). The ``Atthas{\=a}lin\=\i'' (II,
Book II, Ch III, 305) explains about them: ``\ldots grasping the great
essentials (great elements), not letting go, such (derived r\=upas)
proceed in dependance upon them.'' Thus, the derived r\=upas cannot
arise without the four Great Elements. 

Four among the derived r\=upas always arise together with the four Great
Elements in every group of r\=upas and are thus present wherever
materiality occurs, no matter whether r\=upas of the body or
materiality outside the body. These four r\=upas are the following: 




\begin{description}
\item visible object (or colour)

\item odour

\item flavour

\item nutrition
\end{description}




The four Great elements and these four derived r\=upas, which always
arise together, are called the ``inseparable r\=upas'' (in P{\=a}li:
avinibbhoga r\=upas). Wherever solidity arises, there also have to be
cohesion, temperature, motion, colour, odour, flavour and nutritive
essence.

As regards visible object or colour, this is a r\=upa that can be
experienced through the eye-door. It is not a thing or a person.
Visible object is the only r\=upa that can be seen.

Colours are different because of different conditions\footnote{See
also Dhamma\-sanga\d ni {\S}617.} , but no matter what colour appears we
should remember that what is experienced through the eye-door is the
r\=upa which is visible, visible object. The ``Atthas{\=a}lin\=\i'' (II,
Book II, Ch III, 318) gives the following definition of visible object
\footnote{See also Visuddhimagga XIV, 54}: 




\begin{quote}\begin{flushleft}
``\ldots For all this matter has the characteristic of striking the eye,
the function or property of being in relation of object to visual
cognition, the manifestation of being the field of visual cognition,
the proximate cause of the ``four great essentials'' (four Great
Elements).''
\end{flushleft} \end{quote}




Visible object has as its proximate cause the four Great Elements
because it cannot arise without them. However, when a characteristic of
one of these four Great Elements, such as hardness or heat, is
experienced, the accompanying visible object cannot be experienced at
the same time.

When there are conditions for seeing, visible object is experienced.
When we close our eyes, there may be remembrance of the shape and form
of a thing, but that is not the experience of visible object. The
thinking of a ``thing'', no matter whether our eyes are closed or open,
is different from the actual experience of what is visible. 

We may find it difficult to know what visible object is, since we are
usually absorbed in paying attention to the shape and form of things.
When we perceive the shape and form of something, for example of a
chair, we think of a concept. A chair cannot impinge on the eyesense.
Seeing does not see a chair, it only sees what is visible. Seeing and
thinking occur at different moments. We do not think all the time, also
moments of just seeing arise, moments that we do not pay attention to
shape and form. Only one citta at a time arises experiencing one
object, but different experiences arise closely one after the other.
When one cannot distinguish them yet from each other, one believes that
they occur all at the same time. If we remember that visible object is
the r\=upa which can be experienced through the eyesense, right
understanding of this reality can be developed. 

As we have seen, odour is another r\=upa among the eight inseparable
r\=upas. Wherever materiality occurs, no matter whe\-ther of the body or
outside the body, there has to be odour. The ``Dhamma\-sanga\d ni'' ({\S}
625) mentions different odours, pleasant and unpleasant, but they all
are just odour which can be experienced through the nose. The
``Atthas{\=a}lin\=\i'' (II, Book II, Ch III, 320) defines odour as
follows\footnote{See also Visuddhimagga XIV, 56.} : 


\begin{quote}\begin{flushleft}
``\ldots all odours have the characteristic of striking the sense of smell, the property of being the object of olfactory cognition, the
manifestation of being the field of the same\ldots''
\end{flushleft}\end{quote}


It has as proximate cause the four Great Elements. Odour cannot arise
alone, it needs the four Great Elements which arise together with it
and it is also accompanied by the other r\=upas included in the eight
inseparable r\=upas. When odour appears we tend to be carried away by
like or dislike. We are attached to fragrant odours and we loathe nasty
smells. However, odour is only a reality which is experienced through
the nose and it does not last. If one does not develop understanding of
realities one will be enslaved by all objects experienced through the
senses. On account of these objects akusala cittas tend to arise. If
someone thinks that there is a self who can own what is seen, touched
or smelt, he may be inclined to commit unwholesome deeds such as
stealing. In reality all these objects are insignificant, they arise
and then fall away immediately. 

As regards flavour, the ``Dhamma\-sanga\d ni'' ({\S}629) mentions
different kinds of flavour, such as sour, sweet, bitter or pungent;
they may be nice or nauseous, but they are all just flavour,
experienced through the tongue. The ``Atthas{\=a}lin\=\i''
(II, Book II, Ch III, 320) defines flavour as follows\footnote{See
also Visuddhimagga XIV, 57} : 




\begin{quote}\begin{flushleft}
``\ldots all tastes have the characteristic of striking the tongue, the
property of being the object of gustatory cognition, the manifestation
of being the field of the same\ldots''
\end{flushleft}\end{quote}




Its proximate cause are the four Great Elements. Flavour does not arise
alone, it needs the four Great Elements that arise together with it,
and it is also accompanied by the other r\=upas included in the eight
inseparable r\=upas. We are attached to food and we find its flavour
very important. As soon as we have tasted delicious flavour, attachment
tends to arise. We are forgetful of the reality of flavour which is
only a kind of r\=upa. When we recognize what kind of flavour we taste,
we think about a concept, but this thinking is conditioned by the
experience of flavour through the tongue.

Nutrition is another kind of r\=upa which has to arise with every kind
of materiality. It can be exerienced only through the mind-door. The
``Dhamma\-sanga\d ni'' ({\S}646) mentions food such as boiled rice, sour
gruel, flour, etc., which can be eaten and digested into the ``juice''
by which living beings are kept alive. The ``Atthas{\=a}lin\=\i'' (II,
Book II, Ch III, 330) explains that there is foodstuff, the substance
which is swallowed (kaba\d link{\=a}ro {\=a}h{\=a}ro, literally,
morsel-made food), and the ``nutritive essence'' (oj{\=a}). The
foodstuff which is swallowed fills the stomach so that one does not
grow hungry. The nutritive essence present in food preserves beings,
keeps them alive. The nutritive essence in gross foodstuff is weak, and
in subtle foodstuff it is strong. After eating coarse grain one becomes
hungry after a brief interval. But when one has taken ghee (butter) one
does not want to eat for a long time (Atthas{\=a}lin\=\i{} , 331).

The ``Atthas{\=a}lin\=\i''(332) gives the following definition of
nutriment\footnote{See also Visuddhimagga XIV, 70} : 




\begin{quote}\begin{flushleft}
``As to its characteristic, etc., solid food has the characteristic of
nutritive essence, the function of fetching matter (to the eater), of
sustaining matter as its manifestation, of substance to be swallowed as
proximate cause.''

\end{flushleft}\end{quote}



Nutritive essence is not only present in rice and other foods, it is
also present in what we call a rock or sand. It is present in any kind
of materiality. Insects are able to digest what human beings cannot
digest, such as, for example, wood.

Nutrition is one of the four factors which produce r\=upas of the body.
As we have seen, the other factors are kamma, citta and temperature
\footnote{See Introduction.}. In the unborn being in the mother's
womb, groups of r\=upa produced by nutrition arise as soon as the
nutritive essence present in food taken by its mother pervades its body
(Visuddhimagga XVII, 194). From then on nutrition keeps on producing
r\=upas and sustaining the r\=upas of the body throughout life. 

We can notice that nutrition produces r\=upas when good or bad food
affects the body in different ways. Bad food may cause the skin to be
ugly, whereas the taking of vitamins for example may cause skin and
hair to look healthy.

Because of attachment we tend to be immoderate as to food. We are not
inclined to consider food as a medicine for our body. The Buddha
exhorted the monks to eat just the quantity of food needed to sustain
the body but not more and to reflect wisely when eating (Visuddhimagga
I, 85). The monk should review with understanding the requisites he
receives. We read in the ``Visuddhimagga'' (I, 124) about the right way
of using the requisites (of robes, food, etc.):




\begin{quote}\begin{flushleft}
``\ldots For use is blameless in one who at the time of receiving robes,
etc., reviews them either as (mere) elements or as repulsive, and puts
them aside for later use, and in one who reviews them thus at the time
of using them.'' 
\end{flushleft}\end{quote}




The monk should review robes, and the other requisites of dwell\-ing, food
and medicines, as mere elements or as repulsive. If he considers food
as repulsive it helps him not to indulge in it. Food consists merely of
conditioned elements. This can be a useful reminder, also for
laypeople, to be mindful when eating.

In the commentary to the ``Satipa\d t\d th{\=a}na Sutta''\footnote{The
Papa\~ncas\=udan\=\i. See ``The Way of Mindfulness'', a translation of
the Satipa\d t\d th{\=a}na Sutta, Middle Length Sayings I, 10, and
its commentary, by Ven. Soma, B.P.S. Kandy. }, in the section on
Mindfulness of the Body, ``Clear Comprehension in Partaking of Food and
Drink'', we read that, when one swallows food, there is no one who puts
the food down into the stomach with a ladle or spoon, but there is the
element of wind performing its function. We then read about digestion:




\begin{quote}\begin{flushleft}
``\dots There is no one who having put up an oven and lit a fire is
cooking each lump standing there. By only the process of caloricity
(heat) the lump of food matures. There is no one who expels each
digested lump with a stick or pole. Just the process of oscillation
(the element of wind or motion) expels the digested food.'' 

\end{flushleft}\end{quote}



There is no self who eats and drinks, there are only elements performing
their functions.

Whatever kind of materiality arises, there have to be the four Great
Elements and the four derived r\=upas of visible object, odour, flavour
and nutrition.

Because of ignorance we are attached to our possessions. We may
understand that when life ends we cannot possess anything anymore. But
even at this moment there is no ``thing'' we can possess, there are
only different elements that do not stay. When we look at beautiful
things such as gems we tend to cling to them. However, through the eyes
only colour or visible object appears and through touch tangible object
such as hardness appears. In the absolute sense it does not make any
difference whether it is hardness of a gem or hardness of a pebble that
is experienced through touch. We may not like to accept this truth
since we find that gems and pebbles have different values. We have
accumulated conditions to think about concepts and we neglect the
development of understanding of realities; we tend to forget that what
we call gems and also the cittas that enjoy them do not last, that they
are gone immediately. Someone who leads the life of a layman enjoys his
possessions, but he can also develop understanding of what things
really are. 

In the ultimate sense life exists only in one moment, the present
moment. At the moment of seeing the world of visible object is
experienced, at the moment of hearing the world of sound, and at the
moment of touching the world of tangible object. Life is actually one
moment of experiencing an object.

The ``Book of Analysis''\footnote{Vibha\.nga, Second Book of the
Abhidhamma, Pali Text Society, 1969.} (Part 3, Analysis of the
Elements, {\S}173) mentions precious stones together with pebbles and
gravel in order to remind us of the truth. It explains about the
internal element of extension (solidity) as being hair of the head,
hair of the body and other ``parts of the body''. Then it explains
about the external element of extension as follows:




\begin{quote}\begin{flushleft}
``Therein what is the external element of extension? That which is
external, hard, harsh, hardness, being hard, external, not grasped. For
example: iron, copper, tin, lead, silver, pearl, gem, cat's-eye,
shell, stone, coral, silver coin, gold, ruby, variegated precious
stone, grass, wood, gravel, potsherd, earth, rock, mountain; or
whatever else there is\ldots''

\end{flushleft}\end{quote}



The elements give us pleasure or pain. When we do not realize them as
they are, we are enslaved by them. We read in the ``Kindred
Sayings''(II, Nid{\=a}na-vagga, Ch XIV, Kindred Sayings on Elements,
{\S}34, Pain) that the Buddha said to the monks at S{\=a}vatth\=\i:




\begin{quote}\begin{flushleft}
``If this earth-element, monks, this water-element, this
heat-element, this air-element were entirely painful, beset with
pain, immersed in pain, not immersed in happiness, beings would not be
lusting after them. But inasmuch as each of these elements is pleasant,
beset with pleasure, immersed in pleasure, not in pain, therefore it is
that beings get lusting after them.

If this earth-element, monks, this water-element, this
heat-element, this air-element were entirely pleasant, beset with
pleasure, immersed in pleasure, not immersed in pain, beings would not
be repelled by them. But inasmuch as each of these elements is painful,
is beset with pain, immersed in pain, not immersed in pleasure,
therefore it is that beings are repelled by them.''


\end{flushleft}\end{quote}


We are bound to be attached to the elements when we buy beautiful
clothes or enjoy delicious food. We are bound to be repelled by the
elements when we get hurt or when we are sick. But no matter whether
the objects we experience are pleasant or unpleasant, we should realize
them as elements that arise because of their own conditions and that do
not belong to us.














\subsection*{Questions}


\begin{enumerate}
\item Is there nutrition with matter we call a table?

\item Why are eight r\=upas called the ``inseparable r\=upas''?

\item Nutrition is one of the four factors which can produce r\=upa. Can it produce the materiality we call ``tree''?
\end{enumerate}














\chapter[The Sense-Organs]{}
\section*{The Sense-Organs (Pas\=ada R\=upas)}




So long as there are conditions for birth we have to be born and to
experience pleasant or unpleasant objects. It is kamma that produces
rebirth-consciousness as well as seeing, hearing and the other
sense-impressions arising throughout our life. For the experience of
objects through the senses there have to be sense-organs and these
are r\=upas produced by kamma as well. The sense-organs (pas{\=a}da
r\=upas) are physical results of kamma, whereas seeing, hearing and the
other sense-impressions are n{\=a}ma, vip{\=a}kacittas which are the
mental results of kamma\footnote{See Introduction}. 

Visible object and also the r\=upa which is eyesense are conditions for
seeing. Eyesense does not know anything since it is r\=upa, but it is a
necessary condition for seeing. Eyesense is a r\=upa in the eye,
capable of receiving visible object, so that citta can experience it.
For hearing, the experience of sound, there has to be earsense, a
r\=upa in the ear, capable of receiving sound. There must be
smellingsense for the experience of odour, tastingsense for the
experience of flavour and bodysense for the experience of tangible
object. Thus, there are five kinds of sense-organs.

As regards the eye, the ``Atthas{\=a}lin\=\i'' (II, Book II, Ch III, 306)
distinguishes between the eye as ``compound organ'' and as ``sentient
organ'', namely the r\=upa which is eyesense, situated in the eye
\footnote{In P{\=a}li: cakkhu pas{\=a}da r\=upa}. The eye as
``compound organ'' is described as follows:




\begin{quote}\begin{flushleft}
``\ldots a lump of flesh is situated in the cavity of the eye, bound by the bone of the cavity of the eye below, by the bone of the brow above, by
the eye-peaks on both sides, by the brain inside, by the eyelashes
outside\ldots Although the world perceives the eye as white, as (of a
certain) bigness, extension, width, they do not know the real sentient
eye, but only the physical basis thereof. That lump of flesh situated
in the cavity of the eye is bound to the brain by sinewy threads.
Therein are white, black, red, extension, cohesion, heat and mobility.
The eye is white from the abundance of phlegm, black from that of bile,
red from that of blood, rigid from the element of extension, fluid from
that of cohesion, hot from that of heat, and oscillating from that of
mobility. Such is the compound organ of the eye\ldots''
\end{flushleft}\end{quote}




As to the ``sentient eye'' or eyesense, this is to be found, according
to the ``Atthas{\=a}lin\=\i'', in the middle of the black circle,
surrounded by white circles, and it permeates the ocular membranes ``as
sprinkled oil permeates seven cotton wicks.'' We read:




\begin{quote}\begin{flushleft}
``And it is served by the four elements doing the functions of
sustaining, binding, maturing and vibrating\footnote{The earth
element performs its function of sustaining, the water element of
holding together, the fire element of maintaining or maturing, and the
wind element of oscillation.} , just as a princely boy is tended by
four nurses doing the functions of holding, bathing, dressing and
fanning him. And being upheld by the caloric order, by thought (citta)
and nutriment, and guarded by life and attended by colour, odour,
taste, etc., the organ, no bigger in size than the head of a louse,
stands duly fulfilling the nature of the basis and the door of visual
cognition, etc. \ldots''
\end{flushleft}\end{quote}




The ``Visuddhimagga'' (XIV, 37) gives the following definition of
eye-sense\footnote{See also Dhamma\-sanga\d ni {\S}597 and
Atthas{\=a}lin\=\i{}  II, Book II, Part I, Ch III, 312.}:




\begin{quote}\begin{flushleft}
``Herein, the eye's characteristic is sensitivity of primary elements
that is ready for the impact of visible data; or its characteristic is
sensitivity of primary elements originated by kamma sourcing from
desire to see. Its function is to pick up (an object) among visible
data. It is manifested as the footing of eye-consciousness. Its
proximate cause is primary elements (the four Great Elements) born of
kamma sourcing from desire to see.''
\end{flushleft}\end{quote}




We have desire to see, we are attached to all sense-impress\-ions and,
thus, there are still conditions for kamma to produce rebirth, to
produce seeing, hearing and the other sense-impres\-sions, and also to
produce the sense-organs which are the conditions for the experience
of sense objects. Also in future lives there are bound to be
sense-impressions. 

Eyesense seems to last and we are inclined to take it for ``self''. It
seems that the same eyesense keeps on performing its function as a
condition for seeing which also seems to last. However, eyesense arises
and then falls away. At the next moment of seeing another eyesense has
arisen. All these eyesenses are produced by kamma, throughout our life.
We may find it hard to grasp this truth because we are so used to
thinking of ``my eyesense'' and to consider it as something lasting. 

The eyesense is extremely small, ``no bigger in size than the head of a
louse'', but it seems that the whole wide world comes to us through
the eye. All that is visible is experienced through the eyesense, but
when we believe that we see the world, there is thinking of a concept,
not the experience of visible object. Our thinking is conditioned by
seeing and by all the other sense-impressions. 

The eye is compared to an ocean\footnote{Dhamma\-sanga\d ni {\S}597.
Atthas{\=a}lin\=\i{} II, Book II, Part I, Ch III, 308.}, because it cannot
be filled, it is unsatiable. We are attached to the eyesense and we
want to go on seeing, it never is enough.

We read in the ``Kindred Sayings'' (IV, Sa\d l{\=a}yatana-vagga, Fourth
Fifty, Ch 3, {\S}187, The Ocean)\footnote{I used the
translation by Ven. Bodhi, ``The Connected Discourses of the Buddha''.
``Form'' is his translation of r\=upa, which is actually visible
object. }:




\begin{quote}\begin{flushleft}
``\ldots The eye, bhikkhus, is the ocean for a person; its current consists
of forms. One who withstands that current consisting of forms is said
to have crossed the ocean of the eye with its waves, whirlpools, sharks
and demons. Crossed over, gone beyond, the brahmin stands on high
ground.'' 
\end{flushleft}\end{quote}

The same is said with regard to the other senses.

We read in the ``Ther\=\i{} g{\=a}th{\=a}'' (Psalms of the Sisters, Canto
XIV, 71, Subh{\=a} of J\=\i vaka's Mango-grove) that the Ther\=\i{}  
Subh{\=a} became an an{\=a}g{\=a}m\=\i\footnote{There are four stages
of enlightenment. The an{\=a}g{\=a}m\=\i{} or ``non-returner'' has
reached the third stage. The arahat has reached the last stage.} ; she
had eradicated clinging to sense objects. A young man, infatuated with
the beauty of her eyes, wanted to tempt her. She warned him not to be
deluded by the outward appearance of things. In reality there are only
elements devoid of self. The Ther\=\i{}  said about her eye (vs. 395):




\begin{verse}
``What is this eye but a little ball lodged in the fork of a hollow 
tree,

Bubble of film, anointed with tear-brine, exuding slime-drops.

Compost wrought in the shape of an eye of manyfold aspects?\ldots''
\end{verse}




The Ther\=\i{}  extracted one of her eyes and handed it to him. The impact
of her lesson did not fail to cure the young man of his lust. Later on,
in the presence of the Buddha, her eye was restored to her. She
continued to develop insight and attained arahatship.

Eyesense is only an element devoid of self. It is one of the conditions
for seeing. The ``Visuddhimagga'' (XV, 39) states about the conditions
for seeing: ``Eye-consciousness arises due to eye, visible object,
light and attention''. 

Earsense is another sense-organ. The ``Atthas{\=a}lin\=\i'' states that
it is situated in the interior of the ear, ``at a spot shaped like a
finger-ring and fringed by tender, tawny hairs\ldots''
\footnote{Atthas{\=a}lin\=\i{}  II, Book II, Part I, Ch III, 310.}
Earsense is the r\=upa which has the capability to receive sound. It is
basis and door of hearing-consciousness. The ``Visuddhimagga'' (XIV,
38) gives the following definition\footnote{See also
``Dhamma\-sanga\d ni {\S}601 and Atthas{\=a}lin\=\i{}  II, Book II, Part I,
Ch III, 312.}: 




\begin{quote}\begin{flushleft}
``The ear's characteristic is sensitivity of primary elements that is
ready for impact of sounds; or its characteristic is sensitivity of
primary elements originated by kamma sourcing from desire to hear. Its
function is to pick up (an object) among sounds. It is manifested as
the footing of ear-consciousness. Its proximate cause is primary
elements born of kamma sourcing from desire to hear.''
\end{flushleft}\end{quote}




Without earsense there cannot be hearing. The ``Visuddhimagga'' (XV,
39) states: ``Ear-consciousness arises due to ear, sound, aperture
and attention.'' ``Aperture'' is the cavity of the ear. If one of these
conditions is lacking hearing cannot arise. 

As to the other pas{\=a}da r\=upas, smellingsense, tastingsense and
bodysense, these are defined in the same way\footnote{See
Dhamma\-sanga\d ni {\S}605, 609, 613, Visuddhimagga XIV, 39, 40, 41,
Atthas{\=a}lin\=\i{} , Book II, Part I, Ch III, 312.}. Smell\-ing\-sense is a
r\=upa situated in the nose. It is one of the conditions for smelling.
The ``Visuddhimagga''(XV, 39) states: ``Nose-consciousness arises due
to nose, odour, air (the element of wind or motion) and attention.'' As
to the element of wind or motion being a condition, we read in the
``Atthas{\=a}lin\=\i'' (II, Book II, Part I, Ch III, 315):




\begin{quote}\begin{flushleft}
``\ldots the nose desires space, and has for object odour dependent on
wind. Indeed, cattle at the first showers of rain keep smelling at the
earth, and turning up their muzzles to the sky breathe in the wind. And
when a fragrant lump is taken in the fingers and smelt, no smell is got
when breath is not inhaled\ldots''
\end{flushleft}\end{quote}




As to tastingsense, this is situated in the tongue and it is one of the
conditions for tasting. The ``Visuddhimagga'' states in the same
section: ``Tongue-consciousness arises due to tongue, flavour, water
and attention.'' Also the element of water or cohesion plays its part
when tasting occurs. We read in the ``Atthas{\=a}lin\=\i'' (same
section, 315) about the element of water being a condition for tasting:




\begin{quote}\begin{flushleft}
``\ldots Thus even when a bhikkhu's duties have been done during the three
watches of the night, and he, early in the morning, taking bowl and
robe, has to enter the village, he is not able to discern the taste of
dry food if it is unwetted by the saliva\ldots''

\end{flushleft}\end{quote}



As to bodysense, this is situated all over the body and inside it,
except in the hairs or tips of the nails. It is one of the conditions
for experiencing tactile object. The ``Visuddhimagga'' states, in the
same section: ``Body-consciousness arises due to body, tangible
object, earth and attention.'' The ``Atthas{\=a}lin\=\i'' (same section,
315) explains:




\begin{quote}\begin{flushleft}
``\ldots Internal and external extension (solidity) is the cause of the
tactile sense seizing the object. Thus it is not possible to know the
hardness or softness of a bed well spread out or of fruits placed in
the hand, without sitting down on the one or pressing the other. Hence
internal and external extension is the cause in the tactile cognition
of the tactile organ.''
\end{flushleft}\end{quote}




Thus, when tactile cognition, bodyconsciousness, arises, there are
actually elements impinging on elements. The impact of tactile object
on the bodysense is more vigorous than the impact of the objects on the
other senses. According to the ``Paramattha Ma\~nj\=usa'', a commentary
to the ``Visuddhimagga''\footnote{See Visuddhimagga, XIV, footnote
56.}, because of the violence of the impact on the bodysense,
body-consciousness (k\=ayavi\~n\~n{\=a}\d na) is accompanied either by
pleasant feeling or by painful feeling, not by indifferent feeling,
whereas the other sense-cognitions (seeing, hearing, etc.) are
accompanied only by indifferent feeling. 

Through the bodysense are experienced: the earth element, appearing as
hardness or softness; the fire element, appearing as heat or cold; the
wind element, appearing as motion or pressure. When these
characteristics appear they can be directly experienced wherever there
is bodysense, thus also inside the body. 

As we have seen, visible object, sound, odour, flavour and tangible
object (which consists of three of the four Great Elements) are
experienced through the corresponding sense-doors and they can also
be experienced through the mind-door. The sense-organs themselves
through which the sense-objects are experienced are r\=upas that can
only be known through the mind-door.

The five sense-organs are the bases (vatthus) or places of origin of
the corresponding sense-cognitions. Cittas do not arise outside the
body, they are dependent on the physical bases where they originate
\footnote{There are also planes of existence where there is only
n{\=a}ma, not r\=upa. In such planes cittas do not need a physical
base. }. The eyesense is the base where seeing-consciousness
originates. The earsense is the base where hearing-consciousness
originates, and it is the same in the case of the other sense-organs. As regards the base for body-consc\-ious\-ness, this can be at any place on the body where there is sensitivity. The sense-organs are bases
only for the corresponding sense-cognitions. All the other cittas
have another base, the heart-base; I shall deal with that later on.

The five sense-organs function also as doorways for the five kinds of
sense-cognitions, as we have seen. The doorway (dv{\=a}ra) is the
means by which citta experiences an object. The eyesense is the doorway
by which seeing-consciousness and also the other cittas arising in
that process experience visible object. As we have seen, cittas which
experience objects impinging on the senses and the mind-door time and
again, arise in processes of cittas.\footnote{See Introduction.} The cittas other than seeing-consciousness which
arise in the eye-door process do not see, but they each perform their
own function while they cognize visible object, such as considering
visible object or investigating it. Each of the five sense-organs can
be the doorway for all the cittas in the process experiencing a
sense-object through that doorway. The sense-organs can have the
function of base as well as doorway only in the case of the five
sense-cognitions. 

The sense-organs arise and fall away all the time and they are only
doorway when an object is experienced through that sense-organ.
Eyesense, for example, is only eye-door when visible object is
experienced by the cittas arising in the eye-door process. When sound
is experienced, earsense is doorway and eyesense does not function as
doorway.

The ``Atthas{\=a}lin\=\i`` (II, Book II, Ch III, 316) states that ``the
senses are not mixed.'' They each have their own characteristic,
function, manifestation and proximate cause, and through each of them
the appropriate object is experienced. The earsense can only receive
sound, not visible object or flavour. Hearing can only experience sound
through the ear-door. We are not used to considering each doorway
separately since we are inclined to think of a person who coordinates
all experiences. We are inclined to forget that a citta arises because
of conditions, experiences one object just for a moment, and then falls
away immediately. In order to help people to have right understanding
of realities, the Buddha spoke time and again about each of the six
doorways separately. He told people to ``guard'' the doorways in being
mindful, because on account of what is experienced through these
doorways many kinds of defilements tend to arise. 

We read in the ``Kindred Sayings'' (IV, Sa\d l{\=a}yatanavagga, Third
Fifty, Ch 3, {\S}127, Bh{\=a}radv{\=a}ja) that King Udena asked the
venerable Bh{\=a}radv{\=a}ja what the cause was that young monks could
practise the righteous life in its fulness and perfection.
Bh{\=a}radv{\=a}ja spoke about the advice the Buddha gave to them, such
as seeing the foulness of the body, and guarding the six doors. We read
that Bh{\=a}radv{\=a}ja said:




\begin{quote}\begin{flushleft}
``\ldots It has been said, Mah{\=a}r{\=a}jah, by the Exalted One\ldots :
`Come, monks, do you abide watchful over the doors of the faculties.
Seeing an object with the eye, be not misled by its outer view, nor by
its lesser details. But since coveting and dejection, evil,
unprofitable states, might overwhelm one who dwells with the faculty of
the eye uncontrolled, do you apply yourselves to such control, set a
guard over the faculty of the eye and attain control of it. Hearing a
sound with the ear\ldots with the nose smelling a scent\ldots with the tongue
tasting a savour\ldots with the body contacting tangibles\ldots with the mind
cognizing mind-states\ldots be you not misled by their outward
appearance nor by their lesser details\ldots attain control thereof\ldots' '' 
\end{flushleft}\end{quote}




We then read that King Udena praised the Buddha's words. He said about
his own experiences:




\begin{quote}\begin{flushleft}
``I myself, master Bh{\=a}radv{\=a}ja, whenever I enter my palace with
body, speech and mind unguarded, with thought unsettled, with my
faculties uncontrolled, at such times lustful states overwhelm me.
But whenever, master Bh{\=a}radv{\=a}ja, I do so with body, speech and
mind guarded, with thought settled, with my faculties controlled, at
such times lustful states do not overwhelm me\ldots''
\end{flushleft}\end{quote}




We read that King Udena took his refuge in the Buddha, the Dhamma and
the Sangha. How can we avoid being misled by the outward appearance or by the
details of phenomena? By understanding realities as they are when they
appear, one at a time. The following sutta in the ``Kindred
Sayings''(IV, Sa\d l{\=a}yatanavagga, Second Fifty, Ch 3, {\S}82, The
World) reminds us not to cling to a ``whole'' but to be mindful of only
one object at a time as it appears through one of the six doors:



\begin{quotation}\begin{flushleft}

``Then a certain monk came to see the Exalted One\ldots Seated at one side
that monk said to the Exalted One: `The world! The world! is the
saying, lord. How far, lord, does this saying go?'

`It crumbles away, monks. Therefore it is called the world'
\footnote{In P{\=a}li there is a word association of loko, world,
with lujjati, to crumble away.}. What crumbles away? The eye \ldots
objects\ldots eye-consciousness\ldots eye-contact\ldots that pleasant or
unpleasant or neutral feeling that arises owing to eye-contact \ldots
tongue \ldots body \ldots mind \ldots It crumbles away, monks. Therefore it is called the world'.''
\end{flushleft}\end{quotation}






\subsection*{Questions}




\begin{enumerate}
\item Can eyesense experience something?

\item Where is the bodysense?

\item Is eyesense all the time eye-door?

\item For which type of citta is eyesense eye-door as well\\ as base
(vatthu, physical place of origin)? 

\end{enumerate}

















\chapter[Sense Objects]{} 
\section*{Sense Objects}



We are infatuated with all the objects which are experienced through the
sense-doors. However, they are only r\=upas that fall away
immediately; we cannot possess them. Sometimes we experience pleasant
objects and sometimes unpleasant objects. The experience of a pleasant
object is the result of kusala kamma and the experience of an
unpleasant object is the result of akusala kamma. 

The objects which can be experienced through the sense-doors are the
following:




\begin{description}
\item colour or visible object

\item sound

\item odour

\item flavour 

\item tangible object
\end{description}




As we have seen in Chapter 2, three of the four Great Elements can be
tangible object, namely: solidity (appearing as hardness or softness),
temperature (appearing as heat or cold) and motion (appearing as
motion, oscillation or pressure). The element of cohesion is not
\ tangible object, it can be experienced only through the mind-door. 

The sense objects which are visible object, odour and flavour are
included in the ``eight inseparable r\=upas'' which always arise
together. As we have seen, r\=upas arise in groups and with each group
there have to be the eight inseparable r\=upas which are the four Great
Elements, visible object, odour, flavour and nutritive essence.
Although these r\=upas arise together, only one kind of r\=upa at a
time can be the object that is experienced. When there are conditions
for the experience, for example, of flavour, the flavour that impinges
on the tastingsense is experienced by tasting-consciousness. Flavour
arises together with the other seven inseparable r\=upas but these are
not experienced at that moment. 

Sound is the object of hearing-consciousness. Sound is not included in
the eight inseparable r\=upas, but when it arises it has to be
accompanied by these r\=upas that each perform their own function.
Whenever sound occurs, there also have to be solidity, cohesion,
temperature, motion and the other four inseparable r\=upas. When sound
is heard, the accompanying r\=upas cannot be experienced.
\footnote{Because each citta can experience only
one object at a time through the appropriate doorway.}

We read in the ``Dhamma\-sanga\d ni'' ({\S}621) about different kinds of
sounds, such as sound of drums and other musical instruments, sound of
singing, noise of people, sound of  substance against substance,
sound of wind or water, human sound, such as sound of people talking.
The ``Atthas{\=a}lin\=\i'' (II, Book II, Part I, Ch III, 319), which
gives a further explanation of these kinds of sounds, defines sound as
follows\footnote{See also Visuddhimagga XIV, 55.}: 




\begin{quote}\begin{flushleft}
\ldots all sounds have the characteristic of striking the ear, the function and property of being the object of auditory cognition, the manifestation of being the field or object of auditory cognition\ldots
\end{flushleft}\end{quote}




Like the other sense objects, sound has as its proximate cause the four
Great Elements. No matter what sound we hear, it has a degree of
loudness and it ``strikes the ear''. Its characteristic can be
experienced without the need to think about it. We may hear the sound
of a bird and it seems that we know at once the origin of the sound.
When we know the origin of the sound it is not hearing, but thinking of
a concept. However, the thinking is conditioned by the hearing. 

It seems that we can hear different sounds at a time, for example when a
chord is played on the piano. When we recognize the different notes
of a chord it is not hearing but thinking. When awareness arises, one
reality at a time can be known as it is. 

Sound can be produced by temperature or by citta. Sound of wind or sound
of water is produced by temperature. Speech sound is produced by citta.


We are inclined to find a loud noise disturbing and we may make
ourselves believe that at such a moment mindfulness of realities cannot
arise. We read in the ``Therag{\=a}th{\=a}'' (Psalms of the Brothers,
Part VII, Canto 62, Vajjiputta) about a monk of the Vajjian clan who
was dwelling in a wood near Ves{\=a}l\=\i. The commentary to this verse
(Paramatthad\=\i pan\=\i) states:




\begin{quote}\begin{flushleft}
``\ldots Now a festival took place at Ves{\=a}l\=\i, and there was dancing, singing and reciting, all the people happily enjoying the festival. And
the sound thereof distracted the bhikkhu, so that he quitted his
solitude, gave up his exercise, and showed forth his disgust in this
verse:
\end{flushleft}\end{quote}




\begin{verse}
Each by himself we in the forest dwell,

Like logs rejected by the woodman's craft.

So flit the days one like another by,

Who more unlucky in their lot than we?
\end{verse}




\begin{quote}\begin{flushleft} Now a woodland deva heard him, and had compassion for the bhikkhu, and
thus upbraided him, `Even though you, bhikkhu, speak scornfully of
forest life, the wise desiring solitude think much of it,' and to show
him the advantage of it spoke this verse:
\end{flushleft}\end{quote}




\begin{verse}
Each by himself we in the forest dwell,

Like logs rejected by the woodman's craft.

And many a one does envy me my lot,

Even as the hell-bound envies him who fares to heaven.

\end{verse}



\begin{quote}\begin{flushleft}
Then the bhikkhu, stirred like a thoroughbred horse by the spur, went
down into the avenue of insight, and striving soon won arahatship.
Thereupon he thought, `The deva's verse has been my goad!' and he
recited it himself.''
\end{flushleft}\end{quote}




By this Sutta we are reminded that aversion to noise is not helpful.
Mindfulness can arise of whatever reality presents itself. When sound
appears, correct understanding of this reality can be developed. It can
be known as a kind of r\=upa and it does not matter what kind of sound
it is. We are infatuated with pleasant sense objects and disturbed by
unpleasant ones. Like and dislike are realities of daily life and they
can be objects of awareness. We often find reasons why we cannot be
mindful of the reality appearing at the present moment.

We would like to hear only pleasant things. When someone speaks
unpleasant words to us we are inclined to think about this for a long
time instead of being mindful of realities. We may forget that the
moment of hearing is vip{\=a}kacitta, result produced by kamma. Nobody
can change vip{\=a}ka. Hearing falls away immediately. When we think
with aversion about the meaning of the words that were spoken, we
accumulate unwholesomeness.

We read in the ``Greater Discourse of the Elephant's Footprint'' (Middle
Length Sayings I, 28) that S{\=a}riputta spoke to the monks about the
elements that are conditioned, impermanent and devoid of self. He also
spoke about the hearing of unpleasant words:


\begin{quote}\begin{flushleft}
`` \ldots Your reverences, if others abuse, revile, annoy, vex this monk, he
comprehends: `This painful feeling that has arisen in me is born of
sensory impingement on the ear, it has a cause, not no cause. What is
the cause? Sensory impingement is the cause.' He sees that sensory
impingement is impermanent, he sees that feeling \ldots perception \ldots the
habitual tendencies (sa\.nkh{\=a}rakkhandha) are impermanent, he sees
that consciousness is impermanent
\footnote{This sutta refers to the five khandhas.
Conditioned n{\=a}mas and r\=upas can be classified as five khandhas or
aggregates: r\=upakkhandha (comprising all r\=upas), vedan{\=a}kkhandha
or the khandha of feelings, sa\~n\~n{\=a}kkhandha, the khandha of
perception or remembrance, sa\.nkh{\=a}rakkhandha, the khandha of
``habitual tendencies'' or ``formations'', including all cetasikas
other than feeling and perception, vi\~n\~n{\=a}\d nakkhandha, including
all cittas.}. His mind rejoices, is pleased, composed, and is set on
the objects of the element. If, your reverences, others comport
themselves in undesirable, disagreeable, unpleasant ways towards that
monk, and he receives blows from their hands and from clods of earth
and from sticks and weapons, he comprehends thus: `This body is such
that blows from hands affect it and blows from clods of earth affect it
and blows from sticks affect it and blows from weapons affect it. But
this was said by the Lord in the Parable of the Saw: If, monks,
low-down thieves should carve you limb from limb with a two-handled
saw, whoever sets his heart at enmity, he, for this reason, is not a
doer of my teaching. Unsluggish energy shall come to be stirred up by
me, unmuddled mindfulness set up, the body tranquillised, impassible,
the mind composed and onepointed. Now, willingly, let blows from hands
affect this body, let blows from clods of earth \ldots from sticks \ldots from
weapons affect it, for this teaching of the Awakened Ones is being
done.' ''
\end{flushleft}\end{quote}




Do we see our experiences as elements to such a degree already that,
when we hear unpleasant words, we can immediately realize: ``This
painful feeling that has arisen in me is born of sensory impingement on
the ear''? In order to see realities as they are it is necessary to
develop understanding of n{\=a}ma and r\=upa. 

There are different ways of classifying r\=upas. One way is the
classification as the four Great Elements (mah{\=a}-bh\=uta r\=upas)
and the derived r\=upas (up{\=a}da r\=upas), which are the other
twentyfour r\=upas among the twentyeight r\=upas.

Another way is the classification as gross r\=upas (o\d l{\=a}rika
r\=upas) and subtle r\=upas (sukhuma r\=upas). Twelve kinds of r\=upa
are gross; they are the sense-objects that can be experienced through
the sense-doors, namely: visible object, sound, odour, flavour and
the three r\=upas that are tangible object, namely: solidity,
temperature and motion, thus, three of the great Elements, and also the
five sense-organs (pas{\=a}da r\=upas) that can be the doors through
which these objects are experienced. The other sixteen r\=upas among
the twentyeight kinds are subtle r\=upas. 

The ``Visuddhimagga'' (XIV, 73) states that twelve r\=upas ``are to be
taken as gross because of impinging; the rest is subtle because they
are the opposite of that.'' The seven r\=upas that can be sense
objects\footnote{They are visible object, sound, odour, flavour and
three tangible objects which are three among the Great Elements. } are
impinging time and again on the five r\=upas which are the sense
organs. Subtle r\=upas do not impinge on the senses. According to the
``Visuddhimagga'', the subtle r\=upas are far, because they are
difficult to penetrate, whereas the gross r\=upas are near, because
they are easy to penetrate. 

Objects impinge on the senses time and again, but we are usually
forgetful of realities. We have learnt about the four Great Elements
and other r\=upas and we may begin to notice different characteristics
of realities when they present themselves. For example, when we are
walking, r\=upas such as hardness, heat or pressure may appear one at a
time. We can learn the difference between the moments when
characteristics of realities appear one at a time and when we are
thinking of concepts such as feet and ground. The ground cannot impinge
on the bodysense and be directly experienced. The Buddha urged the
monks to develop right understanding during all their actions. We read
in the commentary to the ``Satipa\d t\d th{\=a}na Sutta''\footnote{In
the Middle Length Sayings I, no 10. See the translation of the
commentary to this sutta in ``The Way of Mindfulness'' by Ven. Soma,
B.P.S. Kandy, 1975.}, in the section on the four kinds of Clear
Comprehension, about clear comprehension in wearing robes:


\begin{quote}\begin{flushleft}
`` \ldots Within there is nothing called a soul that robes itself. According
to the method of exposition adopted already, only, by the diffusion of
the process of oscillation (the element of wind or motion) born of
mental activity does the act of robing take place. The robe has no
power to think and the body too has not that power. The robe is not
aware of the fact that it is draping the body, and the body too of
itself does not think: `I am being draped round with the robe.' Mere
processes clothe a process-heap, in the same way that a modelled
figure is covered with a piece of cloth. Therefore, there is neither
room for elation on getting a fine robe nor for depression on getting
one that is not fine.''
\end{flushleft}\end{quote}


This passage is a good reminder of the truth, also for laypeople. We are
used to the impact of clothes on the body, most of the time we do not
even notice it. Or we are taken in by the pleasantness of soft material
that touches the body, or by the colour of our clothes. We can be
mindful of softness or colour as only elements. In reality there are
only elements impinging on elements. 

We read in the ``Gradual Sayings'' (II, Book of the Fours, Ch XVIII,
{\S}7, R{\=a}hula) that the Buddha said to R{\=a}hula:

\begin{quote}\begin{flushleft}

``R{\=a}hula, what is the inward earth-element and what is the
external earth-element, these are just this earth-element. Thus it
should be regarded, as it really is, by perfect wisdom: `This is not of
me. Not this am I. Not to me is this the self.' So seeing it, as it
really is, by perfect wisdom, one has revulsion for the
earth-element; by wisdom one cleanses the heart of passion.''

\end{flushleft}\end{quote}



The same is said of the elements of water, heat and wind. The Buddha
then said:

\begin{quote}\begin{flushleft}
``Now, R{\=a}hula, when a monk beholds neither the self nor what
pertains to the self in these four elements, this one is called `a monk
who has cut off craving, has loosed the bond, and by perfectly
understanding (this) vain conceit, has made an end of Ill.' ''
\end{flushleft}\end{quote}














\subsection*{Questions} 




\begin{enumerate}
\item Which factors can produce sound?

\item When someone speaks, by which factor is sound produced?

\item Why are gross r\=upas so called?

\item Which r\=upas among the inseparable r\=upas are gross?

\item Through which doorways can gross r\=upas be known?

\end{enumerate}



















































\chapter[Subtle R\=upas and Kamma]{}
\section*{Subtle R\=upas produced by Kamma}



The objects that can be experienced through the sense-doors as well as
the sense-organs themselves are gross r\=upas, the other r\=upas are
subtle r\=upas. As we have seen, seven r\=upas are sense objects,
namely, colour or visible object, sound, odour, flavour and tangible
object including three of the four Great Elements which are solidity,
temperature and motion. Five r\=upas are sense-organs, namely,
eyesense, earsense, nosesense, tonguesense and bodysense. The sense
objects impinge on the relevant senses so that seeing, hearing and the
other sense-cognitions can arise time and again in daily life. 

Among the twentyeight kinds of r\=upas, the sense objects and the
sense-organs are twelve r\=upas which are gross, whereas the other
sixteen r\=upas are subtle r\=upas.


The sense-organs are produced solely by kamma, not by the other
three factors of citta, temperature and nutrition which can produce
r\=upas. There are also subtle r\=upas which are produced solely by
kamma. They are: the femininity-faculty, the masculinity-faculty,
the life-faculty and the heart-base.

With regard to the femininity-faculty (itthindriya\d m) and the
mascu\-linity-faculty (purisindriya\d m), collectively called
bh\-{\=a}\-var\=upa or sex, these are r\=upas produced by kamma from the
first moment of our life and throughout life. Thus, it is due to kamma
whether one is born as a male or as a female. The ``Atthas{\=a}lin\=\i''
(II, Book II, Part I, Ch III, 322) explains that birth as a male and
birth as a female are different kinds of vip{\=a}ka. Being born as a
human being is kusala vip{\=a}ka, but since good deeds have different
degrees also their results have different degrees. Birth as a female is
the result of kusala kamma of a lesser degree than the kusala kamma
\ that conditions birth as a male. In the course of life one can notice
the difference between the status of men and that of women. It is a
fact that in society generally men are esteemed higher than women.
Usually women cannot so easily obtain a position of honour in society.
But as regards the development of wisdom, both men and women can
develop it and attain arahatship. We read in the ``Kindred Sayings''
(IV, Sa\d l{\=a}yatana-vagga, Part III, Kindred Sayings about
Womankind, 3, {\S}34, Growth):

\begin{quote}\begin{flushleft}
``Increasing in five growths, monks, the ariyan woman disciple increases
in the ariyan growth, takes hold of the essential, takes hold of the
better. What five?

She grows in confidence (saddh{\=a}), grows in virtue (s\=\i{} la), in
learning, in generosity, in wisdom. Making such growth, monks, she
takes hold of the essential, she takes hold of the better \ldots''
\end{flushleft}\end{quote}




The ``Atthas{\=a}lin\=\i'' (II, Book II, Ch III, 321) explains that women
and men have different features, that they are different in outer
appearance, in occupation and deportment. The feminine features, etc.
are conditioned by the r\=upa that is the femininity faculty. The
``Atthas{\=a}lin\=\i'' states about these features:




\begin{quote}\begin{flushleft}
`` \ldots They are produced in course of process because of that faculty.
When there is seed the tree grows because of the seed, and is replete
with branch and twig and stands filling the sky; so when there is the
feminine controlling faculty called femininity, feminine features, etc.
, come to be \ldots''
\end{flushleft}\end{quote}




The same is said about the masculinity faculty. 
Femininity and masculinity are ``controlling faculties''. A controlling
faculty or indriya is a ``leader'' in its own field, it has a
predominant influence. The controlling faculties of femininity and
masculinity permeate the whole body so that they are manifested in the
outward appearance and features of a woman and a man.

The ``Atthas{\=a}lin\=\i'' (same section, 322) gives the following
definitions of the femininity faculty and the masculinity faculty:




\begin{quote}\begin{flushleft}
``Of these two controlling faculties the feminine has the characteristic
of (knowing) the state of woman, the function of showing ``this is
woman'', the manifestation which is the cause of femininity in feature,
mark, occupation, deportment. 

The masculinity controlling faculty has the characteristic of (knowing)
the state of man, the function of showing ``this is man'', the
manifestation which is the cause of masculinity in feature, etc.
\footnote{See also Dhamma\-sanga\d ni {\S}633, 634 and Visuddhimagga
XIV, 58.}'' 
\end{flushleft}\end{quote}




These two faculties which, as the Visuddhimagga (XIV, 58) explains, are
``coextensive with'' or pervade the whole body, are not known by visual
cognition but only by mind-cognition. But, as the
``Atthas{\=a}lin\=\i'' (321) states, their characteristic
features, etc., which are conditioned by their respective faculties,
can be known by visual cognition as well as by mind-cognition. 

Seeing experiences only visible object, it does not know ``This is a
woman'' or ``This is a man''. The citta which recognizes feminine or
masculine features does so through the mind-door, but this
recognizing is conditioned by seeing. When the commentary states that
these characteristic features are known by visual cognition as
well as by mind-cognition, it does not speak in detail about the
different processes of cittas experiencing objects through the
eye-door and through the mind-door.

Generally, women like to emphasize their feminin\-ity in ma\-ke up and
clothes and also men like to emphasize their masculinity in their
outward appearance and behaviour. One clings to one's feminine or
masculine features, one's way of deportment. We should not forget that
it is the femininity faculty or masculinity faculty, only a r\=upa
produced by kamma, which conditions our outward appearance or
deportment to be specifically feminine or masculine. We take our sex
for self, but it is only a conditioned element devoid of self.




Life faculty, the r\=upa which is j\=\i vitindriya, is also a subtle
r\=upa produced by kamma from the first moment of life and throughout
life. 

Apart from r\=upa-j\=\i vitindriya there is also
n{\=a}ma-j\=\i vitindriya. N{\=a}ma-j\=\i vitindriya is a cetasika
among the ``universals'', cetasikas which accompany every citta. This
cetasika supports the citta and the cetasikas it arises together with,
it maintains their life. 

The r\=upa that is life faculty, r\=upa-j\=\i vitindriya, sustains and
maintains the r\=upas it accompanies in one group of r\=upas. This kind
of r\=upa is produced solely by kamma, it arises only in living beings.
Therefore, the r\=upas in the bodies of living beings are different
from those in dead matter or plants which are produced solely by
temperature or the element of heat. The r\=upa that is life faculty is
contained in each group of r\=upas of the body produced by kamma. 

Life faculty is a ``controlling faculty'' (indriya), it has a dominating
influence over the other r\=upas it arises together with since it
maintains their life. The ``Visuddhimagga'' (XIV, 59) states about life
faculty:\footnote{See also Dhamma\-sanga\d ni {\S}635.
The Atthas{\=a}lin\=\i{}  refers to its definition of
n{\=a}ma-j\=\i vitindriya (I, Part IV, Ch I, 123, 124)} 




\begin{quote}\begin{flushleft}
``The life faculty has the characteristic of maintaining conascent kinds
of matter.\footnote{The r\=upas arising together
with it.} Its function is to make them occur. It is manifested in the
establishing of their presence. Its proximate cause is
primary elements that are to be sustained.''
\end{flushleft}\end{quote}




Life faculty maintains the other r\=upas it arises together with in one
group, and then it falls away together with them. The ``Visuddhimagga''
(in the same section) states:




\begin{quote}\begin{flushleft}
``It does not prolong presence at the moment of dissolution because it
is itself dissolving, like the flame of a lamp when the wick and the
oil are getting used up\ldots''

\end{flushleft}\end{quote}



We cling to our body as something alive. R\=upas of a ``living body''
have a quality lacking in dead matter or plants, they are supported by
the life faculty. We are inclined to take this quality for ``self'',
but it is only a r\=upa produced by kamma.

The heart-base (hadayavatthu) is another r\=upa produced solely by
kamma. In the planes of existence where there are n{\=a}ma and r\=upa,
cittas have a physical place of origin, a base (vatthu).
Seeing-consciousness has as its base the eye-base, the r\=upa which
is eyesense, and evenso have the other sense-cognitions their
appropriate bases where they arise. Apart from the sense-bases there
is another base: the heart-base. This is the place of origin for all
cittas other than the sense-cognitions.

At the first moment of life the rebirth-consciousness
(pa\d tisan\-dhi-citta) which arises is produced by kamma. If this citta
arises in a plane of existence where there are n{\=a}ma and r\=upa it
must have a physical base: this is the heart-base, which is produced
by kamma. Kamma produces this r\=upa from the first moment of life and
throughout life.

The r\=upa which is the heart-base has not been classified as such in
the ``Dhamma\-sanga\d ni'', but it is referred to as ``this r\=upa'' in
the ``Book of Conditional Relations'' (Pa\d t\d th{\=a}na), the Seventh
Book of the Abhidhamma. In the section on ``Dependance Condition''
(Part II, Analytical Exposition of Conditions) it is said that
dependant on the five sense-bases the five sense-cognitions arise
and that dependant on ``this matter'' mind-element and
mind-consciousness-element arise. ``This matter'' is the r\=upa
which is the heart-base; the mind-element and
mind-consciousness-element comprise all cittas other than the five
sense-cognitions.\footnote{Mind-element are the
five-sense-door adverting-consciousness and the two types of
receiving-consciousness, which are kusala vip{\=a}ka and akusala
vip{\=a}ka. Mind-consciousness-element are all cittas other than
the sense-cognitions and mind-element.} The sense-cognitions of
seeing, etc. have the appropriate sense-base as physical base, and
all other cittas have the heart-base as physical base. 

The ``Visuddhimagga'' (XIV, 60) gives the following definition of the
heart-base\footnote{The Atthas{\=a}lin\=\i{}  does
not classify the heart-base separately, but it mentions the
``basis-decad'', a group of ten r\=upas including the heart-base (
Book II, Ch III, 316). As I shall explain later on, from the first
moment of our life kamma produces three decads, groups of ten r\=upas:
the bodysense-decad, the sex-decad and the heart-base-decad.}:





\begin{quote}\begin{flushleft}
``The heart-basis has the characteristic of being the (material)
support for the mind-element and for the
mind-consciousness-element. Its function is to support them. It is
manifested as the carrying of them \ldots''
\end{flushleft}\end{quote}




The ``Visuddhimagga'' (VIII, 111,112) states that the heart-base is to
be found inside the heart. It is of no use to speculate where exactly
the heart-base is. It is sufficient to know that there is a r\=upa
which is base for all cittas other than the sense-cognitions. We may
not experience the heart-base as such, but if there would be no
heart-base we could not think at this moment, we could not know which
objects we are experiencing, we could not feel happy or unhappy. In the
planes of existence where there are n{\=a}ma and r\=upa all cittas must
have a physical base, they cannot arise outside the body. When we, for
example, are angry, cittas rooted in aversion arise and these originate
at the heart-base.

If we had not studied the Abhidhamma we would have thought that all
cittas originate in what we call in conventional language ``brain''.
One may cling to a concept of brain and take it for self. The
Abhidhamma can clear up misunderstandings about bodily phenomena and
mental phenomena and the way they function. It explains how physical
phenomena and mental phenomena are interrelated. Mental phenomena are
dependant on physical phenomena\footnote{In the
planes of existence where there are n{\=a}ma and r\=upa. } and
physical phenomena can have mental phenomena as conditioning factors.

The conditioning factors for what we call body and mind are impermanent.
Why then do we take body and mind for something permanent? We read in
the ``Kindred Sayings'' (III, Khandh{\=a}-vagga, Kindred Sayings on
Elements, First Fifty, Ch 2, {\S}18, Cause) that the Buddha said to
the monks at S{\=a}vatth\=\i:




\begin{quote}\begin{flushleft}
``Body, monks is impermanent. That which is the cause, that which is the
condition for the arising of body, that also is impermanent. How,
monks, can a body which is compounded of the impermanent come to be
permanent? \ldots''
\end{flushleft}\end{quote}




The same is said about the mental phenomena (classified as four
aggregates or khandhas). We then read:




\begin{quote}\begin{flushleft}
``Thus seeing, the welltaught ariyan disciple\footnote{An ariyan is a
person who has attained enlightenment.} is repelled by body, is
repelled by feeling, by perception, by the `activities'
\footnote{Cetasikas other than feeling and perception are classified
as one khandha, that of the activities or formations,
sa\.nkh{\=a}rakkhandha.}. He is repelled by consciousness. Being
repelled by it he lusts not for it: not lusting he is set free. Thus he
realizes: `Rebirth is destroyed, lived is the righteous life, done is
my task, for life in these conditions there is no here-after.' ''
\end{flushleft}\end{quote}
























\subsection*{Questions}




\begin{enumerate}
\item Why can life faculty not arise in plants?

\item What is the base for citta rooted in aversion?

\item Does the brain have the function of base for cittas? 

\item What is the base for rebirth-consciousness in the human plane of 
existence?

\end{enumerate}
















\chapter[Intimation]{}
\section*{Intimation through Body and Speech}




Citta is one of the four factors that produces r\=upa. We look different
when we laugh, when we cry, when we are angry or when we are generous.
Then we can notice that citta produces r\=upa.

Bodily intimation (k{\=a}yavi\~n\~natti) and speech intimation
(vac\=\i{} vi\~n\-\~natti) are two kinds of r\=upa, originated by citta. They
are not produced by the other three factors that can produce r\=upa, by
kamma, temperature or nutrition.

As to bodily intimation, this is a specific way of expression by r\=upas
of the body that display our intentions, be they wholesome or
unwholesome. Our intentions can be expressed by way of movement of the
body, of the limbs, facial movement or gestures. The intention
expressed through bodily intimation can be understood by others, even
by animals. Bodily intimation itself is r\=upa, it does not know
anything. We read in the ``Dhamma\-sanga\d ni'' ({\S}636):




\begin{quote}\begin{flushleft}
``What is that r\=upa which is bodily intimation (k{\=a}yavi\~n\~natti)?


That tension, that intentness, that state of making the body tense, in
response to a thought, whether good or bad, or indeterminate
(kiriyacitta), on the part of one who advances, or recedes, or fixes
the gaze, or glances around, or retracts an arm, or stretches it forth
- the intimation, the making known, the state of having made known
- this is that r\=upa which constitutes bodily intimation.''

\end{flushleft}\end{quote}



Citta is one of the factors that produces groups of the ``eight
inseparable r\=upas'' of the body\footnote{The four Great Elements
of solidity, cohesion, temperature and motion, and visible object,
odour, flavour and nutrition.} and among them the element of wind or
motion plays its specific part in causing motion of r\=upas of the body
so that intimation can be displayed. 

The ``Atthas{\=a}lin\=\i'' (I, Book I, Part III, 82, 83) states about
bodily intimation:

\begin{quote}\begin{flushleft}
``Because it is a capacity of communicating, it is called
``intimation''. What does it communicate? A certain wish communicable
by an act of the body. If anyone stands in the path of the eye, raises
his hands or feet, shakes his head or brow, the movement of his hands,
etc. are visible. Intimation, however, is not so visible; it is only
knowable by the mind. For one sees by the eye a colour-surface moving
by virtue of the change of position in hands, etc.\footnote{Because
of sa\~n\~n{\=a}, remembrance, one can perceive the movement of a
colour surface. Seeing sees only colour, it cannot see movement of
colour.} But by reflecting on it as intimation, one knows it by
mind-door-consciousness, thus: `Imagine that this man wishes me
to do this or that act.' \ldots''
\end{flushleft}\end{quote}




The intention expressed through bodily intimation is intelligible to
others, not through the eye-door but through the mind-door.
Knowing, for example, that someone waves is cognition through the
mind-door and this cognition is conditioned by seeing-consciousness
that experiences visible object or colour. The meaning of what has been
intimated is known by reflection on it, thus it can only be cognized
through the mind-door. 

The ``Visuddhimagga'' (XIV, 61) defines intimation in a similar way and
then states about its function, manifestation and proximate cause:




\begin{quote}\begin{flushleft}
`` \ldots Its function is to display intention. It is manifested as the
cause of bodily excitement. Its proximate cause is the
consciousness-originated air- element.''
\end{flushleft}\end{quote}




As to the proximate cause, as we have seen, the element of wind (air) or
motion plays its specific role in the intimating of intention by bodily
movement or gestures. 

We are inclined to take intimation as belonging to self, but bodily
intimation is only a kind of r\=upa, originated by citta. There is no
person who communicates by gestures. Are we aware of n{\=a}ma and
r\=upa when we gesticulate? Are there kusala cittas or akusala cittas
at such moments? Most of the time akusala cittas arise, but we do not
notice it. Do we realize which type of citta conditions the bodily
intimation when we wave to someone else in order to greet him, when we
gesticulate in order to tell him to come nearer, when we nod our head
while we agree with something or shake it while we deny something? Such
gestures are part of our daily routine and it seems that we make them
automatically. Perhaps we never considered what types of citta
condition them. Akusala citta conditions bodily intimation, for
example, when we with mimics ridicule someone else or show our contempt
for him. In such cases it is obvious that there is akusala citta. We
should remember that bodily intimation is more often conditioned by
akusala citta than by kusala citta. There may be subtle clinging that
is not so obvious while we are expressing our intention by gestures.
When mindfulness arises we can find out whether kusala citta or akusala
citta motivates our gestures. Someone may also commit akusala kamma
through bodily intimation, for example when he gives by
gesture orders to kill. Kusala cittas may condition bodily
intimation when we, for example, stretch out our arms to welcome people
to our home, when we stretch out our hand in order to give something,
when we point out the way to someone who is in a strange city, when we
by our gestures express courtesy or when we show respect to someone who
deserves respect. However, we may also perform such actions because of
selfish motives, or we may be insincere, and then akusala cittas
condition bodily intimation. More knowledge about citta and r\=upas
which are conditioned by citta can remind us to be aware of whatever
reality appears, also while gesticulating. Then akusala citta has no
opportunity to arise at such moments. 

Our intentions are not only communicated by gestures, but also by
speech. Speech intimation (vac\=\i{} vi\~n\~natti) is a kind of r\=upa,
originated by citta. The ``Dhamma\-sanga\d ni'' ( Ch II, {\S}637) states:





\begin{quote}\begin{flushleft}
``What is that r\=upa which is intimation by language
(vac\=\i{} vi\~n\~natti)?

That speech, voice, enunciation, utterance, noise, making noises,
language as articulate speech, which expresses a thought whether good,
bad, or indeterminate - this is called language. And that intimation,
that making known, the state of having made known by language - this
is that r\=upa which constitutes intimation by language.''

\end{flushleft}\end{quote}



When someone's intention is intimated through speech, it is intelligible
to others. The meaning of what is intimated is known by reflection
about it, thus, it is cognizable through the mind-door. Speech
intimation itself does not know anything, it is r\=upa. 

Citta is one of the factors that produces the ``eight inseparable
r\=upas'' of the body and among them the element of earth or hardness
plays a specific part in the conditioning of speech intimation. 

The ``Visuddhimagga'' (XIV, 62)\footnote{See Atthas{\=a}lin\=\i{}  I,
Book I, Part III, Ch 2, 86,87, and II, Book II, Part I, Ch 3, 324.}
states that the function of speech intimation is to display intention,
its manifestation is causing speech sound, and that its proximate
cause is the earth element originated from citta. 
The proximate cause of bodily intimation is the element of wind or
motion which is produced by citta, whereas the proximate cause of
speech intimation is the element of earth or hardness which is produced
by citta. When speech-intimation occurs it is the condition for the
r\=upas which are the means of articulation, such as r\=upas of the
lips, to produce speech sound. 

R\=upas can be classified as sa\=bh{\=a}va r\=upas, r\=upas with their own
distinct nature (sa meaning: with, bh{\=a}va meaning: nature) and
asa\=bh\-{\=a}va r\=upas, r\=upas without their own distinct nature. The
eight inseparable r\=upas are sabh{\=a}va r\=upas, they each have their
own distinct nature and characteristic. Bodily intimation and speech
intimation are r\=upas conditioned by citta, but these two kinds of
r\=upa are not r\=upas with their own distinct nature and
characteristic. They are, as the ``Atthas{\=a}lin\=\i'' expresses it, a
``certain, unique change'' in the great elements which are produced by
citta and which are the condition for the two kinds of intimation. They
are qualities of r\=upa and therefore, asabh\-{\=a}va r\=upas. The eight
inseparable r\=upas on which the two kinds of intimation depend are
produced by citta, according to the ``Atthas{\=a}lin\=\i'' (II, Book II,
Part I, Ch 3, 337).

Do we realize whether speech intimation is conditioned by kusala citta
or by akusala citta? We may know in theory that we speak with akusala
citta when our objective is not wholesomeness, such as generosity,
kindness or the development of understanding of the Buddha's teachings,
but do we realize this at the moments we speak? Even when akusala kamma
through speech, such as lying or slandering, is not committed, we may
still speak with akusala citta. We may find out that often our speech
is motivated by akusala citta. We speak with cittas rooted in
attachment when we want to gain something, when we want to be liked or
admired by others. With this objective we may even tell ``tales'' about
others, ridicule or denigrate them. We are attached to speech and we
often chatter just in order to keep the conversation going. We tend to
feel lonely when there is silence. Usually we do not consider whether
what we say is beneficial or not. We have to speak to others when we
organize our work in the office or at home. Do we realize whether there
are at such moments kusala cittas or akusala cittas? When we lie we
commit akusala kamma through speech.

Speech intimation is produced by kusala citta when we, for example, with
generosity and kindness try to help and encourage others in speaking to
them. When we speak about the Buddha's teachings there may be kusala
cittas, but at times there also tend to be akusala cittas, for example,
when we are conceited about our knowledge, or when we are attached to
the people we are speaking to. Many different types of citta arise and
fall away very rapidly and we may not know when the citta is kusala
citta and when akusala citta. There can be mindfulness of n{\=a}ma and
r\=upa while speaking. One may believe that this is not possible
because one has to think of the words one wants to speak. However,
thinking is a reality and it can also be object of mindfulness. There
are sound and hearing and they can be object of mindfulness when they
appear. We are usually absorbed in the subject we want to speak about
and we attach great importance to our speech. We live most of the time
in the world of ``conventional truth'', and we are forgetful of
ultimate realities (paramattha dhammas). In the ultimate sense there is
no speaker, only phenomena, devoid of self, conditioned n{\=a}mas and
r\=upas. 

When we gesticulate and speak, hardness, pressure, sound or hearing may
present themselves, they can be experienced one at a time. If there is
mindfulness at such moments, understanding of the reality that appears
can be developed. 

The ``Visuddhimagga'' (XVIII, 31) uses a simile of a marionette in order
to illustrate that there is no human being in the ultimate sense, only
conditioned phenomena. We read:




\begin{quote}\begin{flushleft}
``Therefore, just as a marionette is void, soulless and without
curiosity, and while it walks and stands merely through the combination
of strings and wood, yet it seems as if it had curiosity and
interestedness, so too, this mentality-materiality is void, soulless
and without curiosity, and while it walks and stands merely through the
combination of the two together, yet it seems as if it had curiosity
and interestedness. This is how it should be regarded. Hence the
Ancients said:
\end{flushleft}\end{quote}

\begin{verse}
`The mental and material are really here,

`But here there is no human being to be found,

`For it is void and merely fashioned like a doll--

`Just suffering piled up like grass and sticks.''

\end{verse}



When one sees a performance with marionettes, it seems that the puppets
have lives of their own: they exert themselves, they are absorbed,
attached or full of hatred and sorrow, and one can laugh and cry
because of the story that is being enacted. However, the puppets are
only wood and strings, held by men who make them act. When one sees how
the puppets are stored after the play they are not impressive anymore,
only pieces of wood and strings. When we study the Abhidhamma it helps
us to understand more that this marionette we call ``self'' can move
about, act and speak because of the appropriate conditions.

As we have seen in the definitions of the two kinds of intimation by the
`Dhamma\-sanga\d ni'' ({\S}636, 637), these two kinds of r\=upa can be
conditioned by kusala citta, akusala citta or ``inoperative'' citta
(kiriyacitta). When we realize that intimation through body and speech
is very often conditioned by akusala citta, we come to see the danger
of being forgetful of n{\=a}ma and r\=upa while we make gestures and
speak. Then we are urged to remember the Buddha's words as to the
practice of ``clear comprehension'' (sampaja\~n\~na) in the
``Satipa\d t\d th{\=a}na Sutta'' (Middle Length Sayings no. 10, in the
section on Mindfulness of the Body, dealing with the four kinds of
clear comprehension\footnote{See the translation in ``The Way of
Mindfulness'' by Ven. Soma.}) :




\begin{quote}\begin{flushleft}
``And further, bhikkhus, a bhikkhu, in going forwards (and) in going
backwards, is a person practising clear comprehension; in looking
straight on (and) in looking away from the front, is a person
practising clear comprehension; in bending and in stretching, is a
person practising clear comprehension; in wearing the shoulder-cloak,
the (other two) robes (and) the bowl, is a person practising clear
comprehension; in regard to what is eaten, drunk, chewed and savoured,
is a person practising clear comprehension; in defecating and in
urinating, is a person practising clear comprehension; in walking, in
standing (in a place), in sitting (in some position), in sleeping, in
waking, in speaking and in keeping silence, is a person practising
clear comprehension.''
\end{flushleft}\end{quote}






\subsection*{Questions}

\begin{enumerate}
\item Can bodily intimation be the body-door through which a 
good deed or an evil deed is being performed?


\item Through which door can what is being intimated by bodily
movement be recognized?

\item When a conductor conducts an orchestra and he makes gestures
in order to show the musicians how to play the music, which
types of citta can produce the bodily intimation?

\item When one slanders, which type of r\=upa is the door through
which such action is being performed?

\item When we speak to others in order to organize our work, can
speech be conditioned by akusala citta?

\end{enumerate}










\chapter[R\=upas from different Factors]{}
\section*{R\=upas from different Factors}
The study of r\=upas produced by kamma, citta, temperature or nutrition
is beneficial for the understanding of our daily life. When we study
the conditions for our daily experiences and bodily functions, we shall
better understand that our life is only n{\=a}ma and r\=upa. This again
reminds us to be aware so that realities can be known as they are. 

In this human plane of existence experiences through the senses arise
time and again, such as seeing and hearing, and these could not occur
without the body. The sense-cognitions have as their physical places
of origin their appropriate sense-bases (vatthus) and these are
produced by kamma throughout our life. All other cittas have as their
physical base the heart-base (hadaya-vatthu) and this kind of
r\=upa is produced by kamma from the first moment of life. In the
planes of existence where there are n{\=a}ma and r\=upa, citta needs a
physical base, it could not arise without the body. The r\=upa that is
life-faculty (j\=\i vitindriya) is also produced by kamma from the
first moment of life. It supports the other r\=upas of the group of
r\=upas produced by kamma. Moreover, it is due to kamma whether we are
born as a female or as a male. The r\=upas that are the
femininity-faculty (itthindriya\d m) and the masculinity-faculty
(purisindriya\d m) have a great influence on our daily life. They
condition our outward appearance, our behaviour, the way we walk,
stand, sit or lie down, our voice, our occupation, our place and status
in society. All these kinds of r\=upa produced by kamma arise in
groups, that always include the eight inseparable r\=upas and also
life-faculty. 

Some kinds of r\=upa are produced solely by kamma, some are produced
solely by citta, such as bodily intimation (kaya-vi\~n\~natti) and
speech-intimation (vac\=\i{}-vi\~n\~natti). Some kinds of r\=upa can
be produced by kamma, citta, temperature or nutrition. The eight
inseparable r\=upas of solidity, cohesion, temperature, motion, colour,
odour, flavour and nutrition can be produced by either one of the four
factors. If kamma produces them, they always arise together with
life-faculty, and in addition they can arise with other r\=upas
produced by kamma. Citta produces groups of the eight inseparable
r\=upas from the moment the bhavanga-citta (life-continuum) that
succeeds the rebirth-consciousness arises. 

The following three kinds of r\=upa are sometimes produced by citta,
sometimes by temperature, sometimes by nutrition. They are:




\begin{description}
\item buoyancy or lightness (lahut{\=a})

\item plasticity (mudut{\=a})

\item wieldiness (kamma\~n\~nat{\=a})
\end{description}


\.n

Because of lightness, our body is not heavy or sluggish. Because of
plasticity it is pliable, it has elasticity and is not stiff. Because
of wieldiness it has adaptability. For the movement of the body and the
performance of its functions, these three qualities are essential. They
arise in the bodies of living beings, not in dead matter. These three
r\=upas are r\=upas without a distinct nature, asabh{\=a}va r\=upas;
they are qualities of r\=upa, namely, changeability of r\=upa
(vik{\=a}ra r\=upas, vik{\=a}ra meaning change)\footnote{As we have
seen in Ch 6, the two r\=upas of bodily intimation, k{\=a}ya
vi\~n\~natti, and speech intimation, v{\=a}ci vi\~n\~natti, are also
qualities of r\=upa that are changeability of r\=upa, vik{\=a}ra
r\=upas. In some texts bodily intimation and speech intimation are
classified separately as the two r\=upas of intimation, vi\~n\~natti
r\=upas.}. The ``Atthas{\=a}lin\=\i'' (II, Book II, Part I, Ch III,
326) gives the following definitions of these three kinds of r\=upa
\footnote{See also Dhamma\-sanga\d ni {\S}639 - 641 and Visuddhimagga
XIV, 64.} : 




\begin{quote}\begin{flushleft}
`` \ldots buoyancy of matter has non-sluggishness as its characteristic,
removing the heaviness of material objects as its function, quickness
of change as its manifestation, buoyant matter as its proximate cause.

Next `plasticity of matter' has non-rigidity as characteristic,
removing the rigidity of material objects as function, absence of
opposition in all acts due to its own plasticity as manifestation,
plastic matter as proximate cause.

`Wieldiness of matter' has workableness suitable or favorable to bodily
actions as characteristic, removal of non-workableness as function,
non- weakness as manifestation, workable matter as proximate cause.''

\end{flushleft}\end{quote}



The ``Atthas{\=a}lin\=\i'' also states that these three qualities ``do
not abandon each other''. When one of them arises, the others have to
arise as well. They never arise without the eight inseparable r\=upas.
Although the qualities of lightness, plasticity and wieldiness arise
together, they are different from each other. The ``Atthas{\=a}lin\=\i''
(in the same section) explains their differences. Buoyancy is
non-sluggishness and it is like the quick movement of one free from
ailment. Plasticity is plasticity of objects like well-pounded
leather, and it is distinguished by tractability. Wieldiness is
wieldiness of objects like well-polished gold and it is distinguished
by suitableness for all bodily actions. When one is sick, the elements
of the body are disturbed, and the body is sluggish, stiff and without
adaptability. We read in the ``Visuddhimagga'' (VIII, 28) about the
disturbance of the elements:




\begin{quote}\begin{flushleft}
`` \ldots But with the disturbance of the earth element even a strong man's
life can be terminated if his body becomes rigid, or with the
disturbance of one of the elements beginning with water if his body
becomes flaccid and putrifies with a flux of the bowels, etc., or if he
is consumed by a bad fever, or if he suffers a severing of his
limb-joint ligatures.''
\end{flushleft}\end{quote}




When one is healthy, there are conditions for lightness, plasticity and
wieldiness of body. The ``Atthas{\=a}lin\=\i'' states that these three
qualities are not produced by kamma, but that they are produced by
citta, temperature or nutrition. This commentary states (in the same
section, 327):




\begin{quote}\begin{flushleft}
`` \ldots Thus ascetics say, `Today we have agreeable food \ldots today we have
suitable weather \ldots today our mind is one-pointed, our body is light,
plastic and wieldy.' ''

\end{flushleft}\end{quote}



When we have suitable food and the temperature is right we may notice
that we are healthy, that the body is not rigid and that it can move in
a supple way. Not only food and temperature, also kusala citta can
influence our physical condition. When we apply ourselves to mental
development it can condition suppleness of the body. Thus we can verify
in our daily life what is taught in the Abhidhamma. 

Lightness, plasticity and wieldiness condition our bodily movements to
be supple. When we are speaking they condition the function of speech
to be supple and ``workable''. Whenever we notice that there are bodily
lightness, plasticity and wieldiness, we should remember that they are
qualities of r\=upa, conditioned by citta, temperature or nutrition.




R\=upas always arise in groups (kalapas) consisting of at least eight
r\=upas, the eight inseparable r\=upas. There are r\=upas other than
these eight and these arise in a group together with the eight
inseparable r\=upas. Our body consists of different groups of r\=upas
and each group is surrounded by infinitesimally tiny space, and this is
the r\=upa that is called space (ak{\=a}sa)\footnote{I used for the
description of space Acharn Sujin's ``Survey of Paramattha Dhammas'',
Ch 4.}. The r\=upas within a group are holding tightly together and
cannot be divided, and the r\=upa space allows the different groups to
be distinct from each other. Thus, its function is separating or
delimiting the different groups of r\=upas, and therefore it is also
called pariccheda r\=upa, the r\=upa that delimits (pariccheda meaning
limit or boundary). The r\=upa space is a r\=upa without its own
distinct nature (asabh{\=a}va r\=upa), and it arises simultaneously
with the different groups of r\=upa it surrounds. 

The ``Atthas{\=a}lin\=\i'' (II, Book II, Part I, Ch III, 326) states that
space is that which cannot be scratched, cut or broken. It is
``untouched by the four great Elements.'' Space cannot be touched. The
``Atthas{\=a}lin\=\i'' gives the following definition of space
\footnote{See also Dhamma\-sanga\d ni, {\S}638 and Visuddhimagga XIV,
63.} : 




\begin{quote}\begin{flushleft}
`` \ldots space-element has the characteristic of delimiting material
objects, the function of showing their boundaries, the manifestation of
showing their limits, state of being untouched by the four great
elements and of being their holes and openings as manifestation, the
separated objects as proximate cause. It is that of which in the
separated groups we say `this is above, this is below, this is across.''
\end{flushleft}\end{quote}





Space delimits the groups of r\=upa that are produced by kamma, citta,
temperature and nutrition so that they are separated from each other.
If there were no space in between the different groups of r\=upa, these
groups would all be connected, not distinct from each other. Space
comes into being as it surrounds the groups of r\=upas produced by
kamma, citta, temperature and nutrition and, thus, it is regarded as
originating from each of these four factors. 

We read in the ''Discourse on the Analysis of the Elements'' (Middle
Length Sayings III, no 140) that the Buddha explained to the monk
Pukkus{\=a}ti about the elements and that he also spoke about the
element of space. This Sutta refers to the empty space of holes and
openings that are, as we have read, the manifestation of space. We
read:




\begin{quote}\begin{flushleft}
`` \ldots And what, monk, is the element of space? The element of space may
be internal, it may be external. And what, monk, is the internal
element of space? Whatever is space, spacious, is internal, referable
to an individual and derived therefrom, such as the auditory and nasal
orifices, the door of the mouth and that by which one swallows what is
munched, drunk, eaten and tasted, and where this remains, and where it
passes out (of the body) lower down, or whatever other thing is space,
spacious, is internal, referable to an individual and derived
therefrom, this, monk, is called the internal element of space.
Whatever is an internal element of space and whatever is an external
element of space, just these are the element of space. By means of
perfect intuitive wisdom this should be seen as it really is thus: This
is not mine, this am I not, this is not myself. Having seen this thus
as it really is by means of perfect intuitive wisdom, he disregards the
element of space, he cleanses his mind of the element of space.''

\end{flushleft}\end{quote}

The Sutta speaks about space of the auditory orifices and the other holes and openings of the body. Space in the ear is one of the conditions for hearing\footnote{Space in the ear or the nose is space that is not conditioned by one of the four factors of kamma, citta, temperature or nutrition; it is unconditioned r\=upa.}. We attach great importance to internal space and we take it for ``mine'' or self, but it is only a r\=upa element.













\subsection*{Questions}




\begin{enumerate}
\item When we notice suppleness of the limbs, is this experienced 
through the bodysense? 

\item Can suitable food, suitable weather and the citta which, for 
example, cultivates lovingkindness be conditions for lightness, plasticity and wieldiness of body?

\item Can these three qualities also be produced by kamma?

\item What is the function of space?

\end{enumerate}



































































\chapter[Characteristics of all R\=upas]{}
\section*{Characteristics of R\=upas}


As we have seen, r\=upas can be classified as sabh{\=a}va r\=upas,
r\=upas with their own distinct nature and asabh{\=a}va r\=upas,
r\=upas without their own distinct nature. The four Great Elements are
sabh{\=a}va r\=upas, they each have their own distinct nature and
characteristic. R\=upas such as lightness, plasticity and wieldiness
are asabh{\=a}va r\=upas, they are qualities of r\=upas. The
Dhamma\-sanga\d ni ({\S}596) incorporates in the list of the twentyeight
kinds of r\=upa not only r\=upas with their own distinct nature but
also qualities of r\=upa and characteristics of r\=upa. 

It mentions four different r\=upas which are characteristics of r\=upa,
lakkha\d na r\=upas (lakkha\d na means characteristic). These four
characteristics common to all sabh{\=a}va r\=upas are the following: 



\begin{description}
\item arising or origination (upacaya)\footnote{Literally: initial accumulation.}

\item continuity or development (santati)

\item decay or ageing (jarat{\=a})

\item falling away or impermanence (aniccat{\=a})
\end{description}




R\=upas do not arise singly, they arise in different groups
(kal{\=a}pas). The groups of r\=upa arise fall away, but they do not
fall away as rapidly as citta. R\=upa lasts as long as the duration of
seventeen moments of citta arising and falling away, succeeding one
another. After the arising of r\=upa there are moments of its presence:
its continuity or development. Decay, jarat{\=a} r\=upa, is the
characteristic indicating the moment close to its falling away and
impermanence, aniccat{\=a} r\=upa, is the characteristic indicating the
moment of its falling away. 

We do not notice that the r\=upas of our body fall away and that time
and again new r\=upas are produced which fall away again. So long as we
are alive, kamma, citta, temperature and nutrition produce r\=upas and
thus our bodily functions can continue. These r\=upas arise, develop,
decay and fall away within splitseconds. 

The ``Atthas{\=a}lin\=\i'' (II, Book II, Part I, Ch II, 327) and also
the ``Visuddhimagga'' (Ch XIV, 66, 67) speak in a general, conventional
sense about the arising of r\=upas at the first moment of life, the
initial arising, and they explain that after the initial arising at
rebirth there is `continuity', that is to say, the subsequent arising
of the groups of r\=upa. 

Thus, we have to remember that the characteristics are taught by
different methods: according to the very short duration of one rupa
that arises and continues before it decays and falls away or in a more
general way, in conventional sense. 

The characteristics of r\=upa are taught in a conventional sense in
order to help people to have more understanding of these
characteristics of r\=upa which denote the arising, the continuity, the
decay and the falling away. The teaching was adapted to the
capabilities to understand of different people. 

The ``Atthas{\=a}lin\=\i'' ((II, Book II, Part I, Ch II, 327)
explains in a wider sense, by way of conventional terms, the
origination of r\=upa at the first moment of life and the continuity of
r\=upa as the subsequent arising of r\=upa. Throughout our life there
is continuity in the production of r\=upa.

We read about continuity:




\begin{quote}\begin{flushleft}
``Continuity has the characteristic of continuous occurrence, the
function of linking or binding without a break, unbroken series as
manifestation, matter bound up without a break as proximate cause.''

\end{flushleft}\end{quote}



This definition of continuity in a more general sense reminds us that
the seeming permanence of the body is merely due to the continuous
production of new r\=upas replacing the ones that have fallen away.




We read in the `` Visuddhimagga'' (Ch XIV, 68) about decay or ageing:




\begin{quote}\begin{flushleft}
`` Ageing has the characteristic
of maturing (ripening) material instances. Its function is to lead on
towards [their termination]. It is manifested as the loss of newness
without the loss of individual essence, like oldness in paddy. Its
proximate cause is matter that is maturing (ripening). This is said
with reference to the kind of ageing that is evident through seeing
alteration in teeth, etc., as their brokenness, and so on (cf. Dhs.
644) \ldots''
\end{flushleft}\end{quote}




The commentary to the Visuddhimagga explains as to the simile of the
paddy, that paddy, when it is ageing, becomes harsh, but that it does
not lose its nature, that it is still paddy. It states: ``The ageing is
during the moments of its presence, then that dhamma does not abandon
its specific nature.''

Thus, here the commentary does not speak in a general, conventional way,
but it refers to decay as one of the four characteristics of a single
r\=upa, to the moment that is close to its falling away. After a r\=upa
such as visible object has arisen, there are the moments of its
presence, it is decaying and then it falls away. It is the same visible
object that is present and decaying, it does not lose its specific
nature. 




The ``Atthas{\=a}lin\=\i'' explains terms used by the ``Dhamma\-sanga\d ni''
in reference to decay, such as decrepitude, hoariness, wrinkles, the
shrinkage in length of days, the overripeness of the faculties:




\begin{quote}\begin{flushleft}
`` \ldots By the word `decrepitude' is shown the function which is the
reason for the broken state of teeth, nails, etc., in process of time.
By hoariness is shown the function which is the reason for the greyness
of hair on the head and body. By `wrinkles' is shown the function which
is the reason for the wrinkled state in the skin making the flesh fade.
Hence these three terms show the function of decay in process of
time \ldots''
\end{flushleft}\end{quote}




As to the terms ``shrinkage in life and maturity of faculties'', these
show the resultant nature of this decay. We read:




\begin{quote}\begin{flushleft}
`` \ldots Because the life of a being who has reached decay shortens,
therefore decay is said to be the shrinkage in life by a figure of
speech. Moreover, the faculties, such as sight, etc., capable of easily
seizing their own object, however subtle, and which are clear in youth,
are mature in one who has attained decay; they are disturbed, not
distinct, and not capable of seizing their own object however
gross \ldots''


\end{flushleft}\end{quote}


When we notice decay of our teeth, wrinkles of the skin and greying of
our hairs, decay is obvious. However, we should remember that each
r\=upa that arises is susceptible to decay, that it will fall away
completely. 




As to impermanence, aniccat{\=a}, the ``Visuddhimagga'' (Ch XIV, 69)
states about it as follows\footnote{See Dhamma\-sanga\d ni {\S}645 and
Atthas{\=a}lin\=\i, II, Book II, Part I, Ch II, 328} : 

\begin{quote}\begin{flushleft}
``Impermanence of matter has the
characteristic of complete
breaking up. Its function is to make material instances subside. It is
manifested as destruction and fall (cf. Dhs. 645). Its proximate cause
is matter that is completely breaking up.''
\end{flushleft}\end{quote}




The commentary to the ``Visuddhimagga'' speaks about the impermanence as
the falling away of each r\=upa: 




\begin{quote}\begin{flushleft}
`` It is said that its function is to make (material instances) subside,
since this (impermanence) causes the materiality that has reached (the
moments of) presence as it were to subside. And since this
(impermanence), because of the state of dissolution of material
phenomena, should be regarded as destruction and fall, it is said that
it is manifested as destruction and fall.''
\end{flushleft}\end{quote}




As soon as r\=upa has arisen, it is led onward to its termination and it
breaks up completely, never to come back again. Remembering this is
still theoretical knowledge of the truth of impermanence, different
from right understanding that realizes the arising and falling away of
a n{\=a}ma or a r\=upa. When understanding has not yet reached this
stage one cannot imagine what it is like. One may tend to cling to
ideas about the arising and falling away of phenomena but that is not
the development of understanding. N{\=a}ma and r\=upa have each
different characteristics and so long as one still confuses n{\=a}ma
and r\=upa, their arising and falling away cannot be realized.
Understanding is developed in different stages and one cannot forego
any stage. First there should be a precise understanding of n{\=a}ma as
n{\=a}ma and of r\=upa as r\=upa so that the difference between these
two kinds of realities can be clearly seen. It is only at a later stage
in the development of understanding that the arising and falling away
of n{\=a}ma and r\=upa can be directly known. 

The ``Atthas{\=a}lin\=\i'' (II, Book II, Part I, Ch II, 329) compares
birth, decay and death to three enemies, of whom the first leads
someone into the forest, the second throws him down and the third cuts
off his head. We read:




\begin{quote}\begin{flushleft}
`` \ldots For birth is like the enemy who draws him to enter the forest;
because he has come to birth in this or that place. Decay is like the
enemy who strikes and fells him to earth when he has reached the
forest, because the aggregates (khandhas) produced are weak, dependent
on others, lying down on a couch. Death is like the enemy who with a
sword cuts off the head of him when he is fallen to the ground, because
the aggregates having attained decay, have come to destruction of
life.''
\end{flushleft}\end{quote}




This simile reminds us of the disadvantages of all conditioned realities
that do not last and are therefore no refuge. However, so long as
understanding (pa\~n\~n{\=a}) has not realized the arising and falling
away of n{\=a}ma and r\=upa, one does not grasp their danger. 

We read in the ``D\=\i ghanakhasutta'' (Middle Length Sayings II, no. 74)
that the Buddha reminded D\=\i ghanakha that the body is susceptible to
decay, impermanent and not self:




\begin{quote}\begin{flushleft}
``But this body, Aggivessana, which has material shape, is made up of
the four great elements, originating from mother and father, nourished
on gruel and sour milk, of a nature to be constantly rubbed away,
pounded away, broken up and scattered, should be regarded as
impermanent, suffering, as a disease, a tumour, a dart, a misfortune,
an affliction, as other, as decay, empty, not-self. When he regards
this body as impermanent, suffering, as a disease, a tumour, a dart, a
misfortune, an affliction, as other, as decay, empty, not-self,
whatever in regard to body is desire for body, affection for body,
subordination to body, this is got rid of.''

\end{flushleft}\end{quote}



Origination, continuity, decay and impermanence are characteristics
common to all sabh\=ava r\=upas. They do not have their own distinct nature,
thus, they are asabh{\=a}va r\=upas. These characteristics are not
produced by the four factors of kamma, citta, food and temperature. 
R\=upas have been classified as twentyeight kinds. Summarizing them,
\ they are:




\begin{description}
\item solidity (or extension)

\item cohesion

\item temperature

\item motion

\item eyesense

\item earsense

\item nose (smellingsense)

\item tongue (tastingsense)

\item bodysense

\item visible object

\item sound

\item odour

\item flavour

\item femininity

\item masculinity

\item heart-base

\item life faculty

\item nutrition

\item space

\item bodily intimation

\item speech intimation

\item lightness

\item plasticity

\item wieldiness

\item origination

\item continuity

\item decay

\item impermanence

\end{description}



R\=upas can be classified as the four Principle R\=upas and the
twentyfour derived r\=upas. The four Principle r\=upas,
mah{\=a}-bh\=uta r\=upas, are the four Great Elements. The derived
r\=upas, up{\=a}d{\=a} r\=upas, are the other twentyfour r\=upas that
arise in dependence upon the four Great Elements. 

R\=upas can be classified as gross and subtle. As we have seen (in
Chapter 4), twelve kinds of r\=upa are gross: visible object, sound,
odour, flavour and three of the four Great Elements which are tangible
object (excluding cohesion), as well as the five sense-organs. They
are gross because of impinging: visible object impinges on the
eyesense, sound impinges on the earsense, and each of the other sense
objects impinges on the appropriate sense-base. The other sixteen
kinds of r\=upa are subtle. What is subtle is called ``far'' because it
is difficult to penetrate, whereas what is gross is called ``near'',
because it is easy to penetrate (Vis. XIV, 73).

Furthermore, other distinctions can be made. R\=upas can be classified
as sabh{\=a}va r\=upas, r\=upas with their own distinct nature, and
asabh{\=a}va r\=upas, r\=upas without their own distinct nature. The
twelve gross r\=upas and six among the subtle r\=upas that are:
cohesion, nutrition, life faculty, heart-base, femininity and
masculinity are rupas each with their own distinct nature and
characteristic, they are sabh{\=a}va r\=upas. 


The other ten subtle r\=upas do not have their own distinct nature, they
are asabh{\=a}va r\=upas. Among these are the two kinds of intimation,
bodily intimation and speech intimation, which are a ``certain, unique
change'' in the eight inseparable r\=upas produced by citta. Moreover, the three qualities of lightness, plasticity and weildiness, classified together with the two r\=upas of intimation as vik\=ara r\=upas (r\=upa as changeability or alteration) are included in the asabh\=ava r\=upas. Furthurmore, the r\=upa space (ak\=asa or pariccheda r\=upa) that delimits the groups of r\=upa, as well as the four r\=upas that are the characteristics of origination, continuity, decay and impermanence, are included.

R\=upas can be classified as produced r\=upas, nipphanna r\=upas, and
unproduced r\=upas, anipphanna r\=upas. The sabh{\=a}va r\=upas are
also called ``produced'', whereas the asabh{\=a}va r\=upas are also
called ``unproduced''\footnote{For details see Visuddhimagga XIV, 73,
77. }. The two kinds of intimation produced by citta, the three
qualities of lightness, plasticity and wieldiness produced by citta,
temperature or nutrition and space which delimits the groups of r\=upa
produced by the four factors and therefore originating from these four
factors, are still called ``unproduced'', anipphanna, because they
themselves are not r\=upas with their own distinct nature, they are not
``concrete matter''. 

The ``produced r\=upas'' which each have their own characteristic are,
as the ``Visuddhimagga'' (XVIII, 13) explains, ``suitable for
comprehension'', that is, they are objects of which right understanding
can be developed. For example, visible object or hardness have
characteristics that can be objects of awareness when they appear, and
they can be realized by pa\~n\~n{\=a} as they are, as non-self. The
``unproduced r\=upas'' are ``not suitable for comprehension'' since
they are qualities of r\=upa such as changeability or the r\=upa that
delimits groups of r\=upas. If one does not know this distinction, one
may be led to wrong practice. 






\subsection*{Questions}

\begin{enumerate}
\item Can the r\=upas of lightness, plasticity and wieldiness be objects
of awareness? 

\end{enumerate}

















\chapter[Groups of R\=upas]{} 
\section*{Groups of R\=upas}



R\=upas do not arise singly, they always arise collectively, in groups
(kal{\=a}pas). Where there is solidity, the Element of Earth, there
have to be the other three Great Elements, and also colour, flavour,
odour and nutrition. These are the eight inseparable r\=upas. A group
of r\=upas consisting of only the eight inseparable r\=upas is called a
``pure octad''. Pure octads of the body are produced by citta,
temperature or nutrition, and pure octads outside the body are produced
only by temperature.

The groups of r\=upas produced by kamma have to consist of at least nine
r\=upas: the eight inseparable r\=upas and life faculty
(j\=\i v\-itindriya), and such a group is called a ``nonad''. Eyesense,
earsense, smelling-sense, tasting-sense, bodysense, heart-base,
femininity and masculinity are other kinds of r\=upa produced by kamma
and each of these r\=upas arises together with the eight inseparable
r\=upas and life faculty, thus, they arise in groups of ten r\=upas,
\ decads. All r\=upas of such a decad are produced by kamma. Thus, one
speaks of eye-decad, ear-decad, nose-decad, tongue-decad,
body-decad, heart-base-decad, femininity-decad and
masculinity-decad. As to the body-decad, this arises and falls away
at any place of the body where there can be sensitivity.

Kamma produces groups of r\=upas from the arising moment of the
rebirth-consciousness (pa\d t\-isan\-dhi-citta). In the case of human
beings, kamma produces at that moment the three decads of bodysense, of
sex (femininity or masculinity) and of heart-base, and it produces
these decads throughout our life. The eye-decad and the decads of
ear, nose and tongue are not produced at the first moment of life but
later on. 

The citta which is rebirth-consciousness does not produce r\=upa. The
citta which immediately succeeds the rebirth-consc\-ious\-ness, na\-mely
the life-continuum (bhavanga-citta)\footnote{The bhavanga-citta
arises in between the processes of cittas; it does not experience
objects that impinge on the six doors, but it experiences the same
object as the rebirth-consciousness. It maintains the continuity in the 
life of the individual.} , produces r\=upa. One moment of citta can be divided into three
extremely short phases: its arising moment, the moment of its presence
and the moment of its falling away. Citta produces r\=upa at its
arising moment, since citta is then strong. At the moment of its
presence and the moment of its dissolution it is weak and then it does
not produce r\=upa (Visuddhimagga XX, 32). When the citta succeeding
the rebirth-consciousness, the life-continuum, arises, it produces
a pure octad. Later on citta produces, apart from pure octads, also
groups with bodily intimation, with speech intimation and with the
three r\=upas of lightness, plasticity and wieldiness, which
always have to arise together. These three kinds of r\=upa also arise
in a group together with bodily intimation and speech intimation. In
the case of speech intimation, also sound arises together with speech
intimation in one group. 

Throughout life citta produces r\=upa, but not all cittas can produce
r\=upa. As we have seen, the rebirth-consciousness does not produce
r\=upa. Among the cittas that do not produce r\=upa are also the
sense-cognitions of seeing, hearing, etc. Seeing only sees, it has no
other capacity. Some cittas can produce r\=upas but not bodily
intimation and speech intimation, and some cittas can produce the two
kinds of intimation. Among the cittas that can produce the two kinds of
intimation are the kusala cittas of the sense-sphere (thus not those
that attain absorption or jh{\=a}na and those that realize
enlightenment), and the akusala cittas\footnote{See
Visuddhimagga XX, 31, and Atthas{\=a}lin\=\i{}  II, Book II,  Part I,
\mbox{Ch III, 325.}}. 

Temperature (heat-element) can produce groups of r\=upas of the body
as well as groups of r\=upas of materiality outside. In the case of
materiality outside it produces groups that are ``pure octads'' and
also groups with sound\footnote{Sound can be produced by temperature
or by citta.}. R\=upas that are not of the body are solely produced by
temperature, they are not produced by kamma, citta or nutrition. When
we see a rock or plant we may think that they last, but they consist of
r\=upas originated by temperature, arising and falling away all the
time. R\=upas are being replaced time and again, and we do not realize
that r\=upas which have fallen away never come back.

As regards groups of r\=upas of the body, temperature produces pure
octads and also groups with lightness, plasticity and wieldiness. 

As we have seen, kamma produces r\=upa in a living being from the moment
the rebirth-consciousness arises. Kamma produces r\=upa at each of
the three moments of citta: at its arising moment, at the moment of its
presence and at the moment of its falling away. In each group of
r\=upas produced by kamma there is the element of heat (utu), which is
one of the four Great Elements, and this begins to produce new r\=upas
at the moment of presence of rebirth-consciousness, thus, not at the
arising moment of rebirth-consciousness. The element of heat produces
other r\=upas during the moments of its presence, it cannot produce
r\=upas at its arising moment\footnote{Temperature and nutrition,
r\=upas which produce other r\=upas, do not produce these at the moment
of their arising, since they are then weak, but they produce r\=upas
during the moments of presence, before they fall away. The duration of
r\=upa, when compared with the duration of citta, is as long as
seventeen moments of citta, thus there are fifteen moments of presence
of r\=upa. Citta, however, produces r\=upas at its arising moment since
it is then strong.}. It originates a pure octad and from that moment
on it produces, throughout life, r\=upas during the moments of its
presence\footnote{The heat-element present in a group which is
produced by temperature, no matter whether of materiality outside or of
the body, can, in its turn, produce a pure octad and in this way
several occurrences of octads can be linked up. In the same way, the
heat-element present in groups of r\=upas of the body, produced by
kamma or citta can, in its turn, produce a pure octad, and the
heat-element present in that octad can produce another octad, and so
on. In this way several occurrences of octads are linked
up. Temperature produced by nutrition can also, in
its turn, produce another octad. }. 

Nutritive essence present in food that has been swallowed
\footnote{See Ch 2. The substance of morsel-made food
(kaba\d link{\=a}ro {\=a}h{\=a}ro) contains nutritive essence, oj{\=a}.
When food has been swallowed the nutritive essence pervades the body
and supports it.} , produces r\=upas and it supports and sustains the
body. R\=upas produced by nutrition arise only in the body of living
beings. Nutrition produces pure octads and also groups of r\=upa with
lightness, plasticity and wieldiness. 

Nutritive essense is one of the inseparable r\=upas present in each
group of r\=upas. Nutritive essence in nutriment-originated octads
originates a further octad with nutritive essence and thus, it links up
the occurrences of several octads. The ``Visuddhimagga'' (XX, 37)
states that nutriment taken on one day can thus sustain the body for as
long as seven days\footnote{Also nutriment smeared on the body
originates materiality, according to the ``Visuddhimagga''. Some
creams, for example, nourish the skin.}. 

The groups of r\=upas produced by kamma, citta, temperature and
nutrition are interrelated and support one another. If only kamma would
produce r\=upas the body could not continue to exist. We read in the
``Visuddhimagga'' (XVII, 196):




\begin{quote}\begin{flushleft}
``\ldots But when they thus give consolidating support to each other, they
can stand up without falling, like sheaves of reeds propped up together
on all four sides, even though battered by the wind, and like (boats
with) broken floats that have found a support, even though battered by
waves somewhere in mid-ocean, and they can last one year, two
years \ldots a hundred years, until those beings' life span or their
merit is exhausted.''
\end{flushleft}\end{quote}




The ``Atthas{\=a}lin\=\i'' (I, Book I, Part III, Ch I, 84)
\footnote{See also Visuddhimagga XIV, 61. }, in the context of bodily
intimation, explains that groups of r\=upa produced by citta are
interlocked with groups of r\=upas produced by kamma, temperature and
nutrition. We read:




\begin{quote}\begin{flushleft}
When the body set up by mind (citta) moves, does the body set up by the
other three causes move or not? The latter moves likewise, goes with
the former, and invariably follows it. Just as dry sticks, grass, etc., 
fallen in the flowing water go with the water or stop with it, so
should the complete process be understood \ldots

\end{flushleft}\end{quote}



The study of the groups of r\=upas produced by the four factors of
kamma, citta, temperature and nutrition and also their interrelation
shows us the complexity of the conditions for the bodily phenomena and
functions from birth to death\footnote{For details see:
Atthas{\=a}lin\=\i{}  II, Book II, Part I, Ch III, 342, 343, and
Visuddhimagga XX, 32-43.}. It reminds us that there is no self who
can control the body. 

Not all types of r\=upa arise in the different planes of existence where
living beings are born. Apart from the plane of human beings, there are
also other planes of existence. Birth in an unhappy plane or a happy
plane is the result of kamma. Birth in the human plane of existence is
the result of kusala kamma. During life there are conditions for the
occurring of results of kusala kamma and of akusala kamma, namely, the
experience of pleasant objects as well as unpleasant objects through
the senses. In the human plane the decads of eye, ear, nose, tongue and
bodysense that are produced by kamma arise, so that the different sense
objects can be experienced. People who see the disadvantages of
enslavement to sense impressions and have accumulated the right
conditions for the development of a high degree of calm, can attain
stages of jh{\=a}na. The result of the different stages of jh{\=a}na is
birth in higher planes of existence where less sense impressions occur
or none at all. In some of the higher planes\footnote{The
r\=upa-brahma planes. Birth in these planes is the result of
r\=upa-jh{\=a}na, fine-material jh{\=a}na.} the decads of nose,
tongue, bodysense and sex are absent, but the decads of eye and ear,
the decad of the heart-base and the nonad of life faculty (life
faculty and the eight inseparable r\=upas) arise. The r\=upas produced
by nutrition do not arise. In these planes one does not need food to
stay alive. 

In one of the higher planes of existence there is no n{\=a}ma, only
\ r\=upa\footnote{The ``perceptionless beings plane''
(asa\~n\~n{\=a}-satta plane) which is one of the r\=upa-brahma
planes. Those who are born here have seen the disadvantages of
n{\=a}ma.}. Here the decads of eye, ear and the other senses, sex and
heart-base are absent. Sound does not arise and neither do r\=upas
produced by citta arise, since n{\=a}ma does not arise. Kamma produces
the nonad of life faculty at the first moment of life and after that
also temperature produces r\=upas. 

In some of the higher planes only n{\=a}ma ocurs and thus r\=upas do not
arise in such planes\footnote{The ar\=upa-brahma planes. Birth in
these planes is the result of ar\=upa-jh{\=a}na, ``immaterial
jh{\=a}na''.}. 









\subsection*{Questions}





\begin{enumerate}
\item Can kamma produce groups of eight r\=upas, pure octads?

\item Which decads produced by kamma arise at the first moment of 
life in the case of human beings?

\item Can rebirth-consciousness produce r\=upa?

\item Can seeing-consciousness produce r\=upa?

\item Can citta rooted in attachment produce r\=upa?
\end{enumerate}





\chapter[Conclusion]{}
\section*{Conclusion}

The study of the different kinds of r\=upa will help us to understand
more clearly the various conditions for the arising of bodily phenomena
and mental phenomena. Gradually we shall come to understand that all
our experiences in life, all the objects we experience, our bodily
movements and our speech are only conditioned n{\=a}ma and r\=upa. In
the planes of existence where there are n{\=a}ma and r\=upa, n{\=a}ma
conditions r\=upa and r\=upa conditions n{\=a}ma in different ways. The
r\=upas that are sense objects and the r\=upas that can function as
sense-doors are conditions for the different cittas arising in
processes which experience sense objects.

In order to develop understanding of n{\=a}ma and r\=upa it is necessary
to learn to be mindful of the n{\=a}ma or r\=upa appearing at the
present moment. Only one object at a time can be object of mindfulness
and in the beginning we may find this difficult. The study of r\=upas
can help us to have more clarity about the fact that only one object at
a time can be experienced through one of the six doors. Visible object,
for example, can be experienced through the eye-door, it cannot be
experienced through the body-door, thus, through touch.
Seeing-consciousness experiences what is visible and
body-consciousness experiences tangible object, such as hardness or
softness. Through each door the appropriate object can be experienced
and the different doorways should not be confused with one another.
When we believe that we can see and touch a flower, we think of a
concept. We can learn to see the difference between awareness of one
reality at a time and thinking of a concept. A concept or
conventional truth can be an object of thought, but it is not a
paramattha dhamma, an ultimate reality with its own inalterable
characteristic. 

It may seem difficult to be mindful of one reality at a time, but
realities such as visible object, hardness or sound are impinging on
the senses time and again. When we have understood that they have
different characteristics and that they present themselves one at a
time, we can learn to be mindful of them. We should remember that at
the moment of mindfulness of a reality understanding of that reality
can be developed. Right understanding should be the goal. There is no
self who understands. Understanding is a cetasika, a type of n{\=a}ma;
it understands and it can develop.

Right understanding is developed in different stages of insight and it
is useful to know more about the first stage. When the first stage of
insight has been reached, pa\~n\~n{\=a}, understanding, distinguishes
the characteristic of n{\=a}ma from the characteristic of r\=upa. In
theory we know that n{\=a}ma experiences something and that r\=upa does
not experience anything, but when they appear there is, at first, not
yet direct understanding of their different characteristics. We may,
for example, cling to an idea of ``I am feeling hot''. What is there in
reality? There is n{\=a}ma that experiences heat and there is r\=upa
that is heat, but we tend to think of a ``whole'', a conglomeration of
different phenomena: of a person who feels hot. Then n{\=a}ma cannot be
distinguished from r\=upa. It is true that when heat is experienced,
also the r\=upa that is heat is present. However, only one reality at a
time can be object of mindfulness. Sometimes n{\=a}ma is the object of
mindfulness and sometimes r\=upa, and this depends on conditions. When
one reality at a time is object of mindfulness, one does not, at that
moment, think of ``self'' or ``my body''. Gradually understanding can
develop and then clinging to self will decrease.
R\=upas impinging on the five senses are experienced through the
sense-doors as well as through the mind-door. N{\=a}ma cannot be
experienced through a sense-door, but only through the mind-door.
Each of the sense-objects that is experienced through the appropriate
sense-door is also experienced through the mind-door. We may
understand that seeing sees visible object, but the experience of
visible object through the mind-door is concealed. The processes of
cittas pass very rapidly and when understanding has not been developed
it is not clearly known what the mind-door process is. At the first
stage of insight pa\~n\~n{\=a} arising in a mind-door process clearly
realizes the difference between the characteristic of n{\=a}ma and the
characteristic of r\=upa, and at that stage it is also known what the
mind-door is. When understanding develops it will eventually lead to
that stage. 

The study of n{\=a}ma and r\=upa can clear up misunderstandings about
the development of understanding and about the object of understanding.
Reading about n{\=a}ma and r\=upa and pondering over them are
conditions for the development of right understanding of the realities
presenting themselves thro\-ugh the six doors.
We read in the ``Ther\=\i g{\=a}th{\=a}'' (Psalms of the Sisters) about
people in the Buddha's time who were disturbed by problems and could
not find mental stability. When they were taught Abhidhamma they could
develop right understanding and even attain enlightenment. While one
studies the elements, the sense-doors, the objects, in short, all
ultimate realities (paramattha dhammas), the truth that there is no
being or self becomes more evident. We read in Canto 57 about
Bhikkhun\=\i{}\footnote{Bhikkhun\=\i{}  means nun or sister.} Vijay{\=a}
who could not find peace of mind. After she had been taught Abhidhamma
she developed right understanding of realities and attained arahatship
\footnote{The fourth and last stage of enlightenment.} We read:




\begin{verse}
``Four times, nay five, I sallied from my cell,

And roamed afield to find the peace of mind

I lacked, and governance of thoughts

I could not bring into captivity.

Then to a Bhikkhun\=\i{}  I came and asked

Full many a question of my doubts.

To me she taught Dhamma: the elements,

Organ and object in the life of sense,

(And then the factors of the Nobler life:)

The Ariyan truths, the Faculties, the Powers,

The Seven Factors of Enlightenment
\footnote{The ariyan Truths are the
four noble Truths: the Truth of dukkha, that is, the impermanence and
unsatisfactoriness of all conditioned realities; the Truth of the
origin of dukkha, that is, craving; the Truth of the cessation of
dukkha, that is, nibb{\=a}na; the Truth of the way leading to the
cessation of dukkha, that is, the development of the eightfold Path.
The Faculties, Powers, Seven Factors of Enlightenment are
wholesome qualities that develop together with satipa\d t\d th{\=a}na so
that enlightenment can be attained. Among them are mindfulness, energy,
concentration and understanding.}, 


The Eightfold Path, leading to utmost good.

I heard her words, her bidding I obeyed.

While passed the first watch of the night there rose

Long memories of the bygone line of lives.

While passed the second watch, the Heavenly Eye,

Purview celestial, I clarified 
\footnote{The Heavenly Eye is the
knowledge of the passing away and rebirth of beings.}. 

While passed the last watch of the night, I burst

And rent aside the gloom of ignorance.

Then, letting joy and blissful ease of mind

Suffuse my body, seven days I sat,

Ere stretching out cramped limbs I rose again.

Was it not rent indeed, that muffling mist?''
\end{verse}

\backmatter

\chapter{Glossary}
\begin{description}
\item[abhidhamma] the higher teachings of Buddhism, teachings on ultimate realities.
\item[ak\=asa] space
\item[akusala] unwholesome, unskilful.
\item[an\=ag\=am\=\i] non returner, person who has reached the third stage of enlightenment, he has no aversion (dosa).
\item[anatt\=a] not self.
\item[anicca] impermanent
\item[aniccat\=a] falling away or impermanence.
\item[apo-dh\=atu] element of water or cohesion.
\item[arahat] noble person who has attained the fourth and last stage of enlightenment.
\item[\=aramma\d na] object which is known by consciousness.
\item[ariyan] noble person who has attained enlightenment.
\item[asabh\=ava r\=upas] r\=upas without their own distinct nature.
\item[avinibbhoga r\=upas] the four Great elements and  four derived r\=upas, which always arise together.
\item[bhikkhu] monk.
\item[bhikkhun\=\i] nun.
\item[Buddha] a fully enlightened person who has discovered the truth all by himself, without the aid of a teacher.
\item[cakkhu-dh\=atu] eye element.
\item[cakkhu-dv\=ara] eyedoor.
\item[cakkhu-vi\~n\~n\=a\d na] seeing-consciousness.
\item[cakkhu] eye.
\item[cakkhuppas\=ada r\=upa] r\=upa which is the organ of eyesense, capable of receiving visible object.
\item[cetan\=a] volition or intention.
\item[cetasika] mental factor arising with consciousness.
\item[citta] consciousness, the reality which knows or cognizes an object.
\item[dassana-kicca] function of seeing.
\item[dhamma-dh\=atu] element of dhammas, realities, comprising cetasikas, subtle r\=upas, nibb\=ana.
\item[dhamma] reality, truth, the teachings.
\item[dhamm\=aramma\d na] all objects other than the sense objects which can be experienced through the five sense-doors, thus, objects which can be experienced only through the mind-door.
\item[Dh\=atukath\=a] Discourse on the Elements, the third book of the Abhidhamma.
\item[di\d t\d thi] wrong view, distorted view of realities.
\item[dukkha] suffering, unsatisfactoriness of conditioned realities.
\item[dv\=ara] doorway through which an object is experienced, the five sense-doors or the mind door.
\item[dvi-pa\~nca-vi\~n\~n\=a\d na] the five pairs of sense-cognitions, which are seeing, hearing, smelling, tasting and body-consciousness. Of each pair one is kusala vip\=aka and one akusala vip\=aka.
\item[gh\=ana-dh\=atu] nose element.
\item[gh\=ana-vi\~n\~n\=a\d na] smelling-consciousness.
\item[gh\=anappas\=ada r\=upa] r\=upa which is the organ of smelling sense, capable of receiving odour.
\item[gh\=ayana-kicca] function of smelling.
\item[hadayavatthu]  heart-base.
\item[itthindriya\d m] femininity-faculty.
\item[jarat\=a] decay or ageing.
\item[jivh\=a-dh\=atu] tongue element.
\item[jivh\=appas\=ada r\=upa] r\=upa which is the organ of tasting sense, capable of receiving flavour.
\item[jivh\=a-vi\~n\~n\=a\d na] tasting-consciousness.
\item[j\=\i vitindriya] life faculty. 
\item[k\=ama] sensual enjoyment or the five sense objects.
\item[kamma patha] course of action performed through body, speech or mind which can be wholesome or unwholesome.
\item[kamma] intention or volition; deed motivated by volition.
\item[kamma\~n\~nat\=a] wieldiness.
\item[k\=aya-dh\=atu] the element of bodysense.
\item[k\=aya-vi\~n\~natti] bodily intimation, such as gestures, facial expression, etc.
\item[k\=aya-vi\~n\~n\=a\d na] body-consciousness.
\item[k\=aya] body. It can also stand for the mental body, the cetasikas.
\item[k\=ayappas\=ada r\=upa] bodysense, the r\=upa which is capable of receiving tangible object. It is all over the body, inside or outside.
\item[khandhas] conditioned realities classified as five groups: physical phenomena, feelings, perception or remembrance, activities or formations (cetasikas other than feeling or perception), consciousness.
\item[kicca] function.
\item[kusala citta] wholesome consciousness.
\item[kusala kamma] a good deed.
\item[kusala] wholesome, skilful
\item[lahut\=a]  buoyancy or lightness.
\item[Mah\=a-bh\=uta r\=upas ]the four Great Elements
\item[moha] ignorance.
\item[mudut\=a] plasticity.
\item[n\=ama] mental phenomena.
\item[nibb\=ana] unconditioned reality, the reality which does not arise and fall away. The destruction of lust, hatred and delusion. The deathless. The end of suffering.
\item[o\d l\=arika r\=upas] gross r\=upas (sense objects and sense organs).
\item[oj\=a] the r\=upa which is nutrition.
\item[pa\d thav\=\i -dh\=atu] The element of earth. 
\item[pa\d tisandhi-citta] the first moment of life the rebirth-consciousness.
\item[P\=ali] the language of the Buddhist teachings.
\item[pa\~n\~n\=a] wisdom or understanding.
\item[pa\~n\~natti] concepts, conventional terms.
\item[paramattha dhamma] truth in the absolute sense: including citta, cetasika, r\=upa and nibb\=ana. 

\item[pariccheda r\=upa] the r\=upa that is space, delimiting the groups of r\=upa (pariccheda meaning  limit or boundary).  
\item[pas\=ada-r\=upas] r\=upas which are capable of receiving sense-objects such as visible object, sound, taste, etc.
\item[Purisindriya\d m] masculinity-faculty.
\item[r\=upa] physical phenomena, realities which do not experience anything.
\item[sabh\=ava r\=upas] r\=upas with their own distinct nature.
\item[sadd\=aramma\d na] sound.
\item[sa\.nkhata or sa\.nkh\=ara dhamma] conditioned dhamma.
\item[sa\~n\~n\=a-kkhandha] remembrance classified as one of the five khandhas.
\item[sa\~n\~n\=a] memory, remembrance or perception.
\item[santati] continuity or development.
\item[sa\.nkh\=ara dhamma] conditioned dhamma.
\item[sati] mindfulness or awareness: non-forgetfulness of what is wholesome, or non-forgetfulness of realities which appear.
\item[satipa\d t\d th\=ana] applications of mindfulness. It can mean the cetasika sati which is aware of realities or the objects of mindfulness which are classified as four applications of mindfulness: Body, Feeling, Citta, Dhamma. Or it can mean the Path the Buddha and the aryan disciples have developed.
\item[savana-kicca] function of hearing.
\item[s\=ayana-kicca] function of tasting.
\item[s\=\i la] morality in action or speech, virtue.
\item[sota-dh\=atu] element of earsense.
\item[sota-dv\=ara-v\=\i thi-cittas] ear-door process cittas.
\item[sota-dv\=ar\=avajjana-citta] ear-door-adverting-consciousness.
\item[sot\=apanna] person who has attained the first stage of enlightenment, and who has eradicated wrong view of realities.
\item[sota-vi\~n\~n\=a\d na] hearing-consciousness.
\item[sukhuma r\=upas ]subtle r\=upas 
\item[sutta] part of the scriptures containing dialogues at different places on different occasions.
\item[suttanta] a sutta text.
\item[tejo-dh\=atu] element of fire or heat.
\item[Therav\=ada Buddhism] Doctrine of the Elders, the oldest tradition of Buddhism.
\item[Tipi\d taka] the teachings of the Buddha.
\item[upacaya] arising or origination.
\item[up\=ad\=a-r\=upa] derived r\=upas, the r\=upas other than the four Great Elements.
\item[vac\=\i{} vi\~n\~natti ]speech intimation.
\item[vatthu] base, physical base of citta.
\item[v\=ayo-dh\=atu] element of wind or motion.
\item[vi\~n\~n\=a\d na] consciousness, citta.
\item[vip\=akacitta] citta which is the result of a wholesome deed (kusala kamma) or an unwholesome deed (akusala kamma). It can arise as rebirth-consciousness, or during life as the experience of pleasant or unpleasant objects through the senses, such as seeing, hearing, etc.
\item[vipassan\=a] wisdom which sees realities as they are.
\end{description}



\vspace{10 mm}
\section*{Other books written by Nina van Gorkom}

\begin{description}
\item[The Buddha's Path] An Introduction to the doctrine of Theravada Buddhism.
\item[Buddhism in Daily Life] A general introduction to the main ideas of Theravada Buddhism.
\item[Abhidhamma in Daily Life] is an exposition of absolute realities in detail. Abhidhamma means higher doctrine and the book's purpose is to encourage the right application of Buddhism in order to eradicate wrong view and eventually all defilements.

\item[The World in the Buddhist Sense] The purpose of this book is to show that the Buddhas Path to true understanding has to be developed in daily life.

\item[The Perfections Leading to Enlightenment] The Perfections is a study of the ten good qualities: generosity, morality, renunciation, wisdom, energy, patience, truthfulness, determination, loving-kindness, and equanimity.






\item[Cetasikas] Cetasika means 'belonging to the mind'. It is a mental factor which accompanies consciousness (citta) and experiences an object. There are 52 cetasikas. This book gives an outline of each of these 52 cetasikas and shows the relationship they have with each other.


\item[The Buddhist Teaching on Physical Phenomena] A general introduction to physical phenomena and the way they are
related to each other and to mental phenomena.\end{description}

\section*{Books translated by Nina van Gorkom}

\begin{description}

\item[Metta: Loving kindness in Buddhism] An introduction to the basic Buddhist teachings of metta, loving kindness, and its practical application in todays world. 


\item[Taking Refuge in Buddhism] Taking Refuge in Buddhism is an introduction to the development of insight meditation. 

\item[A Survey of Paramattha Dhammas] A Survey of Paramattha Dhammas is a guide to the development of the Buddha's path of wisdom, covering all aspects of human life and human behaviour, good and bad.
\end{description}



These and other articles can be seen at:

www.zolag.co.uk or www.scribd.com (search for zolag).














\end{document}
