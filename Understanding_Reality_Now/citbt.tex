\part{Confidence in the Buddha's Teachings}



\chapter{Preface}

Acharn Sujin and her sister Khun Sujit
were invited to Vietnam at the end of December 2014 by Tam Bach and
other Vietnamese friends for a ten days sojourn. They sponsored their
flight and hotel accommodation. Friends from Thailand, Australia,
Canada, the U.S. and myself joined this journey. In Vietnam Tran Thai
made the travel and accommodation arrangements for all of us. Afterwards
we returned to Thailand for more Dhamma discussions. We went twice to
Kaeng Krachan, to the place where Acharn Sujin and Khun Duangduen
regularly stay, and here other friends joined us for Dhamma discussions.
Towards the end of my stay there was a brief journey to the North, to
the mountainous regions of Chiengmai and Chiengrai. 

In Vietnam the discussions took place
in Saigon, in a large hall of the hotel where we stayed, and since about
hundred people attended the discussions every day the place became
rather crowded. Among the audience were two monks and many ``nuns'',
that is to say, women who wore robes and observed eight
precepts\footnote{In addition to the five precepts
there are others. Among the eight precepts are: refraining from eating
after midday, refraining from lying on high beds, refraining from
entertainments and from adorning oneself.}
They all were listening very keenly while Tam Bach translated into
Vietnamese what Acharn explained in English. Sarah and Jonothan assisted
Acharn Sujin untiringly with further explanations of the Dhamma. 

Our Vietnamese friends took care of us with great hospitality and every day they left fruits and biscuits on tables set out in the hall. They also gave us large packages with sweets
and different objects which were thoughtfully selected. 

Throughout our time in Vietnam and in Thailand Acharn repeated that theoretical knowledge is not enough.
Without developing understanding of the reality appearing at this moment
all book knowledge is in vain. ``It is now, it is now'', she said time
and again. Many times she spoke about the Buddha's enlightenment and his
compassion to teach. Without his teachings we would not understand
anything about our life. We should have confidence that it is worthwhile
to study his teachings so that we shall know realities as they are, as
non-self, anattā. She explained that confidence will grow along with
right understanding. We should live in order to understand realities.
That is the goal of our life.




\chapter{The Dream of Life}

Acharn Sujin came at the first session
immediately to the essence of the Buddha's teachings: ``The truth is
here and now but it is hidden. It is difficult to find out what it is.
There is seeing and visible object, but no one is there. Visible object
could not appear without seeing.''

If the Buddha had not realized the
truth of all phenomena of life we would be still in darkness. We would
be dreaming about all that occurs in life. We cling to people and we
believe that all that we experience can stay, at least for a while. The
Buddha woke up to the truth and he taught that whatever object is
experienced has to fall away, that it is impermanent. He taught that
there is no person, no self, only ever-changing mental phenomena and
physical phenomena. 

Because of ignorance and clinging we
keep on dreaming about life. The Buddha spoke time and again about
seeing and the object that is seen, visible object, hearing and sound,
smelling and odour, tasting and flavour, the experience through the
bodysense and tangible object, and the experience of objects through the
mind-door. There are six doorways and through each of these doorways an
object can be experienced. 

As the Buddha taught: there is seeing
and visible object but there is no one there. This is his teaching of
non-self, anattā. We are taken in by all objects we experience and we
take them for something that exists, for self. He taught in detail about
two kinds of reality: that which experiences, in Pāli: nāma, and that
which does not experience anything, in Pāli: rūpa. 

We are used to thinking of our mind and
our body as if they were existing. Mind is actually a momentary reality,
it arises and falls away all the time. It is citta or consciousness that
is accompanied by mental factors, cetasikas. Only one citta arises at a
time and then falls away immediately to be followed by the next citta.
The arising and falling away of the cittas that succeed one another is
so rapid that it seems that there is one citta that lasts. It seems that
seeing lasts, but in reality there is a rapid succession of cittas. 

Every citta is accompanied by several
cetasikas. After seeing there may be like or dislike of what is
experienced. Like and dislike are not cittas but cetasikas. Citta is the
chief in knowing an object and the cetasikas that accompany citta
perform each their own function while they share the same object. 

Every citta experiences an object.
Seeing experiences what is visible, visible object. Hearing is another
citta that experiences sound. Smelling is again another citta that
experiences odour. Visible object, sound and odour are sense objects,
objects experienced through the senses. These objects do not know
anything, they are experienced by cittas. We are so absorbed in the
objects we experience that we forget that they could not appear if there
were no citta that experiences them. 

Acharn emphasized time and again that
we may think and speak about realities that do not experience anything,
and realities that experience an object, citta and its accompanying
cetasikas, but that this is only book knowledge. The Buddha taught about
these realities so that one would attend to their characteristics when
they appear at the present moment. Acharn would always help the
listeners to attend to the characteristic of what is appearing right
now. She often spoke about seeing and visible object since seeing and
visible object appear time and again, at the present moment. We believe
that we see people and many different things such as trees and
mountains, but we are mistaken. Seeing only sees what impinges on the
eyesense, visible object. It arises and falls away very rapidly and
afterwards thinking of what is seen arises. Since cittas arise and fall
away so rapidly, it seems that seeing and thinking appear at the same
time, but this is not so. 

We should study very carefully and respectfully each word of the
Buddha's teachings. Today we may understand very little of what he
taught, but when we listen to his teachings again and consider again,
understanding can grow very gradually. We shall have more confidence in
his teachings when we see that what he taught can be verified in our
daily life.

Those who attended the Dhamma sessions
listened very attentively and as the days of discussion passed by, there
was an increasing number of questions asked by the listeners. They were
used to thinking that a quiet place was necessary to have an attentive
mind, but now they heard that understanding of realities can be
developed no matter where one is and no matter when. In fact, there is
no attentive mind, only different cittas that arise because of their own
conditions. Nobody can cause the arising of seeing, and even so, nobody
can cause the arising of understanding of the present reality. 

In the morning there was a session of
two hours and in the afternoon again another session of two hours. Tam
Bach was an untiring interpreter, translating every day, until her voice
needed a rest for one day, and then ``Tiny Tam'' took over. Every day we
went out for a delightful luncheon at a different place. Our Vietnamese
friends took great care so that everybody was well served. When we went
out on our way to a restaurant we saw an old church built in the time of
the colonization by the French and we passed a lane of trees that were
taller than the houses, dating from the same time. There were not many
cars, but the streets were crowded by motorcycles. 

The motorcycles managed to follow each
their own way and one wondered how they could do this without
collisions. Sarah made a simile, saying that no policeman could make an
effort to try to direct the traffic, that everyone went his own way.
Even so, nobody can direct his cittas, they arise because of their own
conditions and then fall away. Cittas proceed according to their own
ways, they are not directed by a self, they are anattā. 

We are bound to confuse seeing and
thinking on account of what was seen. The Buddha taught what is real in
the ultimate sense and what is not real but only a concept we think of.
Without his teachings we would not know this. There are two kinds of
truth: conventional truth (sammutti sacca) and ultimate truth
(paramattha sacca). Visible object can only appear because there is
seeing, a mental reality. Seeing arises because of conditions: if there
were no eyesense, a material reality, rūpa, and visible object, another
rūpa, there could not be seeing. We are immediately absorbed in what is
seen and we often forget that if there were no seeing we could not think
about visible object. Sound could not be heard if there would not be
hearing. Nothing could appear if there were no citta that experiences an
object. We are inclined to take the reality that experiences for self. 

Because of remembrance we recognize
persons and we know that this is a motorcyclist and that is a tree. We
believe that all these things can stay. But in reality all phenomena,
mental phenomena and physical phenomena, arise and fall away
immediately. The whole day we live in the world of illusions, of dreams.
In our dreams we see people who speak to us and we speak to them. Our
impressions are very vivid, our dream world seems real. However, when we
are dreaming our eyes are closed and visible object does not impinge on
the eyesense. We are merely thinking, not seeing. Also when we are awake
we believe that we see people, mountains and lakes, but this is not
seeing, only thinking of concepts. Seeing only sees visible object that
impinges on the eyesense. 

Because of listening to the Dhamma
there are conditions to gradually learn to attend to the different
characteristics of realities that appear now. We can learn what is real
in the absolute or ultimate sense: citta, mental factors (cetasikas)
arising with the citta, and rūpa, physical phenomena. Everything else
such as persons or tables are merely ideas formed up on account of what
was experienced through the senses and the mind and these are concepts,
not realities. When we know the difference between realities and
concepts or ideas, there will be more understanding of what life really
is: realities, elements devoid of a
self. 

There is no mind that can stay and no
body that can stay. We may think of ourselves: ``I am sitting'', but in
reality there are only different bodily phenomena, rūpas, that arise
because of their own conditions and fall away. We speak of our body and
different postures it can assume, but these are notions of conventional
speech.

Realities appear one at a time through
the six doors. A rūpa such as hardness appears through the bodysense.
When something is touched, it appears. We do not think of it, it can be
experienced without naming it. Its characteristic is always hardness, no
matter how we name it, its characteristic is inalterable. It arises and
falls away immediately. But we can think about hardness and the cetasika
remembrance (saññā) which accompanies every citta remembers that it is
for instance a chair that is touched. A chair is a concept, not a
reality. 

We could not lead our daily life
without thinking of concepts. We have to know what different things such
as a chair or a computer are used for. But we can learn the difference
between realities and concepts. When we cling to a ``whole'' such as a
person or the body, we are ignorant of ultimate realities. Ultimate
realities appear one at a time through one of the six doorways. There is
the world of visible object, of sound, of odour, of flavour, of tangible
object or of objects experienced through the mind-door, thus, six
worlds. They appear one at a time. When several objects seem to appear
as a whole, as a collection of things, we are thinking of concepts, not
attending to ultimate realities. We are often entranced in the general
appearance and the details of things.

We read in the ``Dialogues of the
Buddha'' (Ch 2, the Fruits of the Life of a Recluse, 64) that the Buddha
said to King Ajātasattu:

``And how, O king, is the bhikkhu guarded as to the doors of his
senses?''

``When, O king, he sees an object with
his eye he is not entranced in the general appearance or the details of
it. He sets himself to restrain that which might give occasion for evil
states, covetousness and dejection, to flow in over him so long as he
dwells unrestrained as to his sense of sight. He keeps watch upon his
faculty of sight and he attains to mastery over it.''

The same is said as to the other sense-doors and the mind.

Before hearing the Buddha's teachings
we were ignorant of the six worlds. We only knew the conventional world
with people and things which seemed to last. Acharn explained that if
one does not understand the reality appearing at this moment one cannot
understand the teachings. There would only be thinking of ``my seeing'',
of person or self. There could never be the eradication of ignorance and
attachment to what appears now. Understanding of ultimate realities
grows extremely slowly and we cannot expect to understand immediately
everything the Buddha taught. 

Life is seeing, hearing, smelling,
tasting, touching and thinking. There is no self who can see, hear or
perform any function. This is the Buddha's teaching of anattā,
non-self.

One of the days in Saigon we had a
session in a meditation center in a high-rise ap-artment and after the
session our host had a delicious luncheon prepared for us on the roof of
the building. People listened very attentively during the session and
one girl started to cry because she found the idea of non-self
unbearable. Only the sotāpanna who has attained the first stage of
enlightenment\footnote{Those who have developed
understanding to a high degree can attain enlightenment. At that moment
nibbāna, the unconditioned dhamma, is experienced and defilements are
eradicated. There are four stages of enlightenment, and at each stage
defilements are successively eradicated until they are all eradicated at
the stage of arahatship.}
has eradicated wrong view and he does not cling to the idea of self
anymore. This shows that it is a long way to develop right understanding
of realities. 

We read in the ``Kindred Sayings on
Sense'' (Ch 4, § 85, Void) that Ānanda said to the Buddha:

```Void is the world! Void is the
world!' is the saying, lord. Pray, lord, how far does this saying go?''

``Because the world is void of the
self, Ānanda, or of what belongs to the self, therefore it is said `Void
is the world.'

And what, Ānanda, is void of the self
or of what belongs to the self?

Eye\ldots{
objects\ldots{eye-consciousness etc. are void of the self. That is why,
Ānanda, it is said `Void is the world.' ''

The Buddha explained that the world is
the eye, visible object, seeing, in short, all realities that appear in
our life. Nobody can create seeing, it arises because of its own
conditions. Seeing is not hearing, they are different dhammas arising
because of different conditions. Nobody can create his body. What we
take for our body are rūpas that are conditioned by kamma, by citta, by
temperature or by nutrition. They arise and fall away very rapidly.
Eyesense, earsense and the other sense-bases are rūpas conditioned by
kamma, they are the result of a deed done in the past. When we speak or
move our hand, there are rūpas conditioned by citta. Furthermore, heat
and nutriment condition the arising of rūpas time and again. When we
understand the conditions for the phenomena that arise, it will be
clearer that there is no person who owns them or who is master of them.
This is the beginning of understanding the world according to the
Buddha's teaching. 

*******



\chapter{What is Pariyatti?}

There are different levels of
understanding. One classification is: pariyatti or intellectual
understanding, paṭipatti or direct awareness of realities, and paṭivedha
or the direct realization of the truth. 

Pariyatti is intellectual understanding
but it is not theoretical; it always pertains to the reality appearing
at this moment. It is not merely knowing terms, but it is considering
the dhamma appearing now, be it sound, visible object, seeing or
attachment. 

Acharn Sujin said: ``When we read or
listen to the teachings we believe that we understand, but actually
there is no understanding. Seeing seems very common, everyone sees. But
is there any understanding of it? Usually there is the idea of `I see a
body or a table'.''

Acharn spoke many times about seeing
and visible object, because there is seeing time and again, but we know
very little about it. Seeing sees only visible object that impinges on
the eyebase, no people, no things. There can be more intellectual
understanding, pariyatti, of seeing and visible object while we are
seeing. I said to Acharn that I try to see visible object, and she
answered that trying is useless, that we are wasting our time. We try
with attachment, lobha, to find out what it is, but that is not
understanding. It is good to be reminded of the truth. Seeing that it is
wrong to try to know realities is the beginning of right understanding.
Pariyatti is not trying to know or doing something specific in order to
have direct understanding. When we listen and we consider carefully of
what we hear there can be more understanding of what appears now. 

Acharn said: ``Develop the
understanding that there is no one there, only different
realities\ldots{ Seeing is the experience of what appears now. Without
the eyebase there can never be seeing. This is the beginning of more
precise understanding of what is real at this moment. Seeing is not that
which is seen. That which is seen cannot experience anything. From now
on we learn to understand the moment of seeing, even without using any
words. Seeing sees and that which is seen appears now. These are two
different kinds of realities: a reality which experiences (nāma) and a
reality which does not experience anything (rūpa). Realities are always
uncontrollable. Where is the self who can control anything?''

The idea of self is bound to come in
time and again, until it is eradicated at the first stage of
enlightenment, the stage of the sotāpanna. When seeing appears it is
time to carefully consider its characteristic, and then no words are
needed. Seeing is a citta, it is different from the rūpa that is visible
object. Citta is an element that knows something. It cannot be seen, it
has no shape or form. 

Acharn remarked: ``There is no hearing,
then there is hearing, and then no hearing. So what is the use of
hearing? Hearing is gone and the sound is gone. This is life from moment
to moment. No thing, no one, only the arising and falling away of
realities by conditions.''

If there were no specific conditions
for the arising of dhammas, they could not arise. Another way of
reminding us of this truth were her words: ``There is nothing, then
something and then nothing again.'' Realities seem to appear out of
nowhere, they arise because of their appropriate conditions and then
they are gone immediately, never to return. It is unpredictable what
reality will appear the next moment, no one can prevent it from arising.
Kamma, a deed done in the past, even countless lifetimes ago, can
produce vipāka, result, at this moment in the form of body-consciousness
accompanied by painful feeling. The pain can be unbearable, or
dying-consciousness may arise. Death can come at any time. 

Seeing, hearing and the other
sense-cognitions arise in a process of cittas that succeed one another.
After seeing has fallen away it is succeeded by other cittas that do not
see but still experience the visible object that has not fallen away
yet. Like or dislike of the object may arise and then there are akusala
cittas. Like and dislike are cetasikas that accompany akusala cittas.
The Buddha taught about kusala (wholesome) and akusala (unwholesome).
Kusala does not lead to the harm of oneself nor of another. Akusala
leads to the harm of both. 

There are certain mental factors,
cetasikas, that are roots. A root is the foundation of the citta. Every
akusala citta has ignorance as a root (in Pali hetu). Some have in
addition to ignorance attachment and some have in addition to ignorance
aversion or anger. When we dislike something the akusala citta is rooted
in ignorance and aversion. Aversion has many degrees, it can be dislike,
fear or even anger.

Kusala cittas have non-attachment and
non-aversion as roots and they may be accompanied by paññā as well.
There are also other types of cittas that are not kusala nor akusala,
but vipākacittas and these are results of kamma. Kamma, a wholesome or
unwholesome deed committed in the past can bring results today in the
form of seeing, hearing, smelling, tasting or experiencing tangible
object through the bodysense. These can be pleasant experiences or
unpleasant experiences. Kamma also produces rebirth that can be a happy
rebirth such as in the human plane or in a heavenly plane, or an unhappy
rebirth in a woeful plane such as a hell plane or the animal plane,
depending on the nature of kamma.

There are also cittas that are neither
kusala, nor akusala, nor vipāka and these are kiriyacittas (inoperative
cittas).
The understanding should be developed
that there is no one there, only different realities. We dislike akusala
citta and we may try not to have it, but it arises because of
conditions. We had attachment and ignorance in the past and these are
accumulated in the citta from moment to moment, from life to life.
Acharn said that we should be courageous to understand akusala as a
conditioned dhamma instead of trying not to have it. Otherwise akusala
can never be eradicated. It is developed understanding, paññā, that can
eventually eradicate akusala at the different stages of enlightenment. 

Acharn explained time and again that
pariyatti has to be understanding of the dhamma that appears now. It is
not different from this moment. The texts help us to consider realities
and to know that they appear now. 

Robert Kirkpatrick who is a professor
at the university in Kuwait visited us for a few days in Saigon with his
wife and little baby Ryan and older daughter Roxanne. On this occasion
Acharn Sujin gave us another opportunity for a Dhamma session in English
in Khun Noppadom's room in the hotel. Although Acharn had spoken on
Dhamma during the morning and afternoon for the Vietnamese listeners,
she graciously gave her time also for Dhamma in the evening. Baby Ryan
who was not yet two years old gave her a ``wai'' greeting, a respectful
greeting with clasped hands. 

Acharn had explained that pariyatti can
condition paṭipatti, direct awareness of realities. When there is not
yet direct awareness, it shows that pariyatti has not developed enough.
We were wondering when it would be enough. I remarked that it never is
enough. Acharn answered: ``So long as you hope, it never is enough.'' We
were reminded that hoping, which is attachment, is not self and that it
is only a dhamma that should be known as it is. It is most beneficial to
be reminded when attachment, even slight attachment, arises. It hinders
the process of the development of paññā if it is not realized as a
dhamma. 

She also said that the Buddha did not
only speak about pariyatti, paṭipatti or direct awareness of realities,
and paṭivedha or the direct realization of the truth. He also spoke
about the three rounds of understanding the noble Truths: sacca ñāṇa,
the firm understanding of what has to be known and what the Path is;
kicca ñāṇa, understanding of the task, that is,
satipaṭṭhāna\footnote{The development of right
understanding of nāma and rūpa as they appear at the present moment.}
kata ñāṇa, understanding of what has been realized, the realization of
the truth\footnote{See Kindred Sayings V, Kindred
Sayings about the Truths, Ch 2, § 1, The Foundation of the Kingdom of
the Dhamma.}
When pariyatti has become firm and more accomplished it is sacca ñāṇa.
Then one does not move away from the dhamma appearing right now and turn
to other practices in order to understand the truth. Sacca ñāṇa realizes
that every dhamma that arises is conditioned. 

If the Buddha had not taught the three
rounds of understanding we would have misunderstandings about pariyatti.
We would believe that when we had studied the texts sufficiently and
understood what the right Path is, pariyatti was already sufficient. We
would believe that it was the right time for the arising of direct
awareness. But then one underestimates the importance of pariyatti. The
explanations about pariyatti were most beneficial and they made it clear
that pariyatti pertains to what appears right now. At the same time we
should remember that it has to be kusala citta accompanied by
understanding that attends to what appears now. It is not self and it
can only arise because of conditions. We need patience to gradually
develop understanding of the dhamma appearing now. 

One of the Vietnamese listeners asked
whether there is any method to control the mind. There is no method.
After seeing, hearing and the other sense-cognitions, akusala cittas
usually arise. We have accumulated attachment and ignorance and,
therefore, they are bound to arise. The development of paññā is the only
way leading to the end of defilements. Also what is akusala should be
known as it is, as only a dhamma. 

We read in the ``Gradual Sayings'',
Book of the Threes (I, Ch 3, § 71,
Ānanda)\footnote{I used the translation by Ven.
Bodhi, in the Numerical Discourses of the Buddha.}
about a conversation between Ānanda and Channa, a wanderer. Channa asked
Ānanda whether he taught the abandoning of lust, hatred and delusion and
when Ānanda said that he did Channa asked what the danger was he had
seen in them. Ānanda explained:

``One excited by lust, friend, overcome
by lust, with mind obsessed by it, intends for his own affliction, for
the affliction of others, and for the affliction of both and he
experiences mental suffering and dejection. But when lust is abandoned
he does not intend for his own affliction, for the affliction of others
and for the affliction of both, and he does not experiences mental
suffering and dejection\ldots{.'' 

He then explained that one, because of
lust, hatred and delusion is engaged in misconduct by body, speech and
mind and one does not understand one's own good, the good of others and
the good of both. But when defilements are abandoned the opposite is the
case. 

Ānanda said: ``Lust leads to blindness,
loss of vision, and lack of knowledge; it is obstructive to wisdom,
aligned with distress, and does not lead to nibbāna.''

He said the same about hatred and
delusion and about the abandoning of them. Channa then asked whether
there is a Path abandoning defilements. Ānanda said that it is the
eightfold Path.

The eightfold Path are the cetasikas of
right understanding, right thought, right speech, right bodily action,
right livelihood, right effort, right mindfulness and right
concentration. Not a self practises, but right understanding develops
and it is assisted by the other path factors. Right thinking as factor
of the noble eightfold Path has to accompany right understanding. It
``touches'' the nāma or rūpa which appears so that paññā can understand
it as it is. Right effort is energy and courage to persevere being aware
of nāma and rūpa which appear one at a time through the six doorways.
Because of right effort one is not discouraged but one continues
developing understanding. At the moment of mindfulness of nāma and rūpa,
right effort has arisen already because of conditions and it performs
its function; we do not need to think of making an effort.

Right mindfulness is mindful of the
nāma or rūpa which appears so that understanding of that reality as
non-self can be developed. When right understanding realizes the true
nature of the nāma or rūpa which appears, right concentration assists
the citta and the accompanying cetasikas to cognize that object.

Right speech, right action and right
livelihood are actually three cetasikas which are abstinence from wrong
speech, wrong action and wrong livelihood. They may, one at a time,
accompany kusala citta when the occasion arises. They do not accompany
each kusala citta. While we abstain from wrong action or speech there
can be awareness and right understanding of nāma and rūpa. Paññā can
realize that the cetasika which abstains from akusala is non-self, that
it arises because of its appropriate conditions. The path-factors of the
noble eightfold Path lead to deliverance from the cycle of birth and
death. At the moment of enlightenment lokuttara citta arises that
experiences nibbāna and eradicates defilements. At that moment all three
abstinences accompany the lokuttara citta. They fulfil their function of
path-factors by eradicating the conditions for wrong speech, wrong
action and wrong livelihood. The path-consciousness, magga-citta,
eradicates the tendencies to evil conduct subsequently at the different
stages of enlightenment. When the last stage of enlightenment, the stage
of the arahat, has been realized, all defilements are eradicated and
there will not be rebirth for him. 

An excursion to My Tho was organised
where one would go on a boat. At first I did not want to go since I
thought that a boat trip would be too difficult for me. But when I heard
that the boat trip would be on the Mekong River, the word ``Mekong''
evoked many memories. I decided that I just had to go. When my late
husband Lodewijk and I were posted in Bangkok (in 1966), Lodewijk was
very actively engaged with the Mekong Comittee and its many projects for
the development of the four Mekong countries of Cambodia, Laos, Vietnam
and Thailand. He pleaded successfully for a larger donation of money
from our government in the Hague. Our Vietnamese friend Thai had
organised the excursion and he made it into an unforgettable experience
for me. In the bus he gave a touching speech explaining how meaningful
this trip was for me. He had very thoughtfully arranged for a taxi when
the distance was too long for me to walk. When we were on the boat
different projects were explained to us: about the way the rice fields
obtained water from the Mekong when the tide was high, about the
importance of the coconut plantations for the export, about the fibre
made from coconut for sleeping mats.

We had lunch on an island and after
that there was a Dhamma discussion. The discussion was about seeing at
this moment and how ignorance and attachment always lead us away from
the present moment. This is very true. We may have thought that we were
on the right Path, but each time there is wishing for more progress we
are clinging to self. Pariyatti is not yet firm enough to be sacca ñāṇa.
Its development depends entirely on conditions and I am most grateful
that Acharn made it very clear that ``no one can do anything''. She
said: ``It it is impossible to go against the current of ignorance.
Listen and listen, this prevents turning away from the Buddha's words.
There is always the idea of self trying. Seeing now sees, not I.'' 




\chapter{What is Sati?}

Every kusala citta is accompanied by
sati, a beautiful (sobhana) cetasika. Sati, mindfulness or awareness, is
non-forgetful of what is wholesome. There is sati of the level of dāna,
generosity, of sīla, morality, of calm and of satipaṭṭhāna, the
development of right understanding of realities. When sati arises it is
a condition for seeing the value of kusala and the disadvantage of
akusala. When there is an opportunity for kusala such as dāna,
generosity, there is usually forgetfulness and one wastes this
opportunity. Or, there may be forgetfulness when there is an opportunity
for sīla, that includes not only abstention from unwholesome deeds, but
also helping others or politeness. When sati arises it is non-forgetful
of kusala and there are conditions for generous giving, abstention from
unwholesome deeds or helping others. 

Sati of the level of
satipaṭṭhāna is non-forgetful,
mindful, of the nāma or rūpa
appearing at the present moment.
Nāma and rūpa are ultimate realities,
different from ``conventional realities'' or concepts, such as person,
mind, body, animal or tree. We tend to think of a ``whole'' of mind and
body, of the human person. When we study the Dhamma we learn that what
we call mind are different types of citta accompanied by different
cetasikas, and that these arise and fall away all the time. What we call
body are different rūpas, some of which are produced by kamma, some by
citta, some by temperature and some by nutrition.

When sati begins to be mindful of the
present reality it is conditioned by right understanding of that
reality, by understanding of the level of pariyatti. When there are the
right conditions direct awareness of a reality can arise and at that
moment understanding can see it as only a dhamma, a conditioned reality
that is non-self. However, in the beginning understanding is very weak
and we cannot expect to understand the truth immediately. We may wish
for direct awareness, but that is clinging. Right understanding develops
along with detachment. Sati accompanies kusala citta, but kusala citta
is very rare. Mostly attachment arises after seeing, hearing and the
other sense-cognitions. We think of the objects that are experienced and
this is mostly with akusala cittas.

Some people believe that there is sati
when one knows what one is doing. When walking, one knows that one is
walking, when sitting, one knows that one is sitting. At such moments
one thinks about different situations and this does not pertain to
satipaṭṭhāna. It is merely thinking of concepts. One may think with
clinging to a self who is walking or sitting. There is no awareness of
realities as they appear one at a time through the six doorways. 

Someone asked questions about
detachment. Acharn answered:

``There should be right understanding from the beginning. What should be
the object of detachment now? There is seeing now.''

We take seeing for ``my seeing'', but
when it is understood as only a conditioned dhamma there will be less
inclination to take it for self. After seeing there are most of the time
ignorance and clinging. Before this life there were other lives and
during those lives there were ignorance and clinging. Akusala citta with
ignorance and clinging arises and falls away, but every citta is
succeeded by a following citta and in this way inclinations are
accumulated from moment to moment, from life to life. That is why the
development of right understanding of realities is an endlessly long
process. It begins by listening to the teachings and carefully
considering what one has heard so that very gradually understanding can
grow. Understanding of realities is the goal, not having many moments of
direct awareness. There is nobody who can do anything to have direct
awareness. Acharn said: ``It arises unexpectedly and who knows when? In
this life?\ldots{Sati can dart in like a flash, just as lobha. No one
chooses it. It is another conditioned dhamma. It arises unexpectedly,
not according to plan.'' Lobha just arises when there are the right
conditions, and even so, sati arises when there are the right
conditions. Nobody can cause the arising of sati. 

Sati and paññā are anattā, non-self.
Acharn said: ``They are all dhammas. Are we sure that what is seen now
is a reality, no one in it? It is only a reality that can be seen, no
doubt.'' She said that one may look into a mirror and believe that there
is someone there. But the mirror does not have anyone in it. What is
seen is only visible object and when one touches the mirror only
hardness may appear. This is the way to understand rūpa that appears and
not just by talking about it. Understanding begins with listening to the
Dhamma. 

Acharn said several times: ``Where is
the visible object? Where the four Great Elements arise there is visible
object''. Rūpas arise in groups, kalapas, and each group consists of at
least eight rūpas. Among these are the four Great Elements of Earth or
solidity, appearing as hardness or softness, of Water or cohesion, of
Fire or temperature, appearing as heat or cold, and of Wind, appearing
as motion or pressure. Furthermore there are: visible object, odour,
flavour and nutrition. Thus, when visible object appears, it is
accompanied by seven other rūpas, but only visible object is seen. The
accompanying rūpas condition visible object but they are not
experienced. Since the accompanying rūpas are varied they condition
visible object to be different every moment. For instance, solidity is
of different degrees of hardness, and temperature is of different
degrees of heat and that is why they condition visible object to be
different. This leads to thinking of different shape and form and
because of remembrance, saññā, a cetasika arising with every citta, we
recognize different people and things. While Acharn was explaining about
the different kalapas she pointed at the draperies, at the table and a
bottle. If visible object would not be different all the time we would
not recognize those things. The teaching of different kalapas helps us
to have more understanding of anattā. 

We may have intellectual understanding
of realities, but it may still be weak and it may be a long time before
direct awareness of realities can arise. Pariyatti conditions paṭipatti,
direct awareness and understanding, but it is useless to wish for it. We
have to continue to listen and consider the reality of the present
moment. We are so fortunate to have become acquainted with the Buddha's
teachings and to be able to consider the reality appearing now. This
moment is the right moment to consider and investigate the present
reality. Who knows when he will die? It may be this afternoon or
tomorrow. Rebirth in another plane of existence may not be favourable
for the development of understanding.

In the ``Gradual Sayings'' (Book of the
Ones, Ch VIII, § 4) we read about right understanding as being the most
precious in life - that which doesn't bring sorrow:

``Of slight account, monks, is the loss of such things as relatives.
Miserable indeed among losses is the loss of wisdom.

Of slight account, monks, is the increase of such things as relatives.
Chief of all the increases is that of wisdom\ldots{.

Of slight account, monks, is the loss
of such things as wealth. Miserable indeed among losses is the loss of
wisdom.

Of slight account, monks, is the
increase of such things as wealth. Chief of all the increases is that of
wisdom\ldots{.

Of slight account, monks, is the loss of such things as reputation.
Miserable indeed among losses is the loss of wisdom.

Of slight account, monks, is the increase of such things as reputation.
Chief of all the increases is that of wisdom. Wherefore I say, monks,
you should train yourselves thus: We will increase in wisdom. You must
train yourselves to win that.''

Acharn often said that what appears now
should be known as just a dhamma. We may think: ``It is just a dhamma'',
but it is very difficult to realize the truth and it may take a long
time. Whatever appears at the present moment should be known as only a
conditioned reality which is not something or somebody. Acharn said:
``Any moment there is understanding of visible object as that which is
seen, right now, understanding begins to know that it is only that which
is seen.'' She explained that when there is more understanding it can
let go of the idea that there is something there, someone there. She
said: ``When there is no detachment, there can be conditions for
attachment to follow instantly, all the time. Paññā has to be keener and
keener to examine everything that appears as it is. Otherwise we are
only talking about attachment, aversion and wrong understanding.''

Paññā has to be precise, it should not
be just theoretical understanding but closer to the moment a reality
appears. When we begin to listen to the Dhamma, there is sati with the
kusala citta, but not yet sati that is directly aware of realities. It
may take a long time before direct awareness and understanding arise,
but we have to continue to listen and consider what we hear. A few times
of considering the present reality is not enough. 

Rūpas have been classified as
twenty-eight, but only seven types are experienced all the time in daily
life. They are: visible object, sound, odour, flavour and tangible
object which is solidity, appearing as hardness or softness,
temperature, appearing as heat or cold and motion, appearing as motion
or pressure. Each of these rūpas is experienced in a sense-door process
and then in a following mind-door process. The objects experienced
through the six doorways are the six worlds and these do not last. They
are ultimate realities, different from concepts. We think mostly of
concepts such as people or different things, but thinking itself is a
reality that is conditioned. Nobody can force himself not to think of
concepts. 

One of the monks had to leave the
sessions earlier and he had a short conversation with Acharn. In
departing she said to him: ``Do not forget about ultimate realities.''
They appear all the time but we do not realize that they are just
dhammas. We are so absorbed in the stories we think of. When we are
involved in an accident we think of the whole situation and we believe:
``that is my kamma.'' However, in reality many different cittas arise
and they arise and fall away extremely rapidly. When we feel pain, there
is only a very short moment of body-consciousness accompanied by painful
feeling and this is vipākacitta, the result of akusala kamma. After that
we have aversion about the pain and this is akusala citta accompanied by
dosa, aversion. We keep on thinking about our suffering for a long time
with aversion. We are likely to take for the result of kamma what is
actually akusala citta. We think: ``Why does this happen to me?'' Acharn
spoke about the ups and downs of life and she said that life is
unbearable without paññā. When there is kusala citta accompanied by
right understanding there is no aversion, no disturbance about the ups
and downs of life. We shall be less inclined to think of a self who has
unpleasant experiences.
There are only dhammas occurring. Some
are cause, some are result, some are neither cause nor result. Whatever
happens are only conditioned phenomena, beyond control. There is no one
there who suffers. When we see the benefit of the teachings we can have
more confidence to continue to develop understanding. This is the most
precious in life.

Acharn spoke in the bus to our friend
Thai about Dhamma. He was holding the microphone all that time so that,
later on, we all could hear this conversation. She explained that the
world is quite different from the way it was before hearing Dhamma. She
said: ``Before understanding can be firmly established about the way the
world is, it takes time to develop it from hearing, considering. The
first time one has listened it is not enough. One cannot let go of
anything. Attachment is there, hiding. We can begin to let go.'' Thai
remarked that this is a long way. Acharn answered: ``A long way, so long
as ignorance and attachment are accumulated. There can be more
confidence in hearing, considering.''

Acharn explained that we live as in a
dream. We think of ourselves as being on the bus. When we will be at the
hotel everything is gone. In a dream there is always ``I'' and different
things. In our life different objects appear, just one characteristic at
a time. The most difficult thing is that the development of
understanding has to be natural. Not a person develops understanding, it
is understanding that develops because of the right conditions. When
there is no expectation realities appear as anattā. 

Nāma and rūpa arise and fall away
extremely rapidly. Every citta is followed by another citta without any
interval, and thus, it seems that citta can stay. Rūpa does not fall
away as rapidly as citta, it lasts as long as seventeen moments of
citta. However, it still falls away rapidly and so long as there are
conditions it is followed by another rūpa. All the sense objects we
experience at this moment seem to last, it seems they were there already
for a while when we experience them. Time and again Acharn reminded us
that dhammas arise and fall away and that only a sign or nimitta is left
of them when they have fallen away already. There is a sign or nimitta
of each of the five khandhas that arise and fall away: of the khandhas
of rūpa, of feeling, of remembrance (saññā), of the other cetasikas
(saṅkhārakkhandha) and of citta (viññāṇa-kkhandha). This is saṅkhāra
nimitta, the nimitta of conditioned realities
\footnote{The ``Path of Discrimination''
(``Patisambhidhamagga''), I, 438 speaks about seeing as terror the signs
(nimittas) of each of the five khandhas, whereas nibbāna is animitta,
without sign. Saṅkhāra nimitta also occurs in the Visuddhimagga XXI, 38.}
It is impossible to have awareness and direct understanding of just one
unit of rūpa or one citta. She said: ``We do not have to think of
nimitta, it appears. There is a little glimpse of that which is seen,
different from when one closes one's eyes. But we should not try to know
it, it is useless. One begins to see the rapidity of the arising and
falling away of realities. One clings to that which has gone. Only a
sign is left.'' 

We cannot catch visible object that
arises now, but there is seeing again and again and visible object
appears again and again. This gives us an idea of continuity. We think
of shape and form and because of the cetasika remembrance, saññā, which
marks and remembers the object that is experienced, we recognize
different people and things. Acharn said: ``There are ignorance and
attachment. There are always ideas of people in the world until right
understanding can see that this is the world of nimitta, and that
nothing is permanent.'' 

Jonothan remarked that we do not have
to stop seeing people. Thinking of people is conditioned, it is real.
The teaching of nimitta is beneficial, we can come to see that dhammas
are beyond control, anattā. It is useless to cling to what has gone
already. One may try to focus on visible object or another sense object
in order to know it, but it has gone when we try to think about it. When
people asked whether nimitta is a concept or reality Acharn answered:
``It is now.'' Instead of thinking of words there should be
understanding of what appears now. 

After
our sojourn in Vietnam we returned to
Thailand and some of our Vietnamese friends joined us in order to have
more Dhamma discussions. There were sessions at the
Foundation\footnote{The Dhamma Study and Support
Foundation. This is the center where all sessions with Acharn Sujin take
place each weekend.}
and we went twice to Kaeng Krachan where we had Dhamma conversations. 

A Cambodian monk often came to the Thai
sessions at the Foundation. At a certain time in the morning he was
invited to have his meal and he went away. He tried to spend as little
time as possible with his meal so that he could promptly return to the
Dhamma session. At the end of my stay a short trip was organised to
Chiengmai and after that to Chiengrai which is higher in the mountains.
The Cambodian monk was also invited to join and in the plane to
Chiengmai he spoke during the whole trip about the Dhamma with Khun
Noppadom.

In Chiengmai I met with many old
friends again. We all had lunch together and after that there was a
Dhamma discussion. Acharn explained about contiguity-condition,
anantara-paccaya: each citta that falls away conditions the arising of
the next citta without any interval. Even so the last citta of this
life, the dying-consciousness (cuti-citta) is immediately succeeded by
the rebirth-consciousness of the next life. All our good and bad
inclinations are accumulated from moment to moment, from life to life.
Listening and considering the Dhamma is never lost: understanding is
accumulated and it can grow in the course of countless lives. 

Acharn explained about citta that
experiences an object. Every citta must experience an object. That
object is a condition for citta by way of object-condition. If sound
does not arise there cannot be hearing. Sound is object-condition for
hearing. We have to consider this now, while there is hearing. This is
the way to know what ultimate realities, paramattha dhammas, are which
each have their inalterable characteristic. Nobody can change their
characteristics. When we hear a harsh sound, the hearing is vipākacitta,
the result of former kamma. We cannot select what is heard, we cannot
say: let hearing not occur. Paramattha dhammas are different from
concepts and ideas that are merely objects of thinking. We live mostly
in the world of concepts but we should not try to change this. 

The second day in Chiengmai we visited
a sick person in hospital. We went there with a small group. The patient
was unable to talk, but he could understand very well what we were
saying. Before his sickness he used to attend the Dhamma sessions in
Chiengmai. Acharn inquired with utmost gentleness and kindness about his
wellbeing and then she spoke about Dhamma to him. This reminds me that
the Buddha, when he visited sick monks, would, full of compassion, first
inquire about their physical condition and then teach Dhamma to them.
Acharn asked each one of us to speak to our sick friend about what we
learnt during the discussions. I spoke to him about the usefulness of
discussions and asking questions. The day before I had encouraged the
listeners to ask questions, explaining that if one does not ask
questions, paññā will not grow. Discussions are a means to test one's
understanding and to consider more. This visit was a real family
gathering.

Later on we went by car to a meditation
center in Chiengrai. The person who had invited us had provided a
luncheon for us in a building high up on a hill. During the luncheon
Acharn talked to our hostess about Dhamma untiringly. Most people here
were used to being in a quiet place in order to meditate. Wherever
Acharn is, she will always speak about the reality appearing now. The
Buddha taught about seeing, hearing, smelling, tasting and the
experience of tangible object through the bodysense. They arise when
there are conditions for their arising and nobody can make them arise at
will. Understanding of them can be developed anywhere, at any time. The
Buddha taught in detail about realities so that people would come to
understand their nature of anattā. 

*********



\chapter{The Right Path and the Wrong Path} 

Akusala citta, unwholesome
consciousness, arises very often in a day and kusala citta arises
seldom. After seeing and hearing clinging to the object arises but this
is unknown. Acharn reminded us many times that the development of right
understanding is a process that takes a long time, even aeons. We have
accumulated life after life ignorance and wrong view, diṭṭhi. Diṭṭhi is
an akusala cetasika which may arise with clinging. We cling to an idea
of self, we want to have more awareness and understanding. Also
understanding is merely a dhamma arising because of conditions and it
cannot be manipulated by a self. We move away from developing
understanding of the present object when we have an idea of wanting to
do specific things in order to have more understanding. This is wrong
practice, sīlabbata parāmāsa, clinging to rites and rituals. Even when
we are slightly trying to have more understanding or we are thinking
about having more understanding, we are already on the wrong Path. 

I had a short discussion with Acharn.

Acharn: Would you like to have
satipaṭṭhāna
\footnote{Satipaṭṭhāna is the development of
right understanding of mental phenomena and physical phenomena appearing
at the present moment.}
right now?''

Nina: ``I would like to.''

Acharn: ``That is already wrong
practice, sīlabbata parāmāsa, clinging to rites and rituals.''

Only paññā can know when there is wrong
view and this is the beginning of right understanding. The sotāpanna who
has attained the first stage of enlightenment has eradicated wrong view
and wrong practice. Thus, when one has not reached that stage one can be
caught up with wrong practice without even realizing this. 

We can easily mislead ourselves. We may
turn our attention to what appears at the present moment, but there may
be a subtle idea of wanting to do this, or trying just a little. Sarah
said: ``A subtle path - the middle way of understanding very, very
naturally any time at all, without thinking about it or trying to make
it happen even a little.''

The word meditation is often used, but
we should know that there are two kinds of meditation or mental
development: the development of samatha or calm and the development of
insight or vipassanā. Also before the Buddha's time samatha was
developed. The aim of samatha is to become freed from sense objects and
from the attachment that is bound to arise on account of them.
Defilements can be temporarily suppressed in samatha but they cannot be
eradicated. The development of vipassanā is the development of right
understanding of nāma and rūpa as they appear in daily life. Nāma and
rūpa, ultimate realities, not concepts, are the objects of insight. The
aim is the eradication of the wrong view of self, ignorance and all
other defilements. 

In Kaeng
Krachan
we had very lively discussions with
one of our friends who said that he finds a meditation center useful in
order to clear up the daily rubbish. He found that after a few days the
mind becomes less restless, more calm, more concentrated. He could have
peace of mind. Actually, this is thinking about one's mind. But there is
no mind that stays. Every moment there is citta, and only paññā can
clearly know whether the citta is kusala citta or akusala citta.
Concentration, samādhi, is a cetasika that accompanies every citta. Its
function is to focus on the object that appears so that citta can
experience it. Each citta can experience only one object because of
samādhi. When samādhi accompanies akusala citta it is wrong
concentration, micchāsamādhi, and when it accompanies kusala citta it is
right concentration, sammāsamādhi. When someone tries very hard to
concentrate on a meditation subject in order to become calm, he may take
for right concentration what is actually wrong concentration with
attachment and wrong view. Some people think when there is no like or
dislike of an object and the feeling is indifferent that they have true
calm. In reality there may be citta rooted in ignorance, moha-mūlacitta,
which is accompanied by indifferent feeling. 

Those who want to develop samatha,
calm, with a suitable meditation subject should know when kusala citta
arises and when akusala citta. They should also have right understanding
of the meditation subject that can condition calm. Thus, for the
development of samatha paññā is indispensable. Our friend who found a
meditation center useful spoke about mettā as a suitable meditation
subject. Acharn asked him whether he knows the difference between
selfish affection and mettā when looking at his child. We all have
attachment and unselfish love alternately and it is most difficult to
distinguish between them. Only paññā can clearly distinguish between
them. Our friend listened attentively and he gave us in the end a
summary of what he learnt from the discussions with Acharn: ``Forget to
try. Do not move away from reality and from paññā. Paññā conditions
detachment from self.''

As soon as one wants to do specific
things in order to have more calm and understanding, there is an idea of
self and one is following a wrong practice. Understanding can be
developed of whatever object appears and there should be no selection,
no preference for particular objects or particular circumstances.
Someone in Vietnam wanted to change her lifestyle in order to have more
favorable conditions for the development of paññā. If one thinks in this
way one forgets again about anattā. One clings to an idea of self who
wants to do specific things. We encouraged her not to change her
lifestyle. It is because of conditions that we are in such or such
circumstances. We should come to know our natural inclinations in order
to see them as conditioned dhamma. Understanding can be developed no
matter where one lives, no matter what one's activities are. Seeing is
seeing everywhere and it experiences visible object. Paññā develops
along with detachment. 

On the occasion of Acharn's birthday we
went to a school near Kaeng Krachan. It is a school for poor children in
the neighbourhood and the support of this school was one of Khun
Duangduen's charity projects. A delicious and nourishing meal was
provided for the hungry pupils. We noticed that many of them stood in
queue twice with their plates to get another helping. Some of the
children had prepared a speech and did their utmost to read it out in
excellent English. We all helped in sharing out sweets and also a gift
of money for the school was handed to the teachers. It was an excellent
way of celebrating a birthday with loving kindness, mettā. 

We should not blindly follow others,
but carefully consider each word of the teachings. Acharn often referred
to the Mahā-Mangalasutta, the Greatest Blessings (Sutta Nipāta, II, 4),
stating: 

``Not associating with fools, but
associating with the wise.'' 

We read in the ``Book of Analysis''
(second book of the Abhidhamma, Ch 17, §901) about wrong friendship:

``Therein, what is `having evil
friends'? There are those persons who are without confidence, of wrong
morality, without learning, mean, of no wisdom. That which is dependence
on, strong dependence on, complete dependence on, approaching,
approaching intimately, devotion to, complete devotion to, entanglement
with them. This is called having evil friends.'' 

We have accumulated so much ignorance
and wrong view, and, thus, we can easily mislead ourselves as to the
truth. Some people believe, in order to develop right understanding,
that one must first clean oneself from impurities, but, when there is
ignorance one takes for kusala what is akusala. How does one know for
sure? There must be clear understanding of kusala and of akusala, and
this can only be right at the moment they appear. Paññā must be very
keen to know this and only theoretical understanding is not sufficient.
~

Before hearing the Dhamma we used to
believe that the whole wide world with all the people in it, and things
such as a table or a tree were real. Now we heard that we can think of a
person or a table but that these are not realities that can be directly
experienced through the five senses and the mind-door. Hearing or
listening to the Dhamma means listening to anyone who can explain
correctly what the Buddha taught in the Tipiṭaka. We do not have to
think of a particular person. Or it can be reading and studying the
Buddha's words. We always have to verify ourselves what we heard, it has
to be our own understanding, not someone else's. We touch what is called
a table and hardness may appear. We can find out for ourselves whether
hardness has an inalterable characteristic. Is hardness always hardness,
no matter how we call it? We learnt that it is an ultimate reality, a
paramattha dhamma, and we should verify whether this is true. It is
different from a table we can only think of. We can only verify the
truth of what appears now, what is real. This is not a matter of naming,
remembering numbers, it is not theory.

Nāma and rūpa appear one at a time and
each one of them has its own characteristic. These characteristics
cannot be changed. Seeing, for example, has its own characteristic; we
can give it another name, but its characteristic cannot be changed.
Seeing is always seeing for everybody, no matter whether it is seeing of
an animal or of any other living being. Concepts are only objects of
thinking, they are not realities with their own characteristics, and,
thus, they are not objects of which right understanding is to be
developed. Nāma and rūpa which are real in the ultimate sense are the
objects of which right understanding should be developed.

In the beginning paññā does not clearly
understand realities but the right way of its development can be
understood. One knows that the realities that appear at the present
moment should be considered and investigated again and again so that
understanding can grow. ~

During one of the Thai sessions in the
Foundation the ``Subrahmā Sutta'' (Kindred Sayings I, Ch 2, § 7) and its
commentary were discussed. 

The commentary to this sutta states
that the young deva Subrahmā who lived in the Tāvatimsa heaven went to
the Nandana park with a thousand nymphs. Five hundred of them sat with
him under the Parichattaka tree and five hundred climbed in the tree,
throwing garlands on him and singing for him. They suddenly deceased and
were reborn in the Avīci Hell. Subrahmā saw with his divine Eye their
fate and he knew that he would only live seven more days. He went to see
the Buddha and spoke about his anxiety. We read in the Subrahmā Sutta
(Kindred Sayings I, Ch 2, § 7)
\footnote{I used the translation of Ven.
Bodhi, in his ``Collected Discourses of the Buddha'', p. 149.}


``Always frightened is this mind,

The mind is always agitated

About unarisen problems

And about arisen ones.

If there exists release from fear,

Being asked, please declare it to me.''

``Not apart from enlightenment and
austerity,

Not apart from restraint of the sense faculties,

Not apart from relinquishing all,

Do I see any safety for living beings.''

The commentary states that, after he
followed the Buddha's advice, he became a sotāpanna. The commentary
explains that he developed the factors leading to enlightenment. At the
moment of right awareness and right understanding of the object
appearing through one of the senses or the mind-door, there is restraint
of the sense faculties (indriya samvara sīla). The sense-doors are
guarded. As to the expression ``relinquishing all'' this denotes
nibbāna. Only paññā developed to the stage of lokuttara paññā that
attains nibbāna leads to complete safety. 

Like Subrahmā, we are overcome by
attachment time and again. Ignorance of realities conditions clinging to
whatever object appears. We cling to beautiful colours, to pleasant
sounds, to delicious food, to dear people and to comfort. It seems that
pleasant objects can stay because each dhamma that falls away very
rapidly is followed by another dhamma. Subrahmā, after he saw the
unhappy rebirth of the five hundred Fairies and realized that he would
not live much longer, had a sense of urgency. Because of the Buddha's
words he saw the benefit of right understanding that is the only way to
safety from an unhappy rebirth. So long as there is ignorance and
clinging there cannot be freedom from the cycle of birth and death. 

There are several radio programs in
Thai of Acharn Sujin broadcasted every day. In the morning and in the
evening some of these programs are preceded by a poem recited by Khun
Unnop about the danger of ignorance and about the benefit of
understanding. The words are as follows:

``Because of ignorance we have to be in the world

Because of ignorance we suffer in misery

Because of ignorance we have to be deluded for a long time

Because of ignorance we associate with fools

Because of ignorance we burn ourselves and destroy ourselves.

Because of knowing the Dhamma we learn continuously

Because of knowing the Dhamma we are diligent in the development of
kusala

Because of knowing the Dhamma we clearly understand that there are no
beings, no persons,

Because of knowing the Dhamma we let go of self, we are free from
danger.''

The arahat who has eradicated ignorance
and all defilements has no more conditions for rebirth. We still have
conditions to be reborn again and again in future lives. We may believe
that rebirth is pleasant, but so long as one is not a sotāpanna one can
still be subject to rebirth in an unhappy plane, or rebirth in a place
where there is no opportunity for hearing the Dhamma. It is sorrowful to
continue being in the cycle of birth and death and it is a blessing to
be freed from the cycle. 



\chapter{Basic Aspects of Dhamma}

Acharn was asked to speak more about
the beginning of development of understanding, about basics. She said:
``The beginning is understanding what is real right now. What is the
nature of seeing now? Understand dhamma now, dhamma is not self. This is
the benefit of hearing\ldots{ Basic is listening to the truth of the
teachings, understanding what is now. Listening to him {[the Buddha{]
is basic. There is seeing right now, but there is no understanding of
anattā. There can be confidence that paññā can develop so that it really
understands that nothing can be taken for self. This is the beginning.
It has to be one's own understanding. Never move away from what
appears.''

We read in the ``Gradual Sayings''
(Book of the Eights, Ch 1, § 5, Worldly Failings) that the Buddha spoke
about eight worldly conditions: gain and loss, fame and obscurity,
praise and blame, contentment and pain. These worldly conditions are the
ups and downs of life. Actually, this sutta exhorts us to develop right
understanding of whatever appears at the present moment. The Buddha
spoke the following verse:

``Gain, loss, obscurity and fame,

And censure, praise, contentment, pain
-

These are man's states - impermanent,

Of time and subject unto change.

And recognizing these the sage,

Alert, discerns these things of change;

Fair things his mind never agitate,

Nor foul his spirit vex. Gone are

Compliance and hostility,

Gone up in smoke and are no more.

The goal he knows. In measure full

He knows the stainless, griefless state.

Beyond becoming has he gone.''

When we read about gain and loss and
the other worldly conditions we are inclined to think of a self who is
subject to such experiences. Our happiness seems to depend on the
pleasant worldly conditions, we attach great importance to them. We want
to be treated well by others and we are disturbed when people blame us.
We forget that there is no self who experiences gain and loss, praise
and blame. We forget that the worldly conditions are just dhammas, not
``me'' or ``mine''. We are inclined to think of a whole situation,
pleasant or unpleasant, but we have to grasp the deeper meaning of this
sutta. This sutta reminds us that the realities of our life arise and
fall away all the time, that they do not last and are beyond control.
They have their appropriate conditions for their arising and it is
unpredictable what will arise next. We may think for a long time about
the loss of a dear person, but in reality there is no one there, as
Acharn reminded us all the time. There are only nāma and rūpa which
arise because of their appropriate conditions and fall away rapidly,
never to return. Merely thinking of gain and loss, praise and blame will
not lead to detachment from the wrong view of self and to the
eradication of ignorance and all defilements. 

The Buddha taught for forty-five years
about paramattha dhammas, ultimate realities: citta, cetasika, rūpa and
nibbāna. Citta, cetasika and rūpa can be objects of understanding when
they appear now, one at a time. When we hear words of praise there are
actually many different dhammas: hearing hears only sound, not words of
praise, but thinking of a whole story of words of praise follows shortly
afterwards. We can learn that hearing is only a conditioned dhamma, that
sound is only a conditioned dhamma and that thinking is only a
conditioned dhamma. None of these is self or belongs to a self. Words of
praise are concepts we think of, they are not paramattha dhammas. 

One may wonder how it is known when a
concept is the object of citta and when a paramattha dhamma. It can only
be known just now. Now we seem to see people and different things such
as a computer or a table. The object of citta is not just visible
object, not just sound, not just one object through one of the six
doorways. There are wholes, conglomerations, a collection of things like
a person, like a landscape. We can be sure that we are mostly thinking
of concepts. When paññā has been developed to the degree of direct
understanding the difference will be clearly discerned. 

On account of what is experienced
through the senses, different feelings arise: pleasant feeling,
unpleasant feeling or indifferent feeling. The Buddha taught that
feeling is only a dhamma, a cetasika, accompanying every citta. He
taught about feeling in detail so that we would be less inclined to take
it for self or mine. Akusala cittas that are rooted in attachment,
lobha, can be accompanied by pleasant feeling or indifferent feeling.
Akusala cittas that are rooted in aversion, dosa, are accompanied by
unpleasant feeling. Akusala cittas that are rooted in ignorance, moha,
are accompanied by indifferent feeling. Kusala cittas are either
accompanied by pleasant feeling or by indifferent feeling. It seems that
feelings stay for a while, but they arise and fall away extremely
rapidly together with the citta they accompany. When we think of
pleasant feeling or unpleasant feeling it is gone already.

When we understand that whatever
feeling arises is only a conditioned dhamma we will be less inclined to
be led by feeling on account of happy experiences or unhappy
experiences. When we read a sutta about pleasant events and unpleasant
events, we should remember that every sutta points to paramattha
dhammas. We should carefully consider what we read so that there will be
conditions for the arising of direct understanding of realities later
on. 

At this moment we are seeing, hearing
and thinking. Seeing sees what is visible, what appears through the
eyes. We may believe that it lasts, but in fact it falls away. We cannot
make seeing arise whenever we want to and we cannot make it last. When
hearing arises, there is no longer seeing. Hearing experiences sound. It
seems that we can see and hear at the same time, but in reality this is
not so. Seeing arises because of the right conditions, because there is
eyesense and visible object. Hearing arises because of earsense and
sound. Seeing and hearing are different moments of consciousness, which
each arise depending on their own conditions. We cannot direct them, we
are not the owner of seeing and hearing.~

Thinking is again another moment of
consciousness. It arises because of its own conditions. There are
different moments of consciousness arising one at a time and falling
away immediately. We believe that I see, I hear, I think, but actually,
there are just different moments of consciousness. The Buddha taught
that there is no self, no person who can direct the phenomena of life.

Acharn said:
``Actually,
what is appearing now? Is there any understanding of it? It cannot be
taken for I. We can talk about satipaṭṭhāna for a long time but what
about this moment? Is there the realization that there is no one at that
moment at all? Most important is understanding that there is actually no
self. Begin with that understanding to follow each word, understanding
that there is no self, even if it is theoretical understanding\ldots{
When there is no understanding at all, it is `I' all the
way.
Even sammā-sati, right awareness, can
be followed by the idea of I, but it is not known. That is why paññā has
to be developed on and on, with the understanding of no self. Until all
realities can appear in daily life as no self.''

The rūpa that is
visible object can be experienced by
several cittas arising in a process. Only seeing sees visible object,
and the other cittas of that process do not see, but they perform other
functions while they experience visible object. When visible object,
sound or another sense object has been experienced by cittas arising in
a sense-door process, it is experienced by cittas arising in a mind-door
process. Thus, rūpa can be experienced through a sense-door and after
the sense-door process is over, it is experienced through the
mind-door.

Seeing, hearing and the other
sense-cognitions are vipākacittas, results of kamma, a wholesome or
unwholesome deed performed in the past. They are not the only
vipākacittas arising in our life. The first citta of our life, the
rebirth-consciousness (paṭisandhi-citta), was the result of kamma. It is
followed by life-continuum, bhavanga-citta, which is the same type of
vipākacitta and conditioned by the same kamma that produced our rebirth.
There are countless bhavanga-cittas arising in our life. They keep the
continuity in life so that we are the same individual during that
lifespan. Bhavanga-cittas do not experience objects impinging on the
senses and the mind-door, they experience the same object as the
rebirth-consciousness. They do not arise in a process of cittas that
experience objects through the six doorways. When we are fast asleep and
not dreaming there are bhavanga-cittas. At such moments we do not know
where we are and who we are. We do not see or hear. Also in between
processes bhavanga-cittas arise. 

Within a process not only vipākacittas
arise, there are kiriyacittas which are neither kusala citta nor akusala
citta and there are kusala cittas or akusala cittas, arising mostly in a
series of seven, that experience the object that impinged on a
sense-door or the mind-door. For example, the object that was seen or
heard can be experienced with attachment, aversion or understanding. We
cling to all sense objects and clinging arises very shortly after
seeing, hearing or the other sense-cognitions have fallen away. We are
ignorant of the different cittas that arise and fall away extremely
rapidly. We believe that we are still hearing a pleasant sound when in
reality the vipākacitta that hears has fallen away and akusala citta
that clings has already arisen. 

It is most beneficial that the Buddha
taught about the processes of cittas; it shows that the cittas that
arise and fall away extremely rapidly and perform each their own
function cannot be controlled. They are anattā. 

Someone asked me whether I was not
bored hearing the same thing over and over again throughout all these
years. I undertake long journeys to listen to Acharn Sujin and to have
Dhamma discussions with friends. It never is enough because an enormous
amount of ignorance and wrong view has been accumulated. We need courage
and patience. Patience to listen again, consider again. We need to be
reminded of the truth very often. When we hear that seeing does not see
people, only the rūpa that is visible object, that impinges on the
eyesense, we should not expect to really grasp the truth immediately. 

Acharn said: ``One can understand that
the development of right understanding is not just in one or two days,
not in a week, a month, not in a life time, more than that. It is better
to have a little understanding from time to time.'' 

This is an excellent reminder not to reach for levels of understanding
one is not ready for. We should be grateful that we can listen to the
teachings and begin to consider the reality that appears now. That is
pariyatti.

What is the Abhidhamma is a question
that is often raised. The Abhidhamma is an exposition of all realities
in detail. The prefix ``abhi'' is used in the sense of ``preponderance''
or ``distinction''. ``Abhidhamma'' means ``higher Dhamma'' or ``Dhamma
in detail''.

The Abhidhamma is not a theory one
finds in a textbook; the teaching of the Abhidhamma is about all the
realities that appear at this moment. The Abhidhamma teaches about
seeing, about thinking of what was seen, about all the defilements
arising on account of what is experienced through the senses and the
mind-door. We should study the Abhidhamma in order to understand what is
appearing now, otherwise the study is useless. The Abhidhamma brings us
back to the present moment all the time. Coming to know what appears at
this moment, be it seeing, hearing or thinking, will be more helpful
than thinking of a situation in the past or a feeling which arose then.
Because what arose is already gone and there is no way to clearly know
its characteristic.

Those who have not studied the
Abhidhamma may think of states of mind that can be there for a long
time. Through the Abhidhamma we know that cittas arise and fall away all
the time. After seeing has fallen away there are likely to be attachment
and ignorance, but they may be subtle, we may not notice them. One may
believe that kindness is a state of mind that can last for some time.
But even when doing kind acts, there arise in between seeing, hearing,
other sense-cognitions or defilements.

Understanding can be developed of
whatever reality appears. We should not mind whether it is kusala or
akusala, and if we mind it shows clinging to the self. Nobody can change
what has appeared already. It is important to know all kinds of dhammas,
not only mettā, but also seeing, hearing, attachment, anger, whatever
appears now.

Every citta is accompanied by mental
factors, cetasikas. Citta conditions cetasika and cetasika conditions
citta, they have to arise together. Some cetasikas accompany every
citta, such as feeling or remembrance (saññā), some accompany only
akusala citta or only kusala citta. The Buddha taught in detail about
the cetasikas which accompany kusala citta, the cetasikas which
accompany akusala citta and the cetasikas which accompany cittas which
are neither kusala nor akusala but vipākacitta (citta that is result of
kamma), and kiriyacitta, inoperative citta. If we do not know this
distinction we can easily be misled when we read the definitions of the
cetasikas. For example, loving kindness, mettā, arises with kusala citta
whereas attachment arises with akusala citta. Generally, people think
that affection and attachment to one's family members is good. Moreover,
they have an idea of ``my attachment''. When one enjoys the company of
dear people it is selfishness, not pure kindness. Kusala citta is always
accompanied by the cetasikas alobha, non-attachment and non-aversion,
adosa. 

During the discussions, questions were
raised about the meaning of different cetasikas such as effort, volition
and concentration. We should not be misled by words in translations.
Only direct understanding can know characteristics of realities. We read
in the texts about viriya, energy or effort, and some people want to
apply effort to have more understanding. Effort arises with many types
of cittas, it arises with kusala citta as well as with akusala citta.
When we are clinging to having more understanding and apply effort
thereto, there is wrong effort, akusala viriya. If we do not realize
when akusala citta arises and when kusala citta we may delude ourselves
and take wrong effort for right effort. As understanding of the level of
pariyatti grows, there is kusala viriya already because of conditions.
We do not have to think of applying effort. 

The cetasikas volition and
concentration arise with every citta. We may mistake volition and
concentration that are akusala for kusala. We may try very hard to focus
on visible object, concentrate on it, in order to know it as it is
without realizing that this is done with clinging. 

We should not try to experience
different cetasikas. Most important, they have to be known as just a
dhamma, a conditioned dhamma that is not mine or self. Paññā grows very
gradually, and if it is not known what nāma is, the dhamma that
experiences an object, different from rūpa, the difference between
cetasikas cannot be realized. Without the Buddha's teaching in detail
about realities we would delude ourselves our whole life.




\chapter{Confidence}

We read in the Vakkali Sutta (Kindred
Sayings (III, Kindred Sayings on Elements, Ch 4, § 87) that Vakkali was
sick and that the Buddha came to visit him. Vakkali said that he had
wanted to see the Buddha. The Buddha answered:

``Hush, Vakkali! What is there in
seeing this vile body of mine? He who sees the Dhamma, he sees me: he
who sees me, Vakkali, he sees the Dhamma. Verily, seeing the Dhamma,
Vakkali, one sees me: seeing me, one sees the Dhamma.'' 

Seeing the Buddha refers to the nine
lokuttara dhammas: the eight lokuttara cittas subsequently arising at
the four stages of enlightenment when the lokuttara dhamma that is
nibbāna is experienced by these lokuttara cittas. When one has attained
enlightenment one comes to understand what Buddhahood means. The Buddha
always taught about dhamma or reality appearing at the present moment,
so that people would come to understand the true nature of anattā.
Without the Buddha's teaching, we would be completely ignorant of
dhammas such as seeing and visible object, hearing and sound, of all
objects that can be experienced one at a time through the six doorways.


During the Dhamma sessions in Vietnam
Acharn would often refer to the Buddha who taught us about all that is
real out of compassion. She said: ``Who is the Buddha? The greatest
understanding and compassion.'' The highest respect we can pay him is
studying carefully each word of his teachings. 

After his enlightenment the Buddha
pondered about the profundity and subtlety of the Dhamma. We read in the
``Kindred Sayings'' (I, Ch 6, Brahmasaṃyutta, I) that the Buddha spoke
the following verse
\footnote{I used the translation by Ven.
Bodhi in his ``Connected Discourses of the Buddha''.}
: 

``Enough now with trying to teach

what I found with so much hardship;

This Dhamma is not easily understood

By those oppressed by lust and hate.

Those fired by lust, obscured by
darkness,

Will never see this abstruse Dhamma,

Deep, hard to see, subtle,

Going against the stream.''

The Brahmā Sahampati asked him to teach
Dhamma. The Buddha surveyed the world with the eye of a Buddha and saw
that some beings had little dust in their eyes and some much dust. Out
of compassion he decided to teach Dhamma.

The teaching of Dhamma goes against the
stream of common thought, because many people do not accept that there
is no self, that all realities are beyond control. It is difficult to
really comprehend the nature of anattā, but there can be a beginning of
understanding of the reality appearing at this moment. Very gradually
understanding can grow. Acharn said that one cannot know who the Buddha
is without understanding his teachings. She compared our understanding
and his understanding with the earth and the sky. Before hearing his
teachings we did not know that seeing at this moment is a dhamma, a
conditioned reality that is non-self. We did not know that noticing
people is not seeing but thinking. We did not see the danger of clinging
to sense objects, to being enslaved to them. We did not see the danger
of ignorance of realities. The development of intellectual understanding
can lead to direct understanding of realities, which is different from
thinking about them. If seeing is not the object of direct
understanding, the idea of self cannot be eliminated. Seeing arises and
falls away and it never returns. 

When there is a degree of understanding
of whatever appears now, we can have gratitude to the Buddha's teachings
and contemplate his wisdom, his purity and his compassion. Since
ignorance and attachment were accumulated for aeons, from life to life,
the development of understanding is bound to take an endlessly long
time. However, it is the only way leading to the eradication of
ignorance and all other defilements. 

During the discussions in Vietnam
Acharn was speaking about the observance of the precepts. We take
avoidance of akusala for self. She said: ``What reality avoids? The
cetasika that is abstinence (virati). There is no one at all. That is
the difference between the teaching of the Buddha and that of others.
Buddhaṃ saraṇaṃ gacchāmi, Dhammaṃ saraṇaṃ gacchāmi, Sanghaṃ saraṇaṃ
gacchāmi\footnote{I take refuge in the Buddha, the
Dhamma and the Sangha.}
I take my refuge in the Buddha. He is my refuge.'' 

One may wonder why it is necessary to
develop right understanding of such common realities as seeing and
hearing. They appear time and again and they are conditioned dhammas
that arise and fall away never to return. We take them for self and
mine. Understanding can come to see them as just conditioned dhammas. As
understanding develops there will be more confidence that it is
worthwhile to know the truth of all realities appearing in daily life.
We shall learn that there are two types of realities: the reality that
experiences an object, nāma, and the reality that does not know
anything, rūpa. First we know this by intellectual understanding, but
this is not sufficient to eliminate clinging to a self. When
understanding grows nāma and rūpa can be directly understood as
non-self, without the need to think about them. 

Very often during our discussions
Acharn mentioned the importance of confidence in the Buddha's teachings.
Confidence, saddhā, is a sobhana (beautiful) cetasika arising with every
kusala citta. It has purifying as characteristic and it sees the benefit
of kusala. Without confidence there cannot be any way of kusala, be it
dāna, sīla, the development of calm or the development of right
understanding of realities. 

The Expositor (Atthasālinī, commentary
to the Dhammasangani, the first book of the Abhidhamma I, Part IV,
Chapter I, 119) states that confidence is the forerunner of all kinds of
kusala. When we see the benefit of kusala we apply ourselves with
confidence to whatever type of kusala there is an opportunity for. 

The conventional sense of confidence or
faith may be misleading. One may mistake faith that goes together with
attachment and happy feeling for the reality of confidence, saddhā.
However, confidence, saddhā, is kusala and it always goes together with
detachment. It is difficult to know its characteristic, but paññā, when
it is developed, can realize its characteristic. 

Confidence is an indriya, a controlling
faculty. It governs the accompanying dhammas, citta and cetasikas, in
its quality of purifying and of confiding in kusala. It overcomes lack
of confidence in kusala. Some dhammas are classified as indriyas, they
are ``leaders'' each in their own field. Five faculties are sometimes
referred to as ``spiritual faculties''. These are sobhana cetasikas
(beautiful mental factors) included in the ``factors of enlightenment''
(bodhipakkiya dhammas) that should be developed for the attaining of
enlightenment. They are: faith or confidence (saddhā), energy (viriya),
mindfulness (sati), concentration (samādhi) and understanding (paññā).

When confidence is still weak it is not
an indriya. There should be confidence that whatever appears can be
object of right understanding, and then one does not move away from the
present object. Confidence prevents wrong understanding and wrong
practice. One should not have an idea of ``I have confidence'', because
confidence is a dhamma that is non-self. For the sotāpanna who has
realized the four noble Truths saddhā has become an indriya. He has an
unshakable confidence in the Triple Gem. 

Acharn gave us invaluable advice
throughout our journey. She said: ``There can be confidence that paññā
can be developed so that it is really understood that nothing can be
taken for self. This is the beginning. It has to be your own
understanding. Never move away from what appears. Is there understanding
of anattā or is there still an idea that you can do anything? Otherwise
we just talk about the words of the teachings without any understanding
of anattā. It takes a long time. Understanding right now is accumulated
for the future. Right understanding now is the condition for right
understanding in the future, not by doing anything. 

Who understands the teachings of the
Buddha will listen more, consider more and leave it to anattā, because
the development of understanding is condi- tioned. If one tries to have
it, it is not anattā. What arises is the world: that which experiences
and that which does not experience anything.'' 

We have to leave the development of
understanding to conditions, to anattā. We are inclined to think that we
have to do something, to apply energy, in order to know the truth.
Acharn often reminds us that there is bound to be an idea of self who is
striving. By listening and considering the reality that appears at this
moment we can have confidence that understanding of the level of
pariyatti will become firmer. When the conditions are right, awareness
and direct understanding can arise. We are fortunate that we are in the
situation to be able to listen and study the Buddha's teachings. 

I am most grateful to Acharn Sujin for
reminding us all the time that there is no one there, only different
realities. One may wonder when understanding of the level of paṭipatti,
that is the level of direct awareness and understanding, can arise. When
we wonder about this it is already clinging to a self. As Acharn said,
we should leave it all to anattā, to conditions. We can appreciate more
the value of understanding of the level of pariyatti, intellectual
understanding of the reality appearing now. Without the Buddha's
teaching we would be completely ignorant of the present reality, we
would not know when there is clinging, even a little, to more
understanding. As Acharn reminded us, instead of wondering when there
can be direct understanding it is better to have a little understanding
from time to time.

The perfection of truthfulness or
sincerity is one of the perfections
\footnote{The perfections or pāramīs are:
generosity, morality, renunciation, wisdom, energy, patience,
truthfulness, determination, loving kindness, equanimity. The Buddha
developed these for aeons in order to become the Sammāsambuddha.
that should be developed along with right understanding leading
eventually to enlightenment. We need truthfulness to investigate all
realities of our daily life, our defilements included.}

Truthfulness is one of the perfections
that should be developed. Do we believe that we see persons? Doesn't it
really seem that we see them? Then we think of concepts. It is citta
that thinks. At that very moment we have to consider what citta is. We
have to consider the nature of citta, the element that experiences,
again and again, when it appears, with great patience. We all have so
much to learn. We have an idea that we touch a spoon, cup, fork, plate.
We have to be sincere and really consider what appears at the present
moment.~

Truthfulness, patience, courage, all
the perfections support the growth of right understanding so that it can
eventually perform the function of detachment from all conditioned
realities. We may take akusala citta for kusala citta, but we have to
verify the truth and this takes courage and patience. The development of
understanding of realities will take many lives, but even when right
understanding just begins to develop we come to know things we did not
know before. Instead of being distressed about our ignorance and
clinging we can be grateful to the Buddha who taught us the wisdom which
can eradicate all defilements. Being distressed or discouraged is
actually clinging to the idea of a self who wants more progress. It is
thanks to the Buddha that we can begin to develop right understanding of
the realities of daily life. Understanding of the present reality can be
developed while working in an office or doing house work, while talking
or being silent, while laughing or crying. When we begin to understand
the reality appearing at the present moment we may remember that this is
due to the Buddha's teachings and then there can be recollection of the
qualities of the Buddha.

We read in the ``Discourse on the
Simile of the Cloth'' (Middle Length Sayings I, no. 7) that the Buddha
speaks about the defilements of the mind which are: greed, covetousness,
malevolence, anger, malice, hypocrisy, spite, envy, stinginess, deceit,
treachery, obstinacy, impetuosity, arrogance, pride, conceit and
indolence. When the monk knows them as they are he can get rid of them.
The text states:

``When, monks, the monk thinks that
greed and covetousness is a defilement of the mind\ldots{ that
indolence is a defilement of the mind, and having known it thus, the
defilement of the mind that is indolence is got rid of, he becomes
possessed of unwavering confidence in the Awakened One and thinks: `Thus
indeed is he the Lord, perfected, wholly self-awakened, endowed with
knowledge and right conduct, well-farer, knower of the worlds,
incomparable charioteer of men to be tamed, teacher of devas and
mankind, the Awakened One, the Lord.' ''



