\part{The Shortness of Life} 

\chapter{Preface} 

Achan Sujin and other friends were invited to Vietnam by Tam Bach and other 
Vietnamese friends for a two week sojourn at the end of October 2016. Sarah 
and Jonothan were assisting Achan untiringly and with great enthusiasm during 
her Dhamma explanations. Friends from Thailand, Canada, Australia, Taiwan 
and myself joined this journey. In Vietnam, Tran Thai made the travel and accommodation arrangements for all of us. 

The Dhamma discussions took place in Hanoi. There had been Dhamma sessions before with Acharn Sujin in Hanoi and at that time there were fewer attendees than at this time when the number of listeners had grown; now there 
were about eighty of them. The listeners became more and more interested to 
understand the present reality. Among the audience were three monks, many 
`nuns', that is to say, women who wear robes and observe eight precepts, and 
many lay followers. Some had come with their parents. Tam Bach translated into Vietnamese the English Dhamma discussions and a team of Vietnamese 
friends translated the questions from the audience into English. We were also 
invited to the mountainous region of Sapa where the discussions were more informal and personal. 

Before going to Vietnam, I spent a few days in Bangkok, at the Peninsula Hotel. 
The King of Thailand, King Bhumibol Adulyadej, had just passed away and a 
period of national mourning was announced, which had been extended to a year. 
Every day the daily newspaper, the Bangkok Post, was dedicated to all the 
achievements of his Majesty. He truly was a man of the people, visiting all parts 
of the country to meet people personally in order to see in which ways he could 
solve their problems. With dedication and self-sacrifice, he set up many projects 
to raise their quality of life. He assisted the hill-tribe villagers who lived in very 
difficult circumstances. He made people replace opium crops with alternative 
crops such as peaches and apples. His Majesty spent several years in hospital 
because of ailing health and also during that time he only thought of the interests of his people. 

In Hanoi we usually had sessions of about six hours a day and in addition one of 
the monks wanted to discuss the conditions\footnote{In the seventh book of the Abhidhamma, the Patthana, twenty-four classes of conditions have been taught.} for the dhammas that arise. Also 
during previous sessions in Saigon we had discussed some of the conditions and 
now he was very interested to understand the way they condition dhammas that 
appear in daily life. 

During the sessions Achan reminded us time and again that we should carefully 
study each word of the Buddha's teachings. People are inclined to speak about 
concentration and calm, but first one should consider the meaning of these 
terms. Otherwise one talks about these subjects without understanding what 
they exactly are. 

\chapter[What is Life?]{}
\section*{What is Life?}

At the moment of seeing what is visible, life is seeing. At the moment of hearing sound, life is hearing. At the moment of smelling odour, life is smelling. At 
the moment of tasting flavour, life is tasting. At the moment of experiencing 
tangible object through the body sense, life is tactile consciousness. At the moment of thinking, life is thinking. We find our experiences most important, but 
we forget that each moment of experience is extremely brief. It arises because 
of its proper conditions and then it falls away immediately. We cling to all objects that are experienced through eyes, ears, nose, tongue, bodysense and mind, 
through these six doorways. We cling to them and take them for things that exist, that belong to us. 

Throughout our discussions we were reminded that realities that appear are impermanent and non-self, anattā. There is no self or person that coordinates all 
the different experiences through the senses. When seeing arises it sees visible 
object, and it is not a self or person who sees. Hearing is again another experience and it is not a self or person who hears. Each moment of experience is actually a moment of consciousness, in Pali: citta. 

Before hearing the Buddha's teachings, we only paid attention to the outward 
world around us, to all the objects that presented themselves. We did not realize 
that nothing could appear if there were no citta that experiences objects. 

There are many different types of cittas and only one citta arises at a time, experiencing one object. Each citta is succeeded by a next one, from moment to moment, from life to life. It seems that we see people and things, but seeing only sees visible object that impinges on the eyesense; persons or things cannot 
impinge on the eyesense. Very soon after seeing has fallen away, we think of 
persons and things and this shows how fast cittas arise and fall away and succeed one another. The thinking of people and things could not occur if there 
were no seeing, but seeing and thinking do not occur at the same time. The fact 
that there is not one type of consciousness that stays but many different types of 
cittas succeeding one another helps us to see that there is no self who could direct or manipulate realities. They all arise because of their own conditions and 
they fall away immediately. Acharn repeated very often: “There is no one”. 

We can acquire intellectual understanding of the reality that appears at this moment, but this is not the same as the direct experience of the truth. To show us 
that it takes a long time to develop intellectual understanding up to the level of 
direct understanding and the full realization of the truth, Acharn said that we are 
under the water, at the bottom of the ocean. It will take aeons before we rise up 
out of the water. From life to life we take realities for something permanent. 

One afternoon I had a conversation with three friends from Taiwan, with Vincent, his older brother Yung and Maggie. I said to them: 

“Now there is seeing, it is not I who is seeing but it is citta that sees. It falls 
away immediately. It is there for a very short time. Nobody can make seeing 
arise. That shows that it is anattā, non-self. If there were no eyesense and visible 
object, all that is visible, there could not be seeing. Hearing is again another experience, a different citta. This is not book knowledge, but it occurs right now. 
Hearing could not arise if there would not be sound and earsense. It falls away 
immediately, it is there for a very short time. After hearing has fallen away, 
there is another citta: a citta that thinks about the meaning of what is heard. 

Each citta falls away and is succeeded by a following citta, there is no gap in 
between. It is like this from moment to moment, from birth to death. 

There are many different types of citta: some are good or wholesome, kusala, 
some are unwholesome, akusala, some are neither. Citta does not arise alone, it 
is accompanied by several mental factors, cetasikas. Some cetasikas are good or 
wholesome, some are unwholesome and some are neither. They all arise because of conditions and they condition the citta they accompany. The citta that 
thinks is sometimes accompanied by cetasikas that are akusala, sometimes by 
cetasikas that are kusala. 

When the cetasika anger, dosa, accompanies the akusala citta, anger falls away 
together with the citta, but the inclination to anger is accumulated in the citta 
and it can be a condition that anger arises again. Anger, or aversion, is one of 
the three unwholesome cetasikas that are roots. The other two are attachment 
and ignorance. They are called roots because a root is the foundation of the citta. Besides the akusala cetasikas that are roots, there are many other akusala 
cetasikas, but each akusala citta is rooted in ignorance and attachmant, or in ignorance and aversion, or in ignorance alone. We have not only accumulated 
aversion, but also attachment. They arise time and again after seeing, hearing 
and the other sense-cognitions. We are taken in by pleasant objects and at such 
moments also ignorance of realities arises. 

Attachment to pleasant objects experienced through the eyes, the ears, the nose, 
the tongue and the bodysense and attachment to people is actually attachment to 
the self. We think of ourselves most of the time. When understanding has been 
developed more, we shall realize that whatever arises is a conditioned dhamma. 
Dhammas arise, nobody can make them arise and they fall away. 

There are three wholesome or sobhana cetasikas: non-attachment, non-aversion 
and understanding or wisdom. Each kusala citta is rooted in non-attachment and 
non-aversion and it may or may not be accompanied by understanding or wisdom. 
When there is a moment of understanding, it arises with the citta and falls 
away with the citta. But understanding can be accumulated and then there are 
conditions for it to arise again and again. I am inclined to think: `I understand', 
but it is understanding that understands.” 

Maggie asked: ``Does citta know or does paññā know?'' 

Nina: ``Citta knows or experiences an object but it is different from paññā that 
understands the true nature of the reality that is the object of citta. Panna can be 
accumulated and it develops little by little. 

What are realities? Citta, cetasika and rūpa are realities of our daily life. Citta 
and cetasika are nāma, mental realities that experience an object, and rūpa are 
all those realities that cannot experience anything. Intellectual understanding of 
the reality that appears now is a foundation for direct understanding.'' 

Question: ``How can intellectual understanding condition direct understanding?'' 

Nina: ``When we listen and consider what we hear, intellectual understanding 
develops and then it can condition direct understanding.'' 

Maggie's question: ``It is all for the purpose of liberation from the cycle of birth 
and death. I will then renounce (worldly life) and become a nun.'' 

Nina: ``It all has to be step by step. Do not forget this moment, we cannot think 
of the end of the cycle yet. The idea of `doing something' is already wrong.'' 

Question: ``What is the difference between conceit, mana, and wrong view of 
self, ditthi?'' 

Nina: ``They cannot arise together. Ditthi is wrong view, whereas when there is 
mana, you find yourself important, thinking, `Here I am'. It is compared to the 
waving of a banner.'' 

Question: ``They are attached to self?'' 

Nina: ``But each in a different way. They cannot arise together. Only the arahat 
(who has attained the fourth and last stage of enlightenment) has eradicated 
conceit, mana. The sotāpanna (who has attained the first stage of enlightenment) 
has eradicated wrong view but not conceit. When you compare yourself with 
someone else, you may think: `I am better, equal or less.' The sotāpanna does 
not have any idea of `I exist' but he can still think that his khandhas are better 
than the other's khandhas.'' 

Question: ``What is the present reality?'' 

Nina: ``We can ask ourselves: can it be directly experienced and through which 
doorway - through the eyes, the ears, the nose, the tongue, the bodysense or the 
mind-door? Can its characteristic be directly experienced, without naming it? 
Then we know that it is a reality. Seeing appears time and again. We can call it 
seeing or give it another name, but its characteristic is always the same. It experiences visible object.'' 

Citta is different from cetasika, it is just the faculty of knowing an object, it 
knows or cognizes an object. It does not like or dislike or understand the nature 
of the object. Cetasikas condition citta and citta conditions cetasikas. When we 
come to know that there are many conditions for each reality that arises, it leads 
to detachment from the view that they are permanent and self. 

Feeling accompanies every citta and we take it for self. There are happy feeling, 
unhappy feeling and indifferent feeling. When the feeling is unhappy and we 
wish it to be happy, we can see that we cannot control feeling. It has its own 
conditions for its arising. This shows us that it is beyond control, or anattā.'' 

In Hanoi there were several young people among the audience who asked questions about the way Dhamma could help them in daily life. They asked how the 
understanding of seeing, for example, could be of use in their social life. 

Seeing is an ultimate reality, different from conventional ideas. Before hearing 
the Buddha's teachings, we only knew the conventional world of people and of 
different circumstances. We did not know that life really is a moment of experience like seeing, hearing or thinking. Acharn explained many times that seeing 
is a reality and that there is no one who sees, that only seeing sees. Seeing arises 
because there are eyesense and visible object, it arises because of several conditions and then it falls away. How can that which falls away be controlled by a\footnote{Conditioned realities can be classified as five khandhas or aggregates: physical realities are 
rūpa-khandha, and four groups of mental realities are: feeling, vedana-khandha, remembrance, sanna-khandha, the other mental factors besides feeling and remembrance which are 
saṅkhāra-khandha, and all cittas, vinnana-khandha.}
self? Seeing is a mental reality, in Pali nāma. Eyesense and visible object are physical realities, in Pali rūpa. They do not experience anything. 

What we take for people are nāma and rūpa. We may wonder why a person is 
always bad-tempered? His accumulated inclinations are the condition for him to 
behave in this or that way. Understanding of conditioned realities leads to more 
patience when we are in the company of others. When we understand that we all 
have different accumulations, we shall become more tolerant of someone else. 
Acharn said: ``We can become an understanding person.'' When there is more 
understanding of our own accumulations, we will also understand the other persons in our surroundings, in our relationship such as married life. Through the 
Buddha's teachings, we will have more understanding of accumulated tendencies, wholesome and unwholesome. They condition the citta arising at the present moment. 

We all have different likes and dislikes and this is conditioned by accumulations 
of different tendencies. When we were in a restaurant in a village in Sapa, we 
had a Dhamma discussion after our lunch. After the discussion, loud music was 
being played. Vincent and I did not like this music and I was inclined to ask to 
have it less loud. Vincent's wife was a concert pianist and, therefore, he likes 
classical music and I like soft Baroque music and classical music. This is clinging, and all clinging is actually clinging to oneself. We noticed that most people 
liked the music, they were applauding time and again. Through the Dhamma, 
we can learn more about the clinging to ourselves in the different circumstances 
of life. The preference for certain objects and the clinging to these is caused by 
attachment to oneself. More understanding will lead to patience and tolerance 
when things do not turn out the way we wish. 

At this moment there are seeing, hearing or thinking. There can be only one 
moment of citta that experiences an object. Acharn reminded us that seeing 
cannot hear or think, and that hearing cannot see or think. These words sound 
simple, but they remind us to consider the truth again and again. 

Acharn said: ``There is always the idea of `I see someone'. That which is seen is 
nobody, nothing at all. Consider in order to begin to understand the truth of 
whatever the Buddha taught. This does not mean only listening without considering the truth of that which is heard. Without seeing, can there be the idea of 
something? Without any reality at all, can there be the idea of `I'? But what is 
taken for I is not permanent at all. Is that not wrong understanding about life, 
about moments of seeing and hearing? As long as seeing is not directly experienced, it is not possible to eradicate the idea of `something' in it.'' 
We may have expectations to have less defilements through the understanding 
of the Dhamma, but then we are clinging to an idea of self who wants to become a better person. The Buddha's Path is the development of more understanding of whatever reality appears at the present moment. Understanding 
should be developed with detachment. If we have expectations, we forget that 
whatever arises, be it wholesomeness, kusala, or unwholesomeness, akusala, has 
specific conditions. The wholesome and unwholesome tendencies that have 
been accumulated in each citta can, at a given moment, condition the arising of 
kusala citta or akusala citta. Most of the time akusala citta arises. 

If we believe that a self can avoid akusala, we forget that akusala is conditioned 
and that it is impossible to push away what has arisen already. We can learn that 
akusala citta is just a conditioned reality, not self. This is the way to follow the 
right Path. The Buddha taught that all realities that arise are just conditioned 
dhammas, non-self. 

Our friends from Taiwan believed at first that having long retreats in a meditation centre would help them to have less akusala. But after listening to the discussions we had in Vietnam, they realized that there is no self who can control 
cittas and that right understanding of realities can eventually lead to the elimination of akusala after countless lives. Maggie came to understand that attachment 
and anger are normal, that they are conditioned. If they would not arise, how 
could one understand them? What arises at the present moment can be understood by paññā. When there are conditions for attachment, it arises and then its 
characteristic can be understood. 

One of the listeners in Hanoi was wondering how the understanding of realities 
like seeing could help him in his social life. He said that when he was planning 
his diverse activities in his social life, he could not avoid taking account of a 
self. Our activities in social life have to be planned, and we have to think of other people. But it is beyond control whether or not our plans come true. We never know the next moment and whatever we experience is dependent on several 
conditions. We may plan to meet someone else at a certain time and place, but 
due to an accident this may not be according to our plans. We talked about success for someone who is in business. Acharn remarked that he also has to die. 
The duration of one's lifespan is dependent on kamma, it is beyond control. It is 
important to know the true nature of realities like seeing, hearing or thinking 
which occur now. Otherwise we shall never understand their conditioned nature 
and their impermanence. 

We should not change our lifestyle or behaviour in order to develop right understanding of realities. Understanding is to be developed naturally in daily life.

But understanding can be a condition to become more patient and metta, unselfish love, can arise more often instead of attachment to people. 

The Buddha spoke time and again about all realities occurring in daily life. We 
read, for example, in the Kindred Sayings (IV) § 32, Helpful (2): 

\begin{quote}
``I will show you a way, brethren, that is helpful for the uprooting of all conceits. Do you listen to it. And what, brethren, is that 
way? Now what think you, brethren, is the eye permanent or impermanent?''

``Impermanent, lord.''

``What is impermanent, is that weal or woe?''

``Woe, lord.'' 

``Now what is impermanent, woeful, by nature changeable, is it 
fitting to regard that as `This is mine. This am I. This is my 
self?''

``Surely not, lord.'' 
\end{quote}

The same is said about all objects experienced through the six doors, about contact through these doorways and all the sense-cognitions. 

We should not just believe these words but carefully consider what appears 
now. Like seeing, there is seeing now. This is the only way to find out about the 
Truth of the Buddha's teaching. 

\chapter[Kamma and its Result]{}
\section*{Kamma and its Result}

Our present life is part of an endless series of lives in the cycle of birth and 
death. When this life has come to an end, it is followed by a next life and after 
that there are countless other lives. Even so, before this life there were countless 
past lives. Acharn reminded us many times that this life was once the future life 
for the past life. This life will be the past life for the coming life, just as today 
will be yesterday for tomorrow. Yesterday, today and tomorrow follow upon 
each other and what is past is forgotten very soon. This shows how short each 
life is. We do not remember our past life, where we lived and whether or not we 
were married. Remembering that the present life, to which we attach so much 
importance, is only a minuscule part of the cycle of birth and death is helpful 
when we lose a dear person through death. 

Patacara had lost all her family including her recently born son. She went to the 
Buddha and his gathering quite mad and without dress. The Buddha taught her 
the impermanence of life and she was able to develop understanding and finally 
reach arahatship. She spoke to other women who had suffered loss. 

We read in the Therigatha (Canto VI, Fifty, Patacara' s Five Hundred) that 
Patacara said: 

\begin{verse}

``The way by which men come we cannot know; \\
Nor can we see the path by which they go. \\
Why mourn then for him who came to thee, \\
Lamenting through thy tears: `My son! my son!'\\
Seeing thou knowest not the way he came, \\
Nor yet the manner of his leaving thee? \\
Weep not, for such is here the life of man. \\
Unasked he came, unbidden went he hence. \\
For ! ask thyself, again whence came thy son \\
To bide on earth this little breathing space? \\
By one way come and by another gone,\\ 
As man to die, and pass to other births\\\ 
So hither and so hence - why would ye weep?''

\end{verse}

A question was raised during the discussions about the last moment of this life 
and the first moment of the next life. There is no person who travels from this 
life to the next life. When the last citta of this life, the dying-consciousness 
(cuti-citta), has fallen away, it is immediately succeeded by the next citta which 
is the rebirth-consciousness (patisandhi-citta) of the following life. There is no 
gap between these cittas. It is just like now when each citta is succeeded by the 
next citta. Actually, also now there is momentary birth and death of each citta 
that arises and falls away. Life lasts as long as one citta and this is extremely 
short. 

The Buddha taught about life at this moment, what life is now. Acharn said: 
``Life does not belong to anyone, it is only one moment of experiencing an object. It is extremely short. There should be more understanding that there is no 
one, only conditioned realities.'' 

We read in the Kindred Sayings (I, Mara Suttas, § 9, Life Span 3 ) that the Buddha said: 
\begin{quote}

``Bhikkhus, this life span of human beings is short. One has to go 
on to the future life. One should do what is wholesome and lead 
the holy life; for one who has taken birth there is no avoiding 
death. One who lives long, bhikkhus, lives a hundred years or a 
little longer.'' 

Then Mara, the Evil One, approached the Blessed One and addressed him in verse: 
\end{quote}

\begin{verse}

``Long is the life span of human beings, \\
The good man should not disdain it. \\
One should live like a milk-sucking baby: \\
Death has not made its arrival.'' \\

[The Blessed One:] \\

``Short is the life span of human beings,\footnote{I am using the translation of Yen. Bodhi.} \\
The good man should disdain it. \\
One should live like one with head aflame: \\
There is no avoiding Death's arrival.'' \\
\end{verse}
\begin{quote}
Then Mara the Evil One\ldots disappeared right there. 
\end{quote}

Acharn reminded us: ``What is there from moment to moment? What has arisen 
just a moment ago? It has all gone immediately. Life is nothing. From nothing 
there is something and then again nothing, completely gone. After seeing, there 
is the idea of I from birth to death. Each life is conditioned to be born and die. 
Today will be yesterday of tomorrow.'' 

Before seeing arose, there was nothing, no seeing, and then, when there were 
the right conditions, it could arise just for a moment and then it was gone, it became nothing. 

We attach so much importance to joyful events and we dislike sorrow, but they 
all arise because of conditions and they do not last. We would like to experience 
only pleasant objects but this depends entirely on conditions. It is unavoidable 
that unpleasant objects are experienced: we all are subject to disease and to loss 
of what is dear to us. 

It is kamma that produces birth in different planes of existence. Nobody can 
choose his birth. The term kamma is generally used for good and bad deeds, but 
kamma is actually cetana cetasika, volition or intention. Cetana arises with each 
citta and hence it can be kusala, akusala, vipaka or kiriya. Cetana directs the associated dhammas and coordinates their tasks (Atthasalini, Book I, Part IV, Ch 
I, 111). There are two kinds of kamma-condition: conascent kamma-condition 
and asynchronous kamma-condition. Cetana which arises with each citta directs 
the associated dhammas to accomplish their functions; it conditions these 
dhammas by way of conascent kamma-condition, sahajata kamma-paccaya. Cetana which accompanies kusala citta and akusala citta directs the tasks of the associated dhammas and it has the function of activity in good and bad deeds. In 
this last function it produces the results of good and bad deeds. 

Kusala kamma and akusala kamma are mental, and, therefore, they are accumulated in the citta from moment to moment. When cetana motivates a good deed 
or a bad deed, the citta falls away, but cetana or kamma is accumulated. It is 
unknown which of the accumulated kammas will produce rebirthconsciousness. Rebirth as a human is a happy rebirth, the result of kusala kamma, because understanding can be developed during that life if there are 
opportunities for listening to the Dhamma. Rebirth in higher planes, heavenly planes, 
is a happy rebirth. Rebirth as an animal or in a Hell plane is an unhappy rebirth 
or the result of akusala kamma. Some kammas produce their results in the same 
life in which they were committed, some in the next life, some in later lives. 

Kusala kamma and akusala kamma through body, speech and mind can be of 
different degrees. Kamma is not always a `completed action' (kamma patha). 
There are certain constituent factors which make kamma a completed action. 
For example, in the case of killing there have to be: a living being, consciousness of there being a living being, intention of killing, effort and consequent 
death (AtthasalinI, I, Book I, Part III, Ch V, 97). If one of these factors is lacking, kamma is not a completed action. Akusala kamma or kusala kamma which 
is a completed action is capable of producing rebirth that may be unhappy or 
happy. 

There were several questions during the discussions on mano-kamma, kamma 
or cetana, performing its function through the mind (mano), and people wondered when mano-kamma would be a completed action. Sarah explained what 
mano-kamma is and whether it can be kamma patha. 

Cittas that experience objects through the sense-doors and the mind-door arise 
in different processes of cittas. Seeing, for example, sees visible object through 
the eye-door. But seeing is not the only citta that experiences visible object, it 
arises in a process of cittas. There are several other cittas that experience the 
same visible object: they do not see, but while they experience visible object, 
they perform their own functions. Soon after seeing has fallen away it is succeeded by several cittas, usually seven, that are either kusala cittas or akusala 
cittas, which have the function of `javana', `running through' the object in a 
wholesome or unwholesome way. The intention or cetana that accompanies 
those cittas is mano-kamma, but it is not a completed action. As to the kusala 
cittas or akusala cittas arising in a mind-door process, it depends on the intensity of the cetana accompanying them whether they are kamma through the doors 
of body, speech or mind that is accomplished kamma, kamma patha. When attachment to a pleasant sound or flavour arises now, it is first experienced 
through the relevant sense-door and then through the mind-door. It is manokamma, it does not harm anyone else. 

Kamma can also be of the degree that it harms another person. As Sarah explained, wrong views are only akusala kamma patha when they lead to deeds 
and speech, not just the views by themselves. There is mano-kamma whenever 
there are kusala or akusala cittas which are not said to be bodily or vocal kamma. However, unless all the conditions are completed, it is not kamma patha capable of producing rebirth. Akusala cetana in the mind door processes may be 
of a strength to condition bad deeds when they are planned or thought about, not 
just impulsively done. So if the wrong view conditions such deeds or speech, it 
is mano-kamma patha. 

It helps to have less misunderstanding about what kamma is and to appreciate 
that mano-kamma is very common, to know that even now after seeing or hearing, there are kusala cittas or akusala cittas immediately. None of them is self 
and most of the time it is not of a strength to be kamma patha which can condition rebirth. 

Acharn reminded us several times that the Buddha's teaching is not theory, not 
book knowledge. Understanding has to be developed of the reality appearing at 
this moment, so that it will be understood as not self. She said: 

``Is there kamma now? Do we have to find out through which doorway or what 
kind of kamma is there? At this moment it is `I' who is thinking about kamma 
so how can there be the understanding now of kamma itself as not self? 

It is not like `what kind of kammas are there in the book', but how to understand it now as not self and by then one can see, without saying it out, what kind 
of kamma it is. At this moment of speaking, it has to be known that without citta there can never be speaking. What conditions the words which are spoken? It 
depends on citta. When it is akusala citta it conditions harsh words, bad words, 
hurting the other. At that moment it is not necessary to classify whether it is the 
kamma through speech which is vaci-kamma or mano-kamma\ldots The most precious moment is to understand the moment which is conditioned as not self.'' 

Sarah explained: ``Akusala kamma patha, no matter through which doorway it is 
committed, is the one that hurts or harms someone else, but if we try to work 
out exactly whether this or that is kamma patha, there will not be right understanding of it. Someone gave the example in the discussion about covetousness 
when you don't take what belongs to someone else but you really wish to have 
it - and then people want to know whether it is kamma patha. But why is one so 
concerned about it whether it is kamma patha? Usually it's because one is concerned about oneself, thinking, `Will I get bad results in another life' or something like this. We cannot know all the details of the Buddha's wisdom, but 
whether kamma is kamma patha depends on the intensity and whether it harms 
the others at that moment.'' 

Kamma is the condition for the experience of pleasant objects and unpleasant 
objects through the senses. We may believe that other people or the outward 
circumstances of our life are the cause of sorrow, but the real cause is kamma 
that produces results in the form of sense impressions. Seeing, hearing, smelling, tasting and body-consciousness may experience a pleasant object or an unpleasant object. Cittas arise and fall away so rapidly that we cannot know 
whether the object that was experienced was pleasant or unpleasant. It is not 
necessary to find out. There can be wise attention or unwise attention to whatever is experienced and this is dependent on conditions, it is beyond anyone's 
power. At this moment, seeing arises and falls away, and it is followed by thinking that arises and falls away. Even so, in the past there was seeing, followed by 
thinking. In the future, there will again be seeing and all other sense-cognitions, 
followed by thinking. 

When we hear an unpleasant sound, we may have aversion and then the citta is 
akusala citta accompanied by dosa, aversion. When we hear a pleasant sound, 
we may have attachment and then the citta is akusala citta accompanied by lobha, attachment. All akusala cittas are accompanied by ignorance. When right 
understanding is being developed of the realities that appear, there are conditions for wise attention. We may realize that no matter what object is experienced through the senses, pleasant or unpleasant, it is only a conditioned 
dhamma. 

\chapter[Attachment]{}
\section*{Attachment}

Whatever arises because of conditions has to fall away immediately. It is not 
worth clinging to what is impermanent by nature, what is dukkha (unsatisfactory). As paññā develops, it is understood more clearly that life is only one moment of experiencing an object. This is a condition for courage to develop understanding of whatever appears now, be it pleasant or unpleasant, wholesome 
or unwholesome. It is the only way to understand that realities are beyond control, anattā. 
Before we heard the Buddha's teachings, we did not have a precise understanding of what attachment is. We knew in general that one is attached to children, 
members of one's family or friends. Through the Dhamma we learn that there is 
attachment time and again on account of what is experienced through the eyes, 
the ears, the nose, the tongue, the body sense and the mind-door. Very often attachment and ignorance arise after seeing has fallen away, but we do not know 
it. Ignorance covers up the truth of realities. Attachment has been accumulated 
from life to life. It is accumulated in each citta that arises and falls away and 
that is why it can arise so easily. It always finds an object. 

We discussed attachment to persons, which can cause a great deal of disturbance in life. Acharn explained: ``But actually it is not the force of that person to 
condition attachment, but the attachment here has been conditioned for a long, 
long time and it is so great, so that when there is the right moment and the right 
object, it is there and paññā (understanding) can understand. In the beginning, 
paññā cannot see the danger of attachment but it begins to see that it is not self. 
That is the main point. No matter how great attachment is, it falls away. The self 
just regrets having it but paññā does not let go of it.'' 

When paññā has not been developed to a higher degree, it cannot let go of it. 
Only those who have reached the third stage of enlightenment, the stage of the 
non-returner, anagami, have eradicated attachment to sense objects. Only the 
arahat has eradicated all kinds of attachment. 

Sarah said: ``One always clings to something or someone because of not understanding reality. We think this is `my problem - I'm so attached to people', but 
actually it is not `my problem'. They are just moments of thinking with attachment that fall away instantly and they don't last at all. Just moments of attachment and thinking long stories because of sañña (memory) and clinging to the 
stories and ideas - taking them for someone or something. But even such moments of clinging do not last.'' 

Nina: ``Thinking again and again all the time - it is so disturbing.'' 

Acharn: ``Panna begins to see that it is wasting of time, because there is nothing, 
only the object of attachment. It can arise any time.'' 

Nina: ``Panna is too weak to see that, too weak.'' 

Acharn: ``We do not mind and just develop understanding. Paññā will work its 
way.'' 

Sarah: ``If we mind that there is clinging or are disturbed by the clinging, then it 
is just more disturbance about the disturbance. More self - `I don't like this, I 
don't want this kind of thinking'. Thinking and attachment are conditioned and 
fall away anyway, so there is no point in minding and thinking that it shouldn't 
be like that.'' 

Acharn: ``The object of attachment does not last at all. Only thinking and at that 
moment there can be attachment to other things instead of that because it is accumulated in the citta. Nothing can take it away - only paññā can purify it little 
by little. Only that, so be happy. Whatever happens is just a moment and it does 
not last\ldots only a reality which has fallen away completely and is no more.'' 

Nina: ``I don't know yet that it is just a moment.'' 

Acharn: ``By understanding, little by little.'' 

Sarah: ``Never mind! Even if there is no understanding now, moments of ignorance - they are just passing realities. So, it doesn't matter if there is a lot or a 
little understanding, attachment or disturbance or moments of kusala - they are 
all gone instantly. When we think we are so attached to people, actually it is visible object and sound and so on that there is so much attachment to and that is 
why there is thinking about them again and again. It is not `my disturbance' but 
just common, ordinary realities. It is like that for everyone.'' 

Acharn: ``Paññā accumulates so little at a time but it is so great when it has accumulated more and more. It will come.'' 

Sarah: ``No need to think about it - how much or little. Let it just perform its 
function and do its job, otherwise if one is thinking about a lot or a little understanding, one is disturbed again, attached again. 

One thinks of attachment to others but one is most attached to oneself. One 
thinks `Oh, no understanding, very little understanding', and this is all attachment to oneself. As soon as it matters, there is the idea of self. There is nothing 
to be concerned about or upset about, attachment is very common. The problem 
is that one thinks `my attachment is so special'. '' 

People go to Acharn with different problems concerning their relationship with 
others, in the family, with their partner, in their work. Acharn always asks: ``Is 
there seeing now?'' She brings us back to the present moment, because without understanding of the present reality, problems cannot be solved. There is no I 
who sees, seeing sees. Seeing arises because of conditions and it cannot be manipulated. This shows us the nature of anattā. There is no one there. Thinking 
thinks of a problem and the problem becomes very great and important. But it is 
only in our thinking. Thinking is also a conditioned reality, and it is not `I 
think'. The thinking thinks. When things in life are not the way we would like 
them to be, we are inclined to wish to control our life. That is attachment and 
with attachment problems will not be solved. We think of long stories instead of 
realizing that whatever happens is beyond control. The realities of our life are 
only citta, cetasika and rūpa, no person who can be master of situations and 
events. They just arise for a moment and then fall away. 

We may not realize how extremely brief one moment of citta is. When we see, 
it seems that we immediately see people, but then seeing has fallen away and 
thinking has arisen already. 

It is helpful to consider again and again the following verse of the ``MahaNiddesa'' quoted in the ``Visuddhimagga'' (VIII, 39): 

\begin{verse}

Life, person, pleasure, pain - just these alone\\ 
Join in one conscious moment that flicks by.\\ 
Ceased aggregates of those dead or alive\\
Are all alike, gone never to return. \\
No [world is] born if [consciousness is] not \\
Produced; when that is present, then it lives;\\ 
When consciousness dissolves, the world is dead:\\ 
The highest sense this concept will allow' (Nd.1,42). 
\end{verse}

Life, person, pleasure, pain: What is life? It is all that appears through the five 
senses and the mind-door. When seeing arises, life is seeing; when hearing arises, life is hearing; when thinking arises, life is thinking. When we think of a 
person, he seems to exist, but what we take for a person are only impermanent 
nāma and rūpa, fleeting phenomena. Pleasure and pain are impermanent: in our 
life happy moments and sad moments alternate, they appear one at a time. We 
attach great importance to our experiences in life, to our life in this world, but 
actually life is extremely short, lasting only as long as one moment of citta. 
As we read:
\begin{verse}
No [world is] born if [consciousness is] not \\
Produced; when that is present, then it lives;\\ 
When consciousness dissolves, the world is dead:\\  
\end{verse}

When we are thinking about the world and all people in it, we only know the 
world by way of conventional ideas. It seems that there is the world full of beings and things, but in reality there is citta experiencing different dhammas arising and falling away very rapidly. Only one object at a time can be cognized as 
it appears through one doorway. Without the doorways of the senses and the 
mind, the world could not appear. So long as we take what appears as a `whole', 
a being or person, we do not know the world. 

If there were no citta, nothing could appear, but since citta arises at each moment, realities appear. We are reminded of the brevity of all experiences, including thinking with worry about our problems. The real cause of problems is 
not in the outside world nor in other people, it is in the citta. People wonder 
what they should do in difficult situations, in their dealings with other people. 
They ask: ``What next?'' But who knows the next moment? This depends entirely on conditions which are beyond control. Because of our clinging to the idea 
of self, we create our own problems and we believe that we can act in this or 
that way to solve our problems. Development of right understanding of one reality at a time as it appears at this moment is the condition for one to be less 
taken in by concepts and ideas of the conventional world with all the problems 
and worries. One begins to see the world in the ultimate sense: citta, cetasika 
and rūpa. That is the world that is real, that is the world that should be understood more and more. 

When we were at the airport, about to leave Vietnam and go back to Thailand, 
Acharn spoke about attachment. She said that we should not be afraid of it. We 
should not mind having it, or try to force ourselves not having it. It arises because there are conditions for it and it falls away immediately. It is only a 
dhamma and when paññā is more developed, it can realize its characteristic. 
Now we are mostly thinking about ideas which are not real and there is likely to 
be the idea of `my attachment'. We are afraid of having attachment, but the reason for being afraid of it is that we take attachment for self. Even not wanting to 
have attachment is attachment already. 

We had a discussion with the listeners about the way how to solve problems. 
Someone suggested that this could be with therapy. Acharn asked: ``Is there a 
problem now? What is the cause of problems?'' The Dhamma is not like a therapy. The cause of problems is that we are thinking of self, that we relate problems to ourselves. All that is arising now is only a conditioned dhamma, not 
self, and nobody can make it arise or do anything about it. When we listen to the 
Dhamma, there can be a little more understanding. The development of understanding is with ups and downs but we can see that even a little more understanding is beneficial. We cannot expect an immediate result of listening and 
considering the Dhamma. When there is any expectation, we cling to an idea of 
self. We can accept that the development of paññā is just step by step. 

Sarah said that one learns to live easily and naturally while developing understanding instead of trying to change our life with the wrong idea of self. She 
said: ``It is like letting go of a big burden. The happiness of understanding is different from the happiness with clinging.'' 

Acharn quotes from the teachings time and again that the development of understanding should be with courage and gladness. She told us to be happy about 
the reality that appears. She said: ``Be happy. Whatever occurs is just a moment 
and it does not last. It is only a reality that has fallen away completely and is no 
more.'' We should be grateful that the Buddha taught that whatever appears is 
only a conditioned dhamma, impermanent and not self. 

Is there seeing now? It does not see a person, it sees only visible object for an 
extremely brief moment. While realities are considered in the right way, there is 
no worry, no disturbance at that moment. 

The Buddha spoke time and again about seeing, hearing, all the sensecognitions and all objects experienced by them. 

We read in the ``Kindred Sayings'' (IV, 32, Second Fifty, § 60, Comprehension): 

\begin{quote}
``I will show you, brethren, a teaching for the comprehension of 
all attachment. Listen to it. What is that teaching? 

Dependent on the eye and the object arises eye-consciousness. 

The union of these three is contact. Dependent on contact is feeling. So seeing, the well-taught Arian disciple is repelled by the 
eye, by objects, by eye-consciousness by eye-contact and by 
feeling. Being repelled by them he lusts not for them. Not lusting he is set free. By freedom he realizes `Attachment has been 
comprehended by me' ''. 
\end{quote}

The same is said about the ear and sounds, the nose and scents, tongue and savours, body and tangibles, mind and mind-states. 

The Buddha explained that the objects that are experienced, the types of consciousness, cittas, the feelings arising on account of them, are all conditioned. 

He spoke separately about all the six doors. The aim is to understand anattā, to 
become detached from the ideas of person, self, situations, things. Understanding leads to detachment. 

We see here that there is Abhidhamma in the suttas. We read about what is real 
in the ultimate sense, different from stories about persons and things we think of 
all day long. 

The conditions for the experience of visible object, for seeing are entirely different from the conditions for the experience of sound. Eyesense and visible object are conditions for seeing. Earsense and sound are conditions for hearing. 

The more we understand about conditions, the less we cling to a self who could 
cause the arising of seeing or hearing. 

In the Abhidhamma texts details are given about the different processes of cittas 
experiencing one object. All with the aim to cling less to the self, to understand 
anattā. 

The cittas of the eye-door process, of the ear-door process, of all processes succeed one another very rapidly so that it seems that we can experience more than 
one object at a time. 

We should distinguish between the world of thinking of concepts, of persons, 
things, situations, from the world of realities that can be experienced one at a 
time. We mostly live in the world of concepts, imaginations, but we can begin 
to know the difference. This cannot be accomplished immediately, since we accumulated ignorance and attachment from life to life. 



\chapter[Conditioned Dhammas]{}
\section*{Conditioned Dhammas} 

One day, when we were having lunch, Acharn explained to Vincent, our friend 
from Taiwan, about realities. Vincent was working hard all the time, translating 
into Mandarin all the Dhamma conversations held during the sessions for his 
brother and for Maggie. 

We usually think that we see people and do not realize that seeing just sees visible object, not a person or thing. While we were eating a salad with tomatoes, 
Acharn said: 

``How could there be an idea of something without visible object? Visible object 
is a reality, it is not a thing like a tomato, it is just that which can impinge on the 
eyebase. It arises and falls away. Attachment and ignorance arise and attachment wishes to understand, but it covers up the truth. Why did he teach visible 
object? If he had not, there would be something in it all the time. 

It would not appear as it is and it would be impossible to let go of the idea of 
someone. The eyebase is a condition for the arising of seeing. Without it, it 
would be impossible to see. Each citta experiences an object, whenever it arises 
it experiences an object.'' 

Then the term saṅkhāra dhamma, conditioned dhamma, was discussed. We 
should study each word of the teachings, otherwise we talk about what we do 
not know. Whatever arises because of conditions and appears is saṅkhāra 
dhamma. Understanding this is the beginning of paññā. Paññā eradicates ignorance. 

Vincent asked whether saṅkhāra is a concept. 

Acharn said: ``We do not talk about concepts. What appears through the 
eyes?'' 

Vincent answered: ``White colour, but this is already thinking.'' 

Acharn: ``We talk about citta, the faculty that experiences. It does not have the 
function of like or dislike. The table has no quality to experience. You think of 
saṅkhāra but there is no understanding of its meaning. What appears now?'' 

Vincent: ``Sound.'' 

Acharn: ``If sound had not arisen it could not appear. Whatever appears has arisen because of conditions, not by anyone's will. The eyebase cannot condition 
hearing. It is a condition for seeing. Does it arise?'' 

Vincent: ``Yes.'' 

Acharn: ``It is saṅkhāra dhamma, a conditioned reality. At this moment of seeing, we do not have to think of a flower or another thing. There is just that 
which is seen. This is the way to let go of the idea of someone or something in 
it. There is nothing mixed in it at all. Smell is smell, sound is sound. Is sound 
sankhara dhamma?'' 

Vincent: ``Yes.'' 

Acharn: ``Is hearing sankhara dhamma? Is thinking saṅkhāra dhamma?'' 
Vincent: ``Is concept not saṅkhāra dhamma?'' 

Acharn: ``We do not talk about concepts, just about absolute realities. That is 
why we have the words paramattha dhamma (absolute or ultimate reality) and 
abhidhamma (subtle dhamma or dhamma in detail). We begin with the word 
dhamma: whatever is real. We know that there are so many different kinds of 
realities. There are realities that can experience something and realities that 
cannot experience anything.'' 

Acharn then explained that there are nāma, realities that experience something, 
and rūpa, realities that cannot experience anything. There is nobody, no one, no 
permanent self. The word saṅkhāra is used to indicate that there is no one, only 
dhammas. She explained that by talking and discussing one will have more understanding. She said that discussing is a blessing, since it brings more understanding. Otherwise one reads the texts but one does not know how much understanding there is. She then spoke about meditation centres. 

Acharn: ``It is useless to go somewhere and meditate. Then one does not have 
understanding of the reality at this moment. What is saṅkhāra dhamma? Whatever arises by conditions. Without conditions nothing can arise. That is why it 
cannot belong to anyone. It is not anyone. All dhammas are anattā. When you 
go somewhere, is that atta or anattā?'' 

Atta means self, one's actions may be motivated by the idea of self, or by anattā, the understanding of non-self. 

Vincent answered: ``Atta.'' 

Acharn: ``No understanding of anattāness when one thinks: `I would like to do 
this or that'. If there is no atta, why do you go there? If there is right understanding, it is now, right understanding of that which appears. What is 
dhamma?'' 

Vincent: ``What is real.'' 

Acharn: ``Can atta make it arise? Atta cannot bring about right understanding. 
That is the reason one goes to a quiet place, but it cannot be the right Path; it is 
motivated by ignorance and attachment. Here we are talking about Dhamma. 
What brings you here is listening to the Dhamma.'' 

Vincent: ``So, that kind of desire is not-attachment. There is the opportunity to 
listen.'' 

Acharn: ``One knows what can bring less attachment: right understanding.'' 

While we were having our Dhamma conversation at the lunch table, preparations were going on for Khun Deng's birthday. Acharn said to her: ``May you be 
happy'' and then went on immediately with the Dhamma explanation. Khun 
Deng found this the best way to celebrate her birthday. There were songs for her 
and a birthday cake was being shared out. She went to Acharn and said that she 
had appreciated so much the simile of the closed fist Acharn had given her. 
When a fist is closed, we do not know what is in it, but when one opens it, one 
sees that there is nothing. Even so, what we take for our life are realities that 
arise and fall away. There is nothing or nobody there. 

We read in the commentary to the ``Satipatthana Sutta'': 

The character of contemplating the collection of primary and derived materiality is comparable to the separation of the leaf covering of a plantain-trunk, or is like the opening of an empty fist. 

As to primary and derived rnpas, these are the four Great Elements of solidity, 
cohesion, temperature and motion, and the derived rūpas of taste, odour, smell, 
nutrition and other rnpas. They arise and fall away all the time. The body does 
not exist, what we take for our body are only fleeting rnpas. 

Acharn: ``It is not easy to understand that there is no one at all. What is there 
when there is no one? What is the reality at this very moment?'' 

Vincent: ``Nāma and rūpa.'' 

Acharn: ``We may name it, but it takes a long time, from life to life, to understand what is real now. Each word of the Buddha should be considered, because 
it represents a reality, just one characteristic at a time. Only when there is a 
great deal of listening to the Dhamma, considering it and wise reflection, understanding can begin to develop. Now it is intellectual understanding but it can 
become firmer and firmer.'' 

Intellectual understanding of the reality that appears at this moment, in Pali 
pariyatti, is the foundation for the development of direct understanding of realities, in Pali patipatti. This again can eventually lead to the direct realization of 
whatever reality appears, pativedha. 

People had questions about sati, mindfulness of realities, that is developed in 
satipaṭṭhāna. Paññā that is accompanied by sati of this level is actually patipatti, 
which is often translated as practice. This translation is misleading since it suggests a self who is acting in a specific way. 

There are many misunderstandings about mindfulness or awareness. Some people think that this means knowing what one is doing, such as walking or applying oneself to tasks in the house or at work. Sati is a sobhana cetasika, a beautiful mental factor, that accompanies every kusala citta. It is non-forgetful of 
kusala. It can be of many levels and degrees. Sati of the level of dana arises 
when one is generously giving gifts. Sati of the level of slla arises when one abstains from harsh speech or when one is helping others. Sati of the level of samatha, calm, is non-forgetful of the object of calm. Sati of the level of 
satipaṭṭhāna is mindful of the reality appearing at the present moment. It accompanies paññā so that it can know this reality as only a conditioned dhamma 
that is not self or mine. When there is an opportunity for kusala, one may be lethargic and lazy, thinking of one's own comfort and pleasure. One is forgetful 
and lets the opportunity go wasted. However, when sati arises, it is nonforgetful of kusala and does not let the opportunity go wasted. 

Some people wish for the arising of sati and they do not see that sati also is a 
conditioned reality that arises because of its own conditions. Acharn spoke 
many times about the conditioned nature of seeing, because there is seeing time 
and again, also right now. What has arisen is real and what has not arisen is not 
real. She would speak about seeing time and again to bring people back to the 
present moment instead of paying attention to abstractions, ideas which are not 
realities. If there is correct understanding of seeing at this moment, one can 
leam the meaning of conditioned reality that is beyond control. Then it will be 
clearer that also sati is beyond control. 

It is difficult to be aware of one reality at a time that appears without thinking of 
the word. One has to get used to their characteristics, each one is different. 

When there is no understanding of what appears, there is attachment, because it 
seems to be permanent. What is the object of attachment that appears and disappears? Actually, there is clinging to that which is no more. When there is understanding of what appears, like seeing right now, it is the beginning of understanding that it is just a conditioned reality. 

The Buddha taught details so that people would have more understanding of 
conditioned realities. Acharn reminded us: ``There is no method at all. It is dependent on right understanding when it has been sufficiently developed to condition direct awareness of seeing right now. Before hearing the teachings, hardness was experienced at the moment of touching with the idea of `something' 
all the time, with the idea of I or `mine'. Softness appears all the time but there 
is no understanding. Nobody can change it.'' 

Acharn explained about conditioned realities: ``How can there be less attachment to seeing right now? The Buddha taught the conditions for the arising of 
seeing. Otherwise one would think that just opening one's eyes is a condition 
for the arising of seeing. That is wrong understanding. Seeing experiences that 
which is now appearing. Without the accompanying cetasikas, seeing could not 
arise.'' 

The accompanying cetasikas are conditions for seeing. Contact, phassa, is a 
cetasika that contacts visible object so that seeing can see it. One-pointedness or 
concentration, ekaggata cetasika, is the condition that seeing only experiences 
visible object and that there is no thinking of other things at that moment. 
Memory or saññā marks or remembers the object that is seen. Even so, sati 
could not arise without the accompanying cetasikas. It needs non-attachment, 
alobha, as a condition. It also needs concentration, so that it is mindful of one 
nāma or rūpa. It needs calm that accompanies every kusala citta. 

Some people believe that there should be calm first before right understanding 
of a reality can arise. They take a feeling of relaxation, or not being disturbed by 
noise, for calm. Calm as it is understood in conventional sense is quite different 
from the reality of calm, passaddhi. This is a cetasika that accompanies every 
kusala citta. It can only arise when there are the right conditions for kusala and 
nobody can cause its arising. When one is generous, there is already calm with 
the kusala citta. When one studies the Dhamma with kusala citta, there is already calm accompanying the kusala citta. If one thinks that one should go to a 
quiet place in order to have calm, it is wrong understanding. One clings to an 
idea of self who can induce calm. 

Nāma and rūpa appear one at a time and each one of them has its own characteristic. These characteristics cannot be changed. Seeing, for example, has its own 
characteristic; we can give it another name, but its characteristic cannot be 
changed. Seeing is always seeing for everybody, no matter an animal sees or 
any other living being sees. It has to be known as only a dhamma. Concepts are 
only objects of thinking, they are not realities with their own characteristics, 
and, thus, they are not objects of which right understanding is to be developed. 
Thinking is a reality, there is no self who thinks. Sometimes when there is 
thinking of beings there can be understanding at that moment, one may realize 
that it is just thinking. 

Only one reality at a time can be experienced by citta and, thus, mindfulness 
which accompanies the kusala citta can also experience only one object at a 
time. Since we are so used to paying attention to `wholes', to concepts such as 
people, cars or trees, we find it difficult to consider only one reality at a time. 
When we know the difference between the moments of thinking of concepts and 
the moments that only one reality at a time, such as sound or hardness, appears, 
we will gradually have more understanding of what mindfulness is. It can only 
arise when there is no expectation. 

The following sutta emphazises the importance of listening and discussing the 
Dhamma in order to have more direct understanding of realities. 

We read in the ``Gradual Sayings'' (Book of the Fours, Ch XV, § 7 Seasons): 

\begin{quote}

``Monks, there are these four seasons which, if rightly developed, rightly revolved, gradually bring about the destruction of 
the asavas. What four? 

Hearing Dhamma in due season, discussion of Dhamma in due 
season, calming in due season, insight in due season. These are 
the four. 

Just as monks, on a hilltop when the sky-deva rains thick drops, 
that water, pouring down according to the slope of the ground, 
fills up the clefts, chasms and gullies of the hill-side; when these 
are filled, they fill the pools; when these are filled, they fill the 
\footnote{There are four kinds of asavas: 
the canker of sensuality (kamasava) 
the canker of becoming (bhavasava) 
the canker of wrong view (ditthasava) 
the canker of ignorance (avijjasava) }
lakes; when these are filled, they fill the rivulets; when these are 
being filled, they fill up the great rivers; the great rivers being 
filled fill the sea, the ocean; - just so, monks, these four seasons, 
if rightly developed, rightly revolved, gradually bring about the 
destruction of the asavas.'' 
\end{quote}

\chapter[Understanding this Moment]{} 
\section*{Understanding this Moment}

Acharn reminded us: ``When there is no understanding, attachment arises to 
what appears; it seems permanent. The truth is that the object of attachment appears and disappears very rapidly. There is clinging to what is no more. Seeing 
arises and falls away but ignorance cannot understand that. So it takes what is 
seen or seeing as `something'. Seeing a moment ago is gone completely. Thinking about it is gone immediately.'' 

Acharn explained that anattā, the truth of non-self, can only be understood by 
paññā, not by trying so hard to make paññā arise. 

Vincent remarked: ``But citta is so fast.'' 

Acharn said: ``No one can stop the rapidity of the succession of cittas. There 
cannot be selection to have this or that as object of awareness. Realities roll on 
very fast. Life is the stream, the flux of realities arising and falling away by 
conditions.'' 

People were asking how there can be direct understanding of realities. The answer is: only when paññā has grown to a higher level with direct awareness of 
whatever reality appears. This will take many lives, but intellectual understanding of what appears now, thus, pariyatti, can condition understanding of the level of patipatti, direct understanding of realities. 

Some people believe that they have to concentrate on nāma and rūpa in order to 
develop direct understanding of realities. But, what is concentration? As Acharn 
often said, we have to study each word of the teachings in order to understand 
the true meaning. People like to have concentration but they do not understand 
what it is. Coming back to this moment, is there concentration now? Someone 
thought that concentration helps a great deal to understand the present moment. 
Acharn asked again: 

``Is there concentration right now? At the moment of seeing, is there concentration? All realities are unknown. We are only talking about the story of them. 
Citta experiences only one object at a time, and it is the function of ekaggata 
cetasika, concentration, to cause citta to focus on that one object.'' 

There are many misunderstandings about concentration. It arises with every citta. When it accompanies akusala citta, it is wrong concentration, and when it 
accompanies kusala citta, it is right concentration. It is a conditioned dhamma 
and nobody can cause its arising. 

One of the listeners said that there was quite a revolution in his way of thinking 
when he gave up wrong ideas about the eightfold Path. He sees now that understanding should be developed in a natural way and that he should not try to focus on particular realities. 

If we try to concentrate or if we think, ``I am concentrated'', we cling to an idea 
of self who has concentration and then we follow the wrong Path. When we 
think that we can do something to develop understanding, it is wrong practice. 
One should consider realities more that appear now in daily life so that understanding can develop naturally. 

Someone asked how one can reach the stage of patipatti, the development of direct understanding, and whether meditation is necessary to reach it. She wished 
for enlightenment. 

Sarah answered: ``Bhavana is the development of understanding. Let us speak 
about understanding now. It does not mean practice. No I who can do anything. Hearing about the realities that arise in a day leads to a little more understanding of what pariyatti means. There can be bhavana right now, there is no 
need to wait for another time or place. When one thinks of going to follow a 
method, it is thinking. There can be understanding of thinking.'' 

The person who asked questions about meditation had an idea of wanting to experience emptiness. Sarah explained: 

``It seems that there is just emptiness, nothing there, no citta, no object. That is 
moha, ignorance. It is not possible for the citta that arises not to experience an 
object. The object is a reality or a concept. It is not nothing.'' 

Sarah then explained, if someone has an idea of having no object but being able 
to have concentration lasting a few hours, experiencing emptiness, that it does 
not bring him closer to the Buddha's teachings. It will induce people to take the 
wrong Path, it is not a condition for understanding conditioned dhammas. Following the wrong Path is so dangerous. If one listens, like now, one will realize 
that dhammas are anattā, each one impermanent and unsatisfactory (dukkha). 

Acharn repeated very often that realities such as sound, hearing or tangible object are experienced in darkness. Only at the moment visible object is experienced by seeing, the world is light. But when visible object has fallen away, the 
world is dark. It seems that the world of light lasts but this is an illusion. Visible 
object impinges again and again on the eyesense and seeing and the other eyedoor process cittas follow. But there are numerous other processes of cittas in 
between. This shows us how rapidly cittas are arising and falling away in succession. When Acharn said that hearing experiences sound in darkness, she reminded us that only one object at a time can be experienced. She reminded us of 
the rapidity of the stream of realities. 

The Buddha taught about cittas that experience objects through the doors of the 
senses and the mind, arising in different processes of cittas. Cittas arise and fall 
away in succession extremely rapidly. When we consider his teaching more 
deeply, it will help us to see that nobody can interfere with these processes, that 
they are beyond control. All realities are anattā. 

Each of the sense-cognitions experiences an object through the appropriate 
doorway. There is not only one citta that experiences visible object, or one citta 
that experiences sound, but each of the sense-cognitions arises in a series or 
process of cittas succeeding one another and sharing the same object. They all 
cognize the same object, but they each perform their own function. 

Seeing is preceded by the eye-door adverting-consciousness, which adverts to 
visible object. It does not see but it merely turns towards the visible object that 
has just impinged on the eyesense. This citta is an ahetuka kiriyacitta (inoperative citta without hetus, roots), it is not akusala citta, not kusala citta and not 
vipakacitta. Seeing, which is an ahetuka vipakacitta, is succeeded by two more 
ahetuka vipakacittas which do not see but still cognize visible object that has 
not fallen away yet. They perform a function different from seeing while they 
cognize visible object. Visible object is rūpa and it lasts longer than citta. These 
cittas are receiving-consciousness (sampaticchana-citta), that receives visible 
object and investigating-consciousness (santlrana-citta), that investigates the object. The investigating-consciousness is succeeded by the determiningconsciousness (votthapana-citta), which is an ahetuka kiriyacitta. This citta is 
followed by seven cittas performing the function of javana, which are in the 
case of non-arahats kusala cittas or akusala cittas. There is a fixed order in the 
cittas arising within a process and nobody can change this order. 

The five-sense-door adverting-consciousness (panca-dvaravajjana-citta) turns towards the 
object through one of the five sense-doors. It is named after the relevant sense-door, such as 
eye-door adverting-consciousness or ear-door adverting-consciousness. 

There is no self who can determine whether the determining-conscious\-ness will 
be succeeded by kusala cittas or akusala cittas. Cittas arise and fall away succeeding one another extremely rapidly and nobody can make kusala citta arise 
at will. Kusala or akusala performed in the past is a condition for the arising of 
kusala or akusala at present. 

When the sense-door process of cittas is finished, the sense object experienced 
by those cittas has also fallen away. Very shortly after the sense-door process is 
finished, a mind-door process of cittas begins, which experience the sense object which has just fallen away. Although it has fallen away, it can be object of 
cittas arising in a mind-door process. The first citta of the mind-door process is 
the mind-door adverting-consciousness (mano-dvaravajjana-citta) which adverts 
through the mind-door to the object which has just fallen away. The mind-door 
adverting-consciousness is neither kusala citta nor akusala citta; it is an ahetuka 
kiriyacitta. After the mind-door adverting-conscious\-ness has adverted to the object, it is succeeded by either kusala cittas or akusala cittas (in the case of nonarahats), which experience that same object. 

When visible object is experienced through the mind-door, the cittas only know 
visible object, they do not pay attention to shape and form or think of a person 
or a thing. But time and again other mind-door processes of cittas follow which 
think of people or things and then the object is a concept, not visible object. The 
experience of visible object conditions the thinking of concepts of people and 
things which arises later on. It seems that while we are seeing we can think already about what is seen, but in reality seeing and thinking arise in different 
processes. Since cittas succeed one another so rapidly, it seems that they last. 

How much understanding is there now of visible object as visible object? Right 
understanding has not been sufficiently developed so as to become detached 
from the idea of self. There should be listening to the Dhamma, considering 
what one hears and understanding of what is now appearing. Thinking always 
follows seeing, hearing and all the other sense-cognitions. What is thinking? It 
is different from seeing but only when understanding is more developed their 
different characteristics can be directly known. 

We may only `think' that seeing is not self but the actual moment of that which 
sees is not known yet. When there is more understanding, there will be less attachment to realities as self. We should have confidence in right understanding, 
it can understand what was not known before. As long as seeing is not directly 
known, it is impossible to give up the idea of something in it. There can be 
more confidence that whatever happens, whatever we do only lasts for an extremely short moment. 

People are often wondering what the conditions are for right understanding and 
right mindfulness of the eightfold Path. The condition for right awareness is 
right understanding from hearing, considering. Paññā is not developed sufficiently to know that there is no one in visible object. We mostly live in the 
world of ignorance; there is no understanding of what is true and real. Considering realities is a precious moment. 

On my last day in Thailand I attended the morning session in Thai at the Foundation. The subject discussed was the real purpose of monkhood. The following sutta was duscussed: 

Gradual Sayings, Book of the Tens, Ch IV, § 1, Upali and the Obligation. Upali 
asked the Buddha what the purpose of the Vinaya was. The Buddha explained 
that this was: 

\begin{verse}
``For the excellence of the Order, \\
for the well-being of the Order, \\
for the control of ill-conditioned monks \\
and the comfort of well-behaved monks,\\ 
for the restraint of the cankers in this same visible state,\\ 
for protection against the cankers in a future life,\\ 
to give confidence to those of little faith,\\ 
for the betterment of the faithful,\\ 
to establish true dhamma,\\ 
and to support the discipline.'' \\
\end{verse}

Khun Unnop stressed that the second point was most important: the well-being 
of the Sangha. Often there is wrong understanding about the meaning of monkhood. Young men take ordination for a short while, to please their parents and 
without any understanding. People give money to monks and they accept it, but 
this is wrong. Monks have left the home life and should not accept money or 
enjoy the idea of it. They should see the danger of being in the cycle of birth 
and death and their life should be directed towards freedom from the cycle, to be reached at the attainment of arahatship. The monk has only two tasks: the 
study of the scriptures and the development of insight. Nothing else. The study 
of the scriptures is not in order to gain theoretical knowledge, but to understand 
the reality of the present moment. Acharn adapted most of her radio programs to 
explain the purpose of monkhood. 

I met Khun Samnuang Sucharitakul, who came very cheerfully in a wheelchair. 
She had recently turned a hundred years. She used to get up at night for a few 
hours to transcribe Acharn' s talks in Thai, so that these could be printed as 
books which were beneficial for many. It enabled me to translate several of 
these books into English. 

At the Foundation\footnote{Dhamma Study and Support Foundation. This is the centre where all sessions with Acharn 
Sujin take place each weekend.}, during the English session, we discussed the accumulation 
of akusala. By listening to the Buddha's teachings, there can be more understanding of attachment (lobha), aversion (dosa) and ignorance (moha) which are 
of many degrees. They arise because of conditions, they have been accumulated 
from life to life. Whenever there is attachment, we think of ourselves. Some degrees are very harmful such as wanting to steal something, and some are not 
harmful for others. We can learn to see the danger and disadvantage of all degrees of akusala. 

We are attached to friends and family members but it is good to know that we 
are actually attached to ourselves when we like the company of dear people. It 
was emphasized several times that it is so common and that we should not consider it as our special problem or find it important. 

The discussions held during these weeks were most beneficial to all of us. Sarah 
and Jonothan added many useful points to Acharn's explanations of dhammas 
that arise in daily life. I am very grateful for all those reminders. It was emphasized in many ways that understanding should be developed naturally. One 
should not change one's lifestyle or give up one's job to study Dhamma with 
the purpose to have kusala citta with understanding more often. Then clinging 
to the idea of self will not be eradicated. 

Acharn had said before many times that paññā was not developed sufficiently so 
as to condition direct understanding of realities. We may repeat these words but 
at first they may not be very meaningful. Now, after all our discussions, it became somewhat clearer that only a higher level of paññā can condition direct 
awareness and understanding of the present reality. We cannot act in any way to 
cause the arising of a higher level of paññā, except persevering in listening and 
considering what we hear. Seeing and thinking that appear now can be understood as just conditioned dhammas, but it will take a long time, even many lives, before this is clearly understood. Gradually there will be more confidence 
in the growth of paññā. How could one interfere with realities that arise and fall 
away extremely rapidly? 

Acharn asked: ``Life is so very short. What is the best moment in life?'' 

The answer is: listening to Dhamma so that we come to know what the truth is 
in life - whatever reality appears is only a conditioned dhamma, not self. 
