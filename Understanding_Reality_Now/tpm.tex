\part{The Present Moment}




\chapter{Preface} 


A ten days journey to Sri Lanka with English discussions was organized in October 
2015 by Sarah and Khun Toey. Preceding this journey we stayed two days in Nakorn 
Nayok, in Thailand, where we had Dhamma discussions with Tadao from Japan, Ann 
from Canada, Sukinder and Khun Metta who live in Bangkok, and other friends. I 
found that it is most valuable to be in the company of good Dhamma friends. I am so 
grateful for all the good reminders of the truth about realities appearing in daily life. 
Early morning in the hotel in Sri Lanka Sarah reminded me that all the problems with 
difficult steps I had to take (very high and very deep), problems with the laundry, the 
bathroom, stiffness, are nothing compared to the Dhamma we received. I was so glad 
to hear again and again: ``understanding is not yet firm enough, not enough yet.'' I 
found it most helpful that Sarah and Jon elaborated on the subjects Achan spoke 
about and gave us reminders of reality time and again. 

I admired Jon's translation from Thai into English of all the details of the Vinaya 
Khun Sangob gave us, even including technical subjects like the Sima (measurement 
pertaining to boundaries). 

The sessions in Sri Lanka were mostly two hours in the morning and two hours or 
more in the afternoon. Achan never thinks of her own comfort or rest. That is why 
she decided, when we were in our hotel in Negombo (near Colombo), to travel during 
our stay there, to Colombo, which meant four hours in the traffic each day. In Colombo we went to the Buddhist Cultural Center to have Dhamma discussions with 
monks, sisters and other people. Even after returning to the hotel she would still be 
with us for Dhamma discussions. I am very grateful for all the reminders she gave 
about ``dhamma now''. 





\chapter{Concepts and Realities} 




The aim of the Buddha’s teachings is to develop right understanding of the dhamma 
appearing at this moment and this will lead to detachment from the idea of self and to 
the eradication of all that is unwholesome. 

We read in the ``Khuddhaka Nikaya'' in the Commentary to the ``Basket of Conduct'', 
the ``Conduct of Yudanjaya'', about the beginning of the development of paññā during the life the Bodhisatta was young Yudanjaya: 

\begin{quotation}
``In his life when the Bodhisatta was Yudanjaya, he was the eldest son 
of the King and had the rank of the viceroy. He fulfilled every day 
maha-dana, the giving of an abundance of gifts. One day when he visited the royal park he saw the dewdrops hanging like a string of pearls 
on the tree-tops, the grass-tips, the end of the branches and on the spiders’ webs. 

The prince enjoyed himself in the royal park and when the sun rose 
higher all the dewdrops that were hanging there disintegrated and disappeared. He reflected thus: `These dewdrops came into being and 
then disappeared. Even so are conditioned realities, the lives of all beings; they are like the dewdrops hanging on the grass-tips.’ He felt a 
sense of urgency and became disenchanted with worldly life, so that 
he took leave of his parents and became a recluse.''
\end{quotation}

The Bodhisatta realized the impermanence of realities and made this predominant in 
accumulating a sense of urgency and disenchantment; it arose once and then became 
a condition leading to its arising very often. 

What falls away immediately is dukkha, not worth clinging to. Seeing appearing at 
this moment is dukkha, it arises and falls away, never to return. We cling to whatever 
can be experienced through the senses and the mind, but actually we cling to what is 
gone already. What is gone never arises again. 

In Sri Lanka we stayed first in a hotel outside Colombo, in Negombo. The first afternoon Venerable U Pandita and Kevin, a Vietnamese student who is living now in Colombo, visited us. We discussed dukkha, the first noble Truth and the difference between what is real and what is an idea or concept. Dukkha pertains to the arising and 
falling away of realities and if we confuse reality with concept and do not know the 
difference we cannot understand the truth of dukkha. The truth of dukkha refers to realities appearing at this moment. 

We may have read texts of the Buddhist teachings about ultimate realities, dhammas, 
appearing one at a time through the six doors of the senses and the mind-door and we 
have learnt that these are different from what is real only in conventional sense such 
as a garden or a person which are a collection of things or a ``whole'' of impressions. 
We have theoretical understanding of the difference, but Achan Sujin helped us time 
and again to have more understanding of the reality appearing now, at the present 
moment. In this way we learn to verify the truth for ourselves. For instance, when we 
are seeing now, it seems that we see persons and different things like a glass or a table. Persons and different things do not impinge on the eye-sense, they are objects of 
thinking after seeing. Seeing experiences what is visible object impinging on the eye-sense. We can come to know the difference between seeing and thinking very gradually, but it will take a long time since ignorance and attachment have been accumulated for a long time. 

Seeing is not a person, it is a citta (moment of consciousness) that sees. It falls away 
and is then succeeded by a next citta. Only one citta arises at a time, but every citta is 
accompanied by different mental factors, cetasikas, which may be wholesome, unwholesome or neither. Our life is a succession of cittas that arise and fall away, succeeding one another. There is no moment without citta. That is why good and bad 
tendencies are accumulated from one citta to the next one, from life to life. 

When we listen to the Dhamma and consider what we hear, there may be a little more 
understanding of what is real in the ultimate sense. Achan said that we discuss a 
great deal about seeing and visible object, because otherwise we would be forgetful. 
The fact whether there is an interest in the Dhamma today is due to what has been accumulated in the past. 

An illustration of this fact was given to us by a family in Thailand we met after our 
short sojourn in Nakorn Nayok, before we travelled to Sri Lanka. We stopped on our 
way back to Bangkok to visit this family at a place where they sold trees and plants. 
The grandmother is a garbage collector who happened to listen to one of Achan’s radio programs and found that she had never heard before such an explanation of the 
Dhamma. She gained confidence and collected and sold old bottles to make money 
for the ``Dhamma Study and Support Foundation''. Her children were not interested 
in the Dhamma, but recently they independently from each other happened to see 
Achan’s program on the T.V. and hear on the radio one of her lectures broadcast. 
They were impressed by her words and from then on they listened regularly. Now a 
family of three generations, the grandmother, her children and grandchildren have 
great confidence in the Dhamma they heard.  Understanding and interest in the Dhamma does not arise without there being conditions. What has been 
accumulated in the past can be a condition for understanding today. 

Different objects are experienced by citta and if there were no citta, these objects 
could not appear. We are so absorbed in the objects that present themselves that we 
are forgetful of citta.  Achan explained that what appears now to seeing is a reality, a 
dhamma. No matter what we call it, it is that which is seen. Visible object impinges 
on the eye-sense and, after it has fallen away, the impression or sign, nimitta, of visible object is what is left. It seems that visible object lasts for a while, but in reality it 
arises and falls away. We know that seeing arises at this moment, but we cannot pin 
point the citta which sees, it arises and falls away very rapidly and another moment of 
seeing arises. We only experience the ``sign'' of seeing. 

The notion of sign or nimitta can remind us that not just one moment of seeing appears, but many moments that are arising and falling away in succession. Also visible 
object is not as solid as we would think, there are many moments arising and falling 
away in succession which leave the sign or impression of visible object.  Achan Sujin 
used the simile of a torch that is swung around. In this way, we have the impression 
of a whole, of a circle of light that seems to stay. In reality there is no whole. 

Visible object that was experienced by cittas of a sense-door process has fallen away; 
sense-door processes and mind-door processes of cittas alternate very rapidly. Visible 
object impinges again and again and seeing arises again and again. When their characteristics appear we cannot count the different units of rūpa or the cittas that see, 
they arise and fall away; the impression or nimitta of what is seen and of the seeing 
appears. 

Achan Sujin said: ``No matter whether we call it nimitta or not, it is appearing now. 
Whatever appears is the sign or nimitta of the dhamma that arises and falls away.'' 

We cling to what appears for a very short moment, but it does not remain. It is the 
same with saññā, remembrance, a cetasika accompanying every citta. It marks or remembers the object experienced by citta so that it can be recognized later on. There is 
not one moment of saññā that marks and remembers, but countless moments, arising 
and falling away. We can speak of the nimitta of each of the five khandhas: of rūpa, 
of feeling, of saññā , of saṅkhāra-kkhandha (the other cetasikas apart from feeling and 
saññā that can accompany citta), of consciousness. There are nimittas of all conditioned dhammas that appear at this moment, arising and falling away extremely rapidly. 

What is seen is not one reality arising and falling away, it is only the succession, the 
rapidity of the succession of visible object, and it appears as ``something''. The 
nimitta of visible object that arises and falls away in succession gives rise to thinking 
of shape and form, of a concept of a person or thing. 

We do not have to think of nimitta, but it is helpful to know about it: we come to understand that it is not possible to experience just one reality such as one moment of 
seeing or one visible object. It is of no use to try to catch it, wondering about it 
whether it is this visible object or that one. It is gone already. The next visible object 
has appeared already, but its characteristic can be known as just a dhamma. The 
teaching of nimitta gives us an idea of the shortness a reality appears; it is insignificant, not worth clinging to. 

Seeing and visible object are realities, dhammas, but a concept we think of on account of what was seen is not real in the ultimate sense. Realities such as seeing, visible object, hearing, sound, all the sense-cognitions and the sense objects appear all 
the time and when they appear they can be objects of study. The study of the 
Dhamma is not theoretical. We may read texts of the Buddha’s Teachings and believe 
that we, while we read them and ponder over them, we already understand them, but 
that is not so. 

Kevin said while he experiences hardness or softness by touch, that he has to think: 
``This is hardness, this is softness''. Otherwise he would forget the Buddha’s words. 

Achan answered: ``When characteristics of realities are experienced, you need not 
think about words. We can know the difference between thinking about the `story’ of 
dhammas and the direct understanding of them. Whatever appears, there is no one 
there.'' 

She reminded us very often that there is no one there. The Buddha taught the nature 
of non-self, anattā, of realities. There is seeing, but it is not a person who sees, the 
seeing sees, the hearing hears. Seeing and all conditioned realities are non-self, there 
is no one who can cause their arising. They are beyond control. We may believe that 
we see a person, but only what is visible, visible object, is seen, there is no one there. 

Seeing is real, it is a dhamma. We have to understand seeing that arises now, who can 
make it arise? Nobody can have anything at will. It can be said that seeing is a paramattha dhamma\footnote{Paramattha: The highest sense.} and this means that its characteristic cannot be altered. We can give seeing another name, but its characteristic does not change: seeing experiences what 
is visible. Or it can be said that seeing is abhidhamma: subtle dhamma or dhamma in detail\footnote{The prefix ``abhi'' is used in the sense of ``preponderance'' or ``distinction''. ``Abhidhamma'' means ``higher Dhamma'' or ``Dhamma in detail''.}. When we hear the word abhidhamma we need not merely think of the text; 
that part of the Tipiṭaka that is the Abhidhamma explains all realities in detail and it 
pertains to whatever reality appears now. In this sense it can be said that seeing, visible object, hearing, sound, that all realities are abhidhamma. 

We are usually living in the world of concepts and we may have no understanding of 
paramattha dhamma, of what arises because of conditions. It is true that we have to 
think of people and things all day long, otherwise we could not lead our daily life naturally. We have to know that this is a book, that a table. But the difference between 
concepts and ultimate realities can be understood. 

When there is wise attention to the object that appears now, understanding of realities 
can grow very gradually. But it takes a long time not to move away from the present 
object. That is why  Achan reminded us many times of the different phases of understanding: pariyatti, which is not theoretical understanding but intellectual understand ing of the present object; patipatti which is direct understanding of whatever appears; 
and pativedha, the direct realization of the truth. Pariyatti can condition direct understanding and this again conditions pativedha. 

Moreover, there are three rounds of the understanding of the four noble Truths that 
can be discerned: sacca nana, the firm understanding of what has to be known and 
what the four noble Truths are; kicca nana, understanding of the task, that is, direct 
awareness and understanding, Satipaṭṭhāna\footnote{Satipaṭṭhāna is the development of right understanding of mental phenomena and physical phenomena appearing at the present moment. } ; kata nana, understanding of what has been realized, the direct realization of the truth\footnote{See Kindred Sayings V, Kindred Sayings about the Truths, Ch 2, § 1, The Foundation of the 
Kingdom of the Dhamma. }. 

When pariyatti has become firm and more accomplished it is sacca nana. Then one 
does not move away from the dhamma appearing right now and turns to other practices in order to understand the truth. Sacca nana is the firm understanding of the fact 
that every dhamma that arises is conditioned. 

After our discussion with the venerable U Pandita and Kevin, in Negombo, they had 
to return to Colombo, which meant for them about two hours or more in the traffic. 
Achan decided that it would be better to visit them instead in Colombo for the next 
three days. There were sessions in the ``Buddhist Cultural Center'' in Colombo where 
monks, sisters and other people listened and showed their interest by their questions. 
In spite of the many hours we spent in the bus Achan was never tired to have 
Dhamma discussions. She always encourages us not to think of ourselves and our 
own well-being, and then we do not mind when we are in difficult circumstances. 
Whatever experiences arise, they are all gone immediately. 

Sarah often reminded me, even early morning before breakfast in the hotel, that what 
really matters is the development of understanding of the dhamma appearing at this 
moment. She said that we have so much confidence and interest in the Dhamma that 
we travel a long way to hear the Dhamma. While we see its benefit we do not think 
of hardship, ``my pain'' or ``my problems''. 

In the Buddhist Cultural Center questions were raised that pertained to science and 
psychology, and people tried to find a common ground for Buddhism and science. 
However, science belongs to the world of conventional realities and its aim is different from the Buddha’s teachings. We all know conventional realities and we need not 
to be taught about them. When we are thinking about the world and all people in it, 
we only know the world by way of conventional truth. It seems that there is the world 
full of beings and things, but in reality there is citta experiencing different dhammas 
arising and falling away very rapidly. Only one object at a time can be cognized as it 
appears through one doorway. Without the doorways of the senses and the mind the 
world could not appear. So long as we take what appears as a ``whole'', a being or 
person, we do not know realities. The Buddha teaches realities, dhammas that are real 
in the ultimate sense and that can be directly experienced. A medical doctor specialized in brain diseases had questions about memory, believing that this was stored in 
the brain. Achan explained that memory, saññā, is a cetasika, mental factor, that accompanies every citta and that it arises and falls away all the time. It marks each object experienced by the citta it accompanies so that it can be recognized later on. 
When we recognize something or someone saññā performs its function. 

With all her answers Achan tried to help the listeners to understand the present reality, such as seeing, visible object, feeling or thinking. We had heard this very often, 
but every time it seems as if it is new. We are forgetful of realities and that is why we 
found it beneficial to listen again and again to her explanations about seeing now. 

She said: ``The seeing that now sees is not `I’. Life is the experience of one object at a 
time. There is always the idea of `I see, I hear, I think’. There are only different realities, no one at all. 

Self is trying so hard to understand realities, but without the right conditions it is impossible.'' 

Even when we do not think of a self who is seeing, there is still a notion of ``I see”. It 
is so deeply engrained. Achan often said that understanding is not yet sufficient to be 
detached from the idea of self. Patience is needed because it takes a long time for understanding to develop. The term pariyatti was often discussed. It is understanding of 
what appears at this moment. We may feel hot and then we usually have a notion of 
``I'' who feels hot. We live in the world of concepts, instead of understanding what is 
reality. But sometimes one characteristic of reality such as heat or bodily feeling may 
appear. Then we begin to understand the reality of the present moment and even at 
this level there can be some detachment from taking heat or feeling for self or mine. 
Without pariyatti there will not be conditions for right awareness and right understanding of the Eightfold Path\footnote{The eightfold Path are the cetasikas of right understanding, right thought, right speech, right bodily action, right livelihood, right effort, right mindfulness and right concentration. They develop 
together so that realities can be seen as impermanent, dukkha and anattā. }. Pariyatti, when it is firmly established, conditions 
patipatti. Patipatti is often translated as practice but there is no one who practices. 
When people hear the word practice they are inclined to think of doing specific things 
in order to attain enlightenment. However, it is a level of paññā that directly understands the dhamma that appears at the present moment. It is of the level of 
Satipaṭṭhāna. It can only arise when there are the right conditions, it is not under any 
one’s control. 

Achan often reminded us that there cannot be patipatti when understanding of the 
level of pariyatti has not been sufficiently developed. 



\chapter{One path}

One of the listeners in the Buddhist Cultural Center had doubts about the truth of the 
Abhidhamma. He was thinking about the history of the text and thought that it was of 
a later date.  Achan answered: ``Whatever appears is Abhidhamma. Is it `I' who 
sees?'' 

In other words, Abhidhamma is not theory, we can come to understand what Abhidhamma is if there is wise attention to the present reality. The text of the Abhidhamma teaches that there is no person, no self, only citta, consciousness, cetasika, 
mental fators accompanying citta and rupa, physical phenomena. There is an unconditioned dhamma, nibbāna, but this can only be experienced through the attainment 
of enlightenment. In our daily life citta, cetasika and rūpa appear all the time. Seeing 
is not self, it is only a conditioned dhamma. Nobody can cause its arising. 

For most people Achan’s explanations were new and they found it hard to grasp immediately what she said. Sometimes her answers are short, but deep in meaning. 
Therefore, it was most helpful that Sarah elaborated on  Achan’s words in answering 
questions. Sarah and Jonothan assisted all the time during the sessions in adding more 
explanations to  Achan’s words. 

Realities are different from concepts that can be objects of thinking but are not real in 
the ultimate sense.  Achan would time and again speak of seeing and explain that 
seeing is real. We are seeing all the time but we know so little about it. Instead of giving theoretical explanations about realities and concepts Achan would speak about 
what appears at the present moment in order to help the listeners to understand what 
is real. At the moment of seeing there is no idea of a human, a bird or ``I''. There is 
just a conditioned dhamma and no one makes it arise. The citta that sees falls away 
but it conditions the following citta to succeed it. 

Some monks from Bangladesh showed great interest and one of them asked why the 
Buddha had spoken about groups of rūpa, kalapas. 

Rūpas do not arise singly, they arise in units or groups. What we take for our body is 
composed of many groups or units, consisting each of different kinds of rūpa, and the 
rupas in such a group arise together and fall away together. 

There are four kinds of rupa, the four ``Great Elements'' (Maha-bhuta rūpas), which 
have to arise together with each and every group of rūpas, no matter whether these 
are rūpas of the body or rūpas outside the body. The types of rūpa other than the four 
Great Elements depend on these four rupas and cannot arise without them. They are 
the following rūpas: 

the Element of Earth (pathavldhatu) or solidity, appearing as hardness or soft 
ness, 

the Element of Water (apodhatu) or cohesion, 

the Element of Fire (tejodhatu) or heat, appearing as heat or cold, 

the Element of Wind (vayodhatu) or motion, appearing as oscillation or pressure. 

Every day we experience a great variety of sense objects, but they are, in fact, only 
different compositions of rūpa elements. When we touch a cushion or chair, tangible 
object may appear, such as hardness or softness. We used to think that it was a cushion or chair which could be experienced through touch. When we are more precise, it 
is hardness or softness that can be experienced through touch. Because of remembrance of former experiences we can think of a cushion or chair and we know that 
they are named ``cushion'' or ``chair''. This example can remind us that there is a difference between ultimate realities and concepts we can think of but which are not real 
in the ultimate sense. 

The Buddha taught about the groups of rūpa and each rūpa that arises is conditioned 
by the other rūpas in that group. It is entirely dependent on conditions and there is no 
body who could cause its arising. The Buddha taught the nature of anattā of each 
dhamma. Visible object is always accompanied by the four great Elements and by 
other rūpas arising in a group. There have to be at least eight rūpas in each group. 
Apart from the four Great Elements these rūpas are visible object, odour, flavour and 
nutritive essence. Visible object arises in every group of rūpas, but only visible object 
impinges on the eyesense, the other rūpas of that group do not. We believe that we 
see a person but a person cannot impinge on the eyesense. However, there could not 
be an idea of ``person'' if visible object did not impinge on the eyesense and there 
would not be seeing. 

Rūpas are classified as twenty-eight, but seven types appear all the time in daily life. 
They are: visible object, sound, odour, flavour, and three kinds of tangible object which 
are solidity (appearing as hardness of softness), temperature (appearing as heat or 
cold) and motion (appearing as motion or pressure). The Element of Water or cohesion cannot be experienced through touch, it can be experienced only through the 
mind-door. 

All the texts of the Tipiṭaka, including the Abhidhamma, are not meant merely for intellectual study or memorizing, they are directed to the development of direct understanding of realities. The classifications in the texts of the Abhidhamma of cittas, 
cetasikas and rūpas are an exhortation to develop understanding of whatever reality 
appears at this moment. This is the development of the eightfold Path leading to the 
eradication of all defilements. 

The Abhidhamma teaches about different cittas: cittas that are kusala, wholesome, 
akusala, unwholesome, vipaka, result of kamma, or kiriya, inoperative, not kusala, 
akusala or vipaka. A citta never arises alone, it is accompanied by several cetasikas, 
mental factors. Some cetasikas accompany every citta, such as remembrance (sanna) 
or feeling. Some accompany only akusala citta or only kusala citta. Each citta cognizes an object, that is its function. Akusala cetasikas or beautiful cetasikas that accompany it cause the citta to be akusala or kusala. When we are attached to people 
we are inclined to believe that it is self who is attached. However, it can be understood that attachment (lobha) is only a cetasika that is conditioned to accompany akusala citta at a particular moment. We were attached in the past and, thus, this inclination is accumulated from one citta to the next citta so that attachment arises again. 

Our life is an uninterrupted series of cittas arising in succession. That is why good 
and bad qualities are accumulated and carried on from moment to moment. 

We read in the text of the ``Path of Discrimination'' (Patisambhidamagga, Ch 69, 585) 
more about the meaning of accumulation: 
\begin{quote}
``Here the Perfect One knows beings’ biases, he knows their underlying tendencies (asayanusaya nana), he knows their behaviour 
(carita), he knows their dispositions (adhimutti), he knows beings as 
capable and incapable\ldots ''
\end{quote}
The Commentary to the ``Path of Discrimination'' (the ``Saddhammappakasinī'') gives 
explanations about the knowledge of beings’ biases and underlying tendencies: 
\begin{quote}
``As to the term anusaya, bias, they explain this as dependence, abode or 
support on which beings depend. This term denotes the disposition to 
wrong view or to right view that has been accumulated. It denotes the 
disposition to all that is unwholesome, such as clinging to sense objects, or the disposition to all that is good, such as renunciation that 
has been accumulated. 


The defilements that lie persisting in beings’ continuous stream of cittas are called anusaya, latent tendencies. This term denotes the defilements such as clinging to sense objects that is strong.'' 
\end{quote}

Thus, anusaya, latent tendency, refers to unwholesome inclinations that lie dormant 
in every citta. They do not arise but they can condition the arising of akusala citta. 
Anusaya refers to both wholesome and unwholesome inclinations that have been accumulated and can condition the arising of kusala citta or akusala citta. 

There were ignorance and clinging in past lives and these have been accumulated 
from life to life and that is why there are conditions for their arising time and again. 
However, in listening to the Dhamma and truly considering it, there may be a little 
more understanding of realities. A moment of understanding is never lost, it is accumulated in the citta so that it can arise again and grow very gradually. 

We should understand first what is dhamma, before we can understand kusala and 
akusala dhammas. As Achan often said, kusala is not a person, akusala is not a person. It is dhamma, only a reality. Generosity (alobha) may arise and this is a cetasika 
that accompanies kusala citta. It arises and then falls away with the citta. It cannot 
stay and it does not belong to a self. Evenso, aversion that arises is a cetasika that 
may accompany akusala citta. It cannot stay and does not belong to a self. 

Seeing and hearing arise time and again. These are vipakacittas, results produced by 
past kusala kamma or akusala kamma. Kamma is accumulated from one citta to the 
next citta and it can produce its appropriate result later on. 

Our world seems to be full of people, but there are only two kinds of reality: mental 
phenomena or nāma and physical phenomena or rūpa. Citta and cetasika are nāma, 
they experience an object, whereas rūpa does not know anything. Visible object and 
eyesense are rūpas that are conditions for seeing, they do not know anything. At each 
moment of life there is citta, accompanied by cetasikas. What we call body are 
groups of rūpa, arising and falling away. If there were no citta, the body could not 
move. 

One of the monks said that there are different Paths for different people. He emphasized good behaviour in family life as essential for lay people. Achan said that all the 
teachings point to right understanding. Kusala sīla is the wholesome behaviour of 
citta. Sīla before the Buddha’s time was different from the sīla he taught. 

Achan said: ``Not killing is what everyone can say, not only a Buddha. He taught 
that there is no one, no self, at any time. All dhammas are anattā . With regard to not 
killing, there is no ``I'' at all, only wholesome mental factors. There is no one. There are seven anusayas: sensuous desire, aversion, wrong view, doubt, conceit, craving for 
existence and ignorance. There may be wrong understanding, taking realities for self, from life to life. All the teachings are about understanding realities as not self. Ignorance, moha, and aversion, 
dosa, kill, not a self. His teachings are different from others' teachings. Morality with 
understanding is his teaching. One should not just follow his words, but develop one’s 
own understanding.'' 

An illustration of the fact that sīla with understanding is the Buddha’s teaching we 
find in the ``Vyagghapajja Sutta'' (Gradual Sayings, Book of the Eights, Ch VI, §4 8 ). 
We read that Vyagghapajja visited the Buddha and asked for an instruction leading to 
happiness in this life and in a future life. First the Buddha explained the conditions 
for worldly progress using conventional terms expressing situations in everyday life. 
He spoke about abstaining from debauching, drunkenness, gambling and from friendship with evil doers. After that he spoke about the conditions for spiritual progress: 
the accomplishment of faith (saddha-sampada), the accomplishment of virtue (sīla-sampada), the accomplishment of charity (caga-sampada) and the accomplishment of 
wisdom (paññā-sampada). The accomplishment of faith is confidence in the Triple 
Gem and this points to right understanding. When right understanding is being developed confidence in the Buddha who taught the truth about realities is ever growing. 
The accomplishment of sīla is abstaining from killing, stealing and the other akusala 
kamma the abstention of which is contained in the five precepts. The sotapaññā who 
does not believe in a self who is abstaining, will never transgress these five precepts. 
He really has the accomplishment of sīla. Even so the sotapaññā who has eradicated 
all stinginess has the accomplishment of caga. Caga in its widest sense is actually relinquishment, giving up, renunciation from all akusala. 

As to the accomplishment of wisdom, we read: 

\begin{quote}

``Herein a householder is wise: he is endowed with wisdom that understands the arising and cessation (of the five aggregates of existence); he is possessed of the noble penetrating insight that leads to the 
destruction of suffering. This is called the accomplishment of wisdom.'' 
\end{quote}

The Buddha taught that all dhammas, including sīla, are anattā. Whenever kusala sīla 
arises we should understand that it does so because of the right conditions and that 
there is no self who can make an effort for kusala sīla. If one thinks that one should 
accumulate more sīla so that later on there will be more understanding of realities, 
this is not according to the Buddha’s teachings. When kusala citta does not arise, akusala citta arises very often. Right understanding sees the danger of akusala. One can 
begin not to neglect any kind of kusala, be it even of a slight degree. There can be more conditions for kindness, compassion and helpfulness. Sīla accompanied by right 
understanding can lead to enlightenment. There is actually only one Path, the development of right understanding of realities. 

Questions about samatha, the development of calm, were raised. Someone thought 
that samatha had to be developed before vipassanā.  Achan explained that one should 
know what calm is. If one expects or wishes to be calm it is attachment and there is 
no understanding. For both samatha and vipassanā right understanding is indispensable. Understanding has to know when the citta is kusala and when akusala, lest one 
mistakes akusala for kusala. One may believe that one is calm whereas in reality 
there is attachment. One may be attached to being in a quiet place without any noise. 
True calm (passaddhi) arises with every sobhana citta. When one assists someone 
else with kindness or one abstains from harsh speech, there is calm with the kusala 
citta. When someone develops samatha there are specific subjects of meditation, such 
as recollection of the Buddha, the Dhamma and the Sangha, or death. Paññā in samatha has to be very keen so that it is known how true calm is to be developed with a 
meditation subject. 

Mindfulness of breathing, anapanasati, is a subject that is often misunderstood. One 
may believe that by concentrating on breath calm can arise. We should first know 
what breath is. It is rūpa, conditioned by citta. Rūpas of the body can be conditioned 
by four factors: by kamma, by citta, by nutrition and by temperature. Breath is conditioned by citta. It is very subtle and very easily one may take for breath what is not 
breath. The rūpa that is breath is different from breath as we use this word in conventional language. Even when we are holding our breath as we say in conventional language, citta still produces the rūpa that is breath. Citta produces breath from birth to death. 

One may believe that samatha is developed by concentrating on a meditation subject. 
Concentration can also arise with akusala citta; it focusses on an object in an unwholesome way. Paññā has to know when it is akusala and when kusala. When some 
one tries very hard to concentrate on a meditation subject, there may be attachment 
instead of calm. 

The word meditation often leads to misunderstandings. In Pāli the word bhavana is 
used and this means developing. There is samatha bhavana and vipassanā bhavana 
which are different ways of development and have a different aim. Before the Buddha’s time samatha was developed by those who saw the disadvantage of clinging to sense objects. They developed calm to a high degree in order to become free from 
sense cognitions and from being involved in sense objects. vipassanā is taught only 
by the Buddha. It is clear understanding of whatever reality appears at the present 
moment. This understanding eradicates ignorance and wrong view. 


Throughout the discussions  Achan would frequently remind us never to move away 
from the present object. She would explain whatever can be understood right now. 

We think of a self who is seeing or hearing, but seeing and hearing should be understood as not self. Ignorance and wrong view are deeply engrained and that is why 
each reality is taken as ``I''. Understanding has not been developed sufficiently so as 
to abandon the idea of self. 

There is much to learn and consider so that understanding can very gradually develop. One may like to understand anattā now, to experience directly the arising and 
falling away of dhammas, but that is impossible. Only paññā can realize the truth, 
there is no one who can do anything.  Achan said: ``The opportunity to listen to the 
Dhamma is not easy to find, it depends on conditions. Who knows what will happen 
the next moment or tomorrow? This moment of hearing the teachings can be accumulated little by little. Confidence in the teachings is more valuable than anything else 
in the world. The most precious thing in life is a moment of understanding. The 
teachings are very subtle and, therefore, more words of explanation and more consideration are necessary, otherwise there are conditions for forgetfulness of realities. 
Paññā can begin to see the danger of not understanding reality as it is, since ignorance will condition more and more akusala.'' 



\chapter{Living Alone} 




Sometimes we may feel lonely, but  Achan reminded us time and again that there is 
no one who lives alone, that there are only different realities arising and falling away. 
We have heard this many times, but when we are in a difficult situation, such as the 
experience of the loss of a dear person, we tend to forget that in truth life is only one 
moment of citta experiencing one object that presents itself through one of the senses 
or the mind-door. When seeing, life is seeing that experiences only visible object, not 
a person. When hearing, life is hearing experiencing sound. When thinking of a person life is thinking, and the object of thinking is a concept, the concept of a person, 
and each of these moments falls away immediately. 

Huong, one of our friends, wanted to join us in Negombo, but while she travelled she 
lost her friends in Vietnam twice since they had missed the plane. She did not know 
the right address and sent an Email to the Dhamma Study Group after she had moved 
into a hotel in Negombo. Sarah happened to check Emails very early in the morning 
and saw Huong’s post. She did not hesitate and immediately took a taxi to fetch Huong at her hotel. It was all by conditions that Huong could meet us. Huong’s mother 
had recently passed away and Huong was going through a difficult time because of 
her loss.  Achan said: 

“The thinking is gone, everything is gone, each moment. Understanding is not sufficiently developed to see that whatever appears is dhamma. There is an idea of ‘my 
mother’, but even what is taken for mother is dhamma, and thinking about her is 
dhamma. Whatever appears, whatever arises is a dhamma. What we take for some 
thing or someone are in reality different dhammas\ldots

We should not forget that whatever appears, whatever arises must be a dhamma only, 
one dhamma at a time\ldots It's only a reality. That's why we leam about reality to understand reality as not self, otherwise there must be `I'. When there's T, there's `my 
Mom', Right? But when we talk about realities, we begin just to learn that whatever 
we take for someone or something are in reality different dhammas.''

Developing understanding of realities does not mean that we do not think of persons 
anymore, but we learn to discern the difference between what is real in the ultimate 
sense and what is a ``story'' we may be thinking of. 

When we feel lonely because of the loss of a dear person, unhappy feeling arises. But 
feeling does not last, it falls away immediately. We continue holding onto our feelings and sad thoughts, and we think of ourselves as being alone. It is natural that we 
think of dear persons with attachment though we understand intellectually that there 
is no one there. 

 Achan said: ``There is no one there, only seeing arises and it sees. There is no one at 
the moment of hearing, no one who hears.''

Sarah remarked: ``The development of understanding is learning to live alone, each 
moment. That is the nature of dhamma: always arising alone, falling away alone. 
From the moment of birth until the moment of death we are alone, there is never any 
one there. Citta arises on its own by conditions. No matter the circumstances, no matter how difficult they are, we have to remember that citta is always alone, at each moment.'' 

After five days in Negombo we travelled by bus for about five hours to our next destination: Nuwara Eliya. This is located in the mountains, past Kandy. The hotel was 
at 1800 meter altitude. We passed tea plantations on the green hills and waterfalls. 
Our guide explained that in Sri Lanka nature is well protected and no more trees are 
to be cut. If someone would cut a Jackfruit tree he can await five years in prison. Our 
bungalow type hotel was situated high above terraces full of flowers. We had 
Dhamma sessions in the morning and in the afternoon. It was the rainy season but in 
the morning we could sit outside for our discussions.  Achan went out together with 
some of our group for a morning walk and Sarah told me that the talk was about 
flowers and plants all the way. This is natural, daily life. There should be no selection 
of objects that are experienced. At another moment  Achan asked people about the 
food they would eat and explained that we should not avoid talking about these 
things. We enjoyed the Singhalese breakfast with the traditional ``egghoppers'', pan 
cakes with an egg. If we believe that there cannot be awareness when talking about 
such subjects we go the wrong way. Thinking is real, but it is gone immediately. 

Our friends Vince and Nancy also came over for the sessions. We have known them 
for a long time. Vince spoke about Lodewijk, my late husband, with great appreciation and kindness. He asked me full of concern whether it was not a great change for 
me to live alone. I reacted later on to his question when the owner of the bungalow, 
Rajid, attended our sessions. We discussed about realities and the importance of 
knowing whatever appears at the present moment. This is the way leading to the understanding that there is no self, no person, only mental phenomena and physical phenomena arising and falling away. It is not easy to see that this truth concerns our daily life and, therefore, I returned to Vince’s question about my reaction to Lodewijk’ s death in order to help Rajid to see the relevance of Dhamma to daily life. I explained that we learn from the Buddha’s teachings that even when a person is still 
alive there are just fleeting phenomena, only citta, cetasika and rūpa that arise and 
pass away instantly. Through the Dhamma our outlook on life can very gradually 
change and we can learn to live alone, even when we have a loss. Citta is always 
alone, each moment. 

Sarah said that the Dhamma is the best medicine but that it is not always easy to take 
this medicine. We hear about realities appearing now and that there is no person. We 
have to consider this, we cannot understand the truth immediately. It takes more than 
one life, but this does not matter. It is good that we begin to listen and consider. Understanding can only grow very gradually. It is useful to have discussions and remind 
one another of the truth. 

I said that although I understand intellectually that a person does not exist, I still find 
it difficult to accept that Lodewijk, after passing from this life and going on to an 
other life, does not care for me anymore.  Achan said that after dying-consciousness 
has arisen and fallen away and rebirth-consciousness has arisen, there is for the reborn being no more attachment to a particular person. Lodewijk cared for me in the 
past and I have some idea that he can still care for me.  Achan asked me whether I remember my past life? It is helpful to consider that this life is only one short period in 
the innumerable lives of the past and the lives yet to come. Thinking with attachment 
is very natural, but there are only dhammas. Dying-consciousness is succeeded by rebirth-consciousness and then there is a new life, a new story. 

She reminded us: ``We have to understand whatever appears. It is conditioned, no 
body can control it, no matter pleasant or unpleasant experiences occur. When it is 
gone, it is gone. But we think that an experience is there all the time. From nothing to 
something to nothing, never to arise again. One characteristic of reality appears and 
then it does not ever appear again. It seems that the world continues to be the same, 
but actually it never is the same. Whatever occurs is only once in the cycle of birth 
and death, and then never again. This is the truth. It seems that seeing continues all 
the time, but there are countless moments of seeing. There is the succession of the 
arising and falling away of realities, appearing through six doorways all the time.'' 

In the ``Kindred Sayings'' (Ch III, Kosala, Persons, § 2 Grandmother) we read that 
King Pasenadi visited the Buddha. The Buddha asked him why he had come at this 
hour of the day. The King said: 

\begin{quote}
`` `My grandmother, lord, is dead. She was aged and full of years; long 
her span of life, long her life’s faring. She has passed away in her 
102th year. 

Now, my grandmother, lord, was dear to me and beloved. If I had 
been offered the gift of a priceless elephant [or that her life might be 
preserved], I should have chosen that my grandmother had not died; 
nay, I would have given the elephant away to save her life. I would 
have done no less had I been offered, or did I possess a priceless 
horse, or the choice of a village, or a province.’ 

`All beings are mortal; they finish with death; they have death in prospect.’ 

`That is notably and impressively said, lord\ldots’ 

`Even so, sire, even so\ldots Even as all vessels wrought by the potter, 
whether they are unbaked or baked all are breakable. They finish 
broken, they have breakage in prospect. 
\end{quote}

\begin{verse}
All creatures have to die. Life is but death. \\
And they shall fare according to their deeds, \\
Finding the fruit of merit and misdeeds: \\
Infernal realms because of evil works;\\ 
Blissful rebirth for meritorious acts\ldots' '' 
\end{verse}

Wholesome deeds, kusala kamma, is accumulated and can produce a happy rebirth or 
pleasant sense impressions during life. Whereas evil deeds, akusala kamma, is accumulated and can produce an unhappy rebirth or unpleasant sense impressions during 
life. A wholesome deed or unwholesome deed committed in a past life can produce 
its appropriate result even after aeons. It is never lost but it is accumulated in the citta 
and passed on from moment to moment. 

We usually think of death as the end of a lifespan, but in reality there is birth and 
death at each moment of citta that arises and passes away. A moment of seeing that 
arises now falls away instantly, never to return. What we take for life are only fleeting realities. 

Huong asked me whether I have attachment now to my late husband. 

I answered: ``It depends on the citta at a particular moment whether attachment to my 
late husband arises. There is no attachment to thinking of him all the time, when seeing, or when having fun and laughing. There is only one citta at a time, experiencing 
one object. We may think of a whole situation of being sad, missing a dear person, 
but that is only a moment of thinking. It seems to last, but that is not according to re ality. There are only seeing, hearing, other sense-cognitions or thinking and they are 
all gone immediately.'' 

The teachings are very subtle and the truth of there being no person cannot be penetrated immediately. We may understand intellectually that there is no person, no self, 
but, as  Achan emphasized many times, there are not yet sufficient conditions for direct understanding and direct awareness of the truth. If we wonder about it how to 
have more conditions there is no understanding of dhamma at all,  Achan said. We 
hear the word anattā and think about it, but there is no direct understanding of a reality that appears as only a dhamma. We have to be truthful to what is real now. 

Achan said many times: ``What is seen cannot be anything at all, this is the way to 
have less attachment.'' 

What is seen, visible object, is only a type of rūpa that arises, impinges on the eye- 
base and then falls away. It is present for an extremely short moment, but we think 
about it for a long time, clinging to what has gone already. It seems that we see people, but they are not there, they are only objects of thinking, not of seeing. There is no 
one there, no person, that is the truth of anattā. 

Visible object is an element, this means: devoid of self. It is a conditioned element, it 
is conditioned by the four great Elements that always arise together with it. The composition of these Elements is different at different moments: sometimes heat is more 
intense or it may be less, or hardness is more intense or it may be softer, and this 
causes the visible objects that are seen to be different. It seems to us that visible object can stay, but since it is conditioned, it falls away, it cannot stay. 

Achan said: ``We talk about what appears, about seeing, hearing, smelling, but there 
is no understanding of these realities as not `I' at all. Then there is no paññā. Every 
one knows ‘I see, I hear, I think’, but at this moment there is seeing and seeing is 
conditioned. Little by little paññā begins to understand that it is not `I'. But it is not 
developed enough to give up the idea of self at all, no matter how many times we 
hear about this, for years, or our whole life. It depends on paññā and sati. Even at the 
moment of kusala there is sati, but sati is not apparent. So how can paññā see sati as 
sati which arises with kusala citta, when it is not apparent. But when paññā develops 
on and on there are moments of understanding reality, not only stemming from hearing, but sufficient to directly understand the nature of it as just a reality. Very naturally.'' 



\chapter{Treasures} 

The subject of pariyatti was often discussed. We are bound to have misunderstandings about pariyatti and take for pariyatti what is only thinking about the teachings. Achan said: ``At the moment of understanding reality as not self, that is pariyatti.'' 

Sarah remarked: ``That is what is meant by not moving away from the present object.'' We should not mind when there is only thinking about the teachings, because 
whatever has conditions for its arising arises. 

One may wonder whether there is just thinking about realities at a particular moment 
or whether there is pariyatti. Sarah said to me: ``Behind such thinking there is strong 
clinging to self, to `my understanding’ and wanting to know how I can have result. 
Such thinking occurs when it is not understanding dhamma as anattā at this moment.'' 

We read in the ``Dhammapada'' (vs 76):
\begin{quote}

 ``Should one see a wise man, who, like a revealer of treasures, points out faults and reproves, let one associate with such a wise 
person; it will be better, not worse, for him who associates with such a one.'' 
\end{quote}

Being an ordinary person (puthujjana), I am full of ignorance, wrong view and clinging. I often do not see when these defilements arise. Even when I do not think: this 
belongs to me, this is myself, I am still full of the idea of me, me, me. As soon as I 
open my eyes and seeing arises, there is still the idea: I see, even when I do not say so 
or think so, it is there. When a good friend reminds me of my ignorance and wrong 
view I am just grateful, otherwise I would not know about them. I receive a treasure. 

There may be a subtle trying to have awareness or to catch realities. Effort or viriya is 
a cetasika that arises with many cittas. We may believe that we do not try to have 
awareness. But there is a very, very subtle trying that is unknown to us. We wish to 
avoid akusala, we wish to have more understanding and not to be full of ignorance. 

At such moments there is a subtle trying. Paññā may come to know such moments. 

Sarah said: ``It is self trying. The most precious moment is understanding the anattāness of realities, not just want to have understanding. This is the way to have less attachment.'' 

We had a discussion about sati and I mentioned that there is sati with each kusala 
citta. Sati of the level of thinking. Our discussion was as follows: 

Achan: ``Thinking about I, all about I.'' 


Sarah: ``It is all about the `story’ of sati.'' 

Nina: ``At the moment of kusala citta?'' 

Sujin: ``It is true, but you do not need to talk about it.'' 

We often think about the ``story'' or concept of realities instead of understanding their 
characteristics.  Achan explained that when there are the right conditions for sati it is 
aware and at that moment there is no thinking about realities, no need to talk about 
sati. It arises unexpectedly, nobody can plan it. We talk so much about sati and about 
what level it is, and then we cling to an idea of ``my sati''. Or we may think that we 
are not yet ready for sati of the level of Satipaṭṭhāna. Also at such moments we are 
clinging to the idea of self who is not yet ready. If we do not mind whatever reality 
arises, even if it is a very unwholesome thought, and it is understood as just a conditioned dhamma, it is pariyatti. If friends would not remind me of the truth, I would go 
around life after life, not knowing about clinging to the self, to my thinking. Thus, 
these are rare gems to receive in the cycle of birth and death. 

If paññā of the level of pariyatti has not been developed, there are no conditions for 
the arising of direct understanding of realities. Pariyatti is not reflecting about words 
and their meaning.  Achan often asked us: ``Is there seeing now?'' It is a reminder that 
there is seeing at this very moment, not an object we merely think about. What appears must be a reality, it is conditioned. If there were no conditions it could not appear. By understanding this, it is of the level of pariyatti. As understanding develops 
there will be detachment from taking things as a whole, as a being. 

Pariyatti is understanding of what appears, but it is not yet direct understanding. Achan said: ``The understanding of realities is pariyatti, it is not understanding 
merely what this word pariyatti means. It is understanding of reality appearing now. 

It is the understanding of any moment, it is not thinking. It is not remembering the 
words of the texts. We can think about them, but pariyatti is the firm understanding 
of realities. Without pariyatti one goes the wrong way. Pariyatti is the understanding 
of realities which will condition direct understanding. If it is not firmly established it 
can never condition patipatti, Satipaṭṭhāna. Pariyatti is not listening and thinking. The 
moment of understanding the truth of reality is pariyatti. We do not have to pinpoint 
whether there is intellectual understanding or direct understanding. Right now we talk 
about seeing, visible object, hearing, thinking, but it is not as clear as when paññā is 
developed to the degree that it can directly experience these realities.'' 


It seems that we understand what dhamma is, but when it arises, there is no understanding. More intellectual understanding of what appears now, pariyatti, will condition patipatti. Intellectual understanding knows just the ``story'' but it is not direct understanding. 

The development of understanding should be very natural, there is no need to go to a 
special place or assume a special posture. One should not select a specific object for 
sati and paññā.  Achan often said: ``no one can do anything''. This is not an excuse to 
be indolent. It is a warning not to cling to an idea of having sati and paññā by engaging in specific actions. Then we fall into the trap of lobha and wrong view. It is difficult to go against the current of the accumulated lobha, ignorance and wrong view. The Buddha’s teachings can condition less attachment to the object that appears. 

Very gradually it can be learned that not a self sees but that seeing sees, that there is 
no thing, no person in visible object. There is no person, only citta, cetasika and rūpa. 
As  Achan reminded us, who understands the teachings of the Buddha will listen 
more, consider more and leave it to anattā, because the development of understanding 
is conditioned. If one tries to have it, it is not anattā. What arises in the world is two 
kinds of realities: the reality that experiences and the reality that does not experience 
anything. We should discern the true nature of nāma and of rūpa but we do not have 
to call them nāma and rūpa. Confidence can become firmer, confidence that there is a 
way to know directly realities, not merely by thinking about them. 

Right understanding has to be developed together with the ``perfections'', paramis, \footnote{The perfections or paramis are: generosity, morality, renunciation, wisdom, energy, patience, 
truthfulness, determination, loving kindness, equanimity. The Buddha developed these for aeons in 
order to become the Sammasambuddha. }
which support right understanding. In the Commentary to the ``Cariya Pitaka'', in the 
introduction (Nidana Katha) we read about four ways of development which indicate 
that paññā has to be developed during innumerable lives. They are: 

The complete development of the entire range of the Perfections, sabbasam-bhara-bhavana.\footnote{Sambhara can mean requisite or ingredient.} 

Development without interruption, nirantara-bhavana. The development of the 
Perfections throughout the minimum period of four asankheyya (incalculable 
period) and a hundred thousand aeons, or the medial period of eight 
asankheyya and a hundred thousand aeons or the maximum period of sixteen 
asankheyya and a hundred thousand aeons, without a break of even a single existence. 






Development for a long time, cirakala-bhavana, the development of the Perfections for a long duration which is not less than the minimum period of four 
asankheyya and a hundred thousand aeons. 

Development with respect, sakkacca-bhavana, the development of Perfections with seriousness and thoroughness. 

Earlier  Achan had given some additional explanations and Sarah rendered these as 
follows: 

Sabbasambhara-bhavana refers to how kusala `ingredients' are conditions for the development of understanding. Without kusala, there is just akusala all day. Any kusala, 
however small, should be developed. Without right understanding, it won't be known 
or developed. 

Nirantara-bhavana refers to the fact that understanding does not develop at once, but 
over lifetimes to come with continuous development. It doesn't matter how long. 

What is important is that the understanding about anattā is firmly established. Nirantara means without interval, in the sense of continuously, forever. The development 
of understanding takes great patience for a long time, with no thought of being engaged with a particular practice in order to have a result more quickly. This is the 
only way to be freed from being enslaved by attachment. 

If there is desire for results or impatience with the path, it shows that understanding is 
not firmly established. 

When people read about development without interruption, continuously, nirantara-bhavana, misunderstandings may arise. This does not mean that there has to be mindfulness and understanding all day long, without interruption. Nobody could force the 
arising of mindfulness, it is anattā. One should be firmly established in anattā as Achan said. 

Achan explained: ``Understanding should not be only once but since all realities are 
anattā, not under one's control, no one can try to force having it all day impossible. 
So this means not just in this life, but in whatever life to come, one is firmly established in anattāness. If there is the idea of ‘atta’, it is against the teachings, the Truth, 
that’s why it’s very subtle. Take courage to really understand what is now, not ‘I try to 
have awareness’. It doesn't mean to have awareness continually, it means no matter in 
what life anytime.'' 





Cirakala-bhavana is the understanding of how long it takes before there is understanding of what the Buddha said. There is no doubt about what had to be known and 
realized by the previous followers of the Buddha. 

Sakkacca-bhavana\footnote{Sakacca means having honoured, respected. } refers to the respect for each word of the Teachings. When respectfully understood, there are conditions for more understanding and less ignorance. 

When we read about the incalculable periods it can remind us that paññā has to continue to develop lifetime after lifetime in order to understand the reality appearing at this moment. When we begin to see that clinging to self is deeply rooted we can understand that the development of paññā together with all the perfections must take innumerable lives. 

We read in the commentary to the Cariyapitaka about the means by which the perfections are accomplished, and it is said that they should be performed perseveringly 
without interruption, and that there should be enduring effort over a long period with 
out coming to a halt halfway. We may become discouraged when we do not see a 
tangible result of the development of understanding. Then we are thinking about self, 
about ``my lack or progress'', instead of developing understanding of whatever reality 
appears at the present moment. We are falling into the trap of lobha, as we were reminded several times during our journey. 

There is lobha time and again, even while studying the teachings or listening to 
Dhamma. One should understand the nature of anattā, also at such moments. Achan 
said: ``When we are talking about what appears, attachment may arise and then reality 
does not appear as it is. Attachment covers up the nature of anattā, so that one does 
not know that seeing and that which is seen are not anyone at all.'' 

The perfections should be developed now, at this moment. There should be patience 
and determination to begin to understand realities that appear, they should be investigated very carefully. We need truthfulness to investigate all realities of daily life. It 
takes time to really understand and follow the teachings. We need the perfection of 
patience, khanti, energy or courage, viriya, truthfulness, sacca and determination, 
aditthana. Without patience paññā cannot grow. We need the perfection of equanimity to face the worldly conditions such as gain and loss, praise and blame, without being disturbed by them. When understanding becomes firm we have more confidence 
in the Buddha’s words. Whatever experiences through the senses occur, pleasant or 
unpleasant, they all have conditions for their arising. We shall be less inclined to 
think of our own well being. Many conditions are necessary for the development of 
paññā, there is no self who can do anything. 

We need energy and courage, viriya, to listen again and again to the Dhamma. When 
we are convinced that hearing true Dhamma and right understanding are the most 
valuable in our life we appreciate the opportunity for listening we still have in this 
human plane. We do not need to think of the perfections or enumerate them, they are 
any kind of kusala though body, speech and mind. 

We may not like having akusala citta, but when there are conditions it arises. Nobody 
can change it. If we are not courageous enough to develop understanding of akusala 
we shall forever be ignorant of akusala cittas that are bound to arise, even in between 
moments of doing generous deeds. We cling to a notion of self who performs kusala 
and forget that each citta is impermanent and non-self. 

When akusala citta arises we can verify how much understanding there is. Do we try 
not to have akusala? We may be looking for different ways in order not to have it. 

 Achan explained: ``I do not think, ‘defilements are so ugly’, they are just realities. 
There should be understanding of them. People want to get rid of all defilements but 
they do not have any understanding of them. Why should our first objective not be 
right understanding? I do not understand why people are so much irritated by their 
defilements. One is drawn to the idea of self all the time, while one thinks about it 
whether one has less defilements or more. There is no understanding but merely 
thinking of kusala and akusala as ‘ours'. So long as there is ignorance there must be 
different degrees of akusala. We should just develop understanding of whatever real 
ity appears. At the moment of developing understanding one is not carried away by 
thoughts about the amount of one's defilements, wondering about it how many defilements one has or whether they are decreasing. Just be aware instantly!''

Time and again  Achan reminded us of the development of understanding at any moment: ``Like now, there's seeing, hearing, thinking. That's all. Whatever arises by conditions just understand it. Usually it's the object of ignorance and attachment, but it 
can be the object of right understanding and detachment when there's more and more 
understanding. Just live by conditions. You cannot change it, you cannot make any 
thing arise at all, whatever is there. Best of all is understanding it not wanting more 
or less or this or that.'' 

Paññā has to be developed together with all the perfections so that it can eradicate the 
wrong view of self and all other defilements. Because of the accumulated ignorance 
we do not realize that we cling to the idea of self, that we have the idea that we see, 
we hear, we think. Paññā will see more and more how deeply engrained the clinging 
to self is. 

The following sutta illustrates how common clinging to self is. We read in the ``Kindred Sayings'' (I, Kosala, § 8, Mallika) that King Pasenadi of Kosala said to Queen 
Mallika: 

\begin{quote}

“Is there, Mallika, anyone more dear to you than yourself?'' 

Mallika answered: ``There is no one, great king, more dear to me than 
myself. But is there anyone, great king, more dear to you than your 
self?'' 

The King answered: ``For me too, Mallika, there is no one more dear 
than myself.'' 

We read that the Buddha recited the following verse: 

\end{quote}

\begin{verse}

“Having traversed all quarters with the mind,\\ 
One finds none anywhere dearer than oneself. \\
Likewise each person holds himself most dear;\\ 
Hence one who loves himself should not harm others.'' 
\end{verse}

The commentary to the ``Verses of Uplift''(Udana)\footnote{Translated by Ven. Bodhi.} , which has the same sutta (Ch V, 
Sona, I) explains that if one wants happiness for oneself, one should not harm, including even a mere ant or other insect. When one harms others one will experience 
the result of akusala kamma. This is the law of kamma. 







\chapter{Clear Comprehension} 


We may come to see that the Dhamma we hear and study is of the highest value in 
life. During this journey and also on many occasions we could listen to the Dhamma; 
we acquired a better understanding that the object of right understanding is any reality that appears at this moment. This is the only way to have less ignorance. 

We discussed ``clear comprehension'', sampajanna, which is classified by way of four 
aspects. This classification as we find it in the Commentary to the Satipaṭṭhāna 
Sutta\footnote{Translated by Soma Thera in the ``Way of Mindfulness''. } and in the ``Fruits of the Life of a Recluse'' (Dialogues of the Buddha, Chapter 
2), reminds us of the purpose of developing right understanding and of the suitable 
means in order to reach the goal, so that we have less delusion about the objects of 
right understanding. 

The four aspects of clear comprehension are: 

clear comprehension of purpose, satthaka sampajanna, 

clear comprehension of suitability, sappaya sampajanna, 

clear comprehension of resort, gocara sampajanna, 

clear comprehension of non-delusion, asammoha sampajanna. 

When we begin to read the text in the commentary about these four kinds of sampajanna we may believe that it all pertains to the life of a bhikkhu, to the ways he 
should behave and do what is suitable. He should not go to crowded places, that is 
not suitable for him. We read how he should walk, wear his robes, eat. All the time 
he should not be forgetful of the four ways of clear comprehension. 

Here we read about situations described by conventional terms. But the commentary 
also points to ultimate realities. Sometimes the truth is explained by way of conventional terms, sometimes by way of ultimate realities. We should remember that sīla is 
the behaviour of citta. Further on we read about the processes of citta, about details of 
nāma and rūpa. The translator speaks about the subject of meditation (kammatthana) 
and this is, according to the subcommentary: ``The subject of meditation of the elements (modes or processes) that is according to the method about to be stated with 
the words ‘Within there is no soul’ and so forth.'' 


Bending and stretching that the monk should do is explained by way of nāma and 
rūpa and this is included in clear comprehension of non-delusion. One may read the 
whole passage about the behaviour of the monk with wrong understanding, as  Achan 
reminded us. 

We should keep in mind that life is nāma and rūpa in the ultimate sense. 

The explanations about the way of behaviour of the bhikkhu point to realities, to citta, 
cetasika and rūpa. It is emphasized time and again that mere processes are going on, 
and that there is no self. 

Sampajanna is paññā that is able to understand what appears. There is clear comprehension of purpose if one sees that listening to the Dhamma and understanding the 
truth of the reality that is appearing is of the highest value in life. Sampajanna is understanding of what is of the highest value for those who are born a human being and 
have the opportunity to listen to the Dhamma. 

Clear comprehension of suitability is listening to true Dhamma. One should not think 
that a particular place or time is not suitable for awareness. There are seeing, thinking, attachment on account of what is seen at any place, at any time. Awareness and 
right understanding should be developed in a natural way. If we think that a certain 
place or situation is not suitable for awareness, one is thinking about oneself. When 
we are in a difficult situation, we may think of ``poor me, why me?'' We are bound to 
forget that this is thinking at that moment, no self who thinks. It is thinking that is 
preoccupied with the self. If one can be mindful and not forgetful, there can be right 
understanding of whatever appears and then falls away. 

One may truly understand that conditioned realities are impermanent, that nothing 
can stay. Each reality arises because of the appropriate conditions and then falls 
away. Clear comprehension of suitability, sappaya sampajanna, is the condition for 
knowing the truth of what is not permanent. When understanding of what is impermanent, conditioned, has been fully developed it leads to the attainment of what is 
unconditioned, nibbāna. 

We read in the ``Sappaya Sutta'' (Kindred Sayings IV, Third Fifty, §146, Helpful) that 
the Buddha said: 

\begin{quote}
``I will teach you, brethren, a way that is helpful for Nibbāna. Do you 
listen to it. And what, brethren, is that way? 

Herein, brethren, a brother regards the eye as impermanent. He regards objects, eye-consciousness, eye-contact, as impermanent. That 
weal or woe or neutral state experienced, which arises by eye-contact, 
that also he regards as impermanent.'' 
\end{quote}

The same is said with reference to the other sense-doors, the mind-door, the objects 
experienced through those doorways, the other sense-cognitions, contacts and feelings. 

The Buddha spoke about what appears now, at this moment; he spoke about what 
arises and falls away, what is impermanent, but this is not yet known. 

As to clear comprehension of resort, gocara sampajanna, gocara is any object that can 
be object of right understanding, also akusala. There should be no selection of the object of right understanding; understanding can be developed at this very moment. 

Gradually understanding of whatever object appears can develop, so that there will be 
asammoha sampajanna, clear comprehension of non-delusion. One will know that the 
truth is at this moment; it is the reality that arises and falls away, but the arising and 
falling away has not been realized yet. Understanding can become firmer, one begins 
to have right understanding of the true characteristics of realities. That is asammoha 
sampajanna. One is not deluded, one has not wrong understanding and clinging to a 
reality one believes to be permanent. 

The four aspects of sampajanna show us the conditions for the arising of clear comprehension. One will have more confidence that it is not ``I'' who can do anything. 

Achan said: ``Be patient enough to let dhamma condition dhamma, not `I' who tries 
so hard to cause the arising of understanding. The arising of paññā is very natural, as 
natural as ignorance. There cannot be many moments of understanding, only very, 
very few. It takes a long, long time to become detached from wanting to experience 
the truth.'' 

The Buddha explained about realities appearing through the senses and the mind- 
door, one at a time. We have heard his teaching about realities often, in many suttas, 
but we can hear his words again and again, they are deep in meaning. He taught about 
what is really appearing at this very moment: citta, cetasika and rūpa that arise and 
fall away. When seeing arises and appears, it experiences visible object through the 
eye-door, not a person or thing. The world experienced through the eye-door is completely different from the world experienced through the ear-door. When hearing 
arises, it experiences sound through the ear-door. Life exists only in one moment, the 
present moment. 

As Sarah reminded us many times: ``All problems in life come down to clinging now. 
Less clinging means more mettā to those around us, regardless of how we are 
treated.'' The following sutta gives us a good illustration of this fact. 

We read in the ``Kindred Sayings'' (IV, Second Fifty, Ch 4, § 88, Punna), that Punna 
asked the Buddha for a teaching in brief. The Buddha taught him about all the objects 
appearing through the six doorways. If he would cling to these objects, this would 
lead to suffering. If he would not cling there would be the end to suffering. 

The Buddha explained about realities appearing through the senses and the mind-door, one at a time. We have heard his teaching about realities often, in many suttas, 
but we should hear his words again and again, they are deep in meaning. He taught 
about what is really appearing at this very moment: citta, cetasika and rūpa that arise 
and fall away. When seeing arises and appears, it experiences visible object through 
the eyedoor, not a person or thing. When hearing arises, it experiences sound through 
the eardoor. Punna had right understanding of realities, he did not take any reality for 
self. The Buddha asked him where he would be dwelling. When Punna said that he 
would dwell in Sunaparanta, the Buddha said: 

\begin{quote}

``Hot headed, Punna, are the men of Sunaparanta. Fierce, Punna are 
the men of Sunaparanta. If the men of Sunaparanta abuse and revile 
you, Punna, how will it be with you?'' 
\end{quote}

Punna answered that they were kind not to smite him a blow with their hands. 

The Buddha then asked him how he would feel if they would throw clods of 
earth.\ldots beat him with a stick\ldots strike him with a sword or slay him. Punna gave in each 
case a similar answer, he had no aversion. As to being stabbed to death, Punna said 
that some disciples who are disgusted with body and life stabbed themselves, but he 
would have come by a stabbing that he never sought. He was not afraid of fierce people, because in reality there are no fierce persons who could cause one to suffer injuries. He did not have any ideas of revenge. Punna had understood that there are no 
people in reality, only conditioned dhammas. Punna clearly understood what is meant 
by ``living alone'' alone with what is experienced through the senses, one at a time. 
Life is only the experience of one reality at a time. 

Experiencing pain or even being killed has nothing to do with people who act. If one 
is convinced that there is no one there, one has no ill feelings about people. When 
bodily painful feeling arises, this is vipakacitta, citta that is the result of akusala 
kamma committed in the past. When one has not heard the Dhamma, one may think 
about one’s afflictions with akusala cittas and blame those who caused injuries to us. 
Punna had right understanding about cause and result and answered the Buddha with 
wise attention when the Buddha asked him what he would do if the people of 
Sunaparanta would afflict him. Even if they would stab him to death he had no fear. 
No one can control in which circumstances one will die and at which moment. Punna 
had no attachment, aversion or ignorance and hence he had endless loving kindness 
and compassion. 

We may believe that we notice different realities such as sound, odour or hearing. We 
may believe that we know the present reality. That is not understanding that knows 
realities one at time as only a dhamma. When paññā arises there is no need to think: it 
is just a dhamma. Very, very gradually understanding can become firmer and it realizes sound that appears as just a dhamma that cannot be changed, without having to 
think about it. What arises does so because of its own conditions. We do not mind if 
the present reality is envy, conceit or stinginess, they are all conditioned and they are 
gone immediately. 

Achan reminded us many times of the subtlety of the Dhamma: ``It is so very difficult to understand that there is not anyone at all. There is only that which impinges on 
the eyebase and can condition seeing to see it. We have to learn to understand realities one at a time so that it can be understood that there is no self. That which appears 
has arisen, just to be seen and then it exists no more, it will never arise again.'' 

I am most grateful for all good advice and reminders of the truth about realities of 
daily life, given by Achan and other friends during our journey. These are real treasures. 



