
\part{My Time with Acharn Sujin}
\chapter[First Meeting]{}
\section*{First Meeting}


Suzaki wrote:

``What I am curious first is to know your vivid, or perhaps inspiring
moment you had at the earlier years with Acharn Sujin. I read
some comment from the book `Buddhism in Daily Life.' But more specifically, how was your impression from the first meeting?”

\textit{Nina}: I met Acharn Sujin for the first time in the Wat Mahathaat temple where a foreign monk was teaching about the jhāna-factors (to be developed in tranquil meditation) and also helped us to read suttas. We read the ``Parinibbāna sutta'' and the ``Kesaputta sutta'' (mostly called Kalama sutta). I was impressed by the realization that you do not have to accept anything from others, but have to find out the truth for yourself. Acharn Sujin kept rather to the background in this temple. I approached her and said that I wanted to learn about meditation that you can apply in daily life. My life was very busy, being in the diplomatic service. (In Japan the teachers at the language school called me ``Mrs Party''.) I felt that there must be something else in life, apart from just being engaged with parties. Acharn Sujin said, ``Yes, vipassanā can be developed in daily life'', and she invited me to her house. From then on I came to her several times a week with many questions. I asked her about belief in God and how to find out the truth. She answered: ``What is truth will appear''. She also helped me to see what clinging is, clinging to a belief. I had never considered this before. She said from the beginning that in the teaching of Dhamma, the person who teaches is not important; it is not the person but it is the Dhamma that matters.

This was new also for Thais; in Asian countries there is a great respect for teachers and people tend to follow what teachers say. When teachers wrote about Dhamma in olden times they would not mention the source of their quotes. Acharn Sujin greatly contributed to a change in this mentality, always encouraging to looking up the texts oneself, verifying the truth for oneself. She started interest in the translations of Commentaries and promoted this. I remember our visits to the library of Wat Bovornives and our conversations with monks. A friend made notes and gradually Commentaries in Thai were printed.

Acharn Sujin gave lectures in a temple every Sunday and quoted suttas. She asked a monk in advance about the Commentary to the relevant text. I tried to look up the suttas in my English editions








\chapter[Abhidhamma in Daily Life]{}
\section*{Abhidhamma in Daily Life}


Suzaki wrote:

``How skilfully did she bring the technical matter, Abidhamma? If I may say
so into the living, daily practice? Any specific event that you can highlight?''

Nina: When I was at her house, she explained about nāma and rūpa, about kusala citta and akusala citta. She answered my questions and very soon made me work for an English radio program. The first chapters that you find in `Buddhism in Daily Life' are from notes with my conversations with Acharn Sujin. Every two weeks I had to finish a new chapter. It was a busy, but happy time. She helped me to see that all those different cittas (consciousness), cetasikas (mental factors arising with consciousness) and rūpas (physical phenomena) occur in daily life. I learnt that whatever occurs is conditioned; that good and bad inclinations are accumulated from moment to moment and that these condition our behaviour. Everything I learnt was relevant to daily life.

An example: we visited a bhikkhu who smiled when I told him about my interest in the teachings. Acharn Sujin asked me whether I knew why he smiled. She explained that this was because of happy feeling, somanassa. This sounds very simple, but it made me realize that feeling conditions our outward appearance.

She reminded me of how conditions affect our daily life under various circumstances. We were waiting near a kuti, a bhikkhu’s dwelling, for a certain monk. He was not there and I suggested that we would find out about him. She said, ``Let us sit at this stone and just wait and see what happens because of conditions''. We sat quietly for quite some time. What a good lesson! I am so grateful for all those reminders I received in the situation of my daily life. It is true: we think of people we want to meet, but in fact, there are only different experiences, such as seeing, hearing and thinking, and they are all conditioned.

I was used to taking notice only of the outward appearance of people, but now I learnt about different cittas which condition our behaviour. People may look very pleasant and peaceful, but what do we know about their cittas which change from moment to moment?''

When crossing a street she said: ``Elements on elements'', and it is so true: hardness appears, and it is only an element. We think of feet and street, but let us consider what can be directly experienced.
However, it took many years before all these lessons were absorbed, and I needed later on during different journeys many explanations about the difference between thinking and awareness, before I understood a little more. (Later on I come back to this).

Acharn Sujin used to go in retreat in a center but one day she realized that actually daily realities are the objects of vipassanā. From then on she did not go anymore in retreat, and this happened not so long before I met her. Since most people were not used to this approach, they had many questions about vipassanā in daily life. I found this approach the only reasonable one and did not doubt about its value. We have to know our own accumulations, our inclinations we take for self. They appear, and thus, they can be objects of insight.

Acharn Sujin always stressed that there is no rule about how one should develop understanding and that one cannot direct what object appears at a particular moment. I find this most reasonable, because whatever is experienced by citta is conditioned.

We went to different temples, also in the province. People asked questions about vipassanā and concentration. Although I was just learning Thai, Acharn Sujin made me talk as well. I enjoyed simple life in the province, without any fringes. People treated me as one of them, and that is what made me happy.

People asked whether slowing down one’s movements would be helpful for the development of vipassanā. Acharn Sujin asked one person to run and to find out whether there is any difference as to what realities are appearing. The conclusion was: it is all the same. True, seeing is always seeing, no matter we run or sit. Seeing is a citta, an ultimate reality that should be known as it is, non-self. I heard a dog barking and asked whether hearing a dog is an object of insight. She explained that hearing just sound is different from thinking of a dog. I listened, but only many years later I understood the point.

People also asked: ``Is this kusala (wholesome), is that akusala (unwholesome)”. Her answer was: ``You can only know for yourself. Nobody else can tell you''. She also explained that it would be very easy if someone else were to tell you: ``Do first this, then that, and you will make progress''. Her advice always was : ``There are no rules, there is no specific order of the objects insight can be developed of''. In the whole of the Tipiṭaka we learn about realities that arise because of conditions and are non-self. Now, also in the practice we have to be consistent, how can we force ourselves to be aware of specific objects. She kept on warning us of subtle clinging to progress, to result. Expectations are lobha, attachment. She repeated many times: ``Don't expect anything''. We should not expect anything from ourselves nor from others. Expectations bring sorrow. I am grateful for her example in this matter, and her example of patience and equanimity. Some people heavily criticized her, but she was always patient and calmly explained about cause and effect: what cause will bring what effect. We should be clear about this. Do we want only calm or is our aim understanding?
















\chapter[The glimpse of Dhamma]{}
\section*{The glimpse of Dhamma}

Suzaki wrote:

``You said before: ‘I became used to the different types of citta, consciousness.’ What were the few specific incidents in your early days that made you find the glimpse of dhamma?''

Nina: At breakfast I listened to Acharn Sujin’s radio program and heard time and again the terms denoting the different cittas arising in sense-door processes and mind-door processes. Thai and Pāli are very close, and in this way I could learn all these terms. But becoming used to these terms does not mean experiencing all the different cittas. Acharn Sujin explained that intellectual understanding is a foundation for awareness and right understanding that can arise later on. She stressed the importance of foundation knowledge: knowledge of the details of cittas, of their different characteristics, of cetasikas (mental factors), such as feeling, akusala cetasikas, beautiful cetasikas and rūpas. Indeed, as we read in the suttas, listening, considering are most important conditions for the arising of satipaṭṭhāna, sati (awareness) and paññā (understanding) that directly realize characteristics of nāma and rūpa.

We begin to recognize attachment, lobha, and aversion, dosa, in our lives, and this is useful, but we should not take this for awareness. For many years I thought that thinking was awareness. We may think without words, recognize realities very quickly, but, when we are very sincere, there is still an idea of self who does so. It is not paññā of satipaṭṭhāna.

I began to know that laughing is conditioned by lobha, and this made me feel somewhat uneasy when laughing. I had an idea of wanting to suppress laughing. Lobha again! Acharn Sujin explained that we should behave very naturally, and not force ourselves not to laugh: ``Just do everything that you are used to doing, but in between right understanding can be developed''. ``We have to know our good moments and our worst moments in a day'', she said. I read a sutta where the Buddha spoke to the monks about women and compared a woman to a snake. I did not like that. Acharn Sujin answered that this sutta can remind us of our accumulated defilements. If right understanding is not developed, accumulated defilements can cause the arising of many kinds of aksuala, and then we are like a snake. In other words, we should profit from the message contained in a sutta, learning how dangerous akusala is. Moreover, by this sutta the Buddha warned the monks of the danger of getting involved with women.

Acharn Sujin helped me to see how accumulations in past lives can lead to harm. We never know how these accumulations can condition cittas at the present moment. We may do things we did not believe ourselves capable of.

When I listened to her lectures in the temple I became sometimes depressed when I realized how difficult the development of right understanding is. Would I ever be able to reach the goal? But I had no inclinations to look for another way that could hasten the development of right understanding. Acharn Sujin explained that clinging to progress will not help us at all. When we realize more that it takes aeons to develop right understanding we shall be less inclined to think in terms of progress. Before this life there were aeons of ignorance, and in this life we are fortunate to be able to listen to the teachings and begin to understand the way of development of the eightfold Path. But it has to be a long way before we reach the goal. We can learn to accept that this will take more than one life.

Time and again Acharn Sujin repeated what the Buddha said in the ``Exhortation to the Pāṭimokkha”: ``Patience is the greatest ascetism.''




\chapter[Practising the `process']{}
\section*{Practising the `process'}



Suzaki wrote:

``So, practising the `process' (may I also say, sīla-samādhi-paññā?) will lead to elimination of suffering.”

Nina: Acharn Sujin taught me what is kusala and what is akusala by her example. Observing the precepts is not a matter of rules one has to follow. She explained that there is no self who can direct the arising of kusala, that it is sati which conditions refraining from akusala and performing kusala. Since I was in the diplomatic service I went to cocktail parties and took drinks. Acharn Sujin would never say, ``Don’t drink''. She would explain that it is sati that makes one refrain from akusala. Gradually I had less inclinations to drinking. I did not know that killing snakes or insects was akusala. When I was in Acharn Sujin’s house, we were having sweets, and when flies were eating some crumbs on the floor, Acharn Sujin said, ``We let them enjoy these too”. I had never considered before to give flies something they would enjoy, it was a new idea to me. I learnt more in detail what was kusala, what akusala. I began to refrain from killing insects and snakes. She also taught me that it is kusala sīla to pay respect to monks, because the monks observe so many rules. She taught me to kneel down and pay respect in the proper way. She taught me the importance of the Vinaya, and she explained that we laypeople should help the monks by our conduct to observe the Vinaya. We should not give money to them, but hand it to the layperson in charge. When we are in conversation with the monks we should not chat on matters not related to Dhamma. Together with her elderly father we visited temples and offered food. We often had lunch with her father in his favoured restaurant where they served finely sliced pork (mu han in Thai). We did not talk about Dhamma very much at such occasions, but I noticed Acharn Sujin’s feeling of urgency, never being forgetful of the Dhamma, whatever she was doing. I was clinging very much to Dhamma talks, but throughout the years I learnt that we do not need to talk about Dhamma all the time, but that we should reflect on Dhamma and apply Dhamma in our life. Acharn Sujin is always such an inspiring example of the application of Dhamma.

When we read the ``Visuddhimagga'' we see the three divisions of sīla (wholesome conduct or virtue), concentration and paññā. We may think of a specific order. However, Acharn Sujin explained that this is the order of teaching, that there is not a specific order according to which we should practise. When we carefully read about sīla, we see that all degrees of sīla are dealt with, from the lower degrees up to the highest degrees: the eradication of all defilements.

Having kindness for flies and abstaining from killing is sīla. Being respectful to monks is sīla. Being patient in all situations is sīla. Satipaṭṭhāna is sīla. We read in the Gradual Sayings (Book of the Threes, Ch II, § 16, The Sure Course) that a monk who possesses three qualities is ``proficient in the practice leading to the Sure Course'' and ``has strong grounds for the destruction of the āsavas''. These three qualities are moderation in eating, the guarding of the six doors and vigilance. We read concerning the guarding of the six doors:

\begin{quote}
And how does he keep watch over the door of his sense faculties?

Herein, a monk, seeing an object with the eye, does not grasp at the general features or at the details thereof. Since coveting and dejection, evil, unprofitable states might overwhelm one who dwells with the faculty of the eye uncontrolled, he applies himself to such control, sets a guard over the faculty of the eye, attains control thereof\ldots
\end{quote}

The same is said about the other doorways. The six doorways should be guarded. How does one, when seeing an object with the eye, not ``grasp at the general features or at the details thereof''? In being mindful of the reality which appears. It is satipaṭṭhāna which is the condition for abstaining from akusala.

As to concentration or calm, there are many degrees of concentration. Each kusala citta is accompanied by calm. Calm is not a feeling of calm, it means the absence of akusala. When we cling to silence and to being calm, there is lobha, not calm. Paññā has to be very keen to know exactly which moment is akusala and which moment of kusala, otherwise we shall not know the characteristic of calm. When there is awareness of nāma or rūpa there is also calm at that moment. As paññā grows, calm grows as well. The eradication of defilements is the highest degree of calm. Acharn Sujin often stressed: when there is right awareness of a nāma or rūpa there is at that moment higher sīla, higher concentration and higher paññā.



Acharn Sujin helped me to see what is akusala and what is kusala in the different circumstances of daily life. She often said, the teachings are ``not in the book'', they are directed to the practice of everyday life. Also the Abhidhamma is not technical, it helps us to have a more refined and detailed knowledge of different cittas as they occur at this moment. When I said that I had enjoyed reading a beautiful sutta, she answered, ``It is so sad when we only think of what is in the book, when we do not apply it.'' I realized that we may cling to what we read instead of seeing it as a reminder to develop understanding.

Acharn Sujin introduced me to her friends at her house, where they consulted books of the Tipiṭaka and discussed points of the Dhamma. She explained to me, ``All we study and discuss is not just for ourselves, it is to be shared with others.'' This impressed me very much because I knew very little about sharing kusala with others. It had not occurred to me that even studying the teachings is not just for oneself. She would always help me to have more kusala cittas. When we were in a temple and we had things to offer to the monks she would hand the gifts and books to me, asking me to present them. I was glad to have the opportunity to pay respect to the Triple Gem and show my reverence to the monks. In fact she was helping others all the time to have kusala cittas. We visited Khun Kesinee who wanted to print my book ``Buddhism in Daily Life''. Khun Kesinee said, ``Khun Sujin has given me life''. This was so true, because she taught us all a new outlook on life, she taught us how right understanding can be developed in our ordinary daily life. She taught us to develop understanding of all phenomena of life in a natural way. Her daughter Khun Amara wrote ``The Lives and Psalms of the Buddha’s Disciples'', inspired by the Thera-therigatha''. These are the stories of men and women in the Buddha’s time who proved in their daily lives that the Path can be developed and enlightenment be attained.
Acharn Sujin and I were very busy to correct the printing proofs of my book, sometimes at night. When we had not heard anything from the printer and I wondered about this, she just answered, ``No news.'' This was a good lesson to leave things to conditions and not to expect anything. Later on I thought many times of these words. It is clinging when we expect things to be the way we like them to be.

I was glad to meet many of her friends and take part in their life of giving and sharing. We went to temples together with Acharn Sujin, presenting dāna, or attending cremation ceremonies. On Sunday, I drove Acharn Sujin to the temple where she gave lectures on satipaṭṭhāna and afterwards we sat outside the temple where people asked her more questions about awareness in daily life. Her lectures were put on tape for a radio program. In the course of years the radio stations which sent out her program expanded all over Thailand and to neighbouring countries.

I accompanied Acharn Sujin to different places where people had invited her for a lecture. People were wondering whether there can be awareness of nāma and rūpa while driving a car. The answer was that it is just the same as being at home, it is normal life. Seeing, thinking or hardness appear time and again. When walking on the street we discussed seeing and thinking of concepts. There were holes in the pavement and if one would only be aware of colour and seeing but not think, one would fall into the holes. We learn that in the ultimate sense there are only nāma and rūpa, that there are no people, no things. This does not mean that we should not think of people and things. Also thinking of concepts is part of our daily life, we could not function without thinking of concepts. Thinking is a conditioned reality, it is nāma, not self. We can think with different types of citta, some are kusala and many are akusala. In the development of satipaṭṭhāna, we come to know our daily life just as it is.

\chapter[`Formal’ meditation]{}
\section*{`Formal’ meditation}



Suzaki  wrote: 

``From just skimming to read ‘Buddhism in Daily Life’, it appears that you do not put high importance to `formal’ meditation. Was this the case in your beginning of the Path? Did
you start to do `formal’ meditation later? If so, how and how effective was it? Or, are you suggesting that it depends on people?”

Nina: I left Thailand after almost five years, but there were opportunities to return many times and take part in pilgrimages to India and Sri Lanka together with Acharn Sujin. She taught at the Thai language school to foreigners and several of them took an interest in the teachings. Among them were the late Bhikkhu Dhammadharo and Jonothan Abbot. Later on I also met Sarah who visited me from England. I found discussions on the Dhamma very useful since these helped me to clear up misunderstandings about nāma and rūpa. I had correspondance with people all over the world and this also helped me to clarify for myself the meaning of satipaṭṭhāna in daily life.

People are always wondering how to act in order to have more understanding. Acharn Sujin would stress that we should not think of ourselves, and that we become less selfish by paying more attention to the needs of others. This is a simple advice, but it is very basic. We cling to ourselves all the time, but the aim is detachment from the idea of self. If we are always selfish, how can we become detached? On all the India trips she would speak about the perfections which should be developed together with satipaṭṭhāna. Generosity, loving-kindness (mettā) and patience are essential qualities that should be developed, they are conditions for thinking less of ourselves. I learnt a great deal from my Thai friends on these trips. I noticed how alert they were to help others, even with small gestures. When we are sitting with others at the table for a meal, we can notice whether we take hold of dishes or reach for food only with the idea of wanting things for ourselves, or whether we are also attentive to the needs of others. I began to understand that there are countless moments of thinking of ourselves. I learnt in the situation of daily life that when kusala citta arises, there is a short moment of detachment. However, very shortly after kusala citta we are likely to cling to an idea of ``my kusala”. Generosity is only a perfection if we do not expect anything for ourselves, if it leads to less clinging. The aim of the development of perfections is detachment, eradication of defilements.

Acharn Sujin would often remind us of the need to apply the Dhamma in our daily life, reminding us how circumstances change from moment to moment. Each moment is actually a new situation. Each moment is conditioned. Whatever we experience through the senses, be it pleasant or unpleasant is conditioned by kamma. Once during a pilgrimage we stayed in a Thai Temple where different rooms were assigned to our group. I received the worst room, without bathroom and full of moquitos. I could hardly sleep and the next day I complained about this. I was used to having Vip treatment in the diplomatic service but Acharn Sujin helped me to see that unpleasant experiences are conditioned. Nāma is nāma and rūpa is rūpa, and it is not important what status of life people have. She asked me whether I was not glad afterwards to have those experiences. I agreed because now I found such experiences a good lesson. She helped us to understand kamma and vipāka in different situations of our life.

Some of her listeners thought that they should look for other circumstances, different from the present one, in order to have more conditions for sati. To them Acharn Sujin explained that seeing here is the same as seeing in another place, hearing here is the same as hearing in another place. Seeing is always seeing and hearing is always hearing, they are ultimate realities with their unalterable characteristics. We learnt that the Abhidhamma is not theory, that it can be directly applied, and this is satipaṭṭhāna. She would often remind us, ``And how about this moment now?'' Whatever questions people asked, she would always guide them to the present moment.

Bhikkhu Dhammadharo said that he was sometimes lost for a long time, without sati. Acharn Sujin asnwered that this shows that one has to develop right understanding in daily life, that one has to understand one’s natural life. Then one can see the conditions for different nāmas and rūpas, conditions one has accumulated. One can check for oneself whether there is clinging to nāma and rūpa.

We need the Vinaya, the Suttanta and the Abhidhamma to support the development of right understanding. We should listen, study and consider the Dhamma. Paññā cannot suddenly arise. When we have intellectual understanding we can compare this with a plant that has to grow. We see at first buds, and we do not know yet when it will bloom. This will happen when the conditions are right.

\chapter[Understanding of mind-matter]{}
\section*{Understanding of mind-matter}

Suzaki wrote: 

`` The aim is understanding of mind-matter relationship, by dissecting or rather becoming aware of specific happenings that we experience in our daily life (that we were unaware of before). Such insight will enable us to become aware of what is going on in terms of cause and effect relationship, to see the cause of suffering, etc.''

Nina: During a pilgrimage in India with Acharn Sujin, Bhikkhu Dhammadharo, Jonothan and other friends we discussed Dhamma all night in the train to Bodhgaya. During that night we discussed the difference between thinking of nāma and rūpa and direct awareness of them. We may notice that realities appear through different doorways, that sound is experienced through ears and hardness is experienced through the bodysense. However, we may take noticing realities for direct awareness of them. Acharn Sujin said, ``You may believe, ‘I have developed a great deal of understanding, I sees that there is nothing else but nāma and rūpa.' '' She then explained that in reality this is only thinking, not direct understanding of one nāma or rūpa at a time. Hearing is nāma, it experiences sound. Sound is rūpa, it does not experience anything. When hearing arises we think almost immediately of the meaning of the sound, its origin, of words which were spoken and the meaning of those words. Thinking is another type of nāma, different from hearing. Her remarks were an eye-opener to me. This shows again how important discussions on the Dhamma are. Without them our misunderstandings of the Dhamma would not appear. That night in the train passed very quickly, and before we realized it we were in Bodhgaya. One of our friends offered breakfast to Bhikkhu Dhammadharo and to the Samanera (novice) who was also present.
We also stayed in Varānasī, in Hotel de Paris. When we were walking in the garden of that hotel, we heard a band with drums, and immediately we had an image of people marching and playing. Acharn Sujin explained that we build up stories on account of what we experience through the senses. Sound, hearing and thinking are ultimate realities, the stories we think of are concepts or ideas, different from ultimate realities. It is difficult to distinguish between different realities; it is direct understanding, paññā, that is able to do so. If we try to separate nāma from rūpa or if we try to think of both nāma and rūpa, there is only thinking, no awareness of either of them. Paññā cannot suddenly arise, it is gradually developed by studying, considering what we learn, discussing, asking questions.

We may be thinking of ourselves and others, walking in the garden of Hotel de Paris, but if we die now, the story comes to an end. Actually, each citta that falls away is a moment of dying. With the citta that falls away, the story comes to an end. Many years later Lodewijk and I walked to Hotel de Paris again, and then we saw that it had become neglected and that nothing of it’s old glory was left.

One may believe that knowing what is going on is right awareness. Someone may know that he sees or that he hears, but that is not satipaṭṭhāna. When right awareness arises, it is mindful of the characteristics of nāma and rūpa as they appear one at a time. Right mindfulness and right understanding arise when there are conditions for their arising.
Throughout all these years with Acharn Sujin we discussed again and again what seeing is: the experience of what appears through eyesense. We discussed what hearing is: the experience of what appears through the earsense. We are always forgetful of seeing and hearing, because we are more interested in concepts such as people, things and events. We can never be reminded enough of nāma and rūpa, because these are ultimate realities paññā has to understand. Right understanding of nāma and rūpa leads to detachment from the idea of self.

We were reminded that awareness is not self, it cannot be induced. Acharn Sujin asked us: ``Who is aware?” When we answered, ``Awareness is aware'', she said, ``That is in the book, but in your mind?'' Such remarks made us realize how much we are still clinging to the idea of ``my awareness''.



\chapter[Elimination of Suffering]{}
\section*{Elimination of suffering}

Suzaki wrote: 

``Such insight will enable us to become aware of what is going on in terms of cause and effect relationship to see the cause of suffering, etc. Such cause and effect relationship lead to the experiential understanding of the four noble truths. So, practising the `process' (may I also say, sila-samadhi-panna?) will lead to elimination of suffering.

Nina: My husband and I took part of many excursions with Acharn Sujin and other friends whenever we visited Thailand again. We went to nature reserves in the north of Thailand, to Nakom Phanom and other places in the provinces. For our Dhamma discussions Acharn Sujin always tries to arrange for pleasant surroundings and a relaxed atmosphere. With the help of her sister Khun Jid and our friend Khun Duangduen she sees to it that we have delicious and well-balanced meals. There is no end to their hospitality. The right climate and suitable food can be favourable conditions for the citta that develops right understanding. During our visits to Thailand and during our pilgrimages to India we discussed Dhamma and whenever we talked about personal problems in daily life, she would give us the most practical advice. This helped us to see our problems in the light of the Dhamma. When we discussed deep subjects of the Dhamma such as the Dependent origination and the four noble Truths, she would always relate these to our daily life.

We read in the Tipiṭaka about the four noble Truths: dukkha, the cause of dukkha which is craving, the cessation of dukkha which is nibbāna and the way leading to the cessation of dukkha, which is the eightfold Path.

Acharn Sujin stressed that we should not have merely theoretical understanding of the four noble Truths. Dukkha and the cause of dukkha pertain to our life at this moment. The way leading to the cessation is the development of right understanding of the realities appearing at this moment. When insight has been developed stage by stage nibbāna can be attained.

We read in the "Kindred Sayings''(V, 420, Dhamma-Cakkappavattana vagga, §1), that the Buddha said, ``in short, the five khandhas are dukkha”. The five khandhas are actually all conditioned realities, nāma and rūpa of our daily life. When the arising and falling away of nāma and rūpa, thus their impermanence, is realized, dukkha can be understood. What falls away immediately is not worth clinging to, it is dukkha.

We have to develop insight stage by stage. We have to develop understanding of hardness when it appears through the bodysense during all our activities in daily life. We do not have to think, this is hard, and we do not have to think of the place where it touches; its characteristic can be known when it appears. Gradually we can learn that the characteristic of nāma is different from the characteristic of rūpa. When we take nāma and rūpa as a whole, the arising and falling away of nāma and rūpa as they appear one at a time cannot be realized. They can not be realized as dukkha and we shall continue to take them for a person or a thing that exists.

Craving, the cause of dukkha, arises time and again and it causes us to continue in the cycle of birth and death. Acharn Sujin reminded us to be aware of clinging at this moment. We should know when there is clinging to awarseness, to having a great deal of understanding. If we do not realize such moments we do not follow the right Path. Intellectual understanding of the fact that each reality arises because of its own conditions can help us to follow the right Path, and then we shall not be inclined to try to select particular realities as objects of mindfulness and try to make mindfulness arise. It arises because of its own conditions. She said, ``Awareness is like an atom in a day'', meaning that there are not many moments. How could this be otherwise; we have accumulated such a great deal of ignorance.

We are in the cycle of birth and death, and during this cycle, cittas arise and fall away, succeeding one another. Each citta that falls away conditions the arising of the following citta, and in this way all wholesome and unwholesome qualities of the past have been accumulated from moment to moment. Even so all wholesome and unwholesome qualities that arise at the present time are accumulated and they will condition our life in the future. When ignorance arises today, it does so because it is conditioned by past moments of ignorance, even during aeons. When understanding arises today, it does so because it is conditioned by past moments of understanding. Even if there is a short moment of right understanding now, it is not lost, it is accumulated and thus there are conditions for its arising later on. Acharn Sujin said that this is like saving a penny a day, which can become a big fortune.

During all our journeys and visits to Thailand she stressed that the four noble Truths are realized in different phases. First there should be firm understanding of what the object of right understanding is and how right understanding should be developed. This is the first phase (sacca ñāṇa, understanding of the truth). When understanding of the truth, the first phase, is firmly established, one will not deviate from the right Path, that is, right awareness and precise understanding of the characteristic of the reality that appears. The first phase is the foundation of the practice, which is the second phase (kicca ñāṇa, understanding of the task). This again is the foundation of the realization of the truth (kata ñāṇa, understanding of what has been done).

I remember that we were walking in India with one of the Thai monks and that Acharn Sujin was repeatedly stressing these three phases. Hearing the Dhamma again and again helps us to remember what was explained and to reflect on it. When we read about the four noble Truths we may not realize that they can only be understood and applied in different phases and that we can begin right now. Acharn Sujin would always remind us that there is seeing at this moment. We do not have to be in a quiet place to understand seeing; there is seeing no matter where we are. Seeing can gradually be known as a reality that experiences only what appears through the eyes, visible object. This is the beginning of the first phase of understanding the four noble truths.

The Buddha taught the development of understanding of our life at this very moment. The Abhidhamma is not technical, not theoretical, it teaches about citta, cetasika and rūpa, realities arising all the time. I am most grateful to Acharn Sujin for pointing out to us time and again that we should understand our life at this very moment. What she explained is completely in conformity with the Buddha’s teachings.
Nina.
