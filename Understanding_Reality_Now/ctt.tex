\part{Considering the Truth}
\chapter{Preface}

Acharn Sujin and friends were invited again for Dhamma discussions in
Vietnam by Tam Bach and other Vietnamese friends in February 2017.
Friends from Thailand, Australia, Taiwan and myself joined this journey.
Sarah and Jonothan were assisting Acharn and they gave us all many
helpful reminders and explanations about the reality appearing at the
present moment. In Vietnam Tran Thai made the travel and accommodation
arrangements for us.

Before the sessions in Vietnam, there were a few days of Dhamma
discussions in Thailand, in Nakorn Nayok, a place outside Bangkok. Here
we could meet old friends from Thailand, U.S.A. and Italy. Vincent from
Taiwan joined us with his wife Jane and their eleven-year-old daughter
Nana. We discussed details of cetasikas, mental factors that accompany
citta, such as contact, feeling and concentration. Knowing more details
helps us to see that citta and cetasikas that arise together condition
one another. Nobody can cause the arising of particular realities, they
are all dhammas devoid of self. We also discussed the factors leading to
enlightenment. Without the proper conditions, enlightenment cannot be
attained.

In Vietnam the sessions took place first in Saigon where we met many
friends who attended sessions at former occasions. The listeners had
numerous questions which were profitable for all of us. After Saigon,
the sessions were held at a very special place in Hoi An, namely at the
newly constructed resort An Villa of Tran Thai and Tiny Tam. It is a
``home-stay'' and it has the friendly atmosphere as we find it in a
family home, a Dhamma home, and here we spent delightful hours of
discussions.

Acharn reminded us many times that only listening and reading is not
enough. We should deeply consider the truth of Dhamma. In this way
understanding can grow. The development of understanding should be very
natural. We should not imagine that we have to go to a special tranquil
place or engage in special actions in order to have more understanding.
Understanding can be developed in the midst of akusala. It does not
matter what types of realities arise, they are all just conditioned
dhammas.




\chapter[The reality of this moment]{}
\section*{The reality of this moment}

In our daily life we are quite occupied with all the chores we have to
do, with our work, with our social life. We meet people and talk to them
and we are absorbed in the stories they tell us. We find our thoughts,
speech and actions very important and we take them for self. But the
Buddha taught that there is no person, no ``I'' or ``you'', only
ever-changing realities. If we truly understand this, we shall know the
meaning of ``living alone'', without the idea of a person or self.

Before we heard the Buddha's teaching of non-self or anattā, we found it
quite normal to believe: ``I see, I hear, I think''. Seeing, hearing and
thinking arise, but there is no person who sees, hears or thinks. They
are realities arising because of their appropriate conditions. They are
present for an extremely brief moment and then they fall away, never to
return. How could they be a self or how could they belong to a self?

We read in the
Visuddhimagga\footnote{This is an Encyclopedia
composed by the commentator Buddhaghosa who lived in the fifth century
A.D.}  (XV, 15 in
the section on the āyatanas, sense-fields) about conditioned realities:

\begin{quote}
``\ldots{}they do not come from anywhere previous to their arising, nor
do they go anywhere after their falling away. On the contrary, before
their arising they had no individual essence, and after their falling
away their individual essences are completely dissolved. And they occur
without power (being exercisable over them) since they exist in
dependence on conditions\ldots{}

Likewise they should be regarded as incurious and uninterested. For it
does not occur to the eye and visible object, etc., `Ah, that
consciousness might arise from our concurrence'. And as door, physical
basis, and object, they have no curiosity about, or interest in,
arousing consciousness.

On the contrary, it is the absolute rule that eye-consciousness, etc.,
come into being with the union of eye with visible object, and so on. So
they should be regarded as incurious and uninterested\ldots{}''
\end{quote}

As Acharn often reminded us: ``Before seeing there is nothing, and after
seeing there is nothing''. Seeing does not come from anywhere, it arises
for an extremely brief moment when there are the right conditions and
then it falls away, never to return.

When we stayed in Nakorn Nayok, before going to Vietnam, we had a
valuable experience, listening to the discussion of Vincent and Jane's
daughter Nana with Acharn, Sarah and others. We can really learn from an
eleven-year-old child. Nana was very happy to have bicycle rides with
Sarah and Jonothan in the early morning. Nana began to take an interest
in the Dhamma and when questions were posed to her, she deeply
considered these before she answered.

Nana asked Acharn: ``What is most important in life?''

Acharn answered: ``Right understanding of whatever appears. Do you
understand what is there now? Seeing, hearing? You are without
understanding, but you think that you understand. Seeing sees, is it you
that sees? Can you see your father?''

Nana answered that this is just thinking. Nana said that she did not
like it when her dad was angry with her. She came to understand that her
daddy was not real and that there was just thinking about him being
angry. Sarah asked her what the problem was: her dad being angry or
thinking and feeling unhappy. Nana answered: ``Thinking and feeling
unhappy''.

Sarah asked: ``Is it good or not good when you are angry?''

Nana said: ``Not good.''

Sarah asked her whether she could prevent being angry and Nana
understood that this is not possible.

These are all basic questions about the real cause of problems we have
in life and in our dealings with other people. We tend to forget that in
reality our own defilements are the cause of all trouble and not other
people or the outward circumstances. We may have intellectual
understanding of the truth but when we face contrariwise situations or
when we have disagreements with people, we tend to forget the truth of
realities.

Nana had truly accumulated an interest in the Dhamma and she considered
the truth. Sarah explained that some realities are good and some are not
good. Nana understood that anger is not good and kindness and helping
others is good.

Hearing about basic principles of the Buddha's teachings such as in the
discussions with Nana is valuable to all of us. We are forgetful of the
truth that a person does not exist. We are taken away by the stories
about people and then there are ignorance and attachment time and again.
We cannot be reminded enough of the truth. Acharn said: ``We have to
begin again and again.''

When someone said that it is so difficult to realize that there is no
one, nothing, Acharn answered:

``There is not direct understanding yet. That is why we talk about
seeing, so there will be a moment of direct understanding. Even touching
now is real, but there is no understanding; no matter how many times
there is touching. But when there is understanding, it is intellectual
understanding until the moment there is touching with right
understanding. We talk a lot about realities but there is not yet a
moment of understanding of what is real as not self.''

I said to Acharn: ``It is so helpful that you explain all the time about
this moment. This moment of touching or this moment of thinking. The
present moment. Until there is direct understanding of that by
conditions.''

This understanding does not arise by anyone's will. Otherwise it is
wrong, because there is the idea of ``I''. The idea of ``I am doing
something, I am concentrating on an object, I am being aware.'' If this
is not clearly understood there is no way that the idea of self can be
eradicated. We forget that it is not ``I'' who studies Dhamma, who is
listening to the Dhamma, who learns about realities.

Sarah asked Nana: ``Why is there the idea of my mommy?''

Nana: ``Thinking.''

Sarah: ``Why is there the idea of my mommy, not my daddy?''

Nana: ``Thinking.''

Sarah: ``Thinking and memory.''

Remembrance (saññā) is a mental factor (cetasika) arising with every
citta. There could not be thinking of a person or a thing without
remembrance assisting the citta that thinks.

We have to listen more to the right friends who explain the teachings
and we have to carefully consider what we hear again and again, so that
there will be intellectual understanding (pariyatti) of what appears at
the present moment. Intellectual understanding can condition direct
understanding of the truth (paṭipatti). There should be no expectations
as to when direct understanding will arise, it depends on conditions,
nobody can cause its arising.

When people visited the Buddha he would ask them whether seeing is
permanent or impermanent. We read in the ``Kindred Sayings on Sense''
(IV, Ch III, § 82, the World) that a monk said to the Buddha:

\begin{quote}
`` `The world! The world!' is the saying, lord. How far, lord, does this
saying go?''

The Buddha answered:

``It crumbles away, monks. Therefore it is called `the world'. What
crumbles away? The eye\ldots{} objects\ldots{}
eye-consciousness\ldots{}''
\end{quote}

It seems that seeing can stay, but it arises because of eyesense and
visible object, just for a moment and then it falls away. It is the same
with hearing and the other sense-cognitions. We tend to cling to an idea
of the whole body that exists, that is sitting, standing, walking or
lying down. When we take the body as a whole it seems to last and we
believe that we can manipulate it. There are only elements arising
because of their own conditions, there isn't anybody who can cause their
arising. They appear one at a time through one doorway at a time. When
we look at the body, only visible object appears, when we touch it only
tangible object appears such as hardness or softness, heat or cold.
These are realities, not concepts of a whole. There is no body that is
sitting. Realities can be directly experienced when they appear one at a
time, without having to think about them, whereas concepts cannot be
directly experienced, they are just objects of thinking. When
understanding of realities is being developed, the whole world crumbles
away, including the body, and the person. There is nothing left.

Some cittas are result, some are cause. Rebirth-consciousness
(paṭi\-sandhi-citta) is the first citta arising in life and it is the
result of a good or a bad deed (kusala kamma or akusala kamma) committed
in the past. Birth as a human is the result of kusala kamma. Birth as an
animal or birth in a hell plane is the result of akusala kamma. Seeing,
hearing and the other sense-cognitions arising in the course of life are
also cittas that are result (vipākacittas), they experience pleasant
objects or unpleasant objects. On account of what is experienced,
akusala cittas or kusala cittas arise that experience the object in an
unwholesome way or in wholesome way. These are cittas that are cause,
they can motivate unwholesome or wholesome deeds that will bring their
results accordingly.

Only one citta arises at a time and every citta is accompanied by
several mental factors, cetasikas, that each perform their own function
while they experience the same object as the citta. Some cetasikas are
beautiful (sobhana), some are akusala and some are neither.

Attachment (lobha), aversion (dosa) and ignorance (moha) are akusala
cetasikas that have been accumulated during countless lives and, thus,
they can condition the arising again of attachment, aversion and
ignorance. Each kusala citta is accompanied by the beautiful cetasikas
of non-attachment (alobha) and non-aversion (adosa), and it may be
accompanied by understanding, paññā, as well. Cetasikas are a condition
for citta and citta conditions cetasikas. The understanding of
conditions is essential for understanding the nature of anattā. Nobody
can cause the arising of any reality, its arising depends entirely on
conditions.

Citta and cetasika are realities that are mental, in Pali: nāma. They
experience an object. They are different from physical realities, in
Pali: rūpa. There are twenty-eight kinds of rūpa, but only seven kinds
appear all the time in daily life. They are: visible object, sound,
odour, flavour, and the three tangible objects of solidity, appearing as
hardness or softness, temperature, appearing as heat or cold, and
motion, appearing as motion or pressure. These rūpas condition citta and
cetasikas just by being the object they experience. The rūpas that are
the sense-bases are essential conditions for the sense-cognitions.
Seeing-consciousness could not arise if there were no eyesense, hearing
could not arise if there were no earsense, all sense-cognitions need
their own physical base. The sense-bases themselves are classified along
with the seven sense objects as gross
rūpas\footnote{Twelve rūpas are gross
because of impinging: the seven sense objects and the five sense-bases.
The other sixteen kinds of rūpas are subtle.}, even though they
are not experienced directly by those who begin to develop the Path.
They can only be experienced through the mind-door.

We are so used to thinking of self and what appears to the self. The
Buddha taught what is real at this moment, like seeing, hearing and
thinking. They arise and are present for such a short time, we cannot
imagine how short.

When there is some understanding of the reality at this moment, such as
seeing that appears now, we can begin to understand what reality is.
Seeing is a mental reality, nāma, that experiences what is visible. It
seems that we immediately see people. This shows us that cittas arise
and fall away extremely rapidly, succeeding one another. Seeing that
sees visible object is one moment and thinking of people is another
moment, different from seeing. Seeing is vipākacitta, the result of
kamma, and thinking is kusala citta or akusala citta. One may think
wisely of unwisely, depending on accumulated conditions.

One may think about someone else with kindness, and this is a cetasika,
mettā cetasika, arising with kusala citta. The citta with kindness falls
away immediately but since each citta is succeeded by a next one,
without a break, kindness is not lost; it is accumulated from one moment
of citta to the next moment so that there are conditions for the arising
again of kindness. It is the same with the accumulation of unwholesome
qualities. One may think with attachment, aversion and ignorance. These
akusala cetasikas fall away with the akusala citta, but they are
accumulated from moment to moment so that there are conditions for their
arising again.

The Buddha taught about realities and their conditions to help people to
realize that whatever arises is uncontrollable, non-self. He spoke time
and again about seeing, visible object, hearing, sound and thinking,
because these arise all the time. We cannot select what kind of object
we shall experience, desirable or undesirable, this depends entirely on
conditions which are beyond control. It is not easy to understand
different realities appearing one at a time. That is why we have to
listen and to consider the realities that appear now, again and again.

During the discussions in Vietnam, someone said that the more she
listens, the more confused she becomes. This is not surprising because
there were countless lives in the past with ignorance. We should be
happy to have an opportunity now to listen to the Dhamma. Ignorance
cannot be eradicated by a few moments or even many years of listening.
During the discussions we hear again the same things such as
explanations of the nature of seeing, hearing or thinking that arise
now. However, this is not boring, because it can remind us to consider
the present reality at those moments. Did we really consider seeing as
being different from thinking of what is seen? We cannot expect to
understand the truth of realities immediately. Seeing does not know what
is seen, it does not know whether there are people and things present in
the room. It just sees. The idea of self, of ``I see, I hear'' is deeply
rooted.

Acharn repeated many times: ``Be patient and truthful to seeing now.
Otherwise it is not possible to understand directly its arising and
falling away.'' We have intellectual understanding of the arising and
falling away of realities, but when paññā is more developed, it can
realize directly their arising and falling away. Then it will become
clearer that it is impossible to control realities.

It is in the beginning not easy to know the difference between
realities, namely, citta, cetasika and rūpa, and concepts of persons and
things we can think of but which are not real.

During the discussions, someone spoke of her child with concern about
his opinions as to the eating of meat. She thought that one should not
eat meat since animals have to be slaughtered in order to obtain it.
Acharn spoke about clinging to an idea of ``my child''. Someone may
cling to his child and strongly believe that the child exists. Because
of ignorance and attachment, there is an idea of ``my child''. The
Buddha taught that there is no person, no self. In that sense we can say
that we are living alone. We forget that there is no one there, only
different cittas, cetasikas and rūpas, conditioned elements. No matter
whether there are people around or not, there are just seeing, thinking,
thinking of stories, situations.

Sarah said: ``No courage to live alone, to understand reality at this
moment, no matter the circumstances. No courage to understand what
appears now, seeing now, visible object now, kusala, akusala, not anyone
who does anything.''

Nina: ``I see people, not visible object.''

Acharn: ``The `I' comes in very quickly.''

Sarah: ``Not enough understanding, not enough courage to live alone,
even at that moment of thinking and clinging to `I'. There is lobha
(attachment) at that moment. It is real, but it is not my clinging, or
my thinking, my wrong understanding. Just thinking, citta, cetasika and
rūpa.''

Acharn: ``Are you alone now? In order to understand realities one does
not need so many stories. Consider that which experiences and that which
now appears. Dhamma means that which now appears.''

We are all alone. Seeing arises for a moment and then falls away, it is
no one. Thinking arises and falls away, it is no one. There can be only
one citta at a time, arising and falling away. Where is the person? But,
at first there is only intellectual understanding which cannot eradicate
the belief in a person. When this has become firmer, there can be more
confidence in the truth and it can condition direct awareness of
realities appearing now, one at a time. Then it will be understood more
clearly that a person does not exist.

We read in the ``Kindred Sayings'' (Second Fifty, Ch 2, § 63, By
Migajāla) that Migajāla addressed the Buddha:

\begin{quote}

`` `Dwelling alone! Dwelling alone!' lord, is the saying. Pray, lord, to
what extent is one a dweller alone, and to what extent is one a dweller
with a mate?''

``There are, Migajāla, objects cognizable by the eye, objects
desirable, pleasant, delightful and dear, passion-fraught, inciting to
lust. If a brother be enamored of them, if he welcome them, if he
persist in clinging to them, so enamored, so persisting in clinging to
them, there comes a lure upon him. Where there is a lure there is
infatuation. Where there is infatuation there is bondage. Bound in the
bondage of the lure, Migajāla, a brother is called `dweller with a
mate.',''
\end{quote}

The Buddha said the same about the other objects experienced through the
other sense-doors and the mind-door. Even if that person would live in
solitary places, he is still called ``dweller with a mate'', because
craving is the mate he has not left behind.

The Buddha explained that when there is no clinging to those objects,
the lure fades away:

\begin{quote}
``When there is no lure, there is no infatuation. Where there is no
infatuation, there is no bondage. Freed from the bondage of the lure,
Migajāla, a brother is called `dweller alone.'

So also with regard to savours cognizable by the tongue, and mind-states
cognizable by the mind\ldots{}

Thus dwelling, Migajāla, a brother, though he dwell amid a village
crowded with brethren and sisters, with lay-brethren and lay-sisters,
with rājahs and royal ministers, with sectarians and their followers,-
yet is he called `dweller alone'. Why so? Craving is the mate he has
left behind. Therefore is he called `dweller alone'.''
\end{quote}



\chapter[Against the stream]{}
\section*{Against the stream of common thought}

A friend asked: ``What is understanding?'' This is a basic question and
I tried to answer this: ``It is the opposite of ignorance. Now there is
ignorance of many things. We think of cups, of food with attachment. We
know that there are many moments of attachment and these are always
accompanied by ignorance. We learn that seeing does not see people or
things, that that is already thinking. This shows how fast cittas arise
and pass away. We are misleading ourselves all the time.''

Our friend thought that there was a choice to be made and because of
this either understanding would arise or attachment and aversion. I
answered: ``There is no choice, there is no one who can choose or select
anything. It just happens by conditions. They are all realities, it does
not matter what arises. We should not try to change realities; some we
would rather not have. They are all dhammas and understanding can
understand them little by little.''

Acharn explained that one should be truthful to the truth, and
understand the present moment. One should not think: ``What can I do to
cling less to the self.''

All that appears naturally can be object of right understanding. When
clinging to the self appears, it can be known as just a dhamma, arisen
because of conditions. It is difficult to really see the difference
between the world of concepts and ideas and the world of realities such
as seeing, visible object, hearing, like, dislike and thinking. Life
lasts as long as one moment of citta that experiences one object at a
time.

We read in the ``Kindred Sayings'' (I, VI, The Brahmā Suttas, Ch I, § 1,
The Entreaty) that the Buddha, after his enlightenment, contemplated the
Dhamma that is subtle and deep, difficult to understand. He thought:



\begin{verse}
``This that through many toils I have won,\\
Enough! Why should I make it known?\\
By folk with lust and hate consumed\\
Not this a Dhamma that can be grasped.\\
Against the stream of common thought.\\
Deep, subtle, fine, and hard to see,\\
Unseen it will be by passion's slaves,\\
Cloaked in the murk of ignorance.''\\
\end{verse}



We read in the commentary to this
sutta\footnote{As given in the
translation by Ven. Bodhi in the ``Connected Discourses of the Buddha''.} :

\begin{quote}
``Living at ease (appossukkatā, lit. ``little zeal'') means lack of
desire to teach. But why did his mind so incline after he had made the
aspiration to Buddhahood, fulfilled the perfections, and attained
omniscience? Because as he reflected, the density of the defilements of
beings and the profundity of the Dhamma became manifest to him. Also, he
knew that if he inclined to living at ease, Brahmā would request him to
teach, and since beings esteem Brahmā, this would instil in them a
desire to hear the Dhamma.''
\end{quote}

We read that then Brahmā Sahampati entreated the Buddha to preach the
Dhamma. The Buddha surveyed the world with the eye of a Buddha and saw
beings who were easy to teach and who were difficult to teach. He
answered Brahmā Sahampati, making known his inclination to teach.

The Dhamma is subtle and deep, difficult to understand. It goes
``against the stream of common thought''. The idea of self is deeply
ingrained. We believe that a self sees, hears or thinks, is attached or
has aversion. That is why the Buddha taught for forty-five years so that
people would understand that what is taken for a person or self is only
citta, cetasika and rūpa that arise because of their proper conditions.

During the sessions we discussed birth and death. We believe: ``I was
born and I will die'', but what was born? Not the ``I'', only citta,
cetasika and rūpa. The first moment of life cannot be by anyone's will,
it has conditions to arise. Life goes on by conditions, from moment to
moment. When we see life in a conventional way, we think of a whole
lifespan that is ended by death. But in order to understand the truth,
we have to consider life from moment to moment. For example, when seeing
arises, life is seeing, and this arises and falls away immediately.
After seeing there may be thinking with attachment or aversion and these
are always accompanied by ignorance. Sometimes understanding may arise.
All these are different moments that are gone immediately. So are birth
and death, there is birth and death of citta at each moment. This is the
only way to really understand the truth of life.

Some people are afraid of death but the dying-consciousness
(cuti-citta), the last moment of this life, falls away and is succeeded
immediately by the following citta that is the rebirth-consciousness
(paṭisandhi-citta) of the next life. There is nothing to fear, cittas
arise and fall away, succeeding one another extremely rapidly. The
change from one life to another life occurs without there being time to
think about it. Even now, the next moment can arise in another plane of
existence.

Sarah reminded us: ``Hearing, considering, understanding and sharing
that understanding of dhammas as anattā is by far the most useful way
to spend this life which may come to an end at any time.''

The arahat has no more conditions for rebirth since all defilements have
been eradicated. Summarizing the three kinds of death: death in
conventional sense which is the end of a lifespan, momentary death
(khanika maraṇa) which is the falling away of every citta that has
arisen, and final death (samuccheda maraṇa) of the arahat.

Considering life at this moment, the momentary birth and death of citta,
helps us to see what is really true, different from conventional ideas
one may think of but which are not real. The Buddha taught realities by
way of many different aspects so that we would understand that there is
no person, no self, only citta, cetasika and rūpa.

In Hoi An, in the resort An Villa where we stayed, the Dhamma
discussions were held on a veranda under a roof, close to a swimming
pool. During the discussions, I saw people catching a snake by the side
of the pool. I was so absorbed in the ``story'' about the snake that I
was forgetful of realities. It is helpful to remember that many visible
objects appear one after the other, all different because of different
conditions. This leads to thinking of shape and form and the concept of
a snake. The snake is not a reality, but the thinking is. It is
essential to know the difference between concepts and realities.

Rūpa is different from nāma. Seeing is nāma and visible object is rūpa.
Rūpa does not think, remember, feel, like or dislike. It is important to
learn the different characteristics of nāma and rūpa. If they are not
distinguished from each other, we are bound to take whatever appears for
self.

Citta and cetasika are different types of nāma. Every citta experiences
an object, it is the ``leader'' in experiencing an object. Only one
citta arises at a time and it is accompanied by several cetasikas,
mental factors, which assist citta in cognizing the object.

Cittas can be classified in many ways and one of these is the
classification by way of ``jāti'' (literally birth or nature). Cittas
can be of the following four jātis: akusala, kusala, vipāka (result of
kusala kamma or of akusala kamma) or kiriya (inoperative, neither cause
nor result). Each citta is accompanied by several cetasikas. Some
cetasikas accompany every citta, some accompany cittas of the four jātis
but not every citta, some accompany only akusala citta and some
accompany only kusala citta.

Citta and its accompanying cetasikas are closely associated and they
condition one another. The cetasikas which accompany the citta arise
together at the same physical base, experience the same object as the
citta, and they fall away together with the citta. The cetasikas which
accompany citta are of the same jāti as the citta they accompany. When
the citta is kusala, all accompanying cetasikas are also kusala, even
those kinds of cetasikas which can arise with each type of citta. When
the citta is akusala, all the accompanying cetasikas are akusala.
Feeling, for example, is a cetasika which accompanies each citta. When
there is pleasant feeling, it can accompany kusala citta or akusala
citta rooted in attachment, but its quality is different in each case.

When generosity arises, there is no person who is generous, generosity
is a cetasika performing its function while it assists the kusala citta.
When attachment arises, there is no person who is attached, attachment
is a cetasika performing its function. Citta is the principal, the
leader in experiencing an object and the cetasikas that experience the
same object as the citta perform each their own function while they
assist the citta.

Seven cetasikas arise with every citta, no matter whether it is kusala,
akusala, vipāka, or kiriya (inoperative, neither kusala, akusala, or
vipāka). We have read about these cetasikas, the ``universals''
(sabbacittasādhāraṇa), in the text books and we have studied them, but
while we were in Nakorn Nayok we had an opportunity to discuss about
them in more detail. We were reminded that they are realities appearing
in daily life, conditioned dhammas that are not self. It is beneficial
to consider their true nature again and again.

Contact, phassa is mentioned first among the seven ``universals''. It is
not physical contact, but it is mental. It accompanies the citta and
contacts the object so that citta and cetasikas can experience the
object for an extremely brief moment. Contact is mentioned first because
without contact there would not be the experience of any object. When
seeing arises it is accompanied by contact that contacts visible object.
When hearing arises it is accompanied by contact that contacts sound. At
first we may believe that there can be seeing and hearing at the same
time. But they arise at different bases and experience different
objects, thus, from different conditions. They are different realities.

One may like to experience a specific object, but it is impossible to
cause phassa (contact) to contact such or such object. This reminds us
that there is no self who can select an object.

Feeling is another cetasika that accompanies every citta. Sometimes
feeling is pleasant, sometimes unpleasant, and sometimes indifferent.
When contact contacts an unpleasant object such as a disagreeable odour,
there is likely to be unpleasant feeling arising with the akusala cittas
after smelling-consciousness has fallen away. Smelling-consciousness is
accompanied by indifferent feeling. When contact contacts a pleasant
object such as a beautiful sound, there is likely to be pleasant feeling
arising after hearing has fallen away. Seeing, hearing, smelling and
tasting are always accompanied by indifferent feeling, but the
vipākacitta that is body-consciousness is either accompanied by
unpleasant bodily feeling or by pleasant bodily feeling. These feelings
are mental, but they are called bodily feeling because they accompany
body-consciousness. They are also vipāka, conditioned by kamma.

It is difficult to distinguish between feeling that is vipāka and
feeling that is akusala. When painful bodily feeling arises, unpleasant
feeling that accompanies akusala citta rooted in aversion may arise
immediately. We find feeling very important but mostly we are ignorant
of realities. People wish to have pleasant feeling all the time but it
falls away immediately. Sometimes it accompanies kusala citta such as
citta with generosity, but mostly it accompanies citta rooted in
attachment.

The following sutta reminds us of the diversity and complexity of
feelings. We read in the ``Kindred Sayings'' (IV, Kindred Sayings about
Feeling, 2. The Chapter on Solitude, § 12, The Sky), that the Buddha
said:

\begin{quote}
``Just as, brethren, divers winds blow in the sky -- some winds blow
from the east, some from the west, some from the north, some from the
south, winds dusty, winds dustless, cool winds and hot winds, winds soft
and boisterous -- even so in this body arise diverse feelings --
feelings pleasant, feelings painful, also neutral feelings.''
\end{quote}

The Buddha then speaks about a monk who has understanding and awareness
of all kinds of feelings and is without defilements.

When feeling appears it can be understood as a conditioned dhamma devoid
of self. Nobody can choose what kind of feeling arises, it is beyond
control.

Another cetasika that accompanies every citta is concentration or
one-pointedness (samādhi or ekaggatā cetasika). It focusses on the
object that citta experiences and it performs this function for an
extremely brief moment. It is the condition that citta knows only one
object. There are actually six worlds: the world of experiencing visible
object, of experiencing sound, of experiencing odour, of experiencing
flavour, of experiencing tangible object and of thinking. The conditions
for citta experiencing an object through one of the six doorways are
different and this makes it clear that there is no self who coordinates
all that is experienced.

Concentration or samādhi is of many degrees and it is developed in
samatha in order to suppress the ``hindrances''\footnote{The hindrances
(nīvaraṇa) are: sensuous desire (kāmacchanda), ill-will (vyāpāda), sloth
and torpor (thīna-middha), restlessness and worry (uddhacca-kukkucca)
and doubt (vicikicchā).}; it is one of the jhāna
factors leading to the attainment of jhāna (absorption). It is also one
of the eight Path factors leading to enlightenment. It focusses on the
nāma or rūpa that appears at the present moment, so that paññā can know
it as it really is, as impermanent, dukkha and anattā.

Remembrance, saññā, is another cetasika accompanying every citta. It
remembers and marks the object that is experienced at that moment so
that it can be recognized later on. The translation of saññā as memory
or remembrance should not mislead us, saññā is not remembrance as we
understand it in a conventional way. It is a cetasika that arises for an
extremely short moment together with the citta and the other cetasikas
it accompanies. It performs its function of recognition or marking the
object experienced by the citta it accompanies. Without saññā there
could not be any thinking about pleasant and unpleasant objects.

One may be attached to a pleasant object with happy feeling. Attachment
and happy feeling fall away immediately but saññā remembers this happy
feeling. This leads to more and more clinging to happy feeling. Saññā
conditions attachment. Since saññā accompanies every citta while it
performs its function, it has the same object as the citta and, thus,
its object can be a reality, a nāma or rūpa, or a concept.

The recognition of a person or a thing, which are concepts, is the
result of many different processes of citta and each moment of citta is
accompanied by saññā. Because of many moments of saññā, we can follow
the trend of thought of a speaker or we ourselves can reason about
something, connect parts of an argument and draw conclusions. All this
is not due to ``our memory'' but to saññā, which is not self but only a
kind of nāma. What we take for ``our memory'' or ``our recognition'' is
not one moment which stays, but many different moments of saññā which
arise and fall away.

There are several aspects to akusala saññā and kusala saññā. Akusala
saññā leads to taking realities for permanent (nicca-saññā) and for self
(attā-saññā).

Jonothan said: ``Attā-saññā accumulates, it influences the way the world
is seen now.''

We think of people and things, of the world in conventional sense, and
we forget that in reality there are only citta, cetasika and rūpa that
are gone immediately after they have arisen.

When saññā accompanies kusala citta, it is kusala. Firm saññā that is
kusala is mentioned in the ``Atthasālinī'' (I, Part IV, Chapter I, 121)
as a condition for the arising of sati of the level of satipaṭṭhāna. It
states that the proximate cause of mindfulness is firm remembrance
(saññā) or the four applications of mindfulness (satipaṭṭhāna). There
can be mindfulness of the nāma or rūpa which appears because of firm
remembrance of all we learnt from the teachings about realities of daily
life. We can learn by reading and by discussing. When we are reading, we
should not forget that the Buddha's words pertain all the time to
whatever appears right now. Listening is mentioned in the scriptures as
one of the most important conditions for the attainment of
enlightenment, because when we listen time and again, there can be firm
remembrance of the Dhamma.

Mindfulness is different from remembrance, saññā. Saññā accompanies
every citta; it recognizes the object and marks it, so that it can be
recognized again. Mindfulness, sati, is not forgetful of what is
wholesome. It arises with sobhana cittas. But when there is sati which
is non-forgetful of dāna, sīla, of the object of calm or, in the case of
vipassanā, of the nāma and rūpa appearing at the present moment, there
is also kusala saññā which remembers the object in the right way, in the
wholesome way. Saññā accompanying insight which remembers in the right
way the reality that appears, as non-self, is anattā-saññā.

Apart from contact (phassa), feeling (vedanā), concentration (samādhi or
ekaggatā cetasika) and remembrance (saññā), there are three more
cetasikas among the ``universals'' accompanying every citta: volition
(cetanā), attention (manasikāra) and life faculty (jīvitindriya). They
condition every citta and also the other cetasikas that they accompany.
When conditions are understood more it will become clearer that all
realities are anattā.

Apart from the seven ``universals'', there are other cetasikas that
accompany cittas of the four jātis of kusala, akusala, vipāka and
kiriya, though not every citta. These are the six ``particulars''
(pakiṇṇakā) and among them are ``thinking'' (vitakka) which touches the
object so that citta can experience it, effort or energy (viriya) and
decision (adhimokkha). Seeing does not need thinking and effort for the
experience of visible object, but the cittas that succeed it in the same
process and still experience visible object, although they do not see,
need these ``particulars''\footnote{The other particulars
are: sustained thinking (vicara), enthusiasm (pīti) and wish-to-do
(chanda).} .

Akusala cetasikas such as attachment (lobha), aversion (dosa) and
ignorance (moha) only accompany akusala cittas. Sobhana (beautiful)
cetasikas, such as non-attachment (alobha), non-aversion (adosa) and
understanding (paññā) accompany sobhana cittas.

All cetasikas are realities occurring in daily life, they are not merely
textbook terms. They condition citta and they condition each other and
when we truly consider their functions it becomes clearer that there is
no self who can control realities. All the classifications of realities
given by the Buddha have as their aim helping beings to understand that
each reality that appears is non-self.


\chapter[Inner and the outer sense-fields]{}
\section*{Inner and the outer sense-fields}

I was talking with Alberto, a friend from Italy, about accumulated
tendencies. Accumulations have such a strong impact and we are
overwhelmed by them at times. I am worried, for example, about many
things I have to do in daily life and I am thinking about how to act.
Alberto reminded me that we always take accumulated tendencies for self.

Acharn said: ``We always forget that all are dhammas. We are concerned
about `myself'. Thinking, `how can I do this, what about me tomorrow
morning'. Develop understanding, it does not matter what is arising. It
is gone completely.''

We were discussing conceit (māna), what conceit is and when it arises.
Acharn made it clear that if we name it and keep on thinking about it,
it is already completely gone. The present reality is not conceit at
such a moment. We are just speculating about what happened in the past
instead of knowing the characteristic of the present reality. She
brought us back to the present moment by saying that there is seeing
right now. Seeing sees visible object which is there, accompanied by the
four great Elements of solidity, cohesion, temperature and motion.

When we consider rūpas, physical phenomena, we may see them by way of
science, in a conventional way, but in reality they arise and fall away
each moment. We think of the body as a whole, but what we take for our
body are in reality many groups or units, consisting each of different
kinds of rūpa, and the rūpas in such a group arise together and fall
away together. Rūpas do not arise singly, they arise in units or groups.

There are four conditioning factors that produce rūpas of the body:
kamma, citta, temperature and nutrition. Kamma is actually the cetasika
volition or intention (cetanā) that arises with every citta. When it is
kusala or akusala it can motivate a good deed or a bad deed that
produces result. Kamma is a mental activity which is accumulated. Since
cittas that arise and fall away succeed one another in an unbroken
series, the force of kamma is carried on from one moment of citta to the
next moment of citta, from one life to the next life. In this way kamma
is capable to produce its result later on. A good deed, kusala kamma,
can produce a pleasant result, and an evil deed can produce an
unpleasant result. Rebirth-consciousness is the mental result of kamma,
vipākacitta, but at that moment kamma also produces rūpas and kamma
keeps on producing rūpas throughout life; when it stops producing rūpas
our lifespan has to end. Kamma produces particular kinds of rūpas such
as the senses.

Citta also produces rūpas. Our different moods become evident by our
facial expressions and then it is clear that citta produces rūpas.

Temperature, which is actually the element of heat, also produces rūpas.
Throughout life, the element of heat produces rūpas.

Nutrition is another factor that produces rūpas. When food has been
taken by a living being, it is assimilated into the body and then
nutrition can produce rūpas. Thus, some of the groups of rūpas of our
body are produced by kamma, some by citta, some by temperature and some
by nutrition. If we see the intricate way in which different factors
condition the rūpas of our body we shall be less inclined to think that
the body belongs to a self.

There are not only rūpas of the body, there are also rūpas which are the
material phenomena outside the body. If we do not study rūpas, we may
not notice their characteristics that appear all the time in daily life.
We shall continue to be deluded by the outward appearance of things
instead of knowing realities as they are.

Seeing sees visible object. Visible object is a rūpa that arises in a
group of rūpas. These are the four Great Elements of solidity, cohesion,
temperature, motion, and in addition odour, flavour and nutritive
essence. Although these rūpas arise together, only one kind of rūpa at a
time can be the object that is experienced. When there are conditions
for seeing, only visible object is seen. Visible object is conditioned
by the other rūpas that arise in the same group. For example, solidity
may be hard or soft at different moments and this conditions visible
object to be different too at different moments. Different visible
objects that appear lead to thinking of shape and form and this again is
a condition to think of concepts of persons and things. On account of
what is seen, heard or experienced through the other senses, we think of
concepts. Gradually we can learn to see the difference between realities
and concepts.

When there are conditions for the arising of sati of the level of direct
awareness, understanding can know what appears as just a conditioned
reality. Acharn also explained that no matter we talk about realities as
stated in the teachings, we have to come back to seeing right now,
hearing right now.

People have many problems in daily life, they have worries and
anxieties, but one should remember Acharn's reminder of ``Is there
seeing now?'' There is seeing time and again and its characteristic can
be known when it appears without having to think about it. After seeing
has fallen away we often think with attachment, aversion and ignorance
about what was seen. These three unwholesome roots (akusala hetus) are
the real problems in life. One thinks that the cause of unhappiness is
one's partner, a colleague at work, or the way one was treated by
others. But the real cause is the accumulated attachment, aversion and
ignorance.

Intellectual understanding (pariyatti) can condition direct
understanding of realities (paṭipatti) and in this connection Acharn
mentioned that the understanding of the āyatanas (bases or sources) can
be a supporting condition for direct understanding.

All realities can be classified as twelve āyatanas, translated sometimes
as ``sense-fields'' (Visuddhimagga, Ch. XV,
1-17)\footnote{See also Book of
Analysis, Vibhaṅga, II, Analysis of Bases.}. There are six
inward āyatanas and six outward āyatanas. The six inward āyatanas are:
eyesense, earsense, smelling-sense, tasting-sense, bodysense and
mind-base, including all cittas.

The six outward āyatanas are: visible object, sound, odour, flavour,
tangible object and mind-object (dhammāyatana), including cetasikas,
subtle rūpas and nibbāna.

Āyatana implies association of realities, and this shows that the
teaching of āyatanas is not abstract. Association of dhammas takes place
at this very moment, but usually we are ignorant of this. There could
not be hearing now if there were no association of sound, earsense and
hearing. Different characteristics appear one at a time, and there can
be understanding of them. That is the real study of āyatanas.

When we see, hear or think, we believe that a self experiences objects,
but in reality there is the association of the inward āyatanas and the
outward āyatanas, the realities ``outside''. When seeing sees visible
object there is the association of eye-sense, visible object,
seeing-consciousness and the seven cetasikas that accompany every citta,
the ``universals''.

Understanding what āyatanas are will help us to know the difference
between reality and concept. When touching the body, there is a meeting
between tangible object and the rūpa that is bodysense, so that there
are conditions for body-consciousness that experiences tangible object.
Only for a very short moment, only at the moment these āyatanas
associate, namely, mindbase (manayatana which is citta), accompanied by
cetasikas (dhammāyatana), tangible object (photabbāya-tana) and
bodysense (kāyāyatana). They meet for an extremely short time and then
they are gone. This reminds us how fragile and insignificant what we
take for body is. There is no whole body, it does not exist. It is only
an idea we may think of.

Sarah explained that the āyatanas associate for a short moment and that
it could not be any other way.

She said: ``It all shows that there is no self at all involved, no one
who can control anything for an instant. Understanding more about these
realities and their conditioned nature and `coming together' is the way
that understanding develops with detachment and awareness which leads to
the abandoning of the idea of self and other defilements. It's the only
way. Right understanding leads to the development of direct awareness of
what appears now very naturally. If there is any expectation of
awareness or expectation of understanding arising now, it shows the deep
clinging to the idea of self again, not the detachment which right
understanding brings at this moment. It really is the `fine point' or
`balancing on a pinhead' of dhammas arising by conditions very naturally
and without any expectation of what will arise next. This is the only
way that satipaṭṭhāna and the Path can develop. The detail about the
āyatanas shows the real subtlety and depth of the Buddha's Teachings.''

Acharn said that there should be no expectation for the arising of
awareness, but when one has understanding of the āyatanas, there will be
a condition for that moment unexpectedly. One will understand more and
more that all dhammas are anattā. It is most helpful that Acharn
emphasized the importance of understanding the āyatanas. We have read
about their classification many times but the truth of these realities
should be considered over and over again. ``Are there āyatanas now?''
Acharn asked us.

The dhammas of our life arise and fall away, they are dukkha,
unsatisfactory or suffering. Someone of the listeners asked what
suffering is. He tried to walk slowly and stop thinking in order to have
less worry. We are so involved in stories, thinking, ``What happens to
me, what shall I do.'' We forget the real meaning of dukkha the Buddha
taught: the arising and falling away of realities right now. Since they
fall away immediately, they are unsatisfactory, not worth clinging to.
So long as we are involved in stories, thinking of concepts instead of
understanding whatever reality appears now, there is no way of solving
any problem in life. When we think of suffering and of what we have to
undergo in our life we forget that each reality that appears now falls
away immediately, even thinking or worry. We need patience to consider
all the different realities of our life.

The Buddha, in his previous lives when he was still a Bodhisatta,
considered all realities over and over again with endless patience. This
was the only way to develop the wisdom leading to enlightenment and
becoming the Omniscient Buddha. Acharn asked me several times to speak
about the three kinds of Bodhisattas, beings destined for enlightenment:
someone who will become a Sammāsambuddha, Omniscient Buddha, someone who
will become a Paccheka Buddha, Silent Buddha, and someone who is a
Savaka Bodhisatta, who is a ``Learner''\footnote{Cariyāpiṭaka-aṭṭhakathā,
nidānakathā.} . In the beginning, I did
not quite understand the meaning of Savaka Bodhisatta, but by repeatedly
discussing this term I better understood what a Savaka Bodhisatta is and
how we can become a Savaka Bodhisatta by developing understanding of
realities, even for countless lives.

The Sammāsambuddha has attained perfect understanding of the truth of
realities, all by himself, without the help of a teacher. Through his
enlightenment, he reached omniscience. The Buddha could attain
enlightenment because he developed for innumerable lives direct
understanding of seeing which appears at the present moment, of visible
object which appears at the present moment, of all realities which
appear at the present moment. Direct understanding of realities can only
be developed now.

The Silent Buddha, Paccheka Buddha, has also attained enlightenment all
by himself, but he has not accumulated wisdom to the same extent as the
Sammā-sambuddha. He cannot proclaim to the world the truth he has
realized.

The Savaka Bodhisatta develops understanding during innumerable lives
until the moment of enlightenment. He knows the value of understanding
of realities appearing at the present moment.

A moment of understanding is not lost, it is accumulated from moment to
moment. Conditions for attaining the first stage of enlightenment, the
stage of the ``stream-enterer'' (sotāpanna)
are\footnote{See ``Dialogues of the
Buddha'', DN 33.1.11(13)} : association with good
people (sappurisa-saṃsevo), hearing the true Dhamma (saddhammassavanaṃ),
thorough attention (yoniso manasikāro), practice of the Dhamma in
accordance with the Dhamma (dhammā-nudhammap\-paṭipatti).

As to ``wise attention'', the commentary explains that this is attention
to impermanence, dukkha and anattā.

As to the practice of the dhamma in conformity with the dhamma, the
commentary states that practice in conformity with the dhamma relates to
lokuttara dhamma, and that previous practice is necessary, which is,
according to the subcommentary, vipassanā, the development of insight.

The word ``practice'' is a translation of the term paṭipatti, meaning
the development of direct understanding.

This phrase is also explained in the commentary to the
Mahāparinib\-bānasutta: ``Those who practise a dhamma consistent with the
dhamma (dhammānu-dhamma-paṭipannā). Those who practise the teaching of
insight (vipassanā) which is consistent with the teachings of the noble
(ariyadhammassa).''

The teaching of the noble is of the path (magga), or else also the
ninefold supramundane
dhamma\footnote{This refers to the eight
lokuttara cittas and nibbāna. For each of the four stages of
enlightenment there is the magga-citta and the phala-citta, the result
of the magga-citta.} .

These are the essential conditions leading to the penetration of the
four noble Truths. Gradually the true nature of the realities that
appear can be penetrated. As we read, there has to be wise attention to
the characteristics of impermanence, dukkha and anattā. However, first
of all there have to be right awareness and direct understanding of the
realities appearing through the six doors. Nāma has to be known as nāma
and rūpa as rūpa. We read, consider and discuss Dhamma just in order to
understand the reality of this moment. When a moment of understanding
arises, understanding is accumulated little by little. This is the way
that paññā can grow to the degree of lokuttara paññā. We need confidence
and courage so that we do not become disheartened about the long way we
have to travel.

We learn from the statements about the three kinds of Bodhisattas that
whatever occurs in life has been conditioned. We can become
``learners'', savaka Bodhisattas, by continuing to develop
understanding of what appears now with confidence and without expecting
anything. We are beginners and what can be understood depends on
conditions. Some people seek peacefulness but everything that arises
now, also when it is not peaceful, should be known as not self.

We heard a great deal about citta, cetasika and rūpa, which are
conditioned realities arising for a short moment and then falling away.
They can be considered in our daily life. However, we are inclined to
cling to situations, to cling to thinking of other people and ``self''.
That is not the world of realities. We forget that all that is real now
in our life are citta, cetasika and rūpa. Intellectual understanding of
what is real can eventually lead to direct understanding, to
satipaṭṭhāna.

Satipaṭṭhāna is developed in being mindful of whatever appears so that
direct understanding can grow. Mindfulness, sati, is a cetasika that
arises only with certain types of citta and lasts for an extremely short
moment. Remembering this helps us not to try to control or manipulate
this type of cetasika. All that can be done is listening and considering
the truth of this moment and understanding it a little more, so that we
can become a ``savaka'', a learner, who will once attain enlightenment
and full understanding. This is possible, the Buddha taught the way. But
when we are hoping or expecting results, there is clinging and this
hinders the development of understanding. We shall listen again and
again to the truth of seeing, visible object, hearing, sound, and all
that is reality. This is the way to have more confidence in the Buddha's
teaching of anattā.

Some people wonder why we discuss seeing and hearing all the time. They
are realities of daily life that arise and fall away extremely rapidly,
but they are unknown. There is seeing now but no one pays attention to
it, it falls away in split seconds. What is seen is only what impinges
on the eyebase but we think immediately of shape and form and take it
for something that stays. Understanding of seeing, visible object and
all other dhammas appearing at this moment develops very gradually and
hearing about them and discussing them is never enough.

Seeing is an element that experiences an object and visible object is an
element that is seen, it does not experience anything. So long as the
different characteristics of mental realities, nāma, and physical
realities, rūpa, are not distinguished, it is impossible to understand
them as anattā. It is not sufficient to merely think that nāma is
different from rūpa. Understanding can come to realize the different
characteristics of seeing and visible object, hearing and sound, of all
mental realities and physical realities, when they appear at this
moment.

Sarah explained: ``Without even an intellectual understanding of them,
there cannot be any understanding of kusala and akusala (wholesome and
unwholesome states) and there can't be even a beginning of following the
right Path.''

People want to know the difference between kusala dhammas and akusala
dhammas when they appear in daily life, but first the difference between
the elements that experience an object and the elements that do not
experience anything should be directly known. Intellectual understanding
(pariyatti), when it is firm enough, can condition direct understanding
(paṭipatti), as was often emphasized during our discussions.

\chapter[Factors leading to enlightenment]{}
\section*{Factors leading to enlightenment}

In Nakorn Nayok we discussed the thirty-seven factors leading to
enlightenment (bodhipakkhiya-dhammas). They are not theoretical terms,
they are realities that can be developed in daily life. We should first
consider what the meaning is of enlightenment. Acharn said people are
usually dreaming about enlightenment as some desirable state, without
understanding its meaning. It is actually perfect understanding of the
truth. This has to begin by understanding the truth of this moment. She
said:

``Most important is understanding this moment, otherwise it is gone
without understanding, thus, with ignorance. Each moment in life is
gone, by conditions. Without conditions nothing can arise, such as
seeing, it cannot arise without eyesense. Just learn in the beginning to
understand the reality and to understand that it is conditioned. Know by
how many conditions in order to understand that it can never be yours.
It is only a reality that is conditioned to arise and nobody can stop
its arising. Like now, nobody can stop the arising of seeing. So, there
is no you. You think, at the moment of seeing, that you see, but
actually seeing is not you. Learn to understand each moment as not me.''

Someone asked: ``What conditions knowing that seeing is not: `we see' or
`I see'?''

Acharn answered: ``One just wants to keep the `I', not to abandon the
`I'. One should not just listen and read, but consider deeply the truth.
It takes time. Do not forget that all dhammas are anattā, no
exception.''

She explained that people think of ``my problem'', but that this is not
the understanding of dhammas as anattā.

The factors of enlightenment, bodhipakkhiya-dhammas, are wholesome
qualities that, when fully developed, can lead to enlightenment.
However, the attainment of enlightenment is entirely dependent on
conditions. We can take our refuge in the Buddha, the Dhamma and the
Sangha. The Dhamma as our refuge includes these factors leading to
enlightenment. They are the following factors:

\begin{itemize}
\item
  the four applications of mindfulness,
\item
  the four right efforts,
\item
  the four roads to power (iddhi-pāda),
\item
  the five faculties (indriyas),
\item
  the five powers (balas),
\item
  the seven factors of enlightenment (bojjhanga),
\item
  the eightfold Path.
\end{itemize}

As to the four Applications of Mindfulness, these are all nāmas and
rūpas that appear now, no matter under what application they have been
classified. Thus, at one moment there can be mindfulness of rūpa, the
next moment of feeling, of citta or of dhamma. The application of
mindfulness of dhamma includes rūpa and nāma under different aspects. It
all depends on sati and paññā what object they take. Sati and paññā are
mere dhammas, not self, and they cannot be directed to specific objects.
The classification as four applications is given so that people can be
reminded that all realities in all situations can be object of
mindfulness and understanding. When pariyatti, intellectual
understanding of the present reality, is quite firm, it can condition
the arising of direct understanding. Direct understanding is developed
through satipaṭṭhāna.

As to the four right efforts, we read in the ``Dialogues of the Buddha''
(Dīgha Nikaya, XXXIII, Sangīti Sutta, D. III, 1, 221):

\begin{quote}
``Four great efforts (sammappadhānā): Here a monk rouses his will
(chanda), makes an effort, stirs up energy, exerts his mind and strives
to prevent the arising of unarisen evil unwholesome mental states. He
rouses his will\ldots{} and strives to overcome evil unwholesome
mental states that have arisen. He rouses his will\ldots{} and strives
to produce unarisen wholesome mental states. He rouses his will\ldots{}
and strives to maintain wholesome mental states that have arisen, not to
let them fade away, to bring them to greater growth, to the full
perfection of development.''
\end{quote}

The word ``will'' used here is a translation of chanda. Chanda is also
translated as zeal or wish-to-do. There are several kinds of chanda:
zeal of craving, of wrong view, of energy, of dhamma. Dhamma-chanda,
zeal for the dhamma, is intended here.

Right effort or energy accompanies right understanding of the eightfold
Path. It is not self, but a cetasika performing its function. When there
are conditions for sati to be mindful of whatever dhamma appears through
one of the six doorways, understanding can develop so that the reality
that is the object can be seen as a mere dhamma, not self. The
accompanying energy performs its function already and there is no need
to think of applying effort.

When there is mindfulness of visible object which appears now, seeing
which appears now, sound which appears now, hearing which appears now,
or any other reality which appears now, right understanding of the
present reality is being developed. This is the most effective way to
avoid akusala, to overcome it, to make kusala arise and to maintain
kusala and bring it to perfection. At that moment right effort performs
its task of strengthening the kusala citta with right understanding so
that there is perseverance with the development of understanding of
realities.

Effort or energy has nothing to do with effort in conventional sense.
Even at the level of pariyatti there has to be wholesome energy, kusala
viriya, and right understanding of realities will lead to the four
right efforts. At this moment effort, viriya, can maintain the kusala
that has arisen. This means it is a support for the arising again of
kusala citta.

There are four ``Roads to Power'', iddhipāda. They are classified among
the factors leading to enlightenment. Iddhipāda is also translated as
basis of success. They are: chanda (wish-to-do), citta (firmness of
kusala citta or concentration), viriya or energy, and vīmaṃsā or
investigation, another word for paññā.

The ``roads to power'' are actually four predominant factors. Whenever
we wish to accomplish a task, one of these four factors can be the
predominance-condition for the realities they arise together with.
Chanda, viriya and citta can be predominant in the accomplishment of an
enterprise or task both in a wholesome way and in an unwholesome way,
whereas vimaṃsa, investigation of Dhamma, which is paññā, a sobhana
cetasika, can only be predominant in a wholesome way. In the context of
the ``bases of success'', only the factors which are sobhana are dealt
with.

Citta can be a predominance-condition for the accompanying cetasikas,
but not all cittas can be predominance-condition. Predominance-condition
can operate only in the case of javana-cittas accompanied by at least
two roots. All mahā-kusala cittas (kusala cittas of the sense-sphere)
and all mahā-kiriyacittas (of the arahat), have the two roots of alobha,
non-attachment, and adosa, non-aversion, and, in addition, they can have
the root which is paññā, thus, they have two or three roots and,
therefore, they can be predominance-condition for the accompanying
dhammas. When we accomplish a task with cittas which are resolute,
firmly established in kusala, the citta can be the
predominance-condition for the accompanying dhammas.

With regard to investigation of the Dhamma, vīmaṃsā, this is paññā
cetasika. When we listen to the Dhamma, consider it and are mindful of
realities, vīmaṃsā can condition the accompanying citta and cetasikas by
way of predominance-condition.

Without the conditioning force of one of the four predominance factors,
it would not be possible to attain jhāna.

We read in the ``Visuddhimagga'' (III,24):

\begin{quote}
``\ldots{}If a bhikkhu obtains concentration, obtains unification of
mind, by making zeal (chanda) predominant, this is called concentration
due to zeal. If\ldots{} by making energy predominant, this is called
concentration due to energy. If\ldots{} by making (natural purity of)
citta predominant, this is called concentration due to citta. If\ldots{}
by making inquiry (vimaṃsā) predominant, this is called concentration
due to inquiry (Vibhanga 216-219)\ldots{}''
\end{quote}

The Bases of Success can also lead to supernatural powers like walking
on water, knowing one's former lives (Visuddhimagga, Ch XII, 50-53).

In the development of vipassanā, right understanding of nāma and rūpa,
one also needs the ``four bases of success'' for the realization of the
stages of insight wisdom and for the attainment of enlightenment. The
arising of awareness and understanding of realities is beyond control,
it is due to conditions. We need patience and courage to persevere
studying and considering nāma and rūpa, and, if there are the right
conditions, to be aware of them in daily life.

Among the factors pertaining to enlightenment are five faculties,
indriyas, sometimes referred to as ``spiritual faculties''. These are
sobhana cetasikas (beautiful mental factors) included in the ``factors
of enlightenment'' (bodhipakkiya dhammas) that should be developed for
the attainment of enlightenment. They are: faith or confidence (saddhā),
energy (viriya), mindfulness (sati), concentration (samādhi) and
understanding (paññā).

Indriya means principal, or leader. There are several indriyas, but each
indriya is a leader, a leader in its own field. Mindfulness is an
indriya, a ``controlling faculty'', a ``leader'' of the citta and
accompanying cetasikas in its function of heedfulness, of
non-forgetfulness of what is wholesome. The five wholesome controlling
faculties, the ``spiritual faculties'', must be developed in samatha in
order to attain jhāna and in vipassanā in order to attain enlightenment.
It is our nature to be forgetful of the reality which appears now, but
gradually mindfulness can be accumulated. It can even become a ``power''
(bala). As understanding develops, the accompanying spiritual faculties
also develop.

We are so often taken in by thinking of people and situations, and this
is with attachment and ignorance time and again. We would wish not to be
absorbed, but that is not the development of understanding. Whatever
arises, even being absorbed, is a reality that can be object of right
understanding. Acharn often tells us to begin again and again with the
development of right understanding.

She said: ``There is not yet direct understanding and that is why we
talk about seeing so that it will be object of right understanding. Even
touching now is real, but there is no understanding, no matter how often
we touch. But when understanding arises, it is intellectual
understanding, until there will be a moment of touching with direct
understanding.''

Among the factors of enlightenment, there are five powers, balas. We
read in the ``Dialogues of the Buddha'' (Dīgha Nikāya, Sangītisutta, IV,
sutta 26) about four powers, since in this section confidence is
omitted. The five powers are: confidence (saddhā), energy (viriya),
mindfulness (sati), concentration (samādhi) and understanding (paññā).

They are unshakable by their opposites. Lack of confidence is the
opposite of confidence, indolence is the opposite of energy,
forgetfulness is the opposite of mindfulness, restlessness the opposite
of samādhi and ignorance the opposite of paññā. The commentary states
that all of them refer to samatha, vipassanā and magga (lokuttara
magga), and that they can be mundane or supramundane. In the case of
magga, lokuttara magga-citta, they accompany lokuttara citta and they
are also lokuttara. They experience nibbāna.

Confidence (saddhā) is also one of the powers. We should have more
confidence in the power of kusala dhammas and this confidence is a
condition for their development.~

Among the factors pertaining to enlightenment, there are seven factors
of enlightenment, bojjhangas. We read in the ``Dialogues of the Buddha''
(Dīgha Nikaya, XXXIII, Sangīti Sutta, D. III, 2, 252):

\begin{quote}
``Seven factors of enlightenment (sambhojjhangā): mindfulness,
investigation of phenomena (dhammavicaya), energy, delight (pīti),
tranquillity, concentration, equanimity.''
\end{quote}

Dhammavicaya, investigation of realities, is another term for paññā. The
enlightenment factors develop together with paññā that understands more
and more the true nature of nāma and rūpa that appear at the present
moment. All factors develop together and are leading to enlightenment
and in that case they are still mundane. When enlightenment has been
attained, they are lokuttara and nibbāna is the object experienced at
that moment. When insight is developed, there is also calm. One is not
disturbed by unwholesome thoughts about persons and situations when
right understanding of dhammas is developed. One begins to see thinking
about them as an impersonal element devoid of self.

Right concentration, sammā-samādhi, focusses on the object of vipassanā
in the right way. When there is mindfulness of one object at a time as
it appears through one of the sense-doors or the mind-door, right
concentration focusses on that object, and at that moment right
understanding can investigate it so that it will be seen as it really
is. When right understanding arises, there is right concentration, which
is conascent with it.

There are many types of concentration and many levels of it. It is
actually the cetasika that is one-pointedness, ekaggatā cetasika. It
accompanies each citta and it is the condition that each citta
experiences only one object. All accompanying cetasikas share that same
object. When seeing arises, one-pointedness focusses on visible object,
and only that object is experienced. When hearing arises, it focusses on
sound. Hearing is a moment that is quite different from seeing, they
experience different objects. There is no self who can choose what
object will be experienced, this is dependent on several conditions.
Although there is concentration with each citta, usually it does not
appear. When jhāna is being developed with right understanding of the
object which can be a condition to be removed from sense objects,
samādhi grows until it is access-concentration, upacāra samādhi, and
attainment concentration, appanā samadhi, which is
absorption-concentration or jhāna.

We can be easily deluded and take for right concentration,
sammā-samādhi, what is wrong concentration, micchā-samādhi. When it is
conascent with lobha, it is wrong concentration. We are inclined to take
samādhi for `my concentration', and, therefore, it is important to
remember that it is only a dhamma conditioned by many different factors.
It is conditioned by the citta it accompanies and by the conascent
cetasikas.

The factors of the eightfold Path are among the factors pertaining to
enlightenment: right view (sammā-diṭṭhi), right thought
(sammā-sankappa), right speech (sammā-vācā), right action
(sammā-kammanta), right livelihood (sammā-ājīva), right effort
(sammā-vāyāma), right mindfulness (sammā-sati), right concentration
(sammā-samādhi).

These are sobhana cetasikas each performing their own function. One
should not just remember their names but they are dhammas each with
their own characteristic and these can gradually be known when they
appear. The factors of the eightfold Path all develop together with
right view, paññā. It is not ``I'' who practices, but the factors of the
eightfold Path that will come to fulfilment at the moment of
enlightenment. The object of the Path factors is a reality such as
seeing or visible object that appears at the present moment.
Sammā-sankappa, right thinking, is vitakka cetasika which is sobhana.
When right thinking is a factor of the noble eightfold Path it has to
accompany right understanding, paññā. Right thinking ``touches'' the
nāma or rūpa which appears so that paññā can understand it as it is.

There are three cetasikas which are sīla, namely: right speech
(sammā-vācā), right action (sammā-kammanta) and right livelihood
(sammā-ājīva). They are actually the three abstinences or virati
cetasikas which are: abstinence from wrong speech (vacīduccarita
virati), abstinence from wrong action (kāyaduccarita virati),
abstinence from wrong livelihood (ājīvaduccarita virati).

Paññā can realize that the cetasika which abstains from akusala is
non-self, that it arises because of its appropriate conditions. The
three abstinences which accompany cittas of the sense-sphere,
kāmāvacara cittas, arise only one at a time. However, when lokuttara
citta arises, all three abstinences accompany the lokuttara citta and
then nibbāna is the object. They fulfil their function of path-factors
by eradicating the conditions for wrong speech, wrong action and wrong
livelihood.

Sammā-vāyāma or right effort is another factor of the eightfold path. It
is viriya cetasika (energy or effort), which strengthens and supports
the accompanying dhammas. When it accompanies right understanding of the
eightfold Path, it is energy and courage to persevere being aware of
nāma and rūpa which appear one at a time through the six doorways. At
the moment of mindfulness of nāma and rūpa, right effort has arisen
already because of conditions and it performs its function; we do not
need to think of making an effort. When we think, ``I can exert effort,
I can strive'', akusala citta has arisen with clinging to the idea of
self reaching the goal. Wrong effort may arise without our noticing it.
Right effort, when it accompanies right understanding, supports the
other factors of the eight-fold Path, but we should remember that it
arises because of its own conditions, that it is non-self.

Sati of the level of satipaṭṭhāna is right mindfulness of the nāma or
rūpa which appears so that understanding of that reality as non-self can
be developed. Mindfulness does not last, it arises just for a moment,
but it can be accumulated. Mindfulness and right understanding cannot
arise without the appropriate conditions: listening to the teachings as
explained by the right friend in Dhamma, considering what one has heard
and applying it in daily life. Moreover, all wholesome qualities
developed together with satipaṭṭhāna are supportive conditions for
paññā. When we learn to be less selfish and develop kindness,
thoughtfulness and patience, these qualities support paññā to become
detached from the idea of self. Sati that accompanies right
understanding of the eightfold Path is a factor of the noble eightfold
Path.

One may wonder how, in the development of insight, the faculty of
mindfulness, the power of mindfulness, the path-factor right mindfulness
and the enlightenment factor of mindfulness can be developed. The answer
is: through mindfulness and understanding of the nāma and rūpa which
appears right now. There is no other way. Sights, sounds, scents,
savours and tangible objects are most of the time objects of attachment,
aversion and ignorance. If mindfulness arises and right understanding of
the object is being developed, one is at that moment not enslaved to the
object nor disturbed by it.

Sammā-samādhi, right concentration, is another path-factor accompanying
sobhana citta. Kusala citta which is intent on dāna, sīla or bhāvanā is
accompanied by right concentration which conditions the citta and
accompanying cetasikas to focus on an object in the wholesome way. Right
concentration, which is a factor of the noble eightfold Path, has to
accompany right understanding of the eightfold Path.

All these enlightenment factors are not theory. They are realities to be
developed. Several factors are the same types of cetasikas but they are
shown under different aspects. For example the faculties (indriyas) when
developed more can become powers, balas. Then they have become
unshakable. When sati has become a power, it can arise any time and at
any place, no matter the circumstances. We see that many conditions are
necessary for the attainment of enlightenment.


\chapter[Everything is dhamma]{}
\section*{Everything is dhamma}

After our discussions in Thailand, we flew to Vietnam where we first had
four days of discussions in Saigon. Among the listeners were many old
friends who had been to Dhamma sessions before. They took a great
interest in the discussions and posed many questions. It was the time of
New Year's celebration, ``Tet'' in Vietnamese, and it is the custom to
give money with one's good wishes for the New Year. People gave money to
be spend on charity such as the printing fund for Acharn's books and a
few of my books that were translated into Vietnamese. These books were
out of print. Friends had also composed a book with collections of
Dhamma discussions with Acharn translated into Vietnamese, that were
held in Vietnam at several occasions. There were sessions in the morning
and the afternoon, and one of the nuns who always listened with great
attention remarked that time went by too quickly while attending the
sessions.

One of the listeners found it difficult to study Dhamma. She found that
she knew just seeing, but she was not mindful of thinking. How could she
be mindful?

Acharn asked: ``What about understanding? Each word of the Buddha should
be carefully studied, like seeing. Without the Buddha's word, it is not
possible to understand seeing as not self. Seeing now is taken for `I
see'. Consider whether this is a reality which is not self. Right
understanding has to develop right now.''

She explained that seeing is not that which is seen. We should be
truthful and patient in order to understand seeing now. Truthfulness and
patience are perfections that should be developed together with paññā,
they are indispensable. Otherwise there are no conditions to know
directly the arising and falling away of each reality that appears. She
often referred to the
perfections\footnote{The perfections are
liberality (dāna), good morality (sīla), renunciation (nekkhamma),
wisdom (paññā), energy (viriya), patience (khanti), truthfulness
(sacca), determination (adiṭṭhāna), loving kindness (mettā) and
equanimity (upekkhā).}  that should
be developed together, because without these there can never be the
eradication of ignorance, attachment and other defilements. We should
learn more about realities so that we shall have more confidence in the
Buddha's teachings. Without the accumulation of the perfections, it is
impossible to understand what appears now. One follows ignorance and
attachment all the time. The perfections support paññā. We need courage
and energy so that we shall not be discouraged when the development of
understanding is so slow. Mettā, lovingkindness, is a condition to be
concerned for others and to think less of ourselves. The perfection of
determination is indispensable so that the development of paññā can
continue.

Acharn said that just considering the words: ``Life is so short'' can be
a perfection, because one understands that the best moment in life is
hearing, understanding whatever appears now. The perfection of
truthfulness means being truthful to reality now as non-self. Acharn
said: ``One has to consider one's blindness and the great understanding
of the Enlightened One. Consider, consider, consider so that there would
be understanding gradually. There is ignorance after seeing, but it is
unknown all the time.''

Acharn said that the moment of listening to the Buddha's teachings is
paying respect to him. Each of his words is most valuable, just one word
like `dhamma'. There is nothing except dhamma. Each word is very deep
and subtle. Whatever is real and appears at the present moment through
one of the six doors, the sense-doors or the mind-door, is dhamma. She
explained:

``Right now there is just thinking about seeing. Who knows what seeing
is. It has already arisen and fallen away. Without conditions, nothing
can arise, like seeing. Without the eye base, seeing cannot arise. One
has read before that all dhammas are anattā, which means no self.''

We have heard many times those words said by Acharn, but actually one
should not forget that they are not Acharn's teaching, but the teaching
of the Buddha. He explained all the conditions for the dhammas that
arise. We have read about these in the textbooks, but we have not deeply
considered them when they occur in daily life now. That is why we
appreciate reminders of realities. It is true that when there is no
understanding at the moment of seeing, there is always an idea of ``I
see''. We may not expressively think that it is self that is seeing, but
the idea of self is still there. Ignorance cannot know it, but when a
moment of understanding arises, it knows when there is an idea of self
that is thinking or acting. It is not eradicated until one has reached
the first stage of enlightenment, the stage of the sotāpanna
(streamwinner). That is why Acharn reminded us that we should listen and
consider the Buddha's words ``on and on and on, until one becomes
enlightened.''

Enlightenment seems far away, it is actually perfect understanding.
However, there can be a beginning of understanding of the reality that
appears at this moment, be it sound, visible object or whatever appears
now. Paññā has not been sufficiently developed so as to distinguish the
reality that experiences and the reality that does not experience.
Seeing is nāma, it experiences visible object, and visible object is
rūpa, it does not experience anything but it can appear. Seeing could
not arise if there were no visible object and no eyesense which is
another rūpa. Nāma and rūpa have different characteristics and as long
as these are not distinguished from each other we take all that occurs
for self. When paññā is more developed it can realize the difference
between nāma and rūpa. This is not a matter of remembering that this is
nāma (a mental reality) and that is rūpa (a physical reality), but we
should consider the characteristic of the reality that appears now. When
hardness appears its characteristic can be gradually understood as a
reality that does not know anything, without having to think about it.
When seeing appears it can be understood as a reality that experiences
visible object.

After our sessions in Saigon, we went to Hoi An where we stayed in An
Villa, a resort with eight rooms adjacent to a swimming pool. It is very
suitable for Dhamma discussions. There is space for many people who come
to Dhamma sessions. They can be seated under a roof as a protection
against rain and sunshine. We were the first guests staying here and we
attended the inauguration ceremony. Photos were shown picturing how the
idea originated for a Dhamma home. A committee was formed by Tran Thai,
Tiny Tam, family members and friends to execute their plans. After the
ceremony, there was a party at the poolside and delicious springrolls
and other snacks were served. The service given by the attendants who
cleaned the place and took care of our rooms was with so much kindness
and thoughtfulness. Every day the breakfast menu was different and
prepared with the utmost care. Hang helped me every day with difficult
steps, and besides this she had early morning ocean swims and she baked
black bread for breakfast. Staying in An Villa was an unforgettable
experience to us since it really is like a family home. One always wants
to return there.

Every day we went out for luncheon to a different restaurant and all
these places had lush tropical gardens full of flowers, or they were
situated at the riverside. Vincent helped me with great patience as we
were walking along to these places, waiting to let the quick walkers
pass.

During the discussions, both in Saigon and in Hoi An, the Vietnamese
listeners were very keen to learn more and they posed many questions.
Someone asked: ``How can we figure out what is citta and how can we have
more kusala cittas? Which citta is kusala and which akusala? What is the
right Dhamma?''

People think of making merit by releasing animals at the Pagoda. It is
very difficult to know when the citta is kusala citta and when akusala
citta. Cittas arise and fall away very rapidly and kusala cittas
alternate with akusala cittas. Acharn would say: ``Only paññā can
know''. It cannot be known by paññā that is still weak. Usually we think
of what is seen, heard or experienced by the other sense-doors but the
kusala citta or akusala citta that follows is not known. They do not
appear until there are conditions to understand this moment directly.
Without listening to the Dhamma, it is not known that often our deeds
and speech are motivated by clinging to the idea of self. Acharn said:

``Even moving one's hands is motivated by akusala citta. How can this be
known without direct understanding? No one can force oneself to have
understanding of citta right now. When the moment of experiencing is not
known as not self, how can one know the exact moment of kusala citta and
of akusala citta? Do not work out: this is kusala, this is akusala. One
should rather know the reality which experiences and the reality which
is experienced.''

If the difference between nāma, such as seeing which experiences, and
rūpa such as visible object that is experienced, is not known, we are
bound to take kusala and akusala for self, but they are mere dhammas
that arise because of conditions and fall away instantly.

When I listen to a talk by Acharn, enthusiasm (pīti) and happy feeling
(somanassa) arise, and these can accompany true appreciation. But, also,
I like enthusiasm and happy feeling, and I cling to those very much. It
is impossible to disentangle all such different moments of kusala and
akusala that are alternating, arising and falling away so rapidly. It is
impossible as long as it is not known that whatever appears at the
present moment is just a dhamma.

We read in the ``Kindred Sayings'' (II, 7. The Great Chapter, § 61) that
the Buddha spoke about the rapidity cittas arise and fall away. We read:

\begin{quote}
``But this, brethren, that we call thought, that we call mind, that we
call consciousness, that arises as one thing, ceases as another, whether
by night or by day. Just as a monkey, brethren, faring through the
woods, through the great forest catches hold of a bough, letting it go
seizes another, even so that which we call thought, mind, consciousness,
that arises as one thing, ceases as another, both by night and by day.''
\end{quote}

Citta is called by different names such as mano (mind) or viññāṇa
(consciousness). Citta does not become another one when it ceases. This
is merely an expression to show how rapidly it arises and falls away.

People sometimes ask why seeing is mentioned first as object of
understanding. In the suttas we can also notice that seeing is mentioned
first. It is the most common reality and the most intricate reality.
Immediately after seeing what is visible, we cling to shape and form and
this leads us to thinking of concepts of the world and all the people in
it. We take them for lasting and for self. We should learn the
difference between seeing visible object and thinking of concepts such
as people and things. There are no people and things but thinking is
accumulated so that we take them for self. Life is just one moment of
experiencing an object.

We also discussed meditation. Someone had strange experiences, such as
not experiencing his physical body anymore. Some people were wishing for
peacefulness. They thought of peace as a state that is lasting for a
while. But each moment of citta passes away immediately. At one moment
there may be kusala citta, at another moment akusala citta. It is hard
to know the difference. Some cittas are cause, such as kusala citta and
akusala citta that can motivate deeds that bring result, and some cittas
are result such as seeing or hearing. They experience different objects
through the eyes, the ears and the other senses. We can easily mislead
ourselves as to the present reality. That is why Acharn reminds us often
of the characteristics of seeing, hearing, thinking and all other
realities. When we find out that the Dhamma can be verified by
considering and testing the truth, there will be more confidence.

Calm was discussed time and again during our discussions. Someone asked
what the conditions for calm are in samatha. Acharn said: ``What is
calm? Who can know it? It is not `I'. It is a reality arisen because of
conditions. It may arise with or without right understanding.''

People talk about calm but it is important to know what it is. It is
different from the conventional idea of calm. It is a cetasika arising
with every kusala citta\footnote{Calm is in Pali
passaddhi. It is actually two cetasikas: kāya-passaddhi, calm of the
mental body, the cetasikas, and citta-passaddhi, calm of citta.}.
Kusala citta and thus also calm can arise with understanding or without
it. Calm is a reality, not self, but one is bound to take it for self.
When the citta is kusala citta, there is calm of citta and cetasikas,
there is no restlessness nor agitation at that moment. There is no
infatuation with the object which is experienced. However, it is not
easy to recognize the characteristic of calm. The different types of
citta succeed one another very rapidly and shortly after the kusala
cittas have fallen away, akusala cittas tend to arise. Right
understanding has to be keen in order to know the characteristic of
calm. If there is no right understanding, we may take for calm what is
not calm but another reality. For example, when we are alone, in a quiet
place, we may think that there is calm while there is actually
attachment to silence.

There are likely to be misunderstandings about calm. Someone may think
that he is calm when he is free from worry, but this calm may not be
kusala at all. There may be citta rooted in attachment which thinks of
something else in order not to worry. Or people may do breathing
exercises in order to become relaxed. Calm of cetasikas and calm of
citta which are sobhana cetasikas are not the same as a feeling of
relaxation which is connected with attachment.

Those who see the danger of attachment to sense objects may want to
develop samatha in order to be removed from sense impressions. As was
explained many times during the discussions, for the development of
samatha, right understanding is needed that knows what wholesome calm is
and how calm can be developed with a suitable meditation subject.
However this understanding is different from right understanding that
knows the present reality as non-self. Those who develop samatha without
developing also right understanding of the eightfold Path can be
temporarily away from sense impressions but they still take calm for
self.

When right understanding of the eightfold Path is developed, calm also
develops and it is understood as non-self. Acharn asked whether there is
attachment to visible object now. One will not know this by thinking
about it; only paññā will really see the danger of attachment. When
there is ignorance of realities, there are conditions to be attached
again and again. Life is so short and, therefore, one should ask oneself
what is more valuable: suppressing clinging to sense objects or
developing understanding of realities which will eventually lead to the
eradication of ignorance and all that is unwholesome.

In the Buddha's time, people who had accumulated the inclination for
samatha would develop it, but after hearing the teachings they would
also develop insight, right understanding of realities. They learnt to
understand the present moment, be it akusala or calm, even to the degree
of jhāna, absorption, as a conditioned reality, non-self.

We read in the ``Kindred Sayings'' ( I, Ch I, The Devas, I, § 10, ``In
the Forest'') that at Sāvatthi, a deva asked the Buddha:

\begin{verse}
``Those who dwell deep in the forest,\\

Peaceful, leading the holy life,\\

Eating but a single meal a day:\\

Why is their complexion so serene?''\\

The Blessed One:\\

``They do not sorrow over the past,\\

Nor do they hanker for the future,\\

They maintain themselves with what is present:\\

Hence their complexion is so serene.''\\

``Through hankering for the future,\\

Through sorrowing over the past,\\

Fools dry up and wither away\\

Like a green reed cut down.''\\
\end{verse}

People spoke about benefits they believed there were in meditation. They
found that it helped them in the situation of their work, in the
relation with their family. They thought that they had more patience and
less aversion. They could stand cold and heat much better and they
thought that they had more compassion. However, one may still keep on
clinging to oneself. One may cling to patience and calm as states that
one can possess. Such moments do not stay. Life is only fleeting
moments. The Buddha's way to eradicate the wrong view of self and all
that is unwholesome is developing understanding life after life and one
cannot expect an immediate effect.

In the Tipiṭaka we read time and again that the Buddha spoke about the
impermanence of seeing, hearing and all other dhammas. Some people
think that this is also taught in other religions. We all know that
people do not live forever and that all things in life are subject to
change. The Buddha did not teach general ideas about impermanence, but
his teaching is very precise. He taught about the reality of this very
moment that arises and passes away. He taught for forty-five years about
what we did not know before. We take it for granted that we see at one
moment and then hear or think at other moments, but we are ignorant of
the arising and ceasing of one reality at a time. The Buddha taught that
the sense-cognitions and also thinking arise in processes of cittas that
arise and fall away, succeeding one another. Seeing is only real when it
arises, and this shows that what the Buddha taught is not theory, it
pertains to this very moment. What arises and falls away, never to
return, can that be anyone?

It is most helpful that Acharn reminds us all the time about this moment
of touching, this moment of thinking, the present moment. Direct
understanding of these realities can arise by conditions, not by
anyone's will. Otherwise there is wrong thinking, because there is the
idea of ``I'', ``I am doing'', ``I am concentrating on the object'', ``I
am being aware''. There is no ``I''. Everything is conditioned from
moment to moment. We forget all the time that it is not ``I'' who is
studying the Dhamma, not ``I'' who is listening, not ``I'' who learns
about realities.


\chapter[What is most valuable in life]{}
\section*{What is most valuable in life}

Sarah said: ``The development of satipaṭṭhāna can be now, no need to
wait. Because there is visible object now, there is seeing, hearing, no
need to wait, this moment.''

I remarked: ``Visible object is not the people here around.''

Sarah said: ``It is so helpful to remember that even at this moment
actually, when we think that we are with our friends we are so attached
to, it is only visible object which is seen now, sound that is heard.
There aren't any people, there aren't any friends seen. What is touched
is only hardness or softness, nothing special that is touched. It is
only thinking that thinks about the people in this room, the people
around the table. Just a moment of thinking that is completely gone.
Nothing of value at all in life, except the moments of understanding. We
mind so much about pleasant feeling, but it does not last an instant.
The visible object does not last, just here or there or in Holland, just
visible object that does not last, sound that does not last. Just the
same, no matter where or when, it just lasts an instant. So the world
becomes smaller and smaller, closer and closer, just what appears now,
just one citta at a time''.

Nina: ``Like Acharn always says to us: `There is no one there.' But when
we hear that, we find it a point we did not get yet.''

Sarah: ``A reminder: even now only sound is heard and then gone. That is
why considering leads to satipaṭṭhāna, to the development of
understanding, to the development of the Path. Even at the moment of
wise consideration, it is enough. One does not need to think about it
that it is so little or that it is not so clear. At that moment, for an
instant, it is clear: it is just sound heard, it is just visible object
seen. When there is doubt, just doubt, just another passing dhamma, not
of any consequence. Begin again and again.''

Jonothan: ``We can understand what is seen as just visible object and an
idea of people and things follows closely after. We do not need to tell
ourselves: `no one there, no one there', because it is not the reality,
the situation for us. That is trying to see something we are not capable
of experiencing. But what we can experience is what is there, and what
is there can be understood as it is. What is not there can't be
understood.''

Sukin: ``When it is said there is no one there, dhamma is there.''

It is true, dhamma is there. There are always dhammas arising and
falling away. There is no person or being, but there are citta, cetasika
and rūpa.

In the texts we read about many classifications of cittas, cetasikas and
rūpas, and the only purpose of this is helping people to have more
understanding of non-self. They are all elements that arise because of
their own conditions and fall away immediately.

People keep on thinking of their problems, but they do not understand
dhammas. They do not know that citta and cetasikas perform their
functions while there is thinking. When we are absorbed in thinking of
persons and situations, we forget that the real world is the world of
whatever reality appears through one of the sense-doors or the
mind-door. We believe that ``we'' see, or ``we'' hear, but in reality
there are only different cittas performing their functions. We live as
it were in the world of a magician who makes people believe in whatever
he does.

We read in the ``Kindred Sayings'' (III, Middle Fifty, 5, § 95,
Foam\footnote{Translated by Bikkhu
Bodhi in ``The Connected Discourses of the Buddha''.}) that the Buddha
said:

\begin{quote}
``Suppose, bhikkhus, that a magician or a magician's apprentice would
display a magical illusion at a crossroads. A man with good sight would
inspect it, ponder it, and carefully investigate it, and it would appear
to him to be void, hollow, insubstantial. For what substance could there
be in a magical illusion? So too, bhikkhus, whatever kind of
consciousness there is\ldots{} it would appear to him to be
void, hollow, insubstantial. For what substance could there be in
consciousness?''
\end{quote}

The commentary to this sutta states:

\begin{quote}
``Consciousness is like a magical illusion (māyā) in the sense that it
is insubstantial and cannot be grasped. Consciousness is even more
transient and fleeting than a magical illusion. For it gives the
impression that a person comes and goes, stands and sits, with the same
mind, but the mind is different in each of these activities.
Consciousness deceives the multitude like a magical illusion.''
\end{quote}

We may at times become discouraged, realizing that understanding is so
little. But even one moment of understanding is valuable. It is
accumulated so that it can arise again. In this way understanding grows.
As Sarah said: ``One does not need to think about it that it is so
little or that it is not so clear. At that moment, for an instant, it is
clear.''

I used to think, ``I have so little understanding and so much
ignorance''. It seems very humble but we have to realize that this can
be conceit (māna), we may be clinging to the importance of self. Sarah
reminded me: ``It is thinking about oneself, rather than considering
dhamma as āyatana (sensefield) or khandha, as different realities
conditioned at each moment.''

We know that we should learn that whatever is experienced are only
conditioned dhammas. Meanwhile we get involved in long stories about
people and things which seem to stay. With lobha, dosa, moha and wrong
view. But we should remember: it does not matter what arises, it is
conditioned already. This is very important. We cannot and do not try to
change it. That is the way to learn what dhamma is. It takes a long
time.

The thinking, mostly akusala, arises, and we should learn to see it as
only a conditioned dhamma. We take thinking for ``my thinking''. But
even such inclination can be known as a dhamma.

Some people want to follow a method and do everything slowly in order to
have more mindfulness. Acharn would always help people to return to the
present moment, saying:

``Understand what appears right now. At the moment of hearing, there is
no seeing. They arise because of different conditions. In a day there
are only different realities arising and falling away. There is touching
now, hardness appears, not what is seen. Just begin: that which is seen
is a reality, but nobody in it. Begin to understand whatever appears
little by little. Just hearing without considering is not pariyatti.''

Acharn reminded us very often, saying: ``Be truthful to the truth.'' We
should carefully consider these words. The truth of life is seeing that
appears now, or attachment, unhappy feeling, conceit, whatever appears
now. We would rather that akusala realities do not arise, but they are
the truth. To be truthful to the truth means developing understanding of
whatever appears in order to know it as just a conditioned dhamma, not
``I''. Actually, when we are truthful it does not matter what arises
since it is conditioned and nobody can change it.

We read in the ``Kindred Sayings'' (III, The Middle Fifty, § 59, The
Characteristic of
Nonself)\footnote{Translated by Bhikkhu
Bodhi in ``The Connected Discourses of the Buddha''.} :

\begin{quote}
Thus have I heard. On one occasion the Blessed One was dwelling at
Bārāṇasi in the Deer Park at Isipatana. There the Blessed One addressed
the bhikkhus of the group of five thus: ``Bhikkhus!''

``Venerable sir!'' those bhikkhus replied. The Blessed One said this:

``Bhikkhus, form\footnote{Rūpa, physical
phenomena.}  is
nonself. For if, bhikkhus, form were self, this form would not lead to
affliction, and it would be possible to have it of form: `Let my form be
thus; let my form not be thus.' But because form is nonself, form leads
to affliction, and it is not possible to have it of form: `Let my form
be thus; let my form not be thus.'

``Feeling is nonself\ldots{} Perception is nonself\ldots{} Volitional
formations are nonself\ldots{} Consciousness is nonself. For if,
bhikkhus, consciousness were self, this consciousness would not lead to
affliction, and it would be possible to have it of consciousness: `Let
my consciousness be thus; let my consciousness not be thus.' But because
consciousness is nonself, consciousness leads to affliction, and it is
not possible to have it of consciousness: `Let my consciousness be
thus; let my consciousness not be thus.' ''
\end{quote}

Nobody can alter the five khandhas, all conditioned realities, and have
them according to his wish.

One of our friends who attended the sessions in An Villa, had just heard
that her mother passed away. She continued to listen to the Dhamma but
the next day she had to leave earlier to go home. She was reading one of
my travel reports ``The Cycle of Birth and Death'' I had written just
after my husband's passing away. At that time I began to understand more
that the world of ultimate realities, citta, cetasika and rūpa and the
world of ideas and concepts are quite different. We cling to dear people
but actually what we take for a person are only passing realities. The
company of dear people gives us pleasure and when they have passed away
we lament our lack of this pleasure.

We dwell in our thoughts for a long time on the loss we suffered; many
moments of citta think with sadness but these are all gone immediately.
There are only conditioned dhammas arising and falling away.

Life is very short and we should know what is most important: developing
understanding of the reality appearing now. Acharn reminded us:

``Most important is to understand this moment. Otherwise it is gone,
gone without understanding. Each moment in life is gone, by conditions.
Without conditions, nothing can arise. Seeing cannot arise without
eyesense. Just learn in the beginning to understand the reality,
understand that it is conditioned. By how many conditions, in order to
know that it can never be yours. It is only reality that is conditioned
to arise, no one can stop its arising. Like now, no one can stop the
arising of seeing. So there is no you. Learn to understand each moment
as not me.''

When we learn about the different processes of citta that succeed one
another extremely rapidly it becomes clearer that each moment is
``gone'' very rapidly. We have read about processes of citta in the
textbook but it is useful to consider them again and again. Seeing
arises in a process of cittas and all of them experience visible object.
It is preceded by eye-door adverting-consciousness,
cakkhu-dvarāvajjana-citta, which just adverts to visible object that has
impinged on the eye-door. It is followed by seeing which sees visible
object, a vipākacitta, the result of kamma. This is followed by two more
vipākacittas, the receiving-consciousness, sampaṭicchana-citta, that
does not see but still experiences visible object while receiving it,
and the investigating-consciousness, santīraṇa-citta. Then the
determining-consciousness, the votthapana-citta, arises, an ahetuka
kiriyacitta (inoperative consciousness without roots) that
``determines'' the object. It is only one moment of citta that will be
followed by seven kusala cittas or akusala cittas performing the
function of javana, ``going through the object''. There is no self that
determines whether there will be kusala cittas or akusala cittas, this
depends entirely on accumulated conditions. Very often akusala cittas
with ignorance and attachment arise. The sense-door process is followed
by a mind-door process of cittas after there have been bhavanga-cittas
(life-continuum) in
between\footnote{They do not experience
an object through one of the six doorways, and they do not arise within
processes of cittas. They are vipākacittas and they just keep the
continuity in a life. They experience the same object as the
rebirth-consciousness.}. Also in the
mind-door process kusala cittas or akusala cittas arise.

When sound appears, hearing arises in another process of cittas, in the
ear-door process. It seems that sound can be experienced at the same
time as visible object, but the eye-door process must have fallen away
when sound is heard. This shows that processes of cittas follow upon
each other extremely rapidly.

Cittas succeed one another without interval. The next citta cannot arise
if the preceding citta has not fallen away. Because of the unbroken
continuity of cittas, past lives condition the present life and the
present life conditions future lives. Kusala cittas and akusala cittas
fall away immediately but the good and bad qualities are accumulated
from moment to moment, from life to life, and, thus, there are
conditions for the arising again of kusala citta and akusala citta.
There is no self who can cause the arising of kusala citta and akusala
citta.

Cittas arising in processes arise in a certain order and nobody can
change this order. For example, seeing that arises has to be followed by
two more vipākacittas that also experience visible object; these cittas
cannot experience sound or any other object.

The processes of cittas experiencing an object through the eyes, the
ears, the nose, the tongue, the bodysense and the mind-door proceed
extremely rapidly from birth to death, and very often the objects are
experienced with ignorance and attachment. Thus, countless moments are
``gone'' with ignorance and in this way we keep on accumulating
ignorance. Before we realize it, our life has come to an end.
Considering the way processes of cittas proceed all the time can be a
condition to develop understanding of the present reality. This does not
mean that one has to act in a specific way in order to know the present
object; then one is taken in again by the idea of self. Acharn explained
that people just want to keep the ``I'', not to abandon the ``I''.

I am very grateful for all the explanations and reminders given by
Acharn, Sarah, Jonothan and other friends. I found the questions posed
by our Vietnamese friends very helpful, they made me consider more the
realities of daily life.

Listening and reading alone are not sufficient, one should consider
deeply the truth. Acharn said that it is a long way and that it is to be
followed with detachment. As she reminded us time and again: most
important is understanding this moment. This is the beginning of
understanding the truth of life.

Understanding, paññā, can be classified in many ways. The following way
demonstrates the different levels of understanding:

\begin{itemize}
\item
  understanding based on hearing, suta-maya-paññā,
\item
  understanding based on considering, cinta-maya-paññā,
\item
  understanding based on mental development, bhāvana-maya-paññā.
\end{itemize}

This was one of our subjects of discussion. The Pali term ``suta'' means
what is heard; we have to listen again and again, a few times is not
sufficient. The words ``everything is dhamma'' that we often hear sound
very simple, but we really have to consider the truth. Listening can
condition considering the truth so that there will be more
understanding. Realities can be considered in daily life, no matter
where one is. One should not think that being with many people or
performing one's daily tasks is a hindrance to considering realities.
There is no self who considers the truth, it occurs by conditions.

Dhamma is what appears now. The reality that appears cannot be selected.
It may be unwelcome, such as strong attachment, great sadness or worry.
It does not matter what appears, it is non-self, uncontrollable. We
should not forget the words ``it does not matter what appears''. It is
not sufficient to merely think that it is not self. Considering over and
over again can be a condition for paññā based on mental development
(bhāvana-mayā-paññā), the development of paññā which understands
directly the characteristics of realities that appear. At first
understanding is not yet firm enough to see them as mere conditioned
dhammas. Only the direct understanding of realities can gradually
lessen the inclination to take realities for self. Finally it will lead
to the eradication of the wrong view of self.



