

\part{Pilgrimage in Sri Lanka}


\chapter{Introduction}

``Buddhism in Daily life'' was the theme of a Buddhist seminar held in
Sri Lanka. Do we really practise the Buddha's teachings in our daily
life? Aren't we often forgetful of them? When we are impatient, where
are the loving kindness (mettā) and compassion (karuṇā) the Buddha
taught? In theory we know about the different ways of wholesomeness he
taught. We think that we have understood how to cultivate wholesome
deeds, wholesome speech and wholesome thoughts, but most of the time we
are forgetful of wholesomeness.

A schoolteacher in Sri Lanka told me that he does not teach the children
in a theoretical way, but that he teaches them how to apply immediately
what they have learnt. I felt like a child who has been taught how to
apply the Dhamma, the Buddha's teachings, in the different situations of
life. I found out that I overlooked many things which are taught in the
suttas, such as kindness, gentle speech, speech at the righ time,
patience and many other ways of wholesomeness. We think that we have
understood the Dhamma, but we have not really understood it. It was most
helpful to be reminded of the practice of the Dhamma and to discuss the
Dhamma with many new friends we made in Sri Lanka.

I was reminded to live in the present moment, not in the past or the
future, and to ``study'' the present moment with mindfulness. ``If there
is no study of the present moment, right understanding (paññā) cannot
grow'', Khun Sujin reminded us every day.

In the past, satipaṭṭhāna
\footnote{The development of mindfulness and
understanding of all mental phenomena and physical phenomena of our life
as they appear through the senses and the mind.}
was taught and widely practised in Sri Lanka by monks, nuns and
layfollowers. Countless people in Sri Lanka attained arahatship. They
attained because they were mindful of any reality appearing at the
present moment through eyes, ears, nose, tongue, bodysense and
mind-door.

Captain Perera of the Buddhist Information Center in Colombo organised a
five week seminar of Dhamma discussions which were held in Colombo,
Anu-rādhapura and Kandy. Ms. Sujin (Acharn Sujin) and Ms. Duangduen had
come from Thailand, Sarah from England and I from Holland. We all met in
Sri Lanka on the occasion of this seminar. The venerable Bhikkhu
Dhamma-dharo and the venerable Bhikkhu Jetananda had come from Thailand
several months ago and Samanera Sundara arrived at the same time as
Acharn Sujin.

The seminar was opened in Colombo by the venerable ``Mahā Nayaka'' (the
chief monk) with the traditional lighting of the oillamp. The sessions
were held nightly in the form of discussions. During the day we met our
Singhalese friends in their homes and discussed Dhamma in a more
personal way. All through those five weeks we spent in Sri Lanka we
enjoyed the wonderful hospitality of the Singhalese while we stayed as
guests in their houses. They gave us every day delicious curry luncheons
and dinners, there was no end to their generosity. Captain Perera looked
after us throughout our stay and when we had problems with visas or
other matters he just smiled and said, ``All wounds get healed.''

The Buddha visited Sri Lanka three times and during these visits he went
to sixteen different places. Relics of the Buddha have been enshrined in
several stupas (dāgabas) and a sapling of the original Bodhi Tree in
Gaya was brought over in olden times. It was planted in Anurādhapura
where it is still growing today. A new sprout developed recently from
this tree. Is this not a hopeful sign that the Dhamma is still
flourishing in Sri Lanka?

I became interested in the history of Sri Lanka and started to read the
``Mahāvaṃsa'', an old chronicle, compiled at the end of the fifth or the
beginning of the sixth century A.D. After the third Council, which was
held in India during the reign of King Asoka (250 B.C.), missionaries
were sent out to different countries
\footnote{At the first Council held shortly
after the Buddha's passing away, the texts of the Vinaya, the Suttanta
and the Abhidhamma were recited by five hundred arahats. The second
Council was held one century later dealing with the wrong interpretation
of the rules of the Vinaya by heretical monks. At the third Council,
held 268 B.C. the canon of the texts in the Pali language was finally
redacted.}.

The arahat Mahinda, King Asoka's son, was sent to Sri Lanka together
with four other monks, a samanera (novice) and a lay-disciple. They went
to Mahintale where they met the Singhalese King Devānampiya Tissa while
he was hunting deer. The King laid aside his bow and after Mahinda had
tested him on his readiness to hear the Dhamma he preached to him the
``Lesser Discourse on the Simile of the Elephant's Foot Print'' (Middle
Length Sayings I, no 27). This sutta describes the life of a bhikkhu who
abstains from ill deeds through body, speech and mind, who ``guards the
six doors'' through mindfulness, develops jhāna
(absorption-concentration) and finally attains arahatship.

The following day Mahinda and the other monks went to Anurādhapura where
the King presented Mahinda with the royal park. This place became the
``Mahā Vihāra'' (Great Monastery), a famous center of Buddhism. The
monastery of Cetiyapabbata and many other monasteries were established
as well.

Mahinda had brought the ``Tipiṭaka'' and the commentaries to Sri Lanka
and these were translated into Singhalese. Many Singhalese wanted to
lead the ``homeless life'' and were ordained monks. Women wished to
become bhikkhunīs, nuns, and bhikkhunī Saṅghamittā, Mahinda's sister,
came to Sri Lanka in order to ordain bhikkhunīs. She brought the sapling
of the Bodhi tree from India to Sri Lanka. During the reign of King
Devānampiya Tissa the ``Thupārāma Dāgaba'', the oldest stupa in Sri
Lanka, was also constructed and in this stupa the relic of the Buddha's
right collarbone was enshrined.

The Buddhist teachings declined in India, but they were preserved in Sri
Lanka. However, when one studies the history of Sri Lanka one sees how
difficult it must have been to preserve them. Invading kings and also
local kings who did not support the Sangha threatened the survival of
the teachings.

After an invasion by Tamils, King Dutthagāmanī (about 150 B.C.) restored
the position of the Sangha and started to build the ``Ruvanvelisāya'',
the great and famous stupa of Anurādhapura, which contains relics of the
Buddha and which is together with the Bodhi Tree the center of worship
in Anurādhapura up to today.

Not only wars, also famines have threatened the survival of the
teachings which were not yet committed to writing. Many people died
during those famines and the arahats who survived on roots and fruits
continued to recite the teachings with heroic fortitude. When they had
no more strength to sit up, they continued reciting while lying down.

Wars, famines and also the introduction of wrong beliefs and wrong
practice made it difficult to preserve the teachings. Finally, in 89
B.C., the teachings were committed to writing. Five hundred monks
undertook this great enterprise in the cave of Aluvihāra (Alulena) which
we visited during our pilgrimage.

Several centuries later (410 A.D.) Buddhaghosa Thera came from India to
Sri Lanka. Here he composed his famous ``Path of Purification''
(Visuddhimagga). He edited all the commentarial material he found in Sri
Lanka and translated these commentaries from Singhalese into Pāli. The
commentaries to the Vinaya, to most of the Suttanta and to the
Abhidhamma were translated and edited by Buddhaghosa. The
``Atthasālinī'' (Expositor) is the commentary to the first book of the
Abhidhamma, the Dhammasangaṇi. Sri Lanka, where the Tipiṭaka and the
commentaries were preserved, is an inspiring country to visit in order
to recollect the Buddha, the Dhamma and the Sangha. The fact that
numerous arahats lived in this country proves that the Dhamma was truly
practised in daily life.

Despite a decline of Buddhism and even persecution, the Singhalese have
maintained many wholesome traditions which were originated in the olden
times, such as the presenting of dāna to the monks, the celebration of
``Uposatha Day''
\footnote{Fasting day, kept on the days of
full moon, new moon and sometimes on the days of the first and last
moon-quarter.}
and many other ceremonies. The Singhalese of today see the relevance of
the Buddhist teachings in their daily life. Numerous books on the Dhamma
written by learned bhikkhus and layfollowers and also a Buddhist
Encyclopedia are being published today. Many Singhalese are well versed
in Pāli and they are able to chant texts from the teachings. Our hostess
in Colombo would spend the evening of Uposatha day in her shrineroom,
chanting in Pali the ``Satipaṭṭhāna Sutta''
\footnote{The sutta about the four
Applications of Mindfulness: of body (all physical phenomena), of
feelings, of cittas and of dhammas. These include all physical phenomena
and mental phenomena that can be objects of mindfulness and right
understanding.}
and other parts of the scriptures. One of our hosts who had invited us
to luncheon recited in the car the ``Karanīya Mettā Sutta'', the sutta
about the development of loving kindness, while his wife was driving. We
noticed that people did not only think about mettā but that they also
practised mettā. Their mettā appears in their generosity and their
thoughtfulness for the guests they receive into their homes.

Shortly after our arrival in Sri Lanka it was ``Uposatha Day'' (Poya
Day). We saw many people clothed in white who observed eight precepts
\footnote{In addition to the five precepts
there are others, such as refraining from eating after midday, from
lying on soft beds, from entertainments and from adornments.}.
Even small children observed these until six at night. We were taken out
to the Kelaniya temple which was the focul point of the Buddha's second
visit to Sri Lanka. Near the temple is a stupa in which relics of the
Buddha have been enshrined and there is also a Bodhi Tree. We heard the
sound of drums and all around on the temple grounds people were sitting
in small groups, reciting the `` Satipaṭṭhāna Sutta'' and other texts.
Oillamps were lit, incense was burnt and flowers were offered.

The abbot of the temple explained to us that people in Sri Lanka ,
before of- fering flowers, take off all the green parts. They do not
keep them in water but let them dry out. It is the course of nature that
flowers have to wither. Elderly people are not afraid of ageing and
death because they realize that they cannot escape from them, just as
flowers cannot avoid withering.

The stanza which is recited in Sri Lanka when one offers flowers is a
beautiful reminder of impermanence. Our host who took us around on that
day chanted it for us:

\begin{verse}

``With diverse flowers, the Buddha I adore;\\
And through this merit may there be release.\\
Even as these flowers must fade,\\
So does my body march to a state of destruction.''

\end{verse}

I found the discussions during the seminar very useful. We spoke about
the many kinds of kusala the Buddha taught. Dāna (generosity), sīla
(morality) and bhāvanā (mental development) can be practised in daily
life. We read in the ``Sigalovāda Sutta'' (Dialogues III, no 31
\footnote{I am using the translation by Ven.
Narada, Wheel Publication 14, B.P.S. Kandy, Sri Lanka.}
) that the Buddha, when he was staying in the Bamboo Wood near Rājagaha,
at the Squirrels' Feedingground, spoke to Sigāla about good qualities to
be developed in daily life. We read, for example, that the Buddha said
to him:

\begin{verse}

``Who is wise and virtuous,\\
Gentle and keen-witted,\\
Humble and amenable,\\
Such a one to honour may attain.\\
Who is energetic and not indolent,\\
In misfortune unshaken,\\
Flawless in manner and intelligent,\\
Such a one to honour may attain.\\
Who is hospitable and friendly,\\
Liberal and unselfish,\\
A guide, an instructor, a leader,\\
Such a one to honour may attain.\\
Generosity, sweet speech,\\
Helpfulness to others,\\
Impartiality to all,\\
As the case demands.\\
These four winning ways make the world go round,\\
As the lynchpin in a moving car.\\
If these in the world exist not,\\
Neither mother nor father will receive,\\
Respect and honour from their children.\\
Since these four winning ways\\
The wise appraise in every way;\\
To eminence they attain,\\
And praise they rightly gain.''

\end{verse}

When we read these words of advice they may seem simple to us, but how
difficult it is to follow them all the time. There are more conditions
for unwholesome moments of consciousness (akusala cittas) than for
wholesome moments of consciousness (kusala cittas) in a day
\footnote{What we take for `our mind' are
different moments of consciousness, citta, which arise and then fall
away immediately, succeeding one another. Cittas can be kusala, akusala,
vipāka (result of kamma), or kiriya (neither cause, namely kusala or
akusala, nor result).}.
The more one sees one's lack of kusala, the more one realizes that it is
important to know oneself, to know precisely the different moments of
consciousness which arise.

Kusala citta and akusala citta arise because of their appropriate
conditions and nobody can cause the arising of kusala at will.
Understanding can be de- veloped so that they can be seen as they are,
as non-self. This understanding will eventually lead to the elimination
of akusala.



\chapter[Kusala and Akusal]{}
\section*{Kusala and Akusala}

The Buddha taught many different ways of kusala and one of these ways is
generosity (dāna). We should cultivate generosity, but do we know when
there is true generosity?

Generosity does not last. There is no abiding mind, no self who is
generous. There are only fleeting moments of consciousness which change
all the time. Citta, a moment of consciousness, arises and falls away
immediately, and then it is succeeded by the next citta. Many different
types of citta arise and fall away, succeeding one another. Generosity
arises with kusala citta, wholesome consciousness, and this does not
stay; it falls away immediately, to be succeeded by the next citta.
Akusala citta, unwholesome consciousness, may follow shortly after the
kusala citta, but we do not notice this. Akusala citta cannot arise at
the same time as kusala citta, because only one citta can arise at a
time. Attachment or clinging, which is unwholesome, and generosity
cannot arise at the same time, but attachment may follow shortly after
generosity has fallen away.

There is very little generosity in a day. From the time we are waking up
until we go to sleep we are trying to obtain things for ourselves. How
few are the moments we are giving things away instead of trying to
obtain them for ourselves. Do we know exactly at which moment there is
generosity? We may take for generosity what is actually attachment. Do
we know when attachment arises to the person who receives our gift,
attachment to the thing we give, attachment to our wholesome deed? We
cling to the pleasant feeling we derive from giving and we do not even
notice that there is clinging. We may cling to an idea of ``my giving'',
we take kusala for ``self''.

Many more moments of attachment arise than we could imagine. We may
think that attachment arises only when we want to possess things, when
we are greedy. But there are many forms of attachment, some of which are
gross and some more subtle. Don't we very often, after we have seen
something, cling to what we have seen? Do we cling to seeing or to our
eyes? We would not want to part with an eye or lose the ability to see.
That shows that there is attachment. Attachment is bound to arise after
we have seen, heard, smelt, tasted, and experienced objects through the
bodysense, and also when we experience objects through the mind-door.

One may wonder what the term ``door'' means. A door is the means through
which citta experiences an object. Seeing experiences visible object
through the eye-door. The eye-door is the eyesense, a physical
phenomenon, rūpa, which is capable of receiving visible object. Eyesense
itself does not see but it is a condition for seeing. There are six
doors of eyes, ears, nose, tongue, bodysense and mind-door. There is no
self who experiences objects through these doorways. There are only
different cittas, succeeding one another, which experience an object
through one of the six doors.

Attachment, aversion and ignorance can arise on account of what is
experienced through each of the six doors. Often we take akusala citta
for kusala citta. For example, I was helping someone to get buckets of
water for an old lady. While I was helping I talked about the Dhamma,
but many moments of attachment arose to ``my kusala''. When people in
the temple wished me well and showed their appreciation of kusala, I
appreciated their generosity, but I was immediately attached to these
kind people and to ``my kusala''. Since different cittas succeed one
another so rapidly, it is extremely difficult to know precisely when the
citta is kusala and when akusala. It is the function of paññā, wisdom,
to know this. We are so ignorant, and ignorance covers up the truth.
When the citta is kusala, there is no attachment, no aversion and no
ignorance.

When we come to know ourselves more, we learn that even kusala such as
dāna can condition attachment. We can come to know when we cling to a
pleasant result of our good deed, such as a happy rebirth. Or we may
realize when conceit arises about our good deed: we may think ourselves
better than other people. One has to develop right understanding in
order to know the difference between kusala and akusala.

Right understanding or wisdom does not always accompany kusala citta.
For example, one may help others because it is one's nature to do so,
without there being right understanding with the kusala citta. One may
not know precisely when kusala citta arises and when akusala citta.
Someone may offer food to the monks or offer flowers in the temple
because these are good traditions he was taught to observe, but the
kusala citta may not be accompanied by right understanding. Kusala citta
does not stay. It falls away and then akusala citta is bound to arise.
It is difficult to know this without right understanding of kusala and
akusala. Someone may think that kusala cittas arise all the time when he
is in the temple or when he is helping others, but in reality many
moments of akusala cittas arise without our knowing it.

During the sessions we spoke about mettā, loving kindness, and karuṇa,
compassion. We may think that there is pure loving kindness while there
are actually many moments of attachment to people. Are we sure when
there is true compassion? We may take for compassion what is aversion.
For example, when we see someone kicking a dog, aversion is bound to
arise. When true compassion arises, there cannot be aversion at the same
time. The kusala citta with compassion is without attachment and without
aversion.

Venerable Dhammadharo said that it is a ``healthy shock'' to see that
akusala citta arises more often than kusala citta. More knowledge of the
truth about ourselves shakes us up and it reminds us to develop right
understanding in order to know more precisely when kusala citta arises
and when akusala citta.

Sīla, morality, is another way of kusala the Buddha taught. Abstaining
from ill deeds through body, speech and mind is kusala sīla. Paying
respect to those who deserve respect and helping others are included in
sīla as well. Especially during the sessions in Anurādhapura people
asked many questions concerning the practice of sīla. Someone who had a
military profession asked whether it is akusala to follow up the order
to kill. Acharn Sujin asked him, ``Did you want to kill, or did you have
to kill?'' There is a difference here. Killing is akusala kamma, an
unwholesome deed, but akusala kamma has many degrees. When one wishes
wholeheartedly to kill, the degree of akusala is higher than when one
follows up orders.

Those who have not attained enlightenment, should not believe that they
will never neglect the five precepts. The tendencies to all kinds of
akusala are latent in us and when an opportunity presents itself, we may
commit akusala kamma. Someone may for a long time not be in a situation
to kill, but when he is in very difficult circumstances, does he know
for sure that he will not kill? One may, for instance, kill insects
because guests are coming to one's house.

A police officer asked whether he could do his duties with kusala citta.
Acharn Sujin said that in his profession there are many opportunities
for helping: helping to keep order, helping people who are in trouble. A
judge asked whether one commits akusala kamma when one has to condemn
someone to death. One has to follow the law. While someone signs the
verdict he commits not necessarily akusala kamma, but he is likely to
have akusala citta at such a moment.

One afternoon the judge and his family had come to meet venerable
Dhammadharao while we were sitting under a tree in the area of the
``Mahā Vihara'', the Great Monastery, which is between the
``Ruwanvelisāya'', the Great Stupa, and the Bodhi-Tree. We found this
place where the Dhamma was discussed in olden times very suitable for a
conversation about the Dhamma. Venerable Dhammadharo spoke about the
danger of ambitions in life. They may cause the arising of many akusala
cittas and even akusala kamma, such as telling a lie in order to attain
one's goal. The receiving of pleasant objects such as honour and esteem
are the result of kusala kamma; they can never be the result of akusala
kamma. Without right understanding we do not know when kusala citta
arises and when akusala citta, and we do not know how to develop kusala.
Thus, we are enslaved to our many defilements.

The judge gave some money to a poor woman who came around to our group.
Acharn Sujin said: ''This moment of giving is conditioned. If there were
no conditions for giving there could not be any giving.'' It is useful
to be reminded that there is no ``self'' who gives, that there is no
person in the giving. At the moment of generosity there is only a citta
that arises because of conditions. Giving in the past is a condition for
giving today. The citta that is generous arises and then falls away, it
does not stay. However, that moment of generosity is a condition for
generosity again, later on. Since each citta conditions the succeeding
one, good and bad tendencies can be carried on from moment to moment,
from life to life.

Abstaining from wrong speech is a form of sīla. We understand this in
theory, but do we remember it in our daily life, when we are about to
say something unpleasant? For example, someone may suggest a plan to us
which does not conform to our wishes. Are we impatient and do we say
straight away that we do not like his plan, or are we patient and do we
abstain from unpleasant speech out of consideration for his feelings? We
may know that when we shout there is wrong speech; that is very obvious.
But do we realize that there is also wrong speech when we speak with
lack of consideration for someone else's feelings, even though we do not
shout? Showing one's dislike through speech is speech motivated by
aversion. How can that be right speech? Even not saying anything, but
keeping quiet with aversion when we do not agree with someone else is
not kusala citta abstaining from wrong speech.

In the suttas we read about gentle speech. For example, in the ``Lesser
Simile of the Elephant's Footprint'', the sutta Mahinda preached to King
Devanampiya Tissa, we read about gentle speech:

\begin{quote}
``\ldots{} Abandoning harsh speech, he is one who abstains from harsh
speech. Whatever speech is gentle, pleasing to the ear, affectionate,
going to the heart, urbane, pleasant to manyfolk, agreeable to the
manyfolk- he comes to be one who utters speech like this\ldots{}''
\end{quote}

Venerable Dhammadharo told me about an event which I find an excellent
reminder to be patient in one's speech. One night the bhikkhus had no
microphone during the Dhamma session and whenever they wanted to speak
they had to wait for the microphone being handed over to them. They all
found this waiting very helpful. If one speaks straight away one may
speak with akusala citta when one does not agree with someone else's
words. If one has to wait one has time to collect oneself. How difficult
it is to always speak with kusala citta. Even when the topic is Dhamma
one may have attachment to one's own words and ideas, one may be proud
of one's knowledge, or one may have aversion towards what others say.
When akusala citta motivates our speech, we cannot be of great help to
others, even when the topic is Dhamma. Thus we see that right
understanding of our different cittas is most helpful for the
development of kusala.



\chapter[Tranquil Meditation]{}
\section*{Tranquil Meditation}

Dāna and sīla can be performed without right understanding or with right
understanding. When they are performed with right understanding they are
of a higher degree of kusala. Bhāvanā, mental development, is another
way of kusala, but mental development is not possible without right
understanding.

There are two kinds of mental development: samatha bhāvanā or tranquil
meditation, and vipassanā bhāvanā or the development of insight. For
both forms of mental development right understanding is indispensable,
but the right understanding in samatha is different from the right
understanding in vipassanā. Samatha and vipassanā have different aims
and their ways of de-velopment are different. The aim of samatha is
calm. In samatha defilements are temporarily subdued, but they cannot be
eradicated.

Samatha is a way of cultivating kusala citta. Those who see the
disadvantage of akusala want to develop more conditions for kusala.
There are not always opportunities for dāna and sīla, but if one has
understood how to develop samatha, there are conditions for calm, even
in one's daily life.

What is calm? Is it enjoyment of nature, listening to the bird's song,
being in quiet surroundings? What we in conventional language call
``calm'' is not the same as the calm that is developed in samatha. The
calm that is developed in samatha has to be wholesome; samatha is a way
of mental development. When attachment arises, there is no calm. One may
have attachment to silence and if right understanding is not developed,
one is likely to take for wholesome calm what is not really wholesome
calm. One may think, when there is neither pleasant feeling nor
unpleasant feeling, but indifferent feeling, that there must be calm.
Indifferent feeling can arise with kusala citta, but also with akusala
citta. It can arise with the citta that is rooted in attachment
(lobha-mūlacitta) and it arises always with the citta that is rooted in
ignorance (moha-mūlacitta). Since it is extremely difficult to know
exactly when the citta is kusala and when it is akusala, a fine
discrimination of one's cittas is necessary for the development of
samatha. Thus, we see that right understanding is indispensable.

Calm arises with every kusala citta. When we are generous or observe
sīla, we are free from attachment (lobha), aversion (dosa) and ignorance
(moha), and that is calm. If someone has right understanding of the
characteristic of calm there can be conditions for more calm and, thus,
calm can develop. The understanding that is needed in samatha is not
merely theoretical understanding. One has to know the characteristic of
calm when it appears and one has to know precisely when the citta is
kusala and when it is akusala.

During the sessions we discussed many times the word ``meditation''.
This word is misleading. Generally people think that sitting in a quiet
place and trying very hard to concentrate is tranquil meditation or
samatha. One may try very hard to concentrate, but which types of cittas
arise at such moments? Does one concentrate with aversion, because
concentration is hard to achieve? Does one concentrate with attachment
and with ignorance? Wrong view may arise when one thinks of ``my
concentration''.

We should remember that concentration or ``one-pointedness'' (ekaggatā
cetasika) arises with every citta. Its function is to focus on one
object. When seeing arises, there is concentration on visible object.
When aversion arises, there is concentration on the object of aversion.
When someone performs dāna or observes sīla, there is concentration on
the objects or dāna or sīla. When someone develops samatha, there is
concentration on the subject of samatha. Right understanding of the
meditation subject of samatha can be a condition for more calm, and then
there will be concentration which is kusala. There will be a higher
degree of concentration, without the need to strive for concentration.
If someone strives for concentration he is bound to have attachment or
aver-sion. If a person is able to develop samatha this is due to
conditions.

Calm has many degrees. In the Buddha's time many people had conditions
for the attainment of jhāna, absorption concentration. At the moment of
jhānacitta sense-impressions do not arise, and attachment, aversion and
ignorance are temporarily subdued.

Can calm arise in daily life? When someone does not lead a secluded life
and he does not have accumulated skill for the attainment of jhāna, he
can still have moments of calm in daily life. The ``Visuddhimagga'' (Ch
IV-XII) describes forty meditation subjects of samatha. It depends on
the inclinations of the individual which of these subjects can be a
condition for calm.

The contemplation of a corpse, which is among the subjects of of
samatha, can for some people be a condition for aversion. But if one
thinks of this subject with right understanding there can be conditions
for kusala citta with calm. We may realize that our body now is not
different from a corpse: it consists of rūpas, physical phenomena, which
do not know anything and which are impermanent.

Mindfulness of breath is another subject among the forty meditation
subjects (kammaṭṭhāna). The ``Visuddhimagga'' explains that this subject
is extremely difficult, one of the most difficult subjects. One should
have right understanding of breath, otherwise calm cannot arise. What we
call breath is rūpa which is conditioned by citta. Bodily phenomena can
be conditioned by kamma, by citta, by temperature or by nutrition.

We cling to life, to our body, to our possessions. However, our life
depends on breath, which is only a rūpa. So long as we are breathing in
and out we are alive, but when we breathe out for the last time that is
the end of this life. Of what use are then our possessions to us, of
what use are all the things we are clinging to? If one has accumulated
conditions to be mindful of breath with right understanding there can be
moments of calm. Depending on one's accumulated skill, jhāna can be
attained through the development of this meditation subject. However, if
mindfulness of breath is not developed in the right way it is not
bhāvanā. Without precise knowledge of the moments of akusala citta and
of kusala citta, one is bound to take for bhāvanā what is not bhāvanā.
Do we like our breath and do we have desire to watch it, because that
gives us a pleasant sensation? That is not calm but clinging. Breath is
very subtle and not everyone is able to be mindful of it. It is hard to
know when it is breath, the rūpa conditioned by citta, which appears,
and when it is something else we take for breath. Breath can be
perceived where it touches the nosetip or the upperlip. Following the
movement of the abdomen is not mindfulness of breath. If one has no
conditions to develop calm with this meditation subject, one should not
force oneself to develop it. For the development of samatha one should
choose the right subject, that is, the subject which can condition
kusala citta with calm. It depends on the individual which subject is
suitable. That is why there are forty meditation subjects of samatha.

The recollection of the Buddha, the Dhamma and the Sangha are also
subjects of samatha. One may pay respect to the Buddha, the Dhamma and
the Sangha because one has been taught to do so, without right
understanding of the virtues of the Buddha and of his teaching. The
citta may be kusala, but without right understanding there is no mental
development. Right understanding of the object of calm is necessary for
its development. Right understanding of the Buddha's virtues and of his
teachings are conditions for citta to be calm, free from lobha, dosa and
moha. Such moments can occur in daily life, it is not necessary to go to
a quiet place. It is right understanding which is indispensable, and if
this is lacking, a quiet place will not induce calm. If one sits in
front of a Buddha statue and repeats the word ``Buddha'' without right
understanding, kusala citta may arise, but this is not mental
development, bhāvanā.

The ``brahmavihāras'' (divine abidings) of loving kindness (mettā),
compassion (karuṇa), sympathetic joy (muditā) and equanimity (upekkhā)
are subjects of samatha, but they cannot be developed without right
understanding of the characteristics of these virtues. One may recite
the ``Karanīya Mettā Sutta'' in the morning, but, if one does not
develop mettā when one is in the company of other people, can one know
the characteristic of mettā? If one does not know the characteristic of
mettā how can one develop it as a subject of samatha?

When we are in the company of other people we should develop mettā and
we should find out when there is attachment which is akusala and when
there is mettā which is kusala. The difference between attachment and
metta should be known very precisely.

One may wonder whether it is possible to develop mettā towards one's
relatives. Is attachment to them not bound to arise? We can develop
mettā towards them if we do not see them as members of ``our family''
who belong to us, but as human beings whom we would like to treat with
kindness and thoughtfulness.

When true loving kindness, true compassion or the other brahmavihāras
arise, calm can be developed with these subjects and then calm can
increase. That is bhāvanā.

Another meditation subject is ``Parts of the Body'': ``Hair of the head,
hair of the body, nails, teeth, skin\ldots{}'' Are there no parts of the
body appearing during the day? Instead of having attachment or aversion
right understanding of this subject can be developed so that there are
conditions for calm. We are attached to the body and we think that it is
beautiful, but when we consider the ``Parts of the Body'', we can be
reminded that what we take for ``our beautiful body'' are only elements.
When we wash our hair or cut our nails, moments of calm can arise while
considering ``Parts of the Body'' as mere elements that do not belong to
us.

We may have thought that calm can be developed only when one leads a
secluded life. We read in the scriptures that many monks in the Buddha's
time lived in the forest. This does not mean that everybody has to go to
the forest or to a secluded place which is quiet in order to develop
calm. Monks who lived in the forest did so because it was natural for
them, it was their inclination. They developed samatha to a high degree
and they could attain jhāna because they had conditions for such a high
degree of calm. Before the Buddha's enlightenment samatha was the
highest form of kusala. The Buddha taught people to understand
jhānacitta as a conditioned element which is not self.

It is beneficial to consider the meditation subjects of samatha. Some of
them can condition moments of calm in daily life. However, there is no
rule that everybody has to develop calm. It all depends on the
individual whether or not he has conditions for the development of calm.
Right understanding is necessary in samatha. The right understanding in
samatha knows the difference between kusala citta and akusala citta very
precisely and it knows the right conditions for calm. Samatha is a way
to be temporarily freed from lobha, dosa and moha but through samatha
defilements are not eradicated. Only the right understanding developed
in vipassanā sees realities as they are: as impermanent, dukkha
(suffering) and anattā (non-self). Through vipassanā the wrong view of
self and the other defilements can be eradicated.



\chapter[Realities and Concepts]{}
\section*{Realities and Concepts}

The right understanding which is developed in vipassanā sees realities
as they are: impermanent, dukkha and anattā. This understanding has to
be developed, it cannot arise without conditions.

We have accumulated such a great deal of ignorance and wrong view during
countless lives. From the teachings we have learnt that seeing is not
self, that hearing is not self, that all realities are not self.
However, when seeing has arisen, do we know it as it is, or do we still
have an idea of self who sees? Is it still ``my seeing''? Do we still
have an idea of my hearing, my thinking, my feeling, my attachment, my
kusala?

The Buddha spoke about all the phenomena which are experienced through
the six doorways of eyes, ears, nose, tongue, bodysense and mind-door.
He spoke about seeing and visible object, hearing and sound and about
all the other phenomena.

We read in the ``Kindred Sayings''(IV, Saḷāyatanavagga, Kindred Sayings
on Sense, Ch III, § 23):

\begin{quote}

``Monks, I will teach you the all. Do you listen to it. And what, monks,
is the all? It is eye and object, ear and sound, nose and scent, tongue
and savour, body and things tangible, mind and mental objects (dhammas).
That, monks, is called `the all'.

Whoso, monks, should say: 'Rejecting this all, I will proclaim another
all'- it would be mere talk on his part, and when questioned he could
not make good his boast, and further would come to an ill pass. Why so?
Because, monks, it would be beyond his scope to do so.''

\end{quote}

Besides the realities which can be experienced through the six doors,
there are no other realities. We read in § 25 of the same section:

\begin{quote}

``I will teach you a teaching, monks, for the abandoning of the all by
fully knowing, by comprehending it. Do you listen to it. And what,
monks, is that teaching?

The eye, monks, must be abandoned by fully knowing, by comprehending it.
Objects\ldots{} eye-consciousness\ldots{} eye-contact\ldots{} that
pleasant feeling, unpleasant feeling or neutral feeling\ldots{} that
also must be abandoned by fully knowing, by comprehending it.

The tongue, savours\ldots{} The mind\ldots{} mindstates\ldots{} that
pleasant feeling, unpleasant feeling or neutral feeling\ldots{} that
also must be must be abandoned by fully knowing, by comprehending it.''

\end{quote}

All these phenomena are elements which arise and fall away, they are not
beings or things which stay. Seeing is not a person, not self, it is a
moment of consciousness, a citta, which arises, performs the function of
seeing and then falls away immediately.

We are not master of seeing, seeing does not belong to us. Seeing can
arise only when there are the right conditions for it. Eyesense is a
condition for seeing. If there is no eyesense, seeing cannot arise. Are
we master of our eysense? Did we create our eyesense? Visible object is
another condition for seeing. When there is no visible object there
cannot be seeing. All phenomena in ourselves and around ourselves can
arise only when there are the appropriate conditions for their arising.
Without the right conditions they cannot arise. We cannot control
phenomena. Do we think that we are master of our mind and of our body?
Can we prevent them from changing all the time? What we take for mind
are only mental phenomena which arise because of conditions and fall
away immediately. What we take for body are only different bodily
phenomena which arise because of conditions and fall away again.

What we call ``life''or ``the world'' are only mental phenomena, nāma,
phenomena that can experience objects, and physical phenomena, rūpa,
phenomena that cannot experience any object. Seeing is a mental
phenomenon, it experiences visible object. Feeling is a mental
phenomenon, it feels. Visible object is a physical phenomenon, it cannot
experience any object.

Someone asked whether one cannot call nāma ``subject'' and rūpa
``object''. Nāma can also be an object that is experienced. Nāma can
experience both nāma and rūpa. Nāma can experience another nāma. For
instance, can attachment or feeling which are nāmas not be noticed by
another nāma? Thus, the terms ``subject'' and ``object'' cannot be of
any use to understand nāma and rūpa.

It may seem complicated to classify all the phenomena within ourselves
and around ourselves as nāma and rūpa. But is this actually not more
simple than all the different names and values we attach in conventional
language to these phenomena? ``Satipaṭṭhāna, mindfulness of nāma and
rūpa `uncomplicates' our life'', Venerable Dhammadharo said.

We try to build up a synthetic vision of Buddhism, ``our vision''. We
try to fit our own philosophy or the scientific terms we have learnt
into Buddhism. Don't we try to make Buddhism into something which
matches our view of life and ``our world''? Why don't we forget for a
moment all we have learnt, all these thoughts, and study through direct
experience any reality which appears now? Only in that way can we verify
what is real.

All phenomena are either nāma or rūpa. Theoretical understanding of nāma
and rūpa is not enough, it does not bring detachment from the concept of
``self''. We have to know nāma and rūpa as they are through direct
experience. What does that mean? We have to know them when they appear,
one at a time, right now. That is the only way to see them as they are,
as not self.

What should be known in vipassanā through direct experience? Can a
person be known through direct experience? Can hardness be known through
direct experience? These are important questions which we discussed.

Hardness can be directly experienced through the bodysense when it
appears. Is there no hardness now, impinging on the bodysense? We do not
have to think of hardness or name it in order to experience it. Hardness
is real, it is a physical phenomenon, a rūpa, which can be directly
experienced.

Can a chair be experienced through the bodysense? We think that we can
touch a chair, but what is actually experienced? Hardness or softness
can be directly experienced. A chair cannot be directly experienced, it
is only an idea we form up in our minds. Thinking can think of many
objects, it can think of realities and also of concepts which are not
real. When we think that we see a person, it is not seeing, but it is
thinking of a concept. Only visible object can be experienced through
the eyesense. When we touch what we take for a person, what appears?
Hardness, softness, heat or cold can be directly experienced through the
bodysense, not a person. The Buddha taught that there is no person, no
self. But we have accumulated so much ignorance and wrong view that it
seems that we see and touch people.

We may find it difficult to understand that there are in the absolute
sense no people. There are no people, but this does not mean that there
are no realities. What we take for people are different mental phenomena
and physical phenomena which arise and fall away. There are realities
such as seeing, thinking or generosity, but they are not people; they do
not stay. When we think that a person is generous, it is in reality a
moment of consciousness which is generous. It arises because of
conditions and then it falls away. ``Why do we always insert a person in
the giving when giving occurs'', Acharn Sujin said.

When seeing arises, no person sees, only a moment of consciousness
arises and falls away. Venerable Dhammadharo said: ''Seeing has no
father or mother, it has no name or address, it cannot walk or sit.''
This simple example makes it clear that it is very unrealistic, even
foolish, to believe in the existence of a person.

Through vipassanā one can come to know what is real and what is not
real. Concepts are not objects of mindfulness in vipassanā since they
are not real. A person or a chair is a concept we can think of, but it
is not a reality that can be directly experienced. Seeing is a reality
with its own inalterable characteristic that can be directly experienced
when it appears. One may change the name seeing, but its characteristic
cannot be altered; it experiences visible object, no matter how one
names it. The same is true for visible object, attachment or generosity.
They are realities, not concepts and when they appear one at a time
right understanding of them can be developed.

What is mindfulness in vipassanā? This was another topic of our
discussions. Is being mindful of an object the same as being conscious
of an object? For example, when one is conscious of hardness does that
mean that one is mindful of hardness?

Mindfulness, in Pāli: sati, arises with every sobhana citta (beautiful
consciousness). Sati is wholesome, it is non-forgetful of what is
wholesome.There are many levels of sati. There is sati of the level of
dāna. The kusala citta that performs dāna could not arise without sati.
There is sati with sīla. When kusala citta arises which observes sīla it
is accompanied by sati. The kusala citta which develops samatha is
accompanied by sati which is aware of the object of samatha.

The kusala citta which develops vipassanā is accompanied by sati. Sati
in vipassanā is mindful of nāma or rūpa which appears right now through
one of the six doors. The object of mindfulness in vipassanā can be
visible object, seeing, sound, hearing, thinking, or any other reality
which appears at the present moment. We should first have more
understanding of the object of sati so that the function of sati in
vipassanā will become more evident.

Sati in vipassanā is mindful of the reality appearing at the present
moment. What is the meaning of ``present moment''? When hearing arises,
hearing itself is not accompanied by sati, it has only the function of
hearing. But when it has just fallen away, the characteristic of hearing
can be the object of mindfulness. Can there not be mindfulness of
hearing right now? Mindfulness accompanies kusala citta, but even
akusala citta can be the object of mindfulness. For example, citta with
dislike can be the object of mindfulness. The dislike has fallen away
when the citta with mindfulness arises, but can the characteristic of
dislike not appear to sati? Dislike is different from like or from
seeing.

Being mindful of a reality is not the same as being conscious of an
object. When, for example, hardness impinges on the bodysense, a citta
arises which merely experiences hardness, it has the function of
experiencing hardness. This type of citta does not like or dislike the
object, neither can it have right understanding of it. Shortly after
this citta has fallen away, akusala cittas or kusala cittas arise. If
there are conditions for kusala citta with mindfulness of the object,
the characteristic of that object can be investigated, so that right
understanding can develop. Right understanding cannot arise immediately,
it has to be developed little by little through mindfulness. We used to
study only by reading, listening or thinking. Study with mindfulness is
different: it is study through the direct experience of the
characteristics of nāma and rūpa as they appear one at a time. Acharn
Sujin often said: ``Without study paññā (wisdom) cannot grow''.

Only one reality at a time can be the object of sati. Can we experience
more than one object at a time? It seems that we can see and hear at the
same time. But each citta which arises can experience only one object
and then it falls away, to be succeeded by the next citta. Seeing
experiences visible object through the eye-door and then falls away.
Hearing is completely different from seeing, it experiences sound
through the ear-door and then falls away. Since cittas arise and fall
away very rapidly it seems that seeing and hearing last for a while and
that they can occur at the same time, but that is not so.

Is there no seeing or hearing now? There is often forgetfulness, no
``study'' of any reality. Hardness impinges on the bodysense time and
again, but hardness is not investigated so that it is known as only a
reality, a kind of rūpa. When we touch something which is hard we have
no doubt that it is hard; even a child can know this. But is the
characteristic of hardness understood as only a rūpa, not mixed up with
a concept of a finger or a chair which is hard? When we think that we
experience a ``whole'' such as a finger or a chair, it shows that there
is no mindfulness of a reality as it appears through one of the six
doors. We may experience hardness many times with attachment, with
aversion and with ignorance. Sometimes sati may arise and then the
charac-teristic of hardness can be investigated so that right
understanding can develop. From the foregoing examples we can see that
mindfulness or awareness in vipassanā is not the same as what we mean in
conventional language by ``awareness'' of something or being
consciousness of something.


\chapter[The Objects of Mindfulness]{}
\section*{The Objects of Mindfulness}

Any reality which appears now can be the object of mindfulness in
vipassanā. Does seeing arise now? That can be object of mindfulness.
Does hearing arise now? That can be object of mindfulness.

We had many discussions about seeing, visible object and thinking of
what is seen, because we all are inclined to confuse different
realities. In vipassanā a very precise understanding of the different
realities has to be developed.

Seeing is a mental phenomenon, it experiences visible object. Visible
object is that which is seen, which is experienced through the eyesense.
We can call it visible object or colour, it does not matter how we call
it, but its characteristic can be known when it appears through the
eyes. When we pay attention to the shape and form of what we see, when
we perceive a person or a particular thing, it is not seeing. Because of
remembrance of past experiences we form up concepts such as ``person''
or ``chair''. It seems that there is a long moment of seeing and that
seeing sees people and things, but seeing falls away immediately and it
is succeeded by other types of cittas. Cittas succeed one another very
rapidly.

When we recognize different colours such as red and blue, it is again
remembrance of concepts. Seeing is only the experience of what appears
through the eyesense. This does not mean that visible object is without
any colour. When visible object is the object of mindfulness, it does
not change into something else. It is visible object that appears. ``It
appears now'', Acharn Sujin reminded us time and again.

Visible object appears now, when our eyes are open. We may think of
something or someone, but that is not the experience of visible object,
since visible object appears through the eyesense. Do we believe that we
see a chair or a person? Venerable Dhammadharo remarked that visible
object has no arms or legs, that one cannot carry it away. Visible
object can only be seen, it cannot be touched.

When visible object appears, there must also be seeing. Seeing is a
mental phenomenon, it is a type of nāma that sees. There is no self who
sees. Seeing can only see, it cannot hear, it cannot think. Seeing which
is a mental phenomenon is different from visible object which is a
physical phenomenon. Mindfulness can be aware of seeing or visible
object, but only of one reality at a time. In that way their different
characteristics can gradually be known as they are.

Several people found the discussions about seeing and visible object,
hearing and sound too academical. Why do we have to know these
realities?

Are seeing and hearing not part of our life? We see and hear pleasant
and unpleasant objects, and soon after seeing or hearing has fallen
away, attachment, aversion and ignorance are bound to arise. We are very
ignorant of seeing, hearing and all the other phenomena of our life. If
there is no understanding of realities such as seeing and visible object
we shall continue to cling to concepts of ``I'' and of ``this or that
person'', and that will cause us much trouble.

Venerable Dhammadharo said: ``We think of that terrible man next door,
but if a brief moment of mindfulness can arise, we shall know that what
is seen is not that man, only visible object.'' In reality no person
exists. Through the eyesense only visible object can be seen. When we
touch someone, hardness, softness, heat or cold may appear, but no
person. All these characteristics can be investigated in order to know
them as they are: only fleeting elements, devoid of self.

We read in the ``Lesser Discourse on the Simile of the Elephant's
Footprint'' (Middle Length Sayings I, no. 27) about the monk who is
mindful:

\begin{quote}
``\ldots{} Having seen visible object with the eye, he is not entranced
by the general appearance, he is not entranced by the detail. If he
dwells with this organ of sight uncontrolled, covetouness and dejection,
evil unskilled states of mind might predominate. So he fares along
controlling it; he guards the organ of sight, he comes to control over
the organ of sight. Having heard a sound with the ear\ldots{} Having
smelt a smell with the nose\ldots{} Having savoured a taste with the
tongue\ldots{} Having felt a touch with the body\ldots{} Having cognized
a mental object with the mind, he is not entranced by the general
appearance, he is not entranced by the detail. If he lives with this
organ of mind uncontrolled, covetouness and dejection, evil unskilled
states of mind might predominate. So he fares along controlling it; he
guards the organ of mind, he comes to control over the organ of mind. If
he is possessed of this ariyan control of the (sense-) organs, he
subjectively experiences unsullied well-being.''
\end{quote}

When we hear the word ``control'' we may think of a self who controls.
How-ever, sati, not self, ``guards'' the six doors.

Should one prepare for vipassanā? Should one sit in a quiet place in
order to become calm first, before one can study the nāmas and rūpas
which appear? We have seen that there is calm in samatha and that right
understanding of the meditation subject can condition calm. In vipassanā
there is also calm and it is conditioned by right understanding. The
right understanding in vipassanā is different from the right
understanding in samatha. Through the development of vipassanā one will
see nāmas and rūpas as they are, as not self. When there is right
understanding of the reality which appears calm arises at that moment,
there is no need to strive for it. Trying to become calm as a
preparation for vipassanā is not the right condition for the arising of
mindfulness and understanding of the realities that appear. Intellectual
understanding of nāma and rūpa and of the development of vipassanā can
be a condition for direct understanding of realities later on.

Intellectual understanding of nāma and rīpa is different from the direct
experience of their characteristics and one should know this difference.
It is important to know when there is sati and when there is no sati. If
we have correct understanding of sati, it can develop.

Many realities are appearing, such as seeing, hearing, attachment,
hardness or heat, but mostly there is forgetfulness, no study of
realities. When there are the right conditions for sati, it may arise,
just for a moment, and it can begin to be aware of one reality at a
time. We may try to explain in many ways what sati is, but it can only
be known from experience, when it has actually arisen already. Sati is
not self, we cannot be master of sati. Sati cannot arise when-ever we
want it to arise, and for as long as we wish, it is beyond control. It
can arise only when there are the right conditions for its arising. When
we listen to the Dhamma as it is explained by the good friend in Dhamma,
when we consider what we have heard, ask questions and discuss Dhamma,
our intellectual understanding will grow and this can condition right
mindfulness. We should know that also intellectual understanding is not
self, that it arises because of conditions. It can arise only when we
have listened to the Dhamma already and pondered over it for a long
time, and when there is steadfast remembrance of what we have heard.

We read in the ``Gradual Sayings'' (Book of the Tens, Ch VIII, § 3)
about ten `helps', helpful conditions, to obtain ten desirable aims:

\begin{quote}
``Energy and exertion are helps to getting wealth. Finery and adornment
are helps to beauty. Seasonable action is a help to health. A lovely
friendship is a help to virtues. Restraint of the sense-faculties is a
help to the Brahma-life. Not quarreling is a help to friendship.
Repetition is a help to much knowledge. Lending an ear and asking
questions are helps to wisdom. Study and examination are helps to
dhammas. Right faring is a help to the heaven worlds.

These are the ten helps to these ten things which are desirable, dear,
charming, hard to win in the world. ''
\end{quote}

We see that listening and asking questions are important for the
development of wisdom. In the same sutta it is said: ``Not to lend an
ear and ask questions is an obstacle to wisdom''. Study and examination
are helpful conditions for dhammas. The commentary to this sutta, the
``Manorathapūranī'' adds that by ``dhammas'' is meant the nine lokuttara
dhammas, namely the eight lokuttara cittas and nibbāna
\footnote{At each of the four stages of
enlightenment two types of lokuttara cittas arise: the magga-citta or
path-consciousness, which is lokuttara kusala citta, and the phala-citta
or fruition, which is lokuttara vipākacitta. These eight lokuttara
cittas experience nibbāna.}.
If one continues to investigate the realities that appear, paññā
develops and eventually enlightenment can be attained.

Venerable Dhammadharo remarked that we should not cling to an idea of
self who is going to practise and will then attain enlightenment
quickly. He said: ``Then we are stuck with the idea of self. We cannot
say, come on sati, come on paññā.'' Are we not sometimes behaving as if
we could induce them?

Even if sati arises we cannot keep it, it arises and falls away. ``It
may be followed by excruciating doubt'', venerable Dhammadharo said. Who
knows the next moment? Realities arise because of conditions and then
fall away. We never know what will happen the next moment. How could we
then plan to have sati and how could we plan what to be aware of?

It is unpredictable when sati will arise and of what it will be aware.
When we recognize something there must have been many different cittas
which arose and fell away. Seeing which experiences only visible object
is one reality, it is different from recognizing someone. When we
recognize someone we think of a concept, but there must also be seeing
in order to recognize someone. One may wonder whether sati should not be
aware of seeing first and then of thinking. There is no rule, no
specific order. We cannot plan of what object there will be awareness,
sati is not self.

Seeing is different from visible object and one may wonder how one can
separate seeing from visible object, they seem to appear together. There
is no self who can separate seeing from visible object. Sati can be
aware sometimes of seeing, sometimes of visible object. One
characteristic at a time can be investigated and in that way right
understanding can know the difference between nāma and rūpa. If one
thinks that one can experience seeing and visible object at the same
time it shows that there is no mindfulness. When one thinks of a
``whole'' of impressions, the object is a concept, not a reality.

The impermanence of nāma and rūpa, their arising and falling away, can
be known by paññā only after a more precise understanding of them has
been developed. One may wonder why the arising and falling away of
realities cannot be experienced before nāma can be distinguished from
rūpa. Why should one first distinguish visible object from seeing or
sound from hearing? Many different realities appear and disappear.
Seeing arises and then hearing, and then other realities appear and
disappear. Is that not the experience of impermanence?

That is only thinking about impermanence, not the direct understanding
of the arising and falling away of nāma and rūpa. ``If one still takes
seeing and visible object together, as a `whole', what arises and falls
away?'', Acharn Sujin asked. What exactly arises and fall away? Is it
seeing or visible object? Only one object can be experienced at a time.

The first stage of insight is directly knowing the difference between
the characteristics of nāma and rūpa. The arising and falling away of
nāma and rūpa can be realized at a later stage. First their different
characteristics have to be investigated. ``If you do not `study' seeing
and visible object now, don't think that you can become a sotāpanna'',
Acharn Sujin said.



\chapter[Right Understanding]{}
\section*{Right Understanding}

In Kandy the venerable Piyadassi Thera was leading the discussions with
mettā and a great deal of patience. He understood which terms used in
the discussions people would find difficult and therefore he asked for
more precise definitions. ``Right understanding'' was one of the terms
he asked us to explain.

What is right understanding? There are many levels of right
understanding, in Pali: sammā-diṭṭhi, which is the cetasika (mental
factor) of amoha (non-delusion) or paññā. When we are generous, the
kusala citta may arise with or without right understanding. We may give
because it is our nature to give, without any understanding of what
kusala is, what kamma and vipāka (deeds and their results) are. We may
also give with right understanding of cause and effect. It is the same
with the kusala that observes sīla: it may arise with or without right
understanding.

As we have seen, the citta which develops samatha must always be
accompanied by right understanding. One should have right understanding
of the meditation subject of samatha. This subject should be the right
condition for the citta to become calm, to become temporarily free from
attachment, aversion and ignorance. Right understanding of the level of
samatha knows the difference between kusala citta and akusala citta and
it knows when these types of citta arise. However, it does not know
kusala citta, akusala citta and the other phenomena as they are: as
elements devoid of self.

The right understanding developed in vipassanā is of a higher level: it
sees nāma and rūpa as not self. This kind of understanding will be able
to eradicate wrong view and the other defilements. Right understanding
developed in vipassanā sees, for example, visible object as only a
reality, no thing or being in it. It sees visible object as not self.
The belief in a ``self'' is wrong view. So long as paññā has not
eradicated wrong view we are inclined to take realities for self.

We may have intellectual understanding of the truth but it is still
difficult to realize visible object as only a rūpa when it appears. We
have to be mindful of visible object when it appears, of seeing when it
appears and of all the other phenomena over and over again, during
countless lives. We may remind ourselves that it is not a person, not a
thing which is seen. That is intellectual understanding and we should
know that it is not direct understanding of the reality that appears.
Intellectual understanding arises because we listened to the Dhamma; it
is conditioned, not self. Intellectual understanding is a condition for
the arising of sati. When sati arises, the realities which appear can be
investigated. Acharn Sujin reminded us many times: ``Is there no seeing
now? Study it. Otherwise paññā cannot grow.''

The word ``study'' is a translation of the Pāli term ``sikkhā''. Sikkhā
can also be translated as ``training''. The word ``study'' can remind us
that the reality appearing at the present moment should be investigated
so that it can be seen as a mere conditioned dhamma, a nāma or a rūpa.
The word ``study'' can be a reminder that right understanding is only
beginning to develop and that realities have to be studied countless
times before realities can be seen as they are. There are many degrees
of right understanding and it develops very gradually.

We should not be discouraged that mindfulness and understanding seldom
arise. The fact that we are interested in the Dhamma today and that we
listen today shows that we have conditions for further development of
right understanding. We are likely to have listened in former lives.

Acharn Sujin said: ``All those people who listened in the Buddha's time
and did not attain enlightenment, where are they now?'' They had
conditions for the development of paññā, but paññā needed more
development; it had not yet been developed to the degree necessary for
the attainment of enlightenment. We may have been one of those who
listened to the Buddha, and now paññā needs to be developed more.

When we were walking along the beach one of our friends remarked that he
was worried that he could not become a sotāpanna in this life. Those who
have not attained enlightenment run the risk of an unhappy rebirth.
Rebirth may occur in a plane where one cannot develop satipaṭṭhāna. I
had been preoccupied with the same question.

Acharn Sujin answered:

``Today we are in the human plane and we are discussing Dhamma. We may
have had births as an animal, but that is forgotten now. Sati which is
accumulated today is never lost. It is a condition for future
development. There can be unhappy births again, but why should we worry
about it?''

She spoke about the Bodhisatta's horse Kanthaka. He had carried the
Bodhisatta outside the palace, after he had renounced worldly life.
Kanthaka could not develop wisdom in that life since he was an animal.
But he was reborn in a deva plane where he developed wisdom and attained
enlightenment. We cannot control anything which happens, but when there
are conditions for right understanding to develop, it will perform its
function.

``What are realities?'', this was a question some people asked. Reality
is not a concept, it is not something abstract. Reality is that which
can be directly experienced, now. Is there no seeing now? Seeing is a
reality, it can be directly experienced. When mindfulness arises, its
characteristic can be investigated in order to know it as it is.

Visible object is a reality, it can be experienced when it appears, now.
Hearing is a reality, sound is a reality. Hardness, softness, heat and
cold are realities; they can be directly experienced through the
bodysense when they appear. Do they not impinge on the bodysense now? If
there is no forgetfulness, understanding of realities can be developed.
This is the way to know them as they are: elements which are devoid of
self.

``Person'' is not a reality, it is only a concept or idea we form up in
our minds. We cling to people and we take them for permanent and for
``self''. We can think of old age and death, but we have not penetrated
the characteristic of impermanence, the arising and falling away of
realities. We should remember that what we take for self or person are
only nāma and rūpa which arise and fall away all the time. Thus, birth
and death actually occur at each moment.

What we call ``life'' is in reality one short moment of cognizing an
object. This moment falls away and is succeeded by the next moment. When
a citta arises which experiences visible object, our life is seeing. At
another moment our life is hearing or thinking. All these moments fall
away as soon as they have arisen. Thus, we can say that life exists only
in one short moment, this very moment.

Nāma and rūpa are realities, they can be experienced. Instead of the
word reality we can use the word ``dhamma''. Dhamma does not only mean
the Buddha's teaching, it has other meanings as well. Everything which
is real is dhamma or ``paramattha dhamma'', which is translated as
``ultimate reality''. Nama and rūpa are paramattha dhammas.

There are two kinds of conditioned nāma: citta, consciousness, and
cetasika (mental factor arising with citta). Seeing and hearing, for
example, are cittas. Attachment and mindfulness are cetasikas which can
accompany citta. Citta is always accompanied by several cetasikas, at
least seven.

Nibbāna is the unconditioned nāma. It does not experience an object, but
it is the object experienced by lokuttara (supramundane) citta.

Summarizing the paramattha dhammas, they are:

\begin{description}

\item citta
\item cetasika
\item rūpa
\item nibbāna

\end{description}

Ultimate realities are different from ``conventional truth'', concepts
or ideas we can think of but which are not real in the ultimate sense.
We need to use concepts such as person, brain, society, in our contact
with our fellowmen. We use these concepts and we would find it difficult
to do without them. How-ever, we should remember that they are not
realities which can be directly experienced when they appear at the
present moment, such as seeing, visible object, hearing or sound.
Paramattha dhammas can be objects of mindfulness in vipassanā.


\chapter[The Right Conditions for Sati]{}
\section*{The Right Conditions for Sati}

Nāma and rūpa appear one at a time through the six doors. They have
different characteristics and these characteristics should be known.
``Characteristic'' was another term people asked us to explain.

Each reality has its own specific characteristic by which it can be
distinguished from other realities. Visible object has a characteristic
which is different from sound. Visible object is experienced through the
eyesense, it cannot be experienced through the earsense. Sound is
experienced through the earsense, it cannot be experienced through the
eyesense. Visible object has a characteristic which is different from
seeing. Visible object is rūpa; it does not know anything, it cannot
see. Seeing experiences visible object, it is nāma, different from rūpa.
We are inclined to ``join'' seeing and visible object into a ``whole'',
instead of being mindful of their different characteristics as they
appear one at a time. So long as we do not distinguish the different
characteristics of nāma and rūpa, we cling to the concept of person or
self and we are ignorant of realities. Is there an idea of ``I'' who
sees, or is there a person or thing in the visible object?

The specific characteristics (visesa lakkhaṇa) of nāma and rūpa can be
known more clearly when we are mindful of them when they appear. Nāma
should be known as nāma and rūpa as rūpa. Later on, when paññā is more
developed, the general characteristics (sāmañña lakkhaṇa) of nāma and
rūpa can be realized and these are: the characteristics of impermanence,
dukkha and anattā. Before the general characteristics can be penetrated,
the specific characteristics of realities should be known. When sati is
mindful of visible object and this is known as rūpa, understanding
begins to see it as not self. It is rūpa, not a person or thing which is
seen. When sati is mindful of seeing and this is known as nāma,
understanding begins to see it as not self. It is nāma which sees, not
``I''.

Sati and right understanding are accumulated little by little.
``Accumulation'' was another term people requested us to define. Someone
found it difficult to understand how a tendency such as lobha can be
accumulated. Each citta which arises falls away completely, how then can
a tendency be accumulated?

Each citta which arises falls away completely, but it conditions the
next citta, it is succeeded by the next citta. That is the reason why
good tendencies and bad tendencies are carried on from moment to moment.
When we are fast asleep and not dreaming lobha does not arise. When we
wake up lobha arises again. Where does it come from? It must have
conditions for its arising. It can arise because lobha has been
accumulated and it is carried on from one moment to the next moment. Our
attachment today is conditioned by attachment in the past, and
attachment today conditions in its turn attachment in the future.

We have accumulated many defilements such as attachment, aversion,
ignorance, jealousy and stinginess. We have also accumulated good
inclinations. Today we take an interest in the Dhamma, we like to listen
to the Dhamma. Where does this interest come from? It must have
conditions, we must have listened to the Dhamma in the past. What we
learn is never lost. If a moment of right understanding can arise now,
it can condition the arising of right understanding later on.

One of our friends remarked that he used to think that only kamma, good
and bad deeds, could be accumulated. He did not think that good and bad
inclinations could be accumulated.

It is true that good deeds and bad deeds are accumulated. When we commit
a bad deed such as killing, the akusala citta is accompanied by the
cetasika volition or intention which motivates that deed. Kamma is
actually the cetasika volition, cetanā. The volition or kamma which
motivates that evil deed falls away together with the citta. But since
each citta conditions the next citta, kamma, the evil volition, is
carried on from moment to moment. That is why kamma can produce its
appropriate result later on. Akusala kamma can produce an unpleasant
result, which may be an unhappy rebirth, or, in the course of life, an
unpleasant experience through one of the senses. Kusala kamma which is
accumulated brings a pleasant result.

Thus, kamma is accumulated and it can produce its result later on. Kamma
is one type of condition: kamma-condition (kamma-paccaya).
Kamma-condition is not the only type of condition, there are twentyfour
classes of conditions.

Not only kamma, also unwholesome and wholesome inclinations are
accumulated. These inclinations which are carried on from one moment of
citta to the next moment are the conditions for the arising of akusala
citta and kusala citta later on. This type of condition is different
from kamma-condition. When we refer to kamma-condition we speak about
kamma which produces result.

The way different types of conditions operate is very intricate. We can
verify that not only kammas, but also our good and bad inclinations are
accumulated from one moment to the next moment. Lobha can arise at any
time, and thus, it must have conditions. It is conditioned by lobha in
the past which has been accumulated. Sometimes generosity or kindness
can arise, and these are conditioned by generosity and kindness in the
past which have been accumulated. Evenso, when right understanding of
realities has been accumulated, it can arise more often.

In Anurādhapura we had discussions about kamma and vipāka. Someone
remarked that he found it unjust that a deed commited in a former life
can cause suffering in this life. The person who suffers today is not
the same person anymore as the being in the past who committed the bad
deed which produces an unpleasant result. Why then do we have to suffer
today because of deeds committed in a past life?

Kamma produces vipāka. Each cause produces its appropriate result. This
is the law of cause and effect which operates, no matter we like it or
not. When we suffer from pain it is the result of kamma. We may be
inclined to think: ``Why does this have to happen to me?'' But why do we
think of ``me''? There was no being in a former life who committed
deeds, neither is there a being in this life who experiences results.
Only realities, nāmas and rūpas, are arising and falling away.

Different types of cittas experience objects and each moment of citta
falls away completely. Some cittas are cause: they can motivate good
deeds and bad deeds which can produce their appropriate results. Some
cittas are the results of good deeds and bad deeds, vipākacittas. Cittas
which experience pleasant or unpleasant objects through the senses, such
as seeing or hearing, are vipākacittas which arise throughout our life.
Vipākacitta arises because of conditions and falls away immediately;
there is no self who experiences a pleasant or unpleasant object. When
we suffer from pain, this is only a short moment of experiencing an
unpleasant object through the bodysense. It is unavoidable, because it
arises because of conditions. It falls away immediately. When we think
of the pain with aversion, not only one citta with aversion arises, but
seven cittas with aversion arise in succession. That is the order of the
cittas arising in a process
\footnote{Cittas which experience objects
through the six doors arise in processes of cittas. Within a process
akusala cittas and kusala cittas arise in a series of seven cittas.
}.
Pain is unavoidable. Life is birth, old age, sickness and death.

The understanding of the Dhamma can help us to cope with problems in
life. We met a business man who complained about the nationalisation of
property. He had lost many of his possessions. After he had studied the
Abhidhamma and pondered over it he worried less about his lost property
and he thought more about the development of kusala. This showed that he
had accumulated right understanding.

We should remember the sutta about the ``marvel'' of the Dhamma. We read
in the ``Gradual Sayings'' (Book of the fours, Ch XIII, § 8, Marvels):

\begin{quote}
``Monks, on the manifestation of a Tathāgata\ldots{} four wonderful,
marvellous things are manifested. What four? Monks, folk take pleasure
in the habitual (sense-pleasures), delight in the habitual, are excited
thereby. But when Dhamma contrary to such is taught by a Tathagata, folk
are ready to hear it, they lend an ear, they apply their minds
thereto\ldots{} Monks, folk take pleasure in pride\ldots{} folk take
pleasure in excitement\ldots{}

Monks, folk are come to ignorance, are become blinded, overcast by
ignorance. But when Dhamma controlling ignorance is taught by a
Tathāgata, they are ready to hear it, they lend an ear to it, they apply
their minds thereto. This, monks, is the fourth wonderful, marvellous
thing manifested when a Tathāgata, Arahat, a fully Enlightened One is
manifested\ldots{}.''
\end{quote}

When there are conditions for the arising of sati it is mindful of the
present reality, appearing through one of the six doors. However,
because of our ignorance we may easily mislead ourselves. We may think
that awareness and understanding of the present object have arisen when
we are actually thinking with attachment, aversion and ignorance. For
example, hardness or softness may impinge on the bodysense. Instead of
developing understanding of these characteristics I found myself
thinking of the places where the impact occurred. This shows that there
was no understanding, only thinking about the body, about concepts. When
hardness presents itself, we can gradually come to understand it as a
kind of rūpa. We can learn that there is no ``place of impact'' in the
hardness, no ``body'' in the hardness. When the characteristic of
hardness is the object of mindfulness, no other object is experienced at
that moment.

I looked at the colourful saris the ladies were wearing and I noticed
that lobha arose as soon as I looked. I was watching ``my lobha''.
Thinking about one's lobha or watching it is not mindfulness of its
characteristic. Once, while I was eating and enjoying my food, Acharn
Sujin asked me whether there was mindfulness. I said: ``Lobha'', without
being mindful of its characteristic. Acharn reminded me that lobha has
its own characteristic and that it can be directly known when it
appears. In that way it can be realized as only an element, not self.
There is no need to think about it or to name it.

Someone remarked that one should practise satipaṭṭhāna methodically,
otherwise there would not be any result.

If one tries to be mindful according to a certain method, who is trying?
There is a concept of self who tries to direct sati to a particular
object. That is thinking, not mindfulness.

We never know whether attachment, anger, seeing or doubt will arise, or
whatever other reality. How can we then direct sati or follow a certain
method?

Sati is not self, it arises only when there are the right conditions for
its arising. ``If sati does not arise, nobody can be aware at that
moment'', Acharn Sujin said. Do we still believe that we can control
sati? If one tries very hard to have sati, one will become tense, it
will not be of any help. One of the monks remarked that he found it such
a relief that one does not have to try to make sati arise.

The present reality is here and now. We obstruct the arising of sati and
right understanding if we think that we have to sit in a room and
practise methodically. Inside the room and outside only seeing, hearing,
hardness and other realities appear one at a time, through the six
doors.

Acharn Sujin said that at this moment of seeing, hearing or thinking we
should have the courage to find out whether awareness arises of the
present reality or not yet. Seeing is real, it sees. Visible object is
real. It is different from seeing. Right understanding of these objects
can be developed in our daily life, during our activities, so that they
are known as they are.



\chapter[The Object of Sati]{}
\section*{The Object of Sati}

``Is sati being conscious of all one's actions, such as eating or
driving a car?'' This was another question asked during the sessions.

When someone thinks of himself as eating or driving a car, it is not
sati but thinking of concepts. Eating is not a reality, driving a car is
not a reality. Sati is a wholesome cetasika and it accompanies all
wholesome cittas. Sati of satipaṭṭhana is mindful of realities, of nāmas
and rūpas. It is mindful of one reality at a time. Sati can be mindful
during activities such as eating or driving a car.

While one eats there are hardness, flavour or thinking. These are
realities and they can be known one at a time, as only different
elements, no ``body'', no self in them. When mindfulness of realities
arises, right understanding of them can be developed. When there is no
mindfulness one is bound to cling to one's body and one's mind.

We read in the ``Satipaṭṭhāna Sutta'' (Middle Length Sayings I, no. 10)
in the section on postures
\footnote{I am using the translation of
the Venerable Nārada Thera.}
:

\begin{quote}

``A disciple while walking, understands `I am walking'; while standing,
he understands `I am standing'; while sitting, he understands `I am
sitting'; while lying down, he understands `I am lying down'. He
understands every position his body assumes.

Thus he lives contemplating the body internally or externally or both
internally and externally.

He lives contemplating the arising nature of the body, or the perishing
nature of the body or both the arising and perishing nature of the
body\ldots{}.''

\end{quote}

Should we be aware of walking? We should read the whole context of the
sutta in order to understand its meaning. We cling to the body that
assumes different postures, but this is only a concept, not a reality.
What we take for the body are only different elements which arise and
fall away. Hardness, softness, heat, cold, motion or pressure, no matter
whether they are internal or external, should be known when they appear
one at a time. In this way one will know later on the arising and
falling away of these elements. Right understanding will eventually lead
to detachment. This sutta reminds us to be aware of any reality which
appears, when we are walking, standing, sitting or lying down.

Is it possible to give simple instructions for the development of
vipassanā? This was a question asked during the discussions.

It would be very easy if a teacher could tell us what to do first and
what next, and if by following these instructions we could be sure of
attaining enlightenment. However, the Buddha taught us not to follow a
teacher blindly, but to develop the Path ourselves. A good friend in
Dhamma can explain the right way of development. We should listen,
consider what we have heard, and then study with mindfulness any reality
which appears. We have to develop the Path ourselves, right now; nobody
else can do that for us.

It is right understanding, not ``self'', which will eventually see
things as they really are. We may wonder how paññā, understanding or
wisdom, can ever know impermanence, dukkha and anattā, and how it can
realize nibbāna.

``Don't underestimate the function of paññā'', Acharn Sujin often said.
It is not self who will know realities, it is paññā. The present moment
is very precious. If realities are considered and investigated, paññā
will work its way.

Sati arises only when there are conditions for its arising, and nobody
can cause its arising. Should we not make an effort to have sati? When
we hear the word effort we are so used to thinking of a self who exerts
effort. Effort is a cetasika, not self. Effort arises with many types of
cittas, though not with all types. Effort arises not only with kusala
citta, but also with akusala citta.

When sati is aware of any reality which appears now right effort has
arisen already; we do not have to think of making an effort. When we
think of effort there is bound to be akusala citta with desire. Akusala
citta is accompanied by wrong effort.

We read in the ``Analysis of the Truths'' (Saccavibhaṅgasutta, Middle
Length Sayings III, no 141) about four right efforts:

\begin{quote}

``And what, your reverences, is right endeavour? As to this, your
reverences, a monk generates desire, endeavours, stirs up energy, exerts
his mind and strives for the non-arising of evil unskilled states that
have not arisen\ldots{} for the getting rid of evil unskilled states
that have arisen\ldots{}for the arising of skilled states that have not
arisen\ldots{} for the maintenance, preservation, increase, maturity,
development and completion of skilled states that have arisen. This,
your reverences, is called right endeavour\ldots{}.''

\end{quote}

When do these four right efforts arise? We read in the Gradual Sayings
(Book of the Ones, Ch VI):

\begin{quote}
``Monks, I know not of any other single thing of such power to cause the
arising of good states, if not yet arisen, or to cause the waning of
evil states, if already arisen, as earnestness. In him who is earnest,
good states, if not yet arisen, do arise, and evil states, if arisen, do
wane.''
\end{quote}

Earnestness, mindfulness of nāma and rūpa, conditions the development of
wholesomeness and it leads to the elimination of unwholesomeness. Right
understanding of the eightfold Path, which is developed in being mindful
of nāma and rūpa, conditions right effort of the eightfold Path. If one
still clings to an idea of self who makes an effort, there is lobha, not
right effort of the eightfold Path.

At the sessions, people were still wondering what could be done in order
not to sit in idleness and wait for the arising of paññā.

There is no self who is ``not doing anything''. Each citta which arises
performs a function. Even when one thinks that one is not doing anything
sati can arise and be mindful of thinking as not self. When there is
right understanding of the object of mindfulness, there cannot be
laziness. When there are conditions for sati, it arises before one has
any intention to be aware. If someone has the intention to be aware he
is bound to have attachment.

We may find it very difficult to be mindful. What should we do during
all those moments when there is no mindfulness of nāma and rūpa? Is
there not bound to be a great deal of akusala?

The Buddha taught us many different kinds of kusala. Sometimes we have
an opportunity for dāna, sometimes for sīla, and sometimes for calm, for
example, when we think of the Buddha's virtues or when we develop mettā.
Sometimes mindfulness of nāma and rūpa may arise. We cannot move the
citta from one kind of kusala to another kind of kusala. It depends on
conditions which kind of kusala arises at a particular moment. Knowing
about the different ways of kusala and seeing their value prevents us
from laziness.

As we have seen, intellectual understanding can condition the arising of
sati. That is why we had discussions about realities such as seeing,
visible object, hearing or sound. We still have many misunderstandings
about nāma and rūpa. We talked about hearing and paying attention to the
meaning of words we hear. Paying attention to the meaning of words is
not hearing, it is thinking of concepts. We remember concepts.
Remembrance, saññā, is a mental factor which arises with each citta.
There is remembrance all the time of visible object, sound and other
realities, and also of concepts.

Acharn Sujin used the name ``Elizabeth'' in order to show that while
this word is pronounced, different sounds are heard that arise and fall
away. The cittas that hear these different sounds arise and fall away.
Ignorance may arise after hearing sounds, or there may be mindfulness of
nāma and rūpa. When we recognize these different sounds, it is not
hearing, but remembrance of concepts. Memory conditions thinking of the
person Elizabeth. Each one of us thinks of the person with this name he
or she knows. We think of her appearance, her voice, or the letters she
wrote. Thinking conditions different feelings: pleasant feeling,
unpleasant feeling or indifferent feeling. In reality there is no
Elizabeth, one only thinks of the concept of Elizabeth.

This example reminded me that there can be ignorance and wrong view even
in between pronouncing the sounds of a word. When there are conditions
sati can arise instead of ignorance, even in between the recognition of
different sounds. Sati is aware of the characteristic of the reality
that appears at the present moment and we do not need to think of the
rapidity of the processes of citta.

``When there is mindfulness of nāma and rūpa one truly lives alone'',
Acharn Sujin reminded us. There may be many people around, but in
reality there are no people. Visible object, seeing, sound, hearing,
many different realities arise which can be the object of mindfulness,
one at a time. They do not belong to anyone. If one thinks of one
person, of two or more people, there is the world of people. But in
reality people do not exist; only nāmas and rūpas arise and fall away.
Life exists only in one moment of experiencing an object.

Sarah said that she found it difficult to know he difference between the
moment of just seeing and the moment of paying attention to shape and
form. Paying attention to shape and form is not seeing.

Acharn Sujin answered that right understanding is not a matter of
catching the moment a particular reality appears. If we believe that
there is a particular order in the appearing of realities, we think of
concepts instead of being aware of whatever presents itself. Sometimes
the object of mindfulness is seeing, sometimes thinking, sometimes
visible object. There is no rule with regard to the object sati will be
aware of.

We all are inclined to try to know the difference between seeing and
paying attention to shape and form, and this is clinging. Venerable
Dhammadharo asked, ``Who is trying?'' and I answered, ``Self wants to
know''. He replied that people usually give the correct answer, but, do
they realize the truth? We forget to be mindful of clinging when it
appears. When we keep on thinking of seeing instead of being mindful of
seeing which appears now, seeing cannot be known as it is. When we have
doubt about characteristics, or when we are discouraged about our lack
of sati, these moments should also be studied. Any reality can be the
object of mindfulness, no matter whether we like that object or not.


\chapter[Rebirth]{}
\section*{Rebirth}

Our journey was a pilgrimage. We visited several places of worship in
order to recollect the Buddha's virtues and the virtues of the arahats
who lived in Sri Lanka and had practised satipaṭṭhāna until all
defilements were eradicated. Khun Sujin encouraged us to keep on
studying realities and developing satipaṭṭhāna. ``It never is enough, it
never is enough until one has attained arahatship'', she said.

In Anurādhapura we stayed in the Government Agent's residence, a
peaceful place with trees all around it, in the old city of
Anurādhapura. His house is within walking distance of the
``Ruvanvelisāya'', the great stupa (dagaba) which King Dutthagāmanī
started to build. Relics of the Buddha have been enshrined in this
stupa. It is illuminated every night and there are always people walking
around it and reciting stanzas. We visited the stupa several times and
on one occasion, while we were walking around it, Acharn Sujin spoke to
our hostess about satipaṭṭhāna. She reminded us to be mindful of only
one reality at a time, as it appears through one of the six doors. We
should not mix up the six doorways. We cannot know visible object and
tangible object, a reality appearing through the bodysense, at the same
time. She said: ``When a reality appears, it does so only through one
doorway. Leave the other doorways alone.'' Don't we try to think of many
``things'' instead of being aware now?

While walking on the stone precincts around the stupa, one may form up
the idea of floor. That shows that there is no mindfulness. Through the
eyes only visible object appears, through the bodysense hardness may
appear. If we do not mix up the different doorways, we shall find out
that there is in reality no floor; there are only different nāmas and
rūpas which appear one at a time.

We are inclined to take seeing and other realities for permanent. Acharn
Sujin reminded us:

``Each reality which appears falls away. The hardness now is not the
same as hardness a moment ago. Seeing now is not the same as seeing a
moment ago. If we think that it is the same it shows that there is no
awareness.''

Acharn Sujin remarked:

``If sati is not accumulated now, it is not possible to attain
enlightenment. Enlightenment can be attained. In the Buddha's time many
attained it. The development of sati is very natural; it is not too
difficult if we are not forgetful. But when sati does not arise, we
should not have regret. When regret appears there can be mindfulness
even of regret.''

While one walks around the stupa and different realities are
``studied'', the past time when arahats walked here and taught
satipaṭṭhāna seems very near. They were never forgetful of realities.

The Bodhi-tree in Anurādhapura which is near to the ``Ruvanvelisāya'' is
another place of worship we visited. The sacred tree stands on a high
terrace and it is surrounded by a golden rail. Generally one does not
have access to the tree, but one of the monks who was in attendance
allowed us to go up to the terrace, in order to pay respect. One night
the same monk arranged for about a hundred white lotus flowers which we
placed all around the tree. The monks who were leading the procession
around the tree chanted stanzas, and we had an opportunity to look at
the new sprout of the tree that had grown recently, several months ago.
It seems that we are far away from the Buddha's time, but so long as
satipaṭṭhāna is taught and pratised we are not far away.

The old city of Anurādhapura and its surroundings are full of stupas,
old monuments and places of commemoration. One of our hosts took us in a
jeep to Tantirimale, which is not far from Anurādhapura. Saṅgamitta and
her retinue who brought the sapling of the Bodhi-tree from India,
stopped in Tantirimale for a rest, on the way to Anurādhapura. A shoot
of the Bodhi-tree was planted in this spot. Today one can still see this
tree which grows on a rocky ground where nothing else will grow. In the
olden times several saplings of the Bo-dhi-tree were planted in
different places, and later on thirty-two saplings were distributed all
over the island.

Many relics of the Buddha have been brought from India to Sri Lanka. The
relic of the Buddha's right collarbone has been enshrined in
``Thupārāma'', which is situated in Anurādhapura.

A few families in Sri Lanka are in possession of very small particles of
the Buddha's relics. One of our hosts in Anurādhapura had in his
shrineroom a particle of a relic of the Buddha and also a relic of an
arahat which had been given to him by his aunt. It is said that so long
as one practises the teachings the relics in one's house will not
vanish. But when one neglects the teachings they will disappear. Our
host showed us the relics, and this was the first time he had shown them
to people outside his family. He took the relics out of their caskets
and we payed respect with flowers, incense and candles. We looked at the
relics, thinking of the Buddha's exhortation to be mindful of the
reality appearing now. Again we found that the Buddha is so near while
one studies the present reality.

The Tooth relic of the Buddha which came to Sri Lanka in the fourth
century A.D. had been enshrined in different capitals in the course of
time. Today the relic is in the ``Dalada Maligawa'' in Kandy. Once a
year a replica of the casket which contains the relic is carried around
in procession: the ``Kandy Perahera''. An elephant with a curled tusk, a
``Tusker'', carries the casket around. The relic itself can never be
taken outside the temple.

The sanctuary where the relic has been enshrined is generally not open
to the public, but we obtained permission to enter.

Afterwards we walked around the shrineroom three times, paying respect
at the ``four quarters''. All these places of worship in Sri Lanka are
occasions to recollect the Buddha and his teachings and to be mindful of
the present reality. Through mindfulness we can learn that life is only
one moment of experiencing an object.

I had offered some money and I expressed the wish: ''May I have less
stinginess.'' Acharn Sujin reminded me that one may have clinging even
while one is expressing such a wish. One may cling to the idea of ``my
stinginess or ``my generosity''. This shows how keen paññā must become.
Otherwise one does not see one's clinging and one takes akusala for
kusala. Even when we do good deeds akusala cittas are bound to arise
shortly after the kusala cittas.

During our stay in Kandy, our host, who was so kind to drive us around
every day, took us to a village school, outside Kandy. Most of the
children of this school came from very poor families. The principal, a
person with great patience and perseverance, had built up the community
of this school in spite of many difficulties. His device was: ``Don't
grumble about what you don't have. Make every difficulty into a
challenge.'' Nobody at this school grumbled.

We had a Dhamma session in the school and one of the teachers translated
English into Singhalese. Many of the questions dealt with rebirth. How
can one prove that there is rebirth and how can one prove that there are
heavenly planes and hell planes?

We explained that today we do not doubt that there was yesterday. Just
as today follows upon yesterday, tomorrow will follow upon today. Evenso
the different cittas (moments of consciousness) which arise and fall
away succeed one another. The preceding citta is completely gone, but
there are conditions that this citta is immediately succeeded by the
next one. The last citta of this life will be succeeded by a following
citta, which is the first citta of the next life: the
rebirth-consciousness.

The first citta of this life was the rebirth-consciousness. It could not
have arisen without conditions for its arising. Its conditions were in a
past life; it succeeded the last citta of the previous life.

If we want to know what our next life will be, we should know our
present life. In this life there are mental phenomena and physical
phenomena arising and falling away, and so it will be in the next life.
The present life will be the past life in the next existence. It depends
on kamma (our accumulated good deeds and bad deeds) in which plane of
existence there will be rebirth. Rebirth in a happy plane is the result
of a good deed, kusala kamma, and rebirth in an unhappy plane is the
result of a bad deed, akusala kamma.

People wonder about the body in the next life. So long as there are
conditions for rebirth, kamma will produce bodily phenomena at the
moment of rebirth-consciousness
\footnote{Unless rebirth occurs in a plane
of existence where there is no rūpa, bodily phenomena.}
. Our body that was yesterday is completely gone, but today there are
again new bodily phenomena we call ``our body''. We have no doubt that
at this moment bodily phenomena arise. Why then do we doubt about
rebirth? Bodily phenomena arise and fall away all the time.

One may perhaps be inclined to prove rebirth by examining cases of
people who claim to remember former lives. Scientific proof and
reasoning will never eradicate doubt and wrong understanding. Neither
are they of any help to take away one's anxiety about what will happen
to the ``self'' after death. Doubt and wrong view can only be eradicated
by right understanding which sees phenomena as they really are.

In Colombo we also had a few sessions with children. We used the
``Sigālovāda Sutta'' (Dīgha Nikāya, Dialogues of the Buddha III, no 31)
as an example of the teaching of different kinds of kusala we should
practise in daily life. In this sutta we read about such good qualities
as kindness, generosity, humbleness and patience. Acharn Sujin spoke
about patience. When we have aversion about an unpleasant object it
shows that there is no patience. But do we have patience when the object
is pleasant? We are attached to pleasant objects and when we are
attached there is no patience. Acharn Sujin said: ''When the food is
very delicious today, do you have patience? Will you eat just enough to
sustain the body, or will you eat more, because you like the food? Then
there is impatience.''

The children wanted to hear ``Jātaka'' stories, stories about the former
lives of the Buddha, and, thus, I explained that the Jatakas teach us
about the many virtues of the Buddha which he accumulated during
innumerable lives. Acharn Sujin asked the children: ``You like to hear
stories, but what about your own story?'' We like to hear about the
story of someone else, but do we really know ourselves? We should find
out more about our ``own story''.



\chapter[Patience]{}
\section*{Patience}

I read in ``History of Buddhism in Ceylon'', by Walpola Rahula, that in
olden times pilgrimages in Sri Lanka were favored by the monks for
various reasons. One of the benefits was traveling with a teacher so
that one could discuss topics of the Dhamma. During our pilgrimage it
was also for us very beneficial to discuss the Dhamma in a personal way
and learn to apply it in the situation of daily life. In theory we know
what is kusala and what akusala, but in our daily life we forget to
apply what we have learnt.

The Buddha taught us to be patient. This may seem simple to us, but we
are impatient when things are not as we would like them to be and when
people do not behave as we would expect them to. Patience was often a
topic of our conversations. The ``Exhortation to the Pātimokkha''
(Ovāda-pātimokkha)
\footnote{The Pātimokkha is the Code of
Discipline for monks.},
recited by the monks, starts with patience:

``Forbearing patience is the highest austerity\ldots{}''

(Khantīparamaṃ tapo tītikkhā\ldots{}.)

We may talk at length about patience without realizing when there is
patience and when there is not. When one is on a journey, things do not
always happen the way one has planned. We had expected to climb the
``Siripada'' (Adam's Peak), a place the Buddha had visited. We had to
cancel this trip twice because the time was not convenient and the rainy
season had started. We always think that we can control situations by
planning, but whether a plan comes true or not depends on conditions. We
cannot force conditions by insisting on our own plans. In such
situations we have to cultivate patience. If we understand that there
are only nāma and rūpa, whether we are on a mountain or in the city, it
helps to be patient.

We should be patient in our speech. Even when we speak about the Dhamma
kusala citta does not arise all the time. We may speak with impatience,
at the wrong time. We may speak with attachment to our own words, and at
such moments we have no mettā, no patience.

We read in the `Discourse on the Parable of the Saw'' (Middle Length
Sayings I, no. 21) that the Buddha spoke about different ways of speech:

\begin{quote}

``\ldots{} Monks, when speaking to others you might speak at a right
time or at a wrong time; monks, when speaking to others you might speak
according to fact or not according to fact; monks, when speaking to
others you might speak gently or harshly; monks, when speaking to others
you might speak about what is connected with the goal or about what is
not connected with the goal; monks, when speaking to others you might
speak with minds of friendliness or full of hatred. Herein, monks, you
should train yourselves thus: `Neither will our minds become perverted
nor will we utter an evil speech, but kindly and compassionate will we
dwell, with a mind of friendliness, void of hatred; and we will dwell
having suffused that person with a mind of friendliness; and beginning
with him, we will dwell, having suffused the whole world with a mind of
friendliness that is far-reaching, widespread, immeasurable, without
enmity, without malevolence.' This is how you must train yourselves,
monks.''
\end{quote}

We should speak at the right time, not at the wrong time. We have to be
considerate of other people's feelings. When it is not the right time
for Dhamma discussion, we can talk about other topics with kusala citta.
Acharn Sujin said to me: ``We apply Dhamma, also when we do not speak
about Dhamma.''

I used to think that talk about flowers, fruits, nature, children and
grandchildren was always motivated by akusala citta; that it was
``animal talk'', mentioned in the ``Vinaya''(Sutta-Vibhaṅga, Pacittiya
85), such as ``Talk of kings, of thieves, of great ministers, of
armies\ldots{}.'' This is talk monks should not engage in.

Venerable Dhammadharo explained to me that even talk which is mentioned
among ``animal talk'' can sometimes be motivated by kusala citta. For
example, when one talks about a king, saying that even kings have to
die, the citta which remembers the impermanence of life is kusala citta.

We can speak about the things other people are interested in with mettā
and karuṇā (compassion), with consideration for their feelings. Acharn
Sujin explained to me that pleasing other people is not necessarily
motivated by attachment; it can be motivated by kusala citta. For
instance, when we say: ``What a beautiful garden you have'', it can be
said with attachment, but it can also be said with kindness or with
sympathetic joy (muditā). It all depends on the citta which motivates
the speaking. At the moment of sympathetic joy there is no envy. It is
precisely when we are with others that we should cultivate the virtues
which are the brahmavihāras of mettā, karuṇā, muditā and upekkhā
(equanimity).

I asked Venerable Dhammadharo what I should say when others tell me
stories with akusala cittas. He remarked: ``What a splendid opportunity
to cultivate mettā and karuṇā at such moments.'' When mettā and karuṇā
arise, the kusala citta will know what to say.

We visited someone who had great aversion towards harsh sounds. He was
angry with people who amused themselves with firecrackers on the
occasion of the New Year. I could sympathize with him because when there
is a radio which is too loud I have aversion immediately. We do not like
aversion and unpleasant feeling, but have we really understood the cause
of dosa (aversion)? When we have problems do we think of the cause of
our problems in the right way? The cause is always within ourselves: our
own defilements. The arahats have no more problems.

We like pleasant objects and we dislike unpleasant objects. Attachment
conditions aversion. I knew that in theory, but I had to be reminded of
it when there was aversion. Acharn Sujin emphasized that when aversion
arises it shows that the attachment which conditions it must be very
strong. That made me see how ugly akusala is. At the moment of aversion
we have no patience, no calm.

It is helpful to consider many aspects of akusala and of kusala. If one
aspect does not help us at a particular moment, another aspect may be
useful. Thinking of kamma and vipāka can help us to be more patient.
When we hear an unpleasant sound and we have aversion, we should
remember that the hearing of the unpleasant sound is the result (vipāka)
of an unwholesome deed (akusala kamma) we performed. The hearing has
been conditioned already and nobody can change it. Hearing experiences
the unpleasant object just for a moment and then it falls away
immediately, it does not stay. But we keep on thinking about vipāka with
aversion and that is akusala.

Whenever we have an unpleasant experience through one of the senses it
is vipākacitta. When we complain, for example, about hot weather, we are
impatient, and we forget that the experience of heat through the
bodysense is only vipāka, caused by kamma.

When we are impatient, there is ignorance; ignorance covers up the
truth. Right understanding sees the disadvantage of akusala , and that
is the condition for the cultivation of kusala.

Patience can be developed with regard to many seemingly unimportant
events in our daily life. When we receive a gift such as a book we do
not like, we should develop patience, for example, in thinking of the
kindness of the giver.

I had a cold and could therefore not wash my hair for many days. Acharn
Sujin reminded me to develop patience even with regard to this. I am
inclined to overlook such facts, but are details not important? So many
moments of our life pass unnoticed.

Kusala citta and akusala citta condition our appearance, they condition
different facial expressions. Is it not lack of consideration for others
if we look sullen? If we remember this it can help us to please others,
even when we feel tired. Our hostess in Anurādhapura always kept
smiling, even when we had to wait a long time for a car. Her conduct
taught me to be more thoughtful, also in small matters.

When we are tired we usually have aversion. This is conditioned by
attachment to our health, to our bodily wellbeing. We may think: ''Well,
I can't help having aversion, it is conditioned.'' What kind of citta
thinks in that way? Even though we say that aversion is conditioned we
may still regard it as ``my aversion'' and then we make it into
something very important. We make the fact of its being conditioned into
an excuse for giving in to a bad mood. When the characteristic of
aversion appears it can be known as only a reality, not self.

Kusala of the level of right understanding may not often arise, but we
should see the value of cultivating all the different ways of kusala.
When we were visiting an old lady who lived alone, in a secluded place,
a friend was cutting her hair and the white flakes of hair were falling
down. At one moment there may be conditions to consider ``Parts of the
Body'': ``Hair of the head, hair of the body, teeth, nails\ldots{}
etc.'', and this can condition calm. There is no need to think
beforehand that one should cultivate calm with such a subject, but when
there is right understanding of calm it can arise naturally. At another
moment one may develop mettā while one helps the old lady or while one
looks at the many ants on her doorpost. At another moment again there
can be ``study'' of visible object as only visible object.

Is it necessary that we have calm first, before mindfulness of nāma and
rūpa can arise? Are the sections of the `Satipaṭṭhāna Sutta'' (Middle
Length Sayings, no 10) about Mindfulness of Breath, Parts of the Body
and Corpses not an indication for this?

When we read the whole context of the sutta we see that the Buddha did
not teach that samatha should be developed first, before one develops
vipassanā. This sutta and all the other suttas teach us that sati can be
aware of any reality appearing at the present moment, no matter what one
is doing, walking, standing, sitting or lying down, cultivating moments
of calm or being engaged in any other activity. Even akusala citta, as
we read in the ``Satipaṭṭhāna Sutta'', in the section on mindfulness of
citta, can be object of mindfulness.

The citta which develops calm of samatha can also be object of
mindfulness. At the moments of calm do nāma and rūpa not arise? When,
for example, a corpse is the object of calm, sati of vipassanā can arise
and be aware of any reality which appears. That is the way to eventually
see nāma and rūpa as they are. There is no other way.

As we have seen, right understanding of samatha is different from right
understanding of vipassanā. Right understanding of samatha does not know
the true nature of visible object, seeing, sound or hearing which appear
now. Right understanding of samatha cannot ``automatically'' change into
right understanding of vipassanā.

All kinds of kusala are of great value. We cannot determine which type
of kusala arises at a particular moment. Every type of kusala, every
reality, can be the object of mindfulness in vipassanā.

Sometimes we feel unable to cultivate any type of kusala. When we are
tired or sick, don't we attach great importance to the way we feel and
don't we make this into an excuse not to develop kusala? Clinging to
ourselves conditions many kinds of akusala.

``Forgetting about oneself conditions the cultivation of kusala'',
Acharn Sujin said. It is inspiring to be with people like Acharn Sujin
and Khun Duangduen who are so kind, patient and considerate. Khun
Duangduen knew that small gestures of kindness are important, she did
not overlook such things. Every day she spoke with Acharn about giving:
what would they give today and to whom? They had brought from Thailand
many useful gifts for the monks. Khun Duangduen looked with kindness at
other people and she was ready to help at any time. I still think of her
generosity as an inspiring example. Someone's example can be more
helpful than words.

Acharn Sujin pointed out that the development of generosity helps us to
have less attachment to our possessions. If we do not develop
generosity, how can we ever become detached from the five khandhas, from
our body and our mind? We cling most of all to our body and our mind, we
do not want to lose them.

If we do not develop kusala now, there are more conditions for akusala,
she said. When we are patient lobha, dosa and moha do not arise.

Venerable Dhammadharo reminded me that we have to develop patience when
we are with people and when we are alone. When we are in the company of
people we are bound to have attachment or aversion. Then there is no
patience. We should develop mettā and karuṇā instead of having
attachment or aversion. When we are alone we may be attached to being
alone, or we may dislike being alone. In that situation we also have to
be patient. If there is mindfulness of any reality which appears, it
does not matter whether we are with people or without them. What
difference does it make? In reality there are no people, only nāma and
rūpa. All that matters is being mindful of them in order to see them as
they are.

Are we impatient when we do not seem to be making progress in wisdom?
Mindfulness of nāma and rūpa should be developed with patience, in the
course of many lives. If we develop patience in all the situations of
daily life, we shall also have more patience as regards the development
of vipassanā. We shall have patience to study with mindfulness any
reality which appears now. We shall not be tired of studying nāma and
rūpa over and over again. It never is enough!

The last day I spent in Sri Lanka was the day the Singhalese celebrated
Vesakha: the day of the Buddha's birth, of his enlightenment and of his
parinibbāna. Many people, including children, were wearing white clothes
and observed eight precepts at home or in the temple.

In Sri Lanka I came to appreciate the observance of the eight precepts,
and on Vesakha Day we also observed them, inspired by the example of the
Singhalese. One of our hostesses told me that she observed eight
precepts once a month in her home, and if the ``Uposatha Day'' was not
convenient for her she would observe them on another day.

Observing the eight precepts is a way of cultivating patience. When one
observes these presepts one realizes how much one clings to eating at
any odd time. Aren't we impatient also with regard to food? On such a
day we are reminded that we are attached to many things we take for
granted in daily life, for example, lying on a soft bed, or sitting on
one's easy chair. These moments usually pass unnoticed, we are not
mindful of them.

The Buddha praised the observance of the eight precepts because on such
a day one follows the example of the arahats. We read in the ``Gradual
Sayings'' (Book of the Eights, Ch. 5, §1, The Observance) :

\begin{quote}
``Monks, the Observance day, when observed and kept with eight
qualifications is very fruitful, of great advantage, very splendid, very
thrilling.

Monks, how is it so observed and kept?

Herein, monks, and ariyan disciple reflects thus: ``All their lives
arahats abandon taking life and abstain therefrom; they dwell meekly and
kindly, compassionately and mercifully to all beings, laying aside stick
and sword. I, too, now, during this night and day, will abandon taking
life and abstain therefrom. I will dwell meekly and kindly,
compassionately and mercifully to all beings, and lay aside both stick
and sword. So, in this way, I shall follow the example of arahats and
keep the Observance\ldots{}.''

The same is said about the other precepts.

\end{quote}

When we observe eight precepts, we have an opportunity to recollect the
excellent qualities of the Buddha and of the arahats who were without
clinging. Clinging is bound to arise, but if we are mindful of it when
it appears, we shall learn to see it as it is, as a conditioned nāma.

On Vesakha Day we offered food to the monks in the ``Buddhist
Information Center'' and afterwards we visited a few temples. In one
temple we saw tiny fragments of the Buddha's bowl which had been
excavated from the ruins of Sopara Stupa, near Bombay. In another temple
we visited, relics of Sāriputta and Moggallāna had been enshrined. In
Sri Lanka there are many opportunities to recollect the excellent
qualities of the Buddha and of the arahats. In the afternoon a Dhamma
discussion was held in the Information Center. The topics were visible
object, seeing, hearing and the other realities which appear. It seems
difficult to know the characteristic of seeing, we are inclined to think
that it is different from seeing at this moment. Acharn Sujin said:

``Study it, this very moment. When hearing appears, the element which
hears should be studied, not the element which sees. When we are
forgetful, not aware, there is ignorance. When awareness arises, right
understanding begins to develop.''

Acharn Sujin said that right understanding of the object of mindfulness
is very important. For example, we should know what seeing is. Often it
seems that we see people and things, but that is not seeing. It is
paying attention to shape and form, which is thinking of concepts. We
should not be discouraged about our ignorance. The characteristics of
the realities which appear have to be investigated with great patience
so that understanding can grow. Gradually we can learn that seeing is
not paying attention to shape and form, and that seeing is different
from visible object.

When awareness arises, it is aware of only one object at a time; at that
moment one is not confused as to the distinction between different
objects. One does not confuse seeing with visible object, or seeing with
paying attention to shape and form.

Captain Perera and Sarah saw me off at the airport. On the way to the
airport we saw the illuminations and the statues people had put up for
the celebration of Vesakha. At the airport Sarah reminded me that when
we think of the people we are attached to and of the country we like, we
think of concepts and our attachment conditions unhappiness. But if we
realize that life exists in only one moment of experiencing an object
and that this moment falls away immediately, we have more understanding
of reality. Sarah said:

``Sri Lanka and all the people we are attached to, all the last five
weeks, it is all in just one moment now, one thought now, and then
gone.''

Life exists in only one moment, the present moment.




