\chapter[The best time to understand Truth]{}
\section*{The best time to understand Truth}

Most precious in life is understanding whatever reality appears now.
Most of the time we are thinking of concepts and ideas about people and
things that seem to stay. Through the Buddha's teachings we learn that
realities such as seeing, feeling or hardness only arise for an
extremely short moment and then fall away. There is no self who can
cause their arising or control them.

Because of ignorance we are misled about the truth of realities that are
impermanent, unsatisfactory (dukkha) and non-self, anattā.

Without the Buddha's teachings no one knows that life is seeing,
hearing, smelling, tasting, experiencing tangible object through the
bodysense and thinking. Each moment of consciousness or citta
experiences only one object at a time and then it falls away, never to
return.

When we are convinced that hearing true Dhamma and right understanding
are the most valuable in our life we appreciate the opportunity for
listening we still have in this human plane of existence.

Most of the time we are forgetful of what life really is: moments of
seeing, hearing or thinking which arise and fall away immediately. We
keep on thinking of different situations which seem to last, of people,
of concepts that are not realities.

Throughout our sessions there were dialogues with children and adults
about the realities of life. These were most useful for all of us. In
this way we were reminded about what is really true.

When we were staying in Nakorn Nayok, Sarah had a lively conversation
about realities with Vincent's twelve year old daughter Nana who lives
in Taiwan. They talked about the different feelings that arise. When we
hear kind words there is mostly happy feeling, whereas when we hear
unkind words unhappy feeling is bound to arise.

Nana was asked what kind of feeling arises when seeing or hearing. There
is not happy feeling nor unhappy feeling but indifferent feeling. She
was asked what feeling arises when one is doing good deeds. Is it not
usually happy feeling? Feelings are different at each moment. Sarah
asked Nana whether it is ``you'' who feels pleasant or unpleasant;
actually, what is the reality that feels it?

Nana answered: ``Only feeling.''

There are two types of reality: one type of reality experiences
something, it is nāma, and one type does not experience anything, it is
rūpa. Conciousness, citta, and its accompanying mental factors,
cetasikas, experience different objects. Sound or hardness are rūpa,
they do not experience anything.

Sarah asked Nana why feeling is nāma. Nana answered that it is nāma
because it experiences things.

We are forgetful of the fact that feeling feels, that not a self feels.
We are led by feelings. We find it very important how we feel in a day.
We are distressed when others do not treat us well. We think that others
are to be blamed instead of understanding that the real cause of our
distress is within us. Whatever occurs is conditioned by many different
factors, and nobody can be master of whatever happens in our life.

When seeing right now we always have the idea of ``I see''. Seeing is a
citta that is accompanied by mental factors, cetasikas, which condition
it. Citta is the leader, the principal, in experiencing an object and
the accompanying cetasikas experience the same object but they each
perform their own function. Citta and cetasika condition each other by
conascence-condition. They arise at the same physical base, and fall
away together at the same time. Some cetasikas accompany every citta,
such as contact, feeling, remembrance or concentration. Contact (phassa)
contacts the object of citta so that citta can experience it.
Remembrance or recognition, saññā, marks or recognizes the object citta
experiences. Concentration, ekaggatā cetasika, focusses on the object
and it is the condition that each citta experiences only one object at a
time. When seeing arises it experiences only visible object, when
hearing arises it experiences only sound. Some cetasikas are beautiful
and some are akusala and they accompany different cittas. Actually,
citta itself is not kusala or akusala, its function is just to
experience an object. The accompanying akusala cetasikas or kusala
cetasikas cause citta to be akusala or kusala.

Sarah asked Nana: ``What is the best time to understand the truth?''
Nana gave the right answer, saying: ``Now''.

We are inclined to think of the past but this has already gone, never to
return. When we are thinking of the future, of what we are going to do,
we are quite absorbed in concepts and we are forgetful of the thinking
itself that is a reality which is conditioned, non-self. We forget that
the only time reality appears is now. This is the time to understand its
true nature.

Seeing appears now. We can learn that there are conditions for its
arising. It is conditioned by eyesense and visible object. Seeing is a
citta that is result, vipāka. It is the result of kusala kamma or
akusala kamma. When it is the result of kusala kamma it experiences a
pleasant object and when it is the result of akusala kamma it
experiences an unpleasant object. Nobody can select the object that is
experienced. Having more understanding of the conditions for the
realities that arise and learning more details helps us to realize that
there is no self who can direct realities or cause their arising. Seeing
is often dealt with as an example, but when we have understanding of the
conditioned nature of seeing we shall also have more understanding of
the other realities that arise.

Each conditioned reality arises for an extremely brief moment and then
it falls away. When we think of the situations of life and of people it
seems that people and things stay on. We are often lost in stories
instead of knowing what is real at this moment. Without the Buddha's
teaching we do not know what life really is. Life is seeing, hearing,
thinking, and they all fall away immediately. Life is only one moment of
experiencing one object at a time.This does not mean that we should not
think of concepts of people. We should not try to change our way of
thinking, but there can be more understanding of what life truly is.

Nana had prepared some of her own questions. She asked what ``nimitta''
(sign) is. While we are seeing now there are numerous cittas arising and
falling away in succession, very rapidly. Countless visible objects are
experienced by moments of seeing-consciousness and what is known is only
the sign of visible object. There is a sign or mark of visible object
that has fallen away. It is impossible to experience just one visible
object since citta that is seeing-consciousness arises and falls away
etremely fast, but a sign of the characteristic of visible object can be
experienced. The sign of visible object that remains is like a shadow of
visible object. As was explained before, the experience of the nimitta
can be compared to a burning torch that one swings around so that a
circle of fire appears. There seems to be a circle that remains but in
reality there is no circle.

The sign or nimitta of what arises and falls away very fast appears as
if it is still there. On account of the nimitta of realities we think of
concepts or ideas. At this moment a dhamma such as visible object
appears for an extremely short moment and then it falls away. But since
dhammas arise in succession there seems to be a continuity which is
steady, so that we take what is experienced for `some thing'. Evenso is
there the nimitta of citta, viññāṇa-nimitta, that experiences different
objects. There is a nimitta of each of the five khandhas.

Nana asked what kamma is. It is cetanā cetasika, translated as intention
or volition. It accompanies every citta. It sees to it that the other
accompanying cetasikas each perform their own task. Cetanā that
accompanies vipākacitta, citta that is result, is different from cetanā
that accompanies kusala citta or akusala citta. When it accompanies
kusala citta or akusala citta and it is of sufficient strength, it is
able to produce the appropriate result. Kindness to others is wholesome
kamma. Nana agreed that knowing the truth is the best kamma.

The Buddha taught that all realities are non-self. However, because of
ignorance and clinging it is hard to accept the truth.

We read in the ``Kindred Sayings''( IV, Saḷāyatana vagga, Kindred
Sayings on Sense, Fourth Fifty, Ch III, § 193, Udāyin):

\begin{quote}
Once the venerable Ānanda and the venerable Udāyin were staying at
Kosambī in Ghosita Park. Then the venerable Udāyin, rising at eventide
from his solitude, went to visit the Venerable Ānanda, and on coming to
him \ldots{} after the exchange of courtesies, sat down at one side. So
seated the venerable Udāyin said to the venerable Ānanda:

``Is it possible, friend Ānanda, just as this body has in divers ways
been defined, explained, set forth by the Exalted One, as being without
the self \ldots is it possible in the same way to describe the
consciousness, to show it, make it plain, set it forth, make it clear,
analyze and expound it as being also without the self?''

``Just as this body has in divers ways been defined, explained, set
forth by the Exalted One, as being without the self, friend Udāyin, so
also is it possible to describe this consciousness, to show it, make it
plain, set it forth, make it clear, analyze and expound it as being also
without the self.

Owing to the eye and visible object arises seeing-consciousness, does it
not, friend?''

``Yes, friend.''

``Well, friend, it is by this method that the Exalted One has explained,
opened up, and shown that this consciousness also is without the self.''

(The same is said with regard to the other doorways.)
\end{quote}

``Owing to the eye and visible object arises seeing-consciousness'', the
Buddha explained. His words are very direct and impressive. Is there
seeing now? Acharn often asked this question during the discussions. It
brings us back to the present moment. There is seeing time and again but
without the Buddha's words we would never know the truth of seeing. It
does not arise because a self wants to see. Nobody can cause its
arising. It arises just for a moment and then it is gone, never to
return. Acharn emphasized again and again that we should carefully
consider each word spoken by the Buddha: ``Owing to the eye and visible
object arises seeing-consciousness.''

Before we went to Vietnam we were invited by Khun Duangduen to her home
in Bangsai. She received us with great hospitality and offered us a
luncheon in a restaurant located in the same area as her house. Here we
enjoyed an incredible variety of health foods. Before and after the
luncheon we had a Dhamma discussion in a small group while seated by the
waterside. Acharn explained about right awareness that should be very
natural. It does not matter whether or not it arises. Attachment to it
is bound to come in very quickly. At the moment of right awareness there
is no thinking about realities. Acharn said: ``Very natural means:
whatever appears by conditions can be object of right awareness.
Unexpectedly. It takes aeons and aeons for sati and paññā to develop,
from moment to moment.''

We discussed about seeing the danger of being in the cycle of birth and
death, saṃsāra. We may well read about the danger or discuss it but
there is not enough understanding of the danger of saṃsāra. I had a
dialogue with Jonothan about this subject:

Jonothan: ``Why are you interested in developing understanding?''

Nina: ``We want to know the truth.''

Jonothan: ``What is the truth we are going to know?''

Nina: ``The truth the Buddha taught about no self, the truth of all
realities. That is what we want to know.''

Jonothan: ``Is that dukkha (unsatisfactory)?''

Nina: ``All conditioned realities are dukkha. But we do not quite
understand dukkha, the impermanence of realities.''

Jonothan: ``All realities are dukkha, there is no future.''

Nina: ``Because they fall away. There is so much thinking about it.''

Jonothan: ``We come to understand this intellectually at first. The
cessation of dukkha takes place only at arahatship, parinibbāna, the end
of saṃsāra.''\footnote{The arahat has
eradicated all defilements and, therefore, he does not have to be
reborn. He has reached the end of dukkha, the arising and falling away
of realities.}

Nina: ``There is hearing, thinking, but not understanding. It is so much
theory for me.''

Jonothan: ``It is theory for all of us, when understanding is of the
intellectual level. You asked about the connection with seeing the
danger of saṃsāra.''

Acharn Sujin: ``We should not try to see the danger, but it is by
understanding on and on and on\ldots Ignorance cannot be eradicated
all at once.''

Acharn explained that that the actual moment of seeing the danger of
life must be vipassanā ñāna\footnote{This is paññā of the
level of direct understanding of realities.}. One can see the danger
of akusala dhammas in daily life. Realizing the danger of life has to be
from now on. There is nothing that does not arise and fall away from
this morning up to now. There is no one there.

I said that realizing this from now on is very hard. Acharn answered
that it is not a matter of self, but of citta and cetasikas. When I said
that we just listen and consider, she answered: ``Citta and cetasikas.
They develop.''

It is beneficial to be reminded that citta and cetasikas perform their
tasks, not a self who can do anything.

Sarah reminded me that wise consideration is paññā that develops
naturally and easily. She said: ``It is not `how can I understand', or
trying to work it out. This is natural, but self, clinging to the self
gets in the way all the time and hinders the natural considering or
understanding.''

Acharn emphasized again: ``Paññā begins to listen carefully, consider
carefully, and in that way it develops.''

\chapter[The Functions of Citta]{}
\section*{The Functions of Citta}

Seeing at this moment is life. It is conditioned, no one can make it
arise. Each moment that arises and falls away is life. As Acharn said,
``Without understanding reality right now, there is no understanding of
life.'' Life is not permanent; each reality that is conditioned to
arise, falls away in splitseconds. Nobody can stop the succession of
realities that arise and fall away. As Acharn said, ``It has gone
completely. So rapid, it seems like it is permanent''. The citta that
has fallen away is succeeded immediately by the next citta and, thus, it
seems that citta can stay.

When citta arises, it performs a function. Seeing is a function of
citta. Seeing is vipākacitta, result of kamma. When it experiences a
pleasant object it is the result of kusala kamma, and when it
experiences an unpleasant object it is the result of akusala kamma. We
should not try to find out whether seeing is kusala vipākacitta or
akusala vipākacitta, it is just one short moment that falls away
immediately.

Different cittas perform different functions. The term ``function'' (in
Pali: kicca), helps us to understand that there is no self who acts, but
that it is citta that performs a function. Seeing arises in a process of
cittas which each perform their own function. Before seeing arises,
there has to be adverting-consciousness (āvajjana-citta) which does not
see but just adverts to visible object that impinges on the eyesense.
Then seeing arises and after it has fallen away it is succeeded by two
more vipākacittas which receive the object and which investigate it.
Then determining-consciousness (votthapana-citta), which is kiriya-citta\footnote{Kiriyacitta is neither
kusala citta nor akusala citta, nor vipākacitta.}, arises. This is only one
moment of citta that will be followed by kusala cittas or akusala cittas
that perform the function of ``javana'' or ``going through'' the object.
It depends on accumulated conditions whether kusala citta or akusala
citta arises. There is no one who could determine this.

Birth is a function of citta and dying is another function of citta.
Birth-consciousness (paṭisandhi-citta) and dying-consciousness
(cuti-citta) are results of kamma. No one can condition
birth-consciousness to arise in a specific place or country. No one can
know when death will come. It depends on kamma when a lifespan will be
ended. It may seem unusual to see dying as a function of citta. But this
is reality. There is no person, only citta, cetasika and rūpa arising
and falling away. Dying-consciousness is the last moment of a life-span
which is followed immediately by the rebirth-consciousness of the
following life. This is just like now: seeing arises and is immediately
followed by the next citta in the process which is
receiving-consciousness: it does not see but still experiences visible
object while receiving that object. There is life and death at each
moment, when a citta arises and falls away.

Cittas can be kusala, akusala, vipāka or kiriya. These are the four
``natures'', ``jātis'', of citta. This was also a topic discussed during
the sessions. In a process of cittas there are not only kusala cittas,
akusala cittas and vipākacittas, cittas that are result, but also
kiriyacittas. Kiriyacitta is not kusala, akusala or vipāka. Before
seeing arises in a process of cittas, there is a kiriyacitta, the
eye-door adverting-consciousness that adverts to visible object. Another
kiriyacitta in the process of cittas is the determining-consciousness,
votthapana-citta, arising after seeing and the two vipākacittas that
follow upon seeing. The determining-consciousness that is only one
moment, is succeeded by kusala cittas or akusala cittas.

It depends on the appropriate conditions of what jāti a particular citta
is. Acharn said: ``There are four jātis every day. One can begin to
understand that there is no one, no self.''

Citta and its accompanying cetasikas are of the same jāti. They
condition one another by conascence-condition, meaning, they arise and
fall away together. Thus, when the citta is kusala, all accompanying
cetasikas are also kusala, and it is the same in the case of akusala
citta, vipākacitta and kiriyacitta. When we are helping someone else
with kusala citta the accompanying feeling which may be pleasant feeling
or indifferent feeling is also kusala. When the kusala citta falls away
there may be akusala citta with clinging to the person we are helping or
to an idea of ``my pleasant feeling''. In that case the accompanying
feeling which may be pleasant or indifferent, is also akusala. We may be
ignorant of realities believing that the feeling that is kusala stays
on, whereas, in reality, it is akusala. It is not self but just feeling
that feels and it only lasts for an extremely brief moment. When we
consider realities wisely it will be clearer that feeling is only a
conditioned dhamma.

Cittas arising in processes experience an object through one of the six
doorways, the doorway of the eye, the ear, the nose, the tongue, the
bodysense or the mind. Only one object through one doorway at a time can
be experienced. Life is the experience of one object and then it is
gone, never to return. Each citta that falls away conditions the arising
of the next citta. This is proximity-condition, anantara-paccaya. Cittas
arise in succession, and also in a specific order. When the
adverting-consciousness has fallen away it is succeeded by seeing or one
of the other sense-cognitions. This order cannot be changed.

Seeing sees only visible object, but very soon after seeing we think of
people and things. Cittas succeed one another so rapidly that it seems
that we can see and think all at the same time. However, we could not
think of persons and things if there had not been seeing. We may
understand this intellectually, but it still seems that we see and
recognize people and things at the same time.

We usually think of people and events with akusala citta, with
attachment or aversion. One may wonder whether one should change one's
way of thinking. We should not try to change our way of thinking, that
would be unnatural. When we try to change our thinking there is clinging
to an idea of self who can do so. Thinking is also a conditioned
reality. We can learn that it is different from seeing or hearing. When
we learn about different realities, there will be more confidence in the
Buddha's teaching of non-self.

There are not only cittas arising in processes, in between the processes
of cittas there are cittas that are life-continuum, bhavanga. They keep
the continuity in the life of an individual. The bhavanga-citta does not
experience an object through a doorway. It experiences the same object
as the rebirth-consciousness, throughout life. Kamma produces the first
citta in life, the rebirth-consciousness which is vipākacitta and it
experiences the same object as that experienced shortly before dying.
The rebirth-consciousess is succeeded by bhavanga-citta. When we are
fast asleep and not dreaming bhavanga-cittas arise. We do not know any
object through the sense-doors or the mind-door, we do not know where we
are and who we are, who our parents are. Acharn asked us: ``Is there
anyone at the moment of being fast asleep? Where are you, your property,
family and friends?''

When we wake up cittas arise again that experience objects through one
of the six doorways.

The cycle of birth and death goes on endlessly, consisting of cittas
arising in processes and cittas that do not arise in processes,
bhavanga-cittas. Dying-consciousness, cuti-citta, is the last citta of
this life-span. It is of the same type as the bhavanga-citta and it
experiences the same object.

Acharn asked whether there is seeing in a dream. We think of many
stories, when dreaming. She explained:

``At this moment there is seeing and memory of people and things. Just
like in a dream. What is the difference between moments of dreaming and
of being awake? There is no seeing in a dream. Memory makes it seem so
real. When one wakes up where are those that appeared in a dream?''

Nothing can arise without conditions. But it is very difficult to
understand conditions. One is occupied with oneself and this causes many
problems. I worry about what will happen to me, tomorrow, or after
tomorrow when I have to travel. I remarked that it is so good to be
reminded that one is occupied with oneself, to learn about conditioned
realities that cannot be controlled. They are all conditioned dhammas.
This is a great support when we face problems. Also akusala citta is a
reality of life and it should be understood as conditioned, non-self.

We read in the ``Kindred Sayings'' (III) ``Kindred Sayings on
Elements'', The First Fifty, § 8, ``Grasping and Worry''(2), that the
Buddha, while he stayed at Sāvatthī, said:

\begin{quote}
    
``I will show you, brethren, grasping and worry, likewise not grasping
and worrying. Do you listen \ldots{}

And how, brethren, is there grasping and worry?

Herein, brethren, the untaught many-folk have this view: `This body is
mine: I am this: this is myself.'\footnote{These three ways of
clinging denote: clinging without wrong view, conceit and clinging with
wrong view.} Of such an one the body
alters and becomes otherwise. Owing to the altering and otherwiseness of
body, sorrow and grief, woe, lamentation and despair arise in him \ldots''

\end{quote}

The same is said about the other four khandhas, the khandha of feeling,
of perception (remembrance or saññā), of saṇkhāra-khandha (the other
fifty cetasikas, apart from feeling and remembrance) and of
consciousness (viññāṇa).

\begin{quote}
    
The Buddha said:

``And how, brethren, is there no grasping and worry?

Herein, brethren, the well-taught Ariyan disciple has this view: `This
body is not mine: I am not this: this is not myself.' But inspite of the
altering and otherwiseness of body, sorrow and grief, woe, lamentation
and despair do not arise in him\ldots{}''

The same is said of the other four khandhas.
\end{quote}

Only the arahat, the perfected one has eradicated all kinds of clinging
and, thus, he has no grasping and worry.

Sarah had a dialogue with Trung, the brother of Tran Thai's wife Tiny
Tam. Sarah asked him what is seen now and he answered that it was people
and things.

Sarah: ``Just visible object that is seen. Afterwards we think of people
and flowers. What sees visible object?''

Trung: ``The `I',''.

Sarah: ``Is the `I' that can hear the same as the `I' that can see?''

Trung: ``Yes.''

Sarah: ``Is that `I' real or just imagination?''

Trung: ``An idea.''

Sarah: ``Just an imaginary `I', no `I' to be found. Does that make
sense?''

Trung: ``Yes.''

Later on Sarah had another dialogue with Trung and it appeared that he
had considered realities more.

Sarah: ``Eyesense cannot experience anything, only citta can experience
something. So visible object contacts the eyesense and seeing sees
visible object. Without these two seeing cannot see.

What is heard now?''

Trung: ``Sound.''

Sarah: ``What hears that sound?''

Trung: ``Hearing-consciousness.''

Sarah: ``What is seen now?''

Trung: ``Light or visible object.''

Sarah: ``Light or visible object, just that which is seen.

When there is the idea of people and flowers there are just ideas that
are thought about. In dreams thinking is real, but are people, flowers,
strange events real or imagined?''

Trung: ``Imagined.''

Sarah: ``Seeing is real, light or visible object is real, thinking is
real. Flowers and people are ideas. Is this interesting or useful?''

Sarah said that she appreciated his interest. Trung came to the sessions
also the following days and he listened attentively to the discussions.

This was a dialogue about all the different realities that appear in a
day and we have heard about this many times. However, it is always
useful to be reminded again and again about what is real since our
ignorance and attachment are deeply rooted. It will take aeons of
listening and carefully considering realities before there will be clear
understanding of realities as impermanent and non-self. It seems, when
we think of our house, that our house is still existing and belongs to
us. Acharn frequently reminded us of the truth. She said:

``You think of your house. Where is your house? You are just dreaming
about your house. Can one say that life is just like a dream, no matter
at night or at day time. There is nothing in a dream. What is the use of
clinging to what appears very shortly.'' We should not try to stop
dreaming, or wish for life to be different. Life should be understood.
Thinking is real, but the stories one thinks of are not real.

During the sessions the listeners had many questions and several of
these pertained to calm as a method to have more kusala. People believed
that if one goes to a quiet place in order to have more loving kindness,
mettā, it would arise more frequently. However, from the beginning it
should be understood what calm is and what mettā is. Mettā can arise
naturally in daily life. At the moments of kindness there is calm
already. Calm (passaddhi) is a cetasika arising with each kusala citta.

One point that was raised was that the thought of foulness of the body
will eliminate lust or attachment. There will not be true calm if one
tries to cause citta to be in a certain way, motivated by an idea of
self. One should know that whatever citta arises is conditioned, and
that nobody can control citta. Many akusala cittas arise in our life and
they can be understood as non-self. They are all dhammas. Recollection
on the foulness of the body is one of the subjects of samatha, the
development of calm. If one understands the right conditions for calm
there can be temporary release from attachment. But if one just focusses
on foulness without right understanding of what true calm is, it will
not be beneficial.

People raised questions as to paying respect to one's ancestors. This is
a duty stemming from the Vietnamese tradition. It is not helpful to
speculate about one's ancestors who could notice one's good deeds done
in order to honour them. When the dying-consciousness has fallen away
and it is succeeded by the rebirth-consciousness, there is no longer the
same individual. However, one can be grateful for what they have done in
the past and think of them with kusala cittas. People also wonder how to
pay respect to the Buddha although he has passed finally away. The best
respect to the Buddha is to study his words and develop understanding of
what is real. Now is the way to show respect to him.

Someone mentioned a Mahāyāna belief that there is a joint or collective
kamma. A group of people receives the same kind of result due to kamma
that they have collectively performed. This is not possible because each
person will receive the result of a deed he has performed himself. He is
heir to his own kamma, as the Budha said. Seeing arises now and it is
the result of past kamma, a deed one has performed oneself.

Each of the questions raised can be brought back to the present moment.

We are thinking about concepts and situations time and again and then we
are forgetful of what is true in the ultimate sense.

When we hear the words ``being alone'' we tend to think about this in
conventional sense. Acharn reminded us of the real meaning of being
alone, of the fact that there is no one there. She said:

``There is always thinking about situations. Each moment is alone.
Seeing is alone. Just live alone. There is thinking about people, but
thinking is alone. Just alone each moment. This will lead to less
attachment. There is no one with us, citta just arises and falls away.
Where are those people we are thinking about. We are happy being with
friends, but this arises alone.''

Citta is always alone, there is nobody there when seeing arises. Besides
citta, cetasika and rūpa there is no person.

When intellectual understanding, pariyatti, develops it will become
quite firm and there will be more confidence in the truth. Then it can
condition right awareness and direct understanding of realities,
paṭipatti. Without confidence there are no conditions for direct
understanding or satipaṭṭhāna. We may be attached to the idea of
pariyatti that has to become firm and condition paṭipatti. Each person
has to find this out by himself, but it is only pañña that knows.
Pariyatti has become really firm when there is no self involved. Instead
of thinking too much about pariyatti and paṭipatti, it would be better
to attend to what is appearing now. This is what is really important.
There is no need to think about terms like pariyatti and paṭipatti, or
all the other terms one finds in the texts. What appears now, just now?
Is it seeing, hearing, hardness or thinking? This is the only way
leading to the development of paññā. Paññā is only an element, devoid of
self.

From the beginning there should be more and more understanding that no
self studies, considers and develops intellectual understanding. This is
already difficult, because the idea of self comes in unnoticed. This is
so because of ignorance and attachment that has been accumulated for
aeons.

Some people believe that they have reached already the stage of
paṭipatti, direct understanding. They believe that they can be aware of
seeing, hearing or thinking. Someone mentioned that there is a fine
borderline between thinking and direct awareness. However, only paññā
can know about the difference between the moments of right awareness and
the moments that there is merely thinking of awareness. It is of no use
to try to find out whether awareness has arisen. Right awareness and
right understanding of the level of satipaṭṭhāna arise together and they
do so because of their own conditions. Nobody can select the object they
take. Whatever reality appears, be it kusala or akusala, pleasant or
unpleasant, can be the object of sati and paññā.

\chapter[What is Death]{}
\section*{What is Death?}

Nothing lasts, there is death at each moment in life. When a citta
arises and falls away there is birth and death in the ultimate sense.

Two boys, Duc, sixteen years old, and Tri, twelve years old, were very
sad because their grandmother had just passed away. Sarah had a dialogue
with them:

``When seeing or hearing now there is no sadness. At the moments of
helping or giving, there is no sadness. Your grandma started a new
journey. What she would wish is that you study, help your parents and
study Dhamma.

The world before hearing the teachings is the whole world. But after
hearing the teachings it is that which arises. Each world is different:
seeing is not hearing. After it has arisen it is old and then it passes
away, never to return. We can think about the world in the sense of
whatever appears now.

Understanding what the Buddha taught brings about respect to anything he
taught. He is the person who enlightened the truth of everything.''

Sarah asked them whether they understood this a little bit and it
appeared that they did. In the case of death and mourning it is
important to consider realities. Instead of thinking again and again
about the dear person one lost and one's sadness it is good to be
reminded of what is really important in life. Mostly we are involved in
stories and concepts instead of considering realities wisely.

Sarah said:

``Is there seeing now? Does it last for a little time or for a long
time? It seems to last for a long time but everything passes very
quickly. What about unhappy feeling? A little while ago you told me you
had unhappy feeling and now you are smiling. So, does unhappy feeling
last? Everything seems to last. We learn a little bit more and then we
understand a little bit more. We learn that it passes very quickly. What
seems so important now will be forgotten very quickly.''

Sarah explained that there is just this moment now and the boys agreed.
They understood that it is useful to find out what life is at this
moment. When we help people to consider the present moment they will be
less inclined to think with sadness about their loss. It is natural to
be sad about a loss, but one can learn that even sadness is a
conditioned reality that does not last. We are attached to pleasant
feeling and when we lack the company of a dear person we cry. We only
think of ourselves.

Sarah explained that when holding the microphone it is not the
microphone that is touched but just tangible object such as hardness or
temperature. The world the Buddha taught is just what is seen now, heard
now, touched now. One may think of one's deceased grandmother but what
is real is just thinking now and then gone.

Duc said that he had regrets about his wasting of opportunities for
kindness towards his grandmother and he was wondering whether he could
come into contact with her in order to correct his former attitude.
Jonothan explained that this is not possible and that it is not useful
to have regrets about the past. One cannot always express how one feels
about someone who is close to us.

I said to them that their interest and the way they asked questions was
helpful to all of us. It was an opportunity to exchange thoughts about
realities, about what is real now. Their dialogue with Sarah helped all
of us to consider more ourselves what is real in life. We should
consider more seeing and hearing. What is touched is not a thing like a
microphone, but a reality such as solidity. We should have more
understanding of what is real and what is only a thought or idea, which
is not real.

I said that if their grandmother could know about their interest in the
Dhamma she would really be happy. Understanding about the truth can
develop very, very slowly. It is normal that it grows so slowly. It may
take many lives, not just one life. It is good that understanding can
begin just now.

The boys were shedding tears but in the course of the conversation about
realities they dried their tears. The next day the cremation took place
and a friend who attended the ceremony said that the boys were smiling.

When we reflect wisely about death, knowing that it can come any time,
we can realize what is precious in life: just understanding this moment.
We should remember that there is birth and death at each moment when a
citta arises and then falls away. It is very natural to be involved in
stories while thinking of grief caused by the loss of a dear person.
However, it is beneficial to know the difference between thinking of
situations and understanding the realities of citta, cetasika and rūpa.
We do not try to change thinking of situations, but understanding can be
developed of realities. In the ultimate sense there is no dear person
who passed away nor people who are grieving. As we heard so many times
these days: there is no one there. Our life is only citta, cetasika and
rūpa. Through the Dhamma we learn to see our life in a different way.

Sarah reminded us that we are not crying for the beloved one but for
ourselves. We are so attached to the pleasant feeling we derived from
his or her company. One by one our dear ones will pass away. She said:
``Wise people who are courageous will understand life as it is: however
life goes, facing difficulties, there are still just dhammas: citta,
cetasika and rūpa arising and falling away.''

We were reminded time and again that no matter life is happy or unhappy
now, the only thing that matters is understanding of the realities
appearing at the present moment. It is useless to wish for more calm and
more kusala.

Kusala citta with calm can only arise when there are the right
conditions.

Just now so many things seem to appear at the same time. All this, we
call the world, appears because citta experiences different objects.
Through the Buddha's teachings we learn that there is one citta at a
time, an extremely short moment. It arises and falls away never to
return. We are misled by the outer appearance of things. We seem to see
continuously, hear continuously. But we learn that, time and again,
there is a different citta that experiences an object through one of the
six doorways, of the senses and of the mind-door.

Whatever subject we consider, we have to distinguish the world of
concepts, of people, I, you, different things that seem to last, and the
real world of ultimate realities. Otherwise we shall not know the truth.

When we think of death we usually think of death in conventional sense,
as the end of a lifespan. But when we consider the arising and falling
away of citta, there is actually birth and death of citta.

Acharn said that in childhood she was very afraid of death. But when she
understood that there is momentary death of citta all the time, all fear
disappeared. She said:

``But now there is each moment. Nothing to be afraid of. Are you afraid
of tomorrow, are you afraid of death? So, understand the death
of reality which arises and falls away, never to return, and that is the
real meaning of death. So, is there time to be afraid of death anymore?
Because it is just right now. When there is more understanding there
will be less sorrow or unpleasant moments\ldots{} Great sorrow will come
from the dear one's death. One can understand what conditions sorrow,
the unknown attachment that arises most in one's life.''

Attachment is mostly not known. When it is strong such as greed for a
delicious meal, we may notice it. But often there are countless more
subtle moments such as attachment to seeing, to what is visible, to the
eye.

We were reminded all the time that the difference between seeing and
visible object, between nāma and rūpa, has to be clearly understood so
that realities can, one at a time, be directly known. But we should not
mind if this is not yet the case. At the moment of right understanding
there is also patience.

Acharn said: ``What is there now. Is there seeing or what? Seeing is
conditioned, it cannot arise by itself. It is conditioned just to arise
and see, that is all. No matter what you call it.''

Those are simple words, but did we really consider them enough? No
matter what we call seeing, this reality appears and can be understood.
We can call it by any name, in any language, but its characteristic does
not change. It experiences visible object. It is an ultimate reality,
different from conventional notions.

Acharn said: ``It is gone completely, never to return in saṃsāra or
anywhere in the world. No matter what you call it, is it true?''

A precious moment is understanding what has not been understood before.
If one has not listened to the Buddha's teachings there will always be
ignorance covering up the truth. We were reminded time and again that
when there is ignorance, seeing is not understood as a moment of seeing,
just a conditioned dhamma. We tend to think of seeing as if it is
lasting. It seems that we see and perceive people and things at the same
time. When intellectual understanding has conditioned direct
understanding of the level of insight knowledge, vipassanā ñāṇa, the
arising and falling away of realities, one at a time, can be clearly
understood. Direct awareness, sati of the level of satipaṭṭhāna, and
direct understanding arise together at those moments.

The reality of mindfulness or awareness, sati, was often discussed. Some
people believe that mindfulness is knowing what one is doing. However,
it has a reality as object, not a situation or a conventional idea one
may think of, such as an idea of doing the washing, of eating or of
walking. Or one may think: ``I am aware''. Awareness cannot arise with
an idea of self. People may have misunderstandings when they read about
applying knowledge during awareness. It entirely depends on conditions
what the object of awareness and understanding will be. We learn from
the teachings that no self can apply anything, but there may still be an
idea of myself applying, even if we do not expressively think this in
words. From the beginning there should be more and more understanding
that no self studies or considers realities. This is already difficult,
because the idea of self comes in unnoticed. This happens because of
ignorance and attachment that has been accumulated for ages.

People confuse mindfulness with an idea of concentration. Mindfulness or
concentration are terms one thinks of in conventional sense, as if they
are lasting. The Buddha taught very precisely about them as specific
cetasikas, dhammas arising because of the appropriate conditions that do
not last.

Concentration or one-pointedness (ekaggatā cetasika) accompanies each
citta and it is the condition that citta experiences only one object at
a time. Seeing only experiences visible object, it cannot know any other
object. It can be kusala, akusala or neither kusala nor akusala. Sati is
a sobhana (beautiful) cetasika that can accompany only sobhana cittas\footnote{Sobhana cittas include
not only kusala cittas but also kusala vipākacittas accompanied by
sobhana cetasikas and kiriyacittas of the arahat accompanied by sobhana
cetasikas.}. Sati is non-forgetful of
what is wholesome. There are many levels of sati: sati of the level of
dāna, of sīla, of samatha, of tranquil meditation or of vipassanā.

Acharn emphasized that paññā can know the difference between moments of
sati and moments without sati. Only paññā that has been developed more
can know the difference. She said: ``Sati and paññā are not you. If
there is only intellectual understanding the characteristics of sati and
paññā cannot be known.''

Acharn explained that when there is understanding of the level of
pariyatti the object does not appear well, but when sati and paññā of
the level of satipaṭṭhāna arise the object of understanding ``appears
well''. The object cannot appear well in the beginning, when paññā has
not been developed enough. When understanding has been developed more,
the object of understanding begins to appear well. Then paññā will know
the difference between the moment of sati and the moment without it. She
said that it is a very short moment but that it is there. The
characteristic of the object is truly understood as not self, as an
element. This is paññā of a higher level, different from intellectual
understanding of an object. But it has intellectual understanding of the
present reality as a condition for its arising.

When understanding has been developed further to the degree that levels
of insight knowledge (vipassanā ñāṇa) arise the object appears clearer
and clearer.

Sati and paññā of the level of vipassanā can directly know the arising
and falling away of realities. Then the real meaning of momentary death
of citta will be understood.

\chapter[Every moment is present moment]{}
\section*{Every moment is present moment}

Understanding what has not been understood before is most valuable.
Without the Buddha's teachings we would always have an idea of ``I see,
I hear'', from life to life. He taught us that there is no self who is
there, no one who experiences, not a person or thing that is
experienced. Ignorance will always cover up the truth. During the
discussions we often heard the words ``seeing is not self'' and we
considered those words. But when seeing arises there is still the idea
of ``I see''. There will be this idea so long as the characteristic of
seeing does not appear to paññā that has been developed more. Even when
we do not expressively think ``it is me'' there is still the idea of
self. We only know this through the Buddha's teaching. The view of self
is only eradicated by the sotāpanna, the person who has attained the
first stage of enlightenment.

Seeing arises for one extremely short moment and then shortly afterwards
we think of people and things we believe that we see. The difference
between seeing and thinking should be known.

Acharn asked someone a few questions about seeing.

Acharn: ``When it is not `I', what is it?''

Answer: ``Seeing.''

Acharn: ``What is seeing?''

Answer: ``Citta.''

Acharn then explained: ``If you do not use words like citta or nature of
citta, there is a reality which sees. A reality which just arises to
see. The more there is the understanding of the reality which arises to
see, the more understanding develops. There is no one there and it falls
away instantly. When there is the idea of people and things there is no
seeing at that moment.''

Acharn often asks the listeners questions in order to help them to
consider the truth for themselves. The Buddha taught the Dhamma in such
way that people would develop their own understanding. Then they will
have more confidence in the teachings.

Acharn asks people why they want to study the Dhamma. It is just for the
sake of understanding the truth. The aim is not becoming a better
person, not trying to have more understanding; then one still thinks of
self. Understanding is not ``I''.

Life just exists in a moment. The moment of seeing is so short, and also
what is seen, visible object, does not last. She asked why one clings to
mountains, trees and people. What is the use of clinging to what is gone
immediately. Clinging accumulates and at such moments there is no
understanding of the truth. Through the development of understanding of
realities there can eventually be a little less attachment to what is
not worth clinging to.

Our whole life is the succession of realities that arise and fall away.
We were often reminded of citta that sees alone, hears alone or thinks
alone. We think of situations and stories about people, but the thinking
itself is alone. Citta arises and experiences an object and then falls
away immediately. Only one citta arises at a time; when seeing arises
there is no other experience at the same time. ``Even right now you are
alone. Citta just arises and falls away'', Acharn said. When we are
happy with friends, happy feeling arises alone. Citta is always alone,
each moment.

Someone may wonder whether helping or giving is meaningful if there is
no person. Here we have again two different worlds, the world of the
situations and persons, and the world of what is real in the ultimate
sense, citta, cetasika and rūpa.

We should consider citta. The citta that arises with generosity is
wholesome. It can think in a wholesome way. There is no need to think of
a person, it is the behaviour of citta, just for a moment, that matters.
Understanding of realities leads to more wholesomeness. It is purer. For
example, when we want to help others they may react with irritation.
That does not matter, we need not think of situations, like reactions of
others; there is the wholesome citta that arises and nobody can change
it.

We lead our ordinary life in different situations, with different
people, but in between thinking of situations, citta with right
understanding of what life really is can arise. Thinking is a reality
and the stories one thinks of are not real. By knowing the difference
between stories and realities understanding can grow and develop. Life
is really seeing visible object, thinking about it, hearing sound,
thinking about it, all different realities that do not last.

This cannot be clearly understood in the beginning. When paññā is more
developed to the stage of direct understanding of realities, it can
penetrate the arising and falling away of whatever appears now. Then it
becomes clearer that all we find so important only stays for a
splitsecond. It will be clearer that what appears now, just one moment
of seeing visible object, is real, and all concepts we think about are
just speculations, thoughts or ideas which are not real. There will
eventually be less ignorance and clinging to self.

After being in Hanoi for a few days we went to Mai Chau for a pleasant
stay in the country with Dhamma discussions. We visited a picturesque
village in Mai Chau Valley by an electric car where we had luncheon. We
passed a traditional ``homestay'', where guests stay together with the
family on the same floor. Those are simple houses of wood and bamboo,
usually without electricity. Women who look after their cattle just
continue with their needle work while walking on the street.

The dining room of the resort in Mai Chau was a roofed verandah, an ideal
place with fresh air for Dhamma discussions. Here Acharn had a
conversation with Tadao, our Japanese friend, about seeing and what is
seen. She explained the Dhamma in the way of asking questions and
would not stop until she had an answer to her questions. People may
answer what they have learnt from the texts without really investigating
the truth themselves. She would ask what is there when seeing. Everyone
should really find out by himself what reality is there.

She explained that in the beginning there should be understanding of
what is real and what is not real. What is seen, visible object, is
real, seeing is real. She asked: ``Can seeing which arises and falls
away be anyone? Is seeing real, can it be you?''

When we say that seeing is not ``I'' there is only an idea about
reality, it is not the actual moment when seeing arises. At that moment
we are usually forgetful and we do not investigate the truth for
ourselves. It has to be one's own understanding of whatever appears now.
She said that understanding of what appears now is the test of one's own
development of understanding.

She asked Tadao ``What is there now when seeing?'' She repeated this
question several times. Tadao said: ``Things''. Acharn then asked what
the thing is that can be seen. She said: ``That which can be seen is a
reality. It is real. Can anyone make it arise? No one can do anything at
all, right? What is seen now? What is that thing that is seen? You did
not answer my question. The question was about the thing, what is it?''

This way of question and answer helps to understand realities. One
should not just think about words and follow what one learnt from the
teachings. One has to investigate the reality appearing now.

People think that they see a table with four legs. Acharn said: ``You
talk about shape and form but it has to be seen. What is really seen
brings about the idea of four legs.'' She reminded us of the truth by
repeating that what has to be understood should be now, now. If there
would not be seeing of visible object there could not be thinking of a
table.

After hearing the teachings one begins to realize what is really seen.
The idea of people or table comes later on, after seeing. She said:

``There must be seeing, like a flash. Thinking comes later. We do not
talk about thinking, we talk about seeing before there can be the
understanding that there is nothing besides seeing and what can be seen.
So, no one there at all. This is the beginning of understanding that
there is no one. We learn to understand the absolute truth of each
moment as it is. At this moment of seeing there is no one. There are
only different realities.''

She said that there is the idea of self all the time but that one should
learn to understand what appears, little by little, in order to
eradicate the idea of self in that which is seen. She talked about
seeing again and again because it is the object of ignorance from aeons
ago.

She asked what nāma is. A reality that arises to experience something.
``Can anyone see nāma ?'', she asked, and the answer was no. She then
asked the same about hearing and the other sense-cognitions. When we
answer that seeing is nāma, not self, it does not mean that we have
really understood what nāma is. We have ideas about reality but there is
no direct awareness at the moment when seeing arises.

By her questions Acharn helps us to understand the difference between
intellectual understanding and direct awareness and direct understanding
of seeing or any other reality that appears now. Paññā that is developed
from moment to moment can clearly know realities that appear now. But as
soon as we are wishing for this knowledge, there is no understanding of
the fact that all realities of our life arise because of their own
conditions and are not self. When sati and direct understanding arise it
will be known precisely when there is wrong view.

The development of paññā to the stage of insight knowledge, vipassanā
ñāṇa, takes aeons. When it arises it knows directly the arising and
falling away of realities. Without direct awareness of seeing it cannot
be known that it arises and falls away. That is why Acharn often said
that paññā is not enough yet. However, understanding can begin to know
different characteristics of realities. If it does not begin now how can
awareness and direct understanding ever arise?

Jonothan reminded us repeatedly that the development of right
understanding does not mean that there will be less akusala, a change of
personality for the better and more calm. There is no point in wishing
for these things. No matter life is pleasant now or unpleasant now, the
only thing that is worth while is understanding realities that appear at
the present moment.

Jonothan said: ``Every moment is present moment''. There are no methods
to be followed for causing the arising of awareness of realities.

Acharn said: ``Why do you think of practising. There is no need to think
about it. Forget it.'' She reminded us that the aim of listening to the
Dhamma is just for the sake of understanding, nothing else.

There were several questions about the way to develop calm. One of the
listeners mentioned that a good location and confortable wheather would
be favourable conditions for the understanding of the Dhamma. Acharn
remarked: ``Why wait.'' One may have the idea that a quiet place is
necessary in order to develop understanding, but that is attachment and
this is not a helpful condition for understanding of whatever appears
now.

One may be attached to comfortable surroundings and wants to have only
pleasant feeling. One may have the illusion that one can control one's
life. We always dream of another place, thinking there will be more
awareness. We are forgetting realities at this moment.

Sarah remarked:

``Whatever arises does so by conditions, not by anyone's will. Can
anyone control what hearing hears, what is experienced through the
bodysense? It arises so quickly by conditions, long before anyone can
even think about it. When we try so hard to avoid unpleasantness, the
realities are attachment and the wrong idea of `I can
control' ''.

Someone said that one should meditate before listening to the Dhamma. By
concentration on breath he would become calm and this would help to
solve the problems of life such as stress. One dislikes stress that
arises when there is aversion (dosa), accompanied by unpleasant feeling.
However, one may not know when there is attachment to pleasant feeling.
The real cause of problems in life are ignorance and attachment. No
matter aversion, attachment, ignorance, whatever dhamma arises, they are
all conditioned realities. It is an illusion that one can control one's
life just by concentration on a subject like breath without any
understanding of what true calm is. Calm, the cetasika passaddhi, arises
with each sobhana citta (beautiful citta). True calm as it is developed
in samatha accompanies right understanding of the way how to develop it
with a suitable meditation subject.

Jonothan explained that the beginning of samatha, the development of
calm, is knowing the difference between kusala and akusala. Without
knowing the difference there cannot be the development of kusala. Some
people believe that they should concentrate on a meditation subject such
as in-and-outbreathing but they may not have any understanding of what
breath is nor of kusala citta and akusala citta.

Kusala citta is always accompanied by the wholesome roots\footnote{Some cetasikas are
roots, hetus, which are the foundation of the beautiful citta or the
akusala citta.} of non-attachment and
non-aversion and it may be accompanied by understanding as well. Akusala
citta is always accompanied by the root of ignorance, moha, and it may
at times by accompanied by attachment (lobha) as well or by aversion
(dosa) as well. There are many degrees of these roots and one may not
know them when they are more subtle. For instance, attachment is not
only desire for a beautiful painting one wants to have, it is also
attachment to seeing right now or to bodily wellbeing, moments that are
not known. Aversion is not only anger or sadness, but it can also arise
when there is a slight feeling of uneasiness. Ignorance follows very
often seeing, hearing and the other sense-cognitions. It is ignorance of
what realities are.

Jonothan explained that non-attachment or detachment arises naturally,
by the development of understanding, after a long time. By thinking
about detachment and by trying to have it, there will not be any
detachment. He said:

``Samatha begins to develop when kusala citta arises naturally in one's
daily life. That is when the characteristic of calm that is kusala can
be known. If one thinks to just take an appropriate meditation subject,
such as in-and-out- breathing or a kasina\footnote{A coloured disc, or a
disc made of earth.} it is wrong. Taking a
particular meditation subject cannot make the citta kusala. That is why
Acharn says that right understanding is necessary for both samatha and
vipassanā.''

The development of samatha has to be right from the beginning. The
development of samatha is not necessary for the attainment of
enlightenment. Many people became enlightened without any attainment of
jhāna.\footnote{See the Susima Sutta (S
II, 199-23) and the Visuddhimagga (666-67) which deals with dry insight,
sukkhavipassanā.}

There can be moments of calm in daily life and then the citta is for
that moment free from akusala, there is detachment. This is different
from trying to be detached. Sarah mentioned that there were conditions
to reflect naturally on death, because a friend's mother had passed
away. Such reflection may condition calm. She said that one can reflect
that life is short: ``Death can come at each moment. There can be
calmness. What is precious in life is understanding at this moment.''

\chapter[The right Path and the wrong Path]{}
\section*{The right Path and the wrong Path}



Just now, at this moment, many things seem to appear at the same time.
All this which we call the world, could not appear if there were no
citta. Citta experiences an object. Through the Buddha's teachings we
learn that there is one citta at a time, an extremely short moment. It
arises and falls away never to return. We seem to see continuously, hear
continuously. But we learn that at each moment there is a different
citta that experiences an object through one of the six doorways, of the
senses and of the mind-door.

Whatever subject we consider, we have to distinguish the world of
concepts, of people, I, you, different things that seem to last, and the
real world of ultimate realities. Otherwise we do not know the truth.

The difference between realities and concepts cannot be grasped at once.
That is why there was so much repetition in all the explanations during
the sessions. We have accumulated ignorance and wrong view for aeons.

Realities can be directly experienced, but first more intellectual
understanding, pariyatti, is needed. We have to listen to the Buddha's
teaching who speaks about seeing, hearing and all realities. Are they
permanent or impermanent he asked. Impermanence is not just thinking
about the fact that things do not last. The falling away of seeing or
hearing can be experienced but not immediately. It needs countless times
of considering what seeing is, as different from thinking about people.
What hearing is, as different from thinking about the rain or thunder
one perceives. Very gradually there can be a little more understanding
of seeing and other realities that appear without trying to know them.
That is why it is useful to discuss about seeing and other realities.

Thus, considering realities is most important. We should not have any
expectation that realities can clearly appear one at a time immediately.
Patience is important.

We are reminded all the time that the difference between nāma and rūpa,
such as seeing and visible object, has to be clearly understood so that
realities can, one by one, be directly known. When we take realities as
a ``whole'', as a collection of phenomena, there is no precise knowledge
of their different characteristics and we take them for self. But we
should not try to distinguish nāma from rūpa, that is done with an idea
of self. Understanding cannot be made to arise by an idea of self.

Acharn said: ``What is there now. Is there seeing or what? Seeing is
conditioned, it cannot arise by itself. It is conditioned just to arise
and see, that is all. No matter what you call it.''

Those are simple words, but did we really consider them enough? No
matter how seeing is called, this reality appears and can be understood.
We can call it by any name in any language, but its characteristic does
not change. It experiences visible object. It is an ultimate reality,
different from conventional notions.

Acharn said: ``It is gone completely, never to return in saṃsāra (the
cycle of birth and death) or anywhere in the world. No matter what you
call it, is it true?''

If one has not listened to the Buddha's teachings there will always be
ignorance covering up the truth.

Acharn said: ``What is most precious in life. Does one understand the
truth of seeing right now? Life goes on from moment to moment. There is
seeing now and it is gone completely. Everything arises and falls away
in splitseconds. Without the Buddha's teaching who knows that seeing
arises and falls away. The truth of life is not easy to understand. What
is precious is understanding what was not understood before. Everything
is gone. Can anything belong to you? Realities arise and fall away each
moment. When there is no understanding there will be the idea of `I see,
I hear', from life to life.''

There are many misunderstandings about practice. In the texts the word
practice is used and it is actually the translation of the Pali term
paṭipatti. There is nobody who practises, practice arises only because
of the right conditions. It is the development of direct understanding
and it is conditioned by intellectual understanding, pariyatti.
Intellectual understanding is not just study of texts and deep
reflection on what one has learnt. It is more than that, it pertains to
reality now, although we know that the understanding of the present
reality is not precise yet. It refers to the reality of this moment,
such as seeing that appears now. It can be understood as a conditioned
reality, not self. It is the beginning of understanding that there is no
one there.

As Acharn often explains: ``One begins to understand: it is this moment
only which is real. And then it is completely gone, there is nothing.
Seeing is gone, where is the `I' who sees? No one.''

When someone asked how understanding develops by reflection, she
replied:

``By understanding better and better, no other way.''

Pariyatti is not the direct understanding of realities, paṭipatti, but
it is the condition for it. Paṭipatti conditions paṭivedha, the direct
realization of the truth by the stages of insight knowledge\footnote{There are different
stages of insight, vipassanā. The first one is knowing the difference
between nāma, the reality that experiences an object, and rūpa, the
reality that does not experience anything. In the course of the
different stages of insight there is more and more detachment from
realities.} and enlightenment.

Some people think that they have to follow a meditation practice in
order to realize the truth, a special method such as focussing on a
subject like breathing. However, when on clings to an idea of self that
can cause the growth of paññā it will only lead to more clinging with
wrong view.

Some people think that there are specific times for practice, such as
sitting quietly. There is an idea of ``I am practising'' and they do not
realize that awareness and right understanding can only arise because of
their own conditions, not when there is an idea of self that can follow
a method to cause their arising. No one can cause the arising of any
reality, such as seeing or hearing, they arise naturally. Instead of
clinging to an idea of practice one should consider what the Buddha
taught about the realities that appear all the time, like seeing,
visible object, hearing, sound or thinking. When we carefully consider
the Buddha's words, for example that seeing arises because of
conditions, it can lead to the development of right understanding. But
there should not be any expectations that understanding can grow
quickly.

Retreats are organized in Vietnam and all over the world. People believe
that it is easier to develop understanding of one's citta when there is
silence. One can quietly follow different moments of thinking. Again,
one should find out whether one is motivated by attachment to the idea
of self who wishes to be wise and calm.

It is said that people's behaviour improves in a meditation center.
However, one cannot tell from someone's outward behaviour whether or not
there are kusala cittas. Kusala citta arises for an etremely short
moment and then falls away, and it is bound to be followed by akusala
cittas which are unknown. One may think of oneself as a good person, and
that is conceit. It is necessary to often hear about realities that
arise and fall away, otherwise there is the idea that one has to do
something special to have more calm and wisdom.

The right Path and the wrong Path was a subject discussed several times
during the sessions. There are misunderstandings as to the development
of the eightfold Path. The factors of the eightfold Path are sobhana
cetasikas which all develop together\footnote{They are: right understanding (sammā-diṭṭhi), right thinking (sammā-sankappa), right
speech (sammā-vācā), right bodily action (sammā-kammanta), right
livelihood (sammā-ājīva), right effort (sammā-vāyāma), right mindfulness
(sammā-sati) and right concentration (sammā-samādhi).}. Some people think that
the factors of the eightfold Path have to be developed one after the
other, but they develop together. The object which the citta and the
accompanying path-factors experience has to be a reality that presents
itself at the present moment, such as seeing, hardness, attachment,
kindness. It is an ultimate reality; conventional notions, such as
person or tree, are not the object of the path-factors. There must be
the factor of right understanding of the eightfold Path accompanying the
other cetasikas. Realities such as right effort and right mindfulness
without right understanding are not factors of the eightfold Path. Right
understanding knows the object that appears as non-self. When someone
thinks of self trying to develop the eightfold Path he is on the wrong
Path.

We usually think of situations, stories which are not realities, we are
dreaming, living in a phantasy world. Acharn always helps people to
return to the present moment. She said:

``Is there seeing right now? Is it truth? Who can make it arise or
change it into thinking? It is not me. Who can stop seeing right now?
One cannot do anything, it is completely gone. The seeing is not you, it
is a reality. There is no one at all, only different realities.''

Some people think that they should first have less akusala cittas before
they can develop understanding. They try to abandon attachment to people
and situations. If one tries to have less attachment one should ask
oneself what type of citta arises at such moments. One wishes to be a
good person with good qualities, and at such moments there is clinging
to an idea of self. It is very meaningful that there are four stages of
enlightenment when defilements are successively eradicated. First wrong
view must be eradicated and this occurs at the first stage of
enlightenment, the stage of the ``streamwinner'' (sotāpanna). Later on,
at the third stage of enlightenment, the stage of the ``non-returner''
(anāgāmī), attachment to sense objects is eradicated. Thus, so long as
realities are taken for self it is impossible to eradicate attachment to
them. The development of understanding takes aeons.

We read in the ``Kindred Sayings'' (III), ``Kindred Sayings on
Elements'', ``Directly Knowing'' (I § 24)\footnote{I am using the
translation by Ven. Bodhi.}, that the Budha said,
while he was at Sāvatthī:

\begin{quote}
    
``Bhikkhus, without directly knowing and fully understanding form\footnote{Rūpa, physical
phenomena.}, without becoming
dispassionate towards it and abandoning it, one is incapable of
destroying suffering\ldots{}''
\end{quote}

The same is said about the four nāma-khandhas: feeling, perception
(saññā), formations (saṅkhāra-khandha, the other fifty cetasikas) and
consciousness (viññāṇa-khandha). The Buddha said:
\begin{quote}

``Bhikkhus, by directly knowing and fully understanding form, by
becoming dispassionate towards it and abandoning it, one is capable of
destroying suffering\ldots''
\end{quote}

The same is said about the other khandhas. First there should be
intellectual understanding of the reality appearing now, such as seeing,
hearing, hardness, or whatever appears. They do not belong to a self but
they are mere passing dhammas. There are many moments of akusala we are
ignorant of. One may not know when feeling at ease, laughing, or looking
at the table, that there are bound to be many moments of attachments and
ignorance. Ignorance usually follows seeing or hearing.

Sometimes people are told to be aware now, but this is impossible, as
Acharn explained. Right awareness of the eighfold Path and also the
other path-factors arise because of the appropriate conditions, they are
not self. It is not a matter of trying to have awareness and
understanding. Trying to think of Dhamma or to walk slowly in order to
have awareness are methods that do not lead to right understanding of
realities, and, thus, they are the wrong Path.

Acharn said: ``When intellectual understanding is not enough to
condition right awareness, it cannot arise. All dhammas are anattā, so,
awareness and right understanding are anattā. Can anyone know the next
moment: seeing, hearing or thinking? Right awareness arises only when
there are conditions.''

It is not known what the next moment will be, it may be seeing, hearing,
clinging, awareness and understanding or dying-consciousness. It depends
entirely on conditions that are beyond our power. Nobody can condition
the arising of seeing, and evenso, nobody can condition the arising of
awareness. It can arise unexpectedly, if one does not cling to it.

People are often wondering when there are sati and paññā. That is why
Acharn explained about these realities time and again.

Acharn asked: ``When listening to Sarah and Jonothan, is there any
understanding? That is sati and paññā, not you. At the moment of hearing
there is no understanding. Understanding comes afterwards, it cannot be
hearing.''

Hearing is a vipākacitta that just hears sound, it does not know the
meaning of what is heard. Acharn said: ``Whenever there is understanding
it is only a reality; there are sati, paññā and other wholesome
cetasikas arising together, but they are unknown. We can understand that
the moments of seeing and hearing are not the moment of understanding.
This is the beginning to understand at which moment there are sati and
paññā and at which moment there are no sati and paññā.''

Acharn explained that in the beginning there is only intellectual understanding of realities. There is not yet direct understanding of the
arising and falling away of realities, which is a higher level of paññā.
There can be thinking about sati and paññā but their characteristics do
not appear as they are, just one at a time.

Someone may take thinking of realities for right awareness. Acharn
explained: ``Usually in a day hardness arises but it is not known. It
appears to ignorance and attachment, not to right awareness and right
understanding.''

Only right understanding can know the difference between such moments.
Someone touches a table and he believes that he clearly knows hardness
at such a moment. He should find out whether there is an idea of self
touching, even if he does not expressively thinks in that way. He may
think of hardness but without right understanding that knows it as just
a conditioned reality that does not last for an instant.

Acharn explained that one misleads oneself while believing that he has
awareness at the moment he is told to be aware. Each word of the Buddha
points to the development of right understanding. Only the right
intellectual understanding of whatever appears now can lead to direct
understanding, to paṭipatti, sati and paññā of the level of
satipaṭṭhāna.

The development of intellectual understanding, pariyatti, may take more
than one life until it is so firm that it can condition direct
understanding. Only direct understanding knows the characteristic of
sati and of paññā. We should not try to find out whether pariyatti is
already firm, as we were reminded so often. We should just persevere to
learn a little more about seeing now, hearing now, hardness now. They
arise because of their own conditions and nobody can manipulate them. It
is as Acharn often said: understanding is not enough yet.

Some people have an idea of practice of vipassanā, insight, by focussing
on an object with concentration, in order to make satipaṭṭhāna,
mindfulness of realities, arise. They try to focus on breathing or try
to have mettā, kindness. Instead of mettā one has attachment. Sarah
explained that this is not the same as mettā arising naturally in daily
life. All such methods are wrong concentration of the wrong Path.

Concentration, ekaggatā cetasika or samādhi, may be right concentration
or wrong concentration. When there is awareness and right understanding
of the reality appearing at this moment without any expectation of a
result, there is at that moment also right concentration. There is no
need to think of concentration, it arises already because of conditions.

Acharn explained that we should remember that all are dhammas, arising
by conditions. There should be no selection of the reality that appears,
its arising depends on conditions.

The words ``no selection'' that she often used are very helpful. It
reminds us that all are just passing dhammas arising each by their own
conditions. ``We cannot do anything, there is nobody there'', as Acharn
said, time and again. Following the right Path with detachment is the
way to have more confidence that all dhammas are anattā. The Path
leading to enlightenment is the Path of understanding of anattā, Acharn
said. At the end of the session on that day someone of the listeners
expressed her great appreciation of these explanations on the right
Path.

\chapter[One cannot do anything]{}
\section*{One cannot do anything}

The Dependent Origination, Paṭiccasamuppāda was briefly referred to
during the sessions. It is the Buddha's teaching about the conditions
leading to the cycle of birth and death. The first link is Ignorance,
leading to kamma, and kamma leads to rebirth and vipākaccittas (result)
throughout life. The first javana-cittas of every living being are
cittas rooted in attachment, lobha, thus, there is immediately craving,
there is clinging to life.

Someone remarked that one should just cut off craving, tanhā, which is
one of the links of the Dependent Origination, and that this would lead
to the end of the cycle. This is impossible since there is no self who
could change conditioned realities. It is with the eradication of
ignorance of realities through fully developed paññā that the chain of
conditions is broken. It will take aeons to develop paññā stage by
stage. Only when right understanding to the degree of arahatship has
been developed there will not be rebirth. Then there will be the end of
the cycle.

Often during the discussions there was reference to the ``three Rounds''
(vaṭṭas) of the cycle of birth and death. When seeing or hearing arises
it is the round of vipāka, the result of kamma. We cling to vipāka and
then there is the round of defilement, kilesa. Defilements lead to the
committing of kamma. So long as defilements have not been eradicated
there will be conditions for kusala kamma and akusala kamma that
condition rebirth. Kamma will produce again vipāka and in this way the
round goes on and on. Attachment and aversion are not always of the
degree of kamma, an evil deed, but they are defilements (kilesa) that
accumulate and when they have become strong they can lead to evil speech
and deeds that harm others, to akusala kamma that is able to produce
result.

The Visuddhimagga\footnote{This is an Encyclopedia
composed by the commentator Buddhaghosa who lived in the fifth century
A.D.} deals
with the three rounds in the chapter on the Dependent Origination (XVII,
299). We read:

\begin{quote}

``With triple round it spins forever: here formations and becoming
\footnote{Formations,
abhisaṇkhāra, refers to kusala kamma and akusala kamma. Becoming, bhava,
to kamma process becoming, kamma in this life.} are the round of kamma.
Ignorance, craving and clinging are the round of defilements.
Consciousness, mentality-materiality sixfold base
\footnote{Consciousness,
viññāṇa, refers to the vipākacitta that is rebirth-consciousness and to
vipākacitta throughout life. Mentality-materiality, nāma-rūpa, to the
cetasikas that are vipāka and rūpa produced by kamma.}, contact and feeling
are the round of result. So this Wheel of becoming, having a triple
round with these three rounds, should be understood to spin, revolving
again and again, for ever; for the conditions are not cut off as long as
the round of defilement is not cut off.''
    
\end{quote}

Sarah explained:

``After seeing attachment arises, it arises beyond anyone's control and
falls away instantly.''

Someone may think that since it does not stay why we should worry about
it, dwell on it, or reason about the causes of attachment.

Sarah: ``If one thinks that one can stop attachment and ignorance
arising after seeing, it is the wrong idea of self that can control or
stop or change reality.

Just as seeing arises naturally by conditions, attachment arises. When
hearing more about realities, sati can arise naturally, but not by
anyone's doing anything.

Listening and considering carefully does not mean how many hours one can
come to these discussions. It means that even a few words, like `seeing
arising by conditions now', or `what is the meaning of dhamma', has to
be very carefully considered. The careful considering leads to awareness
and understanding without any expectation.''

Acharn explained that life is anattā even from the beginning, from the
moment of birth-consciousness. Kamma conditioned the citta and
accompanying cetasikas at that moment; they are vipāka, result of kamma.
Kamma also produced three groups of rūpa, kalapas, at that moment: one
group with bodysense, one group with the heartbase and one group with
sex, femininity or masculinity. Even the moment of birth is anattā.
Hearing the teachings now is kusala kamma and this can condition the
next birth-consciousness.

We still have the opportunity to hear the teachings and develop understanding. She said: ``Understanding means understanding of seeing and
hearing in daily life.'' Pariyatti is not theoretical understanding, it
is understanding of what appears now, time and again, like seeing and
hearing.

If there is only thinking of the fact that all dhammas are anattā, there
will not be any understanding right now of dhamma as anattā. Seeing
lasts only for one moment but people think: `I see'. Life exists only in
one moment. Acharn repeated this very often and this is most beneficial.
We always forget this when we are taken in by the events of life.
Ignorance is a kind of sickness and the Buddha is the true physician who
can cure it.

Acharn explained that just knowing that there is ignorance is not
enough. One should understand more and more that it is conditioned, by
the accumulation of aeons ago. No one can stop conditions for the
arising of anything.

Hearing a few words is not enough. When we were in a restaurant Acharn
said: ``We have to continue developing understanding amongst akusala.''
Delicious food was waiting for us and it is natural to have attachment
to flavour. Even after seeing, shortly after it has fallen away, we are
bound to be attached to seeing or to visible object. This is completely
unknown. Cittas arise and fall away succeeding one another so fast. When
we consider the countless moments of akusala, Acharn's words that
understanding is developed in the midst of akusala become more
meaningful. Akusala cittas arise almost all the time, but, in between,
understanding of whatever appears can be developed, just for an
extremely short moment.

``One cannot do anything'', this was often repeated. It was said to
remind us that there are only conditioned realities arising and falling
away. When seeing has arisen can one prevent it from arising? Can one
cause the seeing that has fallen away to return and see again?

Seeing was taken as an example of a conditioned reality, but when we
have more understanding of its conditioned nature we shall also
understand that other realities that arise are conditioned and cannot be
altered or manipulated. One cannot do anything. This can also be said of
different kinds of defilements that arise, that are not welcome.
However, thanks to the Buddha's teachings understanding of whatever
appears can be developed, in the midst of akusala.

Acharn said: ``What we take for the world, people and things are only
different realities. It is the absolute truth, nobody can change this
characteristic. When you touch something hard is it you who makes it
hard? Actually, there is no you, only different dhammas.''

Someone asked whether good deeds, like giving, releasing animals\footnote{This is a custom done
near a temple or pagoda: releasing birds that have been caught, in order
to make merit.} or chanting texts would
be a way of sharing merit with the dead. When someone is reborn in
another plane of existence, such as one of the heavenly planes, where he
can know about someone else's good deed he may appreciate the wholesome
deeds of someone else and have kusala citta on account of it. This can
be called the sharing of merit. However, we cannot know in which plane
someone else is reborn, a plane where he could know about someone else's
kusala. It is of no use to speculate about this.

It is difficult to know whether the citta of someone else is kusala or
akusala. When one sees other people doing good deeds or abstaining from
deeds that can harm others it cannot be known from the outward
appearance of a deed whether or not the citta is kusala. Every citta
that arises does so for an extremely short moment and it falls away
instantly. Kusala citta can be followed closely by akusala citta, but
since cittas succeed one another so rapidly one may take akusala citta
for kusala citta.

Someone thought that he could constantly be aware of postures. There
cannot be awareness and understanding all the time. When one thinks in
that way there is no understanding. Posture is an impression of a whole,
a situation, such as walking. It is not a nāma or rūpa that can be
directly experienced. We can just think of it.

Some people want to change their lifestyle or give up their job in order
to have more time for the Dhamma. However, such a wish is motivated by
clinging to the idea of self . This hinders the development of
understanding. One should lead one's life naturally.

What is morality, sīla? This was one of the topics of discussion that
came up. It is not following rules. There may be the conventional way of
thinking about good morality when considering kusala sīla. However, we
should know what kusala sīla is in the ultimate sense. It is the
behaviour of citta, only for a very short moment. Kusala citta does not
last, it falls away within splitseconds. From the outward appearance of
deeds by others we can never know what the citta is like.

When performing a good deed such as helping others, one may have an idea
that such moments last. Or when one abstains from akusala one may think
of a story of not retorting unkind words to someone who spoke in an
unpleasant way. All the time during which one does not retort unkind
words one may keep silent with aversion. Then one misses the real
meaning of abstaining from akusala which is just a short moment of
kusala citta. Or there may be moments of conceit, while thinking, ``I am
better than the person who spoke unkindly, or see how good I am''. When
we have more understanding of citta it will be clearer that kusala sīla
is the behaviour of kusala citta which arises and then falls away
immediately.

If one knows the difference between understanding the ultimate realities
of citta and cetasika and the conventional way of thinking, there will
be more understanding of what sīla is, the behaviour of citta now. We
are used to thinking in a conventional way but if we do not know the
difference between the conventional way of thinking and the
understanding of ultimate realities, there will be a great deal of
confusion in life.

In the ultimate sense, there is no person, only citta (consciousness),
cetasika (mental factors accompanying citta) and rūpa (physical
phenomena) which arise and fall away instantly. What arises and falls
away immediately cannot be taken for a person or self.

We have to lead our daily life naturally, and so, we think of our
fellowmen, to treat them with respect and concern for their wellfare. At
the same time there can be more understanding of ultimate realities.
What is the citta like that thinks of others, is it kusala or akusala?
It is not my kusala or akusala, just dhammas arising because of their
own conditions. When there is right understanding, kusala will be purer,
there will be less thinking of self.

One may think in a conventional way of monkhood, thinking of the person
who wears a yellow robe. However, we have to consider the true meaning
of monkhood by way of ultimate realities. It is not the wearing a yellow
robe that makes someone a monk. It is the behaviour of citta of someone
who leads the monk's life, different from the laylife. The monk should
lead the life of an arahat, a perfected one, who truly sees the danger
of being in the cycle of birth and death.

Acharn, her sister and I had a wheelchair at the airport in Bangkok. We
had a conversation with the young men who pushed those wheelchairs and
it appeared that several of them had been ordained as a monk for three
months, just to please their parents. They said that they did not have
much understanding of the Vinaya. One should have understanding of the
teachings first, before being ordained.

In Vietnam Acharn spoke about the true meaning of monkhood. She stressed
the importance of first studying the teachings. One should not think of
being ordained in order to study the teachings later on. Even the first
five disciples of the Buddha were not ordained yet before they had
understanding of the teachings. One should have true understanding of
the teachings in order to know whether one has accumulations for leading
the life as a monk or for leading the laylife.

In the Buddha's time laypeople who developed understanding and attained
enlightenment, could even attain the third stage, the stage of the
``non-returner''(anāgāmī), and this meant they had eradicated clinging
to all sense objects. But those who attained the fourth stage, the stage
of the arahat, had eradicated all kinds of clinging and all defilements
and they could not lead the laylife anymore. The monk's life is the life
of the arahat. It is a life of fewness of wishes. They should never ask
for anything, they are dependent on the requisites of robes, food,
lodging and medicines given to them by laypeople. It is their duty to
``review'' these requisites every time they use them.

Acharn explained that it is disrespect to the Buddha to just wish to
become a monk and be his son or heir without developing understanding of
his teachings. Only those who have accumulations to lead the monk's
life, study the teachings and develop understanding could ask to be
ordained as a monk.

Acharn said: ``The monk's life is quite different from the laylife. When
the monk wakes up he remembers, `I am not a layman anymore.' Before
having his meal he must think about it that this is given. So he would
make use of this gift in the best way. The best thing in life,
especially in the monk's life, is studying the teachings carefully and
follow the Vinaya, the rules of Discipline. Each word of the Vinaya
comes from the Buddha himself. Who else is better than him to lay down
the Vinaya, the rules of the monk's life.'' Acharn said that if the monk
does not live in this way it is so dangerous to be a monk. He is like a
thief who steals what is given to the virtues ones.

We read in the ``Visuddhimagga'' (I, 125) about four kinds of use and
one kind is ``use as a theft''. The commentary to the ``Visuddhimagga'',
the ``Paramattha-mañjūsā'' (61) states:
\begin{quote}
    

`` `Use as theft': use by one who is unworthy. And the requisites are
allowed by the Blessed One to one in his own dispensation who is
virtuous, not unvirtuous; and the generosity of the givers is towards
one who is virtuous, not towards one who is not, since they expect great
fruit from their actions''.
\end{quote}
Acharn said: ``That is why in the Buddha's time someone would understand
the teachings before becoming a monk and he would realize that his
accumulations are so great that he could renounce the laylife.''

Acharn explained that the Buddha's Path, no matter for laypeople or for
monks, is the way to have less and less attachment. Clinging to the idea
of self should be eradicated first, before clinging to sense objects can
be eradicated. Each word of the Buddha pertains to understanding of
reality right now.

At the end of the last session one of the monks spoke with great
appreciation and confidence about Acharn's explanations on the
development of understanding. He said that this was the right Path. He
was hoping that at rebirth in a future life he would be able to listen
to Acharn again.

There were long photo sessions and many of the listeners found that
these days they had learnt a great deal from the discussions. They
expressed their gratefulness and appreciation to Acharn, Sarah, Jonothan
and friends. It was very inspiring to see the confidence and enthusiasm
of those who had listened during these sessions.

The Buddha's words about the truth of life should be understood, not
just followed. The best thing he shared with everyone is understanding
and this is what is most precious in one's life.

