\part{The Meaning of Dhamma}

\chapter[The Meaning of Dhamma]{}
\section*{The Meaning of Dhamma}

The Dhamma Study group in Vietnam had organised Dhamma
sessions in June, 2019, in Nha Trang and Hoi An.
Acharn Sujin could not come this time
since she was recovering in hospital after a journey in
India but Sarah and Jonothan were going to lead the
discussions. Much of what I am reporting from these has been taken from
their inspiring remarks. 

Ann from Canada and Roti from Mexico also joined the sessions. The
Vietamese Dhamma Study Group is very active and Dhamma books are printed
all the time. They had just finished printing Acharn's book on Metta,
Loving Kindness, which I had translated from Thai into English. They had
designed a beautiful cover with a hand of the Buddha. 

Before leaving Bangkok for Vietnam Sarah had asked Acharn
whether she had any message for the listeners at the sessions. She said:
``One word: dhamma.'' Sarah elaborated: ``What is the meaning of dhamma?
Dhamma is the meaning of life at this moment. Usually there is no
understanding of life. We follow our daily routine without any
understanding of life.'' 

Sarah explained that in whatever situation we are, there are
just passing dhammas, each arising because of their own conditions and
impermanent. We can learn to see life in a different way, for example,
when we lose dear persons through death. Wherever we traveled, there was
time and again a case of someone who had lost a dear person. This time I
met someone who was depressed because of the loss of her husband ten
years ago. She often went to
the movies to forget about her loss. 

We all cling to persons, but we can learn from the Buddha
that what we take for a person are only citta, consciousness, cetasikas,
mental factors accompanying consciousness, and rūpa, physical
phenomena. They arise for an instant and then fall away immediately.
What we used to find so important, the different feelings, our
experiences through the senses and our thinking, are just passing
dhammas. 

The Buddha taught what life is. Before hearing the Buddha's
teachings we had ideas about life different from his teaching. In
reality, life is only one moment of consciousness, citta, and it changes
all the time; it falls away immediately. At the moment of seeing, life
is seeing, at the moment of hearing, life is hearing. The Buddha taught
what can be directly experienced one at a time, through the sense-doors
and the mind-door. He taught that each reality arises because of its own
conditions and that there is not any person who can make it arise. Each
reality arises just for a moment and then it is gone immediately, never
to return.

At each moment in life there is the loss of an object we hold
dear, there is a kind of death. We may experience a pleasant sound or a
delicious flavour, but it is gone immediately. We are attached to all
objects appearing through the senses and the mind, but actually we cing
to nothing and are misleading ourselves with regard to what is real.
When we lose a dear person through death we feel sorrow because of our
clinging to pleasant sights, sounds and other objects experienced
through the senses. We no longer see or hear the beloved person, but we
are really thinking of ourselves, we mourn ourselves. 

We always thought that life could last for some time, and
that there is a self coordinating all our experiences in life. We
thought of a self who sees and thinks about what is seen, who hears and
thinks about what is heard, all at the same time. The Buddha explained
that there is no self, only momentary realities that change all the
time. None of these moments can stay and they cannot be controlled, they
cannot be caused to arise. We usually live in the fantasy world of
``I'', thinking of ``I see, I hear, I touch''. They are only different
dhammas. 

One of the listeners realized that he often confused thinking
of stories and concepts with realities. He realized this while he was
thinking of the problems that arose within his family. That shows how
useful discussion about this subject is. In this way understanding of
the level of pariyatti can grow. 

Because of the experiences through the senses and the
mind-door different feelings arise: pleasant feeling, unpleasant feeling
and indifferent feeling. We find feeling so important and take it for
``my feeling''. We wish to experience only pleasant things and when
there is an unpleasant object such as a loud sound, unpleasant feeling
arises. Whatever is experienced, also feeling, are passing dhammas that
cannot be controlled. We wish to control what will happen in our life,
but we never know what the next moment will bring and during my journey
I was reminded of this fact several times. 

When we have problems in life we find them very important,
and we worry. In fact, these are only moments of thinking with different
cetasikas such as aversion which falls away immediately. We believe that
we can control the events of life but in the end problems are solved in
a way that is totally different from what we expected. I was worried
about my traveling I find so difficult. 

Sarah gave me very good reminders about worry. She had just
had an accident, an electrical shock because of touching an electric
device in the hotel room. She was flung from one side of the room to the
other side. Shortly afterwards she spoke very helpful words to me to
remind me of the truth while I was worrying, but she could not remember
what she had said since she still had to recover from the shock. Kusala
cittas conditioned her speaking. This was beyond control but it happened
because of conditions. She explained that when one is worried one is
occupied with oneself and one forgets that whatever reality appears is
beyond control. We can make useful plans, believing that we can
determine the outcome and then we are thinking of situations instead of
understanding the reality which appears now, such as seeing, hearing,
attachment or aversion. When we are just thinking of situations we
usually do this with an idea of self and this will not lead to the
understanding of realities. 

I was worried how I could manage to go around without a
walker when arriving in Thailand and Vietnam. But there were walkers
when I arrived, people had given them to me with great generosity. I
could not have known this ahead of time. Problems are solved beyond
expectation. 

Hang and I wanted to eat some noodles and we went to the
restaurant of the hotel. The owner of the hotel had offered food to
monks who were visiting from Myanmar and since there was much food left
she invited us for 
luncheon. At the table we were sitting together with her
family members and friends. We had not expected to enjoy a meal together
with such a delightful company when we
went out for some noodles. It happened all because of
conditions. During the meal I met Jotamoi from Myanmar who had studied
the Abhidhamma for many years and given lectures on this subject. Hang
and I spoke to him about the Dhamma Study Group and invited him to join
our sessions. He had no more time since he had to accompany the group of
monks from Myanmar, but the next day he sent us a few useful questions.


One of his questions was about the difference in teaching as
we find it in the Suttanta, the
Abhidhamma and the Vinaya. It seems that we read in the
suttas about things that should be done whereas in the Abhidhamma there
is reference to realities. It seems that the Vinaya mainly deals with
the establishment of rules, with sīla.

The Buddha taught about the reality appearing at the present
moment and the way to develop understanding in order to eliminate
ignorance. He taught that whatever reality appears is non-self. The
teachings contained in the three parts of the Tipiṭaka are actually the
same. In the Suttanta he would speak about persons and situations, but
also at those moments he was referring to realities arising because of
conditions. He spoke time and again about all realities of daily life as
they are experienced through the six doors and he asked the listeners
whether these realities are permanent or impermanent and whether they
can be taken for self. He would speak about making an effort for the
arising of kusala, and the elimination of akusala, but then he was
referring to the sobhana cetasika (wholesome mental factor) right effort
which is a factor of the eightfold Path and which is always arising
together with right understanding of realities. Right effort does not
belong to a person. It arises because of wholesome accumulations. Each
wholesome or unwhwolesome quality that arises with the citta falls away
immediately with the citta, but it is not lost, it is accumulated in the
citta. Since each citta is succeeded by a following citta wholesome and
unwholesome qualities are carried on from one citta to the following
citta, from moment to moment, from life to life. 

Not only in the Abhidhamma but also in the suttas we read
about the ultimate realities of the five aggregates 
khandhas), the elements (dhātus),
the sense-fields (āyatanas), about all realities which are classified in
different ways as nāma and rūpa. 

As to the Vinaya, this deals among others with akusala sīla.
Sīla comprises kusala dhamma, akusala dhamma, and for the arahat
avyākata (indeterminate) dhamma which is neither kusala not
akusala\footnote{He does not commit
kamma that can bring a result in the future. For him there are no
conditions for rebirth.}
People are inclined to think that sīla are just rules to be followed,
but sīla is the behaviour of the citta at this moment. Kusala sīla
develops through right understanding of the citta at this moment. Any
reality which appears now can be investigated and considered in order to
understand its true nature. The development of understanding is the
highest kind of sīla.

We read in the teachings about sīla (morality), samādhi,
(calm) and paññā (understanding) and people are inclined to think that
there is a certain order of developing these qualities. There is no rule
that they should be developed in a certain order. The factors of the
eightfold Path can be classified in this threefold way, but all factors
develop together with right understanding of the eightfold Path\footnote{The three abstinences,
right speech, right action and right livelihood, are the sīla of the
eightfold Path, right mindfulness and right concentration the calm,
samādhi, of the eightfold Path, right thinking and right understanding
the wisdom of the eightfold Path. }


Kusala sīla comprises all levels of sīla and the highest sīla
is the development of right understanding of realities. As to samādhi,
calm, this arises with every sobhana (beautiful) citta. When a moment of
right understanding of a reality arises, there are sīla and samādhi as
well.

There are several ways by which sīla, samādhi and paññā have
been classified. They can be considered under various aspects. One way
is the classificication of the ariyans who have reached the perfection
of sīla, of the ariyans who have rached the perfection of
samādhi and of the ariyans who have reached the perfection of pa{ññā.
The sotāpanna who has eradicated the wrong view of self has
no more conditions to violate the five precepts, he has perfected sīla.
He has no more conditions to commit the kinds of akusala kamma that lead
to an unhappy rebirth. The non-returner (anāgāmī) who has realized the
third stage of enlightenment, has perfected calm, samādhi. He has
eradicated all attachment to sense objects, he is no longer absorbed by
them. The arahat who has eradicated all ignorance and other kinds of
defilements has perfected paññā. 

It is because of the development of right understanding of
the reality appearing at the present moment that the different stages of
enlightenment can be attained and defilements are eradicated stage by
stage. Not because one should develop sīla first, then calm and then
paññā. 

Right understanding of what is reality should be developed
and this is different from knowing conventional notions such as a table,
a tree or a person. Acharn Sujin reminded us of what dhamma is by her
message to Sarah about dhamma. Dhamma is what can be directly
experienced through one of the senses or the mind-door. Sound is a
dhamma that can be directly experienced through the earsense, without
having to think about it. We believe that we hear the sound of a barking
dog, but then there is thinking of a concept, a story. There is no
barking dog in the sound.

Before coming into contact with the Buddha's teachings we ony
knew conventional truth. But the Buddha taught what can be directly
experienced one at a time, through the sense-doors and the mind-door. He
taught that it only lasts for one moment and then it is gone, never to
return. He taught that each reality arises because of its own conditions
and that there is not any person who can make it arise. 

We can lead our daily life as usual, thinking of our friends,
of persons, of food, of our house, but we can learn the difference
between thinking of concepts of persons and things and understanding of
what can be directly experienced at the present moment. 

It is beneficial to understand the difference because in that
way we can learn how much ignorance there is of the truth of life. We do
not try to change anything that arises because of conditions, but
understanding can grow. Understanding the truth of life is more precious
than anything else. 




\chapter[Understanding Ultimate Realities]{}
\section*{Understanding Ultimate Realities}


Understanding of our life now can be developed. We usually
follow our daily activities with attachment and ignorance, but any
reality that appears can 
be the object of understanding. One of our friends liked
playing the guitar and spent much time on this. Time and again he was
wondering whether he should spend more time on studying Dhamma instead
of playing guitar. Whatever arises in life happens by conditions. One
person likes music, another person painting or sports. It is not useful
to think of how little awareness and understanding there is and what one
will do to have more. Then one is concerned about oneself and this will
not be helpful to have more understanding of the truth of non-self.
Satipaṭṭhāna is not ``doing something'' but the arising of mindfulnes
and understanding by conditions whereby
direct understanding of
nāma and rūpa
is developed. We were often reminded by Acharn with the words: ``Let
understanding work its way''.

No one can stop attachment from arising right now. One of our
friends remarked that conditions can be created for the non-arising of
attachment. However, if one tries to make things happen in a specific
way one fails to see the truth of anattā. 

When people spoke about their different defilements Jonothan
kept on saying: ``It does not matter, it does not matter.'' When asked
to explain what he meant by this, he said: 

``What has arisen, has arisen. It can be understood. It is of
no use to analyse it, finding out whether it is kusala or akusala. Then
we are choosing a specific object and it is not understanding the
present moment. There is no understanding of any reality that is
appearing now.''

We can lead our life naturally, swimming, playing guitar,
doing our job. It is of no use to try to change our life style.
Understanding can be developed in any situation. If one just thinks of
being in favorable situations there will never be understanding of what
dhamma is that just arises because of its own conditions. 

When one thinks that the development of the Path is too
difficult, there is again the idea of self, an idea of ``I who cannot do
it''. When one thinks about what one shall do, thus, about the future,
there is forgetfulness of the present moment. One does not understand
that there are only conditioned dhammas which are beyond control,
non-self. Thinking arises now, and then it is gone immediately. If there
is no understanding of the present moment yet, it does not matter. When
understanding has been accumulated more the present moment can be known
naturally, without any expectation. 

What can be directly experienced now, without having to think
about it, is just dhamma. Usually realities such as hardness, heat or
cold are directly experienced through the bodysense but they are not
known as dhammas. Right understanding has to grow so that they can be
understood as realities appearing through the bodysense that arise for a
moment and then fall away immediately. 

Each reality arises and falls away very rapidly. Seeing is
immediately followed by other cittas. Then seeing arises again and again
in other processes. We could not know a single moment of seeing, only a
sign, in Pali: nimitta, remains. Of each reality that arises a nimitta
is experienced. This can remind us that realities arise and fall away
extremely fast.

At the end of the Dhamma sessions, in Nha Trang, the monk who
followed the sessions asked Jonothan to resume in a few points the
contents of the sessions. 

Jonothan mentioned four points: Listen carefully, consider
carefully, remember this while going about one's daily life, and have
confidence. He elaborated on these points in the following way:
 As to listening carefully, this means hearing an explanation
of the Buddha's teaching and every opportunity to hear that can be taken
when understanding is being developed. One needs to hear everything more
than once, from different aspects and different parts of the teachings.


As to the second point: just hearing is not enough, there
need to be some reflection as one goes about one's daily life. The
meaning is deep and subtle. It cannot be comprehended just on a single
hearing without reading, turning it over in one's mind, considering how
it relates to the present moment. 

This has to be done while going about one's daily life and
that means that there is no need for a special kind of place, special
circumstances or environment for this to happen. One can hear Dhamma
from an unexpected source, consider it during unlikely activities.
Anything other is not the Path of the Buddha. 

There should be confidence that the first and second points
are sufficient if properly understood, and that they can condition
awareness to arise. It may seem that there must be more that can be
done, but there should be confidence that those factors mentioned by the
Buddha are sufficient for awareness to arise, in time, when the
conditions are there. Be patient also.

These are precious points. As to confidence, one should not
be disheartened that the development of understanding of the present
reality takes a long, long time. Jonothan also said about confidence:
``Confidence in the Dhamma, confidence in the development of kusala,
regardless of the situation.'' We may be disappointed when things do not
work out the way we expected, but we should not forget that whatever
reality presents itself is conditioned already. 

At the end of the sessions in Nha Trang one of the children
who listened to the Dhamma conversations made a touching speech, showing
her gratefulness for all she had learnt those days. She spoke with great
confidence. It was inspiring to notice people's enthusiasm. The mother
of someone who regularly attended the sessions was listening to
recordings during the time she worked in the rice field. This is heavy
work in a hot climate but it did not prevent her from listening and
considering the truth of Dhamma with confidence. Different families
sponsored our lunches each day and they walked with their children past
our tables so that we had an opportunity to meet them and to express our
appreciation.

After Nha Trang we had a short flight to Hoi An where some of
us stayed in Anicca Villa. Those were happy days when we enjoyed the
hospitality of Sun and Mai. Every day a delicious Vietnamese meal was
prepared and before, during and after the meal we had Dhamma
conversations. Nam and his younger brother Nguying joined our
discussions and Nam, who has a great musical talent, played the piano
for us. The two French architects who had designed this villa and
several other resorts came along and they had basic questions which were
useful for everybody. We discussed how one is inclined to wish to
control whatever occurs and if things do not work out according to one's
wish one may vex oneself and believe that this is one's own fault. Sarah
explained that right understanding that whatever occurs is because of
the appropriate conditions is like the removal of a heavy burden. There
will be less clinging to an idea of self one used to find very
important, and no feelings of guilt. 

Again we have to remember the true meaning of dhamma;
whatever arises is dhamma and this means that nobody can make good
qualities arise, that there is no person. It is natural to think of
persons, of our friends, but in order to understand the truth we have to
consider the present reality. There is no reality to be known that is a
person, a friend. Visible object is experienced through the eyesense and
in visible object there is no person, in sound that is heard there is no
person. We tend to forget that what we take for a person is only citta,
cetasika and rūpa that do not last, even for a splitsecond. When someone
speaks unkindly to us we have aversion or anger and we keep on thinking
of ``that terrible person''. We forget that the real problem is always
the attachment, aversion and ignorance that arise within ourselves. In
reality no person is heard, only sound is heard and there is no one who
acts or speaks. It is important to know what is real in the ultimate
sense and what is only imagination. I have heard this often but we can
hear it again and again in order to be reminded of the truth. It is not
wrong view to think by way of conventional notions such as this or that
person or situation. It is necessary for leading our daily life
naturally and to communicate with others. At the same time we can
develop understanding of what is real in the ultimate sense. There is
wrong view when we believe that a person or self really exists.

If we have a problem we tend to think about it with
attachment or aversion and we take our thinking for self. Sarah
explained that the aim of the study of Dhamma is not to stop such
thinking, but that this is the time for understanding Dhamma. Whatever
occurs can be understood as just passing dhammas. Sarah said: ``When
confidence develops there will be less thinking of `how can I have more
understanding.' There can be understanding just now of what appears.'' 

If we wonder about the way to have more understanding, it
takes us away from the present moment. We should not try to focus on
particular realities, because realities appear naturally, just when
there are conditions for them to arise and appear. We should not forget
that life is in a moment, just in this moment. Very gradually we can
come to understand the difference between life in conventional sense,
life as different situations, and life just in a moment, such as seeing
appearing now or thinking appearing now. We think of many stories in a
day, but they are just fantasy, not reality. 

When we wonder whether we should or should not do this or
that, we are inclined to control the situation we are in or we think
that another situation, not the present one, is more favorable for the
development of understanding. 

Some people believe that the development of samatha is
necessary for the understanding of realities. One may wonder whether the
Buddha taught 
samatha. During the sessions people asked questions about
samatha. Also before the Buddha's time samatha was developed. The Buddha
taught about all kinds of kusala, and these can be developed with right
understanding. Samatha should not be developed without understanding of
what true calm is. Some people believe that when they close their eyes
and think of a wholesome subject such as metta, that they develop calm.
One should have right understanding of calm. Calm is a sobhana cetasika
that arises with each wholesome citta\footnote{There are actually calm
of citta and calm of cetasikas.}
At the moment of calm there are no attachment, aversion or ignorance. If
one has accumulated the inclination to develop higher degrees of calm,
even to the degree of jhāna, one should see the danger of clinging to
sense objects. If one does not see this, calm cannot be developed. Right
understanding of the characteristic of calm and of the way to develop it
with an appropriate subject is indispensable. 

There can be wise reflection at any time and at any place.
There is no need to wait for a favorable time or to go to a quiet place.
Some people believe that they have to be calm first before they can
develop understanding of realities. Or they think that when there are
less defilements there is more opportunity for the arising of awarenes
and understanding. Akusala has been accumulated and it will continue to
arise. Only direct understanding that has reached the level of lokuttara
can eradicate stage by stage
the latent tendencies of attachment and other accumulated defilements. 

We often hear that the development of understanding begins at
the present moment. We can listen more and consider the reality
appearing now, be it seeing, sound or attachment, even for a few
moments. Life will never be as we think or plan. It cannot be known what
hearing hears the next moment or what thinking thinks the next moment.

There is no self who is listening and considering the Dhamma.
These are only moments of citta arising because of conditions. Someone
may hear for the first time that there is no ``I'' who sees or hears. It
is only the seeing that sees, the hearing that hears. There can be a
beginning of understanding the truth. Considering the realities that
appear are wholesome moments that do not last, that fall away
immediately. Each moment of citta falls away immediately but it
conditions the following moment. In this way understanding can be
accumulated from moment to moment. At a following moment there may be a
little more understanding of realities. Then we listen and consider
again and gradually understanding can grow. This helps us to see that it
is not ``I'' who understands. 

Intellectual understanding, pariyatti, can condition direct
understanding, paṭipatti. Pariyatti does not mean theoretical knowledge
of the teachings, it is always related to the present moment, to what
appears now.

Seeing appears time and again and we can consider its nature.
We can begin to understand that it is a conditioned reality, conditioned
by kamma, by eyesense and visible object, and that it only experiences
visible object, no persons or things. Thinking about persons and things
is another citta, and it follows so closely after seeing, that it seems
to arise at the same time as seeing. But there can be only one citta at
a time experiencing one object. 

Before realities can be understood as non-self, it should be
known what appears now. Understanding of the level of pariyatti does not
experience realities directly, it experiences concepts of realities.
It should be emphasized that the objects
are concepts of reality, not concepts in the sense of
stories, situations or imagination, such as a person, a building or a
table. Seeing, hearing, sound or feeling may appear, and understanding
can investigate these realities so that they will be known as only
dhammas. When understanding of the level of pariyatti has become very
firm, it can condition direct understanding of them. There should be no
expectation when there will be direct understanding of realities, but
there can be confidence that the truth of Dhamma can be penetrated
little by little. Otherwise the Buddha would not have taught it.



\chapter[Sati of Vipassanā]{}
\section*{Sati of Vipassanā}

Sati is a wholesome cetasika that is non-forgetful of what is
wholesome. Sati can be of different levels, of dāna, of sīla, of samatha
and of vipassanā. When there is an opportunity to be generous in giving
or assisting someone else, we may be lazy and forgetful so that it is
impossible for us to be generous or to help. When sati arises it uses
the opportunity for kusala and it is non-forgetful of generosity and
mettā. At such moments it guards the six doorways: whatever object is
experienced through one of the six doorways is experienced without
attachment, aversion or ignorance.

Sati is not awareness or mindfulness as we use these words in
conventional sense. It is not: knowing what one is doing, like walking,
or focussing on an object. One should know what the object of sati is:
any object that appears at the present moment by conditions. It is not a
situation or a concept but a reality like sound, hardness, attachment or
thinking. There should not be any selection of specific objects; also
unpleasant objects and unwholesome objects can be known one at a time
when they appear. That is the only way to understand that whatever
appears is anattā; it is not in one's power to have any control. If one
believes that the situation is not favorable for sati or that one should
create conditions for sati one is on the wrong Path leading one further
away from the truth. 

Anger or attachment can be object of mindfulness. They may
arise because these realities also arose in the past. They arose and
fell away with the citta but they are accumulated from one moment of
citta to the succeeding moment of citta, from life to life. Kusala and
akusala lie dormant in each citta and when there are conditions they can
arise.

When we remember that kusala cittas and akusala cittas arise
in processes in a certain order that cannot be altered it will be
clearer that they are beyond control. Seeing arises within a process and
after it has fallen away three more moments of cittas arise before
kusala cittas or akusala cittas performing the function of javana arise.
They are all gone before one can think about them. Nobody can prevent
them from arising. Who knows what seeing will see the next moment or
what thinking will think of. Whatever arises because of conditions can
be object of mindfulness. 

It is important to know what is real and what is only
imagination. We believe that we can hear a dog
barking. Sound is real, but
a barking dog does not exist in the ultimate sense. What we take for a
dog is only citta, cetasika and rūpa which do not last for a moment.
Sati can only be mindful of the present reality, not of a conventional
notion. When sati is mindful of a reality like sound there can be
understanding of that reality as anattā. It was explained time and again
that when there is more understanding of the level of pariyatti, thus,
intellectual understanding of this moment, it will lead to realizing
that there are only passing dhammas which are beyond control. Even when
one has heard this many times, it is beneficial to be reminded of the
truth. There is usually absorption in stories, in a fantasy world,
instead of understanding the present reality. 

When one does not know the difference, misunderstandings of
what the Buddha taught may arise. An illustration of this fact is the
way some people understand the Buddha's teaching about kamma and the
result, vipāka, produced by kamma. Some people think of this truth by
way of a situation. When they have an unpleasant experience they say:
``It is my kamma.'' In reality kusala kamma or akusala kamma produces a
pleasant object or unpleasant object at the moment of rebirth or during
life through one of the senses, such as seeing or hearing. A moment of
vipāka is gone immediately but one may think of it as a long lasting
event. The difference should be remembered between thinking of a whole
situation, of concepts or ideas, and understanding the truth of the
reality appearing at the present moment. 

During the discussions Sarah and Jon pointed out very often
that when we cling to the idea of a person, of a self and believe that a
person or self really exists, we live in a fantasy world and do not
understand realities. It is natural to think of persons and situations,
but they are not real in the ultimate sense. They are not realities that
can be directly experienced one at a time as they appear at the present
moment. 

Jonothan said: ``The ignorance, misunderstanding, that needs
to be overcome is not the thinking of the concept of a being, a person,
but the ignorance or misunderstanding of the reality appearing at the
present moment.'' Thus, we need not avoid thinking of this or that
person, but at the same time understanding can be developed that in
reality what we take for a person are only citta, cetasika and rūpa that
arise and fall away, that do not stay on. In that way we shall know the
difference between the fantasy world of stories and concepts and the
real world. 

People had questions about attaining nibbāna. Sarah and Jon
pointed out again and again: the development of awareness and
understanding begins now in daily life, that is the only Path to
nibbāna. The reality appearing now can be understood and thinking of
attaining nibbāna is speculation that is not helpful. 

All we find so important in life such as a pleasant place to
stay, good friends, is just a moment of thinking. Life exists only in
one moment. At the moment of seeing, life is seeing, at the moment of
thinking, life is thinking. Gradually we can learn to attach less
importance to the stories we think of and which are only fantasy.
Thinking is real. It arises because of the appropriate conditions, but
the object we are thinking of is mostly a story, a concept, and thus not
real. While listening and considering more we learn that the way we used
to consider the events of life is quite different from what the Buddha
taught and what we can verify at this moment. Before we had no idea of
what the present moment is. Gradually it becomes clearer what the
meaning is of ``just passing dhammas''.

Acharn asked time and again: ``Why do you listen?'' Is it
because we want to be a better person with less ignorance and less
akusala, we want to have more calm? Then we are thinking of a self who
wants to improve his life. When we are in a difficult situation we
usually think about a self who wants to solve problems and we cling to
the idea of a change of situation. In reality there are only seeing,
hearing, visible object or hardness. On account of such experiences we
think of stories, pleasant or unpleasant. Seeing or hearing which arise
now are vipākacittas, results of kamma committed long ago. After the
vipākacittas have fallen away, kusala cittas or akusala cittas arise
which react with wholesomeness or unwholesomeness towards what is
experienced.

The Buddha said: ``Avoid evil, do good and purify the mind.''
These words are not a prescription to be followed without any
understanding of realities. When there is more understanding of anattā,
of reality arising because of conditions, we can take his words in the
right way. When people are told to have mettā all day there is an idea
of ``I'' who can do something. Mettā is adosa cetasika, (non-aversion)
and it can only arise when there are the right conditions.
When one has kindness towards people one
likes there may be moments of metta, but there are also
likely to be moments with attachment. Kusala cittas and akusala cittas
alternate and one may take for mettā what is attachment. Without paññā
that knows the characteristic of the reality which arises we are likely
to confuse mettā and attachment.

Before we went to Vietnam we visited Acharn a few times in
Teptharin Hospital. One topic we discussed was the development of the
perfections (paramis)\footnote{These are the excellent
qualities the Bodhisatta developed in order to attain Buddhahood:
liberality, morality, renunciation, wisdom, energy, forbearance,
truthfulness, resolution, loving kindness and equanimity.}
We should not try to know whether or not there is at this moment
development of the perfections. She said: ``Daily life is the proof how
much the perfections are developed.'' There must be the firm
understanding of non-self. Several times Acharn emphasized that the
development of understanding is much more difficult and deeper than
anyone can think about. Acharn remarked: ``Understanding does not mind
how much it develops. It is self who thinks about this.''

I said that it is very
important to remember that it is more difficult than you
would think. Acharn remarked that otherwise akusala cannot be
eradicated. Ignorance and attachment continue on, from life to life, for
aeons. We are just reading and talking about realities, but there must
be the understanding of non-self at any time. ``It is harder and
harder'' she said. She emphasized the difficulty of eradication of
ignorance to show how deeply ignorance and wrong view are accumulated.
Acharn said: ``Lobha is making its way all the time and this is so very
difficult.'' We do not realize it that we cling to an idea of ``self
that should have understanding''. We take akusala citta with clinging
for kusala citta and that takes us further away from the truth. We may
underestimate the power of the latent tendencies that can condition the
arising of wrong view at any time. 

The different levels of akusala were brought up by Sarah and
Jon during the discussions. There are three levels: the anusaya kilesa
(latent tendencies), the pariyuṭṭhāna kilesa (arising with the akusala
citta) and the vītikkama kilesa (transgression, misconduct). The latent
tendencies do not arise with the citta but they lie dormant in each
citta, they are subtle defilements that can condition the arising of
akusala citta. The medium defilements (pariyuṭṭhāna
kilesa) arise with akusala citta but they are not of the
degree of unwholesome courses of action. The vītikkama kilesa are
transgressions or misconduct.

The latent tendencies are called subtle defilements, but they
are very powerful and tenacious. They can only be eradicated by the
magga-citta of the different stages of enlightenment. Since they are
latent and do not arise, they cannot be known. 

We talked about the intoxicants, the āsavas, subtle
defilements that arise. After seeing which is vipākacitta there are just
a few more cittas and then during the moment of cittas performing the
function of javana, the āsavas often arise with the akusala citta: the
intoxicant of clinging to sense objects (kāmāsava), of clinging to
existence (bhavāsava), of wrong view (diṭṭhāsava) and of ignorance
(avijjāsava). The āsavas arise time and again, even right now,
immediately after seeing, hearing and the other sense-cognitions, but
they are unknown. Cittas succeed one another extremely rapidly and they
have already gone in a flash. 

Seeing arises now and then the āsavas of ignorance and wrong
view are likely to follow. They pass very quickly but they can condition
the arising of ignorance and wrong view again and again. Acharn said:
``Even when you do not say `I see', the `I' is there. Only paññā can
understand it. The way to eradicate is so far away. Only paññā can
understand better and deeper, it can see what appears now as it is.
There is no need to say that it appears as not self, that is thinking
again.'' 

Acharn reminded us that it is very difficult not to cling to
the idea of self: ``No matter when, it comes in instantly as long as one
forgets it's not self. Even right now, whatever arises passes away never
to return. This moment is the test, any time!''

Jonothan remarked that it is wrong to wish for more
understanding. Acharn answered: ``One is trapped by ignorance and
cinging, having expectations about sati.'' 

Acharn explained about letting go of the object that is
experienced. When paññā begins to understand a characteristic of a
reality, there may be the inclination to hold on to that object and then
there is no opportunity to investigate other realities that follow. One
does not let go of the object. At the moment of understanding there is
no thought of ``I know''. If there is no letting go there is clinging to
an idea of self who thinks about realities. I remarked that it is
difficult to kow the difference between the moments with wrong view and
without wrong view. Sarah asked me: ``Does it matter to know? Otherwise
there is clinging again. When it is time it is known naturally, and we
are not trying to work it out.''

Seeing now is not the same as seeing a moment ago. Seeing
arises and then it falls away never to return and this is life. We have
heard this often but we did not consider this enough. We still have an
idea that seeing can stay, that it is always there. There are actually
six worlds, the world appearing through the eyes, through the ears and
through the other senses and the mind-door. Acharn was speaking about
the world during a session at the Foundation. She spoke about the first
stage of insight when there is clear understanding of nāma as nāma and
of rūpa as rūpa, of only one reality at a time: ``Where is the world?
The world is lost. It is amazing, no one, no world; only one
characteristic as it is. Hardness is hardness, where are the arms and
the legs, the whole world? They are all gone.''

She said several times that it is so amazing that only one
reality appears, no self. We think of our whole body but this is only an
idea. When the body is touched hardness may appear and hardness is a
reality that can be directly experienced. At such a moment there is not
clinging to our whole body. We talk about nāma and rūpa but they do not
appear clearly as is the case when moments of insight-knowledge arise.
There has to be firm understanding of the level of pariyatti and it can
condition later on direct understanding. 

What is the first noble truth of dukkha, was a question that
was raised. The noble truth of dukkha is dukkha of all conditioned
realities, saṅkhāra dukkha. It is the unsatifactoriness of conditioned
realities that arise and fall away and can, therefore, not be a refuge.
When paññā is of the level of pariyatti, it is understood intellectually
that nāma and rūpa are dukkha, but the arising and falling away of
realities can only be penetrated when paññā is of the level of
paṭipatti, direct understanding. After the third stage of insight
knowledge it is really understood what dukkha is. In one of the sessions
at the foundation Acharn said about dukkha: ``What is the use of what
arises and falls away? To understand this is to understand dukkha. There
is nothing, and that is dukkha\ldots{ Whatever has arisen must have
conditions for its arising and whatever arises falls away rapidly. It is
no more, never to return in the cycle of birth and death.'' It cannot
return and be something that one likes. We cling actually to what is
nothing. 

The second noble truth is the cause of dukkha and that is
lobha, clinging. It hinders the understanding of the truth of dukkha,
that is why it is the second noble Truth. It is opposed to wisdom. When
we cling to a reality we think that it is there all the time, that it
does not arise and fall away. 

Because of lobha there is clinging to what cannot last and we
go on to think of people and things as being lasting and real. This
leads to more clinging and craving throughout many successive lives. So
long as there is clinging there are conditions for rebirth and
we shall continue to
be subject to the unsatisfactoriness
inherent in the impermanence of conditioned realities. The
third noble truth is the ceasing of dukkha. When there is no more
clinging there will not be rebirth. The fourth noble truth is the Path
leading to the ceasing of dukkha. This is the development of
understanding of all realities appearing at the present moment.

Acharn often repeated that one has to study each word of the
teachings, one word at a time. She said: ``Study with respect, great
respect to each word. Otherwise one takes it that's very easy, very
simple; it cannot be like that at all.'' This reminds us not to
underestimate the subtlety of dhamma. 

We should remember Acharn's message to the listeners of the
Dhamma session, her message of one word: dhamma. We have to consider the
meaning of dhamma as it appears now, in our daily life. It has no owner
and it cannot be controlled.

When someone asked Acharn to speak some encouraging words,
she answered: ``All are dhammas''. One may be upset by troubles in life,
by problems, but there are only dhammas, conditioned realities. 


