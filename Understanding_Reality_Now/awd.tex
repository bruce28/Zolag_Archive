----------------------- Page 1-----------------------

Alone with Dhamma 



                                          Pilgrimage in India, October 2005 


----------------------- Page 2-----------------------


----------------------- Page 3-----------------------

                                                                                          ● 1 



 Chapter 1 



Alone with Dhamma 



“We live alone in the world”, this was one of the striking points Acharn Sujin  

explained   to   us  during   our   pilgrimage   in   India   with   hundred   and   twenty  

Dhamma-friends from Thailand and elsewhere. 



   We read in the Kindred  Sayings  (IV,  144, Kindred  Sayings on  Sense,  §165,  

Abandoning Wrong View, translated by Ven. Bodhi) that the Buddha said: 



      “Bhikkhu, when one knows and sees the eye as impermanent, wrong view is  

      abandoned. When one knows and sees forms as impermanent... eye- 

      consciousness as impermanent... eye-contact as impermanent... whatever  

      feeling arises with mind-contact as condition... as impermanent, wrong view is  

      abandoned. It is when one knows and sees thus that wrong view is abandoned.” 



The Buddha spoke thus with regard to all dhammas appearing through the six  

doorways. 

   When a person dies we may think about the impermanence of life, but this  

is not the realization of the truth of impermanence, the truth that each reality  

that   arises   because   of   its   appropriate   conditions   falls   away.   The   Buddha  

teaches   us  what   life  really  is:  it  is  one  moment  of  experiencing  an  object  

through  one  of the  six  doorways, the  doorways  of the  senses  and the  mind- 

door. Visible object, sound, these are dhammas appearing at this moment, but  

we are ignorant of the truth. 



Acharn  Sujin  said that we live  alone in the world, that we believe that there  

are many people around us, but that this is thinking. It is hard to accept this  

truth. Citta thinks of relatives and friends who exist. However, in the ultimate  

sense, a person is citta, cetasika and rūpa. Citta is consciousness, cetasikas are  

the  mental  factors  arising with  the  citta,  and  rūpa  are  physical  phenomena.  

Seeing  is  a  citta,  hearing  is  another  citta  and  thinking  again  another  citta.  

Citta  and  the  accompanying  cetasikas  arise  and  then  fall  away  immediately  

and also the rūpas of which the body consists arise and fall away. 

   Understanding that  in the ultimate  sense  a person  is  impermanent mental  

phenomena   and   bodily   phenomena   does   not   mean   that   there   cannot   be  

kindness and compassion for others. On the contrary, the Buddha exhorted us  


----------------------- Page 4-----------------------

2 ● Alone with Dhamma 



to  develop  all kinds  of kusala  and  to  assist  our  fellowmen.  However,  at  the  

same time we can develop understanding of what life really is: the experience  

of one object through one of the six doors. When there is less clinging to ‘my  

personality’ we shall be more concerned for other people’s welfare. 

   Acharn Sujin explained that we are born alone: the rebirth-consciousness is  

a citta that arises and falls away and is succeeded by a following citta. There  

cannot be  more  than  one  citta  at  a  time. We  see  alone, we  think  alone, we  

sleep alone, we die alone. The citta that falls away never returns; after passing  

away from this plane there is no return of the same individual. 

   Whenever   citta   arises,   it   experiences   one   object   and   then   falls   away  

immediately.   When   visible   object   appears   we   take   it   immediately   as   this  

person   or  my  friend,  but   that   is  thinking  on  account  of  the   experience  of  

visible  object. The Buddha taught  about  all  dhammas  appearing through the  

six  doors,  and  during  our pilgrimage this was  a topic  of discussion time  and  

again.  



During our pilgrimage we visited the holy places where the Buddha was born,  

attained enlightenment, preached his first sermon, and passed finally away. It  

was  a long  and difficult journey,  but,  as  a monk in Kusināra  said, it was our  

confidence in the Buddha’s teachings that brought us to these places with the  

purpose to pay respect to him. 

   When we were in Lumbini Acharn Sujin reminded us that the teachings are  

declining and will eventually disappear. Then nobody will know anymore the  

meaning of the holy places. Now we still have the opportunity to pay respect  

to the Buddha  at these places,  and  it  is  as  if we pay respect  at his feet. The  

Buddha fulfilled all the perfections and after dwelling in the Tusita Heaven he  

became  a  human being  and was born  in  Lumbini.  That was his  last  life.  In  

Bodhgaya   he   became   a   Sammāsambuddha   in   order   to   help   the   world   to  

become free from dukkha. 



Khun  Sujin helped us to have more understanding of the dhamma appearing  

at   the   present   moment,   because   without   such   understanding   we   only  

speculate  about  the  truth.  Her  energy  to  speak  extensively  on  the  Dhamma  

was   truly   amazing  and   it  showed  her   concern  for  us.   She  exhorted  us  to  

develop understanding, no matter whether the circumstances were disturbing,  

no matter whether we were sick or tired. She would repeatedly say: “develop  

understanding now.”  

   During  our journey   we  also  visited   different  Thai  monasteries  where   we  

offered robes  and other requisites. The Abbot of the Thai temple  in Nalanda  

expressed   his   appreciation   of   Acharn   Sujin’s   Dhamma   talks   to   which   he  

listened regularly. In the Thai temple of Kusināra the Abbot gave Acharn Sujin  

a blessing and said that he had listened to her for forty years, since the time he  


----------------------- Page 5-----------------------

                                                                   Alone with Dhamma ● 3 



was a monk in the temple of Mahā-dhātu in Bangkok. During the rainy season  

he  listened for  one  and  a half hour  in the morning,  and  also  after the  rainy  

season he listened regularly. He said: “How could we understand the Dhamma  

without Acharn Sujin, where would we be without her.” 

   We   rejoiced   in   his   appreciation   of   Acharn   Sujin’s   efforts   to   explain   the  

Dhamma. This monastery also has a Dispensary where many people from the  

province are coming. It is supported by devoted volunteers. 

   While we were in  Sarnath, the place of the Buddha’s first sermon, we were  

given the opportunity to pay respect to the Buddha’s relics which are kept in a  

ten  meter  deep  cellar. A  monk  had  to  descend  into  it  to  fetch  them.  Since  

Lodewijk just  had his  eightieth birthday, Acharn  Sujin had  asked us to  carry  

the relics in turn towards the altar, and Lodewijk also carried them back to the  

shrine at the end of the ceremony. 

   Afterwards   we   offered   Sangha   Dāna   to   the   monks.   Lodewijk   spoke   the  

following words of thanksgiving to them : 



      Venerable Monks, 



      On behalf of this group of Thai and foreign pilgrims under the spiritual  

      leadership of Acharn Sujin Boriharnwanaket and the practical leadership of Mr.  

      Suwat Chansuvithiyanant, I wish to thank you for giving us this opportunity to  

      perform Sangha Dāna and to pay you our deep respect. Your community of  

      monks reminds us of the vital importance of the Sangha, the third of the Triple  

      Gem, now and in the future. 

         Last week, my wife Nina and I celebrated my eightieth birthday by paying  

      respect to the place Kuru in New Delhi, where the Lord Buddha preached the  

      Satipaṭṭhāna Sutta. Nina recited the text to me and I was, again, struck by the  

      power of this Sutta and its significance for our daily life. 

         This morning, I received the most precious birthday present one can wish for:  

      the honour to carry the relics of the Lord Buddha. 

         The two most important and happiest events in my life were marrying Nina  

      and our encounter with Buddhism through the hands of Acharn Sujin who, ever  

      since, has guided us on the Path and who, during this tour, tirelessly explained  

      the Dhamma to us, wherever and whenever possible. 

         Looking back on my life, I feel distressed by the amount of accumulated  

      akusala committed in the past. 

         I feel distressed by dukkha, by the burden of the five Khandhas of grasping,  

      so well explained in the teachings: rūpa khandha, vedanā khandha (feeling),  

      saññā khandha (remembrance), saṅkhāra khandha (mental formations) and  

      viññāṇa khandha. 

         I feel distressed by the destructive power of the five hindrances, so forcefully  

      put forth in the teachings, which are: desire of sense pleasures, aversion,  


----------------------- Page 6-----------------------

4 ● Alone with Dhamma 



      restlessness and worry, sloth and torpor and doubt. 

        And yet, I understand at least in theory, that regret of the past makes no  

      sense, that there is no self in the past, and that it is understanding of the  

      present moment that counts. 

        And, in fact, I have every reason to be grateful. 

         Every day, I am encouraged and inspired by Nina’s tireless efforts to  

      understand the Dhamma and to help others to understand it. 

         I was inspired by the courage of Nina’s father who recently passed away at  

      the age of hundred and four and who, despite his incapacities of body and  

      mind, never gave up and always looked towards the future. 

        And above all, who should be distressed when he hears the voice of the Lord  

      Buddha: “Abandon evil, O monks. One can abandon evil, O monks. If it were  

      impossible to abandon evil, I would not ask you to do so. But as it can be done,  

      therefore, I say: Abandon evil!”, and similarly on cultivating the good. 

         On our long j ourney towards wisdom, we need the support and the  

      inspiration of the Sangha and therefore, I urge you, venerable monks, to  

      persevere in your task of preserving and propagating the teachings. 

         We thank you for giving us this opportunity to perform Sangha Dāna and as  

      a token of our thanks, I wish to present to you, Venerable Head Monk, Acharn  

      Sujin’s book, “A Survey of Paramattha Dhammas”, translated from Thai by  

      Nina and recently published in Bangkok. It is a masterful, all encompassing  

      treatise on the Dhamma and I hope that it will be of use to your community. 



                                             ****** 



When   we   were   in   Lumbini,   the   Buddha’s   birth   place,   circumambulating  

Asoka’s pillar, I could not find Lodewijk. I was very sad because I thought that  

this would be the last occasion that we would circumambulate together in this  

place.  After   a  while   he  found  me  and  we   could  continue  together,  paying  

respect  to  the  Buddha.  I  remarked  to  him  that  the  dosa  that   arose  was   a  

perfect  subject  of  satipaṭṭhāna.  We   never  know  what  will  happen  the  next  

moment   and   I   realized   that   the   greatest   respect   to   the   Buddha   is   being  

mindful  of the  dhamma that presents  itself  at the present  moment. We  may  

feel lonely, but in the ultimate sense we are alone with Dhamma, the teaching  

of the truth of life and death. 



We  notice  dosa  and  feel  unpleasant  feeling. We  believe  that we  experience  

dosa,  but   we   do  not  realize  it  as  a  dhamma,  arising  because   of  its  proper  

conditions. I realized that noticing dosa is only thinking about it, and at such a  

moment   there   is   still   an   idea   of   “my   dosa”.   Acharn   Sujin   said   that  

understanding based on listening leads to thinking in the right way  of nāma  

and rūpa; only if right understanding has become more firmly established, it  


----------------------- Page 7-----------------------

                                                                   Alone with Dhamma ● 5 



can condition direct awareness of realities. If there are no conditions for it, it  

is impossible to be directly aware of dosa and realize it as a dhamma, non-self.  

   At   the  first  stage  of  insight,  the  difference  between  the   characteristic  of  

nāma  and  of  rūpa  is  clearly  realized,  not  before.  One  begins  to  understand  

nāma as nāma and rūpa as rūpa. This means, we begin to see the nāma that  

appears as a dhamma and the rūpa that appears as a dhamma. 

   Before this  stage of insight is reached, there  is  still  a notion of  “my dosa”,  

and  “my  lobha”,  even though we have  intellectual understanding  of the  fact  

that  they  are  cetasikas,  non-self. We have  not  really  penetrated the  truth  of  

anattā. 



The wrong view of self is eradicated by the sotāpanna. He still has lobha and  

dosa  but   he  does  not  take  them  for  “self”  and  this  makes  a  great  deal  of  

difference.   At   this   stage   paññā   can   see   defilements   as   dhammas,   arisen  

because of their own conditions. Attachment to sense objects and aversion are  

eradicated at the third stage of enlightenment, the stage of the non-returner.  

Paññā  has  to  be  developed  in  the  right  order,  the  right  order  in  stages  of  

insight, and stages of enlightenment. 

   We should begin to develop understanding of what appears now. When our  

eyes are open seeing appears. Seeing could not arise if there were no eyesense  

and visible   object  or  colour.  Seeing  is  a  dhamma  that  arises  because  of  its  

proper conditions. Realities appear one at a time through the six doors. When  

hardness  appears,  it  seems  that  it  was   there  already  for  some  time,  but  in  

reality it arises and then falls away. There is hardness again, but it cannot be  

the same. Each dhamma that arises and falls away never returns. So it is with  

lobha and dosa, they seem to last for a while, but they fall away immediately.  

However, we think of them for a long time. When we are thinking, there is no  

realization   of   their   characteristics.   We   think   of   what   is   past   instead   of  

attending to the characteristic appearing right now. 

   All that is experienced is dependent on citta that arises and then falls away  

immediately. Citta is the chief in knowing an object and it is accompanied by  

several  cetasikas  that  each  perform  their  own  function.  Citta  and  cetasikas  

experience  objects  through   the   five   sense-doors   and   the  mind-door.   Rūpa,  

physical   phenomena,  can  be   experienced  through   the   sense-doors   and   the  

mind-door  and  nāma,  mental  phenomena,  can be  experienced  only  through  

the mind-door. 

   The   rūpas   of   visible   object,   sound,   odour   and   flavour   are   experienced  

through  the  relevant  sense-doors  and  subsequently  through  the  mind-door.  

The   rūpas   that   are   tactile   object   are:   solidity   appearing   as   hardness   or  

softness,  temperature,  appearing  as  heat  or  cold,  and  motion,  appearing  as  

motion and pressure. These rūpas are experienced through the bodysense and  

subsequently through the mind-door. Thus, seven rūpas appear all the time in  


----------------------- Page 8-----------------------

6 ● Alone with Dhamma 



daily life, and without citta and cetasikas they could not appear. 

   The  Buddha  taught  us  to  develop  understanding  of  ultimate  realities,  of  

rūpa and nāma which includes citta and cetasikas, mental factors arising with  

the   citta.   In   this   way   wrong   view   and   all   defilements   can   be   eradicated.  

Ultimate realities or paramattha dhammas are different from concepts such as  

persons,  things  or  events we  may think  of. Ultimate  realities  are the  objects  

right understanding should be developed of. 

   We may study citta, cetasika and rūpa, but we are still bound to take citta  

for “my experience” and rūpa for “my body” or “my possessions”. We need to  

listen again and again and consider what we heard so that understanding can  

gradually develop. 


----------------------- Page 9-----------------------

                                                                                          ● 7 



Chapter 2 



Remembrance of the Dhamma 



When we were visiting the Thai Temple in Bodhgaya, two neatly dressed boys  

with neckties entered. Young as they were, they recited for us the beginning of  

the   Dhammasangaṇi,   the   first   Book   of   the   Abhidhamma,   enumerating   all  

sobhana   cetasikas   that   accompany   kusala   citta.   They   recited   the   Pāli   text  

without mistakes, helping each other. It was impressive to hear the Pāli text so  

well recited. 

   When we asked them about the meaning of kusala cittas and akusala cittas  

in daily life they could not answer our questions. Their teacher explained that  

the application of the Dhamma was a subject dealt with in the higher grades,  

not in the beginning. This shows that reciting and learning the terms by heart  

is very  different  from  applying  them  in  daily  life.  The  understanding  of  the  

realities of daily life is the purpose of studying Abhidhamma. 



Acharn  Sujin  said  to  us:  “One  does  not  know  that  studying Abhidhamma  is  

right   now.   If   one   does   not   understand   this   moment,   one   does   not   study  

Abhidhamma.” 

   We asked many questions during our pilgrimage,  and Acharn  Sujin  always  

led us back to the dhamma appearing at the present moment. 

   Without  awareness  and  understanding  of what  appears  now we  shall  not  

know what citta, cetasika and rūpa are. One may wonder what the difference  

is between thinking of realities and direct awareness of them. 

   Sati  arises with  sobhana  (beautiful) citta. There is sati of the level of dāna  

which is non-forgetful of generosity. There is sati of the level of sīla which is  

non-forgetful with regard to abstinence from akusala. There is sati of the level  

of   samatha   which   is   mindful   of  the   development   of   calm   with   a  suitable  

meditation subject. There is sati of the level of satipaṭṭhāna which is mindful  

of one nāma or rūpa at a time in order to realize their true nature. 

   We  discussed  sati  of  the  level  of  satipaṭṭhāna  time  and  again  since  it  is  

important to have right understanding  of what  sati  is  and what its  object  is.  

We   know   that  sati  is  a  sobhana  cetasika  and  not  self,  but  have   we   really  

understood  this?  We  learnt  that  sati  is  aware  of  one  object  at  a  time  as  it  

appears through one doorway at a time, but can we apply this knowledge? 

   We  have   to  listen  again  and  again  and  consider  in  order  to   have  more  

understanding. Acharn Sujin repeated three times: have more understanding. 


----------------------- Page 10-----------------------

8 ● Alone with Dhamma 



The  development  of  satipaṭṭhāna  is  the  development  of  right  understanding  

that   sees   visible   object,   seeing,   sound,   hearing,   whatever   appears,   as   a  

dhamma that  arises because of the  appropriate conditions. Acharn  Sujin  said  

many times that seeing arises because of eyesense, which is rūpa, and visible  

object  which   is  also  rūpa.  Seeing  experiences  visible   object  or  colour,  it  is  

nāma. It has no shape or form. 

   Acharn Sujin untiringly helps us to understand what is appearing now, such  

as seeing and visible object. She repeats time and again “Is there seeing now?”  

and she explains the characteristic of the dhamma that appears at the present  

moment.  It  is  always  new  to  me  to  hear  again  and  again  about  seeing  and  

visible   object.   Gradually   her   words   become   more   meaningful   so   that  

understanding  of  realities  can  grow,  although  it  may  be   hardly  noticeable.  

Acharn used a simile of a tree with buds. One may not notice that the tree will  

sprout, but one day it will. 

   Seeing has a specific characteristic, it is different from thinking about people  

and  things  we   notice. Visible  object  has  a  specific  characteristic,  it  appears  

through eyesense. People and things do not impinge on the eyesense, they are  

concepts   we   think   of.   We   have   to   be   reminded   of   realities   so   that  

understanding  of  them  can  grow.  Such  understanding  can  condition  direct  

awareness,  but we  should  have  no  expectations,  as  she  often  said.  There  is  

seeing now, but do we study it with mindfulness? We are forgetful of seeing,  

we think rather of persons and things we perceive. 

   Hardness appears many times a day and usually we are forgetful. Hardness  

is experienced by body-consciousness, but this is different from  awareness of  

hardness. When hardness appears, anyone, even a child can tell how hard it is.  

Hardness is experienced by body-consciousness, and this is not awareness. 

   When there  are conditions for the  arising of mindfulness the characteristic  

of  hardness   or   of  the   experience   of  hardness   can   be   understood.   At   such  

moments  one  does  not  cling  to  concepts  such  as  a  hand  or  a  table  that  is  

touched. 

   We  may  think   of  “our  possessions”,   but   through   which   doorway  do  we  

experience possessions? Through eyes only colour is experienced and this falls  

away immediately. Through touch only tangible object is experienced and this  

falls away immediately. What arises because of the appropriate conditions has  

no   owner.   We   have   to   consider   this   again   and   again.   Awareness   and  

understanding of one object at a time as it appears through one doorway at a  

time will lead to detachment from the idea of self and “mine”. 

   Everything  is  dhamma.  We   know  this  by   intellectual  understanding,  but,  

later on, it can be known more deeply by direct understanding. In this way a  

higher level of understanding is reached, the level of insight knowledge. 


----------------------- Page 11-----------------------

                                                           Remembrance of the Dhamma ● 9 



We   read  in  the  “Kindred  Sayings”  (IV,  Ch  III,  §81,  a  Number  of  Bhikkhus,  

translated  by Ven.  Bodhi)  that  a  number  of  bhikkhus  told  the  Buddha  that  

sectarians asked them what the purpose of the holy life lived under the ascetic  

Gotama was.  They  asked whether  they  had  answered  rightly  in  saying  that  

this was for the full understanding of suffering, dukkha. The Buddha said that  

this  was  right,  but   if  they  would  ask  what   that  suffering  was   they  should  

answer thus: 



                                                 1 

      “The eye, friends, is suffering... Forms  are suffering... Whatever feeling arises  

      with eye-contact as condition... that too is suffering... the mind is suffering...  

      Whatever feeling arises with mind-contact as condition...that too is suffering: it  

      is for the full understanding of this that the holy life is lived under the Blessed  

      One. This, friends, is that suffering for the full understanding of which the holy  

      is lived under the Blessed One.” 



The Buddha taught the development of right understanding of all dhammas as  

they appear through the six doors. 



At the end of our pilgrimage we spend a few days in Kashmir on houseboats. I  

was  clinging  to  Dhamma  discussions  and  I  asked Acharn  Sujin whether we  

could have a discussion the next day. She answered that we do not know what  

the next moment will bring and that also hearing Dhamma is anattā. 

   The truth has to be applied in daily life. Whatever we hear is conditioned,  

hearing   is   the   result   of   kamma.   Hearing   Dhamma   is   the   result   of   kusala  

kamma;  hearing  is vipākacitta   which  is  conditioned  and  which   nobody  can  

cause to  arise.  It  is  of no use to wish for the  arising  of  certain vipākas  or to  

have any expectations. 



Acharn Sujin said that whenever there is more understanding it indicates that  

there   has   been   right   consideration   of   realities.   When   we   consider   and  

investigate  different   dhammas,  this   is  accompanied  by   a  level  of  sati,  sati  

stemming   from  listening  to   the  Dhamma.  In  this   way  direct  awareness  of  

realities will  arise  naturally, without  one  trying  to be  aware.  If  one  tries  to  

make   awareness   arise   it   is   counteractive;   clinging   to   self   obstructs   the  

development of paññā. 

   She also said that nobody can tell whether there will be sati now.  Sati can  

arise before we are thinking about it. We should know the difference between  

forgetfulness  of  dhammas  and  mindfulness  of  one  dhamma  at  a  time  as  it  

appears through  one  doorway.  Otherwise we keep  on talking  about  sati but  

we are ignorant of its characteristic. 

   We may be discouraged to realize that very few moments of sati arise in a  



1   Rūpas, visible objects. 


----------------------- Page 12-----------------------

10 ● Alone with Dhamma 



day   or   none   at   all.   Acharn   Sujin   said:   “Instead   of   attending   to   the  

characteristics  of  realities  there  is  thinking  about  them.  But  we   should  not  

have any expectations, otherwise it is me who would like to have progress.” I  

was  grateful  for  such  reminders. Time  and  again  attachment to  result  arises  

but   we   do   not   notice   this.   Acharn   Sujin   said:   “The   development   of  

understanding   has   to   go  along   with   detachment   all   the   way,   and   that   is  

against  the  current  of  life.”  We   are  inclined  to  think  of  a  self  who   has  to  

become proficient. 



People wonder whether there are ways to induce the arising of sati such as the  

development   of   calm.   They   doubt   whether   listening   to   the   Dhamma   is   a  

condition for the arising of sati. 

   We read in the “Gradual Sayings” , Book of the Fours, Ch XXV, §6: 



      “Monks, these four states conduce to growth in wisdom. 

         What four? 

        Association with a good man, hearing Saddhamma, thorough work of mind,  

      and behaviour in accordance with Dhamma. 

         These are the four.” 



 A good man is the translation of sappurisa, which usually denotes an ariyan  

who   is  a  good  friend  in  Dhamma.  Saddhamma  is  true  dhamma.  Thorough  

work of mind  stands for yoniso manasikāra, which  is right  attention to what  

one hears. 

   Behaviour   in   accordance   with   Dhamma   is   dhammānudhammapaṭipatti,  

practice in accordance with the Dhamma. It is the application of the Dhamma  

one   has   heard   and   thoroughly   considered   through   the   development   of  

satipaṭṭhāna. 

   When we read that listening is an important condition we should remember  

that also considering the Dhamma and its application are implied. 



During   our   pilgrimage   Acharn   Sujin   emphasized   the   role   of   saññā,  

remembrance,  and  in  particular  remembrance  of  Dhamma  as  the  proximate  

cause of satipaṭṭhāna. 

   Saññā arises with each citta and its function is “marking” and remembering  

the object that citta experiences. At the moment of seeing, saññā marks visible  

object and when seeing has fallen away saññā arising with the following cittas  

performs its function of marking and remembering. It accompanies the cittas  

that  define  and  name what  has been  seen.  On  account  of visible  object we  

think   about   persons   and   things   and   saññā   performs   its   function   while   it  

accompanies thinking. We think time and again of persons, things and events,  

but without saññā there could not be such thinking. Cittas arise and fall away  


----------------------- Page 13-----------------------

                                                      Remembrance of the Dhamma ● 11 



very  rapidly  and  it  seems, that  seeing,  defining  and  thinking  all  arise  at  the  

same   time,   but   only   one   citta   arises   at   a   time.   When   we   recognize   or  

remember things we should know that it is saññā, not self, that is doing so. 

   Lodewijk   remarked   that   memories   of   the   past   can   worry   us.   There   are  

things we  do  not want to  remember, but  memories  still  come back  and this  

shows  that  saññā  is  beyond  control.  Acharn   Sujin  said:  “When  there  is  no  

paññā it is not known that saññā is anattā. We think about it and do not want  

to   have   it,   but   instead   we   should   understand   how   it   arises   because   of  

conditions.” 

   I  said  that   we   are  more   inclined   to   think   than   to   be   directly   aware  of  

realities.  

   Acharn Sujin answered: “Paññā can see the difference between thinking and  

the direct experience of the truth. We cling to wholesome thinking and to the  

importance of self who thinks.” 

   We   usually   remember   concepts   of   people   and   things   and   we   continue  

thinking of them.  Seeing  arises and it experiences only visible object. Acharn  

Sujin said: “No one can change the characteristic of seeing which experiences  

visible object now. But there is not always remembrance of the Dhamma.”  

   Saññā which remembers the Dhamma is different from saññā which arises  

when   thinking   of   concepts.   On   account   of   what   is   seen   we   think   with  

attachment  about  events,  people  and  things  and  are  quite  taken  in  by  our  

thoughts.   However,   gradually   saññā   can   remember   what   we   heard   when  

listening   to   Dhamma.   Saññā   arises   with   intellectual   understanding   of   the  

Dhamma,   it   remembers   the   terms   and   their   meanings   and   when   there   is  

mindfulness saññā can also remember characteristics of realities. Saññā which  

remembers  the  Dhamma  will  be  firmer  so  that  it  will  become  a  proximate  

cause of satipaṭṭhāna. 



   We read in the Expositor (I, Part IV, Chapter 1, 122) about mindfulness: 



      ... Mindfulness has  “not floating away” as its characteristic, unforgetfulness as  

      its function, guarding, or the state of facing the obje ct, as its manifestation,  

      firm remembrance (saññā) or application in mindfulness as regards the body,  

      etc., as proximate cause. It should be regarded as a door-post from being firmly  

      established in the object, and as a  door-keeper from guarding the door of the  

      senses. 



I  had  read  this  text  before  but  I  had  not yet   considered  all  implications  of  

saññā’s  role.  Considering  the  function  of  saññā  helps  us  to  understand why  

listening to the Dhamma is most important. Saññā performs its function when  

we  listen to  the  Dhamma  or  read  Suttas.  If we  are  not  passive  listeners but  

also understand what we hear  or read, we  shall  not be  forgetful.  Saññā  can  


----------------------- Page 14-----------------------

12 ● Alone with Dhamma 



become a firm foundation for the arising of direct awareness of the dhammas  

that  appear. Another word  for  awareness  is  non-forgetfulness. We  are  often  

forgetful  of  nāma  and  rūpa,  but  since we  have  listened  and  considered  the  

Dhamma   there   are   conditions   for   remembrance   of   the   Dhamma   and   this  

supports sati. 

   It  seems to us that  dhammas last,  at least for  some time. We think  of our  

body  as  a  whole  that  exists  because  of  saññā  that  remembers  it.  In  reality  

there are rūpas arising and falling away all the time. When hardness appears it  

seems to last, but  in reality  it falls  away  immediately  and never returns. We  

keep on remembering the hardness that has fallen  away, but very  gradually,  

when   paññā   has   been    developed,   there    can   be   remembrance    of  

impermanence,  anicca  saññā  instead  of  remembrance  of  permanence,  nicca  

saññā. The clinging to the concept of self is so deeply rooted, but when paññā  

has been developed to the stage of enlightenment of the sotāpanna, there will  

be anattā saññā instead of attā saññā. 

     

We read in the Gradual Sayings, Book of the Tens, Ch VI, §6, Ideas, about ten  

kinds of saññā, here translated as ideas: 

     

      Monks, these ten ideas, if made to grow and made much of, are of great fruit,  

      of great profit for plunging into the deathless, for ending up in the deathless.  

      What ten ideas? 

         The idea of the foul, of death, of the repulsiveness in food, of distaste for all  

      the world, the idea of impermanence, of ill in impermanence, of not-self in Ill,  

      the idea of abandoning, of fading, of ending... 



Plunging   into   the   deathless   means   the   attainment   of   nibbāna.   This   sutta  

implies   that   all   these   ten   kinds   of   saññā   are   developed   together   with  

satipaṭṭhāna, otherwise they could not lead to the deathless. 



The following Sutta explains the connection of similar kinds of saññā with the  

“thorough comprehension of lust”. 

   We read in the Gradual Sayings, Book of the Tens, Ch XXII, §8, Lust: 



      Monks, for the thorough comprehension of lust ten qualities should be made to  

      grow. What ten? 

         The idea of the foul, of death, of the repulsiveness in food, of non-delight in  

      all the world, of impermanence, of ill, of the not-self, of abandoning, of fading  

      interest, and the idea of ending... 



Thus, it is emphasized here that paññā leads to detachment  . The arahat has  

thorough comprehension of lobha, so that it can be completely eradicated. 


----------------------- Page 15-----------------------

                                                                                          ● 13 



Chapter 3 



Samatha and Vipassanā 



If  one  is  disturbed  by  strong  defilements  it  is  most  difficult  to  be  aware  of  

one’s akusala cittas. Defilements prevent the  arising of  satipaṭṭhāna. Clinging  

to sense objects may be so strong, it can even motivate akusala kamma patha.  

Should one not develop calm first, so that insight can be developed afterwards  

with more ease? This is  a question that is often  asked,  and in India this was  

also discussed. 



People  of  old  saw  the  disadvantages  of  sense  impressions.  They  knew  that  

seeing is very often followed by attachment. Therefore they developed calm to  

the   degree   of   absorption   concentration,   jhāna ,   so   that   they   would   be  

temporarily   freed   from   sense   impressions   and   the   defilements   arising   on  

account of them. However, defilements are not eradicated by samatha. When  

one emerges from jhāna  insight of all nāmas and rūpas is to be developed so  

that enlightenment can be attained. 

   In the development of samatha paññā must be very keen so that it discerns  

precisely   the   different   cetasikas   that   are   jhā na-factors   which   have   to   be  

developed, and, in order to reach the higher stages of jhā na, one has to know  

which   are  the  coarser jhāna -factors which   have  to  be  abandoned. A   person  

must have accumulated great skill in order to attain jhā na. Paññā is necessary  

so that calm  is  developed in the right way.  One has to know precisely when  

kusala citta  arises  and when  akusala citta. If there is  subtle clinging to calm,  

one may mislead oneself. 

   We  discussed  in  India whether  there  is  a  certain  order  of  development  of  

sīla,  samādhi  and  paññā.  It  seems  that  the  texts  of  the  suttas  point  in  this  

direction. 



We read in the Mahāparinibbānasutta  (Dialogues of the Buddha, no.16) that  

the Buddha repeatedly said: 



      Such and such is sīla, such and such is concentration, such and such is wisdom. 

         Great becomes the fruit, great is the gain of concentration when it is fully  

      developed by sīla. 

         Great becomes the fruit, great is the gain of wisdom when it is fully  

      developed by concentration. Utterly freed from the intoxicants (āsavas) of Lust,  


----------------------- Page 16-----------------------

14 ● Alone with Dhamma 



      of becoming and of ignorance is the mind that is fully developed in wisdom. 



When   one   reads   this   text   it   seems   that   there   has   to   be   sīla   first,   then  

concentration and then paññā. We discussed this with Acharn Sujin who said:  

“Can sīla and samādhi be fully developed without paññā?” 

   The   sotāpanna   has   fully   developed   sīla,   he   cannot   transgress   the   five  

precepts   nor   commit   akusala   kamma   leading   to   an   unhappy   rebirth.   The  

anāgāmī  has   fully   developed  calm,  he  has   eradicated  all  clinging  to  sense  

pleasures. Sīla and samādhi become fully developed by paññā. 



We read in the Commentary to the Mahāparinibbānasutta: 



      Such and such is sīla (virtue), meaning, it is indeed sīla, sīla to that extent; here  

      it is sīla which are the four purities of sīla.  

         Samādhi is concentration. Wisdom should be understood as insight wisdom  

      (vipassanā). 

         As to the words, when it is fully developed by sīla, this means, when he has  

      abided in that sīla etc., these produce concentration accompanying the path- 

      consciousness and fruition-consciousness; when this is fully developed by that  

      sīla it is of great fruit and of great benefit.  

         When he has abided in this concentration, they produce wisdom  

      accompanying the path-consciousness and fruition-consciousness, and this,  

      when it is fully developed by this concentration, is of great fruit, of great  

      benefit.  

         When he has abided in this wisdom, they produce the path-consciousness  

      and fruition-consciousness, and thus when it is fully developed by this (wisdom)  

      he is completely freed from the intoxicants. 



Thus, when we  read  about  full  development  this  pertains  to  lokuttara  cittas  

arising at the different stages of enlightenment. 



The   “Visuddhimagga”,   in   the   Chapter   on   Virtue,   Sīla,   gives   the   following  

fourfold classification of purity of sīla  (pārisuddhi sīla): 



      the restraint of “Pāṭimokkha” including 227 rules of discipline for the monk, 

      the restraint of the sense faculties (indriya saṁvara sīla), 

      the purity of livelihood (ājīva pārisuddhi sīla), 

      the use of the four requisites of robe, dwelling, food and medicines, that is  

      purified by reflection (paccaya sannissita sīla). 



As   regards   restraint   of   the   sense   faculties,   there   are   different   levels   of  

restraint. We read in the “Middle Length Sayings”  (no. 27, Lesser Discourse on  


----------------------- Page 17-----------------------

                                                            Samatha and Vipassanā ● 15 



the Simile of the Elephant’s Footprint) that the Buddha spoke to the brahman  

Jāṇussoṇi about the monk who has restraint as to the sense-faculties: 



      ... Having seen visible object wit h the eye he is not entranced by the general  

      appearance, he is not entranced by the detail. If he dwells with this organ of  

      sight uncontrolled, covetousness and dejection, evil unskilled states of mind,  

      might predominate. So he fares along controlling it; he guards the organ of  

      sight, he comes to control over the organ of sight.... 



The   Buddha   taught   satipaṭṭhāna   so   that   the   wrong   view   of   self   can   be  

eradicated.   Through   satipaṭṭhāna   right   understanding   is   developed   and  

without satipaṭṭhāna sīla cannot become “well established”. For the sotāpanna  

who has developed vipassanā, sīla is “well established”.  Satipaṭṭhāna includes  

training  in  “higher  sīla”  (adhi-sīla  sikkhā),  “higher  citta”  (adhi-citta  sikkhā)  

and “higher wisdom”  (adhi-paññā sikkhā). 

   Instead  of  thinking  of  classifications  and  names  or  thinking  of  a  specific  

order   as   to   the   development   of   sīla,   concentration   and   paññā,   we   can  

gradually   develop   understanding   of  the   nāma   and   rūpa   appearing   at  this  

moment and this is training in higher sīla, higher citta and higher paññā. 

   As to higher citta or concentration, this includes all levels of concentration,  

not   merely   jhā na.   Concentration,   samādhi,   is   the   cetasika   which   is   one- 

pointedness,  ekaggatā cetasika. It  arises with each  citta  and has the function  

of   focussing   the   citta   on   one   object.   When   satipaṭṭhāna   arises,   ekaggatā  

cetasika  “concentrates”  for  that   short  moment  on  the  nāma  or  rūpa  which  

appears so that understanding of that reality can develop. In the development  

of  samatha concentration is developed to  a high degree  so that jhāna  can be  

attained,   but   this   cannot   be   achieved   without   paññā   which   has   right  

understanding   of   the   citta   and   cetasikas   which   develop   calm.   In   the  

“Visuddhimagga” all levels of concentration, jhāna included, are described, but  

this   does  not   mean   that   everybody   must   develop jhāna   in   order  to   attain  

enlightenment. 



We   read   in   the   Tipiṭaka   about  jhāna ,   but   we   should   remember   that   the  

Commentaries   distinguish   between   two   kinds   of   jhāna.   We   read   in   the  

Commentary to the Sallekhasutta  (M.N. I, sutta 8) about two meanings of the  

expression: meditate  (jhāyathā). The objects  of meditation  or contemplation,  

jhāna,   can   be   the   thirty-eight   objects   of   samatha   or   the   characteristics  

beginning   with   impermanence   (anicca)  of  the   khandhas   and  the   āyatanas  

 (sense-fields).   The   Commentary   states:   “It   is   said:   ‘Develop   samatha   and  

vipassanā’.” It repeats that one should not be negligent. 

   We  read  in  the   Subcommentary  to   this   passage:   “With  mindfulness   and  

clear   comprehension   (sati-sampajañña),   which   means:   by   grasping   with  


----------------------- Page 18-----------------------

16 ● Alone with Dhamma 



thorough comprehension.” 

     

Acharn  Sujin said: “One  says that calm is helpful, but why does one not  say:  

all moments  of kusala  are helpful? If one has more kusala  cittas in daily life  

there is also calm. Why does one not develop more kusala in daily life instead  

of high levels of calm?” 

   Any kind of kusala through body, speech and mind brings calm, and at such  

moments one does not think of oneself or one’s problems. Each kusala citta is  

accompanied by calm  (passaddhi). 

   Different kinds of kusala have been classified as the ten bases of meritorious  

deeds  and these  can be  developed  in  daily life. These  include three kinds  of  

dāna:  giving  away useful things to  others,  appreciating the kusala  of  others,  

and extending merit, which means: making known one’s kusala to others, no  

matter whether  they  are  alive  or have  passed  away  to  another  plane where  

they are able to rejoice in one’s kusala. Moreover, they include three kinds of  

sīla:  abstaining  from  akusala,   helping   others  and  paying  respect  to  others.  

Then there is bhāvanā, which includes:  studying  or  explaining the Dhamma,  

samatha and vipassanā. 

   Furthermore, the tenth base is rectifying one’s views  (diṭṭhujukamma). This  

is connected with all other kinds of kusala and there are different degrees of  

it. One degree of rectifying one’s views is knowing the value of kusala and the  

disadvantage   of   akusala.   Another   level   is   understanding   that   one   can  

eliminate akusala by means of generosity, sīla and other good deeds. Another  

degree  is knowing that  one  can  subdue  defilements by  developing  calm  and  

another level is understanding that paññā can be developed with the purpose  

to  eradicate  defilements.  The  ten  bases   of  meritorious  deeds  show  us  that  

there  are  always  opportunities  for kusala  in  daily  life. When we  read  in the  

texts about calm we should not forget that there are many kinds and degrees  

of calm. Calm does not only pertain to the calm of jhāna, but also to calm that  

accompanies  the  different ways  of  kusala  performed  in  daily  life,  dāna,  sīla  

and bhāvanā, mental development. 

   We  read  in  the  subcommentary  to  the  Satipaṭṭhānasutta  (M.N.  10)  about  

meditation subjects that can condition calm in daily life: 



      The words, the meditation subjects on all occasions, mean: recollection of the  

      Buddha, loving-kindness, mindfulness of death, and meditation of foulness. 

         This set of four meditations which is guarded by the yogi (practitioner), he  

      called ‘the meditation subjects on all occasions’. 

         They should be guarded by the power of thorough comprehension,  

      uninterruptedly, with sati that is called samatha, calm, because of its being  

      included in the group of concentration, samādhi. 


----------------------- Page 19-----------------------

                                                               Samatha and Vipassanā ● 17 



The   factors   of   the   eightfold   Path   can   be   classified   as   three   divisions:   as  

wisdom,   sīla  and  samādhi,  concentration  or  calm.  Right  understanding  and  

right thinking constitute the wisdom of the eightfold Path, right speech, action  

and  livelihood  the  sīla  of  the  eightfold  Path,  and  sati,  right  effort  and  right  

concentration the calm of the eightfold Path. The factors of the eightfold Path  

are   developed   together.   When   right   understanding   develops,   also   calm  

develops together with it. 

   The  four  meditation  subjects  mentioned  above  are very  suitable  for  daily  

life, for all occasions  (sabbatthika). There may be conditions for their arising,  

but  one  should  not  cling  to  such  moments. When  calm  arises  there  can  be  

awareness and right understanding of it as a type of nāma. Kusala, akusala, all  

types of dhammas arise because of their own conditions and nobody can make  

them arise or prevent them from arising. 

   When   we   develop   understanding   of  nāma   and   rūpa   we   can   think   with  

gratefulness  of  the  Buddha   who   taught  us  the  Path  leading  to  the  end  of  

defilements.  This  is  a  short  recollection  of  the  Buddha  and  at  that  moment  

mindfulness can arise of kusala citta as a conditioned nāma which is non-self. 

   In  addition,  loving-kindness  is  to be  developed  in  daily  life.  However, we  

should know that it is very difficult to  see the difference between true mettā  

and  selfish  affection, which  is  called the  near  enemy  of  mettā. Akusala  citta  

follows upon kusala citta very closely and they succeed one another extremely  

rapidly. We are likely to mislead ourselves time and again and take for kusala  

what is akusala. 

   In India we were time and again disturbed, even molested by beggars, and  

we were inclined to turn away from them. Acharn  Sujin said that if we think  

of  life  as  a beggar,  of kamma that  conditions  such vipāka, kusala  citta with  

compassion   and   loving-kindness   can   arise   instead   of   akusala   citta   with  

aversion. We can see other people, beggars included, as our children. At such  

moments we  can  notice  that  kusala  citta  has  a  characteristic  different  from  

akusala  citta.  When  kusala   citta  arises,  we  are  calm  and  not  disturbed  by  

someone  else’s   contrarious  behaviour.   Both  samatha  and  vipassanā   can  be  

developed together in daily life. 



We  read that when bhikkhus wanted to leave the  order the  Buddha  advised  

them to  contemplate  asubha, foulness. When  one  contemplates  foulness  one  

does  not  indulge  in  sense  pleasures.  Some  people  believe   that,  before   one  

develops satipaṭṭhāna, one should subdue sense desires by focussing the mind  

on  foulness.  This,  however  is  not  correct.  There  is  no  rule  that  one  should  

perform   particular   actions   before   one   develops   satipaṭṭhāna.   Through  

satipaṭṭhāna right understanding  is  developed of whatever dhamma  appears,  

be it kusala or akusala. But also when one thinks of foulness, sati can be aware  

of a reality, for example of the dhamma that thinks of foulness. 


----------------------- Page 20-----------------------

18 ● Alone with Dhamma 



   The   contemplation   of   foulness   can   lead   one   to   a   deeper   way   of  

contemplation,  the  realization  of  impermanence.  This  is  the  development  of  

insight. 



We read in the Theragāthā, Canto CXVIII, Kimbila, that the Buddha, in order  

to stir him, conjured up  a beautiful woman in her prime, and showed her to  

him passing to old age. 

   Kimbila uttered the verse: 



      As bidden by some power age over her falls. 

      Her shape is as another, yet the same. 

      Now this myself, who never have left myself, 

      Seems other than the self I recollect. 



Thus, when  a person becomes  older his body  change  although he  is  still the  

same individual. The body consists of rūpas that arise and fall away. 

   What  arises  and  falls  away   is  not  beautiful,  not  attractive.  The  colour  a  

person sees is only colour, not feminine beauty. That colour falls away, never  

to return. Where is the beauty? 

   At the first  stage  of principal  insight the  arising  and falling  away  of nāma  

and   rūpa   is   realized.   Kimbila   listened   to   the   Buddha   and   developed  

understanding. 



One may want realities to be different from what they are, but the dhammas  

that have arisen already cannot be changed. One may unknowingly cling to an  

idea of a self who must subdue lobha and dosa. One may cling to the idea of  

wanting to be a good person without defilements. When one fails to suppress  

lobha and dosa there is frustration and disappointment, even despair. 

   We may  notice that we have  aversion,  dosa, but we  do  not realize  it  as  a  

dhamma, arising because of its proper conditions. At the first stage of insight,  

the   difference   between   the   characteristic   of   nāma   and   of   rūpa   is   clearly  

realized, not before that. One begins to understand nāma as nāma and rūpa as  

rūpa. This means, we begin to  see the nāma that  appears  as  a  dhamma  and  

the rūpa that appears as a dhamma. 

   Before   this   stage   of   insight   is   attained,   there   is   still   a   notion   of   “my  

aversion”,   and   “my   attachment”,   even   though   we   have   intellectual  

understanding of the fact that they are cetasikas, non-self. We have not really  

penetrated the truth of anattā. 

   Acharn   Sujin   said:   “You   have   to   understand   your   own   life,   your  

accumulated inclinations, otherwise you can never become a  sotāpanna. One  

should be very  courageous  in  order to  develop the  real Path,  not the wrong  

Path.” 


----------------------- Page 21-----------------------

                                                                                     ● 19 



Chapter 4 



The Present Moment 



Acharn Sujin brought us back to the present moment time and again by asking  

us: “Is there no seeing now?” I was glad because I am always inclined to think  

of  concepts  about  people  and  things  I  perceive.  Concepts  are  not  objects  of  

vipassanā, they are different from visible object, sound, and all the objects that  

appear through the six doorways. However, thinking itself is a citta and it can  

be an object of insight. 

   We  discussed  seeing  and visible  object  time  and  again. Visible   object  is  a  

rūpa that impinges on the eyesense. It is experienced by seeing-consciousness  

that arises in a process of cittas. Visible object or colour is an extremely small  

rūpa arising in a group of rūpas, it does not arise alone. It arises together with  

the  four  Great  Elements  of  solidity,  cohesion, heat,  motion,  and  other  rūpas  

which support it. It falls away immediately and soon afterwards it arises and  

falls  away  again.  There  is  not  one  unit  of visible  object but  countless  units  

arising   and   falling   away.   We   cannot   pinpoint   which   visible   object   is  

experienced  at  the  present  moment.  There  is  only  an  impression  or  mental  

image, nimitta, of visible object. This causes us to think that visible object does  

not fall away. 

   The following sutta deals with the notion of ‘sign’ or mental image, nimitta. 

   We  read  in  the  Kindred  Sayings  (IV,  Ch  II,  §80,  Ignorance,  translated by  

Ven.  Bodhi)  that   a  bhikkhu   asked  the   Buddha  whether   there  is  one  thing  

through the abandoning of which ignorance is abandoned and true knowledge  

arises. 

   We read that the Buddha answered: “Ignorance, bhikkhu, is that one thing  

through the  abandoning  of which  ignorance  is  abandoned by  a bhikkhu  and  

true knowledge arises.” 



Ven. Bodhi states in a note to this passage: “Though it may sound redundant  

to say that ignorance must be abandoned in order to abandon ignorance, this  

statement underscores the fact that ignorance is the most fundamental cause  

of bondage, which must be eliminated to eliminate all the other bonds.” 



We read further on: 



     Here, bhikkhu, a bhikkhu has heard, ‘Nothing is worth adhering to’. When a  


----------------------- Page 22-----------------------

20 ● Alone with Dhamma 



      bhikkhu has heard,  ‘Nothing is worth adhering to’, he directly knows  

      everything. Having directly known everything, he fully understands everything.  

      Having fully understood everything, he sees all signs (nimitta) differently. He  

      sees the eye differently, he sees forms differently...whatever feeling arises with  

      mind-contact as condition... that too he sees differently... 



As to the term adhere, this pertains to clinging with wrong view. 

   The   Commentary   explains   the   words,   “he   sees   all   signs   differently  

(sabbanimittāni   aññato   passati)”   as   follows:   “He   sees   all   the   signs   of  

formations  (saṅkhāranimittāni)  in  a way   different  from  that  of  people who  

have not fully understood the adherences. For such people see all signs as self,  

but one who has fully understood the adherences sees them as non-self, not as  

self. Thus in this sutta the characteristic of non-self is discussed.” 



In this Commentary the word “saṅkhāra-nimitta”, the nimittas, signs or mental  

images, of conditioned dhammas, is used. When we were returning from the  

Bodhi tree walking up the long  stairways,  a friend  asked Acharn  Sujin  about  

this term. Nimitta has different meanings in different contexts. The nimitta or  

mental image in samatha refers to the meditation subject of samatha. We also  

read in some texts that one should not be taken in by the outward appearance  

of things  (nimitta)  and the details. However, the term saṅkhāranimitta has a  

different meaning as I shall explain further on. 

   Acharn  Sujin emphasized that whatever we read in the texts about nimitta  

should  be  applied  to  our  life  now.  “What we  read  is  not  theory”  she  often  

explains. 



We   read  in  the  “Mahāvedallasutta”  (Middle  Length  Sayings,  no  43),  about  

freedom of mind that is “signless”, and we read that there are two conditions  

for attaining this: “non-attention  (amanasikāra) to all “signs” and attention to  

the signless element”. The Commentary states that the signs, nimittas, are the  

objects such as visible object, etc. and that the signless is nibbāna. The signless  

liberation   of  mind   is  explained  in   a  way   that   clearly  connects  it  with   the  

fruition  of  arahantship:  lust,  hatred  and  delusion  are  declared  to  be  “sign- 

makers” (nimittakarana), which the arahant has totally abandoned. 

     When we  read  about  object  (ārammaṇa)  as  a  sign, we  should  remember  

that  this  is  not  theory. An   object  is  what  citta  experiences  at  this  moment.  

When  the  rūpa  that  is  the  eye-base  has  not  fallen  away  yet   and  colour  or  

visible object impinges on it, there  are conditions for the arising of  seeing. If  

there were no citta which sees visible object could not appear. 

   When we asked Acharn Sujin whether the impression or sign  (nimitta) of a  

dhamma is a concept or a reality she answered: “These are only words. If we  

use the word concept there is  something that is experienced by thinking. We  


----------------------- Page 23-----------------------

                                                                   The Present Moment ● 2 1 



should  not j ust  know  words,   but   understand  the  reality  that  appears  right  

now. There is not merely one moment of experiencing visible object, but many  

moments arising and falling away. When right understanding arises we do not  

have to use any term.” 

   She  repeated  that  there  is  the  impression  of visible  object  right  now.  She  

said: “It is this moment.” Visible object impinges on the eyesense and after it  

has fallen away, what is left is the impression or sign, nimitta of visible object. 

   It seems that visible object lasts for a while, but in reality it arises and falls  

away. Acharn  Sujin  used  the  simile  of  a  torch that  is  swung  around.  In  this  

way, we have the impression of a whole, of a circle of light. 

   We know that seeing arises at this moment, but we cannot pinpoint the citta  

which sees, it arises and falls away very rapidly and another moment of seeing  

arises. We only experience the “sign” of seeing. 

   The  notion  of  nimitta  can  remind  us  that  not just   one  moment  of  seeing  

appears,  but  many  moments  that  are  arising  and  falling  away.  Also   visible  

object is not as solid as we would think, there are many moments arising and  

falling away which leave the sign or impression of visible object. 

   Visible  object  that  was   experienced  by  cittas  of  a  sense-door  process  has  

fallen away; sense-door processes and mind-door processes of cittas alternate  

very rapidly. Visible object impinges  again  and  again and  seeing  arises  again  

and  again. When  their  characteristics  appear we  cannot  count  the  different  

units of rūpa or the cittas that see, they arise and fall away; the impression of  

what is seen and of the seeing appears. 

   Acharn   Sujin   said:   “No   matter   whether   we   call   it   nimitta   or   not,   it   is  

appearing now. Whatever  appears  is the  sign or nimitta of the dhamma that  

arises and falls away.” 

   We cling to what appears for a very short moment, but is does not remain.  

It is the  same with  saññā, there is not one moment of  saññā that marks  and  

remembers, but countless moments, arising and falling away. 

   Thus, we can speak of the nimitta of each of the five khandhas: of rūpa, of  

feeling, of  saññā, of  saṅkhārakkhandha, of  consciousness. There  are nimittas  

of  all  conditioned  dhammas  that  appear  at  this  moment,  arising  and  falling  

away extremely rapidly. 

   Seeing  arising  at  this  moment  sees visible  object. We  notice visible  object  

and while we notice it, we have a vivid impression of it, but it has just  fallen  

away. Seeing falls away but extremely shortly after it has fallen away another  

moment  of  seeing  arises  that  experiences visible   object.  It  arises  again  and  

again and in between one notices that there is seeing, or, if there are the right  

conditions   a   citta   with   sati   can   arise   that   is   mindful   of   its   characteristic.  

However, mindfulness of seeing arises after seeing has fallen away, not at the  

same time as seeing. 

   People are wondering how one can be mindful of anger, dosa. Mindfulness  


----------------------- Page 24-----------------------

22 ● Alone with Dhamma 



that  accompanies  kusala  citta  cannot  arise  at  the  same  time  as  anger  that  

accompanies akusala citta. 

   Anger,   dosa,  arises  and  falls  away  and  other  processes  of  citta with  dosa  

arise  again.  In between,  sati  can be  aware  of  its  characteristic, just   as in the  

case   of   seeing   that   has   fallen   away.   Awareness   of   dosa   is   different   from  

thinking  of:  “I have  dosa, there  is  dosa.” But  also  such thinking  can  arise  in  

between. Understanding can develop in considering the characteristics of the  

dhammas that appear and there is no need to think: “It has fallen away”, or,  

“This is remembrance”. 

   We  have  to  face  akusala  with   courage  and  sincerity,  otherwise  we  shall  

always  cling  to  an  idea  of  “my  dosa”,  or  “my  lobha”. Wrong view has  to be  

eradicated first, before lobha and dosa can be eradicated. 

   I quote what Acharn Sujin said at an earlier occasion: “When akusala arises  

it can remind us of the truth about our accumulations and this is the way to  

develop   paññā.   Ignorance   conditions   more   akusala   and   paññā   conditions  

kusala.”   People   may   be   distressed   when   they   notice   akusala,   but   at   the  

moment of understanding the citta is kusala, it is free of disturbance. 

   Some  people   may   believe   that   they   have   to   apply   energy   and   perform  

specific  actions  so  that  they  have  less  akusala  cittas  and  more  moments  of  

mindfulness.   Listening   and   considering   are   conditions   for   the   arising   of  

insight. But there are other conditions stemming from the past: kusala in the  

past   conditions   our   interest   at   this   moment   to   consider   and   investigate  

realities.  This  process has been  set  in  motion  already,  there  is  not  a  person  

who could regulate this. 

   Nobody  can  create  conditions  for  the  arising  of  sati.  Acharn   Sujin  asked  

several times:  “Can you  create hardness  now?” Nobody  can  create  anything,  

because dhammas arise because of their own conditions. 

   Hardness is the rūpa that is solidity or the Element of Earth. This rūpa arises  

and falls away all the time in split-seconds. If  someone says: create hardness  

now, it is impossible, it has already arisen and fallen away and then there is a  

new   hardness  in  another  group  of  rūpas.  It  is  present  with  every  group  of  

rūpas, it supports other rūpas in that group. It arises with sound, with visible  

object, with  any  other  type  of  rūpa. All  of  them  arise because  there  are  the  

right conditions. It may seem that one can create sound, but without the right  

conditions  it  is  impossible.  It  is  the   same  with   sati  and  paññā,  which   are  

sobhana cetasikas. Nobody  can create them. We  are not  a creator, master or  

owner of any dhamma. 



When we were  in  India, we were  sometimes  sick, we had  a  fever  or violent  

pains due to water, food or climate. Acharn  Sujin reminded us then to know  

the   characteristic   of   the   present   reality,   for   when   there   is   thinking   about  

tonight or tomorrow or worrying about it obviously there is too much interest  


----------------------- Page 25-----------------------

                                                               The Present Moment ● 23 



in ‘self’.  She also felt sick, mostly from a severe cold, and exhausted at times  

but she did not show it. She never thinks of herself. 

   When we read the many suttas about dhammas appearing through the  six  

doors we can be reminded of the truth. The Buddha taught all the time about  

the dhammas  appearing through the  six doors. There is only the dhamma of  

this moment, nothing  else, this is the truth. Nobody  can  cause the  arising of  

specific  dhammas.  When  feeling  sick,  that  is  the  dhamma  at  this  moment.  

There  are only dhammas, not me who feels sick. We cling to our feeling, we  

are  commiserating  with   ourselves,  and  also  that   is  a  dhamma.  We   cannot  

escape nāma and rūpa, so long as we are living in this world. 

   Acharn  Sujin said that there can be understanding of the dhamma that has  

already  arisen because  of  conditions.  If we understand  dhamma  as  dhamma  

we   know   that   nobody   can   interfere   with   what   arises   because   of   the  

appropriate conditions. We may have intellectual understanding of anattā, but  

we  should  come to understand the reality that  is  anattā.  She  said:  “There  is  

always  an  idea  of  I  who   is  reflecting,  but   actually,  citta  and  cetasikas  are  

performing their functions and then they fall away immediately.” 

   Acharn Sujin reminded us that it is not sufficient to think that everything is  

anattā. Precisely  at this very  moment we  must try to understand  anattā. We  

are  alone,  only  dhammas  appear  one  at  a  time  through  the  senses  and  the  

mind-door. We think time and again of people we love, we find our thoughts  

about them very important. But what we take for people are dhammas arising  

and falling away immediately. What has fallen away never returns. 

   We have to develop paññā at this moment so that we shall understand the  

truth  of  anattā.  There  is  seeing  at  this  moment  and very  gradually  we  can  

learn that it is a dhamma that sees. 

   Jonothan   remarked   that   one   should   be   honest   with   regard   to   one’s  

defilements. Truthfulness, sacca, is one of the perfections that the Bodhisatta  

developed during countless lives. Lodewijk said that he found the perfection of  

Truthfulness  essential,  but   very   difficult  to  develop.  When  he  reads  in  the  

Sutta that the Buddha said, ‘Abandon evil... it can be done...’ he feels that he is  

insincere, since he clings to all the pleasant things of life and does not want to  

give them up. He said that before one realizes it, one is misleading oneself as  

to kusala and akusala. 

   Jonothan   said   that   there   are   moments   of   understanding   dhammas,   and  

moments of ignorance. When there is awareness and right understanding of a  

dhamma one is on the Path the Buddha taught. Actually, being on the Path is  

momentary,  and when understanding does not  arise, which happens most of  

the time, the Path is not developed. However, he said, this is not a cause for  

concern. Even when the moments of developing the Path are very few, it is a  

great gain hearing the right Dhamma and listening to it with sincere interest.  

There   can   be   moments   of   reflecting,   considering,   moments   of  kusala   and  


----------------------- Page 26-----------------------

24 ● Alone with Dhamma 



paññā. 

   We  may  reflect  on  the  Path  leading  to  enlightenment,  but  Acharn   Sujin  

always says: “But what about this moment?” 

   I find it very helpful to remember that citta and the Path-factors which are  

cetasikas  arise  for  a  moment  and  then  fall  away. This  is  in  conformity with  

real life. 

   Lodewijk said that it seemed to him that he had more understanding when  

in   India,   visiting   the   holy   places.   But   now   it   seems   that   the   little  

understanding he had is lost when he is back in the routine of life in Holland. I  

reminded him of what Jonothan had said to him in India about the Path being  

momentary.  It  is  good  to be  reminded  that  the  moments  of  sati  and  paññā  

arise  and  fall  away  and  that we  cannot  keep  them.  The  understanding  that  

arises is never lost, it is accumulated and it can grow. 

   Acharn Sujin reminded us: “The Buddha explained what is kusala and what  

akusala, but can he force anyone? He could show the way leading to the end  

of defilements. Someone may like to have sati, but who can have it if there are  

no  conditions  for  it?  One  may want the  dhammas that  arise to be  different,  

but  instead  of  such  clinging  there  should be  detachment.  One  may  try very  

hard to make sati arise, but if there is no understanding of it as a conditioned  

reality that arises and falls away it is useless. 

   And   on  the   other  hand,   when   there   are  conditions   for  sati  nobody  can  

prevent its arising. This shows us that there is no need to try to have sati. 

   The entire Tipiṭaka explains the nature of anattā of realities. The arising of  

direct awareness and understanding depends on the right conditions and they  

cannot arise even if one is told a hundred times.” 

   When  we   were   in   India  we   had   many   opportunities  to   rejoice  in  other  

people’s   kusala,   their   generosity   and   readiness   to   help   others.   Jonothan  

performed kusala all day long, without interruption, in recording the Dhamma  

discussions. He applied a great deal of effort to hold the microphone close to  

Acharn Sujin. 

   When   Lodewijk   said  that   he   was   distracted   by   this,   I   said   to   him   that  

Jonathan develops all the perfections while doing this. He performs dāna, he  

gives the great gift of Dhamma. He performs  sīla while helping many people  

by bodily acts, while holding the microphone so that people all over the world  

can   hear   our   dhamma   discussions.   He   develops   renunciation   since   he  

renounces his own comfort while he has little time for relaxation. He develops  

wisdom   while   listening   with   understanding,   and   asking  questions   that   are  

useful  to  all.  He  applies  energy  since  he  is  not  inert  to  perform wholesome  

deeds. He has patience and endurance which is needed day after day in order  

to continue to perform kusala. He develops truthfulness to perform the kusala  

he has determined to do: acting according to what he has promised he would  

do.  He  has  determination   to  perform  kusala  and  continue  with   it.  He  has  


----------------------- Page 27-----------------------

                                                                    The Present Moment ● 25 



mettā   because   he   thinks   of   many   people’s   benefit   and   welfare.   He   has  

equanimity because even when he is tired he continues with equanimity in all  

circumstances. 

   This shows that  all the perfections can be developed together,  at the  same  

time. We do not have to think about the perfections; they are developed while  

we  perform  kusala  through body,  speech  and  mind without  thinking  of  our  

own gain or profit. Jonothan and Sarah are always working hard to edit all the  

recordings of Dhamma discussions just  for our benefit. They even listened to  

them and edited them during our long bus drives.  

   At the beginning  of  this  pilgrimage  Lodewijk  stated  categorically  that  this  

would   be   his   last   trip   to   India.   At   the   end   of   the   trip,   however,   on   the  

houseboat in Srinagar, he said just as categorically that he would definitely go  

on   the   next   pilgrimage.   Why?   The   holy   places,   the   discussions   and   the  

recollection of the Buddha had inspired him to go back to India in order better  

to be able to understand the Truth, wherever we go and live. And so it should  

be for all of us. 
