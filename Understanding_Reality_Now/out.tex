\part{Once Upon a Time}



\chapter{Preface}

In January 2014, a year after my last journey, Acharn\footnote{Acharn is the Thai word
for teacher. In Pali: ācariya.} Sujin and Sarah had
organised another sojourn in Thailand for our Vietnamese friends,
including a girl of five years old, and other Dhamma friends from
different countries: from Canada, Australia, from the U.S., from Japan
and from Italy. In the Hague I had had an accident in the tramway and
broken my hip. I was happy to be able to make the journey after several
months of hard training with my therapist. A day trip was planned to
Bangsai, near Ayuthaya, and shortly after that we would go to Kaeng
Krachan, the place where Acharn Sujin and Khun
\footnote{Khun is the Thai word
for Mr. or Mrs.} Duangduen regularly stay
and then to Suanpheung in the mountains outside Ratchbury. At the end of
my stay I went to Chiengmai with my Thai friends.

Our stay in Thailand coincided with political unrest. Those who opposed
the government and prime minister Yingluck whose brother Thaksin was
ousted, organised demonstrations and blocked roads. However, we could
make all our planned trips inspite of the political situation.

I received great assistance from all my friends, whenever there was a
difficult high step to be taken, or when I had to walk in a dark garden.
Khun Noppadom, one of our Thai friends, was asked to look after me. At
breakfast in Kaeng Krachan he fetched the food for me all the time and
also later on in Chiengmai he saw to it that I would not go hungry. In
Suanpheung several Thai friends provided us with an abundance of fruits
and sweets whenever we had a break in between the discussions.

In Bangkok I stayed again in Hotel Peninsula where Sarah and Jonothan
often stay. I was next door to them which gave me a safe feeling. I
listened four times a day to Acharn's radio programs in Thai and heard
that one should not be impatient in the development of understanding.
One should not have expectations as to its development since we have
accumulated such an amount of ignorance. If one is discouraged it shows
that one clings to the idea of self. Understanding should be developed
with courage and cheerfulness.

Acharn asked me to write a summary of our discussions and she even
suggested a title: ``Once upon a time''. I am very grateful that Acharn
explained with great patience that the characteristic of seeing and
visible object appearing at this moment should be investigated. If we do
not know what seeing is, only a dhamma, a conditioned reality, we shall
continue to cling to a self. Everything is dhamma and ``there is no one
there'' she repeated many times. We cannot hear this often enough.

\chapter[Once upon a Time]{}
\section*{Once upon a Time}


``Once upon a time\ldots{}'' Stories of the past begin with these words.
We do not really know the past. We do not even know what we did and
thought yesterday from moment to moment, it is all gone. We do not know
who we were in a past life, it is forgotten. We were happy and unhappy
but all those experiences are completely gone, never to return. Also in
this life it is true that all we find so important is gone immediately.
This life will be the past life in the next life.

In reality the past can be as recent as one moment of citta
(consciousness). What we call mind is citta that falls away immediately.
There are different types of citta and each citta experiences an object:
seeing is a citta that knows visible object and hearing is another citta
that knows sound. Citta arises, experiences an object and then falls
away immediately, never to return. Each citta is accompanied by mental
factors, cetasikas, that assist the citta in cognizing the object. The
mental factor remembrance or saññā accompanies every citta and it marks
and remembers the object that is experienced. That is why we recognize a
chair and know that it is for sitting, or we recognize a person who is
in the room. Seeing only sees what is visible, it does not see people
and things. After seeing has fallen away, there can be thinking of
people and things which are remembered by saññā. In reality there is no
one there. 

During our discussions, Acharn repeated many times: ``There is no one
there''. She said: ``Dhamma means: no one there. It is just a
characteristic of reality that appears''. I am grateful for this
reminder, because we are deluded most of the time and we believe that
people exist. What we take for a person is in reality only citta
(consciousness), cetasika (mental factor accompanying consciousness) and
rūpa (physical phenomena) which arise for an extremely short moment and
then fall away. When we see, we are immediately attached to seeing and
visible object but before we realize it they are completely gone. They
are past already, they were present ``once upon a time''. We may think
of a dear person who passed away, but there is only the idea or memory
of what is gone completely. Only attachment and ignorance are left,
Acharn said. This helps us to begin to understand, at least
intellectually, the disadvantage and uselessness of clinging to persons.

We believe that we live with many people, but when we consider the
different cittas that arise one at a time and experience different
objects through the senses or the mind-door, we can understand what
``living alone'' means. Life is only the experience of one object at a
time such as visible object or sound. When visible object is
experienced, there is the world of visible object and when sound is
experienced there is the world of sound. Different worlds appear through
the senses and the mind-door. They could not appear if there were no
citta which experiences them.

Sarah said that this is an encouragement to wake up from our dreams.
Understanding of the reality appearing now is the only way to lessen
attachment to whatever appears. She also said that we usually live in
``once upon a time'' stories, but, that just for a moment now, there can
be truly living alone with the world that appears. When we appreciate
this, we begin to have a sense of urgency, with understanding.

It takes an extremely long time before the truth can be realized. It is
realized by paññā, a mental factor that is understanding. This is
developed stage by stage, during countless lives. Intellectual
understanding of the Buddha's teachings is a foundation for the
development of direct understanding. But if we wish for direct
understanding we are on the wrong Path. There is clinging instead of
understanding. Ignorance and clinging have been accumulated for aeons. 

We are heedless and we need many reminders of the truth. Our life is
very short and therefore, we should not waste opportunities to listen
and consider the Dhamma. Actually, life is as short as one moment of
citta. Each moment of seeing or hearing is one moment in the cycle of
birth and death. Seeing is only once in a life time and then it falls
away. Hearing is only once in a life time and then it falls away. Life
is only one moment of citta experiencing an object.

One of the first days of our stay we went to Bangsai. Bangsai is near
Ayudhaya. Here Khun Duangduen has a peaceful place, surrounded by fields
and near a temple. In the background we could hear the monks preaching,
because it was Uposatha day
\footnote{Special day of
vigilance.}.

Acharn asked: ``Do you know me? What you see is only visible object, and
you do not know visible object yet. It is very difficult to eliminate
the idea of self and it can only be achieved by paññā. Is there visible
object or are there people around here? It takes a long time to develop
the understanding of not me, not anyone, no self, no thing in it. Seeing
sees only visible object. It has to be right now, it should be very
natural. Understanding begins to develop, there is no `I' who tries. It
is a very long way but one can begin to see that the Buddha knew through
his enlightenment whatever reality appears. The development of right
understanding has to be the understanding of whatever appears now.''

She wanted to remind us that we see only visible object, not a person.
She said: ``Me or visible object, exactly the same. But you don't know
visible object. Understanding has to be developed until there is no one
at all, no thing at all in that which can just impinge on the eyesense.
Citta arises to see it and then falls away. Visible object cannot be
anyone. What is left is only the sign (nimitta) of reality, no matter
there is seeing, hearing, thinking. Even intellectual understanding is
not easy. Whatever arises is like a flash. Attachment cannot be known by
a self, only right understanding can know it.''

It is true that persons cannot impinge on the eyesense, only visible
object or colour can impinge on it so that it can be seen.

Someone asked whether the ``Element of Wind'' or motion can be
experienced through the bodysense. This is a kind of rūpa (physical
phenomenon) that can be experienced as motion or pressure.

Acharn Sujin answered: ``You like to experience it, and there is not the
understanding of it when it appears. That is the point. Attachment or
craving is the second noble Truth
\footnote{The Buddha taught four
noble Truths: the Truth of suffering, dukkha. The Truth of the cause of
suffering which is craving or attachment. The Truth of the cessation of
suffering which is nibbāna. The Truth of the Path leading to the
cessation of suffering which is the eightfold Path, the development of
right understanding of realities.}. If this is not gradually
eliminated, it hinders. Someone may try very hard to make the Element of
Wind or motion appear. Right now, many realities have passed without
there being understanding of them, including motion, heat or anything.
It is not under anyone's control to let it appear. Mindfulness, sati,
can be aware of it\footnote{Sati, mindfulness or
awareness, is non-forgetful of the realities that appear. Usually there
is forgetfulness, we are absorbed in thinking of ``stories''. When
kusala citta with sati arises there can be mindfulness of one reality at
a time as it appears through one of the senses or the mind-door.}. Sati
is very rapid. Before we can think about it, it is already aware. There
is no need to think that one would like to know a particular reality. It
is time to accumulate understanding so that there are conditions for
having less attachment to experiencing particular realities. Would you
like to have satipaṭṭhāna\footnote{Satipaṭṭhāna is the
development of right understanding of mental phenomena and physical
phenomena.}
right now?''

Nina: ``I would like to.''

Acharn: ``That is already wrong practice, sīlabbata parāmāsa, clinging
to rites and rituals.''

Nina: ``Already? That is very heavy.''

Sarah: ``It is very common.''

Acharn: ``Anything which does not lead to the understanding of reality
is sīlabbata parāmāsa.''

Nina: ``That is so strong. It had not thought of that. Such a strong
word.''

Acharn: ``Only paññā can see reality as it is. Otherwise there is no
understanding of anything.''

Understanding of realities is developed by listening to the Dhamma and
carefully considering it. When intellectual understanding has been
developed sufficiently, there are conditions for direct understanding of
realities. When the mental factor sati is aware of a characteristic of
reality, understanding, paññā, can know its true nature. Paññā is
another cetasika that may accompany kusala citta (wholesome citta).
Cittas may be akusala (unwholesome), kusala, vipākacitta (result of
kamma) or kiriyacitta, inoperative citta\footnote{Kiriyacitta performs
different functions within a process. The arahat has no more kusala
cittas but he has kiriyacittas instead.}. Nobody can make a
particular citta arise, they arise because of their own conditions. 

Realities have each their own characteristic that can be directly
experienced. Concepts are not realities, they can only be objects of
thought, they do not have characteristics that can be directly
experienced. The truth of non-self pertains to realities. Person or
chair do not have the characteristic of anattā.

It is important to know the difference between realities and concepts.
Seeing and visible object are realities. Seeing sees what is visible,
what has impinged on the eyesense. There is no person who sees, only
seeing sees. Dhammas that appear one at a time through one of the senses
or the mind-door are ultimate realities or paramattha dhammas
\footnote{Paramattha means the
highest sense. In Pali ``parama'' is highest and ``attha'' is meaning or
sense. Paramattha dhammas are: citta, cetasika, rūpa and nibbāna.}. Ultimate realities have
each their own unalterable characteristic. We may call them by another
name but their characteristics cannot be altered. Seeing is always
seeing, no matter how we call it. Persons, trees, chairs are not
ultimate realities, they are concepts formed up by thinking.

We dream of persons, mountains or trees. These are all stories we think
of. When we see someone in our dreams it is not really seeing, but
thinking of what is remembered, of what we saw before. It seems so real,
it seems that we really see. When we believe that we see a person now,
while we are awake, it is exactly the same; this is not seeing of what
is visible, it is only thinking.

We read in the ``Middle Length Sayings'', ``Potaliyasutta'' (I, 365)
that the Buddha used different similes for sense pleasures. The text
states:

\begin{quote}
``And, householder, it is as if a man might see in a dream delightful
parks, delightful woods, delightful stretches of level ground and
delightful lakes; but on waking up could see nothing. Even so,
householder, an ariyan disciple reflects thus: `Pleasures of the senses
have been likened by the Lord to a dream, of much pain, of much
tribulation, wherein is more peril.' And having seen this thus as it
really is by means of perfect wisdom\ldots{} the material things of the
world are stopped entirely.''
\end{quote}



It is necessary to consider why we want to study the Dhamma. We study to
have more understanding of what is real, to have more understanding of
the fact that there is nobody in what is seen or heard, nobody who sees
or hears. We have accumulated so much ignorance and wrong view for aeons
and aeons and, thus, we cannot expect to get rid of these soon. Only
paññā, wisdom or understanding, can eradicate ignorance, but it develops
only very little at a time. If we think that we can control or
manipulate understanding or make it grow, there will only be more
attachment and wrong view. Thus, there should not be any expectations as
to the growth of paññā, it develops according to its own conditions. It
does not belong to us.

Acharn said: ``We talk very often about visible object and seeing.
Otherwise we are always forgetful of realities. We think of a collection
of several realities as `something'. When sati is aware of a reality it
is time to know that all the stories we think of are useless. They are
only the object of thinking. Without thinking there is no situation.
Ignorance conditions attachment.''

During our discussions Acharn emphasized very much the uselessness of
experiencing objects. They are gone immediately, but we are clinging to
objects, life after life. What is the use of clinging to what falls away
immediately? Acharn wanted to remind us that life is dukkha (suffering),
not worth clinging to. But just now we do not see the danger and
disadvantage of all our experiences in life. Only paññā that sees
realities as they are can realize this. Paññā can condition detachment.
There can be a letting go, even of paññā, not wanting it again and
again. There should be no selection of the objects of awareness and
understanding.

Acharn said: ``The characteristic of hardness appears as `no one'. Paññā
can see that this is part of the cycle of birth and death (saṃsāra). The
cycle is the succession of the arising and falling away of realities.
There is no one there.''

The English discussions in Bangkok took place in the ``Dhamma Study and
Support Foundation''\footnote{This is the center where
all sessions with Acharn Sujin take place each weekend.}. On
Sunday I attended Thai sessions the whole day. For luncheon we walked to
a restaurant nearby. The widow of Khun Denpong sponsored one of these
luncheons. Khun Denpong passed away three years ago and before he died
he said to Acharn Sujin: ``I would like to live just somewhat longer in
order to develop more understanding.'' I have known him as someone who
always had many good questions. His widow said that he was a wise man.

After luncheon Elle helped me to take the difficult, steep steps down
from this restaurant on the way back to the Foundation. We talked about
the deaths of our husbands and spoke about it how sudden death comes.
There is no time to take leave of our dear ones. We were dwelling on
stories of the past, ``once upon a time''. This is thinking and Acharn's
words always bring us back to reality now.

I remember what Acharn once said to Khun Weera when his wife, Khun Bong,
was about to die:

``Dukkha is heavy, nobody likes it. It is a danger, it causes citta to
be sorrowful, troubled. Nobody is freed from it, but we must understand
it. When we have more understanding of the Dhamma we shall see that what
arises must fall away, this cannot be altered.

Birth is really troublesome. We have to eat to stay alive, we have to
see, there is no end to seeing. Seeing is a burden, because of seeing
there is attachment. Is seeing beneficial or is it a danger and
disadvantage? When there is seeing, there will be clinging to what is
seen. We are searching for the things we like, but if we do not search
for what we like we live more at ease. From where comes the burden? From
seeing and from wanting the things we see. We can come to understand
that each citta that arises and falls away is a burden. Everything that
arises and falls away is great dukkha. Defilements cannot be eradicated
by ignorance, only by understanding. When we listen more and develop
understanding more there will be less dukkha. Everyone has to die, this
cannot be changed. What arises now has to fall away, and then there is
nothing left. When a dhamma arises and there is ignorance, one clings
and takes it for `self' or `mine'.''



\chapter[Ignorance]{}
\section*{Ignorance}


Not knowing conditioned realities which arise and fall away is
ignorance. Ignorance, in Pali avijjā or moha, is an akusala cetasika
(unwholesome mental factor) that accompanies each akusala citta. It is
the root of all that is akusala. We read in the following text
(``Sammadiṭṭhi Sutta: The Discourse on Right View'', MN
9)\footnote{Translated from the
Pali by Ñanamoli Thera and Bhikkhu Bodhi. Access to Insight, Legacy
Edition, 30 November 2013.} that ignorance is not
understanding the four noble Truths. We read :

\begin{quote}
``Not knowing about suffering, not knowing about the origin of
suffering, not knowing about the cessation of suffering, not knowing
about the way leading to the cessation of suffering --- this is called
ignorance.''
\end{quote}

We have to apply this text to the present moment. The first noble Truth,
suffering, dukkha, is the arising and falling away of reality now.
Seeing now falls away and it never comes back. All our experiences fall
away never to return. What is impermanent is not worth clinging to,
clinging only brings sorrow.

The second noble Truth, the origin of suffering, that is craving or
attachment. So long as we have attachment there are conditions for
realities to arise again and again in new births. Also now we have
attachment, attachment to all sense objects, and we often have subtle
attachment we do not notice. Whatever we do, whatever we say, whatever
we are thinking, the idea of self is there. During our discussions
Acharn reminded us time and again of this fact. Even when we engage in
kusala, we do this often for the sake of ourselves.

The third noble Truth, the cessation of suffering, is nibbāna\footnote{Nibbāna is the
unconditioned dhamma, it does not arise and fall away. It is experienced
by lokuttara citta, supramundane citta, when enlightenment is attained
and defilements are eradicated. There are four stages of enlightenment
and at each stage defilements are eradicated until they are all
eradicated at the attainment of arahatship.}. We cannot imagine what
it is like but Acharn said that no arising and falling away is to be
preferred to arising and falling away, which is the dukkha of life.

The fourth noble Truth, the way leading to cessation, this is the
eightfold Path, the development of right understanding of realities\footnote{The eightfold Path
consists of sobhana cetasikas, beautiful cetasikas, of which the
foremost is right view or paññā. The factors of the eightfold Path have
to develop on and on so that enlightenment can be attained.}. Only paññā, right
understanding, can eliminate ignorance. Right understanding can be
developed by listening to the Dhamma and carefully considering it. Even
one moment of understanding can condition the arising of understanding
again later on. Understanding, a cetasika that accompanies kusala citta
(wholesome citta), falls away together with the citta, but it is not
lost. Cittas arise and fall away in succession, and, thus, understanding
is accumulated in the citta from moment to moment so that there are
conditions for its arising again. We have accumulated ignorance and
wrong understanding for aeons and therefore, they cannot be eliminated
immediately. Courage and patience are needed to continue to listen and
develop more understanding. That is the reason why Acharn explained time
and again about seeing and visible object and all realities of daily
life.

We listen to the Dhamma in order to have more understanding of
realities. A beginning can be made now: seeing appears now and what is
the nature of seeing? Seeing only sees what is visible, seeing is not a
person, no ``I'' who sees. Visible object is a type of rūpa, a physical
phenomenon, and it can impinge on the eyesense which is another rūpa.
Visible object and eyesense are rūpas, they do not know anything. They
are conditions for seeing. Seeing is a type of nāma, a mental
phenomenon, a citta that experiences visible object. Cittas arise and
fall away in succession very rapidly. It seems that we see immediately
the shape and form of persons and things, but in reality there are many
different cittas arising and falling away.

It seems that there is one moment of seeing and perceiving people and
things all at the same time, but in reality there are many different
moments. Seeing arises in a process of several cittas that experience
visible object. When that process is over, there is another process of
cittas experiencing visible object through the mind-door. Later on other
processes of cittas arise that think of shape and form and take this for
a person or thing. It seems that there is a long period of seeing people
and things, but in reality there are many different cittas succeeding
one another.

Seeing does not think, it only sees, but when it has fallen away we
think of long stories, forgetting that thinking of stories is
conditioned by seeing, hearing and the other sense-cognitions. Acharn
said: ``After once upon a time, then what? Right now there is past all
the time. Even now, it is once upon a time.''

Sarah remarked: ``Not only when we are asleep, but even now we are
always dreaming, building up stories with worry about how to take steps
in the dark. Always stories like `once upon a time', continuing the
story again and again.''

Citta knows an object, each citta knows or experiences a particular
object and the cetasikas that accompany it also experience that object,
but while they are doing so, they each have their own function or task
while they assist the citta. Citta is the leader in knowing the object
and the cetasikas are the assistants of citta. When we read about
cetasikas, we should not get lost in names or terms. It is not the name
that is important, but the characteristic of cetasika that can be
gradually understood. Studying them helps us to see that citta is
conditioned by the cetasikas that accompany it. Citta cannot arise
without cetasikas and cetasikas cannot arise without citta. The Buddha
taught conditions for the dhammas that arise in order to make it clear
that they do not belong to us, that they are not ``self'' or ``mine''.

Feeling is a cetasika that accompanies each citta, and we find feeling
so important. We cling to it all day long. Feeling may be happy, unhappy
or indifferent. It is only a conditioned dhamma. Attachment (lobha) and
aversion or anger (dosa) are unwholesome cetasikas (akusala cetasikas).
We do not have to name them in order to come to understand their
characteristics when they appear. When they appear now, at the present
moment, their different characteristics can be known very gradually.
When there are conditions they arise and nobody can prevent their
arising. They can be understood as anattā.

We have to know the difference between intellectual understanding of a
reality such as seeing, and the actual, direct understanding of seeing
when it sees, just now. That is understanding without words. We usually
pay attention only to that which is known, seen or heard, and we forget
that without citta there would not be anything that appears, no world.
Visible object could not appear if there were no seeing, sound could not
appear if there were no hearing. There can be less attachment to citta
that experiences and to that which is known by citta. But we should not
have any expectations. Understanding cannot arise by wishing or wanting.
We can come to know that all the time the idea of self comes in that
wishes to know, wishes to observe, and this works counteractive. When
there is more understanding of realities it leads to detachment from the
idea of self who wants to do something, who is trying to know.

We had planned to go to Kaeng Krachan outside Bangkok and this was on
the first day that Bangkok would be ``shut down'' by those who opposed
prime minister Yingluck and the government. Streets would be barricaded.
The day before our departure was a Sunday and the Foundation was closed
so that people could prepare for the ``shut down''. This happened to be
Acharn's birthday, of which she said that she would rather be without
it. However, now people still came with presents on Saturday. We had an
opportunity, with Betty's help, to give her presents and appreciate
other people's generosity. They smiled and kept on telling her how much
they appreciated her teaching. Some people presented her with huge
vegetables. In no time the whole room was packed with presents.

We could go on our journey as planned and we stayed four nights in Kaeng
Krachan. We stayed in bungalows situated in a large park with flowering
trees. Early morning we walked from the bungalow where we stayed through
the park to the restaurant for breakfast, outside along a lake. For the
discussions we were sitting in the garden at the place where Acharn and
Khun Duangduen stayed. The subject of our discussions was mental
phenomena and physical phenomena, the many defilements that arise and
kamma that brings result. A good deed, kusala kamma or a bad deed,
akusala kamma, can produce result later on, even after many lives. The
kusala citta or akusala citta that motivates a deed falls away but
kusala and akusala are accumulated from one citta to the next citta,
from life to life. When it is the right time kamma produces result,
vipākacitta, in the form of rebirth-consciousness or the
sense-cognitions arising throughout life, such as seeing, hearing,
smelling, tasting or body-consciousness experiencing bodily ease or
pain. I mentioned that one never knows when kamma will produce result.
My accident, when I broke my hip, was completely unexpected; I never
thought that it would happen. Acharn reminded me that we always think of
people and situations, but that in reality there are eyesense, seeing,
earsense, hearing, conditioned dhammas that are only there for an
extremely short time. She said: ``There is the flux of the elements that
arise and fall away, uncontrollable. We should understand them as not
`me'. There is no one there. We do not have precise understanding of a
reality that is seen but we keep on thinking in terms of people and
situations. The conditions are not sufficient to make us understand what
appears now.''

She always referred to the present moment since that is the moment a
dhamma appearing through the senses or the mind-door can be investigated
and understood. They appear one at a time and they each have their own
characteristic. When we think about situations, the reality is thinking;
it is usually akusala citta that thinks, and the situation is not a
reality. Acharn reminded us to develop understanding of this moment,
just one moment in the cycle of birth and death.

We were discussing realities, and we can also use the word dhammas or
paramattha dhammas. For example, sound is a reality, it can be directly
experienced when it appears. We do not have to name it sound, but its
characteristic can be directly experienced. Thinking about sound is not
the same as the direct experience of it. We can learn that its
characteristic cannot be changed into something else. Sound is always
sound, it is the object of hearing. Attachment is always attachment, no
matter how we call it.

We think of concepts most of the time. It seems that we hear dogs
barking, words spoken, that we see persons in the room, mountains or
trees. But the difference between concepts and ultimate realities,
paramattha dhammas, should be known, at least on the level of
intellectual understanding. This can lead to direct understanding. Then
we shall know that there is no one there, no person. We shall know that
realities are anattā. At this moment anattā is just a word we repeat.
But the truth of anattā has to be directly realized.

When we are thinking about realities they have fallen away already. We
all try very hard to find out the truth about realities, reasoning about
them. That is not the way. What about now, while we ask questions about
something or have doubts? Acharn said:

``At the moment of not understanding, what is there? Usually we think
without understanding, so it is like a dream. At this moment, what is
real? Now, when there is not direct awareness and understanding, it is a
dream. Even when talking about paramattha dhammas the object is a
concept of paramattha dhammas, they do not appear. When there is direct
understanding, you are not thinking of that subject.''

Also when we ask questions she reminded us to consider the citta that
does so. Instead of wondering or having doubts shouldn't we attend to
the present moment, such as seeing right now? We should know what type
of citta motivates our questions. Often it is akusala citta.

We had Dhamma discussions in the morning and later in the afternoon,
even after dark. In between we went out for luncheons in different
places where we had panoramic views of a lake and mountains or we sat
along the waterside. When the steps to reach the place were too deep for
me I always had support from my friends. Acharn, her sister Khun Jeed
and Khun Duangduen offered us a luncheon on the first day and for the
other days we took turns in sponsoring them. Even during luncheon Acharn
untiringly explained about mental phenomena, nāma, and physical
phenomena, rūpa. We were asking about the characteristic of sati,
mindfulness. This is a sobhana cetasika, beautiful cetasika, that can
only arise when there are conditions. Nobody can cause its arising. We
touch many times during the day different things and body-consciousness
experiences hardness, but there is no mindfulness of a characteristic of
a reality. Body-consciousness is not accompanied by sati, it is
vipākacitta that merely experiences tangible object. When sati arises it
is mindful of the characteristic of tangible object without thinking of
the hardness of ``my body'' and at the same time paññā, understanding,
which is another sobhana cetasika, can investigate that characteristic
so that it is known as just a dhamma, not belonging to a self.

Listening and discussing are conditions for awareness but we should not
be wishing or wanting to have it.

Acharn explained: ``When one is touching and hardness appears it is
different from thinking about what is touched. When a characteristic of
a reality appears it is not as usual because there is direct
awareness\footnote{Sati can be translated
as mindfulness or awareness.} of it. You do
not have to name it and you do not expect to have it. Understanding
knows the difference between the moments of sati and the moments there
is no sati. This is the only starting point for the development of
awareness. Paññā knows when there is attention with awareness to the
characteristic that appears. Attachment or aversion may arise when one
does not have awareness as much as one would like to. Sati is only a
reality, a dhamma, not different from other realities.''

When there is awareness of hardness which is a kind of rūpa, there is
not some ``thing'' in the hardness such as a hand or the table. Only
hardness appears and nothing else. It seems that seeing and hearing can
arise at the same time, but when awareness arises one knows that
realities appear one at a time. Seeing experiences visible object and
hearing experiences sound, these cittas cannot experience more than one
object. At the moment of awareness just one reality appears at a time
and there is nothing else, no world. When this is not realized one knows
that understanding has not been accumulated sufficiently. We need to
listen again and consider again and again. Since ignorance is deeply
rooted we cannot expect that paññā develops rapidly. In each life very
little understanding is being accumulated, Acharn said. Now and then
just a glimpse of understanding arises. When we have an interest in the
Dhamma now and listen to the Dhamma there are conditions for listening
again in a future life. In this way paññā develops gradually from life
to life.

I had a conversation with Acharn about awareness:

Acharn: ``Is there anyone in visible object which is seen? This is the
beginning of seeing the world as it is. Otherwise one is born and dies
without any understanding of reality.''

Nina: ``I have regret when there is no awareness''.

Acharn: ``One can see clinging, it is always there. Only paññā can lead
to detachment.''

Nina: ``When I ask `how can I develop paññā\ldots{} how can I have more
detachment\ldots{}', I know that this indicates clinging.''

Acharn: ``It is a reminder how much ignorance and clinging are there.''

Nina: ``We have regret about what is all gone.''

Acharn: ``If there is no understanding of the present reality there will
not be any understanding of the past and the future. There is only
thinking. Life is just the arising of different realities. We begin to
understand the reality of dhamma, not just the word dhamma. Seeing, for
example, is real and there is no need to say that seeing is dhamma. It
is the same for hearing. We begin to understand the nature of dhamma: it
is arising and falling away and never comes back.''

Sati is aware of the reality appearing now, at the present moment.
Acharn repeated many times that there is seeing now and that its
characteristic can be investigated with awareness. When we think about
seeing or talk about it, it is not the same as attending to the
characteristic of seeing when it appears at the present moment. We do
not know the past since it is gone, nor do we know the future which has
not come yet. The reality appearing at the present moment can be
investigated.

We read in the ``Kindred Sayings'' (I, the Devas, Ch I, a Reed, 10,
Forest)\footnote{Translated by Ven.Bodhi.} that the Buddha
spoke about the benefit of attending to the present moment:

\begin{verse}
At Sāvatthi. Standing to one side, that devatā recited this verse in the presence of the Blessed One:

``Those who dwell deep in the forest,\\

Peaceful, leading the holy life,\\

Eating but a single meal a day:\\

Why is their complexion so serene?''

The Blessed One:

``They do not sorrow over the past,\\

Nor do they hanker for the future,\\

They maintain themselves with what is present:\\

Hence their complexion is so serene.\\

Through hankering for the future,\\

Through sorrowing over the past,\\

Fools dry up and wither away\\

Like a green reed cut down.''
\end{verse}



\chapter[Gradual Development]{}
\section*{Gradual Development}


Understanding of the realities that appear through the eyes, the ears,
through the other sense-doors and through the mind-door should be known
as they are, as non-self. First there should be intellectual
understanding of realities and this can condition later on direct
understanding. Intellectual understanding is called in Pali: pariyatti.
Pariyatti pertains to the reality at this moment, be it seeing, visible
object, body-consciousness or hardness. Pariyatti is not mere
theoretical knowledge, it is not different from considering reality
appearing at this moment. There cannot be direct awareness and
understanding of these realities yet, but one can begin to consider them
when they appear. The texts help us to consider the realities that
appear now. When we read the teachings we should remember that they
pertain to this very moment.

We read, for example, in the ``Kindred Sayings'' (IV, Third Fifty, 5,
§152) that the Buddha said to the monks:

\begin{quote}

``Is there, brethren, any method, by following which a brother, apart
from belief, apart from inclination, apart from hearsay, apart from
argument as to method, apart from reflection on reasons, apart from
delight in speculation, could affirm insight, thus: `Ended is birth,
lived is the righteous life, done is the task, for life in these
conditions there is no hereafter'?''
\end{quote}

The Buddha then explained that there is such method:

\begin{quote}
``Herein, brethren, a brother, beholding an object with the eye, either
recognizes within him the existence of lust, malice and illusion, thus:
`I have lust, malice and illusion', or recognizes the non-existence of
these qualities within him, thus: `I have not lust, malice and
illusion'. Now as to that recognition of their existence or
non-existence within him, are these conditions, I ask, to be understood
by belief, inclination, or hearsay, or argument as to method, or
reflection on reasons, or delight in speculation?''

``Surely not, lord.''

``Are not these states to be understood by seeing them with the eye of
wisdom?''

``Surely, lord.''
\end{quote}

The Buddha said that this was the method. He then said the same about
the other sense-cognitions. The Buddha spoke time and again about
seeing, hearing and the other sense-cognitions. One should know one's
defilements when they arise. There should not be mere intellectual
understanding; dhammas should be ``seen with the eye of wisdom''.

We learn that nāma (citta and cetasika) is a reality that experiences an
object and that rūpa is a reality that does not experience anything.
Hardness which is a rūpa could not appear if there were not a citta that
experiences it. We may begin to understand that not a self experiences
hardness or any other object. That is understanding of the level of
pariyatti. Pariyatti, when it has been sufficiently developed, leads to
paṭipatti, awareness and direct understanding of the reality that
appears now. Paṭipatti leads to paṭivedha, the direct realization of the
truth beginning with the stages of insight-knowledge
\footnote{Insight, direct
understanding of nāma and rūpa, is developed in the course of several
stages of insight leading on to enlightenment, when nibbāna is
experienced and defilements are eradicated.} and leading on to
enlightenment. But if one wishes to have direct understanding and clings
to it, it will not arise.

Acharn spoke many times about pariyatti, explaining that it is different
from just reading the teachings: ``It is this moment. It is the same as
coming to the Buddha and listening to his teaching. It all pertains to
whatever appears now.''

We read in the ``Mahāparinibbāna
Sutta''\footnote{Wheel Publication
67-69, Kandy. Ven. Nyanaponika added in a note the explanation of the
commentary to this sutta.} that the Buddha,
before he passed away, exhorted the monks: ``Behold now, Bhikkhus, I
exhort you: Transient are all the elements of being! Strive with
earnestness!''

The commentary explains: ``You should accomplish all your duties without
allowing mindfulness to lapse!''

We should listen heedfully and learn to understand the present reality,
then we follow the Buddha's teachings. Acharn explained that all of the
teachings deal with the present reality as non-self. She said several
times that we should carefully study each word of the teachings. I
remarked that life is so short and that we should not waste
opportunities to hear true Dhamma.

Acharn answered: ``Understanding of the words `once upon a time` can
condition detachment. There can be more understanding of each moment as
just once in a life time.''

We think of a whole life that lasts but actually life is only one short
moment of citta, such as seeing, hearing or thinking. They are part of
the cycle of birth and death which goes on and on so long as there is
ignorance and attachment. Each reality that arises falls away and never
returns. It occurs only once in a life-time.

We read in the ``Mahāniddesa'' (I, 42) quoted by the Visuddhimagga (VII,
39):

\begin{verse}
``Life, person, pleasure, pain- just these alone\\

Join in one conscious moment that flicks by.\\

Ceased aggregates of those dead or alive\\

Are all alike, gone never to return\\

No world is born if consciousness is not\\

Produced; when that is present, then it lives;\\

When consciousness dissolves, the world is dead:\\

The highest sense this concept will\\
allow.''\footnote{As to the word ``the
highest sense this concept will allow'', the commentary to the
``Visuddhi-magga'' explains: ``the ultimate sense will allow this
concept of continuity, which is what the expression of common usage
``Tissa lives, Phussa lives'' refers to, and which is based on
consciousness (momentarily) existing along with a physical support; this
belongs to the ultimate sense here, since, as they say, ``It is not the
name and surname that lives'' (Paramattha-mañjūsā 242, 801).
}
\end{verse}

As we read: ``ceased aggregates of those dead or alive, are all alike,
gone never to return.'' The nāma and rūpa that fall away at this moment
will never return and so it is at the moment of dying. When
understanding is developed of the present reality there will be less
clinging to a self who could make realities arise or be master of them.

Acharn explained that when the citta is full of akusala there will not
be much interest in listening to the Dhamma and developing
understanding. Akusala has been accumulated in many lives and, thus,
very few moments of kusala citta arise. The good qualities which are the
perfections\footnote{The perfections or
pāramīs are: generosity, morality, renunciation, wisdom, energy,
patience, truthfulness, determination, loving kindness, equanimity. The
Buddha developed these for aeons in order to become the Sammāsambuddha.} are
supportive to the development of paññā up to the stage of enlightenment.
We should develop them, not because we expect a result of kusala, but
because we see the danger of each kind of akusala. Our aim is the
eradication of defilements and eventually to reach the end of the cycle
of birth and death.

Kusala citta can arise with paññā or without it. Kusala is not a
perfection when it is not accompanied by paññā, but paññā does not arise
very often. Acharn spoke about ``pre-pāramīs'', indicating that kusala,
even without understanding, can precede the arising of the perfections.
At the moment of kusala citta there is no opportunity for the arising of
akusala citta, and, thus, there is no accumulation of akusala. It
depends on conditions what type of kusala citta arises. If we try very
hard to make kusala citta with paññā arise, we are clinging to the idea
of self who can exert control over realities. We need the perfection of
truthfulness so that we do not mislead ourselves, believing that there
is kusala citta whereas in reality there is the wrong view of self. We
need patience and courage so that we are not discouraged and paññā
continues to investigate the characteristic of the present reality.

The present reality is nāma or rūpa. Nāma, citta and cetasika,
experiences an object, whereas rūpa does not experience anything. There
are twentyeight classes of rūpa, but seven rūpas appear all the time in
daily life. These are: visible object appearing through the eye-door,
sound appearing through the ear-door, odour appearing through the
nose-door, flavour appearing through the tongue-door, and the three
tangibles of solidity (the Earth Element), temperature (the Fire
Element) and motion (the Wind Element) to be experienced through the
bodysense. Solidity appears as hardness or softness, temperature appears
as heat or cold, and motion appears as motion (oscillation) or pressure.

Rūpas do not arise solely, they arise in groups. The four Great Elements
which are the Element of Earth or solidity, of Water or cohesion, of
Fire or temperature, and of Wind or motion, always arise with each group
of rūpas, they are the foundation for each group. The Element of Water
or cohesion cannot be experienced through the bodysense, only through
the mind-door. Visible object is always accompanied by the four Great
Elements. The four Great Elements arise in different combinations with
visible object and that is why they condition visible object to be seen
as different colours. For example, the Element of Earth or solidity that
accompanies visible object may have different degrees of hardness or
softness, the Element of Fire or temperature that accompanies it may
have different degrees of heat or cold. There are many varieties in
these Elements.

A rūpa such as visible object is not only experienced by seeing, it is
experienced by several cittas arising in a process. Rūpa does not fall
away as rapidly as nāma\footnote{Rūpa lasts as long as
seventeen moments of citta.}.
One rūpa such as visible object can be experienced by several cittas
arising in a process. Only seeing sees visible object, and the other
cittas of that process do not see, but they perform other functions
while they experience visible object. When visible object, sound or
another sense object has been experienced by cittas arising in a
sense-door process, it is experienced by cittas arising in a mind-door
process. Thus, rūpa can be experienced through a sense-door and after
the sense-door process is over, it is experienced through the mind-door.
Nāma is only experienced through the mind-door. We should not try to
find out when there is a sense-door process and when a mind-door
process. Cittas arise and fall away in different processes extremely
rapidly and only when the first stage of insight arises will we know
what a mind-door process is.

Acharn said that we discuss seeing and visible object, hearing and sound
so that there are conditions for the arising of awareness. Without
intellectual understanding the arising of awareness is not possible. She
said about the experience of hardness: ``When hearing again and again
that there is no one in hardness, no arms, no legs, that it is only
hardness, there can be conditions for understanding the characteristic
of hardness. It just appears and there is no need to name it. Usually it
does not appear. There is touching and then other things are experienced
immediately. But when sati arises hardness appears, even if it is very
short. It is different from the moment when it does not appear to sati,
there is just a slight difference. When there is more attention to that
characteristic with the understanding that there is no one in it, paññā
develops. Because of conditions one does not pay attention to other
things at that moment. Paññā begins to understand that characteristic as
not `me' or `I'. Hardness appears to the reality that is aware. There is
no idea of `I am aware.' One can understand the anattāness of reality.
It arises unexpectedly.''

Our discussions about nāma and rūpa were held in different places. The
location was changed, but the subject of discussion was always about
realities appearing now. Nāma and rūpa appear, wherever we are. We went
to Hotel Toscana, outside Bangkok and this was a resort in the
mountainous region of Suanpheung past Ratchbury. Our hostesses were Khun
Luk and Khun Ten. They were very concerned about my handicap and
arranged things in such a way that I would stay in their bungalow, in a
room near Sarah and Jonothan so that I would be more comfortable. All
discussions were in a new building where Acharn and her sister Khun Jeed
stayed. The place was hilly with a large orchard. Our hostesses
supported me whenever I had to take big steps to enter the bungalow or
to go out. We took turns to sponsor the luncheons which were nearby in
woods or near waterfalls. All around we had a panoramic view of the
mountains.

At that time there was a cold period for a few days, and in the morning
even frost was on the grass. Maeve became ill and had to go to hospital
where two of our friends, Elle and Azita, were allowed to stay overnight
with her. They discussed different cittas that arise in such situations.
Kusala cittas arise when helping, but there were many akusala
vipākacittas when unpleasant odours were experienced. Elle and Azita,
while walking, saw a picture and each of them was taken in by what she
saw according to her different accumulations. Elle who is always engaged
with flowers and who arranges the flowers at the Foundation in Bangkok,
saw immediately flowers on that picture. Whereas Azita, who is a nurse,
saw on the same picture a mother nursing her child. We all follow our
different accumulations in life. It is due to the different
accumulations as to what is interpreted and imagined on account of the
visible object which is seen.

Awareness of nāma and rūpa should be very natural so that paññā comes to
know accumulations. When lobha, attachment, arises, paññā can come to
know it. If it is ignored, paññā will never know it. I said that when I
find the akusala that arises very ugly, I do not want to know it, I
rather suppress it. Acharn explained that if one tries not to have
akusala with an idea of self who is trying there is wrong practice.
Awareness should be very natural. Natural is the way of anattā, she
said. I thought before that the natural way of development is easy, but
now I see that it is not easy. The natural way is difficult when
defilements are in the way. When paññā becomes stronger it is a
condition for the natural way of development. There can be awareness and
understanding of whatever dhamma appears, pleasant or unpleasant,
wholesome or unwholesome. This is the way to know our accumulations. The
perfection of truthfulness is necessary, so that we do not delude
ourselves into thinking that we have a great deal of kusala.




\chapter[The hidden Self]{}
\section*{The hidden Self}

So long as we are not a sotāpanna\footnote{The sotāpanna or
``streamwinner'' is the person who has attained the first stage of
enlightenment. He has eradicated wrong view, but he still has
defilements. There are four stages and at each stage different
defilements are eradicated. The stage of the arahat, when all
defilements are eradicated, is the fourth stage.} who has eradicated the
wrong view of self we are not free from clinging to the idea of self.
Acharn helped us to realize that clinging to the idea of self happens
more often than we ever thought. I had a conversation with Acharn about
wrong view. I thought that there was just ignorance, not clinging to the
idea of self.

Nina: ``I am not thinking all the time that this is my eye or that I am
seeing. So, there is just ignorance.''

Acharn: ``What is there?''

Nina: ``Ignorance.''\footnote{Rūpa lasts as long as
seventeen moments of citta.}

Acharn: ``But the idea of `I see' is there. Not the other person sees,
it is `I see',''.

Nina: ``Where is it when I do not think `It is I?'\,''

Acharn: ``If there would be no I at the moment of seeing it would be
completely eradicated.''

Nina: ``We usually think of concepts like a table or a person who is
sitting here.''

Acharn: ``At the moment of seeing, who is seeing? The other person? Not
the other person is seeing.''

Nina: ``I, I who is seeing''.

So long as it is not directly understood that seeing sees, we are bound
to take seeing for self, even if it is not apparent. That is why Acharn
spoke about seeing and visible object every day. It seems that there is
no wrong view but it is there. She also reminded us that when we read or
study, this may be with the idea of self. One may think: ``O, I have
read this, I understand better'' and that is reading with the idea of
``I want to have more understanding''. Paññā has to become keener and
keener to see when the idea of self is there, no matter how large or
slight it is. It seems that we have understanding of words like nāma and
rūpa, or of dhamma, but these are just words and there is no
understanding of a characteristic of reality that appears. If one would
never consider what appears now it means that there is no understanding.
Acharn repeated again: ``Dhamma means `no one there' in reality.''

We were talking about accumulated inclinations and I mentioned that I
like to appreciate what is wholesome in others, that I am inclined to
``anumodana dāna''\footnote{Anumodana means
gratefulness, and dāna is generosity. It is the appreciation of someone
else's kusala.
}.
Acharn mentioned that there may be attachment at such moments: one likes
to have such thoughts and one may be clinging to the idea of self at
those moments. It is true, most often one clings to a self, a self which
is thought to be kusala. This is not known most of the time.

We may want to have more understanding than we actually have at this
moment, and that is clinging to a self, that is wrong. Whatever we say
or think, mostly it is done with the idea of self. There may be clinging
to the idea of self even when we do not think, that it is ``I'' or
``mine''. Clinging is a yoke, it is like the thread of a spider's web,
very fine but strong and hard to cut through. 

Seeing sees visible object. When sati arises one can begin to know that
it is not ``I'' who sees. Seeing is different from visible object. Only
very little at a time can be understood. Visible object may appear, but
we should not try to make ourselves experience visible object with
nobody in it. When we learn more about nāma and rūpa there will be
conditions for awareness of them.

Very shortly after seeing, hearing or the other sense-cognitions akusala
citta with clinging arises already. Acharn said that it is not easy to
understand that there is clinging to seeing right now. It sees. When
asked ``who sees?'' we would answer that it is ``me''. She explained
that the more understanding develops, the more it realizes how difficult
and subtle the Path is. As understanding develops, it has to understand
more and more subtle defilements and other dhammas as not self. Paññā
can see how complex it is to have more understanding of each reality. If
there is no understanding latent tendencies cannot be eradicated. When I
remarked that we would have less problems when there is more
understanding, she answered: ``Right, but paññā goes deeper, deeper than
we can imagine. Paññā has to become very keen and develop, otherwise it
cannot understand realities as not self. Paññā has to see lobha in order
to let go of taking lobha for `me'. Energy or effort (viriya cetasika)
encourages one to continue all the way.''

Clinging can be so subtle that it is not noticed. That is why the Buddha
taught us the akusala cetasikas which are āsavas, intoxicants or
cankers.

There are four āsavas (Dhammasangani §1096-1100):

\begin{enumerate}


\item the canker of sensuous desire (kāmāsava),

\item  the canker of becoming (bhavāsava),

\item the canker of wrong view (diṭṭhāsava),

\item the canker of ignorance (avijjāsava).

\end{enumerate}

The āsavas keep on flowing from birth to death, they are also flowing at
this moment. Are we not attached to what we see? Then there is the
canker of sensuous desire, kāmāsava. We are attached to visible object,
sound, odour, flavour and tangible object. We are infatuated with the
objects we experience through the senses and we wish to go on
experiencing them. One of the cankers is clinging to becoming. Every one
clings to becoming, to being alive. We want to experience all objects
through the sense-doors.

Another group of defilements is the group of the Floods or Oghas
(Dhammasangani §1151). There are four floods which are the same
defilements as the cankers, but the classification as floods shows a
different aspect. The ``floods'' submerge a person again and again in
the cycle of birth and death.

Another group of defilements are the Yoghas or Yokes. They are the same
defilements as the cankers and the floods. The yoghas or yokes are
stronger than the āsavas, they tie us to the cycle of birth and death.

We often ask questions with an idea of self, and it is unknown when we
cling to the idea of self at such a moment. Acharn would remind us all
the time: ``There is a yoke.'' When I answered that I would not say
anything any more, she said ``Yoke again''. We cannot escape the yokes
but they are there to be known. We have to develop understanding, only
paññā can know realities precisely.

Acharn said: ``Does one mind about having kusala or akusala? When one
minds it is `me', the yoke is there. If one tries to stop akusala, how
can one know one's accumulations? The idea of `self` is so strong. There
is no understanding that it is there, while one is wishing. Many people
just want to be good and they do not know their defilements at this
moment. Right mindfulness can arise before you can think about wanting
to have it, or waiting for its arising. In the same way as seeing
arises. This is the understanding of its nature of anattā. Paññā
understands when there is a moment with sati and when without sati.
Otherwise sati cannot develop. It does not develop with desire and this
is very difficult. Usually there is attachment but paññā can begin to
understand attachment. One is trapped all day.''

Seeing only sees what is visible object and after it has gone we think
of many stories on account of what was seen. Seeing arises only for one
moment and at that moment people and things do not appear. After that
many moments of thinking arise. Every reality arises only once, ``once
upon a time'', and then it is gone completely. Acharn asked several
times: ``Is it worth clinging to what is completely gone, each moment?''
We think of what is past, once upon a time, and then we live in a dream.
Without awareness and direct understanding life is like a dream. Even
when we talk about ultimate realities we are dreaming, we are not
mindful of them. When direct understanding arises we are awake just for
a moment. At the moment of direct understanding no words are needed and
as soon as we use a word we are thinking. At that moment the reality has
gone completely.

We learn from the texts that kamma produces result, vipāka, in the form
of rebirth-consciousness and of sense-cognitions throughout life, such
as seeing or hearing. Without understanding of the characteristic of
seeing, we cannot know what vipākacitta is. ``It is still me, not
vipāka'', Acharn said. In the beginning it is not possible to understand
seeing as vipāka. Seeing has to be known as a reality, as a dhamma.
Seeing is nāma, it has no shape and form; it arises because of
conditions and it sees now. It is different from thinking. We do not
have to name it vipāka or think of vipāka. No one can prevent seeing
from arising. It is uppatti (origin, coming forth). It just appears for
a moment, but we believe we see people and think of many stories, and
that is nibbatti (generation, resulted)
\footnote{These terms occur in
the ``Visuddhimagga'' XXI, 37, 38, under ``appearance as terror''.
``Herein, arising (uppādo) is appearance here in this becoming
with previous kamma as condition (purimakammapaccayā idha
uppatti)\ldots{} Generation (nibbatti) is the generating of aggregates
(the khandhas)''.}. There are five pairs
of the sense-cognitions of seeing, hearing, smelling, tasting and the
experience of tangible object through the bodysense, and of each pair,
one is kusala vipākacitta, the result of kusala kamma, and one is
akusala vipākacitta, the result of akusala kamma. These cittas directly
experience a sense object as it arises at the appropriate sense-base.

We can come to know the difference between seeing that directly sees
visible object and the other cittas that follow. Even the citta that
succeeds seeing and that, though it does not see, still experiences
visible object, needs the cetasika vitakka, thinking, in order to be
able to experience visible object. Vitakka is translated as thinking,
but it is not thinking in conventional sense. It ``strikes'' or touches
the object so that citta can experience it. Afterwards in that
sense-door process kusala cittas or akusala cittas arise in a series of
seven and they experience visible object in a wholesome way or
unwholesome way. In the following mind-door process kusala cittas or
akusala cittas arise, and after that mind-door processes of cittas arise
that think about the object. Thus, the vipākacittas that are the
sense-cognitions of seeing, hearing, smelling, tasting and the
experience of tangible object directly experience the relevant sense
object and they do not need vitakka. They are completely different from
all following cittas. In order to distinguish them from the following
cittas they are called ``uppatti''. Whereas the following cittas are
called ``nibbatti''.

Throughout our discussions Acharn emphasized time and again the
difference between uppatti and nibbatti. Uppatti, such as seeing or
hearing, is what appears now. One moment of seeing or hearing is quite
different from the following moments of citta when we think of what was
seen or heard, when we think of stories that are not real, when we live
in a dream. It reminds us of the fact that seeing arises and falls away
very rapidly and that after they have fallen away we are thinking on
account of what is seen for a long time. We believe that the stories we
think of are true. We like what has already fallen away. We continue to
live in the past, in what is ``once upon a time''. Thinking is
conditioned and we should not try not to think but it can be understood
as a reality different from seeing. It is beyond control. The notions of
uppatti and nibbatti remind us of the nature of anattā of realities. The
Buddha explained time again about the sense-cognitions and the objects
experienced by them. After the sense-cognitions kusala cittas or akusala
cittas may arise. When there are mindfulness and understanding,
ignorance will be eliminated and even arahatship may be attained.

In the ``Bāhiyasutta'' (``Minor Anthologies'', Khuddaka Patha, the
``Verses of Uplift'' Udāna, I, 10) we read that Bāhiya Dārucīriya
thought of himself as an arahat. A deva advised him to visit the Buddha
at Sāvatthī. He asked the Buddha to give him a teaching but the Buddha
refused this two times. The
commentary\footnote{Translated by Peter
Masefield, Volume I. PTS.} explained
that the reason for this was that Bāhiya was too excited to listen. When
Bāhiya asked for a teaching the third time, the Buddha said:

\begin{quote}
``Then, Bāhiya, thus must you train yourself: In the seen there will be
just the seen, in the heard just the heard, in the imagined just the
imagined, in the cognized just the cognized. Thus you will have no
`thereby'. That is how you must train yourself. Now, Bāhiya, when in the
seen there will be to you just the seen, in the heard just the heard, in
the imagined just the imagined, in the cognized just the cognized, then,
Bāhiya, as you will have no `thereby', you will have no `therein'. As
you, Bāhiya, will have no `therein', it follows that you will have no
`here' or `beyond' or `midway between'. That is just the end of Ill.''
\end{quote}

We read that Bāhiya attained arahatship. Not long after the departure of
the Exalted One, Bāhiya was attacked by a cow and gored to death.

We read in the commentary as to ``with respect to the seen\ldots{}
merely the seen\ldots{}'': ``It is of the extent seen (diṭṭhamattaṃ)
since it has the extent seen (diṭṭhā mattā), meaning, the thought
process will be of the same extent as eye-consciousness. This is what is
said: `Just as eye-consciousness is not excited, is not blemished, is
not deluded with respect to the form that has gone into its range, so
will there be for me an impulsion of the same extent as
eye-consciousness in which lust and so on are absent, I will set up an
impulsion of the same measure as eye-consciousness.'\,''

Seeing-consciousness is vipākacitta which is not accompanied by the
unwholesome roots of attachment, aversion and ignorance. It merely sees
visible object. Usually akusala cittas with attachment and ignorance
follow upon seeing, but when there is awareness and right understanding
instead they do not arise. Bāhiya could not have reached arahatship
without realizing nāma and rūpa as mere dhammas. We read further on in
the commentary about visible object and seeing:''\ldots{} occurring (as
they do) in accordance with conditions, being solely and merely dhammas;
there is, in this connection, neither a doer nor one who causes things
to be done, as a result of which, since (the seen) is impermanent in the
sense of being oppressed by way of rise and fall, not-self in the sense
of proceeding uncontrolled, whence the opportunity for excitement and so
on with respect thereto on the part of one who is wise?''

The same is said with respect to the other objects experienced by the
sense-cognitions through the relevant doorways.

It will take a long time to know seeing as it is, as a mere dhamma.
Sometimes a moment of understanding may arise and after that ignorance
arises again and covers up the truth. As Acharn said, we should not
mind, because that is the way it is. If we long for more understanding
we are yoked again.



\chapter[Thinking of the Past]{}
\section*{Thinking of the Past}

We think of death as the end of a lifespan but in reality there is at
each moment death of citta that falls away. There are three kinds of
deaths\footnote{``Dispeller of
Delusion,'' Commentary to the Book of Analysis, Classification of the
Truths, 101.
}: momentary
death, khaṇika maraṇa, which is the arising and falling away of all
conditioned dhammas; conventional death, sammuti maraṇa, which is dying
at the end of a lifespan; final death, samuccheda maraṇa, which is
parinibbāna, the final passing away of the arahat who does not have to
be reborn.

Life goes on without understanding the truth. Seeing sees visible object
and because of our delusion we think that we see people who seem to be
already there. It also seems that seeing can stay, that we are seeing
all the time. Ignorance covers up the truth. Seeing falls away
immediately, but since dhammas arise and fall away so rapidly it seems
that the different moments of seeing that arise again and again are one
period of time that lasts for a while. Understanding can be developed of
the characteristic of seeing and there is no need to pinpoint in what
process it has arisen.

Seeing falls away but the sign (in Pali: nimitta) of seeing remains.
Even so visible object falls away but the sign or nimitta of visible
object remains. The nimitta covers up the truth of realities which arise
and fall away very rapidly in succession. No one can really directly
experience one particular reality, because there are so many realities
arising and falling away. The rapid succession of dhammas, such as
visible object, leads to the experience or impression of shape and form
and to the idea of people and things. A simile can be used to explain
this: when we take a torch that we swing around we notice a circle of
light. In fact what we take for a circle of light consists of many
moments, but it seems to be a continuous whole. There is a reality that
can be seen and paññā begins to understand that what is seen cannot be
any one at all. Only memory and thinking condition the idea of someone
or something.

The term saṅkhāra nimitta, the sign of conditioned dhammas
\footnote{The ``Path of
Discrimination'' (``Patisambhidhamagga''), I, 438 speaks about seeing as
terror the signs of each of the five khandhas, whereas nibbāna is
animitta, without sign. Saṅkhāra nimitta also occurs in the
Visuddhimagga XXI, 38.} pertains to the fact
that each of the five khandhas\footnote{All conditioned
dhammas, saṅkhāra dhammas, can be classified as five different khandhas
or aggregates. One khandha is rūpa, and four are nāma.} which arise and fall
away has a nimitta: rūpa-nimitta, vedanā-nimitta (feeling),
saññā-nimitta (remembrance), saṅkhāra-nimitta (the other fifty cetasikas
apart from feeling and saññā)\footnote{In this context
saṅkhāra refers to saṅkhārakkhandha. The term saṅkhāra dhammas refers to
all conditioned dhammas, to all khandhas. Saṅkhārakkhandha refers to one
khandha, the khandha of ``formations''.} and viññāṇa-nimitta
(citta). Since nibbāna does not arise and fall away it is without
nimitta, it is animitta.

Conditioned dhamma falls away but the nimitta remains. It is a sign or
nimitta of the reality that arises and falls away, but we do not realize
the arising and falling away. We mislead ourselves, taking for permanent
what is impermanent. We take for self what is beyond control. There is
no need to think all the time: ``it is a nimitta'' or, ``the reality has
fallen away''. Characteristics are appearing and they can be
investigated. Saṅkhāra nimitta denotes a nimitta of a reality appearing
right now. The reality and its nimitta can be compared to a sound and
its echo. We should remember what Acharn said: ``The reality and the
nimitta of it appear like sound and its echo, who knows which is which?
Instead of finding out whether nimitta is a paramattha dhamma, know that
it is now. No one can pinpoint a moment of experiencing an object or the
object itself.''

We learn to be aware of characteristics of dhammas that appear but
knowing about nimitta makes it clearer that dhammas fall away so fast.
It helps to understand their nature of anattā, they are beyond control.
What has arisen is gone already before we realize it. ``Once upon a
time'' can be seen as an extremely short moment ago. We can remember
what Acharn said long ago: ``We have dear people, people who are close
to us, but dhamma arises and then falls away. Seeing has fallen away and
there is nothing left. Thinking and all dhammas fall away completely.
This is not different from the moment a dear person dies. We are
thinking about a dear person but thinking falls away completely.''

The colour that appears through the eyes is the nimitta, the sign
referring to the visible object that is accompanied by the four Great
Elements of Earth (solidity), Water (cohesion), Fire (temperature) and
Wind (motion). There is a great variety of the four Great Elements, and
since they have different degrees of hardness, softness, heat or cold,
it is a condition for the nimittas to be varied. Whenever visible object
appears or seeing appears, there is the sign of the rapidly arising and
falling away of realities. A single moment of seeing cannot be
experienced.

The succession of the arising and falling away of visible object leads
to an idea of continuity, the perceiving of shape and form. Acharn
explained: ``Memory just marks and forms up the idea of a particular
shape and form of this or that person. It is all that can be seen. Close
your eyes and there is no more that which can be seen\ldots{} Without
reality there is no nimitta but the arising and falling away is so rapid
that it cannot be directly experienced.''

Because of wrong remembrance of self, attā-saññā, the nimitta is taken
for something. Concepts are thought of because of different nimittas.
Sarah also gave some more explanations: ``Thinking has an idea of shape
and form and that leads to the idea of eyebrows, people and things.
Without experiencing visible object many times there could not be the
sign of visible object and without that sign there could not be thinking
about the outward appearance and details of things. One thinks of
concepts of people and things on account of what is seen.''

When we have no understanding there are just concepts about realities as
permanent phenomena which don't arise and fall away instantly. When we
have more understanding of nimitta, we see that whatever we experience
arises for a moment and is then completely gone. We cannot hold on to
it. Acharn said:

``That is life. No matter how happy or unhappy we are, all these moments
are gone. What we take for so very important in life is gone. Such
moments are just objects of ignorance and attachment. What is the use of
experiencing all these realities at this moment? Understanding this is
the beginning of seeing dukkha, which is the arising and falling away of
realities. Each conditioned reality is dukkha. It just arises and falls
away and it cannot be controlled. It is time to eat, to sleep, to move,
to think, but we have an idea that `I will do this'. We can come to
understand the paññā of the Buddha and his compassion to teach, to let
others understand whatever appears.''

At the end of my stay in Thailand a short visit to Chiengmai was planned
and I wanted to join this. We took the plane and stayed one night, but
there were two full days of Dhamma discussions. I had been to Chiengmai
before and, thus, it was a happy meeting again with old friends. We had
lunch in the cultural center where we were offered traditional Northern
dishes, like bamboo filled with pork and a great variety of vegetables.
The sessions were in an auditorium in the hotel where we stayed
overnight. I had to climb a podium with very steep steps, but people
assisted me from all sides.

Acharn explained about nimitta that when seeing, there is clinging to
the nimitta as something or somebody. The impingement of visible object
on the eyesense is a condition for seeing that arises and falls away
very rapidly. Because of the arising and falling away again and again a
nimitta or sign of continuity appears. She said: ``There is a nimitta of
different shapes and forms. Saññā remembers them wrongly as something
that stays. There is wrong remembrance of self, attā-saññā. Concepts are
known because of different nimittas. Because of a concept we know what
something is. Because of thinking of nimitta we know when and where
there is food. If there is no reality, there is no nimitta and no shape
and form.''

We have to know the extent of our understanding and if we try to find
out more than we can understand, we are clinging again. When
understanding develops we can let go of clinging very gradually. We have
possessions in our house but do we have them now? We are only thinking
of them. When we return home, they may not be there anymore. We should
develop understanding with courage and cheerfulness. People mostly
follow their own ideas and do not study the teachings with respect.
Therefore, the teachings will dwindle and disappear.

The last afternoon, before our departure, one of our friends sang a song
in honour of Acharn. She praised her wisdom in explaining the Dhamma to
all of us. The song was very charming with a melody in the Northern
style of music. When we were at the airport on our way back we waited in
the VIP room where we had a Dhamma discussion for another hour. People
showed a great interest and had many questions. Acharn reminded us again
of the Buddha's last words, saying that we should not be neglectful,
also with regard to listening to the Dhamma. There are dhammas all the
time but we do not know that they are dhammas. Their different
characteristics should be investigated.

The Buddha taught the four noble Truths: the Truth of dukkha, the Truth
of the origin of dukkha, the Truth of the cessation of dukkha and the
Truth of the Path leading to the cessation of dukkha.

Dukkha is the arising and falling away of dhammas. What arises and falls
away is not worth clinging to, it is unsatisfactory. The origin or cause
of dukkha is craving, because of craving we have to reborn and that
means the arising and falling away of nāma and rūpa again and again. The
cessation of dukkha is nibbāna and the Path leading thereto is the
eightfold Path.

Paññā has to be developed on and on for aeons before the four noble
Truths can be penetrated. The Buddha showed in his first sermon\footnote{Kindred Sayings, V,
420. The Foundation of the Kingdom of the Norm.
} that there are three
phases in the development of understanding and these pertain to each of
the four noble Truths.

There are three ``rounds'' or inter-twining phases of the understanding
of the four noble Truths. They are: understanding of the truth, sacca
ñāṇa, knowledge of the task to be performed, kicca ñāṇa, which is the
development of understanding of realities, and knowledge of the task
that has been done, kata ñāṇa, which is the direct realization of the
truth.

Acharn referred very often to these three ``rounds'' or phases and
explained that without the first phase, sacca ñāṇa, the firm
understanding of what the four noble Truths are, there cannot be the
second phase, kicca ñāṇa, the performing of the task, that is,
satipaṭṭhāna, nor the third phase, kata ñāṇa, the fruit of the practice,
that is, the penetration of the true nature of realities.

With regard to the first phase, she said that there should be the firm
intellectual understanding of the first noble Truth, and that means
understanding that there is dhamma at this moment, that everything that
appears is dhamma. Acharn said that it must be the firm understanding
that seeing arises and falls away, and that we should not be ignorant of
seeing. All dhammas should be known, otherwise the idea of self cannot
be eradicated. She said:

``Who sees? When anattā is understood it is the beginning of the right
Path.'' When we listen to the Dhamma and consider what we hear the
intellectual understanding of realities, that is, the first phase, sacca
ñāṇa, gradually develops and then it can condition the arising of
satipaṭṭhāna. This means that the second phase, knowledge of the task,
kicca ñāṇa, begins to develop. The practice, paṭipatti, is actually
knowledge of the task that has to be performed.

The second noble Truth is craving or attachment. Craving or clinging in
daily life should be understood. The clinging to self has been deeply
accumulated and we should consider this more. We cling to satipaṭṭhāna
and this can induce wrong practice. We should learn at what moment this
occurs, the test is always at this moment. Understanding of what appears
at the present moment through one of the six doorways leads eventually
to the abandonment of craving. Seeing, for example appears now and it
can be known as only a conditioned dhamma, no self who sees. However,
attachment takes us away from the present object, time and again, so
that we are forgetful of seeing that appears now. Also attachment can be
known as a dhamma.

The ceasing of dukkha, namely nibbāna, is the third noble Truth. Also
with regard to the third noble Truth there are three phases:
understanding what the ceasing of dukkha is, sacca ñāṇa. Paññā can come
to see the danger and disadvantage of the arising and falling away of
conditioned dhammas and it will see the unconditioned dhamma that does
not arise and fall away as freedom from dukkha. We should have the firm
understanding that detachment and the eradication of defilements is the
goal. We should be convinced that it is possible to attain this goal
only if we follow the right Path. Understanding of the task in order to
reach this goal is kicca ñāṇa. At the moment of enlightenment nibbāna is
experienced and defilements are eradicated. Understanding of the task
which has been performed, the realization of nibbāna, is kata ñāṇa.

The way leading to the ceasing of dukkha, namely the eightfold Path, is
the fourth noble Truth. Also with regard to the fourth noble Truth there
are three phases or rounds. The first round is understanding what the
development of this Path is, sacca ñāṇa. This is not theoretical
understanding, but it pertains to the development of understanding of
the dhamma appearing at this moment. Nāma and rūpa, paramattha dhammas,
are the objects of which understanding should be developed. These are
different from concepts, from the image of a `whole' of a person, of the
body, of a thing. When there is firm understanding of what the Path is,
we shall not deviate from it. The teaching of the three phases shows us
that the development of paññā is bound to be an age-long process. We
need to develop it with courage and patience.

Acharn was invited to speak at the ``World Fellowship of Buddhists''. I
went to their center with Jonothan. It was a long taxi drive because at
that time several streets were blocked during anti-government
demonstrations. We had to walk through a park to reach the place.

Acharn said:

``Reality is very daily. It should be studied, otherwise we never know
the truth. Does anything belong to you? Even seeing does not belong to
you. Right understanding, when it arises, begins to see realities as no
being. Seeing is seeing. At the moment of hearing there is no seeing. Is
sound real? It has its own characteristic. Nobody can change the
characteristics of realities. When one has not heard the Dhamma one
thinks: `I see a person'. Visible object is a reality that is seen and
after that one thinks of shape and form because of saññā. Each moment is
conditioned. Understanding is conditioned.''

Several people showed a real interest and asked questions. We discussed
the fact that it is not by chance that someone comes to a particular
place at a particular time to listen to the Dhamma. It must be because
there was an interest in the past and this has been accumulated so that
there are conditions to listen again, to consider again. In this way
understanding can grow.

The Buddha taught anattā all the time. Anattā of what? Of realities or
dhammas. We should not think so much about names and terms, but
understand the reality represented by a name. We may stare at the texts
but it may happen that the meaning escapes us. Then we may go all the
wrong way, and this is very dangerous.

When reading suttas it may seem that the Buddha spoke about impermanence
of concepts, such as persons or possessions, but this was the method of
teaching to certain people who needed first conventional truth until
they were ready to accept ultimate truth (paramattha sacca). So, he
often spoke about people in different situations. When reading about
conventional truth we can consider the deeper meaning, the truth of
realities.

When people had deep sorrow about the loss of dear ones, they needed at
first a gentle approach by way of situations and persons. Not everybody
is ready to accept the truth that each reality falls away very rapidly,
never to come back, and that there is nothing left. When we read in the
suttas about death we can be reminded of momentary death. At each moment
dhammas arise and then fall away never to return. If we believe that
people stay or that possessions are there all the time, we live in a
dream.

We read in the ``Sutta Nipāta'', the Group of Discourses, the Chapter of The Eights No.6\footnote{Translated by K.R.
Norman, PTS 1992.}:

\begin{quote}
`` Truly this life is short; one dies less than one hundred years old.
Even if anyone lives beyond (one hundred years), then he dies because of
old age.

People grieve for their cherished things, for no possessions are
permanent. Seeing that this separation truly exists, one should not live
the household life.

Whatever a man thinks of as `mine', that too disappears with his death.
Knowing thus indeed, a wise man, one of my followers, would not incline
to possessiveness.

Just as a man, awakened, does not see whatever he met with in a dream,
even so one does not see beloved people when they are dead and gone.

Those people are seen and heard of, whose name is `so and so'. When he
has departed, only a person's name will remain to be pronounced. Those
who are greedy for cherished things do not abandon grief, lamentation
and avarice. Therefore the sages, seeing security, have wandered forth,
abandoning possessions. Of a bhikkhu who lives in a withdrawn manner,
resorting to a secluded residence, of him they say it is agreeable that
he should not show himself in any dwelling.

Not being dependent upon anything, a sage holds nothing as being
pleasant or unpleasant. Lamentation and avarice do not cling to him, as
water does not cling to a (lotus)-leaf.

Just as a drop of water does not cling to a lotus(-leaf), as water does
not cling to a lotus, so a sage does not cling to what is seen or heard
or thought.

Therefore a purified one does not think that purity is by means of what
is seen, heard or thought, nor does he wish for purity by anything else.
He is neither empassioned nor dispassioned.''
\end{quote}

The commentary explains as to the words ``a bhikkhu who lives in a
withdrawn manner'', that he practises so that the citta becomes
detached. The word bhikkhu refers to the ``excellent worldling''
(kalyāṇa putthujana) or the ``trainer'' (sekha puggala, the ariyan who
is not arahat). As to not showing himself in any dwelling, this means
that the wise person is free from dying, he does not have to be reborn.

The development of understanding of whatever reality appears now leads
to detachment. As we read: ``A sage does not cling to what is seen or
heard or thought''. He understands realities as they are.


\chapter[Courage]{}
\section*{Courage}

The world with all the people is quite different from what we used to
think, before we heard the Dhamma. Even though we listened for a long
time we have not penetrated the truth of realities. We may repeat the
word ``There is no one there. Everything is dhamma'', but as we listen
more to Acharn we come to realize how little we know. This is
beneficial, we have to continue to listen and consider the Dhamma with
courage and cheerfulness.

The world seems so large, but there is only one citta that experiences
an object and then falls away. Acharn reminded us many times that we are
not together with another person but with citta that experiences visible
object and with citta that thinks, with citta that experiences sound and
with citta that thinks, with citta that smells odour and with citta that
thinks. We are alone in our own world. We think of another person but
there is only citta that thinks and then falls away.

Acharn said: ``Paññā can arise and it can accumulate. It is not a matter
of `doing something' but of understanding. Everyone would like to have
paññā, but the moment of understanding is paññā. When a reality appears
paññā can know the truth. Do not try to have it. At this moment it can
be known to what extent paññā has developed.''

We have to understand seeing and visible object. Time and again seeing
arises and, thus, we should not be forgetful of the present reality.
Some people may find seeing too ordinary to consider, not interesting
enough. But it arises because of the coming together of different
factors. Visible object and eye-base are rūpas that have not fallen away
yet. Rūpa does not fall away as rapidly as nāma. There are conditions
for them to associate exactly at the time they have not fallen away yet,
so that kamma, a deed committed in the past, can produce seeing. We
always took seeing for granted, but actually, it is amazing that seeing
arises.

Seeing experiences visible object and only for that extremely short
moment the world is bright. When seeing has fallen away other cittas
succeed seeing in the eye-door process which, although they do not see,
still experience visible object, but the world is no longer bright. It
seems that when we notice persons on account of what has been seen, that
the world is still bright, but this is not so. We are thinking and,
although our eyes are open, the world is dark. Thinking and other
experiences are interspersed with moments of seeing visible object very
rapidly, and it seems that we are seeing all the time. However, the
moment of seeing is extremely short, it arises and falls away. Thus, in
reality only one short moment is bright and all other moments are dark.

Because of our ignorance we take phenomena for permanent and self. It
seems that we see people and things and that whatever we see was there
already for a long time and that the world keeps on being bright.

In the beginning the momentary arising and falling away of realities,
one at a time, cannot be realized. Understanding has to be developed
further so that impermanence can be directly penetrated.

We read in the ``Kindred Sayings'' (IV, First Fifty, Ch 3, § 23,
Helpful), that the Buddha said:

\begin{quote}
`` `I will show you a way, brethren, that is helpful for the uprooting
of all conceits. Do you listen to it. And what, brethren is that way?

Now what think you, brethren? Is the eye permanent or impermanent?'

`Impermanent, lord.'

`What is impermanent, is that weal or woe?'

`Woe, lord.'

`Now what is impermanent, woeful, by nature changeable, -- is it fitting
to regard that as: This is mine. This am I. This is myself?'

`Surely not, lord.' ''
\end{quote}

The Buddha then explained the same about the other sense-doors, the
objects experienced through them, and the cittas that experience these
objects. He said that the person who realizes the truth attains
arahatship and eradicates conceit.

The Buddha draws our attention to realities such as the senses, all the
objects that can be experienced and all the cittas that experience these
objects. The Dhamma is very precise. We do not even know what hearing
is. It seems that we hear words that are spoken, or that we hear dogs
barking, but only sound is heard. We have to get used to realities one
at a time. It will take very long before the arising and falling away of
precisely this or that reality is directly known. In this sutta we see
that the Buddha mentions realities (dhammas) and not concepts, no
collection of things. In this sutta he truly teaches Abhidhamma, higher
dhamma or dhamma in detail. Or in other words, paramattha dhamma, dhamma
in the highest sense. The Buddha asked after each reality he mentioned
whether it was permanent or impermanent. He wanted the listeners to
consider the truth with respect to each reality, one at a time, right at
that very moment. It is only dhamma at this moment that can be
investigated.

A collection of things does not exist. Where is a person? Is it seeing,
hearing or thinking? Only one citta arises at a time. Once we understand
this, we know the difference between reality and concept.

The Great Disciples at the Buddha's time could after only a few words
realize the truth of impermanence: the arising and falling away of
seeing that appeared at that moment, of visible object and of the other
realities that appeared at that moment. But we are beginners.
Impermanence is not realized by thinking, it is by direct understanding
and no words are needed. It is not thinking: ``everything comes to an
end''. Anybody could come to this conclusion.

We read and repeat: ``all conditioned dhammas are impermanent'', but
these are only words to us. What arises and falls away at this moment: a
nāma or a rūpa? Citta with understanding and mindfulness can take only
one object at a time. Does seeing fall away now, or visible object? This
has to be known very precisely.

There are specific characteristics (visesa lakkhaṇa) and general
characteristics (samañña lakkhaṇa). The general characteristics are:
impermanence, dukkha, anattā. These general characteristics cannot be
realized immediately. First it has to be known precisely what seeing is
as different from visible object. The specific characteristics have to
be known first. So long as we join realities together we take them for
some ``thing'', for a self, for permanent. Seeing is different from
thinking, different from attachment, these are different realities, each
with their own specific characteristics. That is the reason why Acharn
always stressed: you have to know realities first as only a dhamma, and
that at the present moment, now.

The Buddha taught for forty-five years so that people would have
conditions for direct awareness and understanding. The Buddha had
immeasurable compassion to teach so that others could understand
whatever reality appears at this moment. Without him we would be in
complete darkness, the darkness of ignorance. We would not know what is
real and what is not real. We would not know our attachment and all
other vices, we would not know how to develop kusala. We should study
what the Buddha taught with genuine respect. Every word he said is
important.

We should begin to learn what dhamma is from this very moment. In the
beginning one does not know anything at all about dhamma, the reality
that is appearing now. When we listen we can begin to see that what
arises and appears at this moment is dhamma; we can understand it as
dhamma. We can understand the characteristic of dhamma instead of
thinking about the ``story'' of dhamma.

We read in the ``Path of Discrimination'' (Patisambhidamagga, Treatise
on Knowledge I, Ch 71, the Great Compassion) that Enlightened Ones when
seeing all the dangers and disadvantages of worldly life, have great
compassion for beings. We read at the end:

\begin{quote}
``Upon the Enlightened Ones, the Blessed Ones, who see thus, `I have
crossed over and the world has not crossed over; I am liberated and the
world is not liberated; I am controlled and the world is uncontrolled; I
am at peace and the world is not at peace; I am comforted and the world
is comfortless; I am extinguished and the world is unextinguished; I,
having crossed over, can bring across; I, being liberated, can liberate;
I, being controlled, can teach control; I, being at peace, can pacify;
I, being comforted, can comfort; I, being extinguished, can teach
extinguishment,' there descends the Great Compassion. This is the
Perfect Ones knowledge of the attainment of the Great Compassion.''
\end{quote}

It was the Buddha's great compassion to teach in such a way that people
who listened could develop their own understanding.

Many conditions are necessary for right understanding to develop. Acharn
often reminded us that there is not a self who is trying to develop
paññā, but that saṅkhāra-kkhandha is operating. Saṅkhārakkhandha (the
khandha of formations) includes all cetasikas apart from feeling and
remembrance. All sobhana cetasikas (beautiful cetasikas) are included
such as sati, paññā and other wholesome qualities. She explained that
the development of right understanding is understanding of whatever
appears. This is conditioned by listening and considering the Dhamma.
She said: ``Leave it to saṅkhārakkhandha. They are working on and on,
all by themselves.'' When we really consider this we shall be less
inclined to think that we have to ``do'' something special in order to
have more understanding.

All wholesome qualities, such as the ``perfections'' have to be
developed together with right understanding. Paññā is very weak, it
needs the support of all kinds of kusala so that the `other shore' can
be reached. This shore is the shore of defilements and the `other shore'
is enlightenment, when defilements are eradicated. We need courage,
viriya, so as not to become downhearted but continue on the right way of
development. We need dāna, generosity, so that we are not self-centered
all the time, thinking of our own pleasure. We need determination
(adiṭṭhāna) to continue on and on considering the reality appearing now,
whatever difficult situations we have to face, since we see the benefit
of right understanding. We need truthfulness, sincerity: to what extent
is there paññā and to what extent still ignorance. We do not want to be
deluded about the truth of realities and be blinded. We should not
mistakenly believe that we have understood what we are still ignorant
of. With sincerity we have to develop all kinds of kusala. We need
patience, to listen and carefully consider each word of the teachings.
We see that many conditions are necessary for the development of paññā.

Sarah spoke about difficulties many people face with anxiety and
depression. She said: ``We learn that these are kinds of aversion, not
liking, not accepting life now as it is. No one likes such states
because of the unpleasant feelings, but no one minds about the
attachment and pleasant feelings which lead to the anxieties and
depressions. So often, we find ourselves lost in the stories about past
and future and just forget that now, the realities are simply the seeing
of what is visual, the hearing of sounds and thinking about such
experiences. The ideas thought about in our imagination are not real.
This is why we look at the actual realities more and more.''

This is true, the more we listen, the more we come to see the importance
of understanding the present dhamma. In our daily life we are absorbed
in many different events that take place, or in what we read in the
newspaper. At the hotel in the small pool that I use for my early
morning swim, a huge snake was found, just ten minutes before I would
enter the pool. On account of this it is natural that we think of many
stories of what could have happened. At this time it was Chinese New
Year, the Year of the Horse. Children were dressed and performed a dance
mimicking a lion's movement. One could throw money inside his wide-open
mouth and then the lion would bow and thank the giver. Only visible
object is seen, but on account of visible object we go on thinking for a
long time. Gradually we can come to see the difference between thinking
of stories, of concepts and the experience of seeing and other ultimate
realities.

At our last session in the ``Foundation'' Acharn stressed all the time:
not the words are important, but what is understood right now at the
present moment. What about seeing now? We do not need any words, we have
to attend to its characterisic when it appears now. The present moment
cannot be emphasized enough. It is very helpful that Acharn stressed the
difference between textbook knowledge and understanding without naming
realities, by attending to their characterstics. We are likely to call
seeing vipāka (result of kamma) and clinging to visible object akusala
but we can learn that their characteristics are different when they
appear. Gradually we can learn that seeing is quite different from
attachment, without calling them by name.

Acharn spoke about ``seeing now'' every day. Once we have some
understanding of it as only a conditioned dhamma we will come to know
what a reality is as different from a concept. Acharn often explained
that what has fallen away never comes back and that this is the meaning
of dukkha: the reality that just appears and disappears and never comes
back. What was experienced in the morning is not now and what will be
experienced in the evening is not now. Each moment is past and there is
just the idea of self, of ``I'', all the time. What from head to toe
could be ``I''?

During our sessions clinging to the ``self'' became more apparent, even
when we do not think expressively: ``it is mine''. We may believe that
hardness is known as only hardness, but when it appears at some location
in the body it shows that we cannot let go of the idea of body, it is
always ``somewhere in my body''. Seeing appears but when it appears at
some location, namely at the eye-base, there is still an idea of my eye.
It takes a long time before there is detachment from the idea of
``self'' or ``mine''. Direct understanding of a dhamma is without words.
Even when we talk about ultimate realities we are thinking of concepts,
concepts of realities. During our discussions this became clearer.

Usually there is no understanding, and, thus, we live in a dream. Now,
when there is not direct understanding and awareness, we are dreaming.
Even when we are talking about ultimate realities, we are dreaming. But
when direct understanding arises we are not merely thinking, we are
awake just for one moment. It takes a long time to realize the true
nature of realities. Acharn explained with endless patience that ``there
is no one there''. To remind me of the truth she said: ``Where is
Lodewijk? He is no more, but also when he was still alive there was no
Lodewijk. No Lodewijk, no Nina''. Her remark helped me to see that the
Dhamma has to be applied in daily life, at this moment. She often asked
whether seeing, hearing or thinking is a person. It is not a person,
because each moment is gone completely. It is hard to accept, but it all
depends on paññā: is it sufficiently developed? We need more listening
and considering so that paññā can grow. In theory we know that person is
a concept, not a reality. But right now we cling to concepts, to
persons, as if they are real. It is beneficial to know what one does not
know yet. I am very grateful to Acharn that she untiringly, with great
compassion, explained that this moment is dhamma, not ``us''. She said
``this moment'', because only what is present can be investigated, it
arises only once and immediately it is past: ``once upon a time''.