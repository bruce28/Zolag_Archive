\part{Perseverance in Dhamma}


\chapter{Respect to the Buddha}

The venerable Bhikkhu asked Acharn Sujin: ``Where can I find a statue of the Buddha in the Foundation building so that I can pay respect to the Buddha. I went around looking everywhere for a statue, but I could not find it. In every temple all over the world one can see Buddha statues, and people attach great importance to them.''

Acharn Sujin answered: ``It may happen that people who attach importance to a Buddha statue are inclined to believe that the Buddha will protect them when they pay respect. They may be attached to ceremonies. But, the understanding of the teachings is what is most important in our life. The Buddha gave us his teachings so that we can develop understanding of all phenomena of our life and eradicate defilements completely. We actually pay respect to the Buddha at any time we understand the teachings. In the Foundation there is no a statue, but relics of the Buddha are being kept here. An image of the Buddha or his relics can remind us of his excellent qualities, of his wisdom, his purity and his compassion.''

The venerable Bhikkhu asked whether one can be sure about the relic being a genuine relic of the Buddha. Acharn Sujin answered that this does not matter. We may wonder about it and speculate about it, but what really matters is the kusala citta that pays respect to the Buddha. We can pay respect with understanding of his teachings and this will condition great confidence in the Buddha, the Dhamma and the Sangha. The venerable Bhikkhu remarked that in this world there is only a very small group of people who value the benefit of developing right understanding of the Buddha’s teachings. Acharn Sujin answered that people are inclined to follow tradition, but that it takes courage being a real buddhist. The Buddha showed the Path leading to the realization of the arising and falling away of conditioned realities. We need perseverance to continue developing this Path.

The venerable Bhikkhu showed us a book with texts he used for the performance of ceremonies that were requested by people who visited his temple. He said that people are greatly attached to rituals and that they liked to hear recitations of the Pāli texts, even though they did not understand the meaning of the texts. He recited for us the first sentence of the ``Mātikā'', the Summary of the Dhammasangani, the first Book of the Abhidhamma: ``kusala dhamma, akusala dhamma, avyākata dhamma''.

Our whole life is contained in these words. Avyākata, indeterminate, are all dhammas not included in kusala dhamma or akusala dhamma. It comprises: vipākacitta (citta that is result of kamma) and the accompanying cetasikas (mental factors), kiriyacitta (inoperative citta) and the accompanying cetasikas, rūpa (physical phenomena) and nibbāna. The simplicity of this text is very impressive, and it is so deep.

When we are sick, all kinds of rūpas and vipākacittas are arising and falling away and these are included in avyākata dhamma. Hardness or heat may impinge on the bodysense and then painful feeling may arise. Then our reactions to what we experience are included in kusala dhamma or akusala dhamma. We are worried about the sickness of others, we have many problems in life regarding our work or our relationship with others. We are absorbed in concepts and on account of these, moments of happiness and misery alternate. When it is time to depart from this life all these stories will be forgotten. Where is ``our important personality''? There are only dhammas just lasting for a moment and then gone. They are only kusala dhamma, akusala dhamma, avyākata dhamma. This can comfort us in times of sickness and misery. As Acharn Sujin often says: ``Most important in life is understanding reality. Otherwise all phenomema of life are still `I', and the cycle of birth and death will continue.''

The conversation between the venerable Bhikkhu and Acharn Sujin took place last time I visited Thailand. A group of Dhamma friends from different countries were together for a few days and we all met in the building of the ``Foundation for the Study and Preservation of Dhamma''. We had several sessions in English and after my friends had left I had more sessions in Thai.

The weekend program in the Foundation is very balanced, it comprises: the study of Suttas with commentaries, of Abhidhamma, of Vinaya and dialogues about satipaṭṭhāna. Sutta reading may seem to be easy when one only pays attention to the stories and the conventional terms such as birth, old age and death, by means of which the Buddha explained the four noble Truths and the Dependent Origination. However, without the study of the Abhidhamma which explains the details of realities, we cannot understand the deep meaning of what is contained in the sutta. All the suttas actually contain Abhidhamma and are pointing to vipassanā, the way to develop right understanding of dhammas occurring in daily life.

On Sundays, one hour is spent with the reading of the rules of the Vinaya and discussions about them. These rules pertain to the monk’s behaviour and speech in his daily life and explain the different degrees of defilements that occur and that should be discerned. Also laypeople can apply the Vinaya in their own situation. Many defilements are unnoticed, but understanding developed through satipaṭṭhāna can investigate their true nature, and this will lead to their eradication.

Several teachers are assisting Acharn Sujin and I was impressed by their thorough and detailed study of the texts and their dedication. After a full day of Dhamma discussion on Sunday with only one hour in between for luncheon, there is another session of two more hours of studying and discussing subtle points of the teachings and the consultation of the Pāli texts. All those sessions were greatly inspiring to me and they helped me to have more confidence in the Dhamma.

Khun Duangduen, with her abundant hospitality, gave a luncheon in her house as she always does on Sundays. Even while taking the food at the buffet table we could exchange observations on the difficult nature of satipaṭṭhāna, and remind each other that we should not become discouraged. Also Mom Betty Bongkojpriya and Sukinderpal who are residents of Bangkok gave a luncheon for the whole group of foreign friends. During my last days in Bangkok, Acharn Sujin and her sister Khun Jid invited us for luncheon, there was no limit to their hospitality and kindness.

At the weekend people bring flowers to be placed in front of the relics of the Buddha. They were so kind to hand them to me, so that I had many times an opportunity to kneel in front of the relics and pay respect. When Acharn Sujin also knelt next to me to pay respect, I had an opportunity to pay respect to her at her feet. With this gesture I expressed respect and gratefulness for learning through her to develop understanding of the Dhamma.

Once a month on Saturday Kunying Nopparath offers great hospitality at her house for many hours of Dhamma discussion and a luncheon. This time when Lodewijk and I attended this session the subject was the ``Discourse on the Manifold Elements'' (Middle length Sayings, no 115, P.T.S.edition). When the Buddha was staying near Sāvatthī in the Jeta Grove, he said to the monks:

\begin{quote}
``Whatever fears arise, monks, all arise for the fool, not the wise man. Whatever troubles arise, all arise for the fool, not the wise man. Whatever misfortunes arise, all arise for the fool, not the wise man.''
\end{quote}

Further on we read:
\begin{quote}

``Wherefore, monks, thinking, `Investigating, we will become wise,' this is how you must train yourselves, monks.''

When this had been said, the venerable Ānanda spoke thus to the Lord:

 ``What is the stage at which it suffices to say, revered sir: `Investigating, the monk is wise?' ''
\end{quote}

The Buddha then explained about the elements classified in different ways, about the sense-fields (āyatanas), the Dependent origination, the (causally) possible and impossible. When Ānanda asked him how the monk was skilled in the elements the Buddha first spoke about the elements as eighteenfold. We read:

\begin{quote}
``There are these eighteen elements, Ānanda: the element of eye, the element of material shape, the element of visual consciousness; the element of ear, the element of sound, the element of auditory consciousness; the element of nose, the element of smell, the element of olfactory consciousness; the element of tongue, the element of taste, the element of gustatory consciousness; the element of body, the element of touch, the element of bodily consciousness; the element of mind, the element of mental states, the element of mental consciousness. When, Ānanda, he knows and sees these eighteen elements, it is at this stage that it suffices to say, `The monk is skilled in the elements.' ''
\end{quote}

All that is real is included in these eighteen elements. In dependence on the sense objects and sense bases arise the sense-cognitions. We then read about the element of mind (mano dhātu), which includes the five-sense-door adverting-consciousness (the first citta in a sense-door process), the receiving-consciousness (vipākacitta, arising in a process after the sense-cognition of seeing. etc) and the determining-consciousness (votthapana, arising in a sense-door process just before the javana-cittas that are kusala or akusala).

As to the element of ``mental states'' (dhamma dhātu), these are the dhammas that are experienced through the mind-door: all rūpas other than the sense objects, cittas, cetasikas and nibbāna. As to the element of mental consciousness (manoviññāṇa dhātu), this includes all cittas, except the five sense-cognitions and the three kinds of cittas classified as mind-element. It includes cittas experiencing an object through six doors as well as door-freed cittas, cittas not arising in processes, namely, rebirth-consciousness, bhavanga-cittas\footnote{Bhavanga-citta or life-continuum arises in between the processes of cittas and also when we are in deep sleep and not dreaming. It arises throughout life and its function is keeping the continuity in the life of an individual. It is of the same type of citta as the rebirth-consciousness.} and dying-consciousness.

All these elements are realities of our daily life. They arise all the time but we do not realize that they are elements, devoid of self.

We read in the ``Dispeller of Delusion'' (I, Ch 3, 55): about the earth element:

\begin{quote}
As regards paṭhavīdhātu (earth element) and so on, the meaning of element has the meaning of ``nature'' (sabhāva); and the meaning of nature has the meaning of ``voidness'' (suñña); and the meaning of voidness has the meaning of ``not a being'' (nissatta)\ldots
\end{quote}

An element or paramattha dhamma has its own characteristic that cannot be changed: seeing experiences visible object, that is its nature. It arises because of its appropriate conditions and then it falls away. It is ``not a being''.

Seeing only sees what appears through the eyes, different from defining what we see. Hardness is a rūpa, but the experience of hardness is nāma. Sound appears, it is heard. There are sound and hearing, but there can be awareness of only one reality at a time. There can be another citta accompanied by paññā, and then paññā can understand sound as rūpa or hearing as nāma.

When we think of the body as a whole, we think of a concept, an idea we have about the body. In reality, what we call body is constituted by many different rūpas that arise and fall away all the time. We notice decay of the body, and we think about its impermanence, but that is thinking of an idea, a concept we have of the body as impermanent. We do not realize the falling away of each rūpa separately, in other words, the momentary impermanence, the true characteristic of impermanence. And therefore, we also fail to see the true characteristic of dukkha: what falls away each moment is no refuge, it is not worth clinging to, it is dukkha.

Before we can blink your eyes, all the rūpas of the body have already gone, from head to toe, there is nothing remaining. Our misconceptions about reality cannot be eradicated if we remain in the world of thinking, of illusions. If we realize that there are six separate worlds appearing through the six doors, one at a time, our world of illusions crumbles apart. We should remember the sutta ``The world'' (Kindred Sayings IV, First Fifty, Ch III, § 82) about the world that crumbles away. We read:

\begin{quote}
``What crumbles away? The eye\ldots objects\ldots eye-consciousness\ldots ''
\end{quote}

Sutta after sutta the Buddha explains about six classes of objects, six bases, six sense-cognitions.
The question is how can the world of illusions crumble away? Listening and reflection is the foundation of beginning to be directly aware of the dhammas that appear. When sati arises and is mindful of one dhamma at a time, understanding develops. However, when we cling to sati and try to focus on specific realities it is not the way to develop right understanding.

Awareness can arise when there are the right conditions for its arising, but we should not expect many moments of it. Awareness and understanding can be gradually accumulated. It is not enough to be aware of rūpas of the body, also nāma has to be known. Kusala and akusala have to be known, thinking has to be known, understanding itself has to be known as only a nāma, otherwise we cling to it and take it for ``my'' understanding.

The Commentary to the ``Discourse on the Manifold Elements'' explains that foolish people cling to rank, they want to obtain a high position in society. The monk wants to escape danger and leaves the home life. However, also monks have defilements, they may quarrel, such as the monks in Kosambi. We read in the Commentary that for those who consider this Dhamma Discourse, all these elements appear clearly, just as the reflection of his face clearly appears to someone using a mirror. Therefore, the Buddha said to Ānanda that this Discourse could also be remembered as the ``Mirror of Dhamma''.

We have many defilements and we are troubled by many kinds of contrarieties in our life. We think of other persons who treat us in a disagreeable way. However, we can learn that, in the ultimate sense, there is not ``me'', not the other person, but that there are only elements or dhammas. Dhammas can appear as clearly as the reflection in a mirror.

We read in the Commentary that someone who learns to develop vipassanā will overcome defilements and eradicates them completely when he becomes an arahat, just as a soldier who conquers in a battle. Therefore, the Buddha also called this sutta the ``Drum of the Deathless''. The Deathless is nibbāna, the arahat has reached the end of birth and death. This sutta refers to the development of insight: one becomes skilled in the elements through insight. We read in the Commentary that vipassanā is compared to munition used in a battle. By means of this munition one can overcome all defilements. The Buddha also called this Discourse the ``Incomparable Victory in the Battle.''

This Discourse can remind us that we have to be courageous in the development of insight so that we eventually reach the goal. The objects of insight are all the elements as explained in this sutta. These elements appear all day through the six doorways. Visible object impinges at this moment on the eyesense so that seeing can arise. If there were no citta, visible object could not appear. We may believe that we see a person, but we cannot see a person, we can only think of him. Thinking is another element.

Buddhaghosa states page after page that the truth about the khandhas, the dhātus (elements), the āyatanas (sense-fields) is taught by the Buddha. Buddhaghosa repeats that vipassanā is to be developed of the khandhas, the dhātus, the āyatanas. In this way he reminds us to be aware of them now, since they pertain to daily life. Buddhaghosa stresses that the Buddha taught our being in the cycle, vaṭṭa, and being released from the cycle, vivaṭṭa. We are in the cycle now, we are subject to dukkha.

In each Sutta the Abhidhamma and vipassanā are implied. The Abhidhamma explains ultimate realities and these are the objects of insight. People at the Buddha's time had no misunderstandings about this, but since we are further away from the Buddha's time we need more detailed explanations. The Commentaries clearly explain about the Abhidhamma and its application in the development of insight. Without the commentaries we would be lost.

We need endless patience and perseverance in the development of vipassanā. We have to consider nāma and rūpa and be aware of them so that we become familiar with their different characteristics. It has to be remembered that nāma experiences an object and that rūpa does not experience anything.

We can be inspired by people’s patience at the time of the Bodhisatta Sumedha. We read in the ``Chronicle of the Buddhas'' (Buddhavaṁsa, II a Sumedha, vs. 71-76) that the Buddha, during his life as the Bodhisatta Sumedha, was proclaimed a future Buddha by the Buddha Dīpaṅkara. We read about people’s reactions to this event:

\begin{quote}
71. When they had heard these words of the great seer who was without an equal, men and deities, rejoicing, thought: ``Sprout of the Buddha-seed is this''.

72. The sounds of acclamation went on; the (inhabitants of the )ten-thousand (world-system) with the devas clapped their hands, laughed and paid homage with clasped hands.

73. (Saying) ``If we should fail of the Dispensation of this protector of the world, in the distant future we will be face to face with this one\ldots ''

\end{quote}

The Bodhisatta Sumedha had to develop all the perfections for aeons in order to attain Buddhahood, and people at that time had great patience. They had courage and perseverance to continue developing right understanding and all the other perfections for aeons so that they would attain the goal.

When we consider how long it takes to develop paññā we can think with respect and gratefulness of the Buddha who had endless patience to develop the perfections for our sake, so that we would have the opportunity to develop understanding at this moment. Acharn Sujin said, ``When you think of the aeons it takes to develop understanding you are actually praising the Buddha’s excellent qualities.''





\chapter{The Divine Messengers}


Lodewijk and I discussed the sutta of ``The Divine Messengers'' when we were having dinner at the riverside. We spoke about the problems that arise on account of my aged father, and Lodewijk said that he was impressed by this sutta which we can immediately apply in our life with him.

In the ``Divine Messengers (Gradual Sayings, Book of the Threes, Ch IV, § 35)\footnote{I used the translation by Ven. Nyanaponika, Wheel 155-158, B.P.S. Kandy} we read that the Buddha spoke to the monks about three divine messengers: old age, sickness and death. A person who has immoral conduct in deeds, words and thoughts is reborn in hell. The warders take him and bring him before Yama the Lord (of Death). We read that they said:

\begin{quote}

``This man, O majesty, had no respect for father and mother, nor for recluses and priests, nor did he honour the elders of the family. May your majesty inflict due punishment on him.''

Then, monks, King Yama questions that man, examines and addresses him concerning the first divine messenger:

``Did you not see, my good man, the first messenger appearing among men?''
And he replies:''No, Lord, I did not see him.''

Then King Yama says to him: ''But, my good man, did you not see among people a woman or a man, aged eighty, ninety or a hundred years, frail, bent like a roof gable, crooked, leaning on a stick, shakily going along, ailing, his youth and vigour gone, with broken teeth, with grey and scanty hair or none, wrinkled, with blotched limbs?''

And the man replies, ``I have seen it, Lord.''

Then King Yama says to him: ``My good man, did it never occur to you who are intelligent and old enough, `I too am subject to old age and cannot escape it. Let me now do noble deeds by body, speech and mind’?''

``No Lord. I could not do it, I was negligent.''
\end{quote}


We then read that King Yama said that he would experience the fruit of his evil action. King Yama then questioned him about the second divine messenger:
\begin{quote}

``Did you not see, my good man, the second divine messenger appearing among men?''

``No, Lord, I did not see him.''

``But, my good man, have you not seen among people a woman or a man who was sick and in pain, seriously ill, lying in his own filth, who had to be lifted up by some, and put to bed by others?''

``Yes, Lord, this I have seen.''

``Then, my good man, did it never occur to you who are intelligent and old enough, `I too am subject to sickness and cannot escape it. Let me now do noble deeds by body, speech and mind'?''

``No, Lord, I could not do it. I was negligent.''

\end{quote}

We read that King Yama said that he would experience the fruit of his evil action. King Yama then questioned him about the third divine messenger:
\begin{quote}

``But my good man, have you not seen among people a woman or a man who had died one day ago or two, or three days ago, the corpse being swollen, discoloured and festering?''

``Yes, Lord, this I have seen.''

``Then, my good man, did it never occur to you who are intelligent and old enough, `I too am subject to death and cannot escape it. Let me now do noble deeds by body, speech and mind'?''

``No, Lord, I could not do it. I was negligent.''

\end{quote}

We then read that he had to suffer as the result of his evil deeds grievous torments in hell.

The next day, we discussed this sutta with Acharn Sujin. She said, ``Your father is a deva messenger.'' Lodewijk remarked that all that is said in this sutta is literally true with regard to our life with my father. He said that he had to lift my father from his own filth, clean him and put him to bed. King Yama’s question about whether we have seen the Divine Messengers is a pertinent reminder of the truth.
We cannot deny that we see the deva messengers, and we should remember not to be neglectful.

Acharn Sujin said that we cannot escape seeing such things in the circumstances of our life, but that it is most important to understand realities. We should develop understanding of seeing and visible object since these are realities that occur all the time. Seeing conditions thinking about what we see. We should know when we are lost in the ocean of concepts, the ocean of ignorance and clinging. If there is no understanding we are full of the idea of self.

Lodewijk said that the sutta reminds us to perform noble deeds through body, speech and mind. He asked whether there are any limits to good deeds?

Acharn Sujin said that deeds and speech depend on the citta that motivates them. When mettā, loving kindness, arises, speech and deeds will be motivated by mettā. We should not merely think about having more mettā and practising it. When we have more understanding, kusala can become purer. If we do not consider the citta that arises, we may merely think of ourselves. When we are in the company of others we may behave in an agreeable manner and speak pleasant words, but if we do not consider the citta at that moment, there is attachment to ourselves or conceit. We may have conceit and we want to be considered a good person by our fellow men.

It is a gain to know at least when we cling to an underlying notion of self, no matter what our actions are, even when we perform kusala. Acharn Sujin asked us whether it is not true that we often perform kusala for our own sake. She said, ``If one understands the teachings and there is less attachment to the self you think of the others more than of yourself. You think of helping others in deed and speech at any time.''

It may seem that other religions also teach this and that there is nothing special in her words. However, the Buddha taught the development of understanding of realities and this is the condition to become more detached from the idea of self. This understanding can inspire to help others, even when one formerly was always inclined to say: not now, it is not convenient now. As we read in the Sutta, we should do noble deeds by body, speech and mind and this includes mental development, bhāvana. Learning about the different cittas that arise and that motivate our deeds is mental development. Mental development is the study of the Dhamma, the explanation of it to others, the development of calm and the development of vipassanā.

We discussed our problems concerning our life with my father, and we mentioned that he grumbles and finds fault with us. Acharn Sujin said that kusala can be purer when we are not engaged in thinking about what others do or say. She said, ``We love him, but we should not think, does he love me? We show affection but we do not mind about the result.'' Satipaṭṭhāna can be a condition for having more mettā. When we think all the time of persons we may be partial, we may have preferences for certain persons, or we may have expectations about their attitude towards us and at such moments there is no mettā. She explained that so long as there is ``I'', the cycle of birth and death will continue.

One of our friends remarked that her father was always annoyed with her and often uttered harsh speech. She said that she tried to understand his accumulated tendencies, which causes him to be in that way. She thought of him as a giver, as a teacher. She said that a father who always grumbles can be our teacher since he reminds us to investigate more our own cittas. Patience is one of the perfections that should be accumulated so that enlightenment can be attained. Acharn Sujin said that at each moment there is an opportunity for patience, and she asked how it could otherwise be accumulated.

Lodewijk asked Acharn Sujin how we can know the latent tendencies that are accumulated in each citta from moment to moment. She answered that when they condition akusala citta we know that there are latent tendencies. Each time akusala citta arises and falls away akusala is added to the latent tendencies. Through the development of satipaṭṭhāna they are gradually weakened, but they can only be eradicated by lokuttara citta.

We have heard many times that there are three levels of understanding:
understanding stemming from listening and reading, pariyatti, understanding that is developed through awareness of nāma and rūpa, paṭipatti or practice,
understanding of the level of the direct realization of the truth, pativedha.
Sarah said that while we are reading texts we may become absorbed in them without any awareness of nāma and rūpa. Realities, nāma and rūpa, appear all the time, but mostly we are only thinking about them. We were reminded time and again by Acharn Sujin that we should know that there is dhamma at this moment, a reality with its own characteristic. If we have merely theoretical knowledge, we know only the names of realities.

One of our friends asked how we should study Dhamma. Acharn Sujin said that listening and considering are conditions for the understanding of the Dhamma. He wondered whether there are other conditions for the development of right understanding apart from listening. Can one do something else about it? Acharn Sujin explained that each moment is conditioned and that wishing to do something specific in order to have more moments of satipaṭṭhāna is only thinking, a nāma that is conditioned. Listening helps to understand conditions for each moment that arises.
We read in the Gradual Sayings (III, Book of the Fives, Ch XXI, Kimbila, § 2, ``On hearing Dhamma''):

\begin{quote}
Monks, there are these five advantages from hearing Dhamma. What five?
He hears things not heard; purges things heard; dispels doubt; makes straight his view; and his heart becomes calm. Verily, monks, these are the five advantages from hearing Dhamma.
\end{quote}

As we read in the text, he purges things heard. The Thai translation has: he clearly understands what he has heard. This means that we should not listen passively, but investigate what we hear, consider it again and again so that we gain more understanding of the Dhamma. So long as we have not attained enlightenment doubt about nāma and rūpa is bound to arise, but right understanding can eliminate doubt. We read that he makes straight his view. We have wrong view of realities, we believe that they last and we take them for self. By listening and considering paññā can grow so that there is less wrong view, our view can be straightened. We read that his heart becomes calm. The Thai translation has: the citta of the person who listens will have confidence. When there is more understanding of the Dhamma, confidence in the Buddha, the Dhamma and the Sangha will grow.

Acharn Sujin stressed again and again that we should consider whether there is dhamma now. What is dhamma? Seeing, visible object, hearing, sound, thinking of all experiences through the senses. We usually think with attachment (lobha), aversion (dosa) and ignorance (moha). Sense-cognitions such as seeing and hearing only last for one moment and then defilements are bound to arise. It seems that we recognize defilements more easily than the moments of seeing and hearing. However, it is essential to understand also the moments of seeing and hearing. It is on account what we experience through the senses that many defilements arise.

The Abhidhamma helps us to have more understanding of our life, to entangle different realities. We see only visible object, and shortly after seeing we define what we see, we remember the image of a person or thing. We cling to such images and we are neglectful in understanding the different cittas that arise and that experience different objects. We take the person or thing we perceive for reality, for something that really exists and that is lasting. This is wrong view, and wrong view is the condition for much confusion and trouble in our life.

We do not define all the time what is seen, we do not think all the time, ``this is a tree, this is a tree''. There are also moments of just seeing, no thinking. We may believe that we see a particular object, like a tree, but is there not also colour at the background? We do not have to think of tree or background, but all that is visible can be seen. Seeing is not focussing on specific colours such as red or blue. When our eyes are open many different colours appear through eyesense and there is no need to enumerate or define all these colours. They just appear through the eyesense and after that we pay attention to the shape and form of things and we know that this is a person and that an animal or tree. A person is not seen. It is impossible that a person impinges on the eyesense, how could he contact eyesense? But to apply this knowledge is difficult, because we are used to believing that we see people and things all the time. Insight can only very gradually be developed.

When we are hearing we may think of the sound of traffic, but we do not think all the time, ``this is the sound of traffic''. There are also moments of just hearing, hearing of what impinges on the earsense, of sound. Earsense is rūpa, it is ready for impact of sound, just sound, nothing else, so that hearing can arise. Sound is rūpa, hearing is nāma, they have different characteristics. Earsense is the physical base for hearing and it is also the doorway for the experience of sound. Hearing arises at the earbase. Many processes of citta occur extremely fast. When we are in conversation with others we communicate by means of the words we speak. When we hear sounds and then recognize different syllables that form up words, many ear-door processes arise and in between many mind-door processes of cittas that remember meanings. Saññā does its task of marking and remembering the object, so that we can remember a whole sentence, sequences of them and understand the meaning of what is spoken.

When someone speaks to us, there are moments of just hearing and moments of understanding of what has been said. When we translate words from Thai into English, it seems that we can do this immediately, without thinking, but in reality there are many different cittas arising in sense-door processes and mind-door processes. If there is no hearing of just sound, how could we translate anything?
It is the same in the case of reading, we actually translate what is seen into meaning. However, there is also seeing of what appears through the eyesense. Seeing is different from paying attention to the shape and form of the letters.

We need perseverance to listen to the Dhamma and carefully consider it, so that we can understand the difference between concepts of people and things, and the conditioned dhammas of our life which are citta, cetasika and rūpa. When there is more understanding of dhammas as objects of satipaṭṭhāna, the difference between concepts and dhammas will be clearer. We can learn the difference between nāma, which includes citta and cetasika, and rūpa. Nāma is the dhamma that experiences an object and rūpa is the dhamma that does not know anything.


\chapter{The Best of Sights}

The Sutta selected for our sutta reading and discussion in the Foundation building was the Bhaddaji Sutta Gradual Sayings, Book of the Fives, Ch 17, § 10). We read that Ānanda, when he was dwelling near Kosambī, in Ghosita Park, asked Bhaddaji:

\begin{quote}
``Good Bhaddaji, what is the best of sights, what the best of sounds, what the best of joys, what the best of conscious states, and what the best of of becomings?''

``There is Brahmā, sir, who is overcomer, by none overcome, he is seer of whatever may be, with power and dominion; who sees him of the Brahmās, that is the best of sights.

There are devas of radiant splendour, in whom joy flows and overflows, who ever and again utter a cry of: ‘Joy, oh joy!’ who hears that sound- it is the best of sounds.
There are the all-lustrous devas, rejoicing just in quiet, who feel joy- that is the best of joys.

There are the devas who go to the sphere of nothingness theirs is the best of conscious states.

There are devas who go to the sphere of neither consciousness nor unconsciousness\footnote{The words, ``having overcome, in this world, covetousness and grief'' are in Pali: ``vineyya loke abhijjhādomanassaṃ''.} theirs is the best of becomings''.
\end{quote}
We read that Ānanda said:
\begin{quote}

``When, while one looks, the cankers are destroyed- that is the best of sights.
When, while one listens, the cankers are destroyed- that is the best of sounds.
When, while one rejoices, the cankers are destroyed- that is the best of joys.
When, while one is conscious, the cankers are destroyed- that is the best of conscious states.
When, while one has become, the cankers are destroyed- that is the best of becomings.''
\end{quote}

This sutta explains that the attainment of arahatship is superior to all other experiences, even to the attainment to the highest stages of immaterial jhāna, arūpa jhāna, which are the sphere of nothingness and the sphere of neither perception nor non-perception. When all defilements are eradicated at the attainment of arahatship, there will be no more rebirth, no more ``becoming'' and this is to be preferred to any kind of ``becoming''.

The Commentary to this Sutta, the ``Manorathapūranī'', explains that arahatship can be attained immediately after seeing, no matter whether a desirable or undesirable object is seen. It explains that when a monk has seen visible object through the eyes, he begins to apply insight and that the attainment of arahatship can be said to arise consecutively after seeing. It states that it is the same in the case of hearing.

Acharn Sujin said: ``People think that they cannot gain understanding of the reality appearing at this moment''. Realities appear one at a time through one of the six doorways. If we believe that we can see and hear at the same time we shall not know that seeing and hearing are impermanent, dukkha and anattā. The characteristic of impermanence refers to the reality that appears now. She said: ``We may read the Tipiṭaka without any understanding of realities. We may read that all conditioned dhammas are dukkha, but we should know that these words represent reality.''

We may think about the impermanence of mind and body, but that is not insight that realizes the impermanence of citta that arises and falls away each moment, nor is it the resolution of the compact of the body into elements that are arising and falling away. Paramattha dhammas, citta, cetasika and rūpa, have the characteristics of impermanence, dukkha and anattā and these can be directly known through the development of insight.

This sutta reminds us to develop right understanding of seeing and hearing at this very moment. Seeing sees a desirable object or an undesirable object depending on the kamma that produces seeing. Seeing is vipākacitta, the result of kamma. It does not matter whether visible object is desirable or undesirable, seeing falls away immediately. Seeing just sees and it does not know whether visible object is desirable or undesirable. It is of no use to find out whether the object is desirable or undesirable.

We can learn that seeing is only a type of nāma and that it experiences visible object which is rūpa. When we understand that seeing is only a type of nāma, we begin to know the meaning of non-self. Seeing arises because kamma produces it, we cannot cause its arising or be the owner of it. It is only nāma, not ``my seeing''. Seeing sees only visible object, not a person or a thing. Sati can be aware only of one nāma or rūpa at a time, but so long as we do not distinguish nāma from rūpa we are bound to take all realities for self.

Insight develops stage by stage, so that eventually enlightenment can be attained. At the first stage of enlightenment, the stage of the sotāpanna, wrong view of self is eradicated, and at the fourth and final stage, the stage of the arahat, all defilements are eradicated. Some people at the Buddha’s time had accumulated a high degree of understanding and when they heard only a few words, when they heard about the impermanence of realities such as seeing or hearing, they could immediately penetrate their true nature and even attain arahatship.

Acharn Sujin stressed many times that the objects insight has to be developed of are very ordinary, occurring in daily life, but that attachment always distracts us from awareness and understanding of what appears at the present moment. Lobha urges us to do something different from understanding seeing or hearing that arises now. Seeing for the arahat is not different from seeing for us at this moment. However, we still have ignorance, wrong view and all the other defilements.

We see all that appears through the eyes. Seeing is a reality, a dhamma that appears. Ignorance does not know that it is dhamma. It is natural that ignorance and wrong view follow upon seeing very often. When we have a notion of ``I see persons and things'', the dhamma at that moment is not seeing but thinking of concepts. Gradually we can learn that there are many different dhammas appearing through the six doors, one at a time. This is a condition for the arising of direct awareness and understanding and it is the sure way leading to the penetration of anattā.

During lunch in Khun Duangduen’s garden we discussed realities while we were enjoying the food. I was reflecting on hardness, a reality that is experienced through the bodysense. Sometimes there can be just a moment of awareness of hardness, and I was discussing this with Acharn Sujin. She reminded me of the truth, saying, ``Even when there is awareness of hardness, there is still an idea of hardness as `mine'.'' She explained that there is an underlying tendency of ``self’, even when we believe that there is awareness. She said that lobha is always with us, that it is the second noble Truth, which is the cause of dukkha\footnote{The four noble Truths are: dukkha, the unsatisfactoriness inherent in all conditioned realities, the origin of dukkha that is craving, the cessation of dukkha that is nibbāna and the eightfold Path leading to the cessation of dukkha.}.

We may be thinking of lobha and wrong view, but only when they arise, there can be understanding of them. Paññā knows that they must be eliminated.

We read in the ``Kindred Sayings''(V, Kindred Sayings about the Truths, Ch 4, § 4, Turban) that the Buddha asked what should be done if one’s turban or head is on fire. The answer was that in order to extinguish the fire one should make extra efforts, and have mindfulness and attention. The Buddha said:

\begin{quote}

Well, monks, letting alone, paying no heed to, the blazing turban or head, for the comprehension as they really are, of the four not penetrated Ariyan Truths, one must put forth extra desire, effort, endeavour, exertion, impulse, mindfulness and attention\ldots''
\end{quote}

This sutta can remind us that we should not delay the development of understanding of all realities arising at this moment. Acharn Sujin said that lobha is attached to everything and that we are always in danger. There is as it were fire on our heads.
For the development of right understanding we do not need to go to a quiet place. We may die before we reach that place. Realities such as seeing, hearing and thinking are the same no matter where we are. All day long dhammas appear through the five sense-doors and through the mind-door, one at a time. Through the eyes visible object is experienced, through the ears sound, through the nose odour, through the tongue flavour, through the bodysense hardness, softness, heat, cold, motion or pressure. The realities that appear exhibit their own characteristics. They arise dependent on many different conditions and nobody can cause their arising. Through satipaṭṭhāna one will understand the nature of anattā of realities.

During the sessions we discussed samatha and vipassanā. Both of them are ways of mental development, bhāvana, and they cannot be developed without sati and paññā, sati-sampajañña. However, the method and aim of samatha and vipassanā are different.
We read in the ``Discourse on Expunging''(M. I, no 8, Sallekhasutta) that the Buddha said to Cunda:

\begin{quote}
``These, Cunda, are the roots of trees, these are empty places. Meditate, Cunda; do not be slothful; be not remorseful later. This is our instruction to you.''
\end{quote}

In Pāli the word ``jhāyathā'' is used, that can be translated as contemplate. The Commentary to this sutta, the ``Papañcasūdanī'', explains that there are two meanings of jhāna:
contemplation on the thirty-eight objects of samatha (aramma.nūpanijjhāna), and contemplation on the characteristics (lakkhaṇūpanijjhāna), beginning with impermanence, with reference to the khandhas, the sense-fields (āyatanas) and so on. The Commentary states: ``It is said, develop samatha and vipassanā. Do not be slothful; be not remorseful later.''

When someone attains jhāna he can be temporarily free from attachment to sense objects, but defilements cannot be eradicated by jhāna. Only the development of vipassanā leads to the eradication of defilements.

When someone develops samatha, sati-sampajañña should realize the difference between kusala citta and akusala citta, and he should realize when akusala citta with attachment arises. Sati-sampajañña is needed and this means, that he should not only have theoretical understanding of kusala and akusala, but that he should be aware right at the moment when attachment arises. He should also be truly motivated to be free from sense objects and to subdue attachment to them. This means that he should lead a life of contentment with little, without any entertainments, without indulging in all the pleasant things of life. If he is not a monk, his life-style should still be similar to the monk’s way of life.

The kasinas, disks, are among the meditation subjects of samatha and these can remind us of the goal of samatha. We discussed the Earth Kasina, one of the meditation subjects of samatha. One of the meanings of kasina is ``entire'' or ``whole''. The kasina disk or circle includes all. In the case of the earth kasina, we can be reminded that all we are attached to is just earth. When we are eating, when we go shopping and buy clothes, or when we use table and chair, we are clinging to earth. The element of earth represents solidity, hardness or softness. This element is present in all sense objects. One cannot attain calm by merely looking at a kasina and concentrating on it. There should be right understanding of this mediation subject and of the goal of its development.

We read about people in the Buddha’s time who could develop all stages of jhāna and then developed vipassanā. For them samatha was the foundation or proximate cause for insight. However, in order to be able to do so they had to acquire ``masteries'' (vasī) of jhāna. We read in the ``Visuddhimagga'' (IV, 131): ``Herein, these are the five kinds of mastery: mastery in adverting, mastery in attaining, mastery in resolving (steadying the duration), mastery in emerging, and mastery in reviewing.'' This means that jhānacitta had become very natural to them and could arise at any time. Therefore, it could become an object of insight. In order to eradicate defilements completely, insight has to be developed. The person who could attain jhāna had to be aware of the cetasikas which are jhānafactors: applied thinking (vitakka), sustained thinking (vicara), zest (pīti), happy feeling (sukha) and concentration (ekaggata cetasika)\footnote{The jhānafactors are the sobhana (beautiful) cetasikas that should be developed so that absorption, jhāna, can be attained.}.

We read in the scriptures that there were disciples who developed jhāna and vipassanā and disciples who developed only vipassanā.

The objects of vipassanā are whatever reality appears in daily life through one of the six doorways. Vipassanā is developed in stages, and in the beginning of its development, the different objects and doorways are not clearly separated. One confuses visible object which is rūpa and seeing which is nāma. It seems that we can see and think or see and hear at the same time. When the first stage of vipassanā is reached the characteristics of nāma and rūpa are clearly distinguished and appear as such through the mind-door. So long as we have not reached that stage we do not know precisely what the mind-door is. Nāma is experienced through the mind-door and rūpa is experienced through its appropriate sense-door and after that through the mind-door. During the moments of insight it is known what the mind-door is.

When vipassanā is developed it is also accompanied by samādhi, concentration. Samādhi or ekaggatā cetasika, one-pointedness, arises with each citta and its function is focussing on one object. Usually its characteristic does not appear. Samādhi becomes stronger as paññā develops. At the moment of insight knowledge, there is momentary concentration, khanika samādhi. Samādhi performs its function of concentration of the eightfold Path. When vipassanā is further developed, dhammas can be realized as they are, as impermanent, dukkha and non-self.

We read in the ``Kindred Sayings'' (IV, Kindred Sayings on Sense, Ch 2, § 71) that the Buddha said that whoever of the monks could not understand the arising and destruction, the satisfaction and the misery of the sixfold sphere of contact, did not live the righteous life and was far from this Dhamma and Discipline. One of the monks said that he could not understand these things and was in despair. The Buddha said to him:
\begin{quote}
``Now, what do you think, monk? Do you regard these thus: `This is mine. This am I. This is myself' ?''

``No indeed, lord.''

``Well said, monk. And herein, monk, by right understanding as it really is: `This eye is not mine. This am I not. This is not myself,' the eye will have been rightly seen. That is the end of dukkha. So also as regards mind\ldots That is the end of dukkha.'
\end{quote}

The sixfold sphere of contact are the āyatanas: the five sense organs and the five classes of objects experienced by means of them and also mind-base or consciousness (including all cittas) and mind-object (dhammāyatana). Mind-object includes subtle rūpas (other than the sense organs and sense objects), cetasikas and nibbāna. Āyatanas are realities that contact each other, such as eye-sense and visible object, they are the conditions for the arising of seeing. Therefore the word phassāyatana is used: sphere of contact. Phassa is contact.

We read in the Commentary to this sutta, the ``Sārattappakasinī'' that the monk who did not understand these ayatanas felt that he was lost, far away from the Dhamma. The Buddha was thinking what kammathāna, meditation subject, would be helpful for that monk who had no energy for the kammathānas of the elements (dhatus), kasinas or other subjects. He thought that the kammathāna of the āyatanas would be a support, a helpful condition (sappaya) for that monk. He then asked whether the eye is I, mine or myself, etc. And so on for the other āyatanas.

The term ``meditation subject'', kammathāna, is used for the meditation subjects of samatha, but also for the objects of vipassanā, and these are all objects appearing through the six doorways. The Buddha spoke to that monk about the objects of vipassanā.

There are objects impinging on the six doors time and again but most of the time there is forgetfulness of realities. Sometimes there is mindfulness of the reality which appears and then we may notice that such moments are different from our usual forgetfulness. Acharn Sujin said: ``Nobody can force the arising of sati. It can arise unexpectedly, but if one tries to force its arising there will not be right understanding of realities. When it arises because of its own conditions, one will know that it is sati, not ‘me’.''

Gradually we can learn the difference between forgetfulness and awareness of realities. Although there is not yet clear understanding of realities, there can be a beginning of noticing or ``studying'' with awareness different nāmas and rūpas which appear. Are we patient enough to study with awareness all the details of our daily life? We may not like it to be aware of unpleasant feeling, or we may not find it interesting enough to know seeing which appears now or hearing which appears now. If there is patience there can be careful consideration of the Dhamma and there can be a beginning of understanding of the present reality. The only way to have less ignorance is mindfulness of the reality which appears now, even if that is ignorance or unawareness. Courage, perseverance and patience are indispensable for the development of right understanding.



\chapter{Listening}

The development of right understanding of dhammas that appear in our life begins with listening. During the Dhamma sessions in the Foundation building with my English speaking friends as well as with my Thai friends there were many opportunities for listening. Moreover, Acharn Sujin’s lectures are broadcasted every day and many people listen to them early in the morning and in the evening. When we listen, we hear what is new to us and also what we heard before but had not yet completely understood. We should carefully consider what we hear so that it goes into our bones. Listening, considering and studying the Dhamma is kusala kamma included in mental development, bhāvanā.

We read in the Commentary to the ``Dialogues of the Buddha'', the Sumaṅgalavilāsinī, about the Buddha’s daily routine. After his breakfast the layfollowers who had offered food would come to see him. The Buddha, with great compassion and due consideration for their different dispositions, would teach them Dhamma so that they would become established in the three refuges and in sīla or even attain enlightenment.

Life is short and we should not miss any opportunity to listen to the Dhamma and carefully consider what we heard. The opportunities to hear the true Dhamma are rare. In some lives there may be no opportunity. We read in the ``Middle Length Sayings'' (II, no 81, Discourse on Ghaṭikāra) that at the time of the Buddha Kassapa the potter Ghaṭikāra tried to persuade his friend Jotipāla to visit the Buddha, but that Jotipāla reviled and abused the Buddha. It is explained in ``Milinda’s Questions'' (Dilemmas V, no 6) that the Buddha, when he was in a past life Jotipāla, was born into a family of worshippers of Brahmā, without confidence in the Buddha Kassapa. A simile is used of a great blazing fire that when it is in contact with water becomes cool and black. We read that Nagasena explained to King Milinda:

\begin{quote}
``Even so, sire, (though) the brahman youth Jotipāla had wisdom and faith and was widely renowned for his knowledge, yet when he came back to birth in a family that was of little faith, not believing, then did he, in the manner of his family, being blind, revile and abuse the Tathāgata.''
\end{quote}

We read in the Sutta that finally the potter Ghaṭikāra took hold of Jotipāla by his hair and brought him to the Buddha. Jotipāla listened to the Buddha, gained confidence in him and received ordination in his presence.

Acharn Sujin said, ``Listening helps to understand conditions. No matter where we are or where we go, each moment is conditioned. Also listening, considering and understanding are conditioned.'' The Bodhisatta Jotipāla had in many previous lives listened to former Buddhas and developed understanding. His accumulated perfections were like a bright fire, but when he was born in a family with worshippers of Brahma, this bright fire was temporarily reduced to black coal. When he met the Buddha Kassapa the development of understanding in his past lives were the right conditions for him to gain confidence in the Dhamma. Acharn Sujin reminded us time and again that whatever appears now through one of the six doorways is dhamma. We should ask ourselves to what extent we understand the meaning of dhamma. We have to listen again and consider again what we hear, in order to understand that everything is dhamma. There is dhamma right now, it has arisen because of its appropriate conditions. It could not appear if it had not arisen. If we do not understand what dhamma is, it is useless to study the Tipiṭaka. Seeing is dhamma, it is a reality that has its own characteristic and that cannot be changed into something else. Seeing sees what appears through the eyesense. Anger is dhamma, it has its own characteristic, it cannot be changed into attachment. When we are angry we usually think of a disagreeable person, but we should know that a person is a concept. Thinking is dhamma but the object of thinking is an idea or concept. When a dhamma appears through one of the six doors understanding of its characteristic can be developed and there is no need for words. The dhamma that arises does so because of its own conditions and it has no owner. It is non-self and it does not belong to a self.
All nāmas and rūpas that appear are dhammas. We read in the Abhidhamma as well as in the Suttas about attachment, aversion and ignorance, about kusala and akusala, but the Abhidhamma classifies nāma and rūpa fully and in detail. The prefix ``abhi'' of Abhidhamma means great, pre-eminent, refined, or in detail. The Abhidhamma is of great benefit because it helps us to understand the akusala cittas and kusala cittas arising in our life, and the ways they are conditioned. This again helps us to realize that they are anattā.

We read in the ``Kindred Sayings'' (III, Khandhā-vagga, Kindred Sayings on Elements, Ch 2, § 15, What is impermanent):

\begin{quote}
At Sāvatthī...Then (the Exalted One) said:-

``Body, brethren, is impermanent. What is impermanent that is suffering. What is suffering, that is void of the self. What is void of the self, that is not mine, I am not it, it is not my self. That is how it is to be regarded by perfect insight of what it really is\ldots''

The same is said of the khandhas of feeling, perception, the activities and consciousness.
\end{quote}

We may contemplate the three characteristics inherent in all conditioned dhammas, but through the Abhidhamma we can gain a deeper understanding of the meaning of the three characteristics as explained in this Sutta. We learn through the Abhidhamma that seeing arises and falls away within a process of cittas, that it is preceded and followed by other cittas that succeed one another very rapidly. After seeing has fallen away kusala cittas or akusala cittas arise. Nobody can change the order of the cittas that succeed one another. What we take for our body are rūpas originated from different factors, and they arise and fall away. What is impermanent is of no refuge, and thus it is dukkha, unsatisfactory. The Abhidhamma can be a foundation for the development of insight that leads to the direct understanding of the three characteristics of conditioned realities.

The Abhidhamma describes what is occurring in our life now: seeing, visible object, like or dislike of what is seen. This is also described in the Suttas, there is actually Abhidhamma in all the Suttas. The Abhidhamma and the Suttanta teach us about the dhammas that appear, they point to insight. Insight is the development of direct understanding of the dhamma appearing now, be it seeing, visible object, hearing, sound, or any other dhamma appearing through one of the six doors. Everything that is real is dhamma. We hear the sound of a car or of a bird and we immediately think about concepts such as car or bird. Sound is real, it impinges on the earsense. It has a characteristic that can be directly known when it appears. Car or bird are only objects of thinking, they are concepts.

When we listen to the Dhamma\footnote{Dhamma with a capital D has several meanings. It can mean: supramundane (lokuttara) Dhammas, that is, nibbāna and the eight lokuttara cittas that experience nibbāna. It also means the teachings of the Buddha. As to the term ``dhamma'', without a capital, this refers to paramattha dhamma, ultimate reality, which is different from a concept.} we learn about ultimate realities: citta, consciousness, cetasika, mental factors accompanying citta, and rūpa, physical phenomena. They are conditioned dhammas, they arise each because of their own conditions and then they fall away. Nibbāna is the paramattha dhamma that is unconditioned, it does not arise and fall away.

We cannot direct the conditions for the phenomena that arise, but we can develop more understanding of the fact that whatever arises is conditioned. Knowing that there are conditions for akusala helps not to keep on reproaching ourselves: I should not have akusala. There is also conceit when we think: `` `I' should be better, I am too good to have akusala''.

We could not develop understanding without listening to the Dhamma, because the Dhamma is the Buddha’s teaching. Nobody else taught the three characteristics of all conditioned dhammas: their nature of impermanence, dukkha and anattā. When we have listened to the Dhamma and we carefully consider what we heard, we can develop understanding ourselves without being dependent on another person.

We should investigate the different dhammas that appear in daily life, and this is the condition for beginning to be aware of their characteristics. One of my friends in Bangkok remarked that there is no other way to develop understanding, and that if we think that there is, it is just lobha, attachment.

We read in the ``Gradual Sayings'' (Book of the Fives, Ch XVI, § 4, The confounding of Saddhamma) that the Buddha said to the monks:

\begin{quote}

Monks, these five things lead to the confounding, the disappearance of Saddhamma. What five?

Herein, monks, carelessly the monks hear Dhamma; carelessly they master it; carelessly they bear it in mind; carelessly they test the good [the meaning] of the things borne in mind; knowing the good and knowing the Dhamma, carelessly they go their ways [practise] in Dhamma by Dhamma.

Verily, monks, these are the five things that lead to the confounding, the disappearance of Saddhamma.
\end{quote}

We read that if the monks carefully listen to the Dhamma, consider it and apply it, it leads to the stability of the Saddhamma (true Dhamma), to its being unconfounded, to its non-disappearance.

In this life we still have the opportunity to hear the Dhamma, to learn about realities. This sutta reminds us to study the Dhamma with the greatest care and respect and to verify what we learn in our life. Only a Buddha can teach about what is true in the ultimate sense: citta, cetasika, rūpa and nibbāna. We see people, trees, houses and many different things. Do we know what dhamma, reality, is? Seeing is dhamma, visible object is dhamma, thinking is dhamma. We have to practise the Dhamma in order to reach the goal. We have to apply what we learn by being mindful and by developing understanding of nāma and rūpa. It seems that we see a person who is lasting, who stays alive, at least for some time, whereas in reality it is only visible object that is present for a moment and then no more. When we close our eyes, the world and all the people in it do not appear. Our thinking of the world is conditioned by the seeing of visible object. We should remember the momentary death of each dhamma that appears, but we are blinded by ignorance. What we call death at the end of our lifespan is the falling away of the last citta in that life and the breaking up of the body. The length of our life is conditioned by kamma, and nobody can cause his life to last forever. This can help us to accept the death of persons who are near and dear to us. Through the development of vipassanā one comes to understand that in the ultimate sense there is death at each moment: at each moment nāma and rūpa are arising and then falling away completely. Each citta lasts only for one moment, one moment of experiencing an object through one of the six doors. It takes very long, even many lives, to realize this. Dhamma is not a medicine one can quickly apply so that it helps immediately. But we can at least begin now. That is the application of the Dhamma: right understanding of what appears.

Many moments in a day we are forgetful of nāma and rūpa, we are absorbed in thinking of concepts, such as people and events. But when we have listened to the Dhamma there may be conditions for a moment of awareness and understanding of one reality at a time that appears.

Sati is a sobhana (beautiful) cetasika that arises only with sobhana citta. Sati is non-forgetful, heedful, of what is wholesome. There are different levels of sati: sati of dāna is non-forgetful of generosity, sati of sīla is non-forgetful of abstaining from evil, sati of samatha is mindful of the meditation subject. Sati of the level of insight is mindful of a paramattha dhamma so that understanding of it can be developed. Its object is the citta, cetasika or rūpa that appears at the present moment. The aim of the development of insight is knowing dhammas as they truly are, as impermanent, dukkha and anattā.

For the development of insight, vipassanā, we should first of all know that each citta experiences only one object at a time. Seeing experiences only colour, hearing experiences only sound. Hearing does not experience words nor does it know the meaning of words, those are cittas that are thinking, different from hearing. When there are the right conditions for the arising of sati it can be aware of any reality that appears, also of akusala. Sati does not arise in the same process of cittas as the akusala cittas, but after the akusala cittas have just fallen away, the characteristic of akusala dhamma can be object of sati and at that moment understanding of it can develop.

We read in the ``Satipaṭṭhāna Sutta'' (M I, 10, as translated with its Commentary by Ven. Soma), that the Buddha said:
\begin{quote}
Here, bhikkhus, a bhikkhu lives contemplating the body in the body, ardent, clearly comprehending (it) and mindful (of it), having overcome, in this world, covetousness and grief\ldots 
The same is said with regard to the contemplation of feeling, of cittas and of dhammas.
\end{quote}

It is said in the Commentary that the ``world'' means the five khandhas, and these are conditioned dhammas that can be objects of right understanding. As to the words covetousness and grief, the Commentary explains that these words stand for the two hindrances of sensuous desire and ill-will. The five hindrances are sensuous desire, ill-will, sloth and torpor, restlessness and worry, and doubt. By samatha all the hindrances are temporarily subdued. By the development of vipassanā the hindrances are completely eradicated. We have to remember that those who could develop jhāna and afterwards developed vipassanā, had to be mindful of all nāmas and rūpas that appeared, including the jhānacittas.

We have to listen and to consider the Dhamma again and again, so that we have right understanding of the way to develop insight and of the objects of insight. Wrong view can mislead us very easily. At the moment of right awareness, desire for the object that appears or aversion towards it can be overcome. We may like certain objects to be the objects of awareness or we want to exclude other objects such as defilements, but then we are not on the right Path. When, for example, desire presents itself and there is no prejudice with regard to the object, sati sampajañña (sati and paññā) may arise and it may have desire as object, so that it can be seen as only a nāma that is conditioned. If we do not understand desire, if we do not see it as it is, it cannot be eradicated. At the moment of right awareness of a dhamma that appears there is wise attention to the object and the citta is kusala citta accompanied by paññā. We may dislike having desire but at the moment of awareness aversion towards it has been overcome and understanding of it is developing. When the hindrances are objects of right understanding, they are no longer hindrances. In this way covetousness and grief can be overcome.

We should understand that every reality that appears has arisen because of its own conditions, and that nobody creates it. Also sati should be realized as only a conditioned nāma, as dhamma, not self. We should not be discouraged if sati does not often arise, we should actually not think about the arising of sati. It is the development of understanding that is essential. We listen to the Dhamma now and we begin to understand what is dhamma. This is the result of aeons of accumulating understanding and from now on we have to continue to develop understanding for many more lives. Even one moment of understanding is beneficial since it is accumulated and can become fully developed.

The goal of the eightfold Path is detachment and we should not forget that each moment of paññā is accompanied by detachment, alobha, a root arising with each kusala citta. For example, when seeing appears it can be understood as only a conditioned nāma, non-self. At that moment there is a degree, be it very slight, of detachment from the concept of self. We read in the ``Verses of Uplift'' (Minor Anthologies, I, 10) that Bāhiya asked the Buddha for a Dhamma Discourse. The Buddha said to him:

\begin{quote}
``Then, Bāhiya, thus must you train yourself: In the seen there will be just the seen, in the heard just the heard, in the imagined just the imagined\footnote{Imagined stands for mutta, meaning, what is experienced through the other senses.}, in the cognized just the cognized. Thus you will have no ‘thereby’. That is how you must train yourself. Now, Bāhiya, when in the seen there will be to you just the seen, in the heard just the heard, in the imagined just the imagined, in the cognized just the cognized, then, Bāhiya, as you will have no ‘thereby’, you will have no ‘therein’. As you, Bahiya, will have no ‘therein’, it follows that you will have no ‘here’ or ‘beyond’ or ‘midway between’. That is just the end of Ill.''
\end{quote}

Bāhiya listened to the Buddha and truly considered his words. He developed insight so that he could clearly understand the characteristic of seeing as dhamma. He understood that seeing is only seeing, that there is nobody who sees.
We read in the Commentary, the ``Paramatthadīpanī, the Udānaṭṭhakathā, by Dhammapāla\footnote{Translated by P. Masefield.}, about the three kinds of full understanding (pariñña) which include the successive stages of vipassanā. The person who develops vipassanā sees that there are merely dhammas, occurring in dependence on conditions. We read:

\begin{quote}

``\ldots there is, in this connection, neither a doer nor one who causes things to be done, as a result of which, since (the seen) is impermanent in the sense of being non-existent after having been, dukkha in the sense of being oppressed by way of rise and fall, not-self in the sense of proceeding uncontrolled, whence the opportunity for excitement and so on with respect thereto on the part of one who is wise?''

\end{quote}

As to the words ``you will have no ‘here’ or ‘beyond’ or ‘midway between’ '', this refers to the end of rebirth. We read that Bahiya attained arahatship. After having been attacked by a calf he passed finally away.

\chapter{Courage}

The objects of sati and paññā in the development of vipassanā are paramattha dhammas, not concepts. The paramattha dhammas that appear now are the objects of understanding. They are citta, cetasika and rūpa.

Acharn Sujin said: ``Dhamma is different from concepts. When we are thinking, what is true in the ultimate sense: the story we think of or the reality that is thinking? We can see the distinction between reality and concept. There is something that can be touched, do you have to give it a name?''

Tangible object can be directly experienced through the bodysense and there is no need to name it. Hardness and softness are the characteristics of the Element of Earth or solidity; heat and cold are the characteristics of the Element of Fire or heat; motion or pressure are the characteristics of the Element of Wind or motion. These are rūpa elements, each with their own characteristic that cannot be changed, no matter how one names it. They are dhammas that are devoid of self.
Acharn Sujin said: ``Kusala dhamma, akusala dhamma and avyākata dhamma (indeterminate dhamma, neither kusala nor akusala) have to be known now, not by thinking about them. Kusala dhamma is not akusala dhamma. Kusala dhamma has its own characteristic which cannot be changed into akusala dhamma. Both of them are citta and cetasika.''

Citta, cetasika and rūpa are realities, not concepts. They are not abstract ideas, they occur now in our life.

What is kusala dhamma is always kusala, that is its characteristic, it cannot be changed into akusala. This is a way to know what paramattha dhamma is: it has its own characteristic that cannot be changed into something else, no matter how we name it.
There are many shades and varieties of kusala, depending on the accompanying cetasikas, but nevertheless kusala is kusala. Each kusala citta is accompanied by non-attachment and non-aversion, and it may be accompanied by paññā, understanding. The Abhidhamma teaches clearly that there cannot be any selfishness at the same time as kusala citta. There are different kinds of kusala: generosity consisting in giving useful things to others. But generosity is also appreciation of the kusala citta of others. At such a moment one has non-attachment, non-aversion, and one is without jealousy. At the moment of kusala citta there is peace, freedom from akusala. Wholesome conduct through body and speech, for example in helping others or paying respect to those who deserve it, is kusala sīla. Mental development, comprising the study and explanation of the Dhamma, the development of samatha and vipassanā, is another form of kusala. We can learn the characteristic of kusala when it appears, we can learn that it is different from akusala dhamma.

Attachment to our own kusala can arise immediately in between the kusala cittas. The Abhidhamma is of great benefit because it teaches that there are many different processes of cittas, some with kusala cittas and others with akusala cittas, succeeding one another extremely rapidly. All these cittas are arising because of different conditions, there is no person who owns them. The understanding of the truth of paramattha dhammas will lead to detachment from the wrong view of self.
We need guidance of the suttas but also of the Abhidhamma. Otherwise we do not learn about the fine distinctions between different moments such as kusala dhammas and akusala dhammas. And if we do not know anything about processes of cittas we have no idea how and when there can be awareness of even akusala dhammas. We would not know how the characteristic of akusala can still be the object of sati and paññā when akusala citta has just fallen away. Kusala citta cannot arise at the same time as akusala citta, but, cittas succeed one another extremely quickly. Kusala citta with awareness can arise very closely after the akusala citta has fallen away, and it can still be aware of the characteristic of akusala. Through the Abhidhamma we learn more about the conditions of the dhammas that arise and this is of immense benefit for the understanding of anattā, no possessor of anything, no self who can steer or manipulate anything.

The Buddha taught that all conditioned dhammas are impermanent, dukkha and anattā. These are the three general characteristics of conditioned dhammas that will be clearly penetrated through insight. However, vipassanā is developed stage by stage. The three characteristics are characteristics of nāma and of rūpa, they are not abstract entities. However, before someone can realize the arising and falling away of realities, thus, impermanence, paññā has to directly understand which nāma or which rūpa has arisen and appears and then falls away. In countless suttas the Buddha explained about all the objects experienced one at a time through the six doors. He explained about seeing, visible object, hearing, sound, feeling, and thinking which arise all the time. These dhammas have each their own characteristic. First the specific characteristics of nāma and rūpa have to be realized before they can be known as impermanent, dukkha and anattā.

Seeing, visible object, cold, hearing, these are all dhammas that each have their own characteristic. When we feel cold there are nāma and rūpa, but we do not distinguish nāma from rūpa; we have an idea, a concept of ourselves feeling cold. In reality cold impinges only on one point of the body at a time, but we join different moments of experiences together into a whole and then we have the impression of feeling cold all over the body. When we are aware of one nāma or rūpa at a time, without thinking, without trying to focus on specific realities, understanding will develop. When the first stage of tender insight, which is only a beginning stage, arises, paññā directly penetrates the characteristic of nāma as nāma and of rūpa as rūpa, without having to name them nāma and rūpa.

Nāma has to be known as nāma, an element that experiences an object, and rūpa has to be known as rūpa, an element that does not know anything. They appear one at a time through the six doorways. Intellectual understanding of nāma and rūpa should be correct, and in this way it can be the foundation for direct understanding. Because of our ignorance of realities we believe that we can see and hear at the same time, feel hardness of the table and see a table at the same time. Through the study of the Abhidhamma we learn that each citta experiences only one object at a time. Seeing experiences only colour, hearing experiences only sound. Hearing does not experience words nor does it know the meaning of words, those are cittas different from hearing. Evenso, the citta with sati and paññā only experiences one object at a time. Any reality that appears, be it seeing, colour, attachment or aversion, can be the object of sati and paññā. We are bound to take thinking for direct awareness. First we have intellectual understanding of realities, but when there are the right conditions there can be direct awareness without thinking. Then we shall know the characteristic of sati that is directly aware.

Acharn Sujin said: ``We may just think of the words, nāma experiences and rūpa does not know anything. However, paññā has to grow so that it can realize the different characteristics of nāma and rūpa as not a person. Then it can realize one characteristic at a time, it can realize that this is a reality which experiences and that is a reality which does not experience anything.''

What we take for ourselves or a person are only different nāma-elements and rūpa elements arising because of their appropriate conditions. Elements, dhātus, are realities devoid of self. U Narada wrote in his Introduction to the translation of Dhātu-Kathā, the third Book of the Abhidhamma (PTS:Discourse on Elements):

\begin{quote}
``The elements are not permanently present. They arise to exhibit their own
characteristic natures and perform their own characteristic functions when
the proper conditions are satisfied, and they cease after their span of
duration. Thus no being has any control over the arising and ceasing of
the elements and they are not at his mercy or will however mighty and
powerful he may be. In other words, the elements have no regard for
anyone, show no favour to anyone and do not accede to the wishes of
anyone. They are entirely dependent on conditions\ldots

For example, when the four conditions: a visible object, the sense of
sight, light and attention, are present, the eye-consciousness element
arises. No power can prevent this element from arising when these
conditions are present or cause it to arise when one of them is absent.''
\end{quote}

The Abhidhamma helps us to understand the meaning of anattā, non-self, in our daily life. The wrong view of self gives rise to many other defilements, it causes sorrow. So long as wrong view is not eradicated, the other defilements cannot be eradicated.
Acharn Sujin said: ``Do not forget anattā. At this moment there is nobody. We have heard the word anattā and now it is time to know that this very moment is anattā. The dhamma that appears does not belong to anyone, it arises because of conditions. If it had not arisen it could not appear and be known. It falls away instantly. `` She explained that one may go to another place and try very hard to develop understanding, but that it is essential to understand the dhamma appearing at this very moment. We may think about the names of realities, but this is different from ``studying'' with awareness, ``studying'' the characteristic of the dhamma that appears without the need of words. However, the notion of self is likely to occur after a moment of awareness. Acharn Sujin said: ``A moment of kusala citta with sati is very short and lobha follows instantly. Lobha does not let go of the object. If we do not understand that lobha is the second noble Truth, the origination of dukkha, it is impossible to eliminate lobha. Lobha is our teacher and follower, it never goes away. Paññā can realize the characteristic of lobha when it appears. It realizes that each reality has its own characteristic and that it is conditioned.''
We read in the Kindred Sayings (IV, Kindred Sayings on Sense, Ch 5, § 150, Resident Pupil) that the Buddha said to the monks:

\begin{quote}
Without a resident pupil, and without a teacher this righteous life is lived.

A monk who dwells with a resident pupil or dwells with a teacher dwells woefully, dwells not at ease.

And how, monks, does a monk who has a resident pupil, who has a teacher, not dwell at ease?

Herein, monks, in a monk who sees visible object with the eye, there arise in him evil, unprofitable states, memories and aspirations connected with the fetters.
Evil, unprofitable states are resident, reside in him. Hence he is called ``co-resident''. They beset him, those evil, unprofitable states beset him. Therefore is he called ``dwelling with a teacher''.
So also with the ear\ldots the tongue\ldots the mind\ldots
\end{quote}

The opposite is true for him who is without a resident pupil and without a teacher.

When a monk had a pupil he would be his co-resident, he would live with him all the time. Lobha and the other defilements are like a resident pupil one lives with continuously, or they are like a teacher who tells one to act in the wrong way. This sutta points to the development of satipaṭṭhāna, reminding us not to be neglectful, but to develop right understanding of visible object, sound and the other objects that present themselves through the six doorways.

The development of the eightfold Path is not just observing or noticing realities. This might be done with lobha or an idea of self who observes.

The development of the Path is detachment from the beginning to the end. When we listen to the Dhamma, the goal should be detachment from wrong view. Also, when we listen, there can be a degree of detachment from our wrong ideas about a self. Acharn Sujin said that by studying visible object with awareness one will have less attachment to visible object as a real being. In the ultimate sense there are only nāma and rūpa arising and falling away. She said: ``One doesn’t have to hurry to get rid of all attachment and aversion because this is impossible. Our goal is to get rid of wrong view and ignorance. One will have lobha and dosa as usual, but they can decrease when right understanding is developed. The Buddha’s teachings are concerned with one thing: developing understanding, because all dukkha comes from ignorance. It has ignorance as its root by clinging to reality which changes all the time. It seems that we can control life, but realities arise by conditions. You don’t want pleasant things to change, but they change all the time. You don’t want to get old, but you are getting old all the time. You don’t want to part from things or people, but one day even this body will be scattered about.

So we live very temporarily in one moment and we do not know what the next moment will be like.''

Through the development of insight we learn the difference between what is real in the ultimate sense, a paramattha dhamma, and a concept (paññatti) that is only an object of thinking.

On may have doubts whether it is beneficial to know, when looking at a flower, that this is colour, that only colour is seen, that colour and seeing are paramattha dhammas, and thinking of a flower is thinking of a concept. The truth of dhammas is deep and difficult to understand. The development of understanding of the truth can lead to a lessening of defilements. Defilements cause sorrow. It is beneficial to understand what dhamma is, to know that dhamma is different from a concept. But, understanding must be developed very naturally, not in a rigid way. We should not try to separate mindfulness of dhammas and daily life. We may be discouraged that paññā and sati seldom arise and that we are thinking of people, things and events most of the time. But thinking is a reality, and if we do not realize it as such we take it for self.

The development of vipassanā is not a matter of focussing on specific objects and avoiding to think of concepts. We do not have to force ourselves to pay attention only to paramattha dhammas, that would be unnatural. Our life is full of concepts of persons and events, but it is most valuable to have more understanding of paramattha dhammas in the midst of life. We should not reject paying attention to concepts such as persons and events, they are part of our daily life. The Middle Way that is taught by the Buddha is the development of understanding of our own life, of our accumulations and inclinations as they naturally arise. Otherwise, the goal cannot be reached.

Unknowingly we may be motivated by lobha to know only nāma and rūpa, and to avoid thinking of concepts, although we have realized in theory that this will only counteract the development.

The persons we meet, the events that occur, our reactions to them with kusala cittas or akusala cittas, all that occurs in daily life can remind us of paramattha dhammas. Citta, cetasika and rupa are arising and falling away all the time. They are paramattha dhammas, they are within us and they are everywhere in our surroundings. They appear in daily life at this moment. If we could only let ourselves be reminded of them in whatever circumstances we are, then we can profit to the full of the Dhamma we learnt. The Abhidhamma teaches us what paramattha dhammas are and we can apply the Abhidhamma in the development of satipatthana, since the objects of sati and paññā are paramattha dhammas.

The Abhidhamma can teach us to develop understanding naturally. Attachment and sadness may overwhelm us, but they can be known as only conditioned dhammas. They are cetasikas that are not ours. Sati and understanding are cetasikas performing their functions.

Understanding is the foremost factor of the eightfold Path and it gradually develops. Right effort is another factor of the eightfold Path that performs its function of persevering with the development of right understanding. It is not ``me'' but a cetasika, viriya cetasika. Viriya, energy or effort is the quality of a heroe, vīro. We have to be heroic to be aware of the reality of this moment, so that we eventually cross the flood of the cycle of existence. That means that we should never lose courage, even when sati of satipaṭṭhāna does not often arise. Understanding is surely growing, even when we do not see much progress.

Acharn Sujin encouraged us with the words: ``Why can't we be brave enough to understand reality at this moment?\ldots If we are not brave enough lobha will turn us away from this moment and hope for the next moment. We can begin to see how lobha dominates our life from moment to moment, and even from life to life. People think that it is impossible to be aware in times of misery or happiness, but we should be courageous.''


































\chapter{Everything is Dhamma}


Acharn Sujin reminded us time and again that everything is dhamma. Gradually these words have become more meaningful to us. We have many moments of ignorance and forgetfulness of dhammas, but this should not discourage us. We shall come to understand that also such moments are conditioned.

Acharn Sujin explained that we may notice that we have attachment, lobha, but that this is different from the actual moment of being aware of the characteristic that is attached. She said: ``Even when we can tell that we have lobha, there is still the idea of my lobha\ldots We cannot do anything because it has arisen already. We should understand that each reality that appears has arisen and that nobody created it.'' So long as we are only thinking about realities we are drowning in the ocean of concepts. When one of our friends asked her what we can do, she answered:

``We should understand more deeply the word dhamma or element, dhātu, as non-self. We should develop understanding based on hearing, studying and considering, so that there are conditions for the arising of right awareness. But awareness will not arise because of our intention to be aware.''

So long as we do not clearly distinguish the characteristic of nāma from the characteristic of rūpa, we shall not know precisely what kusala is and what akusala. We can begin to be aware of different realities, but we may not know them yet as dhammas devoid of self. Acharn Sujin said: ''Whatever we say about lobha is only thinking. Realities arise and fall away very quickly. Instead of trying to pinpoint and ask ourselves whether this is kusala or akusala there must be the understanding of nāma and rūpa. One thinks that one knows what kusala is, but it is not known as dhamma, it is still `me’.'' Listening and considering, again and again, these are the right conditions for the arising of awareness and direct understanding. We can consider Dhamma in the midst of our activities. Daily life is full of pungent reminders that brings us back to reality: to dhamma now. Some dhammas are pleasant and we cling to them, some are unwelcome, like sickness, death or the daily news we read, and on account of these we have distress or sadness. But we should not ignore any of these reminders. If we do, we are really negligent.

When we are dreaming or thinking, the object is a concept. Seeing is different from thinking, seeing is a paramattha dhamma. This has to be known over and over again by sati and paññā, by sati-sampajañña, at the moment they occur. When we have a notion of an image, of details, of shape and form, it is not seeing that experiences visible object, but it is thinking of concepts. If we try to focus on seeing or visible object with an idea of self, we are thinking, not seeing visible object. Then the understanding of paramattha dhammas is doomed to failure. We should asked ourselves whether there is any understanding of what dhamma is. Such understanding is the foundation for satipaṭṭhāna. Only through satipaṭṭhāna we shall know without fail what dhamma is and what a concept.

People wonder what satipaṭṭhāna exactly is. The word satipaṭṭhāna has three meanings. In the Commentary to the Book of Analysis, the Dispeller of Delusion (I, Ch 7, A. Suttanta Division) it is said: 

``\ldots There are three kinds of foundation of mindfulness, satipaṭṭhāna:

\begin{enumerate}
\item the domain of mindfulness (sati gocaro)
\item the Master’s threefold surpassing of resentment and gratification (delight) as regards the entry of his disciples [on the way of practice].
\item mindfulness (sati).''
\end{enumerate}

As to the domain of mindfulness, sati gocara, this refers to the object of sati, the objects of mindfulness grouped as the four Applications of Mindfulness: Mindfulness of body, of feelings, of cittas and of dhammas.

As to the third meaning: mindfulness, sati, this refers to sati cetasika that is aware of the characteristics of realities.

As regards the second meaning, satipaṭṭhāna that is the Master’s threefold surpassing of resentment and gratification (delight) as regards the entry of his disciples [on the way of practice], this is the way along which the Buddha and his disciples went.
The ``Discourse on the Analysis of the Sixfold Sense-field'' (Middle Length Sayings'' III, no 137) explains that the Buddha is untroubled, mindful and clearly conscious when disciples who listen to the Dhamma turn away, when some of them pay attention but others do not, or when they pay attention to his words.
It is said that disciples who are like the Tathāgata in this way are ``fit to instruct a group'', thus, fit to explain the Dhamma to others.

The objects of mindfulness grouped as the four Applications of Mindfulness are of infinite value and a source of inspiration, reminding us of dhamma appearing in daily life. Often we are forgetful of nāma and rūpa, we are absorbed in concepts of people and events. However, all the sections of the ``Satipaṭṭhānasutta'' can bring us back to nāma and rūpa as they appear one at a time through the six doorways. Here we can see the power of the Buddha's teachings.

In the first Application, Mindfulness of the Body, all the aspects of the body that are explained here serve as a means of being non-forgetful of rūpas.
Mindfulness of the Body begins with Mindfulness of Breathing. One may wonder whether this does not indicate that it is necessary to develop first samatha with this subject. Those who have accumulations for samatha can develop Mindfulness of Breathing up to the stage of jhāna, but, in order to reach the goal, they must also develop insight when they have emerged from jhāna. They should penetrate with insight the jhāna-factors and the jhānacittas lest they take these for self. The whole ``Satipaṭṭhānasutta'' deals with insight. Breath is actually rūpa conditioned by citta. We cling to breath, we cannot live without it, and we take it for granted that we are breathing day in day out. When we are breathing, rūpas may appear that are tangible object: hardness, softness, heat, cold, motion and pressure. We are usually forgetful of nāma and rūpa, but the subject of Mindfulness of Breath can bring us back to realities appearing at this very moment.

We read in the Commentary to the ``Satipaṭṭhānasutta'' (Middle Length Sayings I,
10)\footnote{I am using the translation of the Satipaṭṭhānasutta and Commentary by Ven. Soma, with the title: The Way of Mindfulness. B.P.S. Kandy, Sri Lanka. } in which ways one should contemplate ``the Body in the Body''. We read:

``The bhikkhu sees, the body in the body, (1) as something impermanent; (2) as something subject to suffering; (3) as something that is soulless; (4) by way of turning away from it and not by way of delighting in it; (5) by freeing himself of passion for it; (6) with thoughts making for cessation and not making for origination; (7) and not by way of laying hold of it, but by way of giving it up.''

This refers to the development of the different stages of insight. The three characteristics of impermanence, dukkha and anattā are penetrated and in the course of the development of insight one can become detached from realities. We read, ``with thoughts making for cessation and not making for origination'', and this refers to freedom of the cycle of birth and death.

We read in the ``Satipaṭṭhānasutta'' about the postures of going, standing, sitting and lying down, and about clear comprehension in all one's actions. We read:
``And further, o bhikkhus, when he is going, a bhikkhu understands: ‘I am going’\ldots''. The same is said of the other postures. The Commentary explains:
``Who goes? No living being or person whatsoever. Whose going is it? Not the going of any living being or person. On account of what does the going take place? On account of the diffusion of the process of oscillation born of mental activity...''
Thus, being aware of the postures is not just knowing what one is doing, but we should realize that there are only elements, nāma and rūpa, arising because of conditions. As we read, the process of oscillation or motion born of mental activity occurs. When mindfulness of the body is applied, it does not mean that there is no awareness of nāma. Both nāma and rūpa occur all the time and their difference should be discerned by paññā.

All the sections in the Satipaṭṭhānasutta give us examples of different situations in life that can remind us of being aware of the dhamma appearing through one of the six doors. We read in the section on clear comprehension, sati-sampajañña, that the monk had to practise clear comprehension in all his actions, such as walking, bending, stretching, wearing robes and bowl, eating, chewing, speaking and being silent.
The Commentary explains that there is no self who eats. The process of digestion goes on because of conditions. We read: 
\begin{quote}
``There is no one who, having put up an oven and lit a fire, is cooking each lump standing there. By only the process of caloricity the lump of food matures. There is no one who expels each digested lump with a stick or pole. Just the process of oscillation or motion expels the digested food.''
\end{quote}

In the section on the Repulsiveness of the Body we read about hair of the head, hair of the body, nails, teeth, skin and so on. This is again a reminder of dhamma in daily life. We can notice these parts of the body time and again, and they can remind us that what we take for the body are only elements that are impermanent and not self. In the same way the section on the Elements can bring us back to reality when we are distracted. What we take for ``our important body'' are only elements devoid of self. The cemetery contemplations are recollections of death. We read that when the bhikkhu sees a corpse he thinks of his own body thus: ``Verily this body of mine, too, is of the same nature as that body, is going to be like that body, and has not got past the condition of becoming like that body.''

The second Application of Mindfulness is the Contemplation on Feeling. One moment we have great anxiety and sadness and this is accompanied by unhappy feeling, and the next moment there can very quickly be a change to pleasant feeling when we laugh about something, even about our worries. Feelings change before we can do anything about them, they are beyond control. It is very difficult to be aware precisely of feeling. The Abhidhamma is very precise, but we usually think of a mixture of many phenomena, bodily and mental. Through the Abhidhamma we can have a basic knowledge of the different processes of cittas that experience different objects and that each have appropriate conditions for their arising. It is important to know that seeing arises in one process of cittas and thinking in another process and that they experience different objects, and also, that on account of these experiences different feelings arise. The understanding of conditions will make it clearer that feelings are non-self.

We should pay attention to the third Application of Mindfulness that includes all kinds of cittas arising now: kusala citta, akusala citta, indeterminate citta. The first citta that is mentioned is citta with attachment. We should not neglect akusala citta as object of mindfulness. We take akusala citta as well as kusala citta for self, but they arise because of their own conditions. It is natural that kusala cittas and akusala cittas are alternating in our relationship with others. Through the Abhidhamma we learn that detachment accompanies each kusala citta. When we are generous, when we try to help someone else, we should have no preferences for specific people and we should not expect any kindness, any recognition in return. This means that we need equanimity and renunciation or detachment all the time. Through satipaṭṭhāna we come to know the different cittas that arise. Mindfulness of citta is a condition for beginning to distinguish kusala citta and akusala citta. However, we should not forget that the stages of insight develop in a specific order. The first stage of insight is knowing nāma as nāma and rūpa as rūpa, and before this stage arises kusala and akusala cannot be clearly realized as nāma and thus their different characteristics cannot yet be precisely known.

The objects of the fourth Application of Mindfulness are the hindrances, the five khandhas, the six internal and external sensebases (āyatanas), the seven factors of enlightenment and the four noble Truths. Thus, these are dhammas under different aspects and different cetasikas. We read in the ``Satipaṭṭhānasutta'':

\begin{quote}
``Here, O bhikkhus, a bhikkhu lives contemplating the mental objects in the mental objects of the five hindrances.

How, O bhikkhus, does a bhikkhu live contemplating mental objects in the mental objects of the five hindrances?

Here, O bhikkhus, when sensuality is present, a bhikkhu knows with
understanding: `I have sensuality,' or when sensuality is not
present, he knows with understanding: `I have no sensuality\ldots '
When anger is present, he knows with understanding: `I have anger,' or when anger is not present, he knows with understanding: `I have no anger\ldots' ''
\end{quote}

We can see how the Abhidhamma helps us to understand the suttas, and how the Abhidhamma is indispensable to start on the right Path. The hindrances are only akusala cetasikas, arising because of conditions, they have no owner. In this sense they are beyond control. But paññā, right understanding developed through satipaṭṭhāna, can eventually eradicate them.

The body is with us all the time, when standing, sitting, going, lying down, but we are forgetful of dhammas. We have pleasant feeling, unpleasant feeling, indifferent feeling all the time, but we are forgetful. It is the same with all the other aspects in the other two Applications of Mindfulness. We are forgetfull, but all sections of the Applications of Mindfulness can bring us back to reality. Often we are dreaming but sometimes there can be non-forgetfulness of visible object, or sound, just one dhamma at a time. When we discern the difference between moments of forgetfulness and a moment, even a single moment, of sati, we can verify its characteristic. This is the right condition for its development.

Insight, vipassanā, is developed by means of mindfulness of all nāmas and rūpas appearing in our daily life and these are classified as the four Applications of Mindfulness. There is no specific order according to which there should be awareness. At one moment rūpa may appear, at another moment nāma may appear. The goal is understanding that can eradicate defilements.

We read in the ``Satipaṭṭhānasutta'' that these four Applications are the only way for reaching the right path, for the attainment of nibbāna. The Commentary states:

\begin{quote}
``The Real Eightfold Path is called the right path. Verily, this preliminary, mundane Way of the Arousing of Mindfulness made to become (grown or cultivated) is conducive to the realisation of the Supramundane Way.''
\end{quote}

While satipaṭṭhāna is being developed, the Path is still mundane, but eventually it leads to enlightenment. At the moment of enlightenment the eightfold Path, paññā and the accompanying factors, have become supramundane, lokuttara.

The Buddha, in his great compassion, taught us that there is dhamma at this moment in our daily life. All the sections of the four Applications pertain to daily life and they can remind us to develop understanding of nāma and rūpa at this moment, in the midst of our activities. We should not be neglectful because life is short and we do not know when we will have an opportunity to listen again to the true Dhamma.

Acharn Sujin exhorted us to have patience and not to expect an immediate result of the development of understanding. She said:

``If we know how much ignorance we have accumulated it will prevent us from trying to hasten the development of paññā. The Abhidhamma helps us to understand that there are many processes of cittas arising and falling away very
rapidly. When we see visible object, not only seeing arises in that process, but also many other cittas. Seeing is closely followed by the javana-cittas (impulsion) that are either kusala cittas or akusala cittas. When we do not apply ourselves to dāna (generosity), sīla (morality) or bhāvanā (mental development), the javana-cittas are akusala cittas. The eye-door process is followed by a mind-door process with javana-cittas that are kusala cittas or akusala cittas. Countless processes of cittas with akusala javana-cittas arise, but we do not realize this. Kusala citta arises seldom and thus we can understand that it must take a long time to develop paññā.

Right understanding can be developed of one object at a time so that there can be firm and clear understanding of nāma and rūpa. When insight knowledge arises there is no doubt that `I’ does not exist, that there are only different elements arising and falling away.''

The understanding that ``there is no person'', only paramattha dhammas arising and falling away, does not mean that we should not care for other people. On the contrary, when we have less clinging to the ``self'' kusala can become purer. We have accumulated so much selfish clinging and therefore we may easily confuse mettā and attachment. We may perform kind deeds in order to be liked by others. We like to have friends so that we do not feel lonely. When we develop genuine mettā we shall cling less to people. When kusala citta with mettā arises there is true friendship, and there is no need to long for kindness from others in return.

Acharn Sujin and all the friends I met in the Foundation building were very earnest in listening, studying and explaining the Dhamma, and this earnestness permeated the development of paññā, mettā and all other good qualities in their daily lives. The sessions in the Foundation were characterized by a friendly, informal atmosphere. The podium established in the main hall was not used and we were all sitting close to each other, just like in a family circle. People were interested in what I was doing in the Netherlands. I told them about the Internet, and my Visuddhimagga and Pāli study. Acharn Sujin was assisted by others who encouraged me all the time to continue asking personal questions on the development of satipaṭṭhāna. Acharn Sujin asked me whether I take kusala and akusala for self, and I answered that I do. She said; ``When we have no understanding of realities, it is always ‘I’. ‘I would like to have more kusala, how can I be better’. Nobody can prevent thinking in this way, but one should understand that such moments are not self. Knowing in which way understanding is developed is most important.''

At the Foundation there are always people caring for others. On weekends when most of the sessions take place, kind people bring food for everybody. Acharn Sujin and her sister Khun Jid discussed at length the luncheon menu in the restaurant where they had invited us. This seems a trivial matter, but it demonstrates their loving care and kind intention to make things as agreeable and pleasant as possible for all of us. They show us how satipaṭṭhāna and all kinds of kusala can be developed in daily life in a natural way. Acharn Sujin never tires of performing kusala, and, in particular, the explanation of the Dhamma. She said that feeling tired shows clinging to ``self''. During my last day in Bangkok she encouraged me, saying: ``When you perform more kusala your body will become stronger.''

A shoot of the Bodhitree was planted in the grounds of the Foundation. These shoots collected in Bodhgaya usually do not grow, but this one thrives very well and is now a beautiful tree. The well kept surroundings of the building are very conducive to personal conversations on Dhamma and contemplation on the words of the Buddha.
Acharn Sujin reminded us time and again to have patience, courage and perseverance in the development of right understanding of realities. Her words, ``Is there any understanding at this moment'' still resound in my thoughts. She encouraged us, saying, ``Begin again, begin again''. Sati can begin to be aware of whatever appears through one of the six doorways, even though the truth is not yet directly realized.
To the extent understanding grows our confidence in the Triple Gem becomes stronger. We can have confidence in the Path we have to continue developing with courage and perseverance. Mindfulness of realities is the highest respect we can pay to the Buddha.

I am most grateful to Acharn Sujin and all my friends for the reminders of satipaṭṭhāna and for their good example of practising the Dhamma in daily life.

Bangkok,
February 2004.








