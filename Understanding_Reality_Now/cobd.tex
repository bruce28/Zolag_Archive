\part{The Cycle of Birth and Death}
\chapter{Preface}

\begin{description}
\item \textit{In Memory of Lodewijk}
\end{description}


Soon after Lodewijk's passing away I decided to undertake alone a
journey to Thailand. In January 2013
Acharn\footnote{Acharn is the Thai word for
teacher. In Pali: ācariya.}
Sujin, Sarah and other friends had organized a three weeks sejourn in
Thailand for a group of Vietnamese friends and other friends from
different countries whom I have known for a long time. There were three
different trips outside Bangkok: to Hua Hin which is near the sea, to
Wang Nam Khiao or Korat, in the North East, and to Kaeng Krachan, a
place where Acharn Sujin and Khun
\footnote{Khun is the Thai word for Mr. or
Mrs.}
Duangduen regularly stay and where we often had visited them before.

I had never thought that I would come to Thailand again, but it all
happened according to conditions. Thanks to Sarah's encouragement I
could undertake this journey, and I am most grateful for the kind
concern and moral support of Sarah, Jonothan and the other friends. I
was surrounded by a group of sympathetic friends who were always ready
to give me assistance.

When I was young and I married my beloved one I did not think that there
must be an ending too. That seemed so far away. When the end comes it is
so hard to accept the unavoidable. We keep on thinking of stories,
beautiful ones and sad ones. Thinking is a reality, it arises for a
moment and then falls away. The stories we think of are not realities,
they are imaginations.

Throughout our journey Acharn Sujin was never tired of explaining again
and again the true nature of what appears right now, at this moment,
like seeing, visible object, hearing, sound or thinking. I am very
grateful to her that she time and again reminded us of the present
moment, the reality appearing now. That is the only moment the true
nature of a reality can be investigated. This helped me to understand
that the truth in the ultimate sense (in Pali: paramattha dhamma) is
quite different from concepts and stories which are made up by our
imagination and which we find so important.

We may think for a long time about what happened in the past, about
other people, what they did and said, but such moments are different
from developing understanding of realities that appear now, one at a
time. The whole of the Buddha's teachings deal with the present moment.

It is beneficial to constantly hear about seeing, visible object,
hearing or thinking that can be directly known when they appear.
Otherwise we forget what is reality and what is not and we spend our
days dreaming about what is not reality. A great lesson I learnt while
in Thailand. These constant reminders were most helpful to me.



\chapter[What is Life?]{}
\section*{What is Life?}

In the ``Kindred Sayings'', II, 180 (Nidāna, Ch XV, § 4, Tears) we read
that the Buddha said at Sāvatthī:

\begin{quote}

``Incalculable is the beginning, brethren, of this faring on. The
earliest point is not revealed of the running on, faring on, of beings
cloakedd in ignorance, tied to craving.

As to that, what think you, brethren? Which is greater:- the flood of
tears shed by you crying and weeping as you fare on, run on this long
while, united as you have been with the undesirable, sundered as you
have been from the desirable, or the waters in the four seas?''

``As we allow, lord, that we have been taught by the Exalted One, it is
this that is greater: the flood of tears shed by us crying and weeping
as we fare on, run on this long while, united as we have been with the
undesirable, sundered as we have been from the desirable- not the waters
in the four seas.''

``Well said! Well said, brethren! Well do you allow that so has been the
doctrine been taught by me. Truly the flood of tears is greater\ldots{}

For many a long day, brethren, have you experienced the death of mother,
of son, of daughter, have you experienced the ruin of kinsfolk, of
wealth, the calamity of disease\ldots{}

Why is that? Incalculable is the beginning, brethren, of this faring on.
The earliest point is not revealed of the running on, faring on, of
beings cloakedd in ignorance, tied to craving.

Thus far enough is there, brethren, for you to be repelled by all the
things of this world, enough to lose all passion for them, enought to be
delivered therefrom.''

\end{quote}

We are born, we die and then we are born again, this goes on and on so
long as we are in the cycle of birth and death. Each life is very short,
before we ralize it it comes to an end. When we are reborn we do not
remember our life as it is at present, just as at this moment we do not
remember our past life. What has fallen away never comes back and this
is true of each moment of consciousness, and each physical reality. Each
moment will be immediately past, but we are deluded and take mental
phenomena and physical phenomena for permanent and self. The Buddha
taught about realities in detail so that they can be understood as
non-self (in Pali: anattā).

For a few days I stayed in the same hotel as my friends Sarah and
Jonothan, the Peninsula hotel in Bangkok. I spent a happy time in their
company and throughout my journey they gave me kind advice when I was in
trouble. From my window I looked across the river to the Oriental Hotel
where Lodewijk and I had enjoyed many pleasant days. These belong to the
past now.

The next day I heard that a good friend, Ivan Walsh, had died suddenly.
We went to the temple where rituals were performed and where later on
the cremation would take place. Here Acharn Sujin and several friends
were present. In the morning Acharn's sister, Khun Sujid, and Khun
Sujid's daughter had still seen Ivan on the street, and now he is
another person. It can all happen so suddenly.

The departing from this life is similar to the departing from last life.
When we passed away from last life and we were born into this life, all
that happened in the past is forgotten. It is difficult to accept this
because of our clinging. We do not like the idea of being forgotten by
our beloved one who passed away to another life.

Acharn explained to me that it is also difficult to accept the truth of
this moment: ``Whom do you see? There is always someone, even now.'' In
reality there is no person, there is no one who can stay. What we take
for a person is consciousness (in Pali:citta), mental factor arising
with consciousness (in Pali: cetasika) and physical phenomena (in Pali:
rūpa). These are only fleeting mental phenomena and physical phenomena
which arise and then fall away immediately.

Seeing-consciousness is a moment of consciousness, a citta, that sees
only what is visible, visible object, which is a physical phenomenon, a
rūpa. It sees visible object just for an extremely short moment, and
then it falls away. After the seeing has fallen away we think with
attachment about things and persons we believe we see. It seems that we
see them, but in reality we do not see them, seeing has fallen away
already. Because of remembrance, saññā, a cetasika (mental factor)
arising with each citta, we think of persons and things and we believe
that they stay. In reality seeing, visible object or thinking arise for
a very short moment and then fall away. They are mere elements and
nobody can change their nature. Acharn said: ``What has fallen away
never comes back again, never, never.''

I said that it is so sorrowful when I think about Lodewijk, that he
never comes back. Acharn answered:

``Think of yesterday. Where were you yesterday? And think of this
morning, where were you? There is no one at all, just this moment. We
have to be very courageous to know that this is true. Even when there is
unpleasant feeling, it is just a moment. It has arisen, and if it had
not arisen it could not be here right now.''

Nina: ``Right understanding is so weak.''

Acharn: ``Yes, because of the self, because of you. But when it is not
you it is only the nature of an element. So, we do not mind how many
lives will come because we cannot force the ending of the cycle without
conditions. It has to be like this. But paññā (understanding of
realities) develops and develops. That is why the Buddha taught us the
Jātakas, the stories of his previous lives as a Bodhisatta. Each reality
has gone, sound, sight, nothing is left. Is one attached to someone in
one's thoughts? But actually there are only seeing, thinking, visible
object.''

The Buddha, during countless previous lives as a Bodhisatta, developed
wisdom, right understanding, so that in his last life he could become
the omniscient Buddha. He developed right understanding again and again
of seeing, visible object, hearing, sound, attachment, generosity, of
all realities of daily life. We also have to develop right understanding
of realities life after life so that eventually enlightenment
\footnote{Enlightenment, in the context of
the Buddhist teachings, is highly developed paññā that eradicates
defilements and experiences the unconditioned dhamma, nibbāna. There are
different stages of enlightenment.}
can be attained and defilements eradicated.

Seeing is a reality, it arises and experiences just what is visible, and
then it falls away. It arises because of conditions: eyesense and
visible object are conditions for seeing, and it is a citta that is the
result of kamma, vipākacitta. It only sees visible object, but we
believe that we see a person or thing. That is thinking, arising on
account of what is seen. Thinking is not vipākacitta. When we think, the
citta may be wholesome citta (kusala citta) or unwholesome citta
(akusala citta). It seems that we can see and think at the same time,
but only one citta can arise at a time and experience one object. Cittas
arise and fall away succeeding one another extremely rapidly and that is
why we are deluded about the truth.

Thinking is usually motivated by akusala (unwholesomeness), and this is
the case when we are not intent on what is wholesome, such as
generosity, helping others or developing understanding. Citta can think
of reality or of what is not a reality, but a concept. When we are
living in a dream world all day, thinking of what is not real, we are
deluded and the citta is akusala. We should remember that there is no
one in the visible object, no person or thing. Visible object is only a
kind of rūpa that impinges on the eyesense and that can be seen. We have
an idea of ``I see'', but there is no self who sees, only seeing sees.

Acharn explained: ``When thinking of Lodewijk or Ivan, there is
attachment and it hinders, it hinders the understanding of seeing, but
it takes a long time to really understand this. The Path is very subtle,
but very effective, paññā really knows what hinders.

Now we do not know what hinders. We cry and we think a lot about the
situation we are in. When paññā sees what is a hindrance it cannot
hinder any more, because it is understood.''

Sarah remarked: ``People often say that they found it so difficult in
the case of separation through death that they did not have a chance to
say farewell, but actually, it is just clinging to one's own thought,
one's own idea.''

Acharn Sujin said: ``Even that moment is gone, not to be thought about
again. It is past and past and past, all the time. Nothing is left, only
thinking and memory. Nothing can belong to anyone at all.''

This was a good lesson reminding us not to attach too much importance to
the stories which are objects of our thinking. Thinking arises, it is
conditioned and we cannot prevent it, but we can remember that what
really matters is learning the truth of the reality appearing right now.

Ivan's body was laid in state in the temple with one hand stretched out
so that we could sprinkle water over it and remember his good deeds.
Ivan had always encouraged me to keep on writing about the Dhamma. I was
disinclined to use a computer but he had persuaded me to start writing
on the computer, so that I could share what I wrote with many people.

Acharn remarked: ``So, I smile to Ivan, and may he appreciate all my
good deeds. The Buddha did not teach anyone to cry, because that is
akusala. At the moment of kusala there is no aversion (dosa), no crying,
but appreciation.''

When I said that there are conditions for aversion and sadness, she
said:

``When there is understanding one can see that paññā is the best of all
conditioned realities, that it is a precious thing in one's life.
Everything is past in one's life, all the time. It passes away never to
come back.''

I remarked that intellectual understanding does not really help. It
helps for a while and then it is gone and sadness arises again.

Acharn answered: ``The accumulation of right understanding can become
stronger, better than other accumulations. Without intellectual
understanding how could there be stronger understanding? We have to go
step by step, like climbing a mountain. We cannot reach the top
immediately. Each step leads to more right understanding.''

We have accumulated such an amount of attachment, ignorance and wrong
view. Acharn explained that it has to be eradicated little by little,
very, very little at a time, but that this is better than none at all.
We have to be courageous and patient to develop understanding of one
reality at a time.

Elle, Ivan's wife, asked Acharn how to cope with sadness and loneliness.
She found it so very difficult to be alone in the house. Acharn
explained that one is not alone when one studies the word of the Buddha;
one is in his presence, he is addressing his words to us. This is true,
but we have to listen again and again until there is more understanding
of whatever appears at the present moment.

Acharn said: ``As to thinking about living alone, as soon as it is known
it is gone, as fast as that. Thinking follows and it seems permanent,
but as soon as it is known it is gone.''

During this journey I began to see that dwelling in the past, in stories
about Lodewijk's sickbed, his last days, his suffering, is quite
different from studying and considering what is real in the ultimate
sense and appearing at this very moment, like seeing and visible object.
We discussed about paramattha dhammas (ultimate realities) for hours,
day after day. There is a great contrast between the world of concepts
and imaginations and the world of realities. This helped me not to be
completely absorbed in what is not real.

Ivan's body was laid down in a case and then the monks chanted texts.
Acharn spoke about Ivan's life, and this is also the life of all of us:

``He was born and he died. What did he get from his whole life?
Everything arises and passes away in splitseconds, all the time, from
day to day, from moment to moment. Nothing belonged to him because there
is no him. The rūpa-elements and the nāma-elements arise and fall away
by conditions and never come back. Everyone's life is like this because
there is no self. That is why we listen to the truth of whatever appears
now, to understand it as truth. To understand seeing as seeing; no one
sees and it does not belong to anyone because it is gone completely,
never to come back. How can it be my seeing? It is only a moment of
experiencing an object. Who can prevent seeing from arising? There are
conditions for its arising, and, thus, it arises.''

Life is only the experiencing of an object through one of the six
doorways of the senses and the mind. Only one citta arises at a time,
experiences an object and then falls away. At the moment of seeing just
what is visible, there cannot be the experience of sound, these are
different cittas, experiencing different objects. The citta which
thinks, thinks of persons or situations. In the ultimate sense a person
is mere elements that arise and fall away. We can learn that one is born
alone, sees alone, thinks alone and dies alone. After passing away from
this life there is no return of the same individual.


\chapter[Living alone]{}
\section*{Living alone}

Acharn Sujin repeated many times that what is now today, will be
yesterday tomorrow. This reminds us that all we find so important now
will be past in no time. Right understanding of realities that arise and
fall away will lead to detachment. We find it very important to be in
the company of friends, but Acharn reminded us that in a short while we
shall not know each other anymore. In a next life we shall have new
friendships. She spoke about an example in the Tipiṭaka about seven
friends who in their last life did not remember that they were friends
before. They attained arahatship.

Acharn said: ``When we listen more there will be more understanding of
seeing. There must be that which sees and that which is seen, only that.
There is no other world, no one there. Cittas arise just one at a time,
there is no hearing, no sound, no idea about the object that is seen and
no thinking.

If nothing arises at all there is no world. Whatever arises, even just
one reality, that is the world. It is the arising and falling away of
different realities. The meaning of arising and falling away is: it
never comes back. No one is there, only different cittas, different
cetasikas, different realities. Understanding is not developed by
anyone. It is developed by listening, considering; no one can do
anything because there is no self.

A moment of understanding is like a drop of water in the ocean of
ignorance.''

Understanding is not developed by anyone because there is no person, no
self who develops it. Understanding itself develops when there are the
right conditions for it. There is such a great deal of ignorance, but
the Dhamma is like an island in the ocean of concepts, the ocean of
defilements.

Seeing is one citta and when it arises there cannot be hearing at the
same time. Seeing experiences visible object. Hearing is another citta
that experiences sound. It may seem that we can see and hear at the same
time, but this is a delusion. Each citta can experience only one object
at a time, and it falls away immediately. After it has fallen away we
think of what has been seen and heard, and then we live in the world of
concepts.

What we take for a person are mere elements arising and falling away. We
read in the ``Visuddhimagga'' (XI, 30):

\begin{quote}

``What is meant? Just as the butcher, while feeding the cow, bringing it
to the shambles, keeping it tied up after bringing it there,
slaughtering it, and seeing it slaughtered and dead, does not lose the
perception `cow' so long as he has not carved it up and divided it into
parts; but when he has divided it up and is sitting there, he loses the
perception `cow' and the perception `meat' occurs; he does not think `I
am selling cow' or `They are carrying cow away', but rather he thinks `I
am selling meat' or `They are carrying meat away'; so too this bhikkhu,
while still a foolish ordinary person--both formerly as a layman and as
one gone forth into homelessness--does not lose the perception `living
being' or `man' or `person' so long as he does not, by resolution of the
compact into elements, review this body, however placed, however
disposed, as consisting of elements. But when he does review it as
consisting of elements, he loses the perception `living being' and his
mind establishes itself upon elements.''

\end{quote}

We think of people as ``this man'' or ``that woman'' and we are not used
to seeing what we take for a person as different elements. We might find
it crude to think of a body which is carved up and divided up into
parts, just as a cow is carved up by a butcher. When a cow is peeled and
carved up and then covered again by the skin we may believe that there
is a cow, but in reality there is no cow at all. Evenso we may believe
that a person exists, but there isn't any person, there are only
elements devoid of ``self''.

We should consider again and again that what we take for a lasting
person are actually mental phenomena (in Pali: nāma) and physical
phenomena (in Pali: rūpa) that arise and fall away. Consciousness,
citta, is nāma. There is only one citta arising at a time, but each
citta is accompanied by several mental factors, cetasikas, which each
perform their own function while they assist the citta in knowing the
object. One can think of something with aversion, with pleasant feeling
or with wisdom. Aversion, feeling and wisdom are mental phenomena which
are not citta; they are cetasikas which accompany different cittas.
Thus, both citta and cetasika are nāma, they experience an object,
whereas rūpa such as sound or eyesense do not experience anything. Some
cetasikas such as feeling or remembrance, saññā, accompany each citta,
whereas other types of cetasikas accompany only particular types of
citta. Attachment, lobha, aversion, dosa, and ignorance, moha, are
akusala cetasikas which accompany only akusala cittas. Non-attachment,
alobha, non-aversion, adosa, and wisdom, amoha or paññā, are sobhana
cetasikas, beautiful cetasikas, which can accompany only sobhana cittas.

When we lose dear people through death we are bound to feel lonely. I
had the following conversation about this subject:

Nina: ``When feeling lonely it is difficult to be aware of one reality
at a time. But if we try to escape this situation there is lobha
(attachment) again.''

Acharn Sujin: ``That does not work. We have to be courageous, brave
enough to see that there is actually no one, not even you at that
moment. This is the best cure.''

Sarah: ``Even when we are with people, we are seeing alone, hearing
alone.''

Nina: ``Akusala cetasikas are bad friends and they are gone.''

Sarah: ``When feeling sorry, there are bad friends.''

Nina: ``They come again and again and again.''

Acharn: ``There is only citta with such realities. It cannot stay, it
will go away. Is it good to have it?''

Nina: ``It is not good to have it.''

Acharn: ``So, it is better to have understanding.''

Nina:``This is not possible on command.''

Acharn: ``At the moment of understanding there is no regret. One is
freed from being enslaved, and this was never realized before because
one enjoyed being enslaved.

When there is more understanding of Dhamma there is no wish for anything
at all. This is the beginning of understanding. It has conditions for
its arising and nobody can do anything at all. We can learn to see
realities, one at a time. Like now, there is seeing and at other moments
there are hearing or thinking, unknown all the time. But if there is a
moment of understanding of a reality, it can arise again and go on to
other realities.''

Citta experiences one object, and it is actually alone. At the moment of
seeing visible object there is no one else, seeing is alone. At the
moment of seeing no hearing or thinking arise. Seeing experiences the
object alone. When realities are taken as a mass, a collection, there is
the world of many people. Cittas arise and fall away in succession very
rapidly, they are like a flash. That is why we have a concept or idea of
what appears as something permanent. Acharn said that we have to be
brave in order to understand that what appears is just a reality. We
need courage to let go of wrong view that clings to the idea of person
or ``self''. Right understanding leads to detachment, but our nature is
attachment.

While we were in Huahin we went to the sea where Ivan's ashes and bones
were to be let down into the water. We went out on a boat that belonged
to the Water Police. While we were waiting for the boat in the harbour
and also while we were on the boat Acharn kept on speaking about the
true nature of the reality appearing at the present moment. We
considered realities instead of dwelling too much on situations, on sad
events. A monk who always listened to Acharn's radio program was present
and after he recited some texts, the ashes were let down into the water.
The boat went three times around this place and we kept on throwing
flowers into the water. In the end there was a circle of flowers around
the place where the ashes went down. Acharn said: this is like the cycle
of birth and death.

It is good to be reminded of the cycle of birth and death. The last
citta of this life, the dying-consciousness (cuti-citta), is succeeded
immediately by the rebirth-consciousness (paṭisandhi-citta) of the
following life. Our life is an unbroken series of cittas. Wholesome
qualities and unwholesome qualities which arose in the past can
condition the arising of such qualities at present. Since our life is an
unbroken series of cittas, succeeding one another, wholesome qualities
and unwholesome qualities can be accumulated from one moment to the next
moment, and, thus, there are conditions for their arising at the present
time. When we listen to the Dhamma and we have a little more
understanding, this is never lost. Understanding is accumulated and it
can grow from life to life.

Each day we had one session of two hours in the morning and one session
of two hours later in the afternoon. In Huahin the sessions were in a
large lounge of a bungalow where Jonothan and Sarah had one room and
where I had another room. After the afternoon session, Thai friends
arranged for fruits, cookies and different snacks. There was such an
abundance of food that there was no need to go out for supper anymore.
Our friends were most attentive to all our needs and looked after us all
the time. We went out for lunch to different places and even while we
were having lunch Acharn would speak about paramattha dhammas appearing
right now. The whole atmosphere was most pleasant while we enjoyed each
other's company, the beautiful panorama and the great variety of dishes.

We had the following conversation about understanding realities:

Acharn: ``Visible object can be understood but memory takes it for a
person or a thing. There should be the development of all realities,
even of thinking. One can begin to see the difference between right
understanding and wrong thinking about people and things. Do not have
the idea that there should not be thinking, but understand thinking as
just a reality.''

Nina: ``Trying not to think is forced.''

Acharn: ``It is not natural. Paññā cannot grow when it is not natural.

It arises by conditions and it can become stronger and stronger.''

Sarah: ``When it is time for thinking, time for sadness, it is
conditioned like that. No one can change it or stop it.''

Nina: ``We should not select, but just be aware of any reality.''

Acharn: ``The self is trying. When there is trying it shows that the
understanding of anattā (non-self) is not firm, not well established.

But no matter whether there is a day without awareness, it is by
conditions. When awareness arises by its own conditions it is much
better than trying the whole day with the idea of self. The idea of self
is building up at that very moment. When awareness arises for only a
moment the difference can be seen between unawareness the whole day and
a moment of understanding of a reality. Only paññā can see when lobha
does not arise and when it arises all the time, after seeing, hearing,
at the moments of trying. Lobha is like a big boss.''

Several times Acharn reminded us of the power of lobha, attachment. It
is dangerous that it is mostly unknown. Only paññā can see when lobha
arises and leads one astray. One may wish to have more understanding but
at such a moment one clings to the idea of self.

After our sejourn in Huahin, we stayed for the weekend in Bangkok. On
Saturday Khun Duangduen offered us a lunch in her garden which is a
pleasant, restful place. On Sunday there were sessions in Thai in the
building of the ``Dhamma Study and Support
Foundation''\footnote{This is the center where all
sessions with Acharn Sujin take place each weekend.}.
It was Acharn's birthday and it was inspiring to see many people who
came with gifts and paid respect to Acharn. We could watch the great
generosity of the Thais. The little room Acharn uses to meet people
privately was full of flowers, fruits and other gifts.

During the session we had conversations about life in conventional sense
and life in the sense of paramattha dhammas. It was stressed that it is
important to know the difference between concept and reality. When we
think of people we live in the world of concepts and when understanding
is developed of reality as it appears through one of the six doors, one
at a time, we come to know the world of paramattha dhammas.

We had lunch in the Foundation building at a long table with Acharn and
other friends. We were enjoying the food offered by a couple who
sponsored the meal. Husband and wife served us with such great concern
and affection, taking care all the time to see if anybody needed
anything. Their children entered the room and paid respect to Acharn. I
found it a special experience to be back again in the Foundation. All my
Thai friends welcomed me with great cordiality and they kept smiling,
radiating kindness. When everyone around us is smiling with sincere
kindness, we just have to smile too and it is impossible to be sad and
depressed.

Our second trip outside Bangkok was to the North East, to Wang Nam
Khiao, also called Korat. On the way we visited a museum of a petrified
forest. It was an exposition of the geological history of the region and
one could see many rare examples of petrified trees. It was crowded with
school children so that we had to wait a long time and since our visit
took many hours we arrived rather late in Wang Nam Khiao. This is a
mountainous region where we went out for walks in the morning before
breakfast. We stayed in peaceful bungalows with a balcony situated at
the waterside. We had to walk from our bungalow to the restaurant for
breakfast. For lunch we went out to a variety of places. The lunch
tables were outside in the garden of the restaurant so that it seemed
that we were in the middle of a forest. One of our outings was to the
best restaurant in the region where very refined food was served and
which, as healthy air was concerned, had the seventh place in the world.
This made me think of Kuru where the outward conditions and the climate
were most favorable for the development of the understanding of Dhamma.

I was sitting next to Acharn in the car and I enjoyed the mountainous
landscape. Meanwhile we had a most beneficial Dhamma conversation.

Acharn: ``Sometimes there is very strong lobha or very strong dosa
(aversion), who can condition that? The nature of attachment is
different from the nature of aversion. Who can control them? There must
be conditions, no matter kusala or akusala arises. The truth can appear
little by little as not permanent. At this moment there can be a little
understanding of what appears as uncontrollable; it does not belong to
anyone. Can that which arises and falls away and never comes back be
anyone? Not at all. That is the way paññā develops from pariyatti
(intellectual understanding), to paṭipatti (development of direct
understanding), to pativedha (direct realization of the truth).

Next life one is a different person, suddenly. But past accumulations go
on. That is why people have different characters, different likes and
dislikes.''

Nina: ``I experience a very pleasant object with pleasant feeling, such
as the mountains.''

Acharn: ``It is a reality, it is conditioned. It falls away before we
know what it is. As soon as it is an object that is experienced, it is
gone. Then another object appears and paññā can understand that. The
intellectual understanding conditions detachment from clinging when time
comes. But it is not as effective as direct understanding. The
difference between the two can be seen.''

Nina: ``It is not so easy to know direct understanding.''

Acharn: ``When awareness arises it can be seen that it is quite
different. Intellectual understanding can condition direct
understanding, and it keeps on going by conditions. Otherwise it is
always, how, how can `I' understand.''

Often we ask questions with ``how can I\ldots{}'' and true, this is
motivated by attachment, lobha. We were reminded by Acharn to keep in
mind that all dhammas are non-self, anattā, and that we, in that way,
never will be lost by our own thinking or by wrong understanding. We
cling to having progress in understanding and this is not effective. As
Acharn often said, we cannot do anything. Realities arise because of
their own conditions and nobody can cause their arising. Seeing arises
when there are the appropriate conditions for its arising. Visible
object and eyesense are rūpas that condition seeing. Visible object
impinges on the eyesense and then there are conditions for seeing.
Seeing is caused by kamma, it is vipākacitta.

Some cittas are results of akusala kamma and kusala kamma, they are
vipākacittas. Kamma is intention or volition. Unwholesome volition can
motivate an unwholesome deed which can bring an unpleasant result later
on, and wholesome volition can motivate a wholesome deed which can bring
a pleasant result later on. Akusala kamma and kusala kamma are
accumulated from one moment of citta to the next moment, and, thus, they
can produce results later on. Kamma produces result in the form of
rebirth-consciousness, or, in the course of life, in the form of seeing,
hearing, smelling, tasting and the experience of tangible object through
the bodysense. Vipākacittas experience pleasant objects or unpleasant
objects, depending on the kamma which produces them.

Kamma also produces rūpas such as eyesense, earsense and the other sense
organs. Without eyesense and without visible object there could not be
seeing.

There are several conditions for each dhamma that arises and this shows
the nature of anattā of dhammas. We cannot cause their arising.

Evenso, nobody can cause the arising of sati, mindfulness, and paññā,
understanding, however much we wish for their arising. They can only
arise when there are the appropriate conditions. They are sobhana
(beautiful) cetasikas that can only arise with sobhana citta and there
are many levels of them. When we listen to the Dhamma and we learn about
the realities that can be experienced through the six doorways, one at a
time, and when we consider again and again what we hear, gradually
intellectual understanding can develop. If the conditions are right,
direct awareness of realities can sometimes arise so that direct
understanding can develop. But this does not occur so long as we are
wishing for it.

Acharn reminded us all the time of clinging to sati and paññā that is
deeply rooted and hard to detect. We tend to forget that sati and paññā
are non-self, anattā. The development of understanding leads to
detachment, detachment from the idea of self.



\chapter[No Return]{}
\section*{No Return}

We read in the ``Sutta Nipata'' (vs. 547-590)
\footnote{Translated by John D. Ireland
(Kandy: Buddhist Publication Society, 1983).}:

\begin{quote}
``Unindicated and unknown is the length of life of those subject to
death. Life is difficult and brief and bound up with suffering. There is no means by
which those who are born will not die. Having reached old age, there is death.
This is the natural course for a living being. With ripe fruits there is the
constant danger that they will fall. In the same way, for those born and subject
to death, there is always the fear of dying. Just as the pots made by a
potter all end by being broken, so death is (the breaking up) of life.

The young and old, the foolish and the wise, all are stopped short by
the power of death, all finally end in death. Of those overcome by death
and passing toanother world, a father cannot hold back his son, nor
relatives a relation. See! While the relatives are looking on and
weeping, one by one each mortal is ledaway like an ox to slaughter.

In this manner the world is afflicted by death and decay. But the wise
do not grieve, having realized the nature of the world. You do not know the
path by which they came or departed. Not seeing either end you lament in vain.
If any benefit is gained by lamenting, the wise would do it. Only a fool would
harm himself. Yet through weeping and sorrowing the mind does not become
calm, but still more suffering is produced, the body is harmed and one
becomes lean and pale, one merely hurts oneself. One cannot protect a
departed one (peta) by that means. To grieve is in vain.''

\end{quote}

As we read, we do not know the path by which a person came into this
world or departed from it. We do not know his past life nor his future
life. We are in this world for a very short time and since we still have
the opportunity to hear the Dhamma and to develop right understanding of
all that appears through the senses and the mind-door, we should not
waste our life away. The understanding of Dhamma makes our life worth
living. Understanding is more precious than any kind of possession.

Visible object, sound and the other sense objects that appear are
present only for an extremely short while. As soon as they have been
experienced they are gone already, never to return. Visible object falls
away and then a different visible object arises and falls away again. It
seems as if visible object can stay for a while. We cling to shape and
form and we are taken in by the outward appearance of things. It seems
that we see people and things, but this is a delusion.

Visible object is that which is seen. It could not appear without the
citta which sees, seeing-consciousness. Seeing-consciousness is an
element that cognizes or experiences, it is nāma, whereas visible object
is rūpa, it does not know anything. Rūpas do not arise alone, they arise
and fall away in groups or units of rūpas. Each group consists of
several kinds of rūpas which always include four kinds of rūpas which
are called the four Great Elements. These are the following rūpas:

\begin{description}

\item the Element of Earth or solidity
\item the Element of Water or cohesion
\item the Element of Fire or heat
\item the Element of Wind (air) or motion

\end{description}

The Element of Earth appears as hardness or softness, the Element of
Fire as heat or cold, and the Element of Wind as motion or pressure.
These are tangible object, they can be directly experienced through the
body-consciousness when they appear. The Element of Water is not
tangible object, it cannot be experienced by body-consciousness. When we
touch what we call water it may be softness, heat or cold which are
experienced. The function of the Element of Water or cohesion is holding
together the accompanying rūpas in one group, so that they do not fall
apart.

These four Great Elements that arise with all other rūpas are their
foundation, they support them. Thus, when visible object appears, there
have to be these four Great Elements together with visible object in one
group, but they are not seen. Only visible object is seen at that
moment. The ``Visuddhimagga'' (XI, 100) states that the four Great
Elements are ``deceivers'':

\begin{quote}
``And just as the great creatures known as female spirits (yakkhinī)
conceal their own fearfulness with a pleasing colour, shape and gesture
to deceive beings, so too, these elements conceal each their own
characteristics and function classed as hardness, etc., by means of a
pleasing skin colour of women's and men's bodies, etc., and pleasing
shapes of limbs and pleasing gestures of fingers, toes and eyebrows, and
they deceive simple people by concealing their own functions and
characteristics beginning with hardness and do not allow their
individual essences to be seen. Thus they are great primaries
(mahā-bhūta) in being equal to the great creatures (mahā-bhūta), the
female spirits, since they are deceivers.''
\end{quote}

Realities are not what they appear to be. Because of saññā, the cetasika
remembrance that arises with every citta, we remember shape and form and
immediately we cling to what we believe are things and persons.

One may be infatuated by the beauty of men and women, but what one takes
for a beautiful body are mere rūpa-elements.

The ``Visuddhimagga'' (XI, 98) states that the four Great Elements are
like the great creatures of a magician who ``turns water that is not
crystal into crystal, and turns a clod that is not gold into
gold\ldots{}'' We are attached to crystal and gold, we are deceived by
the outward appearance of things. When we touch crystal or gold, only
hardness or cold is experienced. There is no crystal or gold in the
ultimate sense, only rūpas which arise and then fall away.

We cling to our body, but in reality what we take for our body are only
different elements that arise and then fall away immediately. We can ask
ourselves: ``where is our body?'' It is nowhere to be found.

We learn about the different rūpas of our body, but intellectual
understanding of what the Buddha taught is not sufficient. Acharn
reminded us all the time to pay attention and investigate the reality
appearing right now. What is past has gone already and the future has
not come yet. Learning the characteristic of what appears at this moment
is the only way to penetrate the truth of realities.

Hardness appears and we immediately have an idea of ``my hand'' or ``my
leg'', it is not understood yet as just a reality, just a dhamma. When
we think of my hand or my leg, we think of a collection of things, of a
``whole'', and that is a concept, not a paramattha dhamma. Hardness
impinges on the rūpa that is bodysense, and then it is experienced by
the citta that is body-consciousness. This is a vipākacitta arising in a
process of cittas. Cittas which experience objects through the six doors
arise in a process of cittas. When, for example, body-consciousness
arises, it occurs within a series or process of cittas, all of which
experience tangible object while they each perform their own function.
Body-consciousness is vipākacitta, it merely experiences tangible
object, it neither likes it nor dislikes it. After body-consciousness
has fallen away there are, within that process, akusala cittas or kusala
cittas which experience the tangible object with unwholesomeness or with
wholesomeness. There are processes of cittas experiencing an object
through the eye-door, the ear-door, the nose-door, the tongue-door, the
body-door and the mind-door. There is a great variety of cittas: they
can be kusala, akusala, vipāka or kiriya, which is ``inoperative''.
Kiriyacitta is neither kusala citta nor akusala citta nor vipākacitta
\footnote{Kiriyacitta performs different
functions within a process. The arahat has no more kusala cittas but he
has kiriyacittas instead.}.
After the cittas of a sense-door process have fallen away, the object is
experienced by cittas arising in a mind-door process, and after that
process has been completed other mind-door processes of cittas may arise
which think of concepts. We may think of hardness with attachment or
wrong view. We take the hardness for a hand or leg that belongs to us.

The teaching about the different processes of cittas helps us to
understand that cittas arise and fall away in succession extremely
rapidly. The processes take their course according to conditions and we
cannot do anything about them and this shows their nature of non-self
(in Pali: anattā).

When we listen again and again to the explanation of nāma and rūpa which
are conditioned dhammas, non-self, there may be conditions for the
arising of sati that is mindful, for example, of the characteristic of
hardness. At that moment paññā can begin to investigate that reality so
that it will be understood as only a dhamma.

Acharn explained that when hardness appears and there can be awareness
of it, it is not the ordinary experience of it by body-consciousness.
The object is the same, but it appears more clearly. At that moment
there is not vipākacitta but kusala citta accompanied by sati. When
direct awareness of a reality arises there is no thinking about it. When
we are thinking about realities there usually is an idea of self, we
take that reality for something or someone. We can learn the difference
between the moments with sati and without sati. When sati arises paññā
can begin to know its characteristic, it can understand it as only a
reality that does not belong to anyone. When hardness appears we tend to
think that it can stay, but it arises and falls away.

We may say that there is no self, but what is it that is non-self? We
may use the names nāma and rūpa, but more important is knowing their
characteristics when they appear at the present moment. We can learn
that what experiences and that what is experienced are different
characteristics, without naming them nāma and rūpa. Knowing a
characteristic is more important than knowing the name of a reality.

I had a beneficial conversation with Acharn about concepts we are
dreaming of and the understanding of realities.

Nina: ``I am absorbed in stories, thinking, O, I would have liked to
share this experience with Lodewijk. He would have liked this so much.
Now I cannot share this with Lodewijk.''

Acharn: ``There is no Lodewijk after his death and not even while he was
alive.''

Nina: ``I am thinking in that way because it is conditioned.''

Acharn: ``Then you are not living alone. In the lone world there is no
one.''

Nina: ``When he was alive I tried to remember that there was no
Lodewijk. There is a great deal of thinking, clinging to concepts and
dreaming about them.''

Acharn: ``How rare it is to just be aware of a reality. That can happen
when there is more intellctual understanding, sufficient to be a
condition for right awareness. By developing more understanding one will
let go of the idea of trying to know.

One may be thinking of the self and trying to understand what does not
appear.''

Nina: ``When people have worries or dreams you will always point to the
present reality. That is the only solution to our problems.''

Acharn: ``You want to have the solution with the idea of self and that
cannot be a solution at all.''

Nina: ``That is quite true, we cling to an idea of how I can solve this
problem while having dreams all the time, sadness all the time.''

Acharn: ``Actually, whose problem?''

Nina: ``Self, self.''

It was most beneficial that Acharn reminded us to what extent we cling
to a self. We do not want sadness which is akusala and we try to find
methods not to have it. There is no method. When it appears it can be
understood as just a conditioned dhamma. We should not try to change the
reality that appears already because of conditions. Ignorance of
realities can be eliminated, but courage and patience are needed to
continue developing understanding of realities.

We tend to hold on to thoughts about the past, but then we should
remember that what we find so important today will be yesterday
tomorrow. It is completely gone. We have no idea who we were in the past
life. All realities we take for a person arise and fall away never to
come back. I had a conversation with Acharn about this subject:

Acharn: ``It is not I, only the way elements are, different all the
time. No one can manage them or have them at will.''

Nina: ``I can accept this, but it is difficult for me.''

Acharn: ``That is because of clinging to the self. This will decrease
only when there is understanding of a reality as a reality. Otherwise we
are always living in a dream. Reality does not appear as it is. Today
will be yesterday tomorrow, completely gone, of no importance. No matter
what it is. It experiences something and then it falls away.''

Nina: ``The second day I was In Thailand I heard that Ivan had died and
we all went to the temple. I never thought that this would happen.''

Acharn: ``Today will be yesterday tomorrow and then you do not think
much about it. Just let it go.

You see visible object and then it is gone, like yesterday. Remembering
this helps to understand anattā. There are no conditions to choose,
realities have arisen already. Understanding this is the best in life,
otherwise there is only akusala.''

Nina: ``It was a very long, tiring day to come here, to Wang Nam
Khiao.''

Acharn: ``One can be very patient because of understanding. Everything
is just temporary, it is conditioned. Why worry about it. Right
understanding saves one from akusala.''

It is helpful to be reminded that sad events that happened the day
before are all gone. When a dear person is gone for good and will never
return we should remember that whatever reality appears now falls away
and will never return. Seeing that appears now falls away and will never
return. What we take for a person is only citta, cetasika and rupa,
elements that are beyond control. When Acharn says that we should
understand a dhamma that appears as just a dhamma, it means that we
should not take it for self or a person. Instead of thinking of a person
who will never return we should remember that each citta and each rūpa
that arises now falls away never to return again. Instead of holding on
to the world of concepts and situations, to our dreamworld, we can
develop understanding of realities so that we will see them as elements
that are beyond control. Even when we think of sad events, the thinking
is only a citta that arises because of conditions, there is not a person
who thinks.



\chapter[Understanding the Present Moment]{}
\section*{Understanding the Present Moment}

In the `` Mughapakkha Jātaka'' (no. 538) we read about the life of the
Bodhisatta as prince Temiya who pretended to be cripple, deaf and dumb.
He did not want to become a king so that he would be in a situation to
commit akusala kamma. The King wanted to find out whether he was really
cripple, deaf and dumb and let him undergo all kinds of trials and
tribulations. 

Finally the King was advised to bury him alive. When the charioteer
was digging the hole for his grave, Temiya was adorned by
Sakka\footnote{King of the Devas.}
with heavenly ornaments. He became an ascetic and preached to his
parents about impermanence:

\begin{verse}



``It is death who smites this world, old age who watches at our gate,\\
And it is the nights which pass and win their purpose soon or late.\\
As when the lady at her loom sits weaving all the day,\\
Her task grows ever less and less- so waste our lives away.\\
As speeds the hurrying river's course on, with no backward flow,\\
So in its course the life of men does ever forward go;\\
And as the river sweeps away trees from its banks upturn,\\
So are we men carried along by age and death in headlong ruin.''
\end{verse}

He explained to his father that he did not want the kingdom, stating
that wealth, youth, wife and children and all other joys do not last.
He said:

\begin{verse}

``Do what you have to do today,\\
Who can ensure the morrow's sun?\\
Death is the Master-general\\
Who gives his guarantee to none.''

\end{verse}

Lodewijk and I often spoke about the lady sitting at her loom and
weaving until her task is done. A life comes to its end so soon.

The text can remind us not to put off our task of developing right
understanding of any reality which appears now. The Bodhisatta was
unshakable in his resolution to develop right understanding. Also when
he was put to severe tests, he did not prefer anything else to the
development of wisdom. We are likely to be forgetful of what is really
worthwhile in our life. Wisdom is more precious than any kind of
possession, honour or praise.

We have learnt that what we take for a person or self are nāma and rūpa.
We were often reminded by Acharn that we may say that there are nāma and
rūpa, but that their characteristics can be known only right at the
moment they appear. Then we do not need the words nāma and rūpa, we do
not have to think about them. There is a reality that experiences and a
reality that is experienced. We pay attention mostly to the object that
is experienced but we should remember that if there is no reality that
experiences, nothing can appear, there is no world.

Acharn wanted to help us to understand the characteristic that appears
right now instead of thinking about it. When we think about seeing and
visible object we only know concepts of realities. Acharn said:

``We do not have to say that seeing is nāma, visible object is rūpa.
There is no need to say this because that is only remembrance of the
terms one has heard many times and thought about. But what about this
moment of seeing? It is so real, because whatever is seen, is seen now
and that which is seen is not that which experiences or that which sees
it. We do not have to say: `It is nāma which sees and rūpa which is
seen.' This is not necessary. That is not the way to understand it. The
way to understand it is knowing that when there is seeing right now that
this is seeing. What does it see, what is seen? The thing that is seen
is not the seeing. So, there is the beginning of understanding the nature of a reality which can be seen as just that which can be seen, not:
that which can be heard.''

Acharn kept on reminding us, saying: ``There is seeing right now, seeing
sees visible object.'' We immediately think of shape and form of things
and we do not know the distinction between seeing and thinking about
what is seen. Cittas arise and fall away in succession so rapidly that
it seems that cittas such as seeing and perceiving shape and form occur
at the same time, but in reality different types of citta arise in
different processes. Many citttas arise and pass away between seeing and
preceiving the shape and form of something, thinking of things and of
persons we believe we see. When the rūpa that is visible object or
colour associates with the rūpa that is eyesense, just for a short
moment, there are conditions for seeing. Acharn said:

``Without the reality that experiences an object, nothing can appear.
One just pays attention to what is experienced and not to that which
experiences. That which experiences can be understood as a reality.
Without it there is no world, nothing can appear. By understanding this
little by little one can know that at the moment of seeing, seeing is
not visible object but that it sees a reality, no shape and form. Now it
sees. It is very difficult to understand this because we have
accumulated a lot of ignorance. We learnt only about concepts. We can
come to understand what is meant by right understanding, paññā. It has
to be right understanding of whatever appears now. Otherwise it is not
paññā, it cannot understand the true nature of the reality which
appears. It is only thinking, dreaming about different things. We can
have theoretical understanding when we say: `what is seen is visible
object and then there is thinking of a concept'. And now? It is time to
understand the distinction between that which is seen and that which is
the object of thinking, taking it for something. Thinking of shape and
form is not thinking in words. Thinking is not always thinking in
words.''

It may seem very simple to know that seeing is the experience of visible
object and visible object is that which is seen. But this may be only
theoretical understanding. Understanding the theory is quite different
from the direct understanding of what appears at the present moment.
Acharn said: ``And now?'' The different characteristics of dhammas have
to be realized one at a time at the moment they appear, right now.
Penetrating characteristics of realities that appear is more important
than remembering their names. Whatever appears has to be realized as
just a dhamma, so that we shall really be convinced of the fact that in
reality there are only dhammas, no person or thing. Dhammas do not stay,
they are only present for an extremely short time. No one can condition
anything.

When we returned from Korat to Bangkok, we stopped on the way back at
Toscana Village for a Dhamma discussion and a lunch. The hilly landscape
is somewhat similar to Toscane in the North of Italy. The area was laid
out by way of terraces and there was an abundance of flowering trees.
After the Dhamma discussion we enjoyed an Italian style lunch. When
looking at the gardens, listening to the Dhamma discussions or tasting
the food, different sense objects impinged on the doorways of the senses
and the mind-door. We are constantly interpreting what we see, hear or
experience through the other sense-doors. This can be compared with
reading. When we are reading a book, visible object is seen, we see
black and white and then we perceive letters and interprete their
meaning. Evenso, there is just seeing, hearing, smelling, tasting and
body-consciousness and aferwards we are thinking on account of what is
experienced. This is our life: seeing and interpreting what is seen,
hearing and interpreting what is heard.

We have to carefully consider and understand each word of the teachings,
even one word, for example, the word ``dhamma''. Dhamma is reality which
has its own characteristic and which cannot be changed into something
else. When we cling to concepts which are denoted by conventional terms
such as ``tree'' or ``chair'', we do not experience any characteristic
of reality. What is real when we look at a tree? What can be directly
experienced? Visible object is a paramattha dhamma, a reality; it is a
kind of rūpa which can be directly experienced through the eyes. Through
touch hardness can be experienced; this is a kind of rūpa which can be
directly experienced through the bodysense, it is real. Visible object
and hardness are paramattha dhammas, they have their own characteristics
which can be directly experienced. We may give them another name, but
their characteristics cannot be altered. They appear only for one moment
and then they fall away. They are uncontrollable. ``Tree'' is a concept
or idea we can think of, but it is not a paramattha dhamma, not a
reality which has its own unalterable characteristic, which arises and
then falls away. Ultimate realities should be clearly distinguished from
concepts or ideas which are objects of thinking.

Intellectual understanding of the teachings is necessary but it is not
enough. It is an introduction to direct understanding. What the Buddha
taught pertains to the present moment. Only the present reality can be
really understood, not what is past or what is future. That is why
Acharn emphasized seeing now, visible object now all the time. These
have characteristics that appear and can be attended to without thinking
of their names. There can be a beginning of considering whatever appears
at this moment even though it cannot be precise.

Visible object which is experienced by seeing-consciousness does not
fall away when seeing-consciousness falls away, because it is rūpa; rūpa
does not fall away as rapidly as nāma. When an object is experienced
through one of the six doors, there is not merely one citta experiencing
that object, but there is a series or process of cittas succeeding one
another, which share the same object. When seeing-consciousness has
fallen away it is succeeded by other vipākacittas and after these cittas
have fallen away kusala cittas or akusala cittas arise. Kusala cittas or
akusala cittas arise because of conditions: kusala and akusala that
arose in the past and that have been accumulated from one citta to the
next citta conditions the arising of kusala and akusala at present. We
cannot do anything, cittas arise because of their own conditions, but
paññā can come to understand the true nature of realities and their
conditions.

After the cittas of a sense-door process have fallen away a mind-door
process of cittas follows which experience visible object through the
mind-door. After that there are other mind-door processes of cittas may
think of concepts. The Buddha taught about cittas arising in processes
according to a certain order so that people could see that they are
beyond control, that nobody can change this order.

Acharn said: ``When seeing arises who knows that it is vipāka, and when
thinking arises who knows whether it is kusala or akusala? Their
characteristics are different, one can see the difference, by not naming
them. There can be understanding that seeing is different from kusala or
akusala. Just like now: seeing sees and thinking thinks. There can be a
beginning to understand that they are so different from each other.
Understanding can grow by considering. One can know that kusala is
different from ignorance, that attachment is different from
non-attachment. There is no rule that `I' should do this or that in
order to have more kusala.''

Before the characteristics of kusala and akusala can be known precisely,
they should be understood as ``just a dhamma''. As Acharn pointed out,
the different characteristics of realities can be known by not naming
them. When we are naming them we are merely thinking about them instead
of penetrating their true characteristics.

We should have no expectations as to the arising of kusala and paññā,
that is attachment. When there is understanding that all cittas are
conditioned it helps to have less clinging to realities as self. The
Buddha taught us realities so that we can develop our own understanding
instead of blindly following what he taught.

When we were back in Bangkok we had for a whole day Dhamma discussions
on a boat. A friend of Pinna had kindly offered us this boat trip and
also the lunch that was included. We passed the Temple of Dawn (Wat
Arun) and enjoyed the familiar view of the buildings and bridges, but
now from a distance, from the waterside. After a delicious meal we
climbed off the boat to have a walk and we looked at the dazzling
colours of the shops. When we noticed all these colours we were
thinking, thinking without words. Even when we do not think in words,
the object of the citta can still be a concept. Some of the shops gave
lively presentations of walking toy animals. This conditioned our
imagination: there could be thinking of a whole story, of a real animal
who was walking. Thinking was leading us away from reality. We were
offered samples of herbal tea in small cups and it was explained that
these herbs could cure all sorts of ailments. Acharn was also walking
and then she stood still explaining for quite a while about realities
appearing right now, she was never tired.

Some of us had to take a smaller boat to return to the Peninsula Hotel.
The captain of that boat looked with approval at what was written on the
back of our shirts: ``Do good and study the Dhamma''. He had been a
monk, even an abbot, for ten years and he spoke about meditation. I
tried to explain about studying with awareness realities, no matter
where one is. There is no need for a quiet place, the realities to be
studied are within us and around us. This boat was noisy, not quiet, but
we could still discuss Dhamma, discuss about visible object appearing
through the eyes, sound appearing through the ears, many realities. It
was a good ending of the day.



\chapter[Momentary death]{}
\section*{Momentary death}

We read in the Kindred Sayings (V, Mahā-vagga, Book XII, Kindred Sayings
about the Truths, chapter V, §6, Gross darkness) that the Buddha said to
the monks:

\begin{quote}
``Monks, there is a darkness of interstellar space, impenetrable gloom,
such a murk of darkness as cannot enjoy the splendour of this moon and
sun, though they be of such mighty magic power and majesty.''

At these words a certain monk said to the Exalted One:

``Lord, that must be a mighty darkness, a mighty darkness indeed! Pray,
lord, is there any other darkness greater and more fearsome than that?''

``There is indeed, monk, another darkness, greater and more fearsome.
And what is that other darkness?

Monk, whatsoever recluses or brahmins understand not, as it really is,
the meaning of: This is dukkha, this is the arising of dukkha, this is
the ceasing of dukkha, this is the practice that leads to the ceasing of
dukkha, such take delight in the activities which conduce to rebirth.
Thus taking delight they compose a compound of activities which conduce
to rebirth. Thus composing a compound of activities they fall down into
the darkness of rebirth, into the darkness of old age and death, of
sorrow, grief, woe, lamentation and despair. They are not released from
birth, old age and death, from sorrow, grief, woe, lamentation and
despair. They are not released from dukkha, I declare.

But, monk, those recluses or brahmins who do understand as it really is,
the meaning of: This is dukkha, this is the arising of dukkha, this is
the ceasing of dukkha, this is the practice that leads to the ceasing of
dukkha, such take not delight in the activities which conduce to
rebirth\ldots{} They are released from dukkha, I declare.

Wherefore, monk, an effort must be made to realize: This is dukkha. This
is the arising of dukkha. This is the ceasing of dukkha. This is the
practice that leads to the ceasing of dukkha.''
\end{quote}

Lodewijk found this text always very awesome and he was highly impressed
by it. So long as we have ignorance there will be no end to being in the
cycle of birth and death. The Buddha showed the danger of ignorance and
exhorted the monks to develop right understanding so as to realize the
four noble Truths.

When we were having breakfast Acharn would usually join us and speak
about Dhamma. During one of our breakfasts, she reminded us of the four
kinds of right effort (samma-padhānas): the effort to avoid akusala, to
overcome akusala, to develop what is kusala, namely the enlightenment
factors
\footnote{Wholesome factors leading to
enlightenment, including the Applications of Mindfulness, confidence,
energy, mindfulness, concentration, wisdom and many others.},
and to maintain what is kusala. With regard to the first right effort,
she exhorted us not to have ignorance anymore, to avoid ignorance which
has not yet arisen. There is no self who can prevent ignorance, but
seeing its danger can condition the development of understanding. There
can be a little more understanding each day. Ignorance is not
understanding whatever appears. Not understanding is like dreaming,
Acharn said. When there is seeing, there is no one in the seeing. We
have to consider this again and again so that there will be detachment
from the idea of self or person.

Acharn reminded us that when we feel lonely, we are lonely with
ignorance, but when we understand the lone world, the world without self
or person, we can be cheerful, without problems. Then there are just
seeing, hearing and the other realities arising and falling away.

She gave us a precious reminder, saying that when one is sad and
depressed one is preoccupied with ``self'', one thinks of oneself. Such
moments can be understood as conditioned realities which arise and fall
away. When one is more attentive to the welfare of others, one will
think less of oneself.

When people would say that the development of the understanding of
realities is so difficult, Acharn would answer: ``Now you are praising
the Buddha's wisdom.'' This is true, he accumulated paññā for countless
aeons, and he developed the perfections, such as dāna, sīla, mettā or
patience. He was determined to develop them in order to reach
Buddhahood, out of compassion for all of us. Had he not become an
omniscient Buddha who could teach us all realities today, we would be
ignorant and we would be enslaved, clinging forever to sights, sounds
and all sense objects. We should also be patient and courageous to
develop paññā and all good qualities with determination. There can be a
beginning now and we should not mind how long the development of the
Path will take. We cannot expect to get rid of defilements on command,
they are anattā.

Sometimes people asked what the conditions are for sati of the level of
satipaṭṭhāna, thus, for sati which is mindful of nāma and rūpa. We read
in the Visuddhimagga'' (Ch XIV, 141) that its proximate cause is strong
remembrance (thirasaññā) or the four ``Applications of Mindfulness''\footnote{As explained in the ``Satipaṭṭhāna
Sutta'', they are: mindfulness of body, of feeling, of citta and of
dhammas.}.
Firm remembrance of the reality right now conditions satipaṭṭhāna. If we
forget that there are now only realities there are no conditions for the
arising of satipaṭṭhana. There is not sufficient understanding of anattā
to condition right awareness now.

The four ``Applications of Mindfulness'' include all nāmas and rūpas
that can be the objects of mindfulness. When they have become the
objects of sati they are a proximate cause of mindfulness. Nāma and rūpa
occurring in daily life are the objects of mindfulness. There can be
awareness of nāma and rūpa no matter whether we are walking, standing,
sitting of lying down. Also when akusala citta arises it can be object
of mindfulness, it is classified under the ``Application of Mindfulness
of Citta''. One should learn not to take akusala citta for self.

Several times Acharn reminded us that the lack of awareness was caused
because there was not firm remembrance (thirasaññā) of what we heard.
When one listens to the Dhamma and considers it again and again there
can be firm remembrance of what one has heard, and, thus, there are
conditions for the arising of sati which is mindful of the nāma or rūpa
appearing at the present moment. Thus, we see the value of listening. We
listen but we often forget what we heard. We ought to listen more, it
never is enough.

Acharn's reminder that today will be yesterday tomorrow is an
exhortation not to waste away our short time in this world as humans
where we can still listen to the Dhamma and develop understanding. How
fast time goes, before we realize it there will be the
dying-consciousness, and we do not know our future.

Knowing the theory of the Dhamma is completely different from attending
to the reality that presents itself now. Time and again Acharn reminded
us of this fact. For example, we have learnt about different feelings:
pleasant feeling, unpleasant feeling and indifferent feeling. In each
process of citta kusala cittas and akusala cittas arise and these are
called in Pali ``javana-cittas''. Seven javana-cittas usually arise in
each process. When we consider the accompanying feeling, we have learnt
that pleasant feeling can arise with kusala citta and with akusala citta
rooted in attachment, lobha. Unpleasant feeling invariably arises with
the citta rooted in aversion, thus, with akusala citta. Indifferent
feeling can arise with kusala citta and also with akusala citta, namely,
with citta rooted in attachment and citta rooted in ignorance. We have
learnt all this in theory, but feelings are realities arising all the
time in daily life. We cling to feeling and take them for mine. We could
ask ourselves: is there feeling now? It seems, when there is indifferent
feeling, that the citta is not akusala, that we do not harm anyone.
However, when our objective is not dāna, generosity, sīla, morality, or
bhāvanā, mental development, we act, speak or think with akusala citta.
Even when we listen to the Dhamma and consider it, thus, when we apply
ourselves to mental development, kusala cittas do not arise all the
time. They alternate with akusala cittas. We can see that the teachings
help us to know the extent of our defilements that arise because of
conditions, because they were accumulated for aeons, from moment to
moment, from life to life. We can understand somewhat more the nature of
anattā of the dhammas that arise. We cannot control the dhammas that
arise, but understanding of them can be developed.

Before we heard the Dhamma we had no understanding of realities, no
understanding of defilements. We accumulated more ignorance and clinging
from day to day. We should be grateful to have listened to the Dhamma
and to be able to begin developing understanding of our life, of the
truth. We learn that there are many different types of conditions for
whatever reality arises.

We find it difficult to accept that a dear person who has died will
never return. However, we should realize that each nāma or rūpa that
arises falls away and can never return. There is dying at each moment:
seeing arises and then falls away for good, and it is the same with
hearing, with the other sense-cognitions and with thinking. We shall
have more understanding of what the world is: only one moment of
experiencing one object at a time, and then gone for good. Even a person
who is alive is actually citta, cetasika and rūpa which arise because of
conditions and then fall away, which are very temporary.

Seeing dies, hearing dies at this moment, so, where are people, where is
a person? Where is a person who dies? In reality there is no person. A
moment of seeing cannot be a person, it arises and falls away. We think
that there is a permanent person who sees, who hears, but actually,
seeing is a conditioned reality that arises and falls away immediately.

Is there any difference between living in the world of concepts and
living in the world of absolute realities? What is the difference? It is
actually: living in the world of ignorance and living in the world of
right understanding. The world of concepts consists of cup, table,
person, things. But in the absolute sense, can whatever appears be
someone or something permanent? They seem to be permanent because
realities arise and fall away so rapidly. It seems as if there is no
arising and falling away of anything at all. Even the arising of seeing
does not appear and, thus, the falling away of it cannot appear.
Whatever is experienced is gone as soon as it is experienced.

From birth to death there are cittas arising in processes, vīthi-cittas,
that experience objects through the doors of eye, ear, nose, tongue,
bodysense and mind. Vīthi-cittas are alternated by bhavanga-cittas,
life-continuum, which arise in between the processes of cittas. The
bhavanga-citta which does not experience sense-objects through the
sense-doors experiences the same object as the rebirth-consciousness.
The rebirth-consciousness, paṭisandhi-citta, is vipākacitta conditioned
by kamma and this citta experiences the same object as the object
experienced by the last javana-cittas that arose shortly before dying.
The dying-consciousness, cuti-citta, experiences the same object as the
rebirth-consciousness and all bhavanga-cittas in one lifespan. The
rebirth-consciousness, the bhavanga-citta and the dying-consciousness in
one lifespan are the same type of citta. The dying-consciousness is
immediately followed by the rebirth-consciousness of the next life and
then one is no longer the same individual. However, all accumulated
kusala and akusala go on to the next life, they go on from life to life.
Thus, the cycle of birth and death goes on until the dying-consciousness
of the arahat. Then the end of the cycle has been reached.

Acharn reminded us of three kinds of citta: The first citta (in Pali:
paṭhama citta) is the bhavangacitta before anything appears. When
something appears, such as seeing, hearing, there is the second citta
(in Pali: dutiya citta). Finally there is the dying-consciousness,
cuti-citta, of the arahat (in Pali: pacchima citta, the last citta).
Each life is like this: the rebirth-consciousness arises and then
bhavangacittas arise and the object is unknown, nothing appears. When
something appears there are process cittas, the second kind of citta. In
this way life keeps going on from moment to moment, from birth to death,
again and again, until the last moment of the arahat.

At this moment we are in the cycle of birth and death, saṁsara.
Yesterday there were seeing and thinking, hearing and thinking, and
today it is the same, and so it will be in the future. We are absorbed
in the objects we experience, time and again, and this is the cycle of
birth and death.

Acharn said:

``What has disappeared does not return again. Realities arise and fall
away, arise and fall away. Should one cling? Then one would cling to
what has fallen away and is no more. Where should we find it? There is
clinging because one does not know the truth. There are only realities
that arise and fall away in succession and this does not stop. This is
the cycle of birth and death, saṁsara. There is an opportunity to begin
to understand this.''

So long as ignorance and clinging have not been eradicated we continue
being in saṁsara. If we do not develop insight, vipassanā, the number of
rebirths will be endless. It was out of compassion that the Buddha spoke
about the dangers of rebirth; he wanted to encourage people to develop
right understanding of the reality appearing at this moment.

Acharn was emphasizing all the time the value of understanding this
moment of seeing, hearing, thinking and all realities that appear. This
helped me to see the disadvantage of being absorbed in sad events that
happened in the past and of clinging to what has fallen away and will
never return. Such ways of thinking are conditioned and instead of
trying to avoid thinking we can learn that also the thinking that arises
can be understood in order to know it as not ``mine'', as only a dhamma.

The contrast between living in a dreamworld while clinging to the past
and beginning to understand the world of paramattha dhammas became more
obvious to me than before. The difference between those two worlds is
actually most striking. I am very grateful to Acharn for pointing this
out time and again, in many different ways.

We listen to the Dhamma in order to have more understanding of the
present moment. During this journey it became clearer to me that
listening to the Dhamma is the most precious in life.


