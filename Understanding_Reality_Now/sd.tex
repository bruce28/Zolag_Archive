\part{Sharing Dhamma}

\chapter{Preface}

The Buddha developed all the
perfections\footnote{The perfections or pāramīs are: generosity, wholesome behaviour (kusala sīla),
renunciation, wisdom, energy, patience, truthfulness, determination,
loving kindness, equanimity. The Buddha developed these for aeons in
order to become the Sammāsambuddha.}  with right
understanding of realities to the greatest extent so that he could
become a Sammāsam\-buddha who teaches the truth of realities to the world.
His teaching is the most precious gift to us all. He taught in such a
way that the listeners could develop their own understanding. It is
thanks to the Buddha that we can still listen to the Dhamma and that we
can meet for discussions. In this way we can share what we hear with
others who are interested in the teachings.

Our Vietnamese friends are very keen to learn and understand the truth
of the Dhamma and they use every occasion to listen, consider and
discuss the Dhamma. They organise several times a year sessions with
Acharn Sujin in different locations in Vietnam. Acharn Sujin's book
``Survey of Paramattha Dhammas'' and my ``Buddhism in Daily Life'' were
translated into Vietnamese. At this moment Tam Bach is translating my
``Conditionality of Life''. During this journey she asked many questions
on the different conditions, paccayas, in order to facilitate her
translation. They are a very active group of friends who are truly
dedicated to make known the Dhamma.

They had invited Acharn Sujin and her sister Khun Sujit to Vietnam in
January 2016 for a ten days sojourn. They sponsored their flight and
hotel accommodation. Sarah and Jonothan were assisting Acharn during her
Dhamma explanations and friends from Thailand, Canada, Australia and
myself joined this journey. In Vietnam Tiny Tam, Tran Thai's wife, made
all the traveling and accommodation arrangements for us.

The Dhamma discussions took place in Saigon, in a hall near our hotel.
Among the audience were usually two monks, many ``nuns'', that is to
say, women who wear robes and observe eight precepts, and many lay
followers. Tam Bach translated into Vietnamese the English Dhamma
discussions and a team of Vietnamese friends translated the questions
from the audience into English.

Our Vietnamese friends took great care of all our needs and they were
most inventive in taking us to different restaurants for luncheon. Once
I had a bad cough, and they prepared a drink called ``birds' nest'' with
great loving care. It proved to be very effective, even after a few
days.

Throughout our journey Acharn reminded us not to move away from the
present object. Whatever is real appears now. Seeing is a conditioned
dhamma, it cannot be a person who sees. I am grateful to hear again and
again what she repeats because the development of understanding takes a
long time to become well established.

\chapter[The World]{}
\section*{The World}


What is the world? We live in the world of persons, of self, of
different things. Before hearing the Buddha's teachings we did not
really understand what the world is. Because of ignorance of the world
we cling to our possessions, our family and friends. In truth the world
is citta (moment of consciousness), cetasika (mental factor accompanying
citta) and rūpa (physical phenomena). Seeing is a citta, experiencing
visible object, just for a moment and then it falls away. Thinking is
another citta that thinks of persons and different things like a table
or tree. Thinking may be unwholesome, akusala, and then it is
accompanied by ignorance and other mental factors, cetasikas, that are
unwholesome such as attachment or aversion. Or thinking may be
wholesome, kusala, and then it may be accompanied by kindness,
compassion or understanding of realities. There could not be seeing
without eyesense which is a physical reality, rūpa. Also what is seen,
visible object, is a type of rūpa which conditions seeing by being its
object.

We cling to our body and we believe that it belongs to us. However, what
we take for body is in reality different rūpas that arise and fall away.
They arise because of different conditioning factors and they are beyond
control.

Citta, cetasika and rūpa arise just for a brief moment because of their
proper conditions and then they fall away immediately. When seeing
arises, only visible object is experienced and both seeing and visible
object do not last, they fall away immediately. When hearing arises
sound is experienced and both hearing and sound fall away immediately.
There cannot be seeing and hearing at the same time; only one citta
arises at a time and experiences one object through one of the six
doorways of eyes, ears, nose, tongue, bodysense and mind-door. Thus,
actually, there are six worlds, appearing one at a time: the world of
the experience of visible object, of sound, of odour, of flavour, of
tangible object and the world of thinking.

Because of ignorance and clinging we have to reborn again and again. So
long as there is birth there have to be old age, sickness and death.
During life we have to experience a great deal of sorrow. The Buddha
taught the end to rebirth by the development of right understanding of
all realities that appear. During the discussions there was reference to
the ``Rohitassa Sutta''.

We read in the ``Gradual Sayings'', Book of the Fours, Ch V, §5,
``Rohitassa'', that the deva Rohitassa asked the Buddha:

\begin{quote}
``Pray, lord, is it possible for us, by going, to know, to see, to reach
world's end, where there is no more being born or growing old, no more
dying, no more falling (from one existence) and rising up (in
another)?''
\end{quote}

The Buddha answered that that end of the world is not by ``going'' to be
reached. Rohitassa said that formerly he was the hermit Rohitassa of
psychic power, a sky walker. The extent of his stride was the distance
between the eastern and the western ocean. Rohitassa said:

\begin{quote}
``Though my lifespan was a hundred years, though I lived a hundred
years, though I traveled a hundred years, yet I reached not world's end
but died ere that.''
\end{quote}

He praised the Buddha saying, how well it is said by the Exalted One
that the world's end is not by going to be reached. The Buddha said:

\begin{quote}
``Nay, your reverence, in this fathom-long body, along with its
perceptions and thoughts, I proclaim the world to be, likewise the
origin of the world and the making of the world to end, likewise the
practice going to the ending of the world.
\end{quote}

\begin{verse}
Not to be reached by going is world's end.\\
Yet there is no release for man from Dukkha\\
Unless he reaches world's end. Then let a man\\
Become world-knower, wise, world-ender,\\
Let him be one who lives the divine life.\\
Knowing the world's end by becoming calmed\\
He longs not for this world or another.''\\
\end{verse}

Through the development of right understanding of mental phenomena (in
Pali: nāma) and physical phenomena (in Pali: rūpa) one can become a
world-knower, wise, a world-ender. The objects of right understanding
are not far away, they are ``this fathom-long body, along with its
perceptions and thoughts'', thus, whatever mental phenomena and physical
phenomena appear in daily life. For him who has completely developed
understanding of the world at this moment and eradicated all defilements
there will not be any rebirth, no more world at any time.

Acharn emphasized time and again that realities such as seeing, visible
object, hardness or sound appear all the time in daily life and that
understanding of them can be developed very gradually. One does not have
to go to a quiet place and concentrate on realities. She reminded us
that intellectual understanding of what appears now, pariyatti, can
gradually condition direct understanding, paṭipatti, and this again
conditions paṭivedha, the direct realization of the truth. Many times
she said that each word of the Buddha points to the truth and leads to
detachment. Study of the teachings is not for speculation or memorizing
but for the understanding of this very moment.

In Saigon there were about a hundred listeners every day. They listened
with great interest and many questions were raised. The topics that were
discussed were the meaning of non-self, anattā, the perfections, free
will and fatalism, meditation and many other subjects. Every day, before
the morning session, a discussion was held in a smaller group with
venerable Bhikkhu Silavamsa. He was first ordained as a Mahāyana monk
and then he became a Theravada monk. He read my ``Letters on Vipassanā''
and became interested to listen to Acharn's explanations and attended
Dhamma discussions with her since 2013. He wanted to study all
twenty-four classes of conditions for realities as laid down in the
``Paṭṭhāna'', the seventh book of the Abhidhamma. But the study of
conditions should not just be book study, theoretical knowledge. We
cannot find out by how many conditions, paccayas, this moment is
conditioned. Acharn always referred to understanding this moment of
reality that arises. Seeing is conditioned by visible object and
eyebase; without these rūpas, seeing could not arise. Visible object and
eyebase are different realities which have different conditions for
their arising. The Buddha taught the conditions for the realities which
arise so that people could understand more deeply the truth of non-self,
anattā. Whatever arises because of conditions cannot be directed by a
self, it is beyond anyone's control. Every day we discussed some aspects
of the conditions which pertain to our daily life. Acharn emphasized the
importance of deeply considering the meaning of the conditions for the
realities that arise at this moment.

The texts we read about conditions always pertain to this moment.
Whenever seeing arises there must be that which is seen, visible object.
Visible object is object-condition for seeing, but we do not have to
name it object-condition. We can come to understand this condition by
understanding the reality appearing at this moment. Many times Acharn
reminded us that life is very short and that what is most valuable is
understanding this moment. Learning about conditions supports the
understanding of seeing and hearing as not self. It depends on the
individual's accumulated understanding to what extent he can penetrate
the truth of the different conditions.

Problems we have in daily life were also discussed and as Sarah stressed
many times, the real problem is our own thinking with defilements.
Instead of developing understanding of what is real at this moment, we
continue to think of difficult situations and persons and we worry a
great deal. Then we live in a dream world. Sarah said that there is
always something to worry about, that there is no end to it. Not the
circumstances or other people are the cause of our problems, the only
problem is our unwholesome thinking arising now.

Sometimes we had to walk in the dark and this caused fear to me since I
walk with difficulty. One night I had to take extremely high steps to
get into a friend's car and then I was surprised to see Acharn sitting
in the car already. She spoke to me about my thinking of fear in the
dark and thinking that this would be my last trip, always thinking of
``self''. She reminded me that I have to build up courage, otherwise I
will take fear with me from life to life.

The greatest courage is perseverance with the development of
understanding of whatever appears now, no matter what the circumstances
of our life are. We may come across the greatest difficulties and
problems, but there are always seeing, visible object, hearing or
attachment with different characteristics to be known.

Sarah had a conversation with Glen, Ann's husband, about life. She spoke
about basic notions of Dhamma and she repeated this conversation when we
stayed in Thailand, in Kaeng Krachan. Some people may not think that it
is important to know more about realities such as seeing or hearing but
Sarah explained the relevance of having more understanding of the
realities of our life. She said:

``We believe that we see people and things, but actually just that which
is seen, visible object, is experienced. Immediately after that there is
thinking about people, chairs and trees. This is thinking, not the
experience of that which is seen, visible object. The question is
whether life just continues as usual with ignorance, or whether we are
interested to understand a little more about life. What we usually find
most important is our family, our work or our possessions, for example.
What is the purpose of having more attachment to what we find important
in life? For some of us what is most important of all is having more
understanding about the truth of life. When people are dreaming and wake
up it is very clear that it was a dream world, a fantasy world, not
real. The dream seems so real, but it is just fantasy, an illusion. What
about now? What about the dream we are having at this moment? When
driving along the road one thinks that one sees a lake, but when one
gets closer one knows that it was just a mirage. Also now we are living
in a world of mirages and fantasies. As soon as we see people, trees and
different things it is the world of mirages and fantasies.''

As Sarah said, when we are dreaming our experiences such as seeing seem
very real. But actually, at such moments there is no seeing, there is
not visible object impinging on the eyesense. When we are asleep our
eyes are closed and while dreaming we are just thinking of different
ideas. Even so, when we are awake we believe that we are seeing when
seeing has fallen away already and we are just thinking of different
things, of people. Seeing is one extremely short moment of citta and
after it has fallen away many other types of citta arise. We take for
seeing what is actually thinking of mirages.

Sarah explained:

``Only visible object is seen now, only sound is heard. We think that we
hear the sound of birds, of traffic, sound in the microphone.
Immediately there is an idea of people and things, all the time.
Actually, only sound is heard. There is just the experience of one world
at a time, through eyes, ears, nose, tongue, bodysense and mind and that
is all. It seems that `I' have such experiences, that I am seeing or
hearing, but who is the `I' that is seeing or hearing? It is just one
moment of experience followed by moments of thinking. We can learn that
what we take for `I' is a mirage, an idea. There is just a moment of
seeing, experiencing that which is seen and then thinking about it.
Different realities, each one arising because of its own conditions.
This is life, different moments of experience arising because of
different causes or conditions which no one can control or make arise at
will.

It is not a matter of the terms, the details or the books, but it is
about what can be tested now at the moment of touching what we take for
a table. Just hardness that appears now. Just touching of hardness. We
may be thinking of a table, but that is not the reality that is
experienced. When there is an idea of table we are back to the world of
ideas. Images are not real, they are mind-created. One may think of the
brain, but who could control the brain? That is a scientific outlook,
not the present moment.''

Acharn said about the brain:

``We are told that the brain experiences. What is seen are different
colours. We hear the word brain and think about its function but it
cannot experience anything at all. When it is touched it is hard and
hardness cannot experience anything at all. We can see a picture of it,
but when it is seen, visible object cannot experience anything. Brain is
that which cannot experience or know anything. It cannot know, it cannot
do anything at all.''

Brain is conventional truth one may think of, and it is different from
what is real in the absolute sense: the physical phenomena or rūpa and
the mental phenomena or nāma that appear one at a time at the present
moment.

Acharn explained: ``This is what we are interested in: to learn more, to
have less selfishness, less clinging to the self, the ego, the `I',
because it harms. All unwholesome realities harm oneself and the other.
We cling to `I', everything must be to our liking, from moment to
moment. We search for pleasure from that which is experienced, from
every doorway, from seeing, hearing, smelling, tasting and the
experience of tangible object. Do you think that attachment is
wholesome?''

Glen answered: ``Attachment is harmful, but it is human.''

Acharn said: ``That is the right answer. It is there, it is natural. It
is conditioned, no one can do anything about it. It is conditioned to
arise and fall away, like all conditioned realities. Each reality has
its own characteristic and its own function.''

We were often reminded that the development of understanding has to be
natural. Whatever reality arises cannot be changed, there were
conditions for its arising. It has fallen away already when we think
about it. It is valuable to hear again and again about basic dhammas,
about what life really is from moment to moment. It is different from
what we used to believe before hearing the Buddha's teachings. We learn
that what we take for `I' are citta, cetasika and rūpa only. No self
sees, but seeing sees, just by conditions. This is life.



\chapter[The Meaning of Anattā]{}
\section*{The Meaning of Anattā}


The Anattā Lakkhaṇa Sutta is the Buddha's second sermon. We read in the
``Kindred Sayings'' (III, Kindred Sayings on Khandhas, Elements, Middle
Fifty, § 59) that Buddha said in the Deerpark in Vārāṇasī:

\begin{quote}
``Body, brethren, is not the Self. If body, brethren, were the Self,
then the body would not be involved in sickness, and one could say of
body: `Thus let my body be. Thus let my body not be.' But, brethren,
inasmuch as body is not the Self, that is why body is involved in
sickness, and one cannot say of body: `Thus let my body be. Thus let my
body not be.',''
\end{quote}

He said the same about the four nāma
khandhas\footnote{All conditioned
realities can be classified as five khandhas or groups: one is
rūpakkhandha, the khandha of physical phenomena, and four are
nāmakkhandhas, citta and cetasikas. Saṅkhārakkhandha are all cetasikas
apart from feeling and saññā, remembrance.} of feeling,
perception (saññā), the activities (saṅkhārakkhandha) and consciousness.
We then read:

\begin{quote}
``Now what think you, brethren. Is body permanent or impermanent?''

``Impermanent, lord.''

``And what is impermanent is that weal or woe?''

``Woe, lord.''

``Then what is impermanent, woeful, unstable by nature, is it fitting to
regard it thus: `this is mine; I am this; this is the Self of me'?''

``Surely not, lord.''

``So also it is with feeling, perception, the activities and
consciousness. Therefore, brethren, every body whatever, be it past,
future or present, be it inward or outward, gross or subtle, low or
high, far or near- every body should be thus regarded, as it really is,
by right insight- `this is not mine; this am not I; this is not the
Self of me.',''
\end{quote}

He then said the same about the four nāma khandhas. We read:
\begin{quote}

``So seeing, brethren, the well-taught Ariyan disciple feels disgust for
body, feels disgust for feeling, for perception, for the activities,
feels disgust for consciousness. So feeling disgust he is repelled;
being repelled he is freed; knowledge arises that in the freed is the
freed thing\footnote{There is the knowledge
that the mind is liberated.}; so that he
knows: `destroyed is birth; lived is the righteous life; done is my
task; for life in these conditions there is no hereafter.' ''
\end{quote}

After this sermon the five disciples who were from the beginning with
the Buddha became arahats.

During our discussions one of the listeners said that the meaning of
anattā is: nothing, no reality. He thought that it means that in the
past, present and future there is no reality. Jonothan asked him whether
there are dhammas now. The meaning of anattā cannot be understood merely
by thinking about it. We should have more understanding of the dhammas
that are arising at this moment. Seeing that arises at this moment is
not self, it is seeing that sees. Nobody can make seeing arise. When
understanding of the level of pariyatti develops, thus, intellectual
understanding of the present reality, the meaning of anattā becomes
clearer very gradually. We cannot expect to fully understand the meaning
of anattā after a few years or even after one lifetime of listening to
the Dhamma and considering it. But very gradually one can become closer
to understanding of whatever appears now. Otherwise we are lost in words
and ideas.

Someone else remarked that if all kusala and akusala arise because of
conditions we cannot do anything. Some people believe that this would
lead to fatalism, to being subject to fate. Acharn answered that what we
take for ``I'' are different realities arising because of different
conditions. In other words, also wholesomeness and unwholesomeness are
cittas accompanied by cetasikas, not self. It makes no sense to hold on
to the idea of someone who can do something to have more kusala. People
believe that if there is no free will, responsibility for one's actions
is denied. Volition or intention is a mental factor, a cetasika, cetanā
cetasika, accompanying every citta. It is conditioned. When there are
conditions for kusala citta, cetanā is also kusala and when there are
conditions for akusala citta cetanā is also akusala. Right understanding
of realities can condition more kusala in one's life.

That may be a beginning of the development of the perfections. In our
life there is usually akusala citta and akusala citta cannot understand
any reality. When kusala citta arises there are no attachment, aversion
or ignorance. The perfections which are actually kusala through body,
speech or mind, support right awareness and right understanding of
realities. Kusala without understanding is not a perfection. Firm and
keen understanding of whatever appears now needs the perfection of
truthfulness (sacca) and this can condition detachment from the idea of
self. When the truth is known that there is no person, that there are
only conditioned elements, there is a degree of detachment that develops
along with understanding. Also the perfection of determination
(adiṭṭhāna) and the other perfections are needed. One has to be firmly
determined not to move away from the present object since the
understanding of seeing or visible object that appears now is the only
way to understand that there is no one there.

The perfection of patience is indispensable for the development of
understanding. Acharn said: ``Not just be patient towards cold and heat,
but understand even such moments as not self who is patient.'' There can
be patience when we experience an unpleasant object, but also when the
object we experience is pleasant there can be patience. When attachment
arises on account of a pleasant object there is no patience.

Very often there is the idea of ``I am patient''. For example, we were
enjoying ourselves sitting in a restaurant outside, but there was mostly
seafood which I cannot eat because of an allergy. But I found myself
very patient, not complaining eating plain rice. Sarah said that we have
a conventional idea of patience. When there is the perfection of
patience there has to be the understanding that it is not I who is
patient.

When listening to the Dhamma and considering it one should not be
impatient and expect clear understanding of realities very soon. Acharn
explained that when one has heard about right understanding that can
penetrate the true nature of reality, there may be attachment and
ignorance trying to attain this goal. That is not detachment from
wanting to experience it. There are so many traps of attachment and that
is why life continues in the cycle.

Robert Kirkpatrick came to visit us in Saigon with his wife and two
young children, Ryan and little Nina, a baby of five months old who was
named after me. Acharn was so kind to give us all an opportunity for
Dhamma discussion in the evening after the usual morning and afternoon
sessions. The discussion was held in Robert's room. Ryan was sweetly
playing with his toy cars and now and then he needed attention and the
approval from his father. He was no longer the only child and his lovely
little sister received a lot of attention. This happens to adults as
well. We like to have other people's attention and to be approved by
them. At such moments we cling to the importance of self, there is
conceit. Without the Buddha's teaching we would not know when there is
conceit.

Acharn explained that when intellectual understanding of the level of
pariyatti has become firm it can condition right awareness of the level
of satipaṭṭhāna. There is not any idea of ``I prepare, I will do'' but
right awareness arises unexpectedly by conditions. It is most important
to remember that understanding is anattā and that it develops because of
conditions. We should not turn away from the present object and look for
a specific method to cause understanding to grow.

One of our friends remarked: ``The more we hear, come to the sessions
and hear new terms, the more we become agitated by intellectual
understanding.'' Acharn answered: ``It indicates that it is `you' who
studies, there is no understanding of realities as not `you'. It brings
about the idea of `shall I do this, shall I do that'. That is the wrong
study of Abhidhamma because it does not eliminate the idea of self. How
deeply rooted it is.''

She explained that only the Buddha's words can condition right
understanding, not one's own thinking or the guidance of someone else.
There can be very firm confidence in the teachings. Realities arise by
conditions and fall away again. Even doubt is real, it is not self.
Usually we have an idea of my doubt, but paññā can see it as it is. She
said: ``It does not matter what reality arises, it is conditioned. That
is the way to let go of the idea of self; by understanding that
particular object, right then.''

People often wonder what they should do to have more wholesomeness in
their life and more understanding. It is important to remember that
there is no doer, no person who can do anything at all. Only citta
accompanied by cetasikas can perform functions. These are fleeting
phenomena, they fall away instantly. When we think of them they have
fallen away already, so, how can they be a self that is doing or acting?
Intellectually, this can be understood, but realizing the truth at the
present moment when citta, cetasika or rūpa appears that is another
matter. We are so used to take them for ``self'' or ``mine'', and this
wrong idea cannot be eliminated soon. Intellectual understanding of
seeing now, hearing now, thinking now can become firmer and then it can
be a condition for satipaṭṭhāna, direct awareness of whatever appears.
There should be no selection of the object of mindfulness, the object is
just whatever appears by conditions. We cannot select seeing as object
of mindfulness, maybe at a given moment there is a condition for
hearing, and only one reality at a time can be object of mindfulness. It
is totally unexpected what reality appears, we never know the next
moment: it may be kindness or selfishness. In this way the truth of
anattā can become a little clearer: there is no doer, there is no one
who makes such or such dhamma arise.

Kusala cittas and akusala cittas alternate all the time and we may
wonder how we can know when there is kusala citta and when akusala
citta. When we rejoice in other people's kusala, kusala citta arises,
but in between also akusala cittas arise when we are attached to them. A
kind and generous friend brought us several times at luncheon a special
dessert. We appreciated her kindness but also akusala cittas with
attachment arose. Acharn remarked: ``Everything is dhamma.'' We are
likely to forget this and we take kusala citta and akusala citta for
self. When we find it important to have kusala citta instead of akusala
citta we cling again to the idea of self. Everything is dhamma,
non-self. Whatever arises is conditioned.

Jonothan remarked that the teaching is about the understanding of the
presently arisen dhammas. It does not matter whether there is attachment
or not, also attachment can be object of awareness and right
understanding. We should not limit the dhammas that can be object of
awareness. Instead of just wanting to have more kusala in a day one
should have more understanding of whatever dhamma arises at the present
moment.~

During the discussions several questions about samatha or the
development of calm were raised. Acharn explained that we should
consider what true calm is: being away from akusala. In samatha calm is
developed to a high degree so that jhāna can be attained.

Calm suppresses the
hindrances\footnote{The hindrances
(nīvaraṇa) are the defilements of sensuous desire, ill will, sloth and
torpor, restlessness and worry, doubt.} and it is
opposed to restlessness, uddhacca.

The aim of samatha is to be free from sense impressions that are bound
up with defilements. Right understanding is necessary for the
development of calm, there has to be precise understanding of the
characteristic of calm so that it is known when kusala citta with calm
arises and when attachment to calm arises.

There is also calm in the development of insight. Acharn explained that
every moment of kusala is calm, there are no attachment, aversion or
ignorance. Every kusala citta is accompanied by the cetasika calm,
passaddhi\footnote{Actually, there are two
cetasikas which are passaddhi: calm of body (kāya passaddhi) and calm of
mind (citta passaddhi). Calm of body pertains to the mental body: And
here `body' means the three (mental) aggregates, feeling, perception and
formations, see Dhs.40.}. When there is
right understanding of nāma and rūpa, the six doors are guarded at that
moment and there is true calm. When one has not heard the Buddha's
teachings, one knows about good and bad deeds, but there is no precise
understanding. There is no understanding of realities as non-self. One
may take for calm what is not calm and cling to a conventional idea of
calm which is actually a feeling of relaxation with attachment.

People are inclined to believe that there is awareness, sati, when one
observes what one is doing. Sarah said that the idea of observing is not
sati that naturally arises. No one can stop akusala citta from arising.
Through more understanding of dhammas as anattā there will be less the
idea of observing or selecting an object of awareness.

At the end of the sessions one of the nuns spoke very well on the
development of right understanding. She used to think that she could
attain nibbāna during this life, but now she realized that the
development of understanding is bound to take a very long time. She had
understood that one should not cling to terms, but truly understand
characteristics of realities. She appreciated the Dhamma she had heard
and she had understood that the conditions for understanding are
listening and wise consideration of what one hears.

\chapter[Diverse Topics of Discussion]{}
\section*{Diverse Topics of Discussion}



``This life was the future life of last life, and it will be the past
life of the next life.'' Acharn repeated this many times during our
journey in order to remind us that we are at this moment in the cycle of
birth and death. We cling to our family and friends, but they will not
follow us to the next life. We find our life very important, but this
life will be past life very soon, since each life does not last long.
Then we will not remember who were our dear ones.

During all the discussions in Vietnam and also in the different
locations in Thailand where we stayed, questions were raised about life
and death, kamma and its result, conditions for kusala and akusala, the
development of calm and of insight. In all her answers, Acharn would
help the listeners to understand the present reality. She said many
times that each word of the Buddha's teachings leads to right
understanding of reality now.

She said that we should even understand one word: ``dhamma'', and she
explained: ``Dhamma is a reality, but it is no one. At the moment of
seeing it is a dhamma. At the moment of hearing it is a different
dhamma. By different conditions it arises and falls away, there is no
self, no one at all.''

Whatever is real is dhamma. Time and again a dhamma appears through one
of the senses or the mind-door. Whatever appears now is a reality that
is conditioned. Otherwise it could not arise at all. Paññā of the level
of pariyatti can begin to understand the characteristic of whatever
appears, and there is no need to name it. Pariyatti is not book study.
Realities work their own way by themselves, no one can make them arise.
When this has been understood we are less inclined to cling to a
collection of things, a whole, or a being, like before. Understanding
leads to detachment.

There are two kinds of reality: the reality that experiences an object
or nāma and the reality that does not know anything, or rūpa. Seeing
experiences an object, it is nāma, whereas visible object does not know
anything, it is rūpa. When seeing arises there is also visible object
but they have different characteristics and these can be known one at a
time. This cannot be realized in the beginning, but when paññā is more
developed it can distinguish their different characteristics.
Understanding of the level of pariyatti is not sufficient yet to
understand directly just one reality at a time.

Acharn reminded us time and again: ``Study one reality at a time, until
one sees it as not self. We may say: `seeing is a reality', but that is
not enough. We should be careful in considering what is heard: `Seeing
sees what?' It takes time to get used to the fact that what appears is
not self. It is only a dhamma that can impinge on the eyebase and that
arises and falls away. When we think: 'I see someone or something' the
understanding is not enough.''

When we cling to the idea that a flower is seen, a table is seen, we
have to listen to the Dhamma and consider the truth again and again in
order to have less ignorance. The Buddha and his great disciples
recognized different people and saw different things such as a mountain
or table, but they had no wrong view, they clearly distinguished between
ultimate realities and conventional truth or concepts that can be
objects of thinking.

People usually think of life and death in conventional sense. However,
the Buddha's teachings lead to the understanding of what is real in the
ultimate sense: citta, cetasika, rūpa and nibbāna. Everything else is
conventional truth, not ultimate truth. The last moment of life is a
citta, the dying-consciousness, cuti-citta. It depends on kamma when it
is time for the arising of the cuti-citta and nobody can prevent its
arising. It is immediately followed by the rebirth-consciousness,
paṭisandhi-citta, of the next life. The dying-consciousness and the
rebirth-consciousness are both results of kamma, vipākacittas. Kusala
citta and akusala citta may have the intensity to motivate deeds, kusala
kamma and akusala kamma through body, speech or mind. Kamma is actually
the cetasika volition or cetanā. When it is kusala kamma or akusala
kamma it can produce result, vipāka. Kamma can produce vipāka in the
form of rebirth-consciousness and in the course of life by way of
pleasant and unpleasant experiences through eyes, ears, nose, tongue and
body-consciousness.

Kusala kamma and akusala kamma are mental and, thus, they can be
accumulated from one citta to the next citta and so also from past lives
to the rebirth-consciousness of this life. All the kammas that have been
accumulated and are carried on to the rebirth-consciousness have the
potential to produce their appropriate results in the following lives.
The rebirth-consciousness is followed by other cittas, bhavanga-cittas
or life-continuum. These cittas are also vipākacittas produced by the
same kamma that produced the rebirth-consciousness and they experience
the same object as the rebirth-consciousness. We do not know what that
object is, it is not experienced through any one of the six doorways.
They arise throughout life at those moments that there is not the
experience of objects through one of the six doorways by cittas arising
in processes of cittas, such as seeing, hearing or thinking. Thus, they
arise time and again in between the different processes of cittas. They
also arise in deep sleep, when we are not dreaming. They keep the
continuity in the life of an individual.

Seeing is vipākacitta and it arises in a process of cittas. It is
preceded and followed by other cittas that do not see but nevertheless
experience visible object while they perform other functions. A rūpa
such as visible object lasts longer than citta, it is experienced by
several cittas arising in a process. Seeing is only one moment of
experiencing visible object and it falls away immediately. Very shortly
after it has fallen away kusala cittas or akusala cittas that experience
the same visible object arise and fall away very rapidly. They
experience it with wholesomeness or unwholesomeness and this is
conditioned by the wholesome or unwholesome inclinations that have been
accumulated from one citta to the next one. When that sense-door process
is over it is followed by a mind-door process which experiences visible
object through the mind-door and later on there are other mind-door
processes which think about the visible object.

During our life the experience of pleasant objects and unpleasant
objects alternate: gain and loss, fame and obscurity, praise and blame,
bodily wellbeing and pain. These are among the ``worldly conditions''
the Buddha spoke about. The moments of vipāka are extremely brief and
when we think of the source of our experiences, they are already gone.
The moments of thinking are no longer vipāka, but usually akusala cittas
and these are conditioned by the accumulation of akusala in the past.
This type of condition is different from kamma that produces
vipāka\footnote{Kusala kamma or akusala
kamma that produces later on its appropriate result, vipāka, is one type
of condition, kamma-condition. Wholesome and unwholesome inclinations
that have been accumulated can condition the arising of kusala citta or
akusala citta later on and that is another type of condition: natural
decisive support-condition, pakatūpanissaya-paccaya.}. When someone else
speaks in a harsh way to us, we are inclined to blame that person and we
take the unpleasant experience for ``mine''. Then we do not think wisely
about cause and result. In the ultimate sense there are only conditioned
realities that just arise and appear very shortly. Kamma produces
hearing which is vipāka, and thinking with akusala citta is caused by
our accumulated defilements. There is no person who inflicts sorrow upon
another person and no person who experiences it. There are only
conditioned realities arising and falling away.

Cittas arising in a process do so in a specific order while they perform
each their own function. The citta that adverts to an object that
presents itself arises before seeing or another one of the
sense-cognitions, and kusala cittas or akusala cittas arise later on in
that process. One may wonder what the use is of knowing such details.
The Buddha taught proximity-condition, anantara-paccaya of cittas,
meaning that cittas succeed one another: when one citta falls away it is
immediately succeeded by the next one. He also taught
contiguity-condition, samanantara-paccaya, meaning that cittas succeed
one another in a fixed order that cannot be altered. This clearly shows
the nature of anattā of cittas, they cannot be directed or controlled.
There isn't anybody who is master of the cittas that arise and could
change the order of their arising.

We are in the cycle of birth and death right now. Vipākacitta arises by
way of an experience through one of the sense-doors, and then
defilements are likely to arise. These may motivate kamma and kamma
produces vipāka. Again defilements will arise and motivate kamma that
produces vipāka. In this way the cycle goes on and on. All this occurs
now, anywhere, at any time.

Many times Acharn said that we should not cling to names and terms, but
that we should understand the characteristics of realities that appear.
Citta is the faculty of experiencing an object and it is assisted by at
least seven cetasikas that accompany it, such as feeling, remembrance
(saññā), one-pointedness (ekaggata cetasika). When we read about
cetasikas such as energy or effort (in Pali: viriya), we may think of
their meaning in conventional sense, and then misunderstandings may
arise. We believe that a self can make an effort to have more kusala and
right understanding. We should remember that the cetasika viriya may
arise with many cittas though not with all cittas, not with seeing and
the other sense-cognitions and a few other cittas. Thus, effort may be
kusala, right effort, or akusala, wrong effort. When we make an effort
to have kusala citta, we are likely to cling to an idea of self who
wants to direct cittas. Without knowing it we may take wrong effort for
right effort. When understanding of the present reality is developed
more, it is paññā that will know when citta and cetasikas are akusala
and when kusala. We cannot find out by thinking about it. Several times
we were warned not to try to work things out, then there is just
thinking with an idea of self behind it.

We learn from the texts that some cetasikas are roots, hetus. Ignorance
(moha) attachment (lobha) and aversion (dosa) are three akusala hetus.
Understanding (paññā or amoha), non-attachment (alobha) and non-aversion
(adosa) are three sobhana hetus, beautiful roots. A root, hetu, is the
foundation of akusala citta or kusala citta. Ignorance accompanies every
akusala citta. Acharn explained the roots with examples from daily life
in order to help us to know their characteristics when they appear. They
are not just classifications in the textbook and we should remember this
whenever we read about the different cittas, cetasikas and rūpas. She
said:

``We are attached to what is completely gone, but because of moha,
ignorance, it seems to last. Each word of the Buddha pertains to right
understanding of realities. Akusala hetus and sobhana hetus are
opposites. At this moment of not understanding realities there is moha.
A moment of understanding is not self but paññā cetasika. The subtlety
of the teachings is that they are all about now. Otherwise it would be
useless to listen, there would only be different words. What we take for
the world are only citta, cetasika and rūpa. It is not you who
understands but paññā cetasika. Each reality is different at each
moment. Moha takes realities as a whole, such as a flower. Without
seeing colours can there be a concept of flower?

When there is right understanding there is some detachment from
ignorance and clinging. Life is like a dream, the world of people, `I'
and things. Actually, there is no one, only realities arising and
falling away in succession\ldots{}

There is seeing; it is citta, it is accompanied by cetasikas. Learn to
understand that it is not `I', that it is citta. There can be conditions
for direct experience, paṭipatti. Without pariyatti this is impossible.
The Buddha did not tell anybody to gain it. He taught to understand what
appears now, as it is. If this is not known how can there be paṭipatti.
If paññā is not fully developed there will never be the realization of
the four noble Truths, paṭivedha. Who knows what has been accumulated
from aeons ago up until now.''

Most of the time there is forgetfulness of realities instead of the
development of understanding. We have accumulated so much ignorance and
forgetfulness.

Sarah gave us some good reminders about forgetfulness of realities. She
said: ``There is forgetfulness and it is just a dhamma, falling away
instantly. There is not my understanding or my forgetfulness. If we
think: `How can I have less forgetfulness' there is more attachment. It
can be known when it appears as just a conditioned reality in daily
life. We always follow the objects of desire and how ridiculous is this,
because they fall away instantly. Like this morning, we were clinging to
visible object, clinging to sound, but they have fallen away, just to be
forgotten. Like now, who can remember them. Usually there is so much
attachment to pleasant feeling, and we are disturbed by unpleasant
feeling. Attachment is so common, it is not a matter of trying not to
have it, trying not to be forgetful. But we should just understand what
appears at this moment.''

We all have defilements that we would rather suppress instead of knowing
them as conditioned dhammas. The sotāpanna, the person who has
eradicated wrong view and wrong practice, knows all the defilements that
arise as only a dhamma, non-self. It may be ignorance, forgetfulness,
subtle or strong lobha or conceit, māna. Their characteristics can only
be known as they arise and appear at the present moment. If we do not
realize as they are the defilements that arise, they will never be
eradicated. Paññā can only investigate attachment when it has arisen.
Paññā of the level of satipaṭṭhāna understands attachment little by
little.

\chapter[The Benefit of listening]{}
\section*{The Benefit of listening}


The understanding of anattā begins with listening, considering,
investigating the present reality. If the Buddha had not taught the
truth of realities we would not know that seeing is not self, hearing or
thinking are not self. We can never listen and consider enough. Acharn
repeated what she had said before about the three kinds of
gocara\footnote{Gocara is resort or
pasture. The Visuddhimagga, I, 49-51 mentions three kinds: resort as
support, as guarding and as anchoring. Proper resort as support,
upanissaya, is a good friend ``in whose presence one hears what has not
been heard, corrects what has been heard, gets rid of doubt, rectifies
one's view, and gains confidence: or by training under whom one grows in
faith, virtue, learning, generosity and understanding.'' As to proper
resort as guarding (arakkha), here the Visuddhimagga gives an example of
the bhikkhu who is restrained. As to proper resort as anchoring
(upanibhanda), this is the four foundations of mindfulness.}, resorts or
objects: upanissaya gocara, the object that is a strong support; arakkha
gocara, the object that is a protection; upanibandha gocara, the object
that one can depend on.

Upanissaya gocara is the object that is a strong support; there should
be considering, understanding right now. Seeing is not self. Without
visible object and eyebase there is no seeing. The explanations of
dhammas, realities, can be a strong support to hear more, consider more,
develop understanding, so that it will be a protection, arakkha gocara,
from akusala, from ignorance and attachment. Acharn said about
upanibhanda gocara, resort as anchoring:

``This is not going away from reality right now, not going astray to
such or such story. But what is there now?''

We are often absorbed in stories, conventional realities, with
attachment or worry. We think of the future, of what will happen to
``self'' in the future. Then we forget that whatever pleasant or
unpleasant experiences we have is conditioned by kamma of the past. We
think with defilements and if we had not listened to the teachings we
would not know that whatever problem we have in life is caused by our
own defilements. Acharn explained about understanding ultimate
realities:

``Seeing arises and falls away, without understanding at all. It has to
be direct understanding with direct awareness of a reality which arises
and falls away. If it is not direct how can there be the understanding
of the arising and falling away of realities? Seeing experiences just
visible object. Where is my hand? Seeing is conditioned. Hardness
appears only at the moment of touching, it is gone completely. Each
moment there is a reality that experiences an object. There are many
things in this room, how many moments of that which experiences are
there to condition that which is seen as some `thing', as images of this
or that?''

On account of what is seen we think of many things like persons or
trees, but do we realize that there must be countless moments of seeing
arising and falling away to condition thinking of concepts? Seeing
arises and falls away again and again and visible object appears again
and again. This gives us an idea of continuity. In reality, dhammas
arise and fall away extremely rapidly and only a sign or nimitta is left
of them when they have fallen away already. One unit of rūpa and one
citta is not known, only a sign or nimitta of a dhamma is known, as
Acharn reminded us time and again. One clings to that which has gone,
but only a sign is left. The reality and its nimitta can be compared to
a sound and its echo. Saññā, the cetasika remembrance, marks and
remembers visible object so that we perceive shape and form and this
leads to recognizing different people. We remember a person wrongly as
permanent. We hear many things Acharn had said before, but it is most
beneficial to hear it again, to consider it again and again. This is the
way understanding can gradually grow.

Acharn had spoken before about subtle defilements that arise and that
are unknown. It is beneficial to hear about these again since this
reminds us how little we know. It reminds us that listening and
considering the Dhamma has to continue on so that understanding can
grow. There are countless defilements arising that are unknown. The
Buddha taught different aspects and different intensities of
defilements. He taught about the subtle defilements, intoxicants or
āsavas\footnote{These are: the canker of
sensuous desire (kāmāsava), the canker of becoming (bhavāsava),the
canker of wrong view (diṭṭhāsava), the canker of ignorance (avijjāsava).}. It seems that
after seeing or hearing that arise now nothing else is appearing, but
subtle defilements that are unknown are bound to arise after seeing and
the other sense-cognitions.

Very soon after seeing has arisen and fallen away there may be a subtle
clinging that is unknown. Attachment is bound to arise on account of the
experience of all the sense objects, of visible object, sound, odour,
flavour and tangible object. Attachment to sense objects conditions
rebirth again. Attachment is a danger, but it has to be understood as a
conditioned dhamma, not to be suppressed. If it is not known when it
appears as only a conditioned dhamma, it can never be eradicated.

After we returned from Vietnam, a side trip to Samir Sakorn was
organised the next day by Khun Keowta for a group of Thais and I was
invited as well. We stayed for two days in ``Ravi Home Resort'' and Khun
Keowta paid for our whole stay as a gift of Dhamma. We appreciated her
generosity. We had a short walk to a pavillion in the middle of nature
where we had a copious luncheon. A choir was singing songs for Acharn,
anticipating her birthday the next day. The next day films were shown
with fragments of her life, there were songs with words of praise and
poems were recited in honour of her. In the morning everyone entered the
room where she stayed and paid respect to her. Actually, paying respect
to her is paying respect to the Dhamma. The best respect we can show is
listening to the Dhamma and discussing it. Someone made a touching
speech, mentioning that his father listened to Acharn's radio programs
and since his father had put on the radio, he could not help listening
too from his childhood on. It is so fortunate to be born a human so that
we have an opportunity to hear the Dhamma. We cannot be sure to have
such an opportunity the next life. He expressed with a song Acharn's
merit.

There were several sessions with Dhamma conversations. Acharn explained
that most people want to eradicate quickly attachment, lobha, the second
Truth. However, first wrong view of self should be eradicated. As long
as we take realities for self, defilements cannot be eradicated. When we
learn that whatever reality appears is only a conditioned dhamma that
cannot be controlled, we shall gradually attach less importance to them.

She also explained that we may cling with wrong view, diṭṭhi, or without
wrong view, or with conceit, māna. Clinging without diṭṭhi may arise,
for example, when we dress ourselves or when we are eating delicious
food. There need not be any wrong view at such moments. When there is
conceit we find ourselves important. Conceit may arise because of
beauty, possessions, rank or work. Or because of one's skills,
knowledge, education or wisdom. There may be the wish to advertise
oneself because of these things. We like to be honoured and praised.
When we are dissatisfied with the way other people treat us there are
bound to be moments of aversion, but there may also be moments of
conceit. We find ourselves important and we are disturbed when others do
not treat us the way we like to be treated. We tend to have prejudices
about certain people, even about our relatives, we may look down on
them. We should find out whether we have conceit when we are together
with other people. There are many moments of forgetfulness and then we
do not notice when there is conceit. A moment of conceit, of upholding
ourselves, can arise so easily.

There were a few other trips in Thailand with Dhamma conversations in
English. We went to Nakorn Nayok, to a place where we had been before.
Vietnamese friends joined us and the sessions were held in their
bungalow. They arranged everything for us with great hospitality,
setting out the chairs, serving us drinks and snacks. They helped me
with great kindness in many ways, when I was tired while walking. We had
breakfast and other meals in the restaurant near a waterfall, and Tran
Thai made delicious Vietnamese coffee for me.

We also went together with our Vietnamese friends to Kaeng Krachan, the
place where Acharn and Khun Duangduen often stay. When we arrived at
Kaeng Krachan, Sarah and Jonothan had to undergo a test of patience.
Although our bungalows were reserved ahead of time, they were not
available since the former occupants did not want to leave. Sarah and
Jonothan spoke to the office and waited for a very long time. At last
bungalows were assigned to us. I was in a bungalow next to Ann and Glen.
One morning Ann helped me walking back from the restaurant which is on a
hill, to the bungalow, in one hand holding Glen's breakfast in a covered
dish to keep his toast and eggs warm and with the other hand giving me a
support while walking.

We had at first sessions outside in the garden but since a cold wind was
blowing one of our friends, Khun Bencha, brought shawls for everyone.
Later on we continued the sessions in the bungalow of our Vietnamese
friends. In the early morning they were already sweeping the place and
cleaning up before setting out all our chairs. I came early and could
lie down on a long chair enjoying the view of nature outside. Our
friends made me feel at home.

During all our discussions we were reminded that book study, remembrance
of names and terms, is not the same as understanding the reality that
appears at the present moment. Acharn said:

``We better not go far away to other subjects, but what about now? The
reality which is seen cannot be the reality which sees it. It takes
years, a long time, to realize this. Seeing, the experience, has no
shape and form. Attend more to seeing now. Get closer to whatever is
now: hearing now, thinking now. Otherwise we are lost in the world of
words and thinking. This is the beginning of the development of the
perfections, pāramīs: really understanding that there is no one, only
different elements. We do not have to call it pāramī, but the
understanding begins to develop and this is the beginning of the
pāramīs. No matter what kind of kusala arises, when there is no
understanding it is not a perfection.

For right understanding to be keen and very firm, it needs the
perfections of truthfulness, resolution and the other perfections
because there is not enough accumulation of kusala. At the moment of
kusala citta there are no attachment, lobha, aversion, dosa, and
ignorance, moha. Each moment is conditioned and nobody can change it.''

Seeing sees visible object, but they are different realities. Seeing is
a mental reality, nāma, and visible object is a physical reality, rūpa.
Even so, hearing is nāma and sound is rūpa. Nāma and rūpa have different
characteristics, but, as Acharn said, it takes a long time to directly
understand these different characteristics. Only paññā that precisely
understands the present reality can distinguish different
characteristics of realities.

Listening to the Dhamma at this moment and all the moments of
considering it are not in vain, but they are accumulated, not lost. The
understanding of realities at this moment falls away together with the
citta it accompanies, but understanding is accumulated from one citta to
the following citta and, thus, there are conditions for the arising of
understanding again.

Towards the end of my stay in Thailand a short journey was organized for
a group of Thai friends to Chiengmai, Lampun and Chiengrai, in the North
of Thailand. Ann, Glen and I joined this group. Acharn explained that
paññā can understand sound appearing now. The future has not come yet
and what is past has gone. What has fallen away cannot be known. Acharn
reminds us time and again to investigate the present moment, since that
is the only way to penetrate the truth of nāma and rūpa. Sometimes there
are conditions for the arising of sati, sometimes not, nobody can cause
the arising of sati. We may enjoy our meal but when sati does not arise
there is ignorance of realities such as hardness or flavour which may
appear.

I was mentioning that appreciating someone else's good deed is kusala
citta, but that also akusala citta arises with attachment to that
person. Cittas arise and fall away so rapidly and therefore it is
difficult to distinguish between kusala and akusala. Acharn
said:~``Everything is dhamma is the answer. Not self.'' I said that I
often forget that understanding is not self. Khun Unnop remarked: ``The
idea of self has not been eradicated yet.'' When akusala citta arises
paññā can know it as a dhamma, not self and this is most important.

Straight after the afternoon session in Chiengmai we went to Lampun, to
visit an annex to ``Dhamma Home'' in Chiengmai. This was the first time
for Acharn to visit this place. It was a traditional Thai house that had
to be reached by following a path that went high up. Khun Porntip had
bought this house to be used for regular Dhamma meetings. We were
received with warm hospitality and many different kinds of food were
offered to us. First Acharn had a more private conversation with friends
who were very interested in the development of right understanding. Then
there was a general introduction by everyone who was present, very
informal and friendly. We returned late to Chiengmai where we had
another session the next morning, and after that we went by car to
Chiengrai, to a meditation center we had visited last year. There we
were received by the same kind lady we knew from last year. She is a
very keen listener to Acharn's explanations of the Dhamma. Acharn said
that paññā cannot be developed quickly, that it takes a long time. But
we should not be neglectful since life is very short. Khun Unnop
remarked that the problem is that it is unknown when there is
neglectfulness.

Acharn explained about seeing: ``Seeing now is seeing, it is not
eye-sight or visible object. It is the result of kamma. Pleasant and
unpleasant objects are seen, we cannot select any object. We cannot know
whether the object that is seen is pleasant or unpleasant, cittas arise
and fall away very rapidly.'' It seems that we see immediately people
and things, but that is thinking on account of what is seen, it is not
the experience of visible object.

Acharn explained that we have to look into the mirror, and then we shall
know when we are in the world of ignorance. It seems that we are really
inside the mirror, but when we touch the mirror only hardness appears.

We climbed up a hill where our hostess had a delicious luncheon prepared
so that we could enjoy the meal and the view outside. In between we
talked about Dhamma.

\chapter[The Four Noble Truths]{}
\section*{The Four Noble Truths}



After the Dhamma sessions in Saigon we went for a few days to Muine, a
seaside resort at the Chinese South Sea. Here we discussed the four
noble Truths. In his first sermon the Buddha explained the four noble
Truths. We read in the ``Kindred Sayings'' (V, Mahāvagga, Ch II, § 1,
Setting in Motion the Wheel of
Dhamma)\footnote{I used the translation
by Ven. Bodhi.}:

\begin{quote}
``Thus have I heard. On one occasion the Blessed One was dwelling at
Bārāṇasī in the Deer Park at Isipatana. There the Blessed One addressed
the Bhikkhus of the group of five thus: `Bhikkhus, these two extremes
should not be followed by one who has gone forth into homelessness. What
two? The pursuit of sensual happiness in sensual pleasures, which is
low, vulgar, the way of worldlings, ignoble, unbeneficial; and the
pursuit of self-mortification, which is painful, ignoble, unbeneficial.
Without veering towards either of these extremes, the Tathāgata has
awakened to the middle way, which gives rise to vision, which gives rise
to knowledge, which leads to peace, to direct knowledge, to
enlightenment, to Nibbāna.

And what, bhikkhus, is that middle way awakened to by the Tathāgata,
which gives rise to vision\ldots{} which leads to Nibbāna? It is this
Noble Eightfold Path; that is, right view, \textit{right
intention}\footnote{Ven. Bodhi and other
translators have right intention, but sammā-sankappa is right thinking.
Intention is the usual translation of cetanā cetasika, but this is not a
factor of the eightfold Path.}, right speech,
right action, right livelihood, right effort, right mindfulness, right
concentration\ldots{}

Now this, bhikkhus is the noble truth of
suffering\footnote{This is the
translation of dukkha.}: birth is
suffering, ageing is suffering, illness is suffering, death is
suffering, union with what is displeasing is suffering; separation from
what is pleasing is suffering; not to get what one wants is suffering;
in brief, the five aggregates subject to clinging are suffering.

Now this, bhikkhus is the noble truth of the origin of suffering: it is
this craving which leads to renewed existence, accompanied by delight
and lust, seeking delight here and there; that is, craving for sensual
pleasures, craving for existence, craving for extermination.

Now this, bhikkhus, is the noble truth of the cessation of suffering: it
is the remainderless fading away and cessation of that same craving, the
giving up and relinquishing of it, freedom from it, nonreliance on it.

Now this, bhikkhus, is the noble truth of the way leading to the
cessation of suffering: it is this Noble Eightfold Path; that is, right
view\ldots{} right concentration.

`This is the noble truth of suffering': thus, bhikkhus, in regard to
things unheard before, there arose in me vision, knowledge, wisdom, true
knowledge, and light.

`This noble truth of suffering is to be fully understood': thus,
bhikkhus, in regard to things unheard before, there arose in me vision,
knowledge, wisdom, true knowledge, and light.

`This noble truth of suffering has been fully understood': thus,
bhikkhus, in regard to things unheard before, there arose in me vision,
knowledge, wisdom, true knowledge, and light.''
\end{quote}

The same is said about the origin of suffering, its cessation and the
Path leading to its cessation.

In the preceding text we see three rounds or phases:

Sacca ñāṇa: `this is the noble truth of suffering', clear understanding
of all dhammas in daily life appearing now that are dukkha. All
conditioned realities that are impermanent, are dukkha.

This is pariyatti, but pariyatti is not theory, it pertains to the
dhamma that appears at the present moment. Sacca ñāṇa is pariyatti that
has become firm so that it can condition direct understanding.

Kicca ñāṇa: `Now this noble truth of suffering ought to be fully
understood'.

Understanding the task (kicca), the development of direct understanding
of the characteristics of all dhammas as they appear one at a time
through the senses and the mind-door. This is satipaṭṭhāna, and this is
the way dhammas will be directly known as impermanent, dukkha and
non-self.

Kata ñāṇa: `Now this noble truth of suffering has been fully
understood'.

This refers to the direct realization, paṭivedha, of the truth that is
reached when understanding of realities has been developed.

These three rounds pertain to each one of the four noble truths, and,
thus, there are twelve modes of the three rounds. Kicca ñāṇa begins when
awareness and right understanding is developed of all realities
appearing through the six doors. This is the only way eventually to
realize the cessation of dukkha.

Acharn often mentioned the levels of pariyatti, paṭipatti and paṭivedha,
and the three rounds of sacca ñāṇa, kicca ñāṇa and kata ñāṇa. People may
think that after reading the texts it is time for them to practise. They
do not realize that the right conditions are necessary for paññā to
develop stage by stage and that is likely to take many lives.

In Muine we discussed the different stages of understanding the noble
Truths. Acharn asked us whether the understanding of the first noble
Truth, of dukkha, is firm enough. Paññā should know nothing else but
what appears now. What else will be the realization of dukkha if it is
not understanding of what appears now.

In the sutta quoted above we read about all the different aspects of
dukkha, and at the end the Buddha said: ``the five aggregates subject to
clinging are suffering.'' The five khandhas are all conditioned
realities. They arise and fall away instantly, and, thus, they are
dukkha, not worth clinging to. Thus, the first noble Truth of dukkha is
not merely bodily and mental suffering, it is the unsatisfactoriness due
to the impermanence of all conditioned realities.

Acharn explained: ``There should be no trying to understand the four
noble Truths, but now is the first noble Truth. The impermanence of this
moment is so fast that it seems that nothing arises and falls away.
Detach from the idea of `I' or something permanent. Pariyatti is not
hearing or thinking but study of what appears now. There should be more
understanding of non-self. There is seeing, but no understanding of
seeing. We need hearing more and more so that there are conditions for
direct understanding in daily life only, of different elements arising
by different conditions.''

The succession of dhammas is so rapid that they seem to stay. Acharn
said that it seems to us that we see not visible object but people, that
we hear words, not sound, because of the continuous arising and falling
away of realities. The first Truth is very subtle and there can be wise
consideration arising just because of conditions. In other words, there
is no self who can make wise consideration arise.

Acharn said that we cling to nothing. What has fallen away is no more,
but we still cling to it. We shall not understand what dukkha is if we
are ignorant of the reality appearing now. We have to distinguish
between what is reality now and what is only a conventional idea or
``story''. Visible object is a reality, it arises and falls away. On
account of what is seen we think of ``something'', such as a glass or a
tree. These are ideas, not realities that can be experienced one at a
time through one of the six doorways.

Acharn explained that attachment and ignorance are the cause of dukkha.
We are attached to all objects we experience but we do not realize when
attachment arises. We fall into the trap of lobha time and again.
Whatever arises does so since it is conditioned by ignorance and
attachment. There is not yet direct awareness and direct understanding,
but it is sacca ñāṇa that really understands what causes the arising of
dukkha. The first Truth should be known and the second Truth should be
abandoned.

When intellectual understanding of the level of pariyatti is firm it is
sacca ñāṇa: clear understanding of all dhammas in daily life that are
dukkha. One does not move away from the present object.

Nibbāna is the end of attachment, and there can be firm understanding
that there is freedom from conditioned realities that are arising and
falling away and that are unsatisfactory. One can come to understand
that there is an unconditioned reality that is the end to dukkha and one
will have strong confidence that there can be the direct realization of
nibbāna.

There is a way to reach the end of the cycle of birth and death and that
is the development of the eightfold Path, the fourth Truth. When
understanding of the Path has become well established one does not
deviate from the right Path anymore. One does not search for another
practice in order to reach the goal more quickly. The more understanding
grows, the less one clings to a result.

When there is firm confidence in the four noble Truths there can be a
condition for direct awareness and understanding which is the second
round of the four noble Truths, kicca ñāṇa.

Acharn said: ``There will be detachment gradually, all the time, at the
moment of understanding. When hearing the Dhamma one finds it so
difficult, but since it is the teaching of the Enlightened One how can
it be easy? But by having confidence there begins to be direct
understanding or satipaṭṭhāna of whatever appears now, by conditions.
The more right understanding grows, the more we see the anattaness and
it becomes firmer and firmer. It is very natural.''

I asked whether the understanding of the arising and falling away of the
present reality is already a highly developed paññā.

Acharn answered: ``Of what degree, of the level of sacca ñāṇa or of
kicca ñāṇa? Sacca ñāṇa is not kicca ñāṇa.''

I asked: ``How can arising and falling away be understood on the level
of sacca ñāṇa?''

Acharn said: ``Is seeing now permanent? Is thinking now permanent?''

I answered that they are impermanent.

Acharn explained that the fact that I said that they are impermanent
showed that there is more confidence in the truth.

The second round, kicca ñāṇa, of the four noble Truths is satipaṭṭhāna,
direct awareness and understanding of whatever reality appears. When
right awareness arises it can be known that it is uncontrollable. It
arises unexpectedly, by conditions. Hearing Dhamma at this moment and
considering the truth of it is never lost, it is accumulated in the
following cittas, on and on, so that there are conditions for its
arising again. Like now, we may have a little understanding of
explanations about realities, and such moments do not stay. Other cittas
arise that experience for example a pleasant flavour we enjoy. We may be
absorbed in savoury food. But still, in all those different cittas there
is paññā accumulated. Then we listen again to the Dhamma and there is a
new opportunity for the arising of understanding. Each time we
understand a little more, a little more.

Patience is needed in the development of understanding and one should
not be discouraged if paññā does not arise often. If it would be
impossible to develop paññā of the levels of sacca ñaṇa, of kicca ñaṇa
and even of kata ñaṇa, the Buddha would not teach it.

We read in the ``Gradual Sayings, Book of the Twos, Ch II, §
9\footnote{I used the translation
of Ven. Nyanaponika, BPS Kandy 1970.} that the Buddha said:

\begin{quote}
``Abandon evil, O monks! One can abandon evil, O monks! If it were
impossible to abandon evil, I would not ask you to do so. But as it can
be done, therefore I say `Abandon evil!'

If this abandoning of evil would bring harm and suffering, I would not
ask you to abandon it. But as the abandoning of evil brings weal and
happiness, therefore I say, `Abandon evil!'

Cultivate the good, O monks! One can cultivate what is good, O monks. If
it were impossible to cultivate the good, I would not ask you to do so.
But as it can be done, therefore I say, `Cultivate the good!'

If this cultivation of the good would bring harm and suffering, I would
not ask you to cultivate it. But as the cultivation of the good brings
weal and happiness, therefore I say, `Cultivate the good!' ''
\end{quote}

People wonder about the characteristic of sati. Sati or awareness
accompanies every citta. It is non-forgetful of the object citta
experiences. When we are giving with generosity, the kusala citta is
non-forgetful of generosity. When intellectual understanding of the
present reality arises, sati is non-forgetful of that reality. When
understanding of the level of satipaṭṭhāna arises sati is non-forgetful
of the nāma or rūpa that appears. At that moment sati is mindful and
paññā understands the characteristic of the present reality.

When people read in the ``Satipaṭṭhāna Sutta'' that the bhikkhu should
be aware while he is standing, walking, sitting or lying down, they take
this as a specific practice. However, the Buddha taught that there can
be awareness of realities during all one's daily activities very
naturally, no matter in what situation. Someone asked when he intends to
drink and takes up a glass whether that is mindfulness. At that moment
one thinks of a situation and of concepts such as a glass, whereas sati
of the level of satipaṭṭhāna is mindful of one nāma or rūpa as it
appears through one of the six doorways. It is essential to know when
one is thinking of concepts or ideas and when there is mindfulness of
one reality at a time. When one touches a glass hardness may appear and
this may be known as a kind of rūpa, a physical reality, and then one
does not think of a ``thing'' that stays.

When one has the intention to be aware, there is still the idea of self
who wants to know realities. At the moment of right understanding of
realities as they appear one at a time there is no one, no world, only
the experience and that which is experienced. Paññā abandons attachment
to wrong practice.

The reality that experiences an object is quite different from the
reality that does not experience anything. Each citta must experience an
object, if there were no citta nothing could appear. We have heard this
many times, but it always seems new to us, we did not consider this
enough. When hardness appears or seeing appears the truth of realities
can be investigated so that pariyatti can become firm intellectual
understanding, sacca ñāṇa, which can condition direct understanding,
satipaṭṭhāna. I am most grateful for all the explanations and reminders
of the truth given by Acharn and friends. Those are the greatest
treasures one could possibly receive. All the discussions we had are a
way of sharing the gift of Dhamma.

Hardness appears at the moment of touching. It is not a table, it is not
my hand but it is only a reality: a khanda, an element, an āyatana
\footnote{Realities can be
classified as āyatanas. The inner ayatanas are: the eyesense and the
other senses and citta, and the outer ayatanas are: the sense objects
and dhammāyatana, including cetasikas, subtle rūpas and nibbāna. The
āyatanas show the aspect of association of realities for the experience
of objects.}, dukkha ariya sacca, a
reality that arises and falls away. All conditioned realities are
impermanent and dukkha and all dhammas, including nibbāna, are anattā.
What the Buddha taught was not his own invention, he taught the true
nature of all realities. He had by his supreme wisdom penetrated the
truth and he taught the truth to others. The Sammāsambuddha had realized
all by himself, through his enlightenment, the truth of all dhammas.

We read in the ``Gradual Sayings,'' Book of the Threes, Ch XIV, §134,
Appearance, that the Buddha said:

\begin{quote}
``Monks, whether there be an appearance or non-appearance of a
Tathāgata, this causal law of nature, this orderly fixing of dhammas
prevails, namely, all phenomena are impermanent. About this a Tathāgata
is fully enlightened, he fully understands it. So enlightened and
understanding he declares, teaches and makes it plain. He shows it, he
opens it up, explains and makes it clear: this fact that all phenomena
are impermanent.''
\end{quote}

The same is said about the truth that all conditioned dhammas are dukkha
and that all dhammas are non-self.

The Buddha respected the Dhamma he had penetrated. We read in the
``Kindred Sayings'' (I, Sagāthāvagga, Ch VI, §2, Holding in Reverence)
that the Buddha, shortly after his enlightenment, while staying at
Uruvelā, was considering to whom he could pay respect. But he could find
nobody in the world who was more accomplished than himself in morality,
concentration, insight, emancipation, or knowledge of emancipation. We
then read that he said:

\begin{quote}
``This Dhamma then, wherein I am supremely enlightened --

what if I were to live under It, paying It honour and respect.''
\end{quote}

