\section{Books written by Nina van Gorkom}

\begin{itemize}
\item
\emph{Buddhism in Daily Life} A general introduction to the main ideas
of Theravada Buddhism.The purpose of this book is to help the reader
gain insight into the Buddhist scriptures and the way in which the
teachings can be used to benefit both ourselves and others in everyday
life.
\item
\emph{Abhidhamma in Daily Life} is an exposition of absolute realities
in detail. Abhidhamma means higher doctrine and the book's purpose
is to encourage the right application of Buddhism in order to eradicate
wrong view and eventually all defilements.

\item
\emph{Cetasikas} Cetasika means 'belonging to the mind'. It is a mental
factor which accompanies consciousness (citta) and experiences an
object. There are 52 cetasikas. This book gives an outline of each
of these 52 cetasikas and shows the relationship they have with each
other.
\item
\emph{The Buddhist Teaching on Physical Phenomena} A general introduction
to physical phenomena and the way they are related to each other and
to mental phenomena. The purpose of this book is to show that the
study of both mental phenomena and physical phenomena is indispensable
for the development of the eightfold Path.

\item
\emph{The Conditionality of Life} By Nina van Gorkom
This book is an introduction to the seventh book of the Abhidhamma,
that deals with the conditionality of life. It explains the deep underlying
motives for all actions through body, speech and mind and shows that these are
dependent on conditions and cannot be controlled by a ‘self’. This book is suitable for those who have already made a study of
the Buddha’s teachings.
\end {itemize}

\section{Books translated by Nina van Gorkom}

\item
\emph{A Survey of Paramattha Dhammas} by Sujin Boriharnwanaket. A Survey of Paramattha Dhammas is a guide to the development of the Buddha's path of wisdom, covering all aspects of human life and human behaviour, good and bad. This study explains that right understanding is indispensable for mental
development, the development of calm as well as the development of
insight.
\item
\emph{The Perfections Leading to Enlightenment} by Sujin Boriharnwanaket. The Perfections is a study of the ten good qualities: generosity, morality, renunciation,
wisdom, energy, patience, truthfulness, determination, loving-kindness,
and equanimity.

