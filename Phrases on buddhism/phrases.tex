\documentclass{article}
\usepackage[mathletters]{ucs}
\usepackage[utf8x]{inputenc}
\setlength{\parindent}{0pt}
\setlength{\parskip}{6pt plus 2pt minus 1pt}


\setcounter{secnumdepth}{0}
\title{Phrases on Buddhism}
\author{Compiled by Sarah Abbott and Alan Weller}
\date{revised 31/7/10}
\begin{document}
\maketitle

\emph{Note:} Pali accents have been removed

\emph{Compiled by Sarah Abbott and Alan Weller} from discussions
with Sujin Boriharnwanaket and Phra Dhammadhara (Alan Driver) in
Sri Lanka.

\$Id: phrases.txt,v 1.1 2010/07/31 09:23:15 alan Exp alan \$

\begin{center}\rule{3in}{0.4pt}\end{center}

\begin{itemize}
\item 
  This moment cannot be lost.

\item 
  Each word of the teachings can be directly experienced.

\item 
  If one does not realise yet that one has wrong understanding, it is
  impossible to develop right understanding.

\item 
  The beginning is understanding the characteristic of awareness
  correctly.

\item 
  Some people are afraid to watch TV, but now we are lost in the
  concepts with no awareness.

\item 
  Live alone with sati, aware of visual object as visual object.

\item 
  One takes subtle attachment for calmness because of lack of
  understanding of calmness.

\item 
  This moment is so real.

\item 
  At a moment of right considering, there is no forgetfulness.

\item 
  When there is awareness, there is no thinking of far or near
  objects.

\item 
  Life is so short, so fragile. Get rid of attachment.

\item 
  Always burning with lobha, dosa, moha \ldots{}renunciation with
  satipatthana\ldots{}

\item 
  Understand accumulations from moment to moment.

\item 
  Who knows the other's cittas?

\item 
  Right action is abstaining from wrong action. There must be
  awareness of a nama or rupa to be right action of the eightfold
  path.

\item 
  The aim of thinking about concepts in the right way is to know more
  about realities.

\item 
  Life is make belief. We make it something it isn't.

\item 
  Seeing sees visible object. What is seen is not a person. We have
  attachment to individuals, but individuality has no separate
  characteristic.

\item 
  If you think you are so clever and others don't think so, you feel
  sorry. Attachment to self brings sorrow.

\item 
  Aversion prevents listening.

\item 
  The understanding that begins to know conditioned realities is also
  conditioned.

\item 
  Right understanding understands not a person or a Buddhist.

\item 
  The arising of any conditioned reality is dukkha because of its
  arising. If there is no arising, there is no dukkha. If there is no
  awareness of the reality now, how can one understand the absolute
  reality of dukkha?

\item 
  The arising happens because there is passing away of previous
  moments. Once there is no arising there can be total peace and
  calm.

\item 
  Can you tell what is beyond this moment?

\item 
  If there is no thinking of this or that concept, can there be
  concept at this moment?

\item 
  The game of life that tanha always wins.

\item 
  Don't be a victim of the conceptual system, but the conqueror of
  your ignorance.

\item 
  Awareness of sati means understanding the moment of sati as
  different from the moment without sati.

\item 
  You can have metta by accumulations, but is requires panna to see
  the value and develop it.

\item 
  What is experienced is hardness, not a table, but it has to be
  known by developed understanding.

\item 
  Always wanting the other, not the dosa, rather than understanding
  the dosa as a conditioned reality.

\item 
  When there is dosa, there is strong lobha somewhere that has
  conditioned it.

\item 
  Panna motivated us to have detachment from all akusala.

\item 
  Panna gives one a more sober, realistic view of life.

\item 
  Start with right beginning. Without right understanding, it cannot
  be right beginning.

\item 
  Propagating wrong view is the most dangerous thing to do.

\item 
  Don't force yourself to think it is the right time and right place
  for the arising of awareness, because awareness can arise anytime
  or place. Don't limit it.

\item 
  If right understanding is well established, what about awareness
  now?

\item 
  At the moment of developing right understanding, there is real rest
  no matter what one is doing.

\item 
  Without satipatthana, there is always cling to self, always wanting
  the best for self, even wanting more understanding.

\item 
  We think it's enough listening so now we need time and place for
  the development, but in reality it's never enough listening.

\item 
  We don't understand the game of tanha, so we follow it wherever it
  goes.

\item 
  The Buddha taught us to listen to dhamma, not people.

\item 
  Let go of desire and attachment for other objects that do not
  appear now. When there's awareness, there's letting go.

\item 
  The Buddha taught everyone to have kusala citta at any moment, at
  any level, because to have kusala citta at any moment is so
  helpful.

\item 
  Right understanding brings detachment. If there is even a little
  attachment, it hinders the progress of right understanding.

\item 
  Samatha doesn't get rid of concepts.

\item 
  As understanding grows, it grows beyond the level of thinking of
  sammuti sacca and knows the difference between paramattha sacca and
  sammuti sacca instead of clinging to sammuti cacca and taking for
  self.

\item 
  We have to learn when there is awareness and when there is no
  awareness, but we nee some awareness for this.

\item 
  While day dreaming, we are lost in concepts and thinking, lost in
  the sense doors, thinking of concepts of past rupas through the
  sense doors.

\item 
  We must be brave enough to study with panna any reality. We need to
  be brave to begin to study visible object as visible object.

\item 
  One doorway is never enough. Each doorway should be a check.

\item 
  It is kindness to others if we don't cling to them or encourage
  them to be attached to us.

\item 
  The test is at this moment. Test now\ldots{}Visual object now is
  the test of whether one has understanding or whether there should
  be more understanding developed.

\item 
  With understanding and awareness of calmness, calmness grows.

\item 
  The world of paramattha sacca is the world of understanding reality
  as it is.

\item 
  One is burnt by one's desire all the time. In reality one is
  attached to one's feeling, not really the person\ldots{}

\item 
  Attachment is only a conditioned moment. Attachment is like a trap
  or a bait.

\item 
  One kills oneself and one's heart by one's attachment and
  ignorance. We are trapped, lured by attachment all the time. It's
  truly poisonous.

\item 
  We are cut up with sammuti sacca when there is no awareness of
  thinking.

\item 
  There are different conditions for different namas and rupas. With
  more understanding of different conditions you will see that there
  is no self.

\item 
  It needs right understanding to know whether this moment is kusala
  or akusala.

\item 
  In a day we can see that there are more moments of akusala than
  kusala.

\item 
  It is not in the texts, but is now at the moment of right
  understanding.

\item 
  When one thinks `I am aware', it is not right awareness.

\item 
  When it is not right awareness, it cannot be accompanied by right
  understanding.

\item 
  Whenever right awareness arises, it is aware before there is time
  to think `I am aware'.

\item 
  When one says it is hearing, does one know anything about hearing?

\item 
  It is a reality sometimes very hard to be experienced.

\item 
  When there is no awareness, right understanding cannot grow.

\item 
  When there is no awareness, no understanding, no learning, there is
  no developing or seeing realities as they are.

\item 
  It is very confusing if there is no understanding of the
  development of vipassana.

\item 
  It is very natural in daily life, the teachings of the Buddha.

\item 
  One cannot get away from thinking of people, so in many suttas the
  Buddha taught many people to develop the four Brahma viharas.

\item 
  At the moment of considering someone's death, there can be the
  condition for calmness instead of trying to force calmness by
  thinking of different objects.

\item 
  Does visual object appear as just visual object now? It cannot
  appear as visual object o moha.

\item 
  By developing vipassana one can see different levels of thinking,
  because there can be thinking before thinking in words or
  concepts.

\item 
  Now if one is asked `what are you thinking', can you tell? At that
  moment it can be moha -mula -citta which thinks.

\item 
  An arahat also thinks about concepts after seeing, but maha -kiriya
  citta thinks about concept without akusala.

\item 
  If there is thinking of `samma Buddha' who can know whether it is
  lobha -mmula -citta which thinks?

\item 
  One begins to see the difference between calmness at the moment of
  kusala and no calmness at the moment of akusala.

\item 
  When right understanding grows, awareness also grows.

\item 
  Right understanding is not in the text, it is at the moment of
  understanding what seeing is.

\item 
  Whatever arises must be some type of reality, kusala or akusala.

\item 
  Nibbana does not arise.

\item 
  Citta goes all the time form moment to moment. It comes and goes
  all the time in the way of kulala or akusala. Nibbana does not come
  and go.

\item 
  The growth of vipaassana must begin with detachment and go the way
  of detachment, because attachment is very subtle and always wins
  when there is no understanding.

\item 
  If there is respect at this moment it is kusala.

\item 
  One knows oneself while the others cannot know.

\item 
  The purpose of the Satipatthana Sutta is to show that any object
  which is real can be the object of awareness. Otherwise this moment
  which is real cannot be known.

\item 
  Learn to see dhamma as dhamma.

\item 
  Almost every object is an object of attachment when there is no
  development of understanding.

\item 
  The Buddha's teaching is for practice, not just for reading or
  intellectual understanding.

\item 
  Right understanding knows everything correctly.

\item 
  Right understanding gradually eliminates attachment and ignorance
  and wrong view of self.

\item 
  At this moment of understanding reality, it is not self that
  understands.

\item 
  Never enough understanding, because each moment is conditioned.

\item 
  One has to understand what is the right object of awareness first.

\item 
  When there is the idea of self with wrong view, it conditions other
  akusala. One is attached to oneself so much that one does not
  realise that whatever one is attached to, one is attached to self.

\item 
  One thinks one is so attached to a person, but really one is
  attached to one's feeling, so one clings to one's defilements.

\item 
  No matter how much one thinks, one cannot eradicate the idea of
  self.

\item 
  Hearing this moment is not hearing a moment ago.

\item 
  3 kinds of death:

  \begin{itemize}
  \item 
    Conventional death

  \item 
    Momentary death

  \item 
    Final death for an arahat

  \end{itemize}
\item 
  Without precise understanding of paramattha and sammuti sacca there
  cannot be the eradication of self.

\item 
  Daily life is paramattha.

\item 
  By developing understanding of realities in one's life as they are,
  one sees one still has lots of akusala.

\item 
  Only right understanding can eliminate wrong understanding,
  gradually, at the moment right understanding arises.

\item 
  Intellectual understanding covers up the truth because there is no
  awareness at that moment of a characteristic as it appears.

\item 
  Intellectual understanding should be the foundation, but if one
  thinks that it is enough, there is no development and it hinders
  the development of higher understanding because one does not
  understand there are more levels of higher understanding.

\item 
  It's possible to have all the intellectual understanding but no
  understanding of the practice, like a blind man carrying a torch.

\item 
  One does not see the value of the eradication of self because one
  clings to oneself all the time. When there is less the idea of
  self, one develops more pure kusala.

\item 
  When there is less clinging to self, the weak points are detected.

\item 
  Craftiness lures to different objects\ldots{}

\item 
  When the monk abandons home life, there are more conditions for
  being virtuous at the degree of being able to leave home.

\item 
  Dukkha in the absolute sense is the arising and falling away of
  each moment because it cannot stay.

\item 
  Whatever is real can be proved.

\item 
  In the beginning sati is slow and awkward.

\item 
  Seeing the lack of any alternative is a way of seeing the value of
  kusala.

\item 
  If the blind man thinks he can see, he's really in trouble.

\item 
  Any intellectual understanding cannot be clear.

\item 
  Do we hope for result for me?

\item 
  One begins with detachment from the very beginning.

\item 
  Better to be a nobody than a somebody. Better to be a good friend
  rather than a teacher.

\item 
  One is attached to oneself, to one's feeling when one cares what
  the other thinks.

\item 
  What is right is right.

\item 
  Akusala is so ugly.

\item 
  We all want to be the object of attachment. We think of self and
  world collapses with the idea of being a nobody. There is no seeing
  the value of no attachment, the real freedom when there is no
  enslavement.

\item 
  Lack of confidence is what sati can and cannot do is not helpful.

\item 
  Wrong view guides one's tanha tanha for one's wrong ideas.

\item 
  Getting to know oneself better is the only way to really help
  others. If one develops more metta, karuna, more understanding and
  a more sincere inclination to other people, one will see that what
  has been most helpful to oneself will be what is most helpful to
  others also. One understands oneself better.

\item 
  Sound is obviously what it is when one's understanding grows. One
  does not have to call it by any name.

\item 
  It is because sound appears that we can think about it in different
  ways.

\item 
  Cannot whatever happens in daily life be a subject for teaching
  dhamma?

\item 
  Whatever we receive in this life has its cause in some good deed in
  this life or in a previous life.

\item 
  Finally if comes down to a cause here which is what makes it
  possible to understand life and do something about it.

\item 
  If metta is strong enough one will be concerned to help.

\item 
  If sati does not arise understanding cannot know which are the
  moments of sati and which are the moments without sati.

\item 
  The purpose should be right understanding.

\item 
  Be an island\ldots{} depend on oneself, one's own understanding
  which can eradicate one's defilements.

\item 
  Attached to rubbish\ldots{}

\item 
  Want to have conditions for the arising of satipatthana, waiting
  for the arising of satipatthana this is not the understanding of
  the development of sati. The moment of thinking it is not the
  moment of direct awareness.

\item 
  Intellectual understanding is useful, but don't think that one has
  to think and think and think so that satipatthana will arise.

\item 
  To know the difference between thinking and sati, there has to be
  sati.

\item 
  Panna which performs the function of detachment is the highest
  meaning of upekkha.

\item 
  Sati is not forgetful to be kusala, not forgetful to think about
  the object in the right way. If you forget, can there be studying?

\item 
  Metta is seeing the loveableness of all beings.

\item 
  The intention to do harm brings harm the place where harm
  originated.

\item 
  Whenever there are results, we know that those resulted in their
  cause.

\item 
  We have the idea that we can run away from vipaka.

\item 
  We never know when vipaka will come.

\item 
  We have an idea of cause and result but it's wrong.

\item 
  There must be right understanding, precisely, of this moment.

\item 
  One can think of one's kusala with lobha.

\item 
  When there are conditions for sati, sati will arise.

\item 
  Attachment likes calmness so much that it clings immediately.

\item 
  If one is not courageous enough, one clings to calmness for sure.

\item 
  At this moment of thinking one begins to see whether one thinks
  with kusala or akusala.

\item 
  Always cling to kusala, clinging to self and upset about akusala.

\item 
  Life is a dream. When one knows the citta that dreams one is
  awake.

\item 
  One wakes up for one short moment of sati and then the dream takes
  over.

\item 
  The sound that is heard now does not hear anything.

\item 
  One cannot afford to be disinterested in reality.

\item 
  At the moment of seeing visible object as visible object, there is
  no attractiveness in visible object.

\item 
  The way to know the present moment is to begin to know this moment
  now.

\item 
  Learn to give without strings attached.

\item 
  We have bad opinions of others, but do we like others to have bad
  opinions of us?

\item 
  With developed right understanding, one knows everything one knew
  before, but one knows something one didn't know before.

\item 
  One will understand more about hearing and thinking in addition to
  hearing and thinking.

\item 
  It's very easy to have misunderstanding about what one thinks to be
  oneself.

\item 
  When one gossips, one doesn't think about the other or the person
  one is gossiping to. Would one like to think he is gossiping about
  us?

\item 
  The words may be all right, but still there is no right
  understanding.

\item 
  The citta which solves the problem arises and falls away. The
  solution to life is not to be born.

\item 
  We cannot develop panna if we don't see the value of dana and
  sila.

\item 
  If one treats all as though they are one's child with metta, this
  way one cannot have any attachment.

\item 
  Even though lay people lead busy lives, they can get to know their
  busy lives better.

\item 
  The taking away of anything from anyone is not wholesome.

\item 
  Even if you look through a microscope you still only see visible
  object.

\item 
  If one always wants to be right, how will one deal with situations
  where one is wrong.

\item 
  One's understanding has to learn to know the obvious, not to
  overlook what is staring one in the face.

\item 
  We overlook the obvious all the time, especially nama, the
  experience which makes it possible to experience colour right now.

\item 
  Colour arises but doesn't appear if there is no seeing.

\item 
  Does one have any understanding of the present reality? If so there
  must be awareness.

\item 
  Lobha creeps in all the time.

\item 
  It's always good to teach people to understand what brings
  happiness.

\item 
  Dosa can't help.

\item 
  People are never impressed when the right thing is said in the
  wrong way.

\item 
  Waiting doesn't bring kusala of any kind.

\item 
  On and on and on we have this idea of self and we don't know it.

\item 
  The more one studies satipatthana, the more honest one becomes.

\item 
  When sati is developed one becomes more resigned to the truth.

\item 
  Know oneself first and one will know the other man better.

\item 
  Satipatthana is indispensable.

\item 
  When one studies more, one sees one's abysmal ignorance.

\item 
  It takes time and patience to develop sati.

\item 
  We have to know what is not known to know what ignorance is.

\item 
  One can't help the other to understand without words, but the words
  are not the understanding. The realities of our life are not
  words.

\item 
  Begin again to know this moment which has not been know yet.

\item 
  The arahat has given up the struggle by understanding realities.

\item 
  We always take refuge in that which is not safe or secure. The
  ultimate refuge is nibbana.

\item 
  Having forgotten, he remembers, he begins again.

\item 
  We continue to be beginners that having forgotten, we will
  remember.

\end{itemize}
\begin{center}\rule{3in}{0.4pt}\end{center}

\emph{Compiled by Alan Weller} from discussions with Phra
Dhammadhara (Alan Driver).

\begin{itemize}
\item 
  More ignorance, more wrong understanding, more attachment and this
  will mean more unhappiness, so matter how difficult we find it what
  choice do we have. We either go forward or we go with the rest of
  the world backwards and down. Even if we go forwards only a little
  bit it's much better than going backwards. If we understand how
  important it is to go forward, if we see the value of progress and
  the dangers of falling backwards then perhaps there will be more
  conditions for us to study, to listen, to develop kusula and to see
  progress taking place.

\item 
  From the beginning it must be Right Understanding. But the only
  moment that we can progress, the one and only moment in our whole
  life \ldots{}is this moment.

\item 
  If our understanding is very, very weak, then the little bit of
  understanding we may gain now will not be enough for us to see the
  truth. But without accumulating one little bit now, the moment will
  never come when wisdom is strong enough to know the truth.

\item 
  We know from our own lives that we don't always give help when help
  is needed.

\item 
  Loving kindness has to be a reality for us in our actions and
  speech from day to day as we live.

\item 
  Loving kindness isn't just a word. It's a reality which has to be
  known through practice in our daily life.

\item 
  Everybody loses when we don't have metta.

\item 
  We have to be kind instead of wishing everyone else was kind.

\item 
  Thinking that everyone else should be kind or wondering why they
  are not kind.

\item 
  The only way we will ever be able to prove the Buddha's teachings
  is by beginning to practice them and see what happens. There's no
  other way.

\item 
  There is never a moment goes by when awareness cannot arise.

\item 
  It is loving kindness that helps us present Dhamma is a gentle way
  without being pushy\ldots{} And that's hard.

\item 
  We don't have kindness when we expect too much of someone. They
  should understand, why? Because we say so, because we teach so
  well, because it's the truth. People should be kind. Why? Because
  we think so, because it would be nice. It depends on conditions.

\item 
  If we don't develop Sati, we can't keep the precepts. So what could
  be kinder?

\item 
  It's kindness when we develop Satipatthana, we are being kind to
  ourselves and kind to everybody else. We're doing the whole world a
  favour. We're removing a little bit of ignorance from the world
  that makes us behave and act the way we do.

\item 
  The way to develop more understanding is exactly the same for every
  living being that was ever born. To be aware of the reality which
  appears now in your life whatever that reality may be.

\item 
  There can be no short cut.

\item 
  The practice of Satipatthana is the most subtle thing.

\item 
  All moments are moments for awareness.

\item 
  What is kusula without Right Understanding, it's just self, self,
  self.

\item 
  A moment of Satipattana that is aware of akusula is so much more
  valuable than kusula without Sati.

\item 
  Who can stop realities from arising?

\item 
  At moments of desire for awareness, at moments of trying to be
  aware of a particular object, to try and force, to try and be aware
  here, here, here, it's all wrong, its not natural.

\item 
  We cannot call back any reality.

\item 
  Results for whom.

\item 
  It's good to know the truth that all realities of our life our
  Dukkha. They arise and fall away.

\item 
  If we think that the way to be happy is to get what you want and to
  have nice things to like at, to hear, taste and smell. We'll make
  lobha our God and devote our lives to serving lobha\ldots{} and
  that's the path to pain.

\item 
  There is really nothing better in life to do, than to find out what
  life is all about.

\item 
  We find out in practice what we are told in theory.

\item 
  Who can say what will happen next.

\item 
  It is conditioned, it is unavoidable, uncontrollable.

\item 
  Dosa is Dosa, it's not Dosa with a body it's not Dosa that can do
  anything.

\item 
  How can you blame Dosa?

\item 
  We can't not have akusala.

\item 
  We have the illusion of control all our life except at the moment
  of sati. It helps to eradicate that illusion.

\item 
  The condition for the development of Sati is Right Understanding of
  it.

\item 
  We can't really develop right understanding, if we are not ashamed
  of our ignorance of realities, if we are not afraid of not knowing
  the truth about realities.

\item 
  True respect for Dhamma is not just listening and understanding
  what you hear but putting into practice what you hear.

\item 
  Only a coward performs akusula because he's afraid of
  inconvenience, afraid of trouble, afraid of poverty therefore he's
  capable of and shamelessly performs akusala.

\end{itemize}
\begin{center}\rule{3in}{0.4pt}\end{center}

\emph{Compiled by Alan Weller} from discussions with Sujin
Boriharnwanaket in England.

\begin{itemize}
\item 
  One cannot live without pleasure and one cannot live without
  dhammma. Wise people cannot live just for pleasure. The wise one
  will live with pleasure and with understanding.

\item 
  The development of awareness is the highest degree of kusala.

\item 
  As the moment of aversion find out whether it's just a name or a
  reality which is the object of aversion.

\item 
  The outside is very clean the body is washed, but how many times
  are citta and cetasika washed with understanding.

\item 
  The fire on one's head is this moment.

\item 
  No matter we are happy or sad. Dhamma should be the important thing
  in life.

\item 
  Life cannot be smooth without Right Understanding.

\item 
  If people think that Dhamma will destroy their happiness, they
  don't understand Dhamma.

\item 
  We cannot rate the prize of the `Best Friend'.

\item 
  Lobha encourages wrong view and wrong view encourages lobha. They
  are friends.

\item 
  When one enjoys something very much nobody can help, because there
  are conditions for that degree of enjoyment, but Right
  Understanding can understand and awareness can be aware. And that
  moment of awareness is the eightfold path. So one should understand
  all conditioned realities that happen one's life.

\item 
  Seeing is conditioned, pleasant feeling is conditioned. It has its
  own conditions already. We shouldn't prepare any other conditions
  for any reality to arise.

\item 
  Understanding can follow all six doorways until there is no doubt
  about conditioned realities.

\item 
  Nobody can condition any reality.

\item 
  The present moment is the most important moment.

\item 
  The disadvantage of seeing is that attachment follows it. The
  advantage of seeing is that understanding follows it.

\item 
  As long as there is expectation Vipassana nana cannot arise.

\item 
  One has to be so very patient to understand the teachings.

\item 
  There is always danger when there is experiencing of an object with
  ignorance. It's like stepping on a thorn. If there is no
  development of sati, we enjoy stepping on thorn.

\item 
  The teachings are the Dhamma mirror, they help to see inside
  clearly.

\item 
  Look carefully and step on the place where there is no thorn.

\item 
  It's not enough just to learn from the book. Right Understanding
  and awareness should be able to understand this moment as no self,
  no being at all.

\item 
  Discussion is the most important factor for the development of
  Right Understanding.

\item 
  The more one wants to have panna, the more one sees that it's only
  wishing or wanting, it's not the right cause for the arising of
  panna. The right cause for the arising of panna must be learning,
  studying, considering. There can be awareness at anytime by
  sankhara khanda conditioning it, not by one's wish.

\end{itemize}
\end{document}
