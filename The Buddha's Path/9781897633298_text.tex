\documentclass{book} 
\linespread{1.05}
\usepackage[centering,paperwidth=6in,paperheight=9in]{geometry}
\usepackage[mathletters]{ucs}
\usepackage[utf8x]{inputenc}
\raggedbottom
\widowpenalty=1000
\clubpenalty=1000

\title{The Buddha's Path}
\author{Nina van Gorkom}
\date{2014}
\begin{document}
\maketitle
\vspace*{10pt}
\noindent First edition published in 2008 \\
Second edition published in 2014 by\\
Zolag\\
32 Woodnook Road\\
Streatham\\
London\\
SW16 6TZ\\
www.zolag.co.uk\\
\vspace{10pt}

\noindent ISBN 978-1897633-29-8\\ 
Copyright Nina van Gorkom\\

\noindent This work is licensed under the: \\
Creative Commons Attribution-NoDerivs 3.0 Unported License.\\
To view a copy of this license, visit:\\
http://creativecommons.org/licenses/by-nd/3.0/ \\
or send a letter to: \\
Creative Commons, \\
171 Second Street, \\
Suite 300, \\
San Francisco, \\
California, \\
94105, USA.\\


\noindent British Library Cataloguing in Publication Data\\
A CIP record for this book is available from the British Library\\
Printed in the UK and USA by
Lightningsource.
\frontmatter
\tableofcontents

\chapter{Preface}

What is Buddhism? It is different from what most people believe: an
Oriental religion full of rituals and ceremonies, which teaches
meditation leading to mystical experiences. Buddhism is most practical
and matter of fact. The Buddha taught all that is real, all mental
phenomena and physical phenomena of our life. By the study of his
teach­ings one learns to investigate one's different mental states which
change very rapidly. One comes to know one's faults and vices, even the
more subtle ones which are not easily noticeable. One learns what is
good and wholesome and how to develop wholesome deeds, speech and
thoughts. The Buddha taught on life and death, on the conditions for all
phenomena which arise and which are impermanent. He pointed out the
suffering and dis­satisfaction inherent in the phenomena of life. He
explained the true nature of man: elements which arise
and then fall away immediately and which are devoid of an abiding
substance, of a ``self''. The Buddha taught the eightfold Path which, if
it is developed in the right way, leads to direct understanding of the
true nature of all the pheno­mena of life. It is by direct
under­standing that defilements can eventually be eradicated.

In this book I try to explain the message, the basic contents and some
details of the Buddha's teachings. What is the use of learning details?
The Buddha's teach­ings are subtle and deep and therefore it is
necessary to go into details. If one does not know that there are many
different aspects to each reality the Buddha taught one will read the
scriptures with wrong understanding. There will be an
over-simplification in the interpretation of the texts. Patience is
needed to grasp the complexity of the teachings in order to avoid a
superficial understanding of them. Wrong interpretation of the texts
leads to wrong practice of the Buddha's Path, and as a consequence there
will not be right understanding of the phenomena within ourselves and
around ourselves. The development of the eightfold Path is the
development of direct understanding of the true nature of realities.
When the way of its develop­ment is correctly understood, the truth of
what the Buddha taught can be verified through one's own experience.
Although theoretical understanding is the founda­tion for the
develop­ment of the Path, it is not sufficient to grasp the deep meaning
of the teachings. One should know that it takes time and patience to
understand what this Path is and how one can begin to develop it.

What is the origin of the Buddhist texts of the Theravāda tradition as
they have come to us today? These texts date from the Buddha's time,
about 2500 years ago. Shortly after the Buddha's passing away a Council
was held in Rājagaha, were the teachings were examined and scruti­nized
as to their orthodoxy. Under the leadership of the Buddha 's eminent
disciple Mahā Kassapa five hundred monks who had reached the state of
perfection recited all the texts of the Vinaya, the Book of Discipline
for the monks, the Suttanta, Discourses, and the Abhidhamma, the higher
teaching on ultimate real­ities. A second Council was held one century
later at Vesāli. This was necessary because of wrong interpreta­tions of
the monks's rules by heretical monks. A third Council was held in 268
B.C. in Pātalīputta. On this occasion the canon of the Theravāda
tradition in the Pāli language as it exists today was finally redacted.
During all this time the teachings were handed down by oral tradition.
About 89 B.C. they were commit­ted to writing in Sri Lanka.

In this book I have used a few Pāli terms which can be of use to those
who intend to deepen their knowledge of Buddhism. The English
equivalents of the Pāli terms are frequently unsatisfactory since they
stem from Western philosophy and therefore give an association of
meaning which is different from the meaning intended by the Buddhist
teachings.

I want to acknowledge my deep gratitude to Ms.~Sujin Boriharnwanaket in
Thailand, who gave me great assist­ance in the understanding of the
Buddhist teachings and in particular in their application. I also wish
to express my gratitude to the ``Dhamma Study and Propagation
Founda­tion'', to the publisher Alan Weller and to my husband. Without
their help the writing and the printing of this book would not have been
possible.

Finally I want to give information on the sources of my quotations from
the texts in the English language. I quoted mainly from the Dialogues of
the Buddha, the Middle Length Sayings, the Kindred Sayings and the
Gradual Sayings. I also quoted from the Path of Purification which is an
Encyclopedia on Buddhism written by the com­mentator Buddhaghosa in the
fifth century A.D. This is only a selection of the texts. I used the                                 translations as given by the Pali Text Society.  
With this book I intend to give an introduction to the Buddhist
teachings. I hope that I can encourage readers to explore the scriptures
themselves in order to deepen their own understanding.

Nina van Gorkom
\mainmatter

\chapter{Introduction}

Why are we in this life? Why do we have to suffer? Men of all times
conceived philosophical systems which could explain the reason for their
existence and give a solution to the problem of suffering. Religions
also try to give an answer to the problem of suffering in teaching that
people should have faith in God and live according to His
com­mand­ments; consequently one can, after death, enjoy eternal bliss
in heaven. The Buddha gave his own, unique answer to the problem of
suffering. He taught that the cause of suffering is within man, namely
his own faults and defilements, and not in the external situation. He
explained that only profound knowledge of his own mind and of all
phenomena of his life can lead to the end of suffering. We read in the
Buddhist scriptures (Kindred Sayings I, Chapter III, Kosala, Part 3, §3,
The World) that King Pasenadi had a conversation with the Buddha at
Sāvatthī about the cause of suffering. We read:

\begin{quote}
``How many kinds of things, lord, that happen in the world, make for
trouble, for suffering, for distress?''

``Three things, sire, happen of that nature. What are the three? Greed,
hate, and delusion-these three make for trouble, for suffering, for
distress''
\end{quote}

The outward circumstances cannot be changed, but the inward attitude
towards the vicissitudes of life can be changed. Wisdom can be developed
and this can eventually eradicate completely greed, hate and delusion.
This wisdom is not developed by speculation about the truth of life, it
is developed through the direct experience of the phenomena of life as
they really are, including one's own mental states. That is the Path the
Buddha taught, but it takes time to understand how it is to be
developed.

The Buddha was not a God, not a saviour, who wanted people to follow him
without questioning the truth of his teaching. He showed the Path to the
understanding of the truth, but people had to investigate the truth and
develop the Path themselves. We read in the scrip­tures (Dialogues of
the Buddha, II, 16, the Book of the Great Decease) that the Buddha said
to his disciple Ānanda:

\begin{quote}
Therefore, Ānanda, be an island to yourselves, a refuge to yourselves,
seeking no external refuge; with the Teaching as your island, the
Teaching as your refuge, seeking no other refuge"
\end{quote}

The Buddha explained that in developing the Path one is one's own
refuge.

The Buddha had found the Path to understanding of the truth all by
himself, without help from a teacher. However, he was not the only
Buddha. Aeons and aeons ago there were other Buddhas who also found the
Path all by themselves and who taught the development of the Path to
others. The Buddha whose teaching we know in this time was called the
Buddha Gotama. His personal name was Siddhattha and his family name
Gotama. He lived in the sixth century B.C. in Northern India. He was
born in Lumbini (now in Nepal) as the son of Suddhodana, King of the
Sākyas. His mother was Queen Māyā. He married Princess Yasodharā and he
lived in great luxury. However, when he drove out to the park with his
charioteer he was confronted with suffering. We read in the Dialogues of
the Buddha (II, 14, The Sublime Story) that the Buddha related the story
of a former Buddha, the Buddha Vipassi, and explained that all
Bodhisattas, beings destined to become Enlightened Ones, Buddhas, have
such experiences. We read that the Bodhisatta, after he saw in the park
someone who was aged, asked the charioteer the meaning of what he saw.
The charioteer explained to him that the person he saw was aged and that
all beings are subject to old age. On a following occasion there was an
encounter with a sick person and the charioteer explained that all
beings are subject to illness. At another occasion the Bodhisatta saw a
corpse. The charioteer explained that that was the corpse of someone who
had ended his days. We read:

\begin{quote}
And Vipassi saw the corpse of him who had ended his days and asked
``What, good charioteer, is ending one's days?''

``It means, my lord, that neither mother, nor father, nor other kinsfolk
will see him any more, nor will he ever again see them.''

``But am I too then subject to death, have I not got beyond the reach of
death? Will neither the King, nor the Queen, nor any other of my
relatives see me any more, or I ever again see them?''

``You, my lord, and we too, we all are subject to death, we have not
passed beyond the reach of death. Neither the King, nor the Queen, nor
any other of your relatives would see you any more, nor would you ever
again see them.''

``Why then, good charioteer, enough of the park for today! Drive me back
from here to my rooms.''

``Yes, my lord,'' replied the charioteer, and drove him back.

And he, monks, going to his rooms, sat brooding sorrowful and depressed,
thinking ``Shame then verily be upon this thing called birth, since to
one born the decay of life, since disease, since death shows itself like
that!''
\end{quote}

After the Bodhisatta had been confronted with an old man, a sick man and
a corpse, his fourth encounter was with a monk. The Bodhisatta asked the
meaning of being a monk and the charioteer answered that it was being
thorough in the religious life, in the peaceful life, in good actions,
in meritorious conduct, in harmlessness, and in kindness to all
creatures. The Bodhisatta decided to leave his worldly life and to
become a monk.

The Buddha Gotama, when he was still a Bodhisatta, had the same
encounters as the Bodhisatta Vipassi. He also became a monk after his
fourth encounter in order to seek the solution to the problem of
suffering. He first practised severe austerity, but he saw that that was
not the way to find the truth. He decided to discontinue such severe
practices and to stop fasting. On the day he was to attain enlightenment
he took rice gruel which was offered to him by the girl Sujātā. Seated
under the Bodhi-tree he attained enlightenment. He realized the four
noble Truths: the truth of suffering, of the cause of suffering, of the
ceasing of suffering and of the Path leading to the ceasing of
suffering. He had attained enlightenment at the age of thirty-five years
and he taught the Path to others for forty-five years. At the age of
eighty he passed away at Kusinārā.

His teachings have been preserved in the Buddhist scriptures of the
Vinaya (Book of Discipline for the monks), the Suttas (the Discourses),
and the Abhidhamma (the ``Higher Teachings''). These scriptures which
have been written in the Pāli language are of the Theravāda tradition.
The term ``Theravāda'' (Hīnayāna or ``Small vehicle'' is no longer used)
could be translated as ``the School of the Elders''. There is also the
Mahāyāna tradition which devel­oped later on. The two traditions are in
agreement with several points of the Buddha's teachings, but they are
different as regards the practice, the development of the Buddha's Path
leading to the realization of the truth. The Theravāda tradition is
followed in Thailand, Sri Lanka, Laos, Cambodia and Bangladesh. The
Mahāyāna tradition is followed in China, Japan, Tibet and Mongolia.

The Buddha, at his enlightenment, understood that the cause of suffering
is craving. He saw that when there is the cessation of craving there
will be an end to suffering. What the Buddha teaches is contrary to what
people generally are seeking in life. Every being has craving for the
experience of pleasant things and therefore wishes to continue to obtain
such objects. The Buddha was, after his enlightenment, for a moment not
inclined to teach the truth he had realized under the Bodhi-tree. He
knew that the ``Dhamma'', his teaching of the truth, would be difficult
to understand by those who delighted in sense pleasures. We read in the
Middle Length Sayings (I, 26, The Ariyan Quest), that the Buddha
related to the monks his quest for the truth when he was still a
Bodhisatta, his enlightenment and his disinclination to teaching. We
read that the Buddha said:

\begin{verbatim}
This that through many toils I've won
Enough! Why should I make it known?
By folk with lust and hate consumed
This Dhamma is not understood.
Leading on against the stream
Deep, subtle, difficult to see, delicate,
Unseen it will be by passion's slaves
Cloaked in the murk of ignorance.
\end{verbatim}

We then read that the Brahmā Sahampati, a heavenly being, implored the
Buddha to teach the truth. The Buddha surveyed the world with the eye of
an Awakened One, and he saw beings with different dispositions, some of
whom were not capable to accept his teaching, and some who were capable
to be taught. We read that the Buddha used a simile of different kinds
of lotuses in a pond:

Even as in a pond of blue lotuses or in a pond of red lotuses or in a
pond of white lotuses, a few red and blue and white lotuses are born in
the water, grow in the water, do not rise above the water but thrive
while altogether immersed; a few blue or red or white lotuses are born
in the water, grow in the water and reach the surface of the water; a
few blue or red or white lotuses are born in the water, grow in the
water, and stand rising out of the water, undefiled by the water; even
so did I, monks, surveying the world with the eye of an Awakened One,
see beings with little dust in their eyes, with much dust in their eyes,
with acute faculties, with dull faculties, of good dispositions, of bad
dispositions, docile, indocile, few seeing fear in sins and the world
beyond.

Out of compassion the Buddha decided to teach Dhamma. His teaching goes
``against the stream'', it is deep and it can only be understood by
studying it thoroughly and by carefully considering it. Generally,
people expect some­thing else from the Buddhist teachings. They believe
that the Buddha taught a method of meditation to reach tranquillity, or
even extraordinary experiences such as a mystical trance. It is
understandable that one looks for a way of escape from a life full of
tension and troubles. Extraordinary experiences, however, cannot give
the real solution to one's problems. It is a wrong conception of
Buddhism to think that the goal of the Buddha's Path are mystical
experiences to be reached by concentration. The Buddha's Path has
nothing to do with unworldly mysticism, it is very concrete and matter
of fact. Understanding should be developed of all that is real, also of
our faults and vices as they naturally appear during our daily
activities. We have to know ourselves when we laugh, when we cry, when
we are greedy or angry, we have to know all our different moods. All
troubles in life are caused by our defilements. It is through the
development of understanding that defilements can be completely
erad­icated. Comprehending, knowing and seeing are stressed time and
again in the Buddhist teachings.

It is felt by some people that, in order to develop understanding of
one's mind, one should retire from daily life and sit still in quiet
surroundings. It may seem that, when one is in isolation, there is no
anger or aversion and that it is easier to analyse one's mental states.
However, at such moments there is bound to be clinging to quietness and
when there is clinging there is no development of understanding. We read
in the scriptures about people who could develop calm in concentrating
on a meditation subject. They were very skilled, they knew the right
method to attain calm, which is a wholesome mental state. However,
through the development of calm defilements are not eradicated, they are
merely temporarily suppressed. The Buddha taught the way to develop the
understanding leading to the complete and final eradication of all that
is impure, of all defilements. In order to reach the goal there is no
other way but developing understanding naturally in one's daily life.

It cannot be expected that there will be the eradication of defilements
soon since they are so deeply rooted. The Buddha had, during countless
lives when he was still a Bodhisatta, developed understanding of all
phenomena of life. Only in his last life, at the moment he attained
enlightenment, all defilements were eradicated. How could we expect to
reach the final goal within a short time?

The Buddha taught the way to the eradication of all defilements.
Defilements are not eradicated by rituals or by sacraments. The way to
eradicate them is an inner way, namely the understanding of all mental
and physical phenomena of one's life. The Buddha taught very precisely
what defilements are. They are not exactly the same as what is generally
meant by ``sin''. By sin is usually meant an evil deed, evil speech or
evil thought which has a high degree of impurity. According to the
Buddhist teachings defilements include all degrees, even slight degrees,
of what is impure. Even slight degrees of defilements are unhelpful, not
beneficial. The term ``unwholesomeness'', that which is unhelpful, not
beneficial, includes all degrees of defilements\footnote{``Unwholesome''
  and ``wholesome'' are terms which usually stand for the Pāli terms
  ``akusala'' and ``kusala''.}. If one thinks in terms of sin one will
not understand that ignorance of the phenomena of life is unwholesome,
that ignorance is harmful since it blinds one to see the truth. Or one
will not understand that even a slight degree of attachment is
unwholesome, even harm­ful, because it is accumulated and it will arise
again and again.

The Buddha, when he was sitting under the Bodhi-tree, realized the four
noble Truths: the Truth of suffering, the Truth of the origin of
suffering, the Truth of the ceasing of suffering, and the Truth of the
Path leading to the ceasing of suffering. As to the Truth of suffering,
this is not merely suffering caused by bodily and mental pain. The Truth
of suffering pertains to all phenomena of life which are imper­manent.
They arise and then fall away immediately, and thus they cannot be our
refuge. Suffering in this sense is the unsatisfactoriness inherent in
all phenomena of life. Only when the arising and falling away of
physical phenomena and mental phenomena can be directly experienced, can
one begin to grasp the Truth of suffering.

The Truth of the origin of suffering is craving. Craving in this sense
is not only strong attachment or greed, it includes many shades and
degrees of attachment. There is craving for pleasant colours, sounds,
odours, flavours and tangible objects, for all that can be experienced
through the senses. There is craving for the continuation of life. It is
craving which conditions rebirth in new existences, again and again.
Craving pushes beings on in the cycle of life, the continuation of
rebirth and death. There is not only this present life, there were also
past lives and there will be future lives. I will deal with this subject
later on. So long as there are ignorance and clinging there are
conditions for being in the cycle of birth and death. Through wisdom,
understanding, there can be liberation from it. When there are no more
conditions for rebirth, there is the end of old age, sickness and death,
the end of all suffering.

The third noble Truth, the cessation of suffering, is nibbāna. The
Buddha experienced at his enlightenment nibbāna. It is difficult to
understand what nibbāna is. Nibbāna (more popularly known in its
Sanskrit form of nirvāṇa) is not a place such as heaven or a paradise
where one enjoys eternal bliss. There are heavenly planes, according to
the Buddhist teachings, where one can be reborn as a result of a good
deed, but existence in such planes is not forever. After one's lifespan
in such a plane is ended there will be rebirth in other planes, and,                                 thus, there is no end to suffering. Nibbāna is only an object of specu­                             lation so long as it has not been realized. It can be realized when there                                  is full understanding of all phenomena of life which arise because of                                  their own conditions and then fall away. The conditioned phenomena of                                    life are, because of their impermanence, unsatisfactory or suffer­ing.
Nibbāna is the unconditioned reality, it does not arise and fall away
and therefore it is not suffering, it is the end of suffering. Nibbāna
is real, it is a reality which can be experienced, but we cannot grasp
what an uncon­di­tioned reality is when we have not realized the truth
of condi­tioned realities. Nibbāna is not a God, it is not a person or a
self. Since negative terms are used to express what nibbāna is, such as
the end of rebirth, it may be felt that Buddhism propagates a negative
attitude towards life. However, this is not the case. It has to be
understood that rebirth is suffering and that nibbāna is the end of
suf­fering. Nibbāna is freedom from all defilements, and since
defilements are the cause of all unhappiness nibbāna should be called
the highest goal. We read in the Kindred Sayings (IV, Kindred Sayings on
Sense, Part IV, Chapter 38, §1, Nibbāna) that the wanderer
Rose-apple-eater came to see the Buddha's disciple Sāriputta and asked
him what nibbāna was. Sāriputta answered:

\begin{quote}
The destruction of lust, the destruction of hatred, the destruction of
illusion, friend, is called nibbāna.
\end{quote}

``Extinction'' and ``freedom from desire'' are meanings of the word
nibbāna. Nibbāna means the end of clinging to existence and thus it is
deliverance from all future birth, old age, sickness and death, from all
suffering which is inherent in the conditioned realities of life. The
Buddha experienced at his enlightenment the unconditioned reality which
is nibbāna. His passing away was the absolute extinguishment of
conditions for the continuation of the life process. When the Buddha was
still alive people asked him what would happen to him after his passing
away. He explained that this belongs to the questions which cannot be
answered, questions which are merely speculative and do not lead to the
goal. The Buddha's passing away cannot be called the annihilation of
life, and there cannot be rebirth for him in another plane, either. If
there would be rebirth he would not have reached the end of all
suffering.

The fourth noble Truth, the way leading to the ceasing of suffering, is
the development of the eightfold Path as taught by the Buddha. I will
deal with the eightfold Path more extensively later on in this book. The
eightfold Path is the development of understanding of all physical
phenomena and mental phenomena which occur in daily life. Very gradually
these phenomena can be realized as impermanent, suffering and ``not
self''. The Buddha taught that there is in the absolute sense no abiding
person or self. What is generally understood as a person is merely a
temporary combination of mental phenomena and physical phenomena which
arise and fall away. The Buddha's teach­ing of the truth of ``non self''
is deep and difficult to grasp. This teaching is unique and cannot be
found in other philosophical systems or religions. I will deal with the
truth of ``non­self'' later on in this book. So long as there is still
clinging to the concept of a self defilements cannot be eradicated.
There has to be first the eradication of the wrong view of self and then
other defilements can be eradicated stage by stage.

There were many monks, nuns and laypeople who dev­el­oped the Path and
realized the goal, each in their own situation. The development of the
eightfold Path does not mean that one should try to be detached
immediately from all pleasant objects and from existence. All realities,
including attachment, should be known and understood. So long as there
are conditions for attachment it arises. The development of
understanding cannot be forced, it must be done in a natural way. Only
thus can understanding, knowing and seeing, very gradually lead to
detachment. When one is a layfollower one enjoys all the pleasant things
of life, but understanding of realities can be developed. The monk who
observes the rules of monk­hood leads a different kind of life, but this
does not mean that he already is without attachment to pleasant objects.
He too should develop understanding naturally, in his own situation, and
come to know his own defilements.

The development of the Buddha's Path is very gradual, it is a difficult
and long way. It may take many lives before there can be the attainment
of enlightenment. Since the development of the Path is so difficult
there may be doubt whether it makes sense to start on this Path. It is
complicated to understand all phenomena of life, including our own
mental states, and it seems impossible to eradi­cate defilements. It is
useless to expect results soon, but it is beneficial to start to
investigate what our life really is: phenomena which are impermanent and
thus unsatis­factory. When we start on the Buddha's Path we begin to see
our many faults and vices, not only the coarse ones but also the more
subtle ones which may not have been so obvious. Before studying the
Buddhist teachings, selfish motives when performing deeds of generosity
were un­noticed. When the deep, underlying, impure motives for one's
deeds are realized is that not a gain? A sudden change of character
cannot be expected soon as a result of the Buddhist teachings, but what
is unwholesome can be realized as unwholesome, and what is wholesome can
be realized as wholesome. In that way there will be more truthfulness,
more sincerity in our actions, speech and thoughts. The disadvantage and
danger of unwholesome­ness and the benefit of wholesomeness will be seen
more and more clearly.

The Buddha taught about everything which is real and which can be
experienced in daily life. He taught about seeing and hearing, about all
that can be experienced through the senses. He taught that on account of
what is experienced through the senses there is attachment, aver­sion
and ignorance. We are ignorant most of the time of the phenomena
occurring in daily life. However, even when we only begin to develop
understanding we can verify the truth of what the Buddha taught. Seeing,
hearing, attachment, anger, generosity and kindness are real for
everybody. There is no need to label what is true for everybody as
``Buddhism''. When we begin to investigate what the Buddha taught there
will gradually be the elimination of ignorance about ourselves and the
world around us.

We read in the Kindred Sayings (IV, Kindred Sayings on Sense, The Third
Fifty, Chapter I, §111, Understanding):

\begin{quote}
By not comprehending, by not understanding, without detaching himself
from, without abandoning the eye, one is incapable of the destruction of
suffering. By not comprehending\ldots{} the ear\ldots{} nose\ldots{}
tongue\ldots{} body\ldots{} mind\ldots{} one is incapable of the
destruction of suffering.

But by comprehending, by understanding, by detaching himself from, by
abandoning the eye\ldots{} nose\ldots{} tongue\ldots{} body\ldots{}
mind\ldots{} one is capable of the destruction of suffering.
\end{quote}

In the following sutta we read that, for the destruction of suffering
colours, sounds, scents, savours, tangible objects and mind-states have
to be understood. This is the way leading to the end of suffering. The
Buddha taught about realities for the sake of our welfare and happiness.

\chapter{The Truth of suffering}

Old age, sickness and death are unavoidable. Separation from dear people
through death is bound to happen. We read in the Group of Discourses
(Sutta-Nipāta, III, 8, The Arrow, verses 574-582):\footnote{ I am using
  the translation of J. Ireland, Wheel Publication 82.}

\begin{quote}
Unindicated and unknown is the length of life of those subject to death.
Life is difficult and brief and bound up with suffering. There is no
means by which those who are born will not die. Having reached old age,
there is death. This is the natural course for a living being. With ripe
fruits there is the constant danger that they will fall. In the same
way, for those born and subject to death, there is always the fear of
dying. Just as the pots made by a potter all end by being broken, so
death is the breaking up of life.

The young and old, the foolish and the wise, all are stopped short by
the power of death, all finally end in death. Of those overcome by death
and passing to another world, a father cannot hold back his son, nor
relatives a relation. See! While the relatives are looking on and
weeping, one by one each mortal is led away like an ox to the slaughter.

In this manner the world is afflicted by death and decay. But the wise
do not grieve, having realized the nature of the world
\end{quote}

The Buddha consoled those who had suffered the loss of dear people by
explaining to them the impermanence of all phenomena of life. We read in
the commentary to the Psalms of the Sisters (Therīgāthā, Canto X, LXIII)
that a woman, named Kisā-gotamī, was completely overwhelmed by grief
because of the loss of her son. She went from door to door with his
corpse asking for medicine which could revive him. The Buddha said to
her: ``Go, enter the town, and bring from any house where yet no man has
died a little mustard seed.'' She did not find any family without
bereavement caused by death and she realized that everything is
impermanent. She went to the Buddha and he said:

To him whose heart on children and on goods Is centred, clinging to them
in his thoughts, Death comes like a great flood in the night, Bearing
away the village in its sleep.

Did the Buddha teach anything new? We all know that there has to be
separation and death, that everything in life is impermanent. Thinking
about impermanence is not of much help when one has suffered a loss. The
Buddha taught that there is change of mental states from moment to
moment and that also the physical units which consti­tute the body are
breaking up from moment to moment. He taught the development of the
wisdom which is the direct experience of the arising and falling away of
mental phenomena and physical phenomena. Kisā-gotamī did not merely
think about the impermanence of life, she realized through direct
experience the momentary break­ing up of the mental phenomena and the
physical phenomena. This changed her outlook on life and she could
recover from her deep sorrow.

Each mental state which arises falls away within split-seconds. At one
moment we may speak kindly to someone else but the next moment the kind
disposition has disappeared and we may be irritated and angry, we may
even shout. It is as if we are a completely different personality at
that moment. Actually this is true. Kindness has disappeared and the
angry disposition is a different mental state which has arisen. Seeing,
hearing or thinking are all different moments of consciousness which
arise and then fall away immediately. They each arise because of their
own conditioning factors. Seeing, for example is dependant on eyesense
and on its object, which is colour, and since these conditioning factors
do not last, also the seeing which is conditioned by them cannot last
either. Every reality which is dependant on conditions has to fall away.
Since the moment of consciousness which has fallen away is followed by a
new one it seems that there is a mind which lasts. In reality our life
is an unbroken series of moments of consciousness which arise and fall
away. Also bodily phenomena which arise, fall away. We know that the
body is subject to decay, that there is old age and death, but this is
not the wisdom which can directly realize the momentary breaking up of
the units which constitute the body. We do not notice their vanishing
after they have arisen because there are new bodily phenomena replacing
the ones that have fallen away. We can notice that there is sometimes
heat in the body, sometimes cold, sometimes suppleness, sometimes
stiffness. This shows that there is change of bodily phenomena. Also
what we call dead matter are physical phenomena which are arising and
vanishing all the time. Physical phenomena arise because of conditioning
factors. When we smile or cry, when we move our hand with anger or
stretch out our hand in order to give, there are different bodily
phenomena caused by different mental states. Bodily phenomena and also
the physical phenomena outside arise because of their own conditioning
factors and they have to fall away. Science also teaches the momentary
change of physical phenomena, but the aim of the Buddha's teachings is
completely different, the aim is detachment from all phe­nomena. The "
eye of wisdom" which sees impermanence is different from a microscope
through which one watches the change of the smallest physical units. The
wisdom which directly realizes the momentary impermanence of phenomena
eventually leads to detachment.

Our life can be compared with the flux of a river. A river seems to keep
its identity but in reality not one drop of water stays the same while
the river is flowing on and on. In the same way what we call a " person"
seems to keep its identity, but in reality there are mere passing mental
phenomena and physical phenomena. These phenomena arise because of their
appropriate conditions and then fall away. It can be noticed that people
have different charac­ters, but what is called " character" are
phenomena which have been conditioned by phenomena in the past. Since
our life is an unbroken series of moments of conscious­ness arising in
succession, the past moments can condition the present moment and the
present moment can condition the future moments. There were wholesome
and unwholesome moments in the past and these condi­tion the arising of
wholesome and unwhole­some moments today. What is learnt today is never
lost, moments of understanding today can be accumulated and in that way
understanding can develop.

We conceive life as a long duration of time, lasting from the moment of
birth until death. If the momentary arising and vanishing of each
reality is taken into consideration, it can be said that there is birth
and death at each moment. Seeing arises but it does not last, it falls
away imme­diately. At another moment there is hearing, but it does not
last either. Thinking changes each moment, there is thinking of
different things all the time. It can be noticed that there can only be
thinking of one thing, not more than one thing, at a time. It may seem
that thinking can last, but in reality there are different moments of
con­sciousness succeeding one another extremely rapidly. Feelings
change, there is pleasant feeling at one moment, at another moment there
is unpleasant feeling and at another moment again indifferent feeling.
The Buddha taught that what is impermanent is suffering, in Pāli
dukkha\footnote{The Pāli term dukkha is to be preferred, since the word
  ``suffering'' does not cover completely the meaning of the first noble
  Truth.}. Bodily pain and mental suffering due to the changeability of
things are forms of dukkha which are more obvious. The truth of dukkha,
however, comprises more than that. The truth of dukkha pertains to all
physical phenomena and mental states which are imper­manent. They are
unsatisfactory because, after they have arisen, they are there merely
for an extremely short moment and then they disappear completely. The
truth of dukkha is deep and difficult to understand.

We read in the Kindred Sayings (V, Mahā-vagga, Book XII, Kindred Sayings
about the Truths, Chapter 2, §1) that the Buddha, after his
enlightenment, when he was staying in the Deerpark at Isipatana, near
Vārānasi, preached to a group of five monks. He explained to them the
four noble Truths: the Truth of dukkha, the Truth of the origin of
dukkha, the Truth of the ceasing of dukkha, which is nibbāna, and the
Truth of the Path leading to the ceasing of dukkha. We read with regard
to dukkha:

Birth is dukkha, decay is dukkha, sickness is dukkha, death is dukkha;
likewise sorrow and grief, woe, lamentation and despair. To be conjoined
with what we dislike; to be separated from what we like, that also is
dukkha. Not to get what one wants, that also is dukkha. In short, these
five groups of grasping are dukkha.

The five groups (in Pāli khandhas) of grasping are all physical
phenomena and mental phenomena of our life which have been classified as
five groups. They are: the group of physical phenomena, and four groups
of mental phenomena comprising: the group of feelings, of remem­brance,
of mental activities (including all wholesome and unwholesome qualities)
and of consciousness. These five groups comprise all phenomena of life
which arise because of their own conditions and then fall away. Seeing
is dukkha, because it arises and falls away. Colour is dukkha, pleasant
feeling is dukkha, even wholesome mental states are dukkha, they all are
impermanent.

There may be theoretical understanding of the fact that all that can be
experienced is impermanent and therefore unsatisfactory or dukkha. The
Truth of dukkha, however, cannot be real­ized through theoretical
understanding alone. There can be thinking of the impermanence of
everything in life, but it is extremely difficult to realize through
one's own experience the arising and falling away, thus, the breaking up
from moment to moment of phe­nomena. Through the development of the
eightfold Path there can eventually be direct understanding of the
imper­manence of the phenomena of life and of their nature of dukkha.

All phenomena are impermanent. There should be pre­cise understanding of
what that ``all'' is. Otherwise there cannot be the realization of
impermanence and dukkha. We read in the Kindred Sayings (IV, Kindred
Sayings on Sense, First Fifty, Chapter 3, §23, The all) that the Buddha
said to the monks while he was at Sāvatthī:

\begin{quote}
Monks, I will teach you the all. Do you listen to it. And what, monks,
is the all? It is eye and visible object, ear and sound, nose and scent,
tongue and savour, body and tangible object, mind and mind-states. That,
monks, is called the ``all''.

Whoso, monks, should say: "Rejecting this all, I will proclaim another
all, it would be mere talk on his part, and when questioned he could not
make good his boast, and further would come to an ill pass. Why so?
Because, monks, it would be beyond his scope to do so.
\end{quote}

From this sutta we see that the Buddha's teaching is very concrete, that
it pertains to all realities of daily life:

\begin{itemize}
\item
  the seeing of visible object through the eyes;
\item
  the hearing of sound through the ears;
\item
  the smelling of odours through the nose;
\item
  the tasting of flavours through the tongue;
\item
  the experience of tangible object through the bodysense;
\item
  the experience of mental objects through the mind.
\end{itemize}

When one first comes into contact with the Buddhist teachings one may be
surprised that the Buddha speaks time and again about realities such as
seeing and hearing. However, the ``all'' has to be known and
investigated. There is such a great deal of ignorance of mental
phenomena and physical phenomena. Generally one is inclined to be
absorbed in thinking about people one saw or words one heard; one never
paid attention to seeing itself or hearing itself. One may even doubt
whether it is useful to do so. Seeing and hearing themselves are neither
wholesome nor unwholesome, but immediately after seeing and hearing all
kinds of defilements are bound to arise. All the different moments of
life should be investigated thoroughly, so that there can be elimination
of delusion about them.

There are different degrees of understanding realities. Thinking about
realities and about their imper­manence is theor­etical understanding
and this is not the realization of the true nature of realities.
Theoretical understanding, however, can be the foundation for direct
understanding of the realities which appear in daily life.

As we study the Buddhist scriptures we will learn about the realities
which are to be understood. There are three parts or ``baskets'' of the
Buddha's teachings: the Vinaya, the Suttanta or Discourses and the
Abhidhamma or ``higher teachings''. The Vinaya is the ``Book of
Discipline'' for the monks. The Suttanta are discourses of the Buddha
held at different places to different people. The Abhi­dhamma is a
detailed exposition of all mental phenomena and physical phenomena and
also of their conditioning factors and their different ways of
conditional relations. Although these three parts of the teachings are
different in form, they point to the same goal: the eradication of
defilements through the direct realization of the truth. When one
studies the different realities which are explained in detail in the
Abhidhamma, the goal should not be forgotten: the development of direct
understanding of realities when they appear. There is also Abhidhamma in
the suttas. The sutta about the ``All'' I quoted above is an example of
this. The deep meaning of the suttas cannot be understood without a
basic study of the Abhidhamma. The field of the Abhidhamma is immense
and we cannot grasp the whole contents. However, when one begins to
study it, at least in part, one will see that it can be of great
assistance for the understanding of our life. Some people have doubt as
to the authenticity of the Abhidhamma, they doubt whether it is the
teaching of the Buddha himself. As one studies the Abhidhamma one will
see for oneself that the Abhidhamma teaches about phenomena which can be
experienced at this moment. The Abhi­dhamma deals with seeing, visible
object, with all experi­ences through the senses and the mind, with all
whole­some qualities, with all defilements. The different parts of the
scriptures are one, they are the Buddha's teachings.

We read in the Kindred Sayings (IV, Kindred Sayings on Sense, Second
Fifty, Chapter I, §53, Ignorance) about the elimination of ignorance. We
read about a conversation of a monk with the Buddha about this subject:

\begin{quote}
``By how knowing, lord, by how seeing does ignorance vanish and
knowledge arise?''

``In him that knows and sees the eye as impermanent, monk, ignorance
vanishes and knowledge arises. In him that knows and sees visible
objects\ldots{} seeing-consciousness\ldots{} eye-contact\ldots{} the
pleasant, unpleasant or neutral feeling arising dependant on eye-contact
as impermanent, monk, ignorance vanishes and knowledge arises\ldots{}''
\end{quote}

The same is said about the realities pertaining to the ear, the nose,
the tongue, the bodysense and the mind. All these phenomena have to be
investigated in order to know them as they are.

Seeing arises, and shortly after that attach­ment to what is
seen arises but most of the time there is ignorance of these phenomena. Even when there is no pleasant feeling on account of what is seen there  can still be clinging. There is clinging time and again to seeing, to
visible object, to hearing, to sound, to all that can be experienced. We
would not like to be without eyesense or earsense and this shows that
there is clinging. We want to continue seeing, hearing and experiencing
all the objects which present themselves through the senses. What is
seen and what is experienced through the other senses falls away
immediately, but we erroneously believe that things last, at least for a
while. Because of our delusion we keep on clinging. When we do not get
what we want, when we lose people who are dear to us, or things we
possess, we are sad or even in despair. It is attachment which
conditions aversion or sadness. When we do not get what we like there is
dislike. All such mental states are realities of daily life and, instead
of suppressing them, they can be investigated when they appear. Then
their different characteristics can be distinguished.

Each phenomenon has a different characteristic and it arises because of
different conditions. For example, when we are in the company of
relatives or friends, we can notice that there are different moments of
consciousness. There are moments of attachment, moments that there is
clinging to our own pleasant feeling on account of the company of dear
people. It may seem that we think of other people's happiness, but we
are merely attached to our own happiness. There are other moments,
however, that we sincerely think of the other people's wellbeing and
happiness, that we do not think of ourselves. Attachment and unselfish
kindness have different characteristics and gradually their difference
can be learnt when they appear. It may seem complicated to analyse one's
mental states. One can, however, lead one 's life naturally, one can
enjoy all the pleasant things of life, and at the same time develop more
understanding of different moments of conscious­ness which arise, be it
clinging, unselfish kindness or generosity. In that way there can be a
more precise under­standing of the different characteristics of
phenomena.

When one begins to investigate the different phenomena of one's life,
one realizes that there is such an amount of ignorance. It is beneficial
to realize this, because that is the beginning of understanding. There
is ignorance of realities such as seeing, hearing or thinking. It is not
known precisely when there is seeing and when there is attachment to
what is seen. Realities arise and fall away very rapidly. There is
clinging to the objects which are experienced and their arising and
falling away is not realized. There is ignorance of the suffering and
the unsatisfactoriness inherent in all conditioned realities. Ignorance
and clinging are the conditions for rebirth into a new existence, for
continuation in the cycle of birth and death. When there is rebirth,
there is suffering again, there will again be old age, sickness and
death.

It is difficult to grasp the truth of dukkha, but one can begin to
develop more understanding of the phenomena which appear in one's life.
The Buddha taught Dhamma in order that people could investigate all
realities. The word ``dhamma'' has different meanings, but in its widest
sense dhamma is everything which is real and which has its own
characteristic. Seeing is dhamma, attachment is dhamma, anger is dhamma.
They are realities which can be expe­rienced by everybody. We can read
about seeing, attach­ment or anger, but when these realities occur we
can learn to distinguish their different characteristics. Knowledge of
realities can be acquired through the study of the Abhidhamma, but this
knowledge should be applied so that there can eventually be direct
understanding of realities. We are full of attachment, anger, avarice,
conceit, jealousy, full of defilements, but understanding of all these
realities can be developed. If dislike, for example, would be
suppressed, instead of knowing its characteristic when it appears, there
would be ignorance of the way it is conditioned. It would not be known
that it is attachment which conditions dislike. If there is ignorance of
what is wholesome and what is unwholesome, wholesome qualities could not
be developed. Understanding can be developed of the countless moments of
attachment which arise after seeing, hearing and the other experiences
through the senses. All realities arise because of their own conditions.

The development of direct understanding of realities is the Path leading
to the end of dukkha. The development of this Path is very gradual and
takes a long time. The characteristics of the different realities which
appear have to be thoroughly investigated and understood. In that way it
can be gradually seen that they arise each because of their own
conditions. What arises because of conditions has to fall away, it is
impermanent. The impermanence of realities, their momentary breaking up,
can only be realized at a later stage of the development of
under­standing. Eventually there can be the realization of the fact that
all conditioned realities which arise and fall away are dukkha. There
are different degrees of understanding the Truth of dukkha. When one
attains enlightenment one has understood the Truth of dukkha, of the
origin of dukkha, of the ceasing of dukkha and of the way leading to the
ceasing of dukkha.

\chapter{The Truth of non-self}

All phenomena of life are impermanent and dukkha. Seeing, colour,
hearing, sound, feeling, anger, greed or generosity, they all arise
because of their own conditions and then they fall away immediately.
There is no abiding ego or ``self'' who could cause the arising of these
phenomena or exert control over them. Realities which are impermanent
and dukkha are non-self.

We read in the Kindred Sayings (IV, Kindred Sayings on Sense, First
Fifty, Chapter 1, §1, impermanent, the personal) that the Buddha, while
staying at the Jeta Grove near Sāvatthī, said to the monks:

\begin{quote}
The eye, monks, is impermanent. What is impermanent, that is dukkha.
What is dukkha, that is void of the self. What is void of the self, that
is not mine; I am not it; it is not my self. That is how it is to be
regarded with perfect insight of what it really is.

The ear is impermanent\ldots{} The nose is impermanent\ldots{} The
tongue is impermanent\ldots{} The body is impermanent\ldots{} The mind
is impermanent. What is impermanent, that is dukkha. What is dukkha,
that is void of the self. What is void of the self, that is not mine; I
am not it; it is not my self. That is how it is to be regarded with
perfect insight of what it really is\ldots{} .
\end{quote}

We then read that through insight all clinging to the senses and the
mind is eradicated and that there are consequently no more conditions
for rebirth. In the fol­lowing suttas the same is said with regard to
colour, sound, scent, savour, tangible object and mind-object. They are
impermanent, dukkha and void of the self.

The truth of non-self, in Pāli anattā, is an essential element of the
Buddha's teachings. This truth has been taught by the Buddha alone, it
cannot be found outside the Buddhist teachings. Those who come into
contact with Buddhism for the first time may be bewildered, even
repelled by the truth of non-self. They wonder what the world would be
without a self, without other people. Do we not live with and for other
people? It is difficult to grasp the truth of non-self and its
implications in daily life.

What is called in conventional language a ``person'' or ``self'' is
merely a temporary combination of physical phenomena and mental
phenomena, which are depending on each other. They have been classified
as five groups, in Pāli khandhas: one group of all physical phenomena
and four groups of mental phenomena---feelings, remembrance, mental
activities and consciousness. The five khandhas are in a flux, in a
constant process of formation and dissolution. There is nothing lasting,
nothing eternal, nothing unchanging in life.

The khandhas which arise, fall away and do not return. Present khandhas
are different from past khandhas but they are conditioned by past
khandhas, and present khandhas condition in their turn future khandhas.
We read in the Dialogues of the Buddha (I,IX, Potthapāda Sutta)
that the Buddha explained to Citta about the three modes of personality:
the past, the present and the future personality. They are different,
but the past conditions the present and the present conditions the
future. We read that the Buddha explained this by way of a simile:

Just, Citta, as from a cow comes milk, and from the milk curds, and from
the curds butter, and from the butter ghee, and from the ghee junket;
but when it is milk it is not called curds, or butter, or ghee, or
junket; and when it is curds it is not called by any of the other
names\ldots{}

Just so, Citta, when any one of the three modes of personality is going
on, it is not called by the name of the other. For these, Citta, are
merely names, expressions, turns of speech, designations in common use
in the world. And of these a Tathāgata\footnote{Literally, " Thus-gone",
  epithet of the Buddha.} (one who has won the truth) makes use indeed,
but is not led astray by them.

We call by such or such a name what are actually the five khandhas.
People have different characters, different per­son­ali­ties. In reality
there is nothing static in what is called a person. The present
personality is different from the past personality, but it has
originated from the past personality. We read in the commentary to the
Debates (to the Kathāvatthu, Chapter I, the Person, 33, 34):

\begin{quote}
Given bodily and mental khandhas, it is customary to say such and such a
name, a family. Thus, by this popular turn of speech, convention,
expression, is meant: ``there is the person'' The Buddhas have two kinds
of discourse, the popular and the philosophical. Those relating to a
being, a person, a deva (divine being), a ``brahmas'', are popular
discourses, while those relating to impermanence, dukkha, non-self, the
khandhas, the elements, the senses are discourses on ultimate meaning. A
discourse on ultimate meaning is, as a rule, too severe to begin with;
therefore the Buddhas teach at first by popular discourse, and then by
way of discourse on ultimate meaning.

The Enlightened One, best of speakers, spoke two kinds of truth, namely,
the conventional truth and the ultimate truth, a third is not known.

Therein, a popular discourse is true in conventional sense. A discourse
on ultimate realities is also true, and as such, characteristic of
things as they are.
\end{quote}

Before studying the Buddhist teachings we only knew con­ventional truth:
the truth of the world populated by people and animals, the world of
persons, of self. Through the Buddhist teachings we learn about the
ultimate truth: the mental phenomena and physical phenomena which are
impermanent.

The truth of non-self is ultimate truth. It is deep and hard to
penetrate. It has been taught by way of similes in the Buddhist
scriptures and in the commentaries. The great commentator Buddhaghosa,
in his book the Path of Purification (Visuddhimagga), illustrates the
truth of non-self with similes from Buddhist scriptures. The Path of
Purification is a comprehensive exposition of the Buddha's teaching
based on old commentaries and the tradition of the monks in Sri Lanka,
written in the fifth century A.D. Buddhaghosa explains that when one
thinks of a whole of mind and body, one clings to the concept of person,
whereas when this ``whole'' is seen as different elements which are
impermanent, one will lose the perception of ``self''.
We read in the Path of Purification (XVIII, 25, 26):

\begin{verbatim}
As with the assembly of parts
The word "chariot" is countenanced,
So, when the khandhas are present,
"A being" is said in common usage
\end{verbatim}

Again, this has been said: ``Just as when a space is enclosed with
timber and creepers and grass and clay, there comes to be the term
Ôhouse', so too, when a space is enclosed with bones and sinews and
flesh and skin, there comes to be the term Ômaterial form'\footnote{ see
  Middle Length Sayings I, 28}.''

Further on (XVIII, 28) we read:

\begin{quote}
So in many hundred suttas it is only mentality-materiality that is
illustrated, not a being, not a person. Therefore, just as when the
component parts such as axles, wheels, frame, poles, etc. are arranged
in a certain way, there comes to be the mere term of common usage
``chariot'', yet in the ultimate sense when each part is examined, there
is no chariot and just as when the component parts of a house such as
wattles, etc. are placed so that they enclose a space in a certain way,
there comes to be the mere term of common usage ``house'', yet in the
ultimate sense there is no house,so too, when there are the five
khandhas of clinging, there comes to be the mere term of common usage
``a being'', ``a person'', yet in the ultimate sense, when each
component is examined, there is no being as a basis for the assumption
``I am'' or ``I''; in the ultimate sense there is only
mentality-materiality. The vision of one who sees in this way is called
right vision.
\end{quote}

If life can be considered as existing in just one moment, it will be
less difficult to understand the truth of non-self. In the Mahā-Niddesa
(number 6, Decay) the Buddha ex­plains that life is extremely short. In
the ultimate sense it lasts only as long as one moment of consciousness.
Each moment of consciousness which arises falls away com­pletely, to be
succeeded by the next moment which is different.

We read in the Path of Purification (XX, 72) a quotation from the
Mahā-Niddesa text about the khandhas which are impermanent:

\begin{verbatim}
No store of broken states, no future stock;
Those born balance like seeds on needle points.
Break-up of states is fore-doomed at their birth;
Those present decay, unmingled with those past.
They come from nowhere, break up, nowhere go;
Flash in and out, as lightning in the sky.
\end{verbatim}

One is used to thinking of a self who coordinates all the different
experiences through the senses and the mind, a self who can see, hear
and think all at the same time, but in reality there can be only one
moment of consciousness at a time which experiences one object. At one
moment life is seeing, at another moment life is hearing and at another
moment again life is thinking. Each moment of our life arises because of
its own conditions, exists for an extre­mely short time and then falls
away. Seeing arises dependant on eyesense, on colour and on other
factors. It exists just for a moment and then it is gone. Seeing arises
and falls away very rapidly, but then there are other moments of seeing
again and this causes us to erro­neously believe that seeing lasts. The
seeing of this moment, however, is different from seeing which is just
past. Colour which appears at this moment is different from colour which
is just past. How could there be a self who exerts control over seeing
or any other reality? Realities such as kindness and anger arise because
of their own conditions, there is no self who could exert control over
them. We would like to speak kindly, but when there are conditions for
anger, it arises. We may tell ourselves to keep silent, but, before we
realize it, angry words have been spoken already. There was anger in the
past and this has been accumulated. That is why it can arise at any
time. Anger does not belong to a person, but it is a reality. We are
used to identifying ourselves with realities such as anger, generosity,
seeing or thinking, but it can be learnt that they are mental phenomena,
arising because of their own conditions. We are used to identifying
ourselves with our body, but the body consists of changing physical
phenomena, arising because of their own conditions. Bodily phenomena are
beyond control; ageing, sickness and death cannot be prevented.
Realities come and go very rapidly, they can be compared with a flash of
lightning. One cannot exercise any power over a flash of lightning, it
is gone as soon as it has been noticed. Evenso, one cannot exert control
over the mental and physical phenomena of one's life.

The outer appearance of things deludes us as to what is really there:
fleeting phenomena which are beyond control. We read in the commentary
to the Dhammapada (Buddhist Legends II, Book IV, Story 2) about a monk
who meditated on a mirage, but was unable to reach the state of
perfection. He decided to visit the Buddha and on his way he saw a
mirage. We read that he said to himself: ``Even as this mirage seen in
the season of the heat appears substantial to those who are far off, but
vanishes on nearer approach, so also is this existence unsubstantial by
reason of birth and decay.''

We read that he meditated on this mirage. Wearied from his journey he
bathed in the river Aciravatī and then sat near a waterfall:

\begin{quote}
As he sat there watching great bubbles of foam rising and bursting, from
the force of the water striking against the rocks, he said to himself,
``Just so is this existence also produced and just so does it burst.''
And this he took for his subject of meditation.

The Teacher, seated in his perfumed chamber, saw the Elder and said,
``Monk, it is even so. Like a bubble of foam or a mirage is this
existence. Precisely thus is it produced and precisely thus does it pass
away.'' And when he had thus spoken, he pronounced the following stanza:
\end{quote}

\begin{verbatim}
 "He who knows that this body is like foam, 
 he who clearly comprehends,
 that it is of the nature of a mirage,
 Such a man will break the flower-tipped arrows of Māra
 and will go where the King of Death will not see him."
\end{verbatim}

We read that the monk at the conclusion of this stanza reached the state
of perfection. Māra represents all that is evil, he is the King of
Death. The person who has era­dicated all defilements will not be
reborn, there will not be for him anymore old age, sickness and death,
thus, the ``King of Death'' will not see him anymore.

Life is like a mirage, we are time and again deceived and tricked by the
outer appearance of things. We believe that what we experience can last,
at least for a while, and that there is a self who experiences things, a
lasting person­ality. We take our wrong perceptions to be true, we have
a distorted view of realities. Through the development of precise
understanding of different realities which appear one at a time, our
distorted view can be corrected.

It is difficult to understand and accept that whatever arises does so
because of its own conditions and that it is beyond control. People
generally want to control their lives, to take their destinies in their
own hands. It can, however, even on the theoretical level, be understood
that it is impossible to control one's life. One cannot control one's
body, one cannot control the different moments of consciousness which
arise. When there is, for example, the tasting of a delicious sweet,
there is bound to be clinging to the flavour immediately after having
tasted it. Tasting arises dependent on tastingsense, on flavour which
impinges on tastingsense and on other conditions; clinging to the
flavour arises because of its own conditions, because of the
accumulation of the tendency to clinging. Different moments of
consciousness succeed one another so rapidly that it seems that several
of them can occur at the same time. So long as there is no precise
understanding they cannot be distinguished from each other. In reality
only one moment of consciousness can arise at a time. I will give an
example of different moments of consciousness, arising each because of
their own conditions. Someone had given me a huge teddybear which I put
in a chair. Time and again it happened that when I walked past it at
dusk there were moments of fear before I realized that it was a
teddybear. There was seeing which experienced colour or visible object
impinging on the eyesense, and then, before knowing that there was a
teddybear, there were many other moments of conscious­ness. There can be
fear on account of what is seen, before it is known that it is a
harmless object. There were moments of recognizing and defining and when
there was the registration that there was only a toy, the fear was gone.
This example illustrates that there are different conditions for the
different moments of consciousness which arise. They arise each because
of their own conditions and in a particular order. They arise and fall
away so rapidly that there would not even be time to control or direct
them. There is no mind, no soul which lasts, merely rapidly changing
moments of consciousness.

It is inevitable that questions arise with regard to the implication of
the truth of non-self in one's life. People generally have questions as
to the existence of a free will. If there is no self, only empty
phenomena which appear and disappear, can there be a free will, can one
have a free choice in the taking of decisions in life? Are a free will
and self-control not essential elements of human life? The truth of
non-self seems to imply that one's whole life is determined, even
predestined, by conditions. The answer is that a free will presupposes a
lasting personality who can exert power over his will. Since there is no
``self'', merely impermanent phenomena arising because of conditions,
there is no free will independent of conditions. The will or desire to
act can be wholesome at one moment and unwholesome at another moment.
When there is anger, there is volition which is unwholesome, and it can
instigate words of anger. When there is generosity, there is volition
which is wholesome, it can motivate deeds of generosity. There can be
the decision to do particular things, such as the development of
generosity or of understanding, but there is no person who decides to do
this. There are different moments of decision arising because of
different conditions. What one decides to do depends on past
accumulations of wholesomeness and unwholesomeness, on one's education,
on the friends one associates with. It may be felt that, since
accumulations of wholesomeness and unwholesomeness in the past condition
one's actions, speech and thoughts today, one would be a helpless victim
of these accumulated conditions. What is the sense of life if everything
is determined. So long as there is clinging to a concept of self there
is enslavement, no freedom. When understanding is developed which can
eliminate the clinging to a self one becomes really free. Also the
development of under­standing is conditioned, it is conditioned by
previous moments of understanding, by association with someone who can
explain the Dhamma, by the study of the Buddhist teachings. Whatever we
think or do is dependent on conditions which operate in our life in an
intricate way. The seventh book of the Abhidhamma deals entirely with
the different conditions for all mental and physical phe­nomena of life,
with the aim to help people to have more understanding of these
condi­tions. Even freedom is dependent on conditions. As understanding 
of reali­ties develops will there be the letting go of clinging to the 
importance of self, the clinging to wrong perceptions of reality. 
Eventually all defilements can be eradicated by right understanding and 
is this not what can be called the highest freedom?

In order to be able to understand the truth of non-self, the difference
has to be known between what is real in the ultimate sense and what is
real in conventional sense. It is difficult to clearly know the
difference and I will deal with this subject again later on. Seeing,
hearing, colour, sound or thinking are real in the ultimate sense. This
does not mean that they are abstract categories. They have each their
own characteristic and they can be directly expe­rienced. Seeing, for
example has a characteristic which is different from the characteristic
of hearing. These char­acter­istics do not change, they are the same for
everybody. Seeing is always seeing, hearing is always hearing, no matter
how one names them. Concepts or ideas such as person, world, animal, are
conventional realities one can think of, but they are not real in the
ultimate sense. Thinking of concepts such as person or animal is not
necessarily unwholesome; we can think of them in a wholesome way or in
an unwholesome way. However, we delude ourselves if we take concepts for
realities. It is essential to learn the difference between realities and
concepts, otherwise there cannot be the development of the Buddha's
Path.

So long as understanding has not been developed to the stage that the
momentary breaking up of physical phenomena and mental phenomena has
been realized, it is impossible to see things as they really are. We
believe that seeing lasts for a while and that what is seen also lasts.
Our world seems to be full of people, we believe that we really see
them. In reality seeing doesn't last and colour which is seen doesn't
last either. When we ``see'' people the situation is the same as
watching the projected images on a screen which are rapidly changing. We
``see'' the image of a person or a thing, but the outer appearance is
misleading. In reality there are many different moments arising and
falling away, succeeding one another. There are processes of seeing,
recognizing, classifying, defining and thinking. When it seems that we
see a ``whole'', the image of a person, it is actually thinking which is
conditioned by seeing, by the experience of what is visible.

The Buddha spoke about all that can be experienced through the senses
and through the mind in order to help people to develop understanding of
realities and to know the truth about them, to realize them as
impermanent, dukkha and not self. Seeing is a reality, but it is not
self, hearing is a reality, but it is not self, thinking is a reality,
but it is not self.

A question which may arise is the following: if people do not exist,
what is the sense of developing kindness, which has to be directed
towards people, what is the sense of committing oneself to the
improvement of the world? The answer is that knowing the truth about
realities is no impediment to deal with people, to perform deeds of
kindness and to commit oneself to the improvement of the world. Buddhism
does not propagate a passive attitude towards the world, on the
contrary, it promotes the performing of one's tasks with more
unselfishness, with more wholesomeness. We usually think of people in an
unwholesome way, with clinging, aversion and delusion. We cling to an
image of ourselves and also to images of other people. We have an image
of how they should behave towards us. When someone else does not conform
to the image we have of him we are disappointed or even angry. Clinging
to images we form up conditions many kinds of defilements, such as
conceit, jealousy, avarice or possessiveness. Through the Buddhist
teachings we can learn to think of people in the right way, that is,
without clinging to false images. While we are in the company of people
and talk to them there can be the development of understanding of
realities which appear through the senses and the mind. The realization
of the truth that there is no lasting person or self, merely fleeting
phe­nomena, does not mean that one has to shun one's task in society.
The Buddha himself was caring for other people, he was thinking of his
disciples, he was intent on the welfare of all beings, but he had no
wrong view of an abiding person, of a self. He was an example of
kindness, patience and compassion. He visited sick monks and looked
after them, he preached Dhamma for fortyfive years. He exhorted people
to develop kindness and compassion towards other beings. Even when one
has realized the truth of non-self one can still think of beings, but
instead of thinking with clinging, with selfishness, there are
conditions to think more often in a wholesome way, and this is to the
benefit of oneself and others.

There is no lasting substance or self in the combination of fleeting
physical phenomena and mental phenomena we call ``person''. Neither is
there a ``higher self'' outside. Some people believe that what we could
call a self will after death be dissolved into a ``higher self'' into
the ``All'', or the cosmos. This is not the Buddha's teaching. Even
nibbāna, the unconditioned reality, is not self. All conditioned
phe­nomena of life are impermanent, dukkha and not self. The
unconditioned reality, nibbāna, does not have the characteristics of
impermanence and dukkha, but it does have the characteristic of
non-self. We read in the Dhammapada (verse 277-279):

\begin{verbatim}
All conditioned realities are impermanent.
Who perceives this fact with wisdom,
Straightaway becomes dispassionate towards suffering.
This is the Path to Purity.
All conditioned realities are dukkha.
Who perceives this fact with wisdom,
Straightaway becomes dispassionate towards suffering.
This is the Path to Purity.
All dhammas are non-self.
Who perceives this fact with wisdom,
Straightaway becomes dispassionate towards suffering.
This is the Path to Purity.
\end{verbatim}

The text states that all dhammas are non-self. Nibbāna is not a
conditioned reality, but it is real, it is dhamma. Therefore, when, the
text states that all dhammas are non-self, nibbāna is included.

The development of the eightfold Path is in fact the development of
understanding of ultimate realities: of seeing, colour, hearing, sound,
of all that can be expe­rienced through the senses and the mind. The
reader may find it monotonous to read in the texts of the scriptures
time and again about these realities. The aim of the teaching on
ultimate realities, however, is the eradication of clinging to the 
concept of self. The clinging to the concept of self has to be 
eradicated first before there can be the elimina­tion of other defile­
ments. When a person can be seen as five khandhas, mere elements, which 
are arising and vanishing, there are conditions for being less inclined 
to attachment and aversion towards the vicissitudes of life, such as 
praise and blame, gain and loss, which play such an important role in 
our life. We read in ``The Simile of the Elephant's Footprint'' (Middle 
Length Sayings I,28) that the Buddha's disciple Sāriputta explained to 
the monks realities by way of elements. He explained that the body 
should not be seen as ``I'', ``mine'' , or ``I am''. We read:

\begin{quote}
Your reverences, if others abuse, revile, annoy, vex this monk, he
comprehends: ``This painful feeling that has arisen in me is born of
sensory impingement on the ear, it has a cause, it is not without a
cause. What is the cause? Sensory impingement is the cause.'' He sees
that sensory impingement is impermanent, he sees that feeling\ldots{}
perception\ldots{} mental activities are impermanent, he sees that
consciousness is impermanent. His mind rejoices, is pleased, composed
and is set on the objects of the element.
\end{quote}

We are inclined to blame other people when they speak in a disagreeable
way, instead of realizing that there is merely sound impinging on the
earsense, elements impinging on elements. So long as there is clinging
to a self realities cannot be seen as mere elements. This sutta makes
clear that it is beneficial to understand the truth of non-self. It can
only be realized very gradually, in developing understanding of the
realities included in the five khandhas.

\chapter{The mind}

The Buddha taught the truth of non-self. What is called the mind or the
soul is not a self, but ever-changing mental elements which are arising
and falling away. The implication of this truth is difficult to grasp.
Before coming into contact with Buddhism we considered the mind to be
the core and essence of the human personality. We considered the mind as
that which thinks, takes decisions and charts the course of our life. In
order to understand the Buddha's teaching on the mind as non-self, it is
necessary to have a more detailed knowledge of the mind. The word mind
is misleading since it is associated with particular concepts of Western
philosophy, it is usually associated merely with thinking. The mind
according to the Buddhist teaching experiences or cognizes an object,
and this has to be taken in its widest sense. I prefer therefore to use
the Pāli term citta (pronounced ``chitta''). Citta is derived from the
Pāli term " cinteti", being aware or thinking. Citta is conscious or
aware of an object.

``Mind'', ``soul'' or ``spirit'' are ``conventional realities''. Through
the Buddhist teachings we learn about ultimate realities as I explained
in the preceding chapter. All mental activities we used to ascribe to "
our mind" are carried out by citta, not by one citta, but by many
different cittas. Cittas are moments of consciousness which are
imper­manent, they are arising and falling away, succeed­ing one
another. Our life is an unbroken series of cittas. If there were no
citta, we would not be alive, we could not think, read, study, act or
speak. When we walk or when we stretch out our hand to take hold of
something, it is citta which conditions our movements. It is citta which
per­ceives the world outside; if there were no citta nothing could
appear. The world outside appears through eyes, ears, nose, tongue,
bodysense and mind. We think of what is seen, heard or experienced
through the other senses. There are not merely cittas which think, the
cittas which think are alternated with cittas which see, hear or
experience objects through the other senses. When we touch something
which is hard or soft, there are cittas which experience tangible object
through the bodysense, and then there are cittas which think of what was
touched, a table or a chair.

Before we studied the Buddhist teachings we did not consider the mind as
a reality which can see or hear. The Buddha taught that also seeing and
hearing are cittas. There is a great variety of cittas which each
experience an object. The citta which sees, seeing-consciousness,
exper­iences an object: visible object or colour. It experiences visible
object through the eyesense. Eyesense is the ``doorway'' through which
seeing-consciousness exper­iences visible object. Hearing-consciousness
experiences sound through the doorway of the earsense. Seeing and
hearing are entirely different cittas which are depending on different
conditions. Cittas experience objects through the doorways of eye, ear,
nose, tongue, bodysense and mind. Before studying the Buddhist teachings
we did not pay attention to seeing as being a citta experiencing visible
object through the eye-door, or to hearing as being a citta experiencing
sound through the ear-door. Cittas, objects and doorways are ultimate
realities taught by the Buddha.

One may doubt the usefulness of knowing details on cittas, objects and
doorways. It is important to know more thoroughly the phenomena of our
life which are occurring all the time. We are deluded as to the truth
when we believe that they are lasting and that they are ``self'', or
belonging to a ``self'', that we can exert control over them. The Buddha
taught that they are impermanent, dukkha and non-self. These
characteristics are not abstract categories, they pertain to seeing,
eyesense, visible object, to all phenomena which are arising and
falling away from moment to moment. Since understanding of the truth of
these phenomena can only gradually develop, we should begin to
investigate them more closely. In the ultimate sense there are merely
mental phenomena and physical phenomena. So long as they cannot be
distinguished from each other, there cannot be a precise knowledge of
them.

The citta which sees, seeing-consciousness, is a mental phenomenon, it
experiences an object. It is dependent on eyesense, which is a physical
phenomenon. Eyesense does not see but it has the quality of receiving
colour, so that seeing-consciousness can experience that colour. Colour
or visible object is also a physical phenomenon, it cannot experience
anything. Seeing, hearing and the experiences through the other senses
are dependent on conditions. If there were no doorways the different
sense objects could not be experienced, and consequently what we call
``the world outside'' could not appear. When we are fast asleep, without
dreaming, the world does not appear. We do not know who our parents or
friends are, we do not know the place where we are living. When we wake
up the world around us appears again. We can verify that there is
impingement of the sense objects on the appropriate senses and this is
the condition for the experience of the world around us. There are
cittas which see, hear and experience the other sense objects, and these
experiences condition thinking about the world of people and things. We
are usually absorbed in our thoughts concerning the people and things
around us and we do not realize that it is citta which thinks. We could
not think of ``self'', person or possessions, which are conventional
realities, if there were not the ultimate realities of colour, sound and
the other sense objects and the cittas which experience them through the
appropriate doorways.

There can be merely one citta at a time, experiencing one object. It
seems that several cittas can occur at the same time, but in reality
this is not so. Different cittas, such as seeing and hearing, experience
different objects and are dependent on different doorways. Seeing,
hearing and thinking are different cittas arising at different moments.
We can notice that seeing is not hearing, that they are different
experiences. If they would occur at the same time we would not be able
to know that they are different. Cittas arise and fall away very
rapidly; the citta which has fallen away is immediately succeeded by the
next citta. It seems that seeing, hearing or thinking can last for a
while, but in reality they exist merely for an extremely short moment.

There is a great variety of cittas which arise because of their
appropriate conditions. There are cittas which see, hear, experience
objects through the other senses and think about these objects. The
cittas which see, hear, smell, taste or experience an object through the
bodysense neither like nor dislike the object, they do not react to the
object in an unwholesome or a wholesome way. These types of citta are
neither kusala, wholesome, nor akusala, unwholesome. However, shortly
after they have fallen away cittas arise which react to the objects
experienced through the senses either in an unwholesome way or in a
wholesome way. Thus, there are kusala cittas, there are akusala cittas,
and there are cittas which are neither kusala nor akusala. Time and
again seeing or hearing arises and on account of the object which is
expe­rienced cittas arise which are either kusala or akusala. When
there is thinking, there is either kusala citta or akusala citta. There
are also cittas which motivate good or bad actions and speech. When we
give a present there are wholesome cittas, kusala cittas with generosity
which motivate our giving. When we speak harsh words, there are
unwholesome cittas, akusala cittas with anger which motivate our speech.

Different inclinations to kusala and akusala have been accumulated.
Accumulated tendencies are lying dormant, but they can condition the arising of kusala citta or akusala citta at any time when there is an opportunity. In this connection the term ``subconsciousness'' is used     in West­ern psychology, designating that part of the mind which is not
ordinarily known, but which shows itself for example in dreams. The term
subconsciousness is misleading, it implies something static. In reality
there are accumulated tendencies, but they are not static, they are
accumulating from moment to moment; they are conditions for the arising
of kusala citta or akusala citta later on. Each moment of kusala citta
or akusala citta arising today is a condition for the arising of kusala
citta or akusala citta in the future. Each citta which arises falls
away, but since it is succeeded by the next citta without any interval,
the process of accumulation can go on from moment to moment.

There are different types of kusala citta and of akusala citta. It is
important to learn more about them in order to understand ourselves, the
way we behave towards others in action and speech, and the way we react
towards pleasant and unpleasant events. It is citta which motivates good
deeds and evil deeds. We read in the Middle Length Sayings (II,78, Discourse to Samaṇamaṇèikā) that the Buddha explained to the
carpenter Pañcaka"ga about akusala cittas and kusala cittas:

\begin{quote}
And which, carpenter, are the unskilled moral habits? Unskilled deed of
body, unskilled deed of speech, evil mode of livelihood these,
carpenter, are called unskilled moral habits. And how, carpenter, do
these unskilled moral habits originate? Their origination is spoken of
too. It should be answered that the origination is in the citta. Which
citta? For the citta is manifold, various, diverse. That citta which has
attachment, aversion, ignorance originating from this are unskilled
moral habits.
\end{quote}

The Buddha also said of skilled moral habits that they originate from
the citta, the citta which is without attachment, aversion and
ignorance. Thus, all evil deeds originate from akusala citta and all
wholesome deeds originate from kusala citta.

Akusala can be described as an unhealthy state of mind, as unskilled,
blameworthy, faulty, unprofitable, as having unhappy results. Kusala can
be described as a healthy state of mind, as skilful, faultless,
profitable, as having happy results.

We read in the above quoted sutta that the citta is manifold, various,
diverse. Akusala citta with attach­ment is quite different from
kusala citta with generosity. What types of reality are attachment and
generosity? Are they cittas or are they other types of reality? They are
mental qualities, mental factors which can accompany citta. Attachment
is an unwholesome mental quality, a defilement, whereas generosity is a
wholesome mental quality. Citta can think, motivate actions or speech
for example, with attachment, with anger, with generosity, with
compassion. There is only one citta at a time, but it is accompanied by
several mental factors or mental co-adjuncts, and these condition the
citta to be so various. Greed, avarice, anger, jealousy or conceit are
unwhole­some mental factors which can accompany akusala citta.
Generosity, loving kindness, compassion or wisdom are whole­some mental
factors which can accompany kusala citta. The mental factors which
accompany citta in various combina­tions arise and fall away together
with the citta.

The commentary to the first book of the Abhidhamma, the Expositor (I,
Part II, Chapter I, 67), uses a simile of the king and his retinue. Just
as the king does not come without his attendants, the citta does not
arise alone but is accompanied by several mental factors. As to the
cittas which arise all the time in daily life, it can be said that citta
is the chief, the principal, in knowing the object, and that the mental
factors assist the citta. The citta which thinks, for example with
generosity, is the chief in knowing the object, and generosity assists
the citta to think in a wholesome way. The citta which thinks with
jealousy is the chief in knowing the object, and jealousy assists the
citta to think in an unwholesome way.

Among the unwholesome mental factors which accom­pany akusala citta
there are three which are called ``roots'', namely: attachment, aversion
and ignorance. Among the wholesome mental factors which accompany kusala
citta there are three roots, namely: non-attachment, non-aversion and
wisdom. The word ``root'' is used in the Buddhist teachings, since it is
the firm support for the citta, being an important condition, just as
the root of a tree is the firm support for the tree, the means of
providing saps, necessary for its growth. The unwholesome roots of
attachment, aversion and ignorance which can be associated with akusala
citta have many shades and degrees; they can be coarse or more subtle.
Attachment can be so strong that it motivates bad deeds such as stealing
or lying, but it can also be of a more subtle degree, a degree of
attachment which does not motivate any deed. Attachment can be expecting
something pleasant for oneself, wishing, liking, longing, affection,
self-indulgence, lust, possessiveness or covetousness. Even when we hope
that other people like us, when we wish to have a good name, there are
akusala cittas rooted in attachment. When we, for example, give a
present to someone else there is generosity, but there can also be
moments of hoping or expecting to gain something in return for our gift.
Such expectations are motivated by clinging. Akusala is not the same as
what is generally meant by sin or immorality. Also the more subtle
degrees of attachment which do not motivate bad deeds are akusala, they
are unhelpful, harmful. They are accumu­lated from moment to moment and
thus attachment increases evermore. Clinging is deeply rooted and it is
important to know our deeply rooted tendencies. Affection is a form of
attachment which is in society not regarded as harmful. One feels
affection for parents, relatives, children or friends. It should be
understood, however, that when there is affection, there is actually
clinging to one's own pleasant feeling, derived from being in the
company of a loved one. When there is mourning for someone who has died,
it is sadness conditioned by clinging to oneself. Affection
conditions fear of loss, aversion and sadness. We read in the Kindred
Sayings (IV, Part VIII, Kindred Sayings about Headmen, §11) that the
Buddha, while staying at Uruvelakappa, explained to the headman
Bhadragaka that clinging is the cause of dukkha. We read that Bhadragaka
said:

\begin{quote}
"Wonderful, lord! Strange it is, lord, how well said is this saying of
the Exalted One: ÔWhatsoever dukkha arising comes upon me, all that is
rooted in desire. Desire is indeed the root of dukkha.'

Now, lord, there is my boy, Ciravāsi is his name. He lodges away from
here. At the time of rising up, lord, I send off a man, saying: ÔGo, my
man, inquire of Ciravāsi.' Then, lord, till that man comes back again, I
am in an anxious state, fearing lest some sickness may have befallen
Ciravāsi."

``Now, what do you think, headman? Would sorrow and grief, woe,
lamentation and despair come upon you if your boy Ciravāsi were slain or
imprisoned or had loss or blame?''

``Lord, if such were to befall my boy Ciravāsi, how should I not have
sorrow and grief, woe, lamentation and despair?''

``But, headman, you must regard it in this manner: `Whatsoever dukkha
arising comes upon me, all that is rooted in desire, is joined to
desire. Desire is indeed the root of dukkha.'\,''
\end{quote}

It is impossible to be without clinging so long as the state of
perfection has not been reached. We cannot force ourselves not to have
clinging, but it is beneficial to realize when there is clinging, even
when it is of a subtle degree, and when there is detachment. There is
attachment when we like landscapes, when we enjoy shopping or talking to
friends, or even when we get up in order to fetch a glass of water.
Attachment can be accompanied by pleasant feeling or by indifferent
feeling. At the moment of indifferent feeling there can still be
attachment, but we may not notice it.

Aversion is another unwholesome root. Aversion dislikes the object which
is experienced, whereas attachment likes it. Aversion cannot arise at
the same time as attachment, but it is conditioned by it. Aversion has
many shades and degrees, it can be dissatisfaction, frustration,
disappoint­ment, dejection, sadness, fear, grief, despair, revulsion,
resentment, moodiness or irritability. Unpleasant feeling invariably
goes together with this unwholesome root. Even when we have a slight
feeling of uneasiness there is citta rooted in aversion. When we have
envy or stinginess there is citta rooted in aversion. In the case of
envy, one dislikes it that someone else enjoys pleasant things, one
wants to obtain them for oneself. In the case of avarice one does not
want to share one's possessions with someone else. Aversion can also
motivate killing, harsh speech rudeness or cruelty.

Another unwholesome root is ignorance. This is not the same as what is
meant by ignorance in conventional language. In Buddhism ignorance has a
specific meaning: it is ignorance of the characteristic of kusala and of
akusala, of the truth of non-self, of the four noble Truths, in short,
of ultimate realities. There are many degrees of ignorance. Ignorance is
the root of all evil. Whenever there is citta rooted in attachment and
citta rooted in aversion, there is also the root of ignorance. When one
hears a pleasant sound, attachment is likely to arise and then there is
ignorance as well. When one hears a harsh sound, aversion is likely to
arise and then there is ignorance as well. Ignorance does not know the
realities which arise, it does not know that attachment and aversion are
akusala. Ignorance is like darkness or blindness. When there is
ignorance the real nature of realities is covered up.

The three wholesome roots of non-attachment, non-aversion and wisdom
have many shades and degrees. Non-attachment can be unselfishness,
generosity, renun­cia­tion or dispassion. Each kusala citta is rooted in
non-attachment. Whenever kusala citta arises, there is no clinging at
that moment but detachment. Each kusala citta is rooted not only in
non-attachment, it is also rooted in non-aversion. Non-aversion has many
degrees: it can be loving-kindness, forbearance or endurance.
Loving-kindness is directed towards beings, and forbearance or
endur­ance can also pertain to situations and things. When the
temperature is too hot or too cold there is bound to be dislike. When
the benefit of forbearance is seen, one is not disturbed by the
temperature and one does not complain. Wisdom is the third wholesome
root. Wisdom does not accompany every kusala citta. Wisdom is a condition
for the arising of kusala citta more often. Wisdom or understanding in
Buddhism is understanding of realities. It has many degrees, it can be
theoretical understanding of realities or direct understanding of the
reality which appears. It can be understanding of kusala as kusala, of
akusala as akusala, of good and evil deeds and their results, of the
truth of non-self, of the four noble Truths. Understanding can be
gradually developed. The direct understanding of realities leads to the
eradication of defilements.

When kusala citta arises there are no attachment, aversion or
ignorance with the citta. Kusala citta motivates wholesome deeds and
speech. It depends on accumu­lations of kusala and akusala in the past
what type of citta arises. Good friends or bad friends one associates
with are also an important condition for the arising of kusala cittas or
akusala cittas. When one associates with a wise friend there are
conditions for the arising of kusala citta more often. There are many
more akusala cittas arising than kusala cittas because of the
accumulated defilements which condition them, but this is unnoticed.
Just as we do not notice the amount of dirt on our hands until we wash
them, evenso do we not know the amount of defilements until
understanding of realities is developed.

Citta experiences pleasant and unpleasant objects through the senses and
through the mind-door. When a pleasant object is experienced, attachment
is likely to arise and when an unpleasant object is experienced,
aversion is likely to arise. It is natural that pleasant objects are
liked and unpleasant objects are disliked. It seems that we are ruled by
the objects which are experienced. The pleasant object or unpleasant
object is a condition for the citta which arises, but there is nothing
compulsive in the nature of the object that could determine the reaction
towards it. It depends on one's accumulated inclinations whether one
reacts in a wholesome way or in an unwholesome way to the pleasant and
unpleasant objects which are experienced through the senses and through
the mind-door. After seeing, hearing or the experience of objects
through the other senses there can be ``unwise attention'' or ``wise
attention'' to the object. When there is unwise attention to the object,
there are akusala cittas, and when there is wise attention to the
object, there are kusala cittas. When there is a pleasant object, 
attachment can arise and in that case there is unwise attention. We may,
for example, only be intent on our own enjoyment of the pleasant object
and not inclined to share it with others. Whereas, when there is wise
attention, we are inclined to share a pleasant object with others, and
then kusala cittas with generosity arise. When there is an unpleasant object, there can be aversion and thus there is unwise attention.    Someone else may for example speak harsh words to us and most of the 
time we dislike such speech, we even blame that person for his harsh speech. Aversion, however, does not necessarily have to arise.
When it is remembered that the person who speaks harshly makes himself
unhappy there may be compassion instead of anger or aversion. When there
is wise attention there can be forbearance and patience even when the
object is unpleasant.

It is beneficial to learn more details about the many different types of
citta: kusala citta, akusala citta and citta which is neither kusala nor
akusala. When there is ignorance of akusala and kusala, the disadvantage
of akusala and the benefit of kusala cannot be seen. We long for
pleasant objects and we dislike unpleasant objects. Through the Buddhist
teachings one learns that whatever arises is dependent on conditions.
Sometimes there are conditions for the experience of pleasant objects
and sometimes for the experience of unpleasant objects, nobody can exert
control over the cittas which arise. Pleasant objects cannot last and
therefore clinging to them will only lead to frustration and sadness.
Time and again attachment, aversion and ignorance arise on account of objects experienced through eyes, ears, nose, tongue, bodysense and    mind. There is enslavement to objects which arise and then fall away immediately. When the foolishness of such infatuation is realized,     there are conditions to develop understanding of the realities of life.  One will understand that there are countless akusala cittas arising on account of the objects experienced through the senses, akusala cittas  which were not noticed before. When the characteristics of kusala and akusala are seen more clearly, there are conditions for the develop­   ment of the roots of non-attachment, non-aversion and wisdom. These      are the roots of kusala cittas which moti­vate the abstaining from unwholesome actions and the performing of wholesome deeds and speech.

As we have seen, there is a great variety of cittas. All cittas have in
common that they cognize an object, but cittas are different as they are
accompanied by different mental factors and experience different
objects. Seeing always experiences visible object and hearing always
experiences sound, but the reactions towards the objects and the
thoughts about them vary for different people. When someone else, for
example, speaks harsh words, there is the hearing of sound, and
afterwards there is thinking of the meaning of the words, thinking of
the person who speaks, thinking of conventional realities. Each person
lives in his own world of thinking. We react to what is experienced not
only with our thoughts, but also with action and speech. At the moments
we do not perform good deeds or we do not develop understanding, we
think, act and speak with akusala cittas. Citta determines our
behaviour, citta is called in the scriptures ``the leader of the
world''. We read in the Kindred Sayings (I, Sagāthā-vagga, Chapter I,
The Devas, Part 7, §2, Citta), the following verse:

\begin{verbatim}
Now what is that whereby the world is led?
And what is that whereby it is drawn along?
And what is that above all other things
That brings everything beneath its sway?
It is citta whereby the world is led,
And by citta it is drawn along,
And citta it is above all other things
That brings everything beneath its sway.
\end{verbatim}

In order to grasp the nature of our own life and the lives of others, it
is essential to understand what citta is. In order to have more
understanding of what citta is, the difference between conventional
truth and ultimate truth has to be known. Conventional truth is the
truth we were always familiar with before we studied the Buddhist
teachings; it is the conventional world of person, of ``self'', of
things which exist. Ultimate truth are mental phenomena and physical
phenomena. Cittas are mental phenomena, they experience something.
Bodily phenomena, such as the sense organs, and physical phenomena
outside do not experience anything. Citta can experience both mental
phenomena and physical phenomena. The physical phenomena and mental
phenomena of our life arise, exist just for an extremely short moment
and then vanish. Ultimate realities have each their own characteristic
which can be directly experienced when it appears, without the need to
think about it. By theoretical understanding we will not know what citta
is. Only if there can be the devel­op­ment of direct understanding of
the citta appearing at this moment, no matter it is seeing, hearing or
thinking, will we truly know what citta is. When the diversity of cittas
and their manifold conditions are seen more clearly the truth of
non-self will gradually be better understood. One will be motivated to
seek the elimination of delusion about the realities of one's life, of
the wrong view of self, of all forms of clinging, aversion and
ignorance.

\chapter{Deeds and their results}

There are many types of cittas, moments of consciousness. Cittas can be
kusala, wholesome, akusala, unwholesome, or neither kusala nor akusala.
The sense-cognitions such as seeing or hearing are neither kusala nor
akusala, but shortly after they have arisen and fallen away there are
cittas which react to the object experienced by the sense-cognitions,
and they react either in a wholesome way or in an unwholesome way. There
are more often akusala cittas which can be rooted in attachment,
aversion or ignorance, than kusala cittas which are rooted in
non-attachment and non-aversion, and which may be rooted in wisdom as
well. Akusala cittas can motivate evil deeds and kusala cittas can
motivate good deeds. We read in the Gradual Sayings (V, Book of the
Tens, Ch. 17, §8, Due to greed, hatred and delusion) that the Buddha
said to the monks:

\begin{quote}
Monks, the taking of life is threefold, I declare. It is motivated by
greed, hatred and delusion. Taking what is not given, sexual
misconduct,falsehood spiteful speech, harsh language, idle babble,
covetousness, ill-will and wrong view is threefold, I declare. It is
motivated by greed, hatred and delusion.

Thus, monks, greed is the originator of a chain of causal action, hatred
is the originator of a chain of causal action, delusion is the
originator of a chain of causal action. By destroying greed, hatred and
delusion comes the breaking up of the chain of causal action.
\end{quote}

We read about a ``chain of causal action''. The Pāli term kamma, also
known in its Sanskrit form karma, literally means action or deed. A good
deed brings a pleasant result and a bad deed brings an unpleasant
result. The results of our own deeds come to us sooner or later, this is
the law of kamma and result, and nobody can alter the operation of this
law. The Buddha's teaching on kamma and result is difficult to grasp. It
is not a dogma one has to accept. There can be theoretical understanding
of kamma and result, but by theoretical understanding this law cannot be
fully comprehended. Only by direct understanding of the physical
phenomena and mental phenomena of our life the condition of kamma which
produces result can be seen more clearly. Therefore, it should not be
expected that the law of kamma and result can be fully understood when
we begin to investigate the Buddha's teaching on this subject.

A deed done in the past can produce result later on. Kamma can be
compared to a seed developing into a tree which bears fruit later on.
Evenso, a bad deed, for example killing, can produce an unpleasant
result such as illness or pain. A good deed, for example a deed of
generosity, can produce a pleasant result, such as the receiving of
beautiful things. When we think of a deed and its result we usually
think of a deed which has an effect on someone else. In order to
understand the law of kamma and its result we should not think in terms
of the conventional realities of persons and situations, but we should
have understanding of the ultimate realities of cittas and their
accompanying mental factors and of physical phenomena, realities which
arise and then fall away immediately. We cannot be sure whether or not 
someone else performs kusala kamma from the outward appearance of things.
We may see someone else giving things away but there may not be the
performing of a deed of generosity. The giving may be motivated by
selfish motives, and then giving is not kusala kamma. It is the
wholesome or unwholesome intention or volition which constitutes kusala
kamma or akusala kamma. The terms kusala kamma and akusala kamma can be
used in the sense of good deeds and evil deeds, but when we are more
precise kamma is the intention or volition motivating deeds performed
through bodily action, through speech and through the mind. When we
speak of the different types of kusala kamma and akusala kamma we should
remember that kamma is intention or volition, a mental reality. Kamma or volition is a mental factor accompanying citta, and it arises and falls away together with the citta.

How can a deed performed in the past produce its result later on? Kamma,
or the volition which accompanies the citta when a good deed or a bad
deed is performed, falls away immediately together with the citta.
However, since each citta which falls away is succeeded by the next
citta, kamma can be accumulated from moment to moment. Its dynamic force
is carried on and when the time is ripe it can produce its result. That
is the chain of causal action we read about in the above quoted sutta.
The same sutta mentions the kinds of akusala kamma performed through
body, speech and mind. Not every akusala citta is of the intensity of
akusala kamma which can produce a result. When there is clinging to a
pleasant sight or sound there is akusala citta but not akusala kamma
which could produce a result. Clinging, however, has many degrees. It
can be more subtle or it can be strong, such as covetous­ness, the
desire for someone else's property. This has the intensity of akusala
kamma when one plans to take away what belongs to someone else. Kusala
kamma comprises abstaining from evil deeds as well as the performing of
good deeds, deeds of generosity and mental development, such as the
study of the Buddha's teachings and the development of understanding of
the realities of our life.

Moments of happiness and misery alternate in our life. The experiences
of pleasant objects and unpleasant objects through the senses do not
occur by chance, they must have a cause: kamma is the cause. We read in
the Gradual Sayings (IV, Book of the Eights, Chapter I, §5, Worldly
Failings) that the Buddha said to the monks:

\begin{quote}
Monks, these eight worldly conditions obsess the world; the world
revolves round these eight worldly conditions. What eight?

Gain and loss, fame and obscurity, blame and praise, bodily ease and
pain.

Monks, these eight worldly conditions obsess the world, the world
revolves round these eight worldly conditions.
\end{quote}

\begin{verbatim}
Gain, loss, obscurity and fame,
And censure, praise, bodily ease, pain-
These are man's states impermanent,
Of time and subject unto change.
And recognizing these the sage,
Alert, discerns these things of change;
Fair things his mind never agitate,
Nor foul his spirit vex. Gone are
Compliance and hostility,
Gone up in smoke and are no more.
The goal he knows. In measure full
He knows the stainless, griefless state.
Beyond becoming has he gone.
\end{verbatim}

The person who has reached the state of perfection has equanimity
towards the vicissitudes of life. He is freed from the chain of causal
action, there is no more rebirth for him. So long as one is full of
attachment, aversion and ignorance, one wants pleasant objects and
dislikes unpleas­ant objects. However, the experience of pleasant
objects and unpleasant objects is not in any one's power, it depends on
kamma which produces result. One day there is gain, the next day loss;
one day there is praise, the next day blame. Sometimes we are healthy,
sometimes we suffer from sickness and pain. The experience of pleasant
or unpleasant objects through the senses is not a reward or a
punishment. The idea of reward or punishment stems from the conception
of a supreme being, a God, who is the judge of man's deeds. The cause of
the experience of pleasant and unpleasant objects through the senses is
within ourselves: it is kamma. There is seeing and hearing of pleasant
and unpleasant objects time and again. Seeing and hearing are the
results of kusala kamma or akusala kamma. These results arise just for a
moment and then they fall away. When we define what was seen or heard or
think of the nature of the object, the moments of result have fallen
away already. It is hard to tell whether seeing or hearing is the result
of kusala kamma or of akusala kamma. Thinking of what was seen or heard
is not result; when there is thinking kusala citta or akusala citta 
arises, but mostly akusala citta arises. In order to understand the ultimate realities of kamma and its result we have to be very precise. Seeing, hearing, smelling, tasting and the experience of tangible object through the bodysense are cittas which are results of kamma. Our 
reactions in a wholesome or in an unwholesome way to the objects which   are experienced are not results of kamma, they are kusala citta or    akusala citta. Kusala citta and akusala citta can be called the active  side of life, since they can perform good deeds and bad deeds which     will cause the appro­priate results later on. The cittas which are  results of one's deeds can be called the passive side of life. We have    to receive results, whether we like it or not.

Cittas arise because of their own conditions, they are beyond control.
Sometimes it seems that we ourselves can cause the enjoyment of pleasant
objects. However, there have to be the right conditions for the
enjoyment of pleasant objects and enjoyment cannot last as long as we
wish. We can enjoy pleasant music by turning on the radio, but kamma is
the cause of hearing, not a self. It also depends on conditions whether
we can afford a radio or not. One may live in poverty and not be able to
afford a radio. It is due to kamma if one is born into a poor family and
has to live in uncomfortable circumstances. It is due to kamma if one is
born into a family which is well-to-do and if one can live in comfort.

In order to understand that birth into pleasant surround­ings and in
unpleasant surroundings is the result of kamma we have to go back to the
first moment of a lifespan. There was a citta at the first moment of our
life, and this is the rebirth-consciousness. This citta must have a
cause and the cause is in the past, it is kamma. Birth is result, we
could not select our parents, nor time and place of our birth. The first
moment of life is called rebirth-consciousness because there is not only
this present life, there were also past lives. It is difficult to
understand that kamma of the past produces the birth of a being. We can
notice, however, that people are born into different circum­stances,
with different bodily features and different mental capacities. This
does not happen by chance, there must be conditions for such
differences. There are different kam­mas which cause different kinds of
birth. In the ``Discourse on the Lesser Analysis of Deeds'' (Middle
Length Sayings III, number 135) we read that Subha asks the Buddha what
the cause is of the different results human beings experience from the
time of their birth:

\begin{quote}
``Now, good Gotama, what is the cause, what is the reason that lowness
and excellence are to be seen among human beings while they are in human
form? For, good Gotama, human beings of short lifespan are to be seen
and those of long lifespan; those of many and those of few illnesses;
those who are ugly, those who are beautiful; those who are of little
account, those of great account; those who are poor, those who are
wealthy; those who are of lowly families, those of high families; those
who are weak in wisdom, those who are full of wisdom.''

The Buddha answered Subha:

``Deeds are one's own, brahman youth, beings are heirs to deeds, deeds
are matrix, deeds are kin, deeds are arbiters. Deed divides beings, that
is to say by lowness and excellence.''
\end{quote}

Some people are born in countries where there is war and famine, others
in countries where there is peace and prosperity. This does not happen
by chance; kamma, a deed performed in the past, is the cause. If kamma
is the cause of birth, what is then the role of the parents? Parents are
also a condition for the birth of a child, but they are not the only
condition. Kamma produces at the first moment of life the citta which is
the rebirth-consciousness. The new human being which comes to life
consists of mental phenomena and bodily phenomena. The physical
phenomena which arise at the first moment of life must also have a
cause: kamma is the cause. Thus, at the first moment of life there is
mental result as well as physical result of kamma. Kamma is not the only
factor from which bodily phenomena originate. There are four factors in
all: kamma, citta, temperature and nutrition. After kamma has produced
bodily phenomena at the first moment of life, the other factors also
produce bodily phenomena. As to the factor of temperature, there has to
be the right temperature for the new being in the womb in order to
develop. When the mother takes food, nutrition is suffused in the body
and then nutrition is also producing bodily phenomena for the being in
the womb. Citta is a condition as well for bodily phenomena arising
throughout our life. If there were no citta we could not stay alive, we
could not move, we could not perform any activities. If we remember the
four factors which produce bodily phenomena, namely kamma, citta,
temperature and nutrition, it will help us to understand that the body
does not belong to a self. What we call ``my body'' consists of bodily
phenomena which arise because of different conditions and then fall
away.

Kamma produces bodily phenomena at the first moment of a lifespan and
also throughout life. It is kamma which produces the sense organs of
eyesense, earsense, smellingsense, tastingsense and bodysense. The sense
organs which are the physical results of kamma are the means for the
experiences which are the mental results of kamma: seeing, hearing and
the other sense-cognitions. Thus, kamma produces result at the first
moment of life, it produces the births of beings, and in the course of
life it also produces pleasant and unpleasant results in the form of
experiences through the senses.

Kamma can cause rebirth in unhappy and in happy planes of existence.
Besides the human plane of existence there are other planes of
existence. Birth in an unhappy plane is the result of akusala kamma and
birth in a happy plane is the result of kusala kamma. Hell planes and
the animal world, for example, are unhappy planes. The human plane and
heavenly planes are happy planes. It may be felt by some that the
existence of hell planes and heavenly planes is mythology. It should be
remembered that conventional terms are used to designate different
degrees of unpleasant results and pleasant results of kamma. Birth in a
hell plane is an unhappy rebirth because in such a plane there are
conditions for the experience of intense suffering. Birth in a heavenly
plane is a happy rebirth because in such a plane there are conditions
for the experience of pleasant objects. Life in a hell plane or in a
heavenly plane does not last forever. There will be rebirth again and it
depends on kamma in which plane rebirth-consciousness will arise. Birth
in the human plane is the result of kusala kamma, but in the course of
life there are conditions for the experience of both pleasant and
unpleasant objects through the senses, depending on the different kammas
which produce them.

It may happen that someone who has obtained wealth with dishonest means
lives in luxury. How can bad deeds have pleasant results? It is not
possible for us to find out which deed of the past produces its
corresponding result at present. A criminal can receive pleasant results
but these are caused by good deeds. His bad deeds will produce
unpleasant results but it is not known when. In the course of many lives
good deeds and bad deeds were performed and we do not know when it is
the right time for a particular kamma to produce result. A good deed or
a bad deed may not produce result during the life it was performed, but
it may produce result in the following life or even after countless
lives have passed. In the scriptures it is said that when kamma has
ripened its fruit is experienced. We read in the Dhammapada, (verses 119
and 120):

\begin{quote}
Even an evil-doer sees good so long as evil ripens not; but when it
bears fruit, then he sees the evil results.

Even a good person sees evil so long as good ripens not; but when it
bears fruit, then the good one sees the good results.
\end{quote}

Several other conditions are needed for akusala kamma or kusala kamma to
produce their appropriate results. The time when one is born or the
place where one is born can be a favourable or an unfavourable condition
for kusala kamma or for akusala kamma to produce result. For example,
when one lives in a time of war there are more conditions for akusala
kamma and less conditions for kusala kamma to produce result. A
particular kamma may be prevented from producing result when there is a
very powerful counteractive kamma which has preponderance. For example,
when someone is wealthy and lives in comfort, there are pleasant results
for him, caused by kusala kamma. However, he may suddenly lose his
wealth and be forced to live in miserable circumstances. His loss is
caused by akusala kamma which has ripened so that it can produce
unpleasant result. This is an example which shows that the way different
kinds of kamma operate in our life is most intricate.

Time and again there is result in the form of the experience of pleasant
and unpleasant objects through the senses and after such experiences 
kusala cittas or akusala cittas arise, but more often akusala cittas  arise. There is likely to be attachment to pleasant objects and
aversion towards unpleasant objects. Like and dislike alternate in our
life. Attachment and aversion are of many degrees, they do not always
have the intensity to motivate evil deeds. In that case there is no
accumulation of kamma, but defilements are accumulated. Attachment       and aversion arise and then fall away, but the conditions for these defilements are accumulated so that they can arise again. There
are different types of condition which operate in our life. Kamma is one
type of condition, it can produce result in the form of rebirth, or, in
the course of life, in the form of the experience of pleasant or
unpleasant objects through the senses. Defilement is another type of
condition, it is the condition for the arising again of defilements. On
account of pleasant and unpleasant results of kamma defilements may
arise which are so strong that they motivate the committing of evil
deeds. Thus, the result of kamma can condition defilements and
defilements can condition the committing of akusala kamma which will in
its turn produce result. This process is like an ever-turning wheel.

The Buddha's teaching on past lives, the present life and future lives,
on the cycle of birth and death, is difficult to grasp. We can have more
understanding of this teaching if we can see that, in the ultimate
sense, life lasts merely as long as one moment of citta which arises and
falls away. We are used to thinking in conventional terms of person,
situation, life and death. In the conventional sense life starts at the
moment of conception and it ends at the moment of death. In the ultimate
sense there is birth and death at each moment a citta arises and falls
away. The citta which has fallen away conditions the arising of the next
citta. There has to be a citta arising at each moment, there is no
moment without citta. Cittas arise in succession in the current of life.
When the end of a lifespan approaches, the last citta, the
dying-conscious­ness, falls away, but it is succeeded by the next citta.
That citta is the first citta of a new life, namely the
rebirth-consciousness. There can be theoretical under­standing of death
and rebirth, but all doubts can only be eliminated by the development of
direct understanding of the mental phenomena and physical phenomena
which arise and fall away. If there is direct understanding of the
conditions for the citta which arises at this moment, doubt about
rebirth can be eliminated. Just as the citta of this moment is succeeded
by the next citta, evenso the last citta of this life will be succeeded
by a following citta, the rebirth-consciousness.

It is dukkha to be in the cycle of birth and death. Why do we have to
receive an unpleasant result of a deed committed in a past life? In a
past life one was another being, different from what one is now. But why
should we receive the result of a deed committed in the past by another
being? A deed in the past which produces result now was committed by a
being from which we have originated. It is indeed sorrowful that
unpleasant results have to be received for evil deeds which may have
been committed many lives ago. This is the law of kamma and its result,
and it operates, whether we like it or not. A person in this life is
different from what he was in a past life, but all that was accumulated
in the past, kusala kamma and akusala kamma, defilements and good
qualities, all accumulations have been carried on from moment to moment
and they condition what is called the present personality. The Path of
Purification (XVII, 167) explains:

And with the stream of continuity there is neither identity nor
otherness. For if there were absolute identity in a stream of
continuity, there would be no forming of curd from milk. And yet if
there were absolute otherness, the curd would not be derived from the
milk, So neither absolute identity nor absolute otherness should be
assumed here.

The rebirth-consciousness has not been transferred from the past life to
this life, it is completely new. However, the conditions for its arising
stem from the past. The Path of Purification (166) illustrates this with
similes. An echo is not the same as the sound but it originates from the
sound. The impression of a seal stamped on wax is not the same as the
seal itself, but it originates from the seal. These similes clarify that
the present life is different from the past life, but that it is
conditioned by the past. There is no transmigration or reincarnation of
a self. The person who is reborn consists of five ``groups of
existence'', the ``khandhas'', namely physical phenomena and mental
phenomena which are arising and falling away. There is no permanent,
unchanging substance which passes from one moment to the next one, from
the last moment of life to the first moment of a new life. We read in
the scriptures about the former lives of the Buddha and his disciples.
The ``Birth Stories'' relate the former lives of the Buddha when he was
still a Bodhisatta and accumulated wisdom and all the other excellent
qualities, the ``Perfections'', which were the right conditions to
become a Buddha in his last life. There were accumulations of wisdom and
of the Perfections, but not a person, not a self who accumulated these.
There were only the khandhas arising and falling away. Since each citta
is succeeded by the next one within the current of countless lives,
accumulations are carried on from one life to the next life.

Can one speak of evolution in the succession of different lives, a
development from animal life to the human life and then to life in
heavenly planes? There is no specific order in the kinds of rebirths,
there is not necessarily develop­ment from life in lower planes to
higher planes. In reality rebirth depends on the kamma which produces
it. Kusala kamma may produce rebirth in a heavenly plane and after that
it may be the right time for akusala kamma to produce rebirth in a hell
plane. Only the person who has attained enlightenment has no more
conditions for an unhappy rebirth. When one has reached the state of
perfection all defilements have been eradicated and, thus, there are no
more conditions for any kind of rebirth. This means the end of dukkha.

The Buddha, in the night he attained enlightenment, had penetrated the
conditions for being in the cycle of birth and death and also the
conditions for being freed from this cycle. Kamma which produces rebirth
is part of a whole chain of conditions for the phenomena which
constitute the cycle of birth and death. It is a cycle ofinter­ dependently arising phenomena, forming a chain of twelve links, the
first of which is ignorance and the last one death. This is called
``Dependent Origination''. The ``Dependent Origination'' is an essential
part of the Buddha's teachings. Ignorance is mentioned as the first
cause of the interdependently arising phenomena of the cycle. So long as
ignorance has not been eradicated there are still conditions for the
performing of kamma which produces rebirth. At rebirth there is the
arising of mental phenomena and physical phenomena. Objects are 
experienced through the senses and the mind-door. On account of the  objects which are experienced different feelings arise and
feeling in its turn conditions craving. Due to craving there is clinging
which conditions the performing of kamma and this produces again
rebirth. So long as there is birth there is old age and death, and, thus,
there is no end to dukkha. This is the teaching on the ``Dependent
Origination'' which shows the conditions stemming from the past life for
phenomena in the present life, and conditions of the present life for
phenomena in the future.

Ignorance is mentioned as the first factor of the Dependent Origination,
but no first beginning of the cycle has been revealed. The Path of
Purification (XIX, 20) explains:

\begin{verbatim}
There is no doer of a deed
Or one who reaps the deed's result;
Phenomena alone flow on
No other view than this is right.
And so, while kamma and result
Thus causally maintain their round,
As seed and tree succeed in turn,
No first beginning can be shown.
\end{verbatim}

It is of no use to speculate about the beginning of the cycle. The
Buddha taught that when ignorance has been eradicated by wisdom, there
aren't any more conditions for the performing of kamma, and thus no
conditions for rebirth. Through wisdom there can be the end of the
cycle made up by the links of the Dependent Origination. This
means the end of dukkha. The commentary to the first book of the Abhi­dhamma, the Expositor (I, Part I, Chapter I, 44) explains by       way of a simile the conditions leading to the contin­uation of the    cycle and those leading to the end of it :

\begin{quote}
``leading to accumulation'' are those states which go about severally
arranging (births and deaths in) a round of destiny like a bricklayer
who arranges bricks, layer by layer, in a wall. ``Leading to
dispersion'' are those states which go about destroying that very round,
like a man who continually removes the bricks as they are laid by the
mason.
\end{quote}

When understanding has been developed to the degree that enlightenment
is attained there will be ``dispersion'', the removal of the conditions
for being in the cycle.

The teaching on the Dependent Origination explains why we are in this
life, why we have to suffer old age, sickness and death. It explains the
conditions for our life, for what we call our body and our mind. We may
know in general that mind and body are dependent on conditions, but
through the study of the Buddha's teachings we will know more in detail
what these conditions are and how they operate from birth to death. It
is kamma which produces bodily phenomena from the first moment of life
and also throughout life. Besides kamma, citta, temperature and
nutrition also produce bodily phenomena. Kamma produces throughout life
the sense-organs, the physical conditions for the pleasant and
unpleasant experiences which are the mental results of kamma. We are
heirs to kamma, it is unavoidable that there are loss, pain and other
adversities of life. There are many kinds of kamma which were performed
in the past, and what is done cannot be undone. When it is the right
time kamma produces its appropriate result. Ignorance of cause and
result in life conditions aversion and frustration on account of
unpleasant experiences and this means more suffering. Understanding of
the cause of suffering does not mean the immediate elimination of grief
and depression. However, understanding can help one to be less overcome
by despair about what is unavoidable, what is beyond control. More
understanding means less suffering. The Buddha did not only teach that
life is dukkha, he also taught the release from dukkha, namely the
development of the wisdom which can eradicate ignorance and all
defilements.

\chapter{Good deeds and a wholesome life}

\begin{verbatim}
Not to do evil, to cultivate good, to purify one's mind,
this is the teaching of the Buddhas.

  (Dhammapada, verse 183).
\end{verbatim}

All religions encourage people to abstain from evil, to perform good
deeds and to lead a wholesome life. In which way is Buddhism different
from other teachings? What is kusala, wholesome, is kusala, and what is
akusala, unwholesome, is akusala, no matter who performs it, no matter
which religion he professes. Buddhism, however, is different from other
teachings in so far as it explains the source of wholesomeness: the
different cittas which perform good deeds. The Buddha explained in
detail all the different cittas and their accompanying mental factors
and also the conditions for their arising. He helped people to know the
characteristic of kusala and of akusala. In that way the cittas which
arise in daily life can be investigated and the different degrees of
kusala and of akusala can be known by one 's own experience. When we
think of good deeds such as giving or helping, we usually have in mind
the outward situation, we think of persons who perform deeds. The
outward appearance of things, however, can be misleading. It depends on
the nature of the citta whether there is the performing of
whole­some­ness or not. We can only know ourselves the nature of our own
citta. It is essential to know when the citta is kusala citta and when
it is akusala citta.

The performing of what is wholesome comprises not only deeds of
generosity but also good moral conduct as well as mental development. It
is important to learn more details of the different ways of kusala which
can be performed. In the Buddhist teachings the ways of wholesomeness
can be classified as threefold, namely as generosity, good moral conduct
and mental development. Learning about these ways is in itself a
condition for the development of kusala in one's daily life.

The performing of deeds of generosity is included in the first of the
threefold classification of wholesomeness, but the giving away of things
may not always be kusala kamma. There may be moments of sincere
generosity, but they are likely to be alternated with akusala cittas. We
may expect something in return for our gift, and then there are akusala
cittas rooted in attachment. Or we may find that our gift was too
expensive and we may feel regret about it. Then there are akusala cittas
with stinginess which are rooted in aversion. The person who receives
our gift may not be grateful and therefore we may be annoyed or sad. We
are inclined to pay attention mostly to the effect of our deeds on
others. In order to develop what is wholesome there should be no
preoccupation with the reactions of others towards our good deeds.
Through the Buddhist teachings one learns to investigate the different
moments of citta which motivate one's deeds. Generosity arises with
kusala citta, it does not depend on gratefulness of other people. When
one is intent on the development of what is wholesome, there will be no
disturbance by other people's reactions. Is it not a selfish attitude to
be constantly occupied with one's own cittas? On the contrary, when one
comes to know when the citta is kusala citta and when akusala citta, one
will be able to develop more wholesomeness and this is beneficial for
oneself as well as for one's fellowmen. We have accumu­lated countless
defilements and, thus, the arising of kusala citta is very rare. When
there are conditions for generosity, there are at such moments no
stinginess, no clinging to one's possessions. The development of kusala
promotes a harmonious society.

Generosity is an inward reality, it arises with kusala citta. Even when
there is no opportunity for the giving of material things to others
there are other ways of generosity which can be developed. The
appreciation of someone else's good deeds which can be expressed by
words of approval and praise is a way of generosity. I learnt of this
way of kusala in Thailand where it is widely practised. People bow their
head with clasped hands and say, ``anumodana'', which is the Pāli term
for thanksgiving or satisfaction. In this way they express their
appreciation of someone else's kusala. At such a moment the citta is
pure, free from jealousy or stinginess. One may be stingy not only with
regard to possessions, but also with regard to words of praise.
Appreciation of someone else's kusala is one way of eliminating
stinginess. When one learns of this way of kusala there will be more
conditions for speaking about others in a wholesome way. We are inclined
to speak about other people's akusala, but when we have confidence in
the benefit of kusala we can change our habits. We can learn to speak in
a wholesome way.

Another way of generosity is giving other people the opportunity to
appreciate one's good deeds. Is this not a condition for pride? When one
tries to impress others there is akusala citta. However, when one has
the sincere inclination to help others to have kusala citta it is a way
of generosity which is called ``the extension of merit''. It depends on
the citta whether or not there is this way of kusala. Extension of merit
does not mean that other people can receive the results of kusala kamma
we performed. Each being receives the result of the kamma he performed
himself. Extension of merit means helping others to have kusala citta on
account of our kusala. In this way we can also help beings in other
planes of existence, provided they are in planes where they can notice
our good deeds and are able to appreciate them. In Buddhist countries it
is a good custom to express with words and gestures the dedication of
one's good deeds to the departed. When a meal or robes have been offered
to monks, one pours water over one's hands while the monks recite words
of blessing. In this way one expresses one's intention to dedicate one's
kusala to other beings.

There are several more aspects of generosity. Abstaining from killing,
lying and other evil deeds can be seen as an aspect of generosity. In
abstaining from evil deeds which harm other beings one gives them a
gift, one gives them the opportunity to live in peace. When we, for
example, abstain from killing insects we give the gift of life. Another
aspect of giving is forgiving the wrongdoings of someone else. When
someone else speaks insulting words to us we may have aversion and
conceit. When we think, ``Why is he doing that to me'', we think in
terms of ``he'' and ``me'', and then there is comparing, with conceit.
Ther can be conceit not only when we think of ourselves as higher than
someone else, but also when we think of ourselves as equal or less than
someone else. Conceit prevents us from forgiving. When we are stubborn
and proud, the citta is harsh, impliable. When we see the benefit of
kusala we can forgive. At that moment the citta is gentle, without hate
or conceit. One wishes the other person to be happy. For this way of
generosity one does not have to look for material things to be given, it
can be performed without delay. Knowing that forgiving is an act of
generosity can inspire us to forgive more readily.

Another aspect of generosity, included in the first of the threefold
classification of the ways of wholesomeness, is the explanation of
Dhamma. When one explains the Buddha's teaching to others, one helps
them to develop right understanding of the realities of life. This is
the way leading to the elimination of suffering and therefore, the gift
of Dhamma is the highest gift.

Not only generosity, but also good moral conduct is a way of
wholesomeness. This is the second of the threefold classification of
wholesomeness. There are many aspects to moral conduct or morality.
Abstaining from evil deeds as well as good actions performed through
body and speech are included in this way of kusala. We may believe that
we are leading a wholesome life so long as we do not harm anybody.
However, we should investigate whether the citta which arises is kusala
citta or akusala citta. Then we will discover that we are full of
defilements. The Buddha taught in detail on all unwholesome and
whole­some mental factors which accompany cittas in different
combinations. He explained all the different degrees of akusala and
kusala. It is necessary to know whether kusala citta or akusala citta
motivates our actions and speech, because the outward appearance of our
actions and speech is misleading. We may speak in a pleasant way, but we
may do so with selfish motives. We may flatter someone else in order to
obtain a favour or in order to be liked by him. Then there is not
wholesome speech, but speech motivated by attachment. We have to know
our attachment, aversion, jealousy and conceit, we have to know all our
defilements.

Abstaining from evil deeds is good moral conduct. There are three
unwholesome deeds performed through bodily action: killing, stealing and
sexual misconduct. There are four unwholesome verbal actions: lying,
slandering, rude speech and idle, useless talk. As regards killing, this
is the killing on purpose of any living being, insects included. Also
ordering someone else to kill is included in this type of akusala kamma.
Does this mean that Buddhists should be vegetarians? The Buddha did not
teach people to abstain from eating meat. The monks had to accept any
kind of food which was offered to them by the layfollowers. The Buddha
explained to the monks that they could eat meat unless they had seen,
heard or suspected that an animal was killed especially for them. We
read in the Book of Discipline (Vinaya IV, Mahā-vagga VI, on Medicines,
237) that the general Sīha attained enlightenment after having listened
to the Buddha. He offered a meal which included meat to the Buddha and
the order of monks. The Nigaṇṭhas, who were of another teaching, found
fault with the offering of meat. We read that after the meal the Buddha
explained to the monks:

\begin{quote}
``Monks, one should not knowingly make use of meat killed on purpose
(for one). Whoever should make use of it, there is an offence of
wrong-doing. I allow you, monks, fish and meat that are quite pure in
three respects: if they are not seen, heard, suspected (to have been
killed on purpose for a monk).''
\end{quote}

This answer may not be satisfactory to everyone. One may wonder whether
one indirectly promotes the slaughtering of animals by buying meat. It
would be good if there were no slaughtering at all, no violence. The
world, however, is not an Utopia. Animals are slaughtered and their meat
is sold. If one in the given situation buys meat and eats it, one does
not commit an act of violence. While one kills there is akusala citta
rooted in aversion; killing is an act of violence. While one eats meat
there may be attachment or dislike of it, but there is no act of
violence towards a living being.

The observing of precepts is included in good moral conduct. When one
undertakes the observance of precepts one makes the resolution to train
oneself in abstaining from akusala. There are precepts for monks,
novices and nuns, and there are precepts for layfollowers. At the
present time the order of nuns does not exist any more, although we can
in Buddhist countries still see women who have retired from worldly life
and try to live as a nun. The monks, as we will see, are under the
obligation to observe many rules. For laypeople there are five precepts,
but on special occasions they can undertake eight precepts. The precepts
are not worded in the form of commandments, forbidding people to commit
akusala. They are principles of training one can undertake with the aim
to have less akusala.

The five moral precepts layfollowers can observe are the foundation for
good moral conduct. When one undertakes them one makes the resolution to
train oneself in abstaining from the following unwholesome deeds:
killing living beings, stealing, sexual misconduct, lying and the taking
of intoxicants, including alcoholic drinks. When one is in circumstances
that one could commit an evil deed through body and speech but one
abstains from it with kusala citta, there is good moral conduct.

Abstaining from slandering, rude speech and idle talk are not among the
five precepts for laypeople. Abstaining from them, however, is kusala
kamma. We may be in situations where we are tempted to speak evil, but
when we abstain with kusala citta from slandering there is good moral
conduct. When someone else scolds us we may abstain from talking back.
However, if we keep silent with aversion there isn't good moral conduct.
We have to investigate the citta in order to know whether or not there    is good moral conduct. Abstaining from useless, idle talk is hard to
observe. Most of the time we engage in idle talk about pleasant objects,
such as delicious food, nice weather or journeys. We may think that such
conver­sations are good since we do not harm other people. Through the
Buddhist teachings we learn to investigate the cittas which motivate
such conversations. We can find out that we speak mostly with attachment
about pleasant objects and in this way accumulate ever more attachment.
We do not have to avoid such conversations, but it is beneficial to know
the nature of the citta which motivates our speech. When there are
conditions for kusala citta we can speak about pleasant subjects with
kindness and consideration for other people. Only the person who has
reached the state of perfection has no more conditions to engage in idle
talk. He has eradicated all kinds of akusala, also the subtle degrees.

Committing evil through bodily action or speech for the sake of one's
livelihood is wrong livelihood. When one abstains from wrong livelihood
there is right livelihood. One may, for example, be tempted to tell a
lie in order to obtain more profit in one's business. If one abstains
from such speech there is right livelihood.

It is hard to observe the precepts perfectly in all circum­stances. If
one has confidence in the benefit of kusala, one can gradually train
oneself in observing the precepts. One may generally not be inclined to
kill insects, but when one's house is full of fleas one may be tempted
to kill. Killing may sometimes seem a quick and easy way to solve one's
problems. One needs more effort to abstain from killing, but if one has
confidence in kusala one will look for other ways to solve one's
problems and abstain from killing. However, only those who have attained
enlighten­ment will never transgress the five precepts, not even for the
sake of their health or their life. The development of right
understanding of the realities of our life leads to the perfection of
moral conduct.

Paying respect to those who deserve respect is included in good moral
conduct. In Buddhist countries it is a tradition to pay respect to
parents, teachers, elderly people, monks and novices. At such moments
there is an opportunity to give expression to one's appreciation for
their good qualities, for their wisdom and guidance, and the assistance
they have given. We see layfollowers paying respect to monks by clasping
their hands and bowing their head, or by prostrating the body and
touching the floor with the forehead, the forearms and knees. When one
has not lived in a Buddhist country one may wonder why people are paying
respect to monks in such a humble way. The monks have retired from
worldly life in order to lead a life of detachment. Even if they are not
perfect, they can remind us of those who have reached perfection. In
Buddhist countries one can also see people prostrating themselves before
a Buddha statue. This is not idol wor­ship or a way of praying to the
Buddha. One cannot pray to the Buddha since he is not in a heavenly
plane or in any other place. He has passed away completely not to be
reborn again. One can remember the Buddha's virtues, his wisdom,
compassion and purity, and give expression to one's respect for his
virtues in gesture and speech. It depends on the individual's
inclination in which way he shows respect. One may show respect to
someone out of selfish motives, such as desire for favours, but in that
case the citta is akusala citta. When one pays respect or shows
politeness with a sincere inclination, it is kusala citta. At such a
moment there is no attachment or pride.

Another way of kusala kamma included in good moral conduct is helping
other people through speech and deeds. In order to know whether or not       there is this way of wholesome­ness we have to investigate the cittas
which motivate helping. One may help someone with selfish motives or
with reluctance, and that is not kusala kamma. Helping with unselfish
kindness is good moral conduct. At such a moment there is detachment. We
are inclined to be lazy and to be attached to our own comfort, but in
order to help someone we have to renounce our own comfort and make an
effort for kusala. When kusala citta arises we are able to think of
someone else's welfare. Also listening to other people when they talk
about their problems and giving them our attention is a way of helping
them.

In the Buddhist scriptures, including the Jātakas (the Buddha's Birth
Stories), there are many practical guidelines for a life of goodwill and
benevolence in one's social relations. There are guidelines for kings to
reign with justice and compassion, and these can be applied by all in
government service. We read in the ``Kūtadanta Sutta'' (Dialogues of the
Buddha I, sutta 5) that the Buddha told the Brahman Kūtadanta about a
King who wanted to offer a great sacrifice and asked his chaplain
advice. The chaplain advised the King about a sacrifice for the sake of
which no living being would be injured. He said to the King that,
instead of punishing the bandits who were marauding the country, the
King could improve the economic situation, a way which would be more
effective in suppressing crime. The King should give grain to farmers,
capital to traders, wages and food to those in government service. Then
tensions would be solved and there would be an end to disorder. The King
followed the chaplain's advice and made abundant gifts. The Buddha
explained to Kūtadanta that a sacrifice is not only the giving of
material things, but that it can also be dedication to spiritual
matters, namely having confidence in the Buddha, the Dhamma and the
``Sangha'', the community of enlightened disciples, as well as mental
development, including the development of the wisdom leading to
perfection.

The ``Sigālovāda sutta'' (Dialogues of the Buddha III, sutta 31)
contains the layman's social ethics. The Buddha explained to Sigāla that
there should be love and goodwill in the relations between parents and
children, teachers and pupils, husband and wife, employees and servants,
laypeople and those who have retired from worldly life. The Buddha
warned Sigāla of all the consequences of bad moral conduct and of the
danger of evil friendship. A bad friend appropriates a friend's
possessions, pays mere lip-service and flatters. Whereas a good friend
gives good counsel, sympathizes and does not forsake one in misfortune,
he is even willing to sacrifice his life for his friend. A good friend
is intent on the spiritual welfare of someone else. We read:

\begin{verbatim}
He restrains you from doing evil,
he encourages you to do good,
he informs you of what is unknown to you,
he points out to you the path to heaven.
\end{verbatim}

The Buddhist principles of goodwill and tolerance can be applied in
today's world in the community where one lives, on the national level
and on the international level, including development cooperation. One
can apply these principles more effectively if one at the same time
develops understanding of the different cittas which arise, kusala
cittas and akusala cittas. This understanding will prevent one from
taking for kusala what is akusala. It is necessary to get to know the
selfish motives with which we may act and speak, to get to know our many
defilements. Otherwise our deeds and speech will not be sincere.

For a layman it is difficult to observe good moral conduct in all
circumstances. He may find himself in situations where it is hard to
abstain from akusala kamma, such as killing. The person who has
inclinations to monkhood leaves the household life in order to be able
to observe good moral conduct more perfectly. There is good moral
conduct of the layman and there is good moral conduct of the monk. The
monk's moral conduct is of a higher level. He leads a life of
non-violence and contentment with little. He has renounced worldly life
in order to dedicate himself completely to the study and practice of the
Dhamma and to the teaching of it to layfollowers.

We read in the Book of Analysis (the Second Book of the Abhidhamma, 12,
Analysis of Absorption) about the life of a monk:

\begin{quote}
Herein a monk dwells restrained and controlled by the fundamental
precepts, endowed with (proper) behaviour and a (suitable) alms resort,
seeing danger in (his) slightest faults, observing (the precepts) he
trains himself in the precepts, guarded as to the doors of the sense
faculties, in food knowing the right amount, in the first watch of the
night and in the last watch of the night practising the practice of
vigilance, with intense effort and penetration practising the practice
of development of enlightenment states.
\end{quote}

The goal of monkhood is the eradication of all defile­ments through the
development of wisdom, the attainment of the state of perfection. The
monk is under the obligation to observe twohundred and twentyseven
training rules. Apart from these there are many other rules which help
him to reach his goal. They are contained in the Vinaya, the Book of
Discipline for the monk. We read that every time a monk did not live up
to the principles of monkhood, the Buddha laid down a rule in order to
help him. The Vinaya should not be separated from mental development, in
particular the development of understanding of the mental phenomena and
physical phenomena of life. Otherwise there would be the mere outward
observance of the rules, no purification of the mind. How could one
``see danger in the slightest faults'' if there is no right
understanding of the realities of his life, including the different
cittas which arise?

The task of the monk is the development of under­standing of the Dhamma
and explaining the Dhamma to others. His task is the preservation of the
Buddha's teachings. Social work is not the task of the monk, it is the
task of the layman. The Rules of Discipline, dating from the Buddha's
time about twothousand fivehundred years ago, are still valid today.
Shortly after the Buddha's passing away the first Great Council was held
at Rājagaha under the leadership of the Buddha's disciple Mahā-Kassapa.
We read in the Illustrator of Ultimate Meaning (commentary to the ``Good
Omen Discourse'' of the " Minor Readings", of the Khuddaka Nikāya) that
five hundred monks who had reached the state of perfection were to
recite all the texts of the Buddha's teachings. We read that when
Mahā-Kassapa asked which part they would rehearse first, the monks
answered:

\begin{quote}
``The Vinaya is the very life of the Teaching; so long as the Vinaya
endures, the Teaching endures, therefore let us rehearse the Vinaya
first.''
\end{quote}

The monk should train himself in fewness of wishes. He is allowed the
use of the four requisites of robes, almsfood, lodging and medicine, but
they are not his personal property, they belong to the Order of monks.
He is dependant on layfollowers for receiving these requisites and he
should be contented with whatever he receives. These requisites are the
monk's livelihood and he should train himself in purity of livelihood.

It is important also for laypeople to learn more about the monk's moral
conduct. The monk and the layman have different lifestyles, but the
layman can benefit from the study of the Vinaya and apply some of the
rules in his own situation in daily life. The rules can also help the
layman to ``see danger in the slightest faults'', to scrutinize his
cittas when he, as a layman, is in similar situations as the monk. Both
monks and laymen can train themselves in good moral conduct in action
and speech as well as in wholesome thoughts. We read, for example, in
the Vinaya (II, Suttavibha"ga, Expiation XVI) that monks took possession
of the best sleeping places, which were assigned to monks who were
elders. The Buddha thereupon laid down a rule, stating that such conduct
was an offence. Such an incident can also remind a layman not to ensure
the best place for himself in a room, in a bus or train. One may think
that one is entitled to the best place, but this is conceit. One can
find out that there is at such a moment no kindness and compassion, but
akusala citta. When we read in the Vinaya about the monk's daily life
and about the situations where he was tempted to neglect his purity of
moral conduct, we can be reminded of our own defilements, it can help us
to recognize our deeply rooted faults and vices.

The monk should remember that the four requisites of robes, food,
lodging and medicine are to be used so that he can stay healthy and
dedicate himself to his task of the development of understanding of
realities. The monk should not hint to lay-followers what kind of food
he would like, he should not indulge in clinging to the requisites by
hoarding food or robes. The monk should not try to obtain the requisites
with improper means, such as by pretending to be more advanced in mental
development than he really is. He may out of hypocrisy reject gifts, so
that layfollowers believe that he is a person with fewness of wishes and
then give to him more abundantly. The monk may try to impress others by
his deportment. We read in the Path of Purification (I, 70):

\begin{quote}
Here someone of evil wishes, a prey to wishes, eager to be admired,
(thinking) ``Thus people will admire me'', composes his way of walking,
composes his way of lying down; he walks studiedly, stands studiedly,
sits studiedly, lies down studiedly; he walks as though concentrated,
stands, sits, lies down as though concentrated; and he is one who
meditates in public.
\end{quote}

The Path of Purification explains that desire for requisites can
motivate speech with akusala citta. We read (I, 72) about different
kinds of unwholesome speech:

\begin{quote}
Ingratiating chatter is endearing chatter repeated again and again
without regard to whether it is in conformity with truth and Dhamma.
Flattery is speaking humbly, always maintaining an attitude of
inferiority. Bean-soupery is resemblance to bean soup; for just as when
beans are being cooked, only a few do not get cooked, the rest get
cooked, so too the person in whose speech only a little is true, the
rest being false, is called a ``bean soup''; his state is bean-soupery.
\end{quote}

Not only monks, also laypeople can be insincere in their deportment and
speech in order to obtain something desirable. We should check whether
our speech is ``bean-soupery''. We may to some extent speak what is true
and to some extent what is not true. We may believe that there is no
harm in ``bean-soupery'', but we accumulate at such a moment the
tendency to lying.

The monk should train himself in virtue concerning the requisites. He
should use them without greed and reflect wisely on their use. We read
in the Path of Purification (I, 85) about the use of almsfood:

Reflecting wisely, he uses alms food neither for amusement nor for
intoxication nor for smartening nor for embellishment, but only for the
endurance and continuance of this body, for the ending of discomfort,
and for assisting the life of purity.

Food is bound to be an object of attachment and it can also be an object
of aversion. If one reflects wisely on the use of food there is kusala
citta. It is natural that one enjoys delicious food, but if one
remembers that food is like a medicine for the body, one will be less
inclined to overeating, which is the cause of laziness. It is the monk
's duty to reflect wisely on the use of the requisites, but also for
laypeople there can be conditions for wise consider­ation of the things
they use in daily life.

The monk should not indulge in sleep, in the company of people and in
idle, useless talk. We read in the Gradual Sayings (V, Book of the Tens,
Chapter VII, §9, Topics of talk) that while the Buddha was staying near
Sāvatthī at Jeta Grove, some of the monks were indulging in idle talk,
namely talk on kings, robbers, ministers, food, relatives, villages and
other useless topics. The Buddha asked them what they were talking about
and then said that such idle, useless talk was improper for them. He
pointed out that there were ten topics of talk monks should engage in:

\begin{quote}
Talk about wanting little, about contentment, seclusion, solitude,
energetic striving, virtue, concentration, insight, release, release by
knowing and seeing.
\end{quote}

It is beneficial also for laypeople to find out which types of citta
motivate talking. Even though one cannot change one's habits yet, it is
beneficial to know the different types of cittas which motivate one 's
actions and speech.

The monk should train himself in purity in all his actions and speech.
There are four kinds of purification of the monk's moral conduct:
restraint with regard to the disciplinary rules, the guarding of the
sense doors, virtue concerning his livelihood, virtue concerning his
requisites. With regard to the ``guarding of the sense faculties'', we
read in the Middle Length Sayings (I, Sutta 27, The Lesser Discourse on
the Simile of the Elephant's Footprint):

\begin{quote}
Having seen material shape with the eye, he is not entranced by the 
general appearance, he is not entranced by the detail. If he dwells 
with this organ of sight uncontrolled, covetousness and dejection, evil states of mind, might predominate. So he fares along controlling it; he guards the organ of sight, he comes to control over the organ of 
sight...
(the same for the other doorways).

\end{quote}

When there is understanding of visible object, sound and the other
realities as they are, as impermanent and not self, one will be less
infatuated by them. In this sense we have to understand the word 
"control". It is by understanding, by wisdom, that there will be the guarding of the sense organs.

As we read in the Dhammapada (verse 183), it is the teaching of the
Buddha to abstain from evil, to develop what is wholesome and to purify
one's mind. Through mental development there can be purification of the
mind, the elimination of what is impure, unwholesome. Mental development
is the third of the threefold classification of the ways of
wholesomeness. For mental development right understanding or wisdom is
necessary, whereas the first way of wholesomeness, generosity, and the
second way of wholesomeness, good moral conduct, can be performed also
without right understanding. One may perform deeds of generosity,
abstain from evil or help others because it is one's nature to do so,
even without understanding of one 's cittas. Right under­standing of     the different cittas which arise can lead to the development of more wholesomeness. Through the development of understanding of one's cittas  one will discover all one's weak points, even the slightest faults, and this means that there is less delusion about oneself. It is beneficial 
to discover that whenever kusala is not performed, our actions, speech and thoughts are akusala.

We read in the Gradual Sayings (Book of the Twos, Chapter 2, §7) about
deeds of commission and omission. We read that the brāhmin Jānussoṇi
asked the Buddha why some beings were reborn in Hell. The Buddha
explained that it was owing to deeds of commission and omission. The
Buddha said:

\begin{quote}
``Now in this connection, brāhmin, a certain one has committed bodily
immoral acts, and omitted bodily moral acts and the same as regards
speech and thought. Thus, brāhmin, it is owing to commission and
omission that beings are reborn in Hell.''
\end{quote}

We read that the Buddha then explained that through the commission of
kusala kamma and the omission of akusala kamma beings were reborn in
Heaven. This text is a reminder not to neglect wholesome deeds. Omission  of kusala will condition regret and worry later on. When kusala is performed with the aim to have less defilements, there will be more motivation to abstain from akusala, to develop kusala and to purify the mind. When more understanding of cittas and their accompanying mental factors is developed, confidence in the benefit of kusala will grow.    When the truth of non-self has been realized, one will clearly see that kusala citta arises because of its own conditions, that there is no   person or self who performs it. Then kusala will be purer, and moral conduct will become enduring. The under­standing of the truth of       non-self however, is the result of a gradual development, it cannot be realized within a short time. The development of understanding of the mental phenomena and physical phenomena of life is a way of kusala 
kamma which is included in mental development.

Selfishness, envy, stinginess, anger, conceit and other defilements
disturb our social life. Such defilements motivate us to engage in wrong
action and wrong speech. In this way we harm both ourselves and others.
If we develop more loving kindness, compassion, tolerance and gentleness, as taught by the Buddha, it will be to the benefit of both ourselves and others. At the moment of the performance of wholesomeness, the citta is pure, without defilements; there is no attachment, no aversion or hate,   no ignorance.
We read in the Dhammapada (verses 3-5):

\begin{quote}
``He abused me, he beat me, he defeated me, he robbed me'', the hatred
of those who harbour such thoughts is not appeased.

``He abused me, he beat me, he defeated me, he robbed me'', the hatred
of those who do not harbour such thoughts is appeased.

Hatreds never cease by hatred in this world; by love alone they cease.
This is an ancient principle.
\end{quote}

We cannot live up to such high principles unless there is the
development of understanding which will eventually lead to the
eradication of defilements. We can, however, begin to apply ourselves to
the ways of whole­some­ness we are able to perform at this moment. All
kinds of wholesomeness which are included in the threefold
classification of generosity, good moral conduct and mental development
are to the welfare of ourselves as well as our fellow beings.

\chapter{Mental development}

Mental development is the third of the threefold clas­sification of
wholesomeness. Mental development includes tranquil meditation as well
as the development of insight wisdom. The first way of wholesomeness,
generosity, and the second way of whole­some­ness, good moral conduct,
can be performed also without right understanding of the cittas which arise. For mental develop­ment, however, under­standing is necessary. 
The under­standing which arises in mental development is of different degrees, as we will see.

The study of the Buddha's teachings, the Dhamma, and the explaining of
it to others are included in mental development. When one listens to the
explanation of the Dhamma and reads the scriptures, one learns what is
kusala and what is akusala, one learns about kamma and its result and
about the ways to develop wholesome­ness. One learns that realities 
are impermanent, dukkha, and non-self, anattā. In order to develop under­standing of the Dhamma, one should not only listen, one
should also carefully consider what one hears and test its meaning.
Explaining the Dhamma to others is included in mental development. Both
speaker and listener can bene­fit, because they can be reminded of the
need to verify the truth of the Dhamma in their own life. Understanding
acquired through the study of the Dhamma is the founda­tion for tranquil
meditation and insight meditation which are also included in mental
development.

Tranquil meditation\footnote{In Pāli: samatha.} and insight
meditation\footnote{In Pāli: vipassanā} have each a different aim and    a different way of development. For both kinds of meditation right
understanding of the aim and the way of development is indispensable. In
tranquil meditation one develops calm by concentrating on a meditation
subject in order to be temporarily free from sense impressions and the
attachment which is bound up with them. Insight meditation is the
development of direct understanding of all realities occurring in daily
life. The goal of insight meditation is the eradication of wrong view
and all other defilements. Insight or insight wisdom is not theoretical
understanding of mental phenomena and physical phenomena, it is
understanding which directly experiences the characteristics of
realities.

There are many misunderstandings with regard to the word ``meditation''.
Some people want to meditate without understanding what meditation is,
what its object and its aim are. Meditation is seen as escapism, a way
to be free from the problems of daily life. One believes that when one
sits in a quiet place and concentrates on one object one can become
relaxed and free from worry. Relaxation is desirable, but it is not the
aim of mental development.

As regards tranquil meditation, this is the development of calm. It is
essential to have right understanding of what calm is. True calm has to
be wholesome, it is freedom from defilements. As I explained before,
akusala citta can be rooted in three unwholesome roots: attachment,
aversion and ignorance. These roots have many degrees, they can be
coarse or more subtle. Kusala citta is rooted in the wholesome roots of
non-attachment and non-aversion and it may be rooted as well in wisdom
or understanding. In order to have right understanding of one's cittas
one should investigate them in daily life. Otherwise one may take for
kusala what is akusala. It is generally believed that there is calm     when one has no aversion, no annoyance. One should know, however, that
when there is no aversion there is not necessarily kusala citta. There
may be a subtle attachment to silence and then there is akusala citta,
no calm. Calm is among the wholesome mental factors arising with each
kusala citta. For example, at the moments of generosity and good moral conduct, there is also calm. At such moments the cite is without
attachment, aversion and ignorance, it is pure. The moments of
kusala citta, however, are very rare, and soon after they have fallen
away akusala cittas arise. Since the moments of calm arising with the
kusala cittas are so few, the characteristic of calm may not be
noticeable. The aim of tranquil meditation is to develop calm with a
meditation subject. Only when there is right understanding of the
difference between the moments of akusala citta and of kusala citta can
calm be developed.

Even before the Buddha's time there were wise people who saw the
disadvantages of sense objects and the clinging to them. They were
able to develop calm to a high degree, even to the degree of
absorption\footnote{In Pāli: jhāna.}. Absorption is not what is
generally understood as a trance. At the moments of absorption only the
meditation subject is experienced and sense-cognitions such as seeing
or hearing do not occur. The citta with absorption is of a plane of
consciousness which is higher than the sensuous plane of consciousness,
that is, the cittas of our daily life which experience sense objects.     At the moments of absorption calm is of a high degree, one is not
infatuated with sense objects and defilements are tempo­rarily subdued.
There are different stages of absorption and at each subsequent stage
there is a higher degree of calm. However, by absorption defilements
cannot be eradicated. After the moments of absorption have fallen away,
seeing, hearing and the other sense-cognitions arise again, and on
account of them defilements occur. Even if one has not accumulated
the inclination and skill for the development of a high degree of calm,
it is still useful to have some general knowledge about its development.
This will help one to eliminate misunderstandings about tranquil
meditation and insight meditation. It will help one to see the
difference between these two kinds of meditation.

For tranquil meditation it is essential to have a keen understanding of
the characteristic of calm and of the way to develop calm with a
suitable meditation subject. The Path of Purity (Chapters IV-XII)
describes forty meditation subjects which can condition calm. Among the
meditation subjects are disks (kasinas), recollection of the excellent
qualities of the Buddha, the Dhamma and the enlightened disciples,
meditations on corpses, mindfulness of death, loving kindness,
mindfulness of breathing. A meditation subject does not necessarily
bring about calm. Only when there is right understanding of calm and the
way to develop it, calm can grow. Moreover, it depends on a person's
inclinations which meditation subject is suitable for him as a means to
develop calm. It is generally believed that calm is developed by means
of concentration. It should be known, however, that there can be right
concentration and wrong concentration. Concentration is a mental factor
which accompanies each citta. As I explained before, there is one citta
arising at a time, but each citta is accompanied by several mental
factors which each perform their own function while they assist the
citta in experiencing an object. Each citta can experience only one
object and it is concentration or one-pointedness which has the function
of focusing on that object. Thus, concentration can be kusala, akusala
or neither kusala nor akusala. When concentration accompanies akusala
citta it is wrong concentration. If one tries very hard to concentrate
there may be attachment to one's practice, or there may be aversion
because of tiredness, and at such moments there is no calm. If there is
right understanding of calm and of the way to develop it, there is also
right concentration without there being the need to try to concentrate.
It is right understanding which has to be emphasized, not concentration.

Mindfulness of breathing is generally believed to be an easy subject of
meditation, but this is a misunderstanding; it is one of the most
intricate subjects. If one tries to concentrate on breath without right
understanding of this subject there will be clinging instead of calm.
Breath is a bodily phenomenon which is conditioned by citta. It can
appear as hardness, softness, heat or pressure. Those who want to
develop this subject and have accumulated conditions to develop it have
to be mindful of breath where it touches the tip of the nose or the
upper-lip. However, breath is very subtle, it is most difficult to be
mindful of it. The Path of Purification (VIII, 208) states:

\begin{quote}
For while other meditation subjects become clearer at each higher stage,
this one does not: in fact, as he goes on developing it, it becomes more
subtle for him at each higher stage, and it even comes to the point at
which it is no longer manifest.
\end{quote}

We read further on (VIII, 211):

\begin{quote}
But this mindfulness of breathing is difficult, difficult to develop, a
field in which only the minds of Buddhas, ``Silent Buddhas'' \footnote{A
  Silent Buddha is an Enlightened One who has found the Truth all by
  himself, but who has not accumulated excellent qualities to the extent
  that he can teach the Truth to others.}, and Buddhas' sons are at
home. It is no trivial matter, nor can it be cultivated by trivial
persons. In proportion as continued attention is given to it, it becomes
more peaceful and more subtle. So strong mindfulness and understanding
are necessary here.
\end{quote}

``Buddhas' sons'' are the Buddha's disciples who had accumulated great
wisdom and who were endowed with excellent qualities. Thus, this subject
is not suitable for everybody.

We cling to breath since our life depends on it. Breathing stops when
our life comes to an end. When this subject is developed in the right
way, it has to be known when one clings to breath or to calm; it
has to be known when there is akusala citta and when kusala citta.
Otherwise it is impossible to develop calm with this subject. It is
difficult to know the characteristic of breath, one may easily take for
breath what is not breath. Following the movement of the abdomen is not
mindfulness of breathing. Some people do breathing exercises for the
sake of relaxation. While one concentrates on one's breathing, one
cannot think of one's worries at the same time and then one feels more
relaxed. This is not mindfulness of breathing, which has as its aim the
temporary release from clinging. Mindfulness of breathing is extremely
difficult and if one develops it in the wrong way, there is wrong
concentration, there is no development of wholesomeness. For the
development of this subject one has to lead a secluded life and many
conditions have to be fulfilled.

Does one have to lead a secluded life for the development of all
meditation subjects? Calm is of different degrees and if one has
accumulated the inclination and capacity to cultivate a high degree of
calm, even the degree of absorption, a secluded life is one of the
conditions which are favourable for the attainment of it. However, only
very few people can reach absorption, as the Path of Purification
states. Even if one has no inclination to develop a high degree of calm
there can be conditions for moments of calm in daily life. Some of the
meditation subjects, such as the development of loving kindness, can be
a condition for calm in daily life. It is felt by some that for the
development of this subject one has to be alone and one has to
concentrate on thoughts of loving kindness. The development of
loving kindness is not a matter of concentration but of right
understanding. Loving kindness can and should be developed when one is
in the company of other people. It has to be clearly understood when
there is unselfish kindness and when there is selfish affection. Moments
of loving kindness are likely to be followed by moments of attachment.
Right understanding of one's different cittas is indispensable for the
development of this subject, as it is for the development of all
subjects of meditation. The Path of Purification (IX, 2) explains that
in order to develop loving kindness one should consider the danger of
ill will and the advantage of patience. It states that one cannot
abandon unseen danger and attain unknown advantages. Thus we see again
that right understanding is emphasized. We may dislike someone and we
may be impatient about his behaviour. When we see the disadvantages of
unwholesome thoughts there may be conditions for thoughts of kindness
instead. That person may not treat us in a friendly way, but we can
still consider him as a friend. True friendship does not depend on other
people's behaviour, true friendship depends on the kusala citta. When we
feel lonely, because we miss the company of friends, we should
investigate our own citta. Is there loving kindness with the citta? This
point of view can change our outlook on our relationship with our
fellow­men, and as a consequence our attitude can become less selfish.
Loving kindness can be extended to anybody, also to people whom we do
not know, whom we pass on the street. We tend to be partial, we want to
be kind only to people we like, but that is a selfish attitude. When
there is true loving kindness there is impartiality as well. We tend to
think of others mostly with akusala citta, with cittas rooted in
attachment or aversion. When we learn, however, what loving kindness is,
there can be conditions for whole­some thoughts instead, and then there
is calm. Calm can naturally arise when there are the right conditions.
When one tries very hard to have thoughts of loving kindness in order to
induce calm there is attachment instead of true calm which has to be
wholesome. Thus, this is not the way to develop the meditation subject
of loving kindness.

Not all meditation subjects are suitable for everybody. There are
meditation subjects on corpses in different stages of decay, but for
some people such a subject can condition aversion instead of calm.
Recollection on Death is a meditation subject which can condition calm
in daily life. We are confronted with death time and again, and instead
of sadness we can reflect with kusala citta on the impermanence of life.
We can be reminded that even at this moment our body is subject to
decay, constituted as it is by physical phenomena, elements, which arise
and then fall away. In the ultimate sense death is not different from
what occurs at this moment.

Is it necessary to develop calm before one develops insight? Some people
believe that when the mind is calm first, it will be easier to develop
insight afterwards. It should be remembered that tranquil meditation and
insight meditation have each a different aim and a different way of
development. Tranquil meditation has as its aim to be free from seeing,
hearing and the other sense impressions, in order to subdue clinging to
sense objects. Insight meditation is the development of direct
under­standing of all realities of daily life: of seeing, hearing and
the other sense-cognitions, of sense objects and also of the
defilements arising on account of them. In this way the wrong view of
self and all other defilements can be completely eradicated. Tranquil
meditation should not be considered a necessary preparation for the
development of insight. The Buddha did not set any rules with regard to
tranquil meditation as a requirement for the development of insight.
Individual inclinations are different. It depends on one's accumulated
inclinations whether one applies oneself to tranquil meditation or not.
People in the Buddha's time who had accumulated great skill developed
calm even to the degree of absorption. In order to attain enlightenment,
however, they still had to develop insight, direct understanding of
realities, stage by stage. They had to have right understanding also of
the citta which attained absorption so as not to take absorption for
``self''. There were many people in the Buddha's time who attained
enlightenment without having developed a high degree of calm first.

The aim of tranquil meditation is the subduing of defilements, but, even
when absorption is attained, they cannot be eradicated. When there are
conditions, akusala cittas arise again. In the development of insight
any reality which appears, no matter whether it is pleasant or
unpleasant, kusala or akusala, is the object of under­standing.
Defilements should be understood as they are: as realities which arise
because of their own conditions and which are not self. So long as
defilements are still considered as ``self'' or ``mine'' they cannot be
eradicated. The development of insight does not exclude calm, there are
also conditions for calm in the development of insight. When defilements
are eradicated stage by stage there will be more calm. When defilements
are completely eradicated there is no more disturbance by akusala and
this is the highest degree of calm.

The development of insight which is included in mental development, is
the development of direct under­standing of realities, of the mental
phenomena and physical phenomena of our life. The development of calm
could be undertaken also by people before the Buddha's time. Absorption
was the highest degree of kusala which could be attained before the
Buddha's enlightenment. The develop­ment of insight however, can only    be taught by a Buddha. He taught the truth of impermanence, dukkha, and    non-self, anattā. What is called a person or an ego is only a temporary combination of mental phenomena and physical phenomena\footnote{Mental phenomena and physical phenomena are called in Pāli: nāma and rūpa.} which arise and then fall away immediately. Through the development of insight there can be the direct experience of the truth and the 
eradication of defilements at the attainment of enlightenment.

When, however, understanding of realities is only theore­tical, the
truth of impermanence, dukkha and anattā is not grasped; there is still
clinging to concepts and ideas of persons, the ego, the world. As I
explained in chapter 3, there are two kinds of truths: the conventional
truth and the ultimate truth. Conventional truth is the world of people
and of the things around us, the world of houses, trees and cars, thus
the things we have always been familiar with. When we study the Buddha's
teachings we learn about ultimate truth. Ultimate realities are mental
phenomena, cittas and their accompanying mental factors, and physical
phenomena. Nibbāna is an ultimate reality but this can only be
experienced when enlighten­ment is attained. Seeing is an ultimate
reality, a citta which experiences visible object through the eyesense.
Seeing can only arise when there are eyesense and visible object, it
arises because of its own conditions. The same is true for hearing and
the other sense-cognitions. Only one citta arises at a time and it
experiences one object. After seeing, hearing and the other sense-
cognitions kusala cittas and akusala cittas arise. Kusala cittas
with generosity may arise, or akusala cittas with attachment, aversion
or stinginess. All these realities arise because of their own
conditions. There is no self who can control these realities or cause
their arising. They arise just for a moment and then they fall away
immediately. Because of ignorance we do not grasp the true nature of
realities, their nature of impermanence, dukkha and non-self. Ignorance
covers up the truth. Insight, the direct under­standing of realities, is
developed in order to eliminate ignorance and wrong view. Direct
understanding of realities is different from theoretical understanding,
but theoretical understanding is the foundation for direct
understanding.

The object of insight, of direct under­standing, is ultimate truth, not
conventional truth. Conventional truth are concepts which are objects of
thinking. For example, after seeing we think of the shape and form of a
person or thing. That is not seeing, but thinking of concepts. A concept
is not real in the ultimate sense. Ultimate realities have each their
own characteristic which is unchangeable, and which can be directly
experienced when it appears. Seeing is an ultimate reality, it has its
own characteristic. It is real for everybody. Its name may be changed,
but its characteristic cannot be changed. Anger is an ultimate reality,
it has its own characteristic which can be experienced by everybody. In
order to be able to develop direct understanding of ultimate realities
it is essential to know the difference between ultimate realities and
concepts. One does not have to avoid thinking of concepts, because 
thinking itself is an ultimate reality which arises because of its own
conditions and which has its own characteristic. Thus, thinking can be
the object of understanding when it appears. Every reality which arises
because of conditions can be the object of direct under­standing. Since
concepts are not real in the ultimate sense and do not have
characteristics which can be directly experienced, they are not objects
of direct understanding.

How can direct understanding be developed? There has to be mindfulness
or awareness\footnote{In Pāli: sati} of the reality appearing at the
present moment in order that direct understanding of it can be
developed. There are many levels of mind­fulness. It is a wholesome
mental factor which accom­panies each kusala citta. It is heedful or
non-forgetful of what is wholesome. At the moments of mindfulness the
opportunity for wholesomeness is not wasted by negli­gence or laziness.
Mindfulness prevents one from committ­ing unwholesome deeds, it is like
a ``guard''. Mindfulness arises with generosity, with good moral
conduct, and with the development of calm. In the development of calm
there is mindfulness of the meditation subject, so that calm can
develop. When insight, the direct under­standing of realities, is
developed, mindfulness is non-forgetful, aware, of the reality      appearing at the present moment: a mental phenomenon or a physical phenomenon. At the very moment of mindfulness direct understanding 
of that reality can gradually develop. Thus, when there is mindfulness  with the development of insight, the opportunity for the investigation    of what appears at the present moment is not wasted.

Mindfulness and understanding are different realities, they are mental
factors which each have a different function while they arise with
kusala citta in the development of insight. Mindfulness is non-forgetful
of the reality appearing at the present moment through one of the six
doorways, but it does not have the function of understanding that
reality. Understanding investigates the true nature of the reality which
appears, but in the beginning it cannot be clear understanding. It is
merely learning and studying the characteristic of the phe­nomenon
appearing at the present moment. It develops very gradually, there are
many degrees of understanding. The moment of mindfulness is so short, it
falls away immediately. In the beginning mindfulness and under­standing
are still weak and thus one cannot be sure what their characteristics
are.

Mindfulness in the development of insight is aware of one object at a
time, either a mental phenomenon or a physical phenomenon. It is aware
of an ultimate reality, not of a concept such as a person or a thing.
The whole day there is touching of different things, such as a chair, a
plate, a cup, a cushion. Usually there is thinking of concepts, one
defines the things one touches, one knows what they are used for. When
one has learnt about ultimate realities and there are conditions for
mindfulness, however, it can be aware of one reality, such as hardness
or softness appearing through the bodysense. At that very moment there
can be a beginning of right understanding of that reality: it can be
seen as only a physical reality, an element, arising because of
conditions. One may touch a precious piece of chinaware, but it should
be remembered that through touch the reality of hardness, not the
china­ware, can be experienced. Hardness is tangible object, it is an
ultimate reality which has its own characteristic. When there is
mindfulness of that reality attachment does not arise. When mindfulness has
fallen away, there may be moments of thinking of that piece of
chinaware, there may be thinking with attachment. Attachment to pleasant
things is real, it does not have to be shunned as object of mindfulness.
In order to develop understanding of ultimate realities it is essential
to know when the object which is experienced is a concept and when an
ultimate reality.

Mental phenomena and physical phenomena appear time and again. Through
the bodysense hardness, softness, heat and cold can be experienced. They
have their own characteristics and when mindfulness arises it can be
directly aware of them. It can be verified by one's own experience that
hardness is only a physical element, no matter whether it is in the body
or in the things outside. Direct understanding of ultimate realities
will gradually lead to detachment from the idea of ``my body'' and ``my
mind'', from the idea of self. Through earsense sound is experienced.
One usually pays attention to the origin or the quality of sound, one
pays attention to the voice of someone or to music. At such moments
there is thinking of concepts. When there are conditions for the arising
of mindfulness, it can be aware of the character­istic of sound, a
physical phenomenon which can be heard. It appears merely for a moment
and then it falls away. Sound does not belong to anyone, it is merely an
element, non-self. Is it helpful to know this? Knowing that even the
sound of music one enjoys is only a physical element seems very prosaic.
One can enjoy the pleasant things of life, but in between there can be a
moment of developing under­standing of ultimate realities. Sound is
real, hearing is real, enjoyment of music is real, they are all
realities which can be known as they are: impermanent and non-self.

Different objects can be experienced through one door­way at a time.
Hearing experiences sound through the ears. Seeing experiences visible
object or colour through the eyes. Hearing cannot see, seeing cannot
hear, there is only one citta at a time. There is no self who sees or
hears, the seeing sees, the hearing hears. Through the develop­ment of
insight one can verify that there is no self who coordinates seeing,
hearing and all the other experiences. In the ultimate sense life is one
moment of experiencing an object. When we are thinking of a person or a
thing, we have an image of a ``whole'', and then the object at that
moment is a concept. At the moment of mindfulness, however, only one
reality at a time, appear­ing through one of the six doorways, is the
object.

A mental phenomenon knows or experiences something, whereas a physical
phenomenon does not experience anything. It is essential to learn the
difference between these two kinds of phenomena. We tend to consider
body and mind as a ``whole'', as a person or self. When there is
mindful­ness of one reality at a time, we learn that there are only
different mental elements and physical elements arising and falling
away. When sound appears there is also hearing, the experience of sound.
Sound and hearing have different characteristics. Sound does not
experience any­thing, whereas hearing experiences an object, the object
of sound. When visible object appears there is also seeing, the
experience of visible object. Visible object does not experience
anything, whereas seeing experiences an object, visible object. 
Mindfulness is aware of only one object, either a physical reality 
or a mental reality. Each citta experiences only one object, and
thus when mindfulness accompanies the citta, it can be aware of only one
object at a time. It is very difficult to distinguish sound from hearing
and visible object from seeing. Only when insight, direct understanding
of realities, has been developed, physical realities and mental
realities can be distinguished from each other. So long as there is
confusion about the difference between what is mental and what is
physical, there is still an image or a concept of a ``whole''. When
there is no precise under­standing of one reality at a time, its arising
and falling away, its impermanence, cannot be directly understood.

There is no self who can choose the object of awareness or who can
direct mindfulness to such or such object. Mindfulness is non-self, it
arises because of its own condi­tions. It is unpredictable of what
object mindfulness will be aware: either a mental reality or a physical
reality. The characteristic of mindfulness cannot be understood by
theoretical knowledge, by describing its nature. Only when mindfulness
arises can one know what it is. It arises when there are the right
conditions. The right conditions are: listening to the Dhamma as it is
explained by someone with right understanding, and studying and
considering Dhamma. Theoretical understanding of ultimate realities and
remembrance of what one has learnt are a necessary foundation for the
development of direct understanding. If one has expectations about the
arising of mindfulness, if one tries to concentrate on realities, or
tries to observe them, there is clinging to an idea of self who can
direct mindfulness, and this is counteracting the arising of
mindfulness.

In the beginning one is bound to take for mindfulness what is not
mindfulness but thinking. When one thinks, ``This is attachment'', there
is no direct awareness of the characteristic of the reality which
appears. There can still be a concept of ``my attachment''. Then
attachment is not understood as a conditioned reality which is non-self.
When one reality appears through one of the six doors, there can be a
moment of investigation or study of its characteristic, and that is the
beginning of understanding of its true nature, its nature of non-self.
At such a moment there is mindfulness, mindfulness of the reality
appearing at the present moment. Even one extremely short moment of
mindfulness and investigation of an ultimate reality is beneficial,
because in that way mindfulness and under­standing can be accumulated.
Then there are conditions for their arising again later on and in that
way direct understanding can grow.

Direct understanding of realities can develop only very gradually. 
There are different stages of insight, and in order that these stages    can arise, understanding has to become very keen. at the first stage      of insight the difference between the reality which is mental and the reality which is physical is clearly distinguished. As I explained,     this is difficult, since one tends to confuse realities such as seeing   and visible object or hearing and sound. The arising and falling away,   the impermanence of realities, can be penetrated only at a later stage    of insight.

All the realities of one's daily life, also defilements, can be the
objects of direct understanding. Defilements can eventually be
eradicated when they are understood as they are: as non-self. If one
tries to change one's life in order to create conditions for insight, or
if one tries to suppress defilements in order to have more awareness,
one is led by clinging and this is not the right way. One should come to
understand all realities, the mental and physical phenomena which
naturally arise in daily life.

The development of insight, of direct understanding of realities, is
very intricate and there cannot be an imme­diate result when one begins
to develop it. Is it worth while to begin with its development, even
though it takes more than one life to reach the goal? It is beneficial
to begin with its development. Theoretical understanding does not
eliminate delusion when there is seeing, hearing and the other
experiences through the senses and the mind-door. On account of the
objects which are experienced there is bound to be attachment, aversion
and ignorance. If there is at least a beginning of the development of
direct understanding we will be able to verify that our life is one
moment of experiencing an object through one of the six doors. When
seeing appears, its characteristic can be invest­igated. It can be
understood that it is only a mental reality arising because of its own
conditions, not a person or self. It sees what appears through
eyesense. When visible object appears it can be understood that it is
only a physical reality, appearing through eyesense, not a person or
thing. All realities appearing through the six doors can be understood
as they are, as non-self. Through direct understanding of realities
wrong view about them can be eliminated.

We read in the Kindred Sayings (IV, Kindred Sayings on Sense, Third
Fifty, Chapter 5, §152, Is there a Method?), that the Buddha spoke to
the monks about the method to realize through direct experience the end
of dukkha:

\begin{quote}
``Herein, monks, a monk, seeing visible object with the eye, either
recognizes within him the existence of lust, malice and illusion, thus:
I have lust, malice and illusion,' or recognizes the non-existence of
these qualities within him, thus: ÔI have not lust, malice and
illusion.' Now as to that recognition of their existence or
non-existence within him, are these conditions, I ask, to be understood
by belief, or inclination, or hearsay, or argument as to method, or
reflection on reasons, or delight in speculation?''

``Surely not, lord.''

``Are not these states to be understood by seeing them with the eye of
wisdom?''

``Surely, lord.''

``Then, monks, this is the method by following which, apart from belief
a monk could affirm insight thus: `Ended is birth, lived is the
righteous life, done is the task, for life in these conditions there is
no hereafter.'\,''
\end{quote}

We then read that the same is said with regard to the experiences
through the doorways of the ears, nose, tongue, bodysense and mind. The
development of under­standing of all that is real, also of one's
defilements, is the way leading to the eradication of defilements, to
the end of rebirth. This is the end of dukkha.

\chapter{The eightfold Path}

The development of the eightfold Path leads to the goal of the Buddha' s
teachings: the end of dukkha, suffering. As I explained in chapter two,
the Buddha taught the four noble Truths: the Truth of dukkha, of the
cause of dukkha, which is craving, of the ceasing of dukkha, which is
nib­bāna, and of the Path leading to the ceasing of dukkha. We read in
the Kindred Sayings (V, The Great Chapter, Book XII, Kindred Sayings
about the Truths, chapter II, §1) that the Buddha, while he was dwelling
at Isipatana, in the Deer­park, spoke to the company of five monks:

\begin{quote}
Monks, these two extremes should not be followed by one who has gone
forth as a wanderer. What two?

Devotion to the pleasures of sense, a low practice of villagers, a
practice unworthy, unprofitable, the way of the world (on the one hand);
and (on the other) devotion to self-mortification, which is painful,
unworthy and unprofitable.

By avoiding these two extremes the Tathāgata has gained knowledge of
that middle path which gives vision, which gives knowledge, which causes
calm, special knowledge, enlightenment, Nibbāna.

And what, monks, is that middle path which gives vision Nibbāna?

Verily, it is this noble eightfold way, namely: Right view\footnote{
  Right understanding of realities.}, right thinking, right speech,
right action, right livelihood, right effort, right mindfulness, right
concentration.
\end{quote}

When there is direct awareness and right understanding of any reality
which appears in daily life, there is at that moment no devotion to
sense pleasures nor self-mortification. One is on the middle way, and
that is the eightfold Path. When direct understanding of realities has
been developed stage by stage, the wrong view of self can be eradicated.
Then it is clearly understood that what is taken for a person or self
are in reality merely mental phenomena and physical phenomena which
arise and then fall away immediately. When the realities which arise
because of their own conditions have been understood as they are, as
impermanent, dukkha and non-self, enlightenment, can be attained. At
that moment nibbāna is experienced. Nibbāna is the unconditioned
reality, the reality which does not arise and fall away. There are four
stages of enlightenment and at these stages defilements are
progressively eradicated. First the wrong view of self has to be
eradicated because the other defilements cannot be eradi­cated so long
as they are taken for ``self''. All defilements are eradicated when the
fourth and last stage of enlighten­ment has been attained, the stage of
the perfected one, the ``arahat''. He has eradicated ignorance and all
forms of clinging completely, there are no more latent tendencies of
defilements left. Ignorance and clinging are conditions for rebirth
again and again, for being in the cycle of birth and death. When
ignorance and clinging have been eradicated there will be the end of the
cycle of birth and death, the end of dukkha.

The development of the eightfold Path leads to the cessation of dukkha.
In order to know what the eightfold Path is, the eight Path factors as
enumerated in the above-quoted sutta have to be examined more closely.
They are mental factors\footnote{In Pāli: cetasika.} which can accompany
citta. As I explain­ed before, there is one citta arising at a time, but
it is accompanied by several mental factors which each perform their own
function while they assist the citta in cognizing an object. Mental
factors can be akusala, kusala or neither kusala nor akusala, in
accordance with the citta they accompany. When the eightfold Path is
being developed, the mental factors which are the Path factors perform
their own specific functions in order that the goal can be attained: the
eradication of defilements. From the begin­ning it should be remembered
that there is no self, no person, who develops the Path, but that it is
citta and the accompanying mental factors which develop the Path. As we
read in the sutta, the eight Path factors are the following:

\begin{itemize}
\item
  right understanding
\item
  right thinking
\item
  right speech
\item
  right action
\item
  right livelihood
\item
  right effort
\item
  right mindfulness
\item
  right concentration
\end{itemize}

Right understanding is the first and foremost factor of the eightfold
Path. What is right understanding of the eightfold Path? There are many
levels and degrees of under­standing. There can be theoretical
understanding of the Buddha's teachings on mental phenomena and physical
phenomena. However, the Path factor right understanding is not
theo­retical understanding of realities. When it is developed it is the
direct understanding of the true nature of physical phenomena and mental
phenomena appearing in daily life. When right understanding begins to
develop, however, it cannot yet be clear, direct understanding
immediately. The mental phenomena and physical phenomena which appear in
daily life have to be investigated over and over again. As I explained
in chapter seven, not concepts but ultimate realities are the objects of
direct understanding. The characteristics of seeing, visible object,
hearing, sound, attachment or generosity can be investigated by right
understanding of the eightfold Path. In that way they can be seen as
only conditioned realities which are non-self. When there are conditions
for the arising of right understanding it arises and investigates the
reality which presents itself at that moment through one of the six
doors. Right understanding arises and then falls away immediately
together with the citta it accompa­nies, but it is accumulated and
therefore, there are conditions for its arising again. In this way
understanding can develop; it develops, there is no person who develops
it. Right under­standing can penetrate the characteristics of
imper­manence, dukkha and non-self and it can eventually realize the
four noble Truths.

Right thinking is another Path factor. Right thinking is not the same as
what we mean by thinking in conventional sense. When we use the word
thinking in conventional language we mean thinking of a concept, an
event or a story. In the ultimate sense thinking is a mental factor
which accompanies many types of citta, although not every type. Thinking
``touches'' or ``hits'' the object which citta experiences and in this
way assists the citta in cognizing that object. The mental factor
thinking expe­riences the same object as the citta it accompanies. The
object can be a concept and also an ultimate reality, a mental
phenomenon or a physical phenomenon. Thinking arises merely for an
extremely short moment together with the citta and then it falls away
with the citta. Thinking can be akusala, kusala or neither akusala nor
kusala. When it is akusala it is wrong thinking and when it is kusala it
is right thinking. The Path factor right thinking arises together with
the Path factor right understanding. The object of the Path factor right
thinking is an ultimate reality, the reality which appears at the
present moment. In the beginning there will be doubt whether the reality
appearing at the present moment is a physical reality, such as visible
object, or a mental reality, such as seeing. When right understanding is
only beginning to develop there is not yet precise understanding of the
difference between the characteristics of these realities. The function
of the Path factor right thinking is " touching" the reality appearing
at the present moment so that right under­standing can investigate its
characteristic. In that way precise understanding of that object can be
developed, until there is no more confusion between the characteristic
of a mental reality and a physical reality. Right thinking assists right
understanding to penetrate the true nature of realities: the nature of
impermanence, dukkha and non-self. Thus we see that the Path factor of
right thinking is essential for the development of understanding.

There are three Path factors which are the factors of good moral
conduct\footnote{In Pāli: sīla}: right speech, right action and right
livelihood. They have the function of abstaining from wrong speech,
wrong action and wrong livelihood. Wrong livelihood is wrong speech and
wrong action committed for the sake of one's livelihood. When there are
conditions for abstaining from these kinds of akusala kamma the factors
of good moral conduct perform the function of abstention. They arise one
at a time. When there is abstention from wrong action such as killing,
there cannot be at the same time abstention from wrong speech. The
development of right understanding will condition good moral conduct,
but only after enlightenment has been attained good moral conduct
becomes enduring. The person who has attained the first stage of
enlightenment has no more conditions for the committing of akusala kamma
which can produce an unhappy rebirth. Thus, right understanding of
realities bears directly on one's moral conduct in daily life. As we
have seen, the three mental factors which are the abstentions from evil
moral conduct arise one at a time, depending on the given situation. At
the moment of enlightenment, however, all three Path factors which are
good moral conduct arise together. The reason is that they perform at
that moment the function of eradicating the cause of misconduct as to
speech, action and livelihood. Latent tendencies of defilements are
eradicated so that they do not arise anymore. As I explained before,
defile­ments are eradicated at different stages of enlightenment and it
is only at the fourth and final stage that all akusala is eradicated.

Right effort is another Path factor. Effort or energy is a mental factor
which can arise with kusala citta, akusala citta and citta which is
neither kusala nor akusala. Its function is to support and strengthen
the citta. When effort or energy is kusala it is the condition for
courage and perseverance in the performing of kusala. Energy is right
effort of the eightfold Path when it accompanies right understanding of
the eightfold Path. It is the condition for perseverance with the
investigation and study of the reality appearing at the present moment,
be it a mental phenomenon or a physical phenomenon. Energy and patience
are indispensable for the development of right understanding. There must
be awareness of mental phenomena and physical phenomena over and over
again, in the course of many lives, so that right understanding can see
realities as they are, as impermanent, dukkha and non-self. When we hear
the word effort we may have a concept of self who exerts an effort to
develop right understanding. Effort is non-self, it is a mental factor
which assists right understanding. When there is mindful­ness of a
reality and understanding investigates its nature, right effort arises    as well and performs its own function. It does not arise because of    one's wish, it arises because of its own conditions.

Right mindfulness is another factor of the eightfold Path. As I
explained in chapter seven, there are many levels of mindfulness. There
is mindfulness with each kusala citta and its function is to be heedful,
non-forgetful of what is wholesome. Mindfulness is a factor of the
eightfold Path when it accompanies right understanding of the eightfold
Path. There may be theoretical understanding of realities acquired
through reading and thinking. One may think in the right way of the
phenomena of life which are imper­manent and non-self. However, in order
to directly expe­rience the truth there must be mindfulness of the
reality which appears at the present moment. The moment of mindfulness
is extremely short, it falls away immedi­ately. However, during that
moment understanding can invest­igate the characteristic of the reality
which appears, and in this way understanding can develop very gradually.
Right understanding arises together with right mindful­ness, but they
have each a different function. Right mindfulness is heedful or    attentive to the reality which appears but it does not investigate
its nature. It is the function of right understanding to investigate and
penetrate the true nature of the reality which appears at the present
moment.

Right concentration is another factor of the eightfold Path.
Concentration or one-pointedness is a mental factor arising with each
citta. Citta experiences only one object at a time and concentration has
the function to focus on that one object which citta experiences.
Concentration can accompany kusala citta, akusala citta and citta which
is neither kusala nor akusala. Right concentration is concentration
which is wholesome. Right concentration is of many kinds and degrees.     As we have seen in chapter seven, there is right concentration in 
tranquil medit­ation. When calm is developed to the extent that absorp­tion is attained, there is a high degree of concentra­tion    which focuses on the meditation subject. Sense-cognitions such as 
seeing or hearing do not arise and defilements are temporarily subdued. There is also right concentration in the development of direct understanding of realities. The Path factor right concen­tration accom­panies right understanding of the eightfold Path.
It focuses rightly on the reality which appears at the present moment
and which is the object of right understanding. It is not necessary to  make a special effort to concentrate on mental phenomena and physical
phenomena. If one tries to concen­trate one clings to an idea of
``my concentra­tion'', and then right under­standing is not being developed. When there are conditions for the arising of right
mindfulness and right understanding, there is right concentration
already which focuses on the reality presenting itself at that moment.

Some people believe that one should first develop morality, after that
concentration and then right under­standing. However, all kinds of
kusala, be it generosity, good moral conduct or calm can develop
together with right understanding. There is no particular order in the
development of wholesomeness. Kusala citta is non-self, anattā. When one
reads the scriptures one will come across texts on the development of
right concentration which has reached the stage of absorption. This does
not imply that all people should develop calm to the degree of
absorption. As I explained in chapter seven, it depends on the   individual' s accumulations whether he can develop it or not. The Buddha encouraged those who could develop calm to the degree of absorption to be mindful of realities in order to see also absorption as non self. There  are many aspects to each subject which is explained in the teachings and one has to take these into account when one reads the scriptures. 
Otherwise one will read the texts with wrong understanding. The Buddha's teachings are subtle and deep, not easy to grasp. We read in the Kindred Sayings (V, The Great Chapter, Kindred Sayings about the Truths, chapter II, § 9, Illustration) that the Buddha said to the monks:

Monks, the noble Truth of This is dukkha. This is the arising of
dukkha.This is the ceasing of dukkha. This is the practice that leads to
the ceasing of dukkha, has been pointed out by me. Therein are
numberless shades and variations of meaning. Numberless are the ways of
illustrating this noble truth of, This is the practice that leads to the
ceasing of dukkha. Wherefore, monks an effort must be made to realize:
This is dukkha, This is the arising of dukkha, This is the ceasing of
dukkha, This is the practice leading to the ceasing of dukkha.

The eightfold Path must be developed in daily life. One should come to
know all realities, also one's defilements, as they arise because of
their own conditions. One cannot change the reality which arises, it is
non-self. Misunder­standings as to the development of right
understanding are bound to arise if one has not correctly understood
what the objects of mindfulness and right understanding are. Therefore I
wish to stress a few points concerning these objects. Some people
believe that a quiet place is more favourable to the development of
right under­standing. They should examine themselves in order to find
out which types of citta motivate their thinking. If laypeople want to
live as a monk in order to have more opportunity to develop right
understanding, they are led by desire. It is due to conditions, to one
's accumulated inclinations, whether one is a monk or a layman. Both
monk and layman can develop understanding, each in his own situation.
Then one will come to understand one's own accumulated inclinations. The
development of the eightfold Path is the development of right
understanding of all realities which arise because of their own
conditions, also of one's attachment, aversion and other defilements. In
order to remind people of the realities which can be objects of
mindfulness and right understanding, the Buddha taught the ``Four
Applications of Mindfulness''. These Four Applications contain all
mental phenomena and physical phenomena of daily life which can be
objects of mindfulness and right understanding. They are: Contem­plation
of the Body, which comprises all physical phenomena, Contemplation of
Feeling, Contemplation of Citta and Contemplation of Dhammas, which
comprises all realities not included in the other three Applications of
Mindfulness. Contemplation in this context does not mean thinking of
realities. It is direct awareness associated with right understanding.
We read in the ``Satipaṭṭhāna Sutta'' (Middle Length Sayings I, 10) that
the Buddha, while he was dwelling among the Kuru people at
Kammāssadamma, said to the monks:

This is the only way, monks, for the purification of beings, for the
overcoming of sorrow and lamentation, for the destruction of suffering
and grief, for reaching the right path, for the attainment of nibbāna,
namely, the Four Applications of Mindfulness.

The teaching on the factors of the eightfold Path as well as the
teaching on the Four Applications of Mindfulness pertain to the
development of right understanding of realities in daily life, but they
each show different aspects. The teaching on the Path factors shows us
that for the development of right understanding there are, apart from
the factor right understanding, other Path factors which are the
conditions for right understanding to perform its function in order that
the goal can be reached, the eradi­cation of defilements. In explaining
the Four Applications of Mindfulness the Buddha encouraged people to be
mindful in any situation of their daily life. In the Contemplation of
Citta, citta with attach­ment is mentioned first and this can remind us
not to shun akusala as object of mindfulness. The Buddha explained that
there can be mindfulness of realities no matter whether one is walking,
standing, sitting or lying down, no matter what one is doing. Those who
develop tranquil meditation and attain calm, even to the degree of
absorption, can be mindful of realities. The Buddha showed that there
isn't any reality which cannot be object of mindfulness and right
understanding. When one develops right under­standing of any reality
which appears there is no need to think of the Four Applications of
Mindfulness or of the Path factors.

The Buddha's teaching on the development of right understanding of
realities is deep, it is not easy to grasp what mindfulness and right
understanding are. For this reason I would like to give a further
explanation of objects of mindfulness which present themselves in daily
life. Some people believe that mindfulness is being conscious of what
one is doing. If one is conscious of what one is doing, such as reading
or walking, there is thinking of concepts, no awareness of realities.
There is clinging to an idea of self who reads or walks. No matter what
one is doing there are mental phenomena and physical phenomena and
understanding of them can be developed. When one, for example, is
watching T.V., one may think of a story which is being enacted.
However, there is not only thinking, Also seeing, visible object,   hearing, sound, feeling or remembrance arise. Most of the time there
is forgetfulness of realities, one thinks of concepts. However,
in between the moments of thinking mindfulness can arise of one
reality appearing through one of the six doors, an ultimate reality
which is either a mental reality or a physical reality. The
characteristics of different realities can gradually be understood. When  we read a book we think of the meaning of the letters and of the story.  But there are also moments of seeing merely what is visible, of what appears through eyesense. It is because of remem­brance that we know    the meaning of what we read. Remembrance is a mental factor arising together with the citta, it is not self. Also remembrance can be  understood as it is. It seems that thinking occurs at the same time as seeing, but they are different cittas with different chara­cter­istics. The characteristics of different realities can be investigated, no matter
whether we are seeing, reading, hearing or paying attention to the
meaning of words. When this has been understood we will see that objects
of awareness are never lacking in our daily life. There are six
doorways, there are objects experienced through these six doorways and
there are the realities which experience these objects. That is our
life.

Right understanding develops very gradually, it has to be developed
during countless lives before it can become full understanding of all
realities which appear. There are several stages of insight in the
course of the development of right understanding. Even the first stage
of insight, which is merely a beginning stage of insight, is difficult
to reach. At this stage the different characteristics of the mental
reality and of the physical reality which appear are clearly
distinguished from each other. At each higher stage of insight
understanding becomes keener. The objects of understanding are the same:
the mental phenomena and physical phenomena which appear, but
understanding of them becomes clearer. When conditioned realities have
been clearly understood as impermanent, dukkha and non-self, nibbāna,   the uncon­di­tioned reality, can be experienced. The citta
which experiences nib­bāna is a ``supramundane'' citta\footnote
{In Pāli: lokuttara citta.}, it is of a plane of citta which is higher than the plane of cittas which experience sense objects or the plane of cittas with absorp­tion. All eight Path factors accompany the supra­mundane citta at the moment of enlightenment. Defile­ments are subse­quently eradicated at four stages of enlightenment. The 
supramundane cittas which expe­rience nibbāna arise and then fall away imme­diately. When the fourth and final stage of enlightenment has not been attained, defilements arise again. However, the person who has attained even the first stage of enlightenment has no more conditions to commit akusala kamma to the degree that it can produce rebirth in an unhappy plane of existence.

When one reads the words ``enlightenment'' and ``supramundane'', one may
imagine that enlightenment is something mysterious, that it cannot occur
in daily life. However, it is the function of right understanding to
penetrate the true nature of realities in daily life, and when it has
been developed to the degree that enlighten­ment can be attained, the
supramundane citta which experiences nibbāna can arise in daily life.
Enlightenment can be attained even shortly after akusala citta has
arisen, if right understanding has penetrated its true nature. We read
in the Psalms of the Sisters (Therīgāthā, Canto I, 1) that a woman
attained enlighten­ment in the kitchen. When she noticed that the curry
was burnt in the oven she realized the characteristic of impermanence
inherent in conditioned realities and then attained enlightenment.
Events in daily life can remind us of the true nature of realities. If
understanding could not develop in daily life it would not be true
understanding. We read that people in the Buddha's time could attain
enlightenment even while they were hearing the Buddha preach or just
after his sermon. One may wonder how they could attain enlighten­ment so
quickly. They had accumulated the right condi­tions for enlightenment
during innumerable lives and when time was ripe the supramundane cittas
could arise.

The development of understanding from the beginning phase to full
understanding is an infinitely long process. That is the reason why many
different conditioning factors are needed to reach the goal. The study
of the teachings, pondering over them, understanding of the way of
develop­ment of the eightfold Path are conditions for mindfulness and
direct understanding of realities. However, apart from these conditions
there are others which are essential. Ignorance, clinging and the other
defilements are deeply rooted and hard to eradicate. Therefore, in order
to reach the goal, the eradication of defilements, all kinds of kusala
have to be accumulated together with the development of right
understanding. The Buddha developed during in­numer­able lives, even
when he was an animal, all kinds of excellent qualities, the
``Perfections''. These were the neces­sary conditions for the attainment
of Buddhahood. Also his disciples had developed the Perfections life
after life before they could attain enlightenment. Since the
accumulation of the Perfections is essential in order to be able to
develop the eightfold Path I would like to explain what these
Perfections are.

The ten Perfections are the following:

\begin{itemize}
\item
  generosity
\item
  good moral conduct
\item
  renunciation
\item
  wisdom
\item
  energy
\item
  patience
\item
  truthfulness
\item
  determination
\item
  loving kindness
\item
  equanimity
\end{itemize}

The Buddha, when he was still a Bodhisatta, developed all these
Perfections to the highest degree. For those who see as their goal the
eradication of defilements, all these Perfections are necessary
conditions for the attainment of this goal, none of them should be
neglected.

First of all I wish to give an illustration of the develop­ment of the
Perfections by the Bodhisatta during the life when he was the ascetic
Akitti. We read in the Basket of Conduct (Cariyā-piṭaka, I, Conduct of
Akitti) that the Buddha spoke about the Perfection of liberality he
accum­ulated in that life. Sakka, King of the Devas (divine beings) of
the ``Threefold Heaven'' came to him in the disguise of a brahman,
asking for almsfood. Akitti had for his meal only leaves without oil or
salt, but he gave them away whole­heartedly and went without food. We
read:

\begin{quote}
And a second and third time he came up to me. Unmoved, without clinging,
I gave as before.

By reason of this there was no discolouration of my physical frame. With
zest and happiness, with delight I spent that day.

If for only a month or for two months I were to find a worthy recipient,
unmoved, unflinching, I would give the supreme gift.

While I was giving him the gift I did not aspire for fame or gain.
Aspiring for omniscience I did those deeds (of merit).
\end{quote}

Akitti performed this generous deed in order to accu­mulate conditions
for the attainment of Buddhahood in the future. The commentary to this
text, the Paramattha­dīpanī, states that Akitti accumulated all ten
Perfections while he was giving his gift.

The Perfection of generosity is developed in order to have less clinging
to possessions. We cling to material things because we want the ``self''
to be happy. If we do not learn to give away material things, how could
we ever get rid of clinging to the concept of self? Akitti also
accumulated good moral conduct, wholesomeness by action and speech. He
was helping the brahman in giving him food. He accumulated renunciation.
Renunciation is not merely renunciation of the household life. All kinds
of whole­someness are forms of renunciation, of detachment. One
renounces clinging to oneself, to one 's own comfort. Akitti renounced
his own comfort when he went without food for three days. Akitti
accumulated the Perfection of wisdom. The Perfection of wisdom is not
only right understanding of the eightfold Path, it comprises all levels
of wisdom. The Bodhisatta, even when he was an animal, accumulated the
Perfection of wisdom. He knew akusala as akusala, kusala as kusala, he
knew the right conditions for the attainment of Buddhahood. When he gave
food to the brahman, there was energy or courage, which is indispensable
for each kind of kusala. Energy strengthens the kusala citta so that
good deeds can be performed. He had patience, he was glad to endure
hunger for three days, and had there be an opportunity for a longer
period of time to give away his food, he would have fasted longer, even
for one or two months. He also accumulated the Perfection of
truth­fulness. Truthfulness has several aspects. It is not only
truthfulness in speech but also sincerity in action and thought. Kusala
must be known as kusala and akusala as akusala. One should not delude
oneself with regard to one's faults and vices, even when they are more
subtle. It should be known that when one is giving a gift with selfish
motives, there is no sincere inclination to kusala, that one may take
akusala for kusala. Akitti had a sincere inclin­ation to give and did
not expect any benefit for himself. The Perfection of truthfulness is
indispensable for the development of right understanding. One has to be
sincere with regard to what one has understood and what one has not
understood yet, otherwise there cannot be any pro­gress. The Bodhisatta
accumulated determination, he had an unshakable determination to
persevere with the devel­op­ment of understanding and the other
Perfections until he would reach the goal. He had loving kindness, he
thought of the brahman's welfare, not of himself. There was equanimity,
he had no aversion even though he went without food for three days. He
had equanimity towards the vicissitudes of life. Right understanding of
kamma and the result of kamma conditions equanimity.

This text illustrates that the Perfections can be accumulated when a
good deed is being performed. As we have seen, there are ten kinds of
good qualities which have been classified as the Perfections. Good
qualities are not always Perfections. They are Perfections only when the
aim of the performing of kusala is the diminishing of defile­ments and
eventually their eradication at enlightenment. The accumu­lation of the
ten Perfections together with the development of right understanding of
realities is the application of the Buddha's teachings in daily life.
There may not be mindfulness of ultimate realities very often, but there
are many opportunities to accumulate other kinds of kusala, the
Perfections. It is encouraging to know that all kinds and levels of
kusala can be Perfections, helpful conditions to reach the goal. Even
when we help other people in giving them good advice or in consoling
them when they are in distress can be an opportunity for the
accumulation of the Perfections, conditions to dimin­ish selfishness and
other defilements.

The development of the eightfold Path which leads to enlightenment seems
to be far-fetched for an ordinary person. It is such a long way before
the goal can be reached. There will be more confidence in the Buddha's
teachings when one sees that what he taught can be verified and applied
in one's own life. The development of understanding can only be very
gradual, just as learning a new skill such as a foreign language has to
be very gradual. The Perfection of patience is indispensable for the
development of right understanding of realities. Learning about the
Path-factors which each perform their own function helps us to see that
no self, no person develops right understanding. We do not have to adopt
a particular life-style for the development of understanding.
Under­standing can develop when it is assisted by the other Path-factors
and supported by other conditions, including the Perfections. There
should be no discouragement about the long way which lies ahead. There
can at least be a beginning of understanding of the reality appearing at
the present moment through one of the six doors.

The Buddha taught the conditions for the development of what is good and
wholesome and the conditions for the eradication of defilements. In
developing the Buddha's Path one will come to know one's ignorance of
realities, one's selfishness and other defilements. The change from
selfishness to detachment, from ignorance to under­standing is immense.
How could such changes take place within a short time? It is a long
process. Also the Buddha and his disciples had to walk a long way in
order to gain full understanding of the four noble Truths and freedom
from the cycle of birth and death. We read in the Kindred Sayings (V,
Kindred Sayings about the Truths, chapter III, §1, Knowledge) that the
Buddha, while he was staying among the Vajjians at Koṭigāma, said to the
monks:

\begin{quote}
Monks, it is through not understanding, not penetrating four noble
Truths that we have run on, wandered on, this long, long road, both you
and I. What are the four?

Through not understanding, not penetrating the noble truth of dukkha, of
the arising of dukkha, of the ceasing of dukkha, of the way leading to
the ceasing of dukkha, we have run on, wandered on, this long long road,
both you and I.

But now, monks, the noble truth of dukkha is understood, is penetrated,
likewise the noble truth of the arising, of the ceasing of dukkha, of
the way leading to the ceasing of dukkha is understood, is penetrated.
Uprooted is the craving to exist, destroyed is the channel to becoming,
there is no more coming to be\ldots{}
\end{quote}

\chapter{Pāli Glossary}


\begin{description}
\item[akusala] unwholesome, unskilful
\item[anattā] non self
\item[anumodanā] thanksgiving, appreciation of someone else's
  kusala
\item[arahat] noble person who has attained the fourth and last
  stage of enlightenment
\item[Buddha] a fully enlightened person who has discovered the
  truth all by himself, without the aid of a teacher
\item[citta] consciousness the reality which knows or cognizes an
  object
\item[dhamma] reality, truth, the teaching
\item[dukkha] suffering, unsatisfactoriness of conditioned realities
\item[jhāna] absorption which can be attained through the
  development of calm
\item[kamma] intention or volition; deed motivated by volition
\item[kasiṇa] disk, used as an object for the development of calm
\item[khandhas] aggregates of conditioned realities classified as
  five groups: physical phenomena, feelings, perception or remembrance,
  activities or formations (cetasikas other than feeling or perception),
  consciousness.
\item[kusala] wholesome, skilful
\item[lokuttara citta] supramundane citta which experiences nibbāna
\item[nāma] mental phenomena,including those which are conditioned
  and also the unconditioned nāma which is nibbāna.
\item[nibbāna] unconditioned reality, the reality which does not
  arise and fall away. The destruction of lust, hatred and delusion. The
  deathless. The end of suffering
\item[rūpa] physical phenomena, realities which do not experience
  anything
\item[samatha] the development of calm
\item[satipaṭṭhāna] applicatioms of mindfulness. It can mean the
  cetasika sati which is aware of realities or the objects of
  mindfulness which are classified as four applications of mindfulness:
  Body, Feeling Citta, Dhamma. Or it can mean the development of direct
  understanding of realities through awareness.
\item[sīla] morality in action or speech, virtue
\item[Tathāgata] literally ``thus gone'', epithet of the Buddha
\item[Tipiṭaka] the teachings of the Buddha
\item[vipassanā] wisdom which sees realities as they are
\end{description}

\end{document}
